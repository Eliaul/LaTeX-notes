

\chapter{向量丛}

\section{向量丛}

令 $M$ 是拓扑空间,\emph{$M$ 上的秩 $k$ 的(实)向量丛}指的是一个拓扑空间
$E$ 附带一个连续满射 $\pi:E\to M$,其满足下面的条件:
\begin{enumerate}
  \item 对于每个 $p\in M$,$p$ 上的纤维 $E_p=\pi^{-1}(p)$ 有一个
  $k$-维的实向量空间结构。
  \item 对于每个 $p\in M$,都存在 $p$ 的邻域 $U$ 和一个同胚
  $\varPhi:\pi^{-1}(U)\to U\times \mathbb{R}^k$ (这个同胚被称为
  \emph{$E$ 在 $U$ 上的局部平凡化}),并且 $\varPhi$ 满足:
  \begin{itemize}[nosep]
    \item 记 $\pi_U:U\times \mathbb{R}^k\to U$ 是投影,
    有 $\pi_U\circ\varPhi=\pi$;
    \item 对于每个 $q\in U$,$\varPhi|_{{E_q}}$ 都是
    $E_q$ 到 $\{q\}\times \mathbb{R}^k$ 的向量空间同构。
  \end{itemize}
\end{enumerate} 
如果 $M,E$ 都是光滑流形,$\pi$ 是光滑映射并且局部平凡化可以
选取为微分同胚,那么 $E$ 被称为\emph{光滑向量丛}。在这种情况下,
我们将局部平凡化成为\emph{光滑局部平凡化}。

秩 $1$ 的向量丛被称为\emph{线丛}。
空间 $E$ 被称为\emph{丛的全空间},$M$ 被称为\emph{底空间},
$\pi$ 被称为\emph{投影}。为了简洁,我们通常写作“$E$ 是 $M$ 上的一个向量丛”
或者 “$E\to M$ 是一个向量丛” 或者 “$\pi:E\to M$ 是一个向量丛”。

\begin{exercise}
  设 $E$ 是 $M$ 上的光滑向量丛,证明投影 $\pi:E\to M$
  是满射的光滑浸没。
\end{exercise}
\begin{proof}
  根据定义 $\pi$ 是满射。任取 $p\in M$,设 $U$ 是 $p$ 的邻域,
  $\varPhi:\pi^{-1}(U)\to U\times \mathbb{R}^k$ 是光滑局部平凡化,那么
  $\pi_U\circ\varPhi=\pi$。由于 $\pi_U$ 是光滑浸没,$\varPhi$
  是微分同胚,所以 $\pi$ 是光滑浸没。
\end{proof}

如果存在 $E$ 在整个 $M$ 上的局部平凡化(这被称为\emph{$E$ 的全局平凡化}),
那么 $E$ 被称为\emph{平凡丛}。在这种情况下,$E$ 本身同胚于
$M\times \mathbb{R}^k$。如果 $E\to M$ 是光滑丛并且有一个光滑
的全局平凡化,那么我们说 $E$ 是\emph{光滑平凡丛},此时
$E$ 微分同胚于 $M\times \mathbb{R}^k$。为了简洁起见,我们说一个光滑丛是平凡的时候,
总是指它在光滑意义上平凡而不是拓扑意义上。

\begin{example}[积丛]
  任意空间 $M$ 上的最简单的秩 $k$ 的向量丛是积空间 $E=M\times \mathbb{R}^k$,
  投影 $\pi=\pi_1:M\times \mathbb{R}^k\to M$。这样的丛被称为\emph{积丛},那么恒等映射
  $\Id_E$ 就是一个全局平凡化,所以积丛是平凡丛。
\end{example}

\begin{example}[M\"obius 丛]
  定义 $\mathbb{R}^2$ 上的等价关系为:$(x,y)\sim (x',y')$ 当且仅当存在 $n\in \mathbb{Z}$ 使得 
  $(x',y')=\bigl(x+n,(-1)^ny\bigr)$。令 $E=\mathbb{R}^2/\sim$ 是商空间,$q:\mathbb{R}^2\to E$
  是商映射。

  为了可视化 $E$,令 $S$ 是带子 $[0,1]\times \mathbb{R}\subseteq \mathbb{R}^2$。$q$ 在 $S$
  上的限制是满射且是闭映射,所以 $q|_S$ 是商映射。显然 $q|_S$ 造成的非平凡的粘合只会在 $S$
  的边界上出现,因此我们可以把 $E$ 视为将 $S$ 的右边界上下翻转然后与左边界粘合得到的。

  考虑交换图:
  \[
    \begin{tikzcd}
      \mathbb{R}^2\arrow[r,"q"]\arrow[d,"\pi_1"'] & E\arrow[d,dashed,"\pi"] \\
      \mathbb{R}\arrow[r,"\varepsilon"] & \mathbb{S}^1,
    \end{tikzcd}
  \]
  其中 $\pi_1$ 是第一个分量的投影,$\varepsilon: \mathbb{R}\to \mathbb{S}^1$ 是光滑覆盖映射
  $\varepsilon(x)=e^{2\pi ix}$。因为 $\varepsilon\circ\pi_1$ 在 $q$ 的每个纤维上是常值的,所以
  其下降到一个连续映射 $\pi:E\to \mathbb{S}^1$。直接验证可知 $E$ 有唯一的光滑结构使得
  $q$ 是光滑覆盖映射并且 $\pi:E\to \mathbb{S}^1$ 是光滑实线丛,被称为\emph{M\"obius 丛}。
  
\end{example}

\begin{proposition}[切丛作为向量丛]
  令 $M$ 是一个光滑 $n$-流形,$TM$ 是切丛。那么
  $TM$ 是 $M$ 上的秩 $n$ 的光滑向量丛。
\end{proposition}
\begin{proof}
  记 $\pi:TM\to M$ 是投影映射。对于每个 $p\in M$,$\pi^{-1}(p)=T_pM$
  是 $n$ 维向量空间。取 $p$ 处的光滑坐标卡 $(U,(x^i))$,那么
  $\varPhi:\pi^{-1}(U)\to U\times \mathbb{R}^n$ 可以定义为
  \[
    \varPhi\left(v^i\left.\frac{\partial}{\partial x^i}\right|_q\right) 
    =\left(q,\bigl(v^1,\dots,v^n\bigr)\right),
  \]
  那么 $\varPhi$ 的坐标表示为恒等映射,所以是微分同胚。
  对于每个 $q\in U$,$\varPhi$ 限制在 $T_qM$ 上显然都是
  向量空间同构。此外容易验证 $\pi_U\circ \varPhi=\pi$。
\end{proof}
 
\begin{lemma}
  令 $\pi:E\to M$ 是秩 $k$ 的光滑向量丛,假设 $\varPhi:\pi^{-1}(U)\to U\times \mathbb{R}^k$
  和 $\varPsi:\pi^{-1}(V)\to V\times \mathbb{R}^k$ 是两个 $E$ 的光滑局部平凡化
  并且 $U\cap V\neq\emptyset$。那么存在一个光滑映射 $\tau:U\cap V\to \GL(k,\mathbb{R})$
  使得复合映射 $\varPhi\circ\varPsi^{-1}:(U\cap V)\times \mathbb{R}^k\to (U\cap V)\times \mathbb{R}^k$
  形如
  \[
    \varPhi\circ\varPsi^{-1}(p,v)=\bigl(p,\tau(p)v\bigr)  .
  \]
\end{lemma}
\begin{proof}
  我们有下面的交换图:
  \[
    \begin{tikzcd}[column sep=2.4em]
      (U\cap V)\times \mathbb{R}^k\arrow[dr,"\pi_1"']
      &
      \pi^{-1}(U\cap V)\arrow[l,"\varPsi"']\arrow[r,"\varPhi"]\arrow[d,"\pi"]
      &
      (U\cap V)\times \mathbb{R}^k\arrow[dl,"\pi_1"]\\
      & 
      U\cap V
      &
    \end{tikzcd}  
  \]
  其中 $\varPhi,\varPsi$ 都是微分同胚。那么对于任意的
  $(p,v)\in(U\cap V)\times \mathbb{R}^k$,有
  \[
    p=\pi_1(p,v)=\pi\circ\varPsi^{-1}(p,v)=\pi_1\circ(\varPhi\circ\varPsi^{-1})
    (p,v),  
  \]
  这表明
  \[
    \varPhi\circ\varPsi^{-1}(p,v)=\bigl(p,\sigma(p,v)\bigr),
  \]
  其中 $\sigma:(U\cap V)\times \mathbb{R}^k\to \mathbb{R}^k$ 是光滑映射。
  对于每个固定的 $p\in U\cap V$,
  \[ 
    \varPhi\circ\varPsi^{-1}:\{p\}\times \mathbb{R}^k\xrightarrow{\varPsi^{-1}}
    E_p\xrightarrow{\varPhi}\{p\}\times \mathbb{R}^k
  \]
  是向量空间 $\{p\}\times \mathbb{R}^k$ 上的可逆线性变换,所以
  $\sigma(p,v)=\tau(p)v$,其中 $\tau(p)\in\GL(k,\mathbb{R})$。
  下面只需要说明 $\tau$ 是光滑的。由于
  \[
    \varPhi\circ\varPsi^{-1}(p,v)=\bigl(p,\tau(p)v\bigr),
  \]
  记 $\pi_2:(U\cap V)\times \mathbb{R}^k\to \mathbb{R}^k$ 的投影,那么
  \[
    \tau(p)v=\pi_2\circ\varPhi\circ\varPsi^{-1}(p,v).
  \]
  记 $e_i\in \mathbb{R}^k$ 是标准基向量,那么 $p\mapsto \tau(p)e_i$
  作为 $U\cap V\to \mathbb{R}^k$ 的映射是光滑映射,所以 $p\mapsto \tau(p)$
  是光滑映射。
\end{proof} 

上述光滑映射 $\tau:U\cap V\to\GL(k,\mathbb{R})$ 被称为局部平凡化
$\varPhi$ 和 $\varPsi$ 之间的\emph{转移函数}。

\begin{lemma}[向量丛坐标卡引理]
  令 $M$ 是带边或者无边光滑流形,假设对于每个 $p\in M$ 我们都赋予一个 $k$ 维实向量空间
  $E_p$。令 $E=\coprod_{p\in M}E_p$,$\pi:E\to M$ 将 $E_p$ 的每个元素
  送到点 $p$。如果我们还有下面的资料:
  \begin{enumerate}
    \item $M$ 的一个开覆盖 $\{U_\alpha\}_{\alpha\in A}$;
    \item 对于每个 $\alpha\in A$,有双射 $\varPhi_\alpha:\pi^{-1}(U_\alpha)\to U_\alpha\times \mathbb{R}^k$,
    并且 $\varPhi$ 限制在每个 $E_p$ 上是向量空间 $E_p$ 到 $\{p\}\times \mathbb{R}^k$ 的同构;
    \item 对于每个 $\alpha,\beta\in A$ 并且 $U_\alpha\cap U_\beta\neq\emptyset$,存在光滑映射
    $\tau_{\alpha\beta}:U_\alpha\cap U_\beta\to \GL(k,\mathbb{R})$ 使得映射
    $\varPhi_\alpha\circ\varPhi_\beta^{-1}:(U_\alpha\cap U_\beta)\times \mathbb{R}^k\to (U_\alpha\cap U_\beta)\times \mathbb{R}^k$
    形如
    \[
      \varPhi_\alpha\circ\varPhi_\beta^{-1}(p,v)=\bigl(p,\tau_{\alpha\beta}(p)v\bigr).  
    \]
  \end{enumerate}
  那么,$E$ 有唯一的拓扑和光滑结构使得其成为 $M$ 上的秩 $k$ 的光滑向量丛,
  且 $\pi$ 是投影映射,$\{(U_\alpha,\varPhi_\alpha)\}$ 是光滑局部平凡化。
\end{lemma}

\begin{remark}
  $E$ 上的拓扑结构为:对于每个 $p\in M$,选取所有满足 $p\in V_p\subseteq U_\alpha$
  的光滑坐标卡 $(V_p,\varphi_p)$,令 $\tilde\varphi_p=(\varphi_p\times \Id_{\mathbb{R}^k})\circ\varPhi_\alpha$,
  那么拓扑基 $\{\tilde\varphi_p^{-1}(W)\}$ (其中 $W\subseteq \varphi_p(V_p)\times \mathbb{R}^k$ 是开集)
  生成 $E$ 上的拓扑。$E$ 的光滑结构为:$\{(\pi^{-1}(V_p),\tilde\varphi_p)\}$
  构成 $E$ 的一组光滑坐标卡。
\end{remark} 

\begin{example}[Whitney 和]
  给定光滑流形 $M$ 以及分别有秩 $k'$ 和 $k''$ 的光滑向量丛 $E'\to M$ 和 $E''\to M$,
  我们构造一个新的向量丛,被称为 $E'$ 和 $E''$ 的\emph{Whitney 和},其在 $p\in M$
  处的纤维是直和 $E_p'\oplus E_p''$。全空间定义为 $E'\oplus E''=\coprod_{p\in M}\bigl(E_p'\oplus E_p''\bigr)$,
  附带显然的投影 $\pi:E'\oplus E''\to M$。对于每个 $p\in M$,选择 $p$ 处的足够小的
  邻域 $U$ 使得存在 $E'$ 的局部平凡化 $(U,\varPhi')$ 以及 $E''$ 的局部平凡化 $(U,\varPhi'')$,
  定义 $\varPhi:\pi^{-1}(U)\to U\times \mathbb{R}^{k'+k''}$ 为
  \[
    \varPhi(v',v'')=\bigl(\pi'(v'),\bigl(\pi_{\mathbb{R}^{k'}}\circ\varPhi'(v'),\pi_{\mathbb{R}^{k''}}\circ\varPhi''(v'')\bigr)\bigr).
  \]
  假设给定其他的任意两个局部平凡化 $\left(\wtilde U,\wtilde\varPhi'\right)$ 和
  $\left(\wtilde U,\wtilde\varPhi''\right)$,令 $\tau':U\cap\wtilde U\to \GL(k',\mathbb{R})$
  和 $\tau'':U\cap\wtilde U\to \GL(k'',\mathbb{R})$ 是对应的转移函数。那么
  $E'\oplus E''$ 的转移函数满足
  \[
    \wtilde\varPhi\circ\varPhi^{-1}(p,(v',v''))=\bigl(p,\bigl(\tau'(p)v',\tau''(p)v''\bigr)\bigr),
  \]
  于是 $\tau(p)=\tau'(p)\oplus\tau''(p)$ 是分块矩阵
  \[
    \begin{pmatrix}
      \tau'(p) & 0 \\
      0 & \tau''(p)
    \end{pmatrix}.
  \]
  这就表明 $E'\oplus E''$ 是 $M$ 上的光滑向量丛。 
\end{example}

\begin{example}[向量丛的限制]
  设 $\pi:E\to M$ 是秩 $k$ 的向量丛,$S\subseteq M$ 是任意子集。定义\emph{$E$ 到 $S$ 的限制}
  为 $E|_S=\bigcup_{p\in S}E_p$,投影 $E|_S\to S$ 就是 $\pi$ 的限制。
  如果 $\varPhi:\pi^{-1}(U)\to U\times \mathbb{R}^k$ 是局部平凡化,那么其可以限制为双射
  $\varPhi|_S:(\pi|_S)^{-1}(U\cap S)\to (U\cap S)\times \mathbb{R}^k$。如果 $E$
  是光滑向量丛,$S\subseteq M$ 是浸入或者嵌入子流形,容易验证 $E|_S$ 是光滑向量丛。
\end{example}

\section{向量丛的局部和全局截面}

令 $\pi:E\to M$ 是向量丛,\emph{$E$ 的(全局)截面}指的是连续映射 $\sigma:M\to E$ 使得
$\pi\circ\sigma=\Id_M$。这意味着对于每个 $p\in M$,$\sigma(p)$ 是纤维 $E_p$ 的元素。

更一般的,\emph{$E$ 的局部截面}指的是一个连续映射 $\sigma:U\to E$,其中 $U\subseteq M$ 是开集,
使得 $\pi\circ\sigma=\Id_U$。注意到 $E$ 在 $U$ 上的局部截面就是限制丛 $E|_U$ 上的
全局截面。如果 $M$ 是带边或者无边光滑流形且 $E$ 是光滑向量丛,我们用光滑截面
代指光滑的截面映射。

\emph{$E$ 的零截面}指的是一个全局截面 $\zeta:M\to E$,其定义为
\[
  \zeta(p)=0\in E_p\quad \forall p\in M.
\]
与向量场的情况一样,截面 $\sigma$ 的\emph{支集}定义为集合 $\{p\in M\,|\, \sigma(p)\neq 0\}$ 的闭包。

\begin{example}[向量丛的截面]
  设 $M$ 是光滑流形。
  \begin{enumerate}
    \item $TM$ 的截面就是 $M$ 上的向量场。
    \item 如果 $E=M\times \mathbb{R}^k$ 是积丛,那么在 $E$ 的截面和
    $M\to \mathbb{R}^k$ 的连续映射之间有一个自然的一一对应。一个连续映射 $F:M\to \mathbb{R}^k$
    决定了截面 $\tilde F:M\to E$ 为 $\tilde F(p)=(p,F(p))$,反之依然。如果 $M$
    是光滑流形,那么截面 $\tilde F$ 光滑当且仅当 $F$ 光滑。
  \end{enumerate}
\end{example}

如果 $\pi:E\to M$ 是光滑向量丛,那么 $E$ 的所有全局截面在逐点加法和数乘下成为一个向量空间,记为
$\Gamma(E)$。与向量场类似,光滑向量丛的截面可以左乘一个光滑实值函数:
如果 $f\in C^\infty(M),\sigma\in\Gamma(E)$,我们可以定义 $f\sigma$ 为
\[
  (f\sigma)(p)=f(p)\sigma(p).
\] 

\begin{lemma}[向量丛的延拓引理]
  令 $\pi:E\to M$ 是光滑流形 $M$ 上的向量丛,$A$ 是 $M$ 的闭子集,$\sigma:A\to E$
  是 $E|_A$ 的截面,如果对于每个 $p\in A$,都存在 $p$ 的邻域 $V$ 和
  $V$ 上的光滑截面与 $\sigma$ 在 $V\cap A$ 上重合,那么对于包含 $A$ 的任意开子集
  $U\subseteq M$,都存在光滑全局截面 $\tilde\sigma\in \Gamma(E)$ 使得
  $\tilde{\sigma}|_A=\sigma$ 以及 $\supp\tilde{\sigma}\subseteq U$。
\end{lemma}


\subsection{局部和全局标架}


\begin{example}[积丛的全局标架]
  如果 $E=M\times \mathbb{R}^k\to M$ 是积丛,那么 $\mathbb{R}^k$
  的标准基 $(e_1,\dots,e_k)$ 导出 $E$ 的一个全局标架 $(\tilde{e}_i)$,
  定义为 $\tilde{e}_i(p)=(p,e_i)$。如果 $M$ 是带边或者无边光滑流形,那么
  这个全局标架是光滑的。
\end{example}

\begin{example}[与局部平凡化相关的局部标架]
  设 $\pi:E\to M$ 是光滑向量丛。如果 $\varPhi:\pi^{-1}(U)\to U\times \mathbb{R}^k$
  是 $E$ 的光滑局部平凡化,我们可以使用和上例同样的思想来构造一个 $E$ 在 $U$ 上
  的局部标架。定义映射 $\sigma_1,\dots,\sigma_k:U\to E$ 为
  $\sigma_i(p)=\varPhi^{-1}(p,e_i)=\varPhi^{-1}\circ\tilde{e}_i(p)$。
  因为 $\varPhi$ 是微分同胚,所以 $\sigma_i$ 是光滑的。下面我们说明
  $\bigl(\sigma_i(p)\bigr)$ 构成 $E_p$ 的一组基即可。由于 $\varPhi$
  限制到 $E_p$ 是向量空间同构 $\varPhi:E_p\to \{p\}\times \mathbb{R}^k$,
  并且 $\varPhi(\sigma_i(p))=(p,e_i)$ 构成 $\{p\}\times\mathbb{R}^k$
  的一组基,所以 $\bigl(\sigma_i(p)\bigr)$ 构成 $E_p$ 的一组基。
  我们说局部标架 $(\sigma_i)$ 是相关于 $\varPhi$ 的局部标架。
\end{example}



\section{丛同态}

如果 $\pi:E\to M$ 和 $\pi':E'\to M'$ 是向量丛,连续映射 $F:E\to E'$
被称为\emph{丛同态},如果存在映射 $f:M\to M'$ 使得
$\pi'\circ F=f\circ\pi$,即
\[
  \begin{tikzcd}
    E\arrow[r,"F"]\arrow[d,"\pi"'] & E'\arrow[d,"\pi'"]\\
    M\arrow[r,"f"'] & M' 
  \end{tikzcd}  
\]
此外,对于每个 $p\in M$,限制映射 $F|_{E_p}:E_p\to E'_{f(p)}$
是线性映射。此时我们还称 \emph{$F$ 覆盖 $f$}。

\begin{proposition}
  设 $\pi:E\to M$ 和 $\pi':E'\to M'$ 是向量丛,$F:E\to E'$
  是覆盖 $f:M\to M'$ 的丛同态,那么 $f$ 是连续映射并且由 $F$
  唯一确定。如果丛和 $F$ 都是光滑的,那么 $f$ 也是光滑的。
\end{proposition}
\begin{proof}
  记 $\zeta:M\to E$ 是零截面,那么 $f=\pi'\circ F\circ\zeta$。
\end{proof}

如果一个双射的丛同态 $F:E\to E'$ 的逆也是丛同态,那么 $F$ 被称为\emph{丛同构}。
如果 $F$ 还是微分同胚,那么被称为\emph{光滑丛同构}。

当 $E$ 和 $E'$ 都是同一个底空间 $M$ 上的向量丛的时候,稍微严格一些的丛同态概念更加有用。
\emph{$M$ 上的丛同态}指的是一个覆盖 $\Id_M$ 的丛同态,换句话说,即一个连续映射
$F:E\to E'$ 满足 $\pi'\circ F=\pi$,
\[
  \begin{tikzcd}
    E\arrow[rr,"F"]\arrow[dr,"\pi"'] & & E'\arrow[dl,"\pi'"]\\
    & M & 
  \end{tikzcd}  
\]
并且 $F$ 在每个纤维上的限制是线性映射。如果存在一个 $M$ 上的丛同态
$F:E\to E'$ 同时是(光滑)丛同构,那么我们说 $E$ 和 $E'$ 
\emph{在 $M$ 上是(光滑)同构的}。

\begin{proposition}
  设 $E,E'$ 是光滑流形 $M$ 上的光滑向量丛,$F:E\to E'$ 是
  $M$ 上的双射光滑丛同态,那么 $F$ 是光滑丛同构。
\end{proposition}
\begin{proof}
  $F$ 是双射表明任取 $p\in M$,$F|_{E_p}:E_p\to E_{F(p)}'$ 是双射的线性映射,
  所以 $F|_{E_p}$ 是同构,所以 $F^{-1}|_{E_{F(p)}'}$ 是同构,故
  $F^{-1}$ 是双射的丛同态,所以 $F$ 是丛同构。
  任取 $p\in M$,设 $\varPhi$ 是 $E$ 在 $(U,\varphi)$ 上的局部平凡化,
  $\varPhi'$ 是 $E'$ 在 $(V,\psi)$ 上的局部平凡化,那么 
  $F$ 在 $U\cap V$ 中的坐标表示为
  \[
    (x,v)\mapsto (\varphi^{-1}(x),\varPhi^{-1}(v))\mapsto \bigl(\varphi^{-1}(x),F\circ\varPhi^{-1}(v)\bigr)  
    \mapsto \bigl(\psi\circ\varphi^{-1}(x),\varPhi'\circ F\circ\varPhi^{-1}(v)\bigr),
  \]
  注意到 $\varPhi'\circ F\circ\varPhi^{-1}:\mathbb{R}^m\to \mathbb{R}^{n}$
  是双射的线性映射,所以 $F$ 是微分同胚。
\end{proof}

\begin{example}[丛同态]
  \mbox{}
  \begin{enumerate}
    \item 如果 $F:M\to N$ 是光滑映射,那么全局微分 $\d F:TM\to TN$
    是覆盖 $F$ 的光滑丛同态。
    \item 如果 $E\to M$ 是光滑向量丛,$S\subseteq M$ 是浸入子流形,那么
    包含映射 $E|_S\hookrightarrow E$ 是覆盖包含映射 $S\hookrightarrow M$
    的光滑丛同态。
  \end{enumerate}
\end{example}

设 $E\to M$ 和 $E'\to M$ 是带边或者无边光滑流形 $M$ 上的光滑向量丛,
令 $\Gamma(E),\Gamma(E')$ 表示它们的光滑全局截面。如果
$F:E\to E'$ 是光滑丛同态,那么诱导出一个映射 $\wtilde F:\Gamma(E)\to\Gamma(E')$
为
\begin{equation}
  \wtilde F(\sigma)(p)=(F\circ\sigma)(p).
\end{equation}
不难验证 $\wtilde F(\sigma)$ 是 $E'$ 的一个截面,并且是光滑截面。

因为丛同态在纤维上是线性映射,所以上述截面上的映射 $\wtilde F$ 在
$\mathbb{R}$ 上是线性的。实际上它们满足更强的线性性,即在 $C^\infty(M)$
上是线性的。即任取 $f\in C^\infty(M)$,有
\[
  \wtilde F(f\sigma)(p)=F\bigl(f(p)\sigma(p)\bigr)
  =f(p)F(\sigma(p))=\bigl(f\wtilde{F}(\sigma)\bigr)(p),
\]
所以这表明 $\wtilde F(f\sigma)=f\wtilde F(\sigma)$。
实际上,这个 $C^\infty(M)$-线性性完全刻画了丛同态。

\begin{lemma}[丛同态表征引理]\label{lemma:bundle homomorphism characterization}
  令 $\pi:E\to M$ 和 $\pi':E\to M$ 是带边或者无边光滑流形 $M$ 上的光滑向量丛,
  映射 $\mathcal{F}:\Gamma(E)\to\Gamma(E')$ 在 $C^\infty(M)$
  上是线性的当且仅当存在一个光滑丛同态 $F:E\to E'$
  使得对于所有 $\sigma\in\Gamma(E)$ 有 $\mathcal{F}(\sigma)=F\circ\sigma$。
\end{lemma}
\begin{proof}
  我们已经注意到光滑丛同态诱导出的截面之间的映射是 $C^\infty(M)$-线性的。
  反过来,假设 $\mathcal{F}:\Gamma(E)\to\Gamma(E')$ 是 $C^\infty(M)$-线性的。
  首先,我们证明 $\mathcal{F}$ 的作用是局部的:如果在某个开集 $U\subseteq M$
  上有 $\sigma_1\equiv \sigma_2$,那么在 $U$ 上有 $\mathcal{F}(\sigma_1)\equiv \mathcal{F}(\sigma_2)$。
  记 $\tau=\sigma_1-\sigma_2$,我们说明 $\mathcal{F}(\tau)$ 在 $U$ 上恒为零即可。
  给定 $p\in U$,设 $\psi$ 是关于 $\{p\}$ 的支在 $U$ 中的光滑鼓包函数,
  那么 $\psi\tau$ 在 $M$ 上恒为零。根据 $\mathcal{F}$ 的 $C^\infty(M)$-线性性,有 
  $0=\mathcal{F}(\psi\tau)=\psi \mathcal{F}(\tau)$,所以
  \[
    \mathcal{F}(\tau)(p)=\psi(p)\mathcal{F}(\sigma)(p)=0.
  \]
  这就表明 $\mathcal{F}(\tau)$ 在每个 $p\in U$ 处的取值都是零。

  下面我们进一步说明 $\mathcal{F}$ 的作用实际上是逐点的:如果 $\sigma_1(p)=\sigma_2(p)$,
  那么 $\mathcal{F}(\sigma_1)(p)=\mathcal{F}(\sigma_2)(p)$。同样,我们只需要
  假设 $\tau(p)=0$ 时有 $\mathcal{F}(\tau)(p)=0$。令 $(\sigma_1,\dots,\sigma_k)$
  是 $E$ 在 $p$ 的某个邻域 $U$ 中的局部标架,设 $\tau=u^i\sigma_i$,其中 $u^i\in C^\infty(U)$。
  那么 $\tau(p)=0$ 表明 $u^1(p)=\cdots=u^k(p)=0$。根据向量丛和函数的延拓引理,
  存在光滑全局截面 $\tilde{\sigma}_i\in\Gamma(E)$ 使得其与 $\sigma_i$ 在
  $p$ 的某个邻域上相等,以及存在光滑函数 $\tilde{u}^i\in C^\infty(M)$
  使得其与 $u^i$ 在 $p$ 的某个邻域上相等。那么在 $p$ 的某个邻域上有
  $\tau=\tilde{u}^i\tilde{\sigma}_i$,所以
  \[
    \mathcal{F}(\tau)(p)=\tilde{u}^i(p) \mathcal{F}(\tilde{\sigma}_i)(p)
    =0.
  \]

  下面定义丛同态 $F:E\to E'$。对于任意 $p\in M$ 和 $v\in E_p$,设 $\tilde{v}$
  是 $E$ 的满足 $\tilde{v}(p)=v$ 的全局光滑截面,令 $F(v)=\mathcal{F}(\tilde{v})(p)\in E_p'$。
  前面的讨论表明这个定义和截面 $\tilde{v}$ 的选取无关。显然 $F$ 满足 $\pi'\circ F=\pi$。
  根据 $\mathcal{F}$ 的线性性,$F$ 在每个纤维上也是线性的。并且根据定义,对于任意
  $\sigma\in\Gamma(E)$ 有 $\mathcal{F}(\sigma)=F\circ\sigma$。下面只需要说明
  $F$ 是光滑的,这只需要说明它在每个点的某个邻域中光滑。

  给定 $p\in M$,令 $(\sigma_i)$ 是 $E$ 在 $p$ 的某个邻域上的局部标架。
  根据延拓引理,存在全局截面 $\tilde{\sigma}_i$ 使得其与 $\sigma_i$ 在 $p$
  的某个(更小的)邻域 $U$ 上相等。充分缩小 $U$,我们可以假设存在
  $E'$ 在 $U$ 上的一个光滑局部标架 $(\sigma_j')$。因为 $\mathcal{F}$
  把 $E$ 的光滑全局截面映射到 $E'$ 的光滑全局截面,所以存在光滑函数
  $A_i^j\in C^\infty(U)$ 使得 $\mathcal{F}(\tilde{\sigma}_i)|_U=A_i^j\sigma_j'$。

  对于任意 $q\in U$ 和 $v\in E_q$,设 $v=v^i\sigma_i(q)$,其中 $(v^1,\dots,v^k)$
  是实数,那么
  \[
    F\bigl(v^i\sigma_i(q)\bigr)=\mathcal{F}\bigl(v^i\tilde{\sigma}_i\bigr)(q)
    =v^i \mathcal{F}(\tilde{\sigma}_i)(q)=v^iA_i^j(q)\sigma_j'(q).
  \]
  如果 $\varPhi$ 和 $\varPhi'$ 分别是 $E$ 和 $E'$ 的相对于标架 $(\sigma_i)$
  和 $(\sigma_j')$ 的局部平凡化,那么复合映射 $\varPhi'\circ F\circ \varPhi^{-1}:U\times \mathbb{R}^k\to U\times \mathbb{R}^m$
  形如
  \[
    \varPhi'\circ F\circ \varPhi^{-1}\left(
      q,\left(v^1,\dots,v^k\right)
    \right)=
    \left(q,\left(
      A_i^1(q)v^i,\dots,A^m_i(q)v^i
    \right)\right),
  \]
  这是光滑的。因为 $\varPhi,\varPhi'$ 是微分同胚,所以 $F$ 在 $\pi^{-1}(U)$ 上是光滑映射。
\end{proof}



\section{子丛}

给定向量丛 $\pi_E:E\to M$,$E$ 的一个\emph{子丛}指的是向量丛 $\pi_D:D\to M$,
其中 $D$ 是 $E$ 的拓扑子空间并且 $\pi_D$ 是 $\pi_E$ 在 $D$ 上的限制映射,
满足对于每个 $p\in M$,子集 $D_p=D\cap E_p$ 是 $E_p$ 的一个线性子空间,并且
$D_p$ 的向量空间结构继承于 $E_p$。注意到 $D$ 是 $M$ 上的向量丛这一条件表明
所有的纤维 $D_p$ 非空并且有同样的维数。如果 $E\to M$ 是光滑丛,那么
$E$ 的子丛被称为\emph{光滑子丛}当且仅当它是一个光滑向量丛且是 $E$
中的带边或者无边嵌入子流形。

下面的引理给出了检验一族子空间 $\{D_p\subseteq E_p\,|\, p\in M\}$ 的无交并 
是否为光滑子丛的方便条件。

\begin{lemma}[子丛的局部标架判别法]
  令 $\pi:E\to M$ 是光滑向量丛,对于每个 $p\in M$,我们给定一个
  $m$-维子空间 $D_p\subseteq E_p$。那么 $D=\bigcup_{p\in M}D_p\subseteq E$
  是 $E$ 的光滑子丛当且仅当下面的条件被满足:
  \begin{quote}
    $M$ 的每个点都有一个邻域 $U$,存在光滑局部截面 $\sigma_1,\dots\sigma_m:U\to E$
    使得在每个 $q\in U$ 处 $\sigma_1(q),\dots,\sigma_m(q)$ 构成 $D_q$ 的一组基。
  \end{quote}
\end{lemma}
\begin{proof}
  如果 $D$ 是光滑子丛,那么根据定义,每个 $p\in M$ 处都有一个邻域 $U$
  使得在 $U$ 上有一个 $D$ 的局部平凡化,那么我们选取与这个局部平凡化相关
  的局部标架 $\tau_1,\dots,\tau_m:U\to D$,与包含映射 $\iota:D\hookrightarrow E$
  复合即可得到 $E$ 的局部截面 $\sigma_j=\iota\circ\tau_j$。

  反之,假设 $E\to M$ 是秩 $k$ 的光滑丛并且 $D\subseteq E$ 满足上面的条件。
  根据假设 $D\cap E_p$ 是 $E_p$ 的子空间,我们只需要说明 $D$ 是 $E$
  的带边或者无边嵌入子流形,以及 $\pi$ 在 $D$ 上的限制使得 $D$ 是光滑向量丛。

  为了说明 $D$ 是带边或者无边嵌入子流形,只需要说明每个 $p\in M$ 处都存在
  一个邻域 $U$ 使得 $D\cap \pi^{-1}(U)$ 为 $\pi^{-1}(U)\subseteq E$
  的嵌入(带边)子流形即可。给定 $p\in M$,设 $\sigma_1,\dots,\sigma_m:U\to E$
  是在 $p$ 的某个邻域上满足上述条件的光滑局部截面。我们可以将其完备化
  为 $p$ 的某个邻域 $U$ 上的 $E$ 的光滑局部标架 $(\sigma_1,\dots,\sigma_k)$。
  这个局部标架对应于一个光滑局部平凡化 $\varPhi:\pi^{-1}(U)\to U\times \mathbb{R}^k$,定义为
  \[
    \varPhi\left(s^1\sigma_1(q)+\cdots+s^k\sigma_k(q)\right)=
    \left(q,\bigl(s^1,\dots,s^k\bigr)\right).
  \]
  $\varPhi$ 将 $D\cap\pi^{-1}(U)$ 送到子集 $\bigl\{\bigl(q,(s^1,\dots,s^m,0,\dots,0)\bigr)\bigr\}\subseteq U\times \mathbb{R}^k$,
  这是一个嵌入子流形(如果 $U$ 带边那么就是带边子流形)。此外,
  定义映射 $\Psi:D\cap\pi^{-1}(U)\to U\times \mathbb{R}^m$ 为
  \[
    \varPsi\left(s^1\sigma_1(q)+\cdots+s^m\sigma_m(q)\right)=
    \left(q,\bigl(s^1,\dots,s^m\bigr)\right),
  \]
  那么 $\varPsi$ 就是 $D$ 的一个光滑局部平凡化,所以 $D$ 本身是一个光滑向量丛。
\end{proof}

设 $E\to M$ 和 $E'\to M$ 是向量丛,$F:E\to E'$ 是丛同态。对于 $p\in M$,
我们说 $F$ 在 $p$ 处的秩为线性映射 $F|_{E_p}$ 的秩。如果 $F$
在所有 $p\in M$ 处的秩都是相同的,那么我们说 $F$ 是常秩的。

\begin{theorem}
  令 $E$ 和 $E'$ 是光滑流形 $M$ 上的光滑向量丛,$F:E\to E'$ 是丛同态。
  定义子集 $\ker F\subseteq E$ 和 $\im F\subseteq E'$ 为
  \[
    \ker F=\bigcup_{p\in M} \ker(F|_{E_p}),\quad 
    \im F=\bigcup_{p\in M} \im(F|_{E_p}).
  \]
  那么 $\ker F$ 和 $\im F$ 分别是 $E$ 和 $E'$ 的光滑子丛当且仅当
  $F$ 是常秩的。
\end{theorem}
\begin{proof}
  因为向量丛的纤维必须有相同的维数,所以必要性是显然的。下面证明充分性。
  假设 $F$ 有常秩 $r$,$E$ 和 $E'$ 分别有秩 $k$ 和 $k'$。任取 $p\in M$,
  选取 $p$ 的某个邻域 $U$ 上的 $E$ 的局部标架 $(\sigma_1,\dots,\sigma_k)$。
  对于每个 $i$,$F\circ\sigma_i:U\to E'$ 是 $E'$ 的光滑局部截面,所以张成了
  $(\im F)|_U$。对角标重新排序,我们可以假设 $\bigl\{F\circ \sigma_1(p),\dots,F\circ\sigma_r(p)\bigr\}$
  构成了 $\im (F|_{E_p})$ 的一组基,并且根据连续性它们在 $p$ 的某个邻域
  $U_0$ 上仍然保持线性无关性。因为 $F$ 是常秩的,所以
  $(F\circ\sigma_1,\dots,F\circ\sigma_r)$ 构成 $\im F$ 在 $U_0$
  上的一个光滑局部标架。我们可以对每个点做这样的论述,根据局部标架判别法
  可知 $\im F$ 是 $E'$ 的光滑子丛。

  下面说明 $\ker F$ 是光滑子丛。令 $U_0$ 和 $(\sigma_i)$ 和上面相同。
  由于 $\im (F|_{E_p})$ 由基 $\bigl\{F\circ \sigma_1(p),\dots,F\circ\sigma_r(p)\bigr\}$
  张成,所以 $\ker(F|_{E_p})$ 由基 $\bigl\{F\circ \sigma_{r+1}(p),\dots,F\circ\sigma_k(p)\bigr\}$ 张成,
  于是 $(F\circ\sigma_{r+1},\dots,F\circ\sigma_k)$ 构成 $\ker F$
  在某个比 $U_0$ 小的邻域上的光滑局部标架。所以 $\ker F$
  是 $E$ 的光滑子丛。
\end{proof}

\begin{lemma}[正交补丛]
  令 $M$ 是 $\mathbb{R}^n$ 的带边或者无边浸入子流形,$D$ 是 $T\mathbb R^n|_M$ 的
  秩 $k$ 的光滑子丛。对于每个 $p\in M$,令 $D_p^\bot$ 是 $D_p$ 在 $T_p \mathbb{R}^n$
  (配备 Euclid 内积)中的正交补,$D^\bot\subseteq T \mathbb{R}^n|_M$ 是子集
  \[
    D^\bot=\left\{(p,v)\in T \mathbb{R}^n\,|\, p\in M, v\in D_p^\bot\right\}.
  \] 
  那么 $D^\bot$ 是 $T \mathbb{R}^n|_M$ 的秩 $(n-k)$ 的光滑子丛。对于每个 $p\in M$,
  存在 $D^\bot$ 的在 $p$ 的某个邻域上的光滑正交标架。
\end{lemma}
\begin{proof}
  任取 $p\in M$,令 $(X_1,\dots,X_k)$ 是 $D$ 的在 $p\in M$ 的某个邻域 $V$ 上的光滑局部标架。
  因为浸入是局部的嵌入,所以通过缩小 $V$,可以假设 $V$ 是某个坐标球或者坐标半球 $U\subseteq \mathbb{R}^n$
  的切片。因为 $V$ 是 $U$ 的闭集,根据 \autoref{prop:local frame} 的 (3),我们可以将
  $(X_1,\dots,X_k)$ 完备化为 $T \mathbb{R}^n$ 的在 $U$ 上的光滑局部标架 $\bigl(\wtilde X_1,\dots,\wtilde X_n\bigr)$。
  然后我们将其正交化为 $U$ 上的光滑正交标架 $(E_j)$。对于每个 $q\in V$,有
  $\spn\{E_1|_p,\dots,E_k|_q\}=\spn\{X_1|_q,\dots,X_k|_q\}=D_q$。这表明 $(E_{k+1},\dots,E_n)$
  可以限制到 $D^\bot$ 在 $V$ 上的一个光滑正交标架。因此 $D^\bot$ 满足局部标架判别法,
  是 $T \mathbb{R}^n|_M$ 的光滑子丛。
\end{proof}




