\chapter{流形上的积分}

\section{微分形式的积分}

回顾 $\mathbb{R}^n$ 中的一个积分区域指的是一个边界为测度零的有界子集。
令 $D\subseteq \mathbb{R}^n$ 是积分区域,$\omega$ 是 $\bar D$ 上的
一个(连续)$n$-形式。那么存在某个连续函数 $f:\bar D\to \mathbb{R}$
使得 $\omega=f\d x^1\wedge\cdots\wedge \d x^n$。我们定义 $\omega$
在 $D$ 上的\emph{积分}为
\[
  \int_D\omega=\int_D f\d V.
\]
这也可以写为
\[
  \int_D f\d x^1\wedge\cdots\wedge \d x^n=\int_D f\d x^1\cdots \d x^n.
\]
注意,我们通常会省略“楔积”的符号。

更一般的,令 $U$ 是 $\mathbb{R}^n$ 或者 $\mathbb{H}^n$ 的一个开子集,
$\omega$ 是 $U$ 上的一个紧支撑 $n$-形式。我们定义
\[
  \int_U \omega=\int_D \omega,
\]
其中 $D\subseteq \mathbb{R}^n$ 或者 $\mathbb{H}^n$ 是任意包含 $\supp\omega$
的积分区域,并且 $\omega$ 在支集外延拓为零。不难验证这个定义与选取的
积分区域 $D$ 无关。

\begin{proposition}\label{prop:change of variables for top form}
  设 $D$ 和 $E$ 是 $\mathbb{R}^n$ 或者 $\mathbb{H}^n$ 中的两个开积分区域,
  令 $G:\bar D\to \bar E$ 是光滑映射并且限制为 $D\to E$ 是保定向的
  或者反定向的。如果 $\omega$ 是 $\bar E$ 上的一个 $n$-形式,那么
  \[
    \int_D G^*\omega=\begin{dcases}
      \int_E\omega & \text{如果 $G$ 保定向},\\
      -\int_E\omega & \text{如果 $G$ 反定向}.
    \end{dcases}
  \]
\end{proposition}
\begin{proof}
  令 $\bigl(y^1,\dots,y^n\bigr)$ 表示 $E$ 上的标准坐标,$\bigl(x^1,\dots,x^n\bigr)$
  表示 $D$ 上的标准坐标。首先假设 $G$ 是保定向的。
  设 $\omega=f\d y^1\wedge\cdots\wedge \d y^n$,那么根据重积分的换元公式
  以及满次形式的拉回公式 \ref{prop:pullback of top form},有
  \begin{align*}
    \int_D G^*\omega
    &=\int_D (f\circ G)(\det DG)\d x^1\wedge\cdots \wedge\d x^n\\
    &=\int_D (f\circ G)|\det DG|\d V=\int_E f\d V=\int_E \omega.
  \end{align*}
  如果 $G$ 是反定向的,那么上述绝对值解开后会多一个负号。
\end{proof}

我们想要把这个定理拓展到定义在开子集上的紧支 $n$-形式。但是,因为我们不能
保证任意开子集或者任意紧子集都是积分区域,所以我们需要下面的引理。

\begin{lemma}
  设 $U$ 是 $\mathbb{R}^n$ 或者 $\mathbb{H}^n$ 的一个开子集,$K\subseteq U$
  是一个紧子集。那么存在一个开积分区域 $D$ 使得 $K\subseteq D\subseteq\bar D\subseteq U$。
\end{lemma}
\begin{proof}
  对于每个 $p\in K$,都存在包含 $p$ 的某个开球或者半球,且其闭包包含在 $U$ 中。
  由 $K$ 的紧性可知,存在有限个这样的开球或者半球 $B_1,\dots,B_m$ 覆盖 $K$。
  这些开球或者半球的边界都是测度零的,所以集合 $D=B_1\cup\cdots\cup B_m$
  就是需要的积分区域。
\end{proof}

\begin{proposition}\label{prop:change of variables for top form on open set}
  设 $U$ 和 $V$ 分别是 $\mathbb{R}^n$ 或者 $\mathbb{H}^n$ 的两个开子集,
  令 $G:U\to V$ 是保定向或者反定向的微分同胚。
  如果 $\omega$ 是 $V$ 上的一个紧支撑 $n$-形式,那么
  \[
    \int_V\omega=\pm\int_U G^*\omega,
  \]
  其中正号当且仅当 $G$ 保定向,负号当且仅当 $G$ 反定向。
\end{proposition}
\begin{proof}
  令 $E$ 是满足 $\supp\omega\subseteq E\subseteq \bar E\subseteq V$
  的开积分区域。因为微分同胚把内部映射为内部,边界映射为边界,
  零测集映射为零测集,所以 $D=G^{-1}(E)\subseteq U$ 也是一个开积分区域
  且包含 $\supp G^*\omega$,再根据 \autoref{prop:change of variables for top form}
  即可。
\end{proof}

\subsection{流形上的积分}

使用前面的结果,我们现在可以定义定向流形上微分形式的积分。令
$M$ 是定向的带边或者无边 $n$-流形,$\omega$ 是 $M$ 上的 $n$-形式。
首先假设 $\omega$ 在某个单个光滑坐标卡 $(U,\varphi)$ 中是紧支的,
且这个坐标卡是正定向或者负定向的。我们定义 $\omega$ 在 $M$ 上的\emph{积分}为
\begin{equation}
  \int_M\omega=\pm\int_{\varphi(U)}(\varphi^{-1})^*\omega,
\end{equation}
其中正号对应于正定向的坐标卡,负号对应于负定向的坐标卡。因为
$(\varphi^{-1})^*\omega$ 是在开集 $\varphi(U)\subseteq \mathbb{R}^n$
或者 $\mathbb{H}^n$ 上紧支的 $n$-形式,所以上面的积分是有意义的。

\begin{proposition}\label{prop:well-definedness of integral on manifold}
  在上述定义下,$\int_M\omega$ 不依赖于包含 $\supp\omega$ 的光滑坐标卡的选取。
\end{proposition}
\begin{proof}
  设 $(U,\varphi)$ 和 $(\wtilde U,\tilde\varphi)$ 是两个光滑坐标卡且使得
  $\supp\omega\subseteq U\cap\wtilde U$。如果它们都是正定向的或者都是反定向的,
  那么 $\tilde\varphi\circ\varphi^{-1}$ 是从 $\varphi\bigl( U\cap\wtilde U\bigr)$
  到 $\tilde\varphi\bigl( U\cap\wtilde U\bigr)$ 的保定向微分同胚,所以
  \autoref{prop:change of variables for top form on open set} 表明
  \begin{align*}
    \int_{\tilde\varphi(\wtilde U)}\bigl(\tilde{\varphi}^{-1}\bigr)^*\omega
    &=\int_{\tilde{\varphi}(U\cap\wtilde U)} \bigl(\tilde{\varphi}^{-1}\bigr)^*\omega
    =\int_{\varphi(U\cap\wtilde U)} 
    \bigl(\tilde\varphi\circ\varphi^{-1}\bigr)^*\bigl(\tilde\varphi^{-1}\bigr)^*\omega\\
    &=\int_{\varphi(U\cap\wtilde U)} (\varphi^{-1})^*\omega 
    =\int_{\varphi(U)} (\varphi^{-1})^*\omega.
  \end{align*}
  如果这两个坐标卡的定向相反,那么 $\tilde\varphi\circ\varphi^{-1}$ 是
  反定向的微分同胚,所以上面的等式最后乘以 $-1$ 即可。
\end{proof}

为了在整个流形上进行积分,我们需要使用单位分解组合前面的定义。
令 $M$ 是定向的带边或者无边 $n$-流形,$\omega$ 是 $M$ 上的紧支 $n$-形式。
令 $\{U_i\}$ 是由正定向或者负定向的光滑坐标卡组成的 $\supp\omega$ 的一个
有限开覆盖,$\{\psi_i\}$ 是从属于这个开覆盖的光滑单位分解。定义
$\omega$ 在 $M$ 上的\emph{积分}为
\begin{equation}
  \int_M\omega=\sum_i\int_M\psi_i\omega.
\end{equation}
因为对于每个 $i$,$n$-形式 $\psi_i\omega$ 在 $U_i$ 中是紧支的,
所以上述求和的每一项都是我们前面所定义的。

\begin{proposition}
  定义 $\int_M\omega$ 不依赖于开覆盖或者单位分解的选取。
\end{proposition}
\begin{proof}
  假设 $\{\wtilde U_j\}$ 是另一个由正定向或者负定向的光滑坐标卡组成的
  $\supp\omega$ 的有限开覆盖,$\{\tilde\psi_j\}$ 是从属于这个开覆盖的光滑单位分解。
  对于每个 $i$,我们有
  \[
    \int_M\psi_i\omega=\int_M\left(\sum_j\tilde\psi_j\right)\psi_i\omega
    =\sum_j \int_M \tilde\psi_j\psi_i\omega.
  \]
  对 $i$ 求和,我们得到
  \[
    \sum_i\int_M\psi_i\omega=\sum_{i,j}\int_M\tilde{\psi}_j\psi_i\omega.
  \]
  注意到每一项的积分 $\int_M \tilde\psi_j\psi_i\omega$ 都是在某个单个 
  光滑坐标卡 (例如 $U_i$) 中紧支的形式 $\tilde\psi_j\psi_i\omega$ 的积分,
  所以根据 \autoref{prop:well-definedness of integral on manifold},每一项
  都是良定义的,与坐标映射的选取无关。从 $\int_M\tilde\psi_j\omega$
  出发,做同样的计算,我们得到
  \[
    \sum_j\int_M\tilde\psi_j\omega=\sum_{i,j}\int_M\tilde{\psi}_j\psi_i\omega.
  \]
  因此,两种定义都给出了 $\int_M\omega$ 的相同值。
\end{proof}

令 $S\subseteq M$ 是定向的浸入 $k$-子流形(带边或者无边),$\omega$
是 $M$ 上的 $k$-形式且限制到 $S$ 上是紧支的。我们将 $\int_S\omega$
解释为 $\int_S\iota_S^*\omega$,其中 $\iota:S\hookrightarrow M$
是包含映射。特别的,如果 $M$ 是紧的、定向的、带边的光滑 $n$-流形,
$\omega$ 是 $M$ 上的 $(n-1)$-形式,我们可以把 $\int_{\partial M}\omega$
无歧义地解释为 $\int_{\partial M}\iota_{\partial M}^*\omega$,其中
$\partial M$ 总是理解为附带诱导定向。

\begin{proposition}[形式积分的性质]
  设 $M$ 和 $N$ 是非空的定向的带边或者无边 $n$-流形,$\omega$ 和 $\eta$
  是 $M$ 上的紧支 $n$-形式。
  \begin{enumerate}
    \item \textbf{线性性}:如果 $a,b\in\mathbb{R}$,那么
    \[
      \int_M a\omega+b\eta=a\int_M \omega+b\int_M \eta.
    \]
    \item \textbf{定向反转性}:如果 $-M$ 是 $M$ 的反定向,那么
    \[
      \int_{-M}\omega=-\int_M \omega.
    \]
    \item \textbf{正定性}:如果 $\omega$ 是正定向的定向形式,那么
    $\int_M\omega>0$。
    \item \textbf{微分同胚不变性}:如果 $F:N\to M$ 是保定向或者反定向的微分同胚,那么
    \[
      \int_M\omega=\begin{dcases}
        \int_N F^*\omega & \text{如果 $F$ 保定向},\\
        -\int_N F^*\omega & \text{如果 $F$ 反定向}.
      \end{dcases}
    \]
  \end{enumerate}
\end{proposition}
\begin{proof}
  (1) 显然 $a\omega+b\eta$ 也是紧支的 $n$-形式,因为 
  $\supp(a\omega+b\eta)\subseteq \supp\omega\cup\supp\eta$
  是紧集的闭子集,从而是紧集。任取 $\supp\omega\cup\supp\eta$ 的一个由正定向或者负定向坐标卡
  组成的有限开覆盖 $\{U_i\}$ 以及从属于它的光滑单位分解 $\{\psi_i\}$。那么
  $\{U_i\}$ 同时是 $\supp(a\omega+b\eta),\supp\omega,\supp\eta$ 的有限开覆盖。
  因此,我们有
  \begin{align*}
    \int_M a\omega+b\eta&=\sum_i\int_M \psi_i(a\omega+b\eta)
    =\sum_i \left(a\int_M \psi_i\omega+b\int_M \psi_i\eta\right)\\
    &=a\int_M \omega+b\int_M \eta.
  \end{align*}

  (2) 取 $\supp\omega$ 的一个由正定向或者负定向坐标卡组成的有限开覆盖 $\{U_i\}$
  以及从属于它的光滑单位分解 $\{\psi_i\}$。那么在 $-M$ 上,每个 $U_i$ 的定向性与之前相反,
  从而每个积分项满足 $\int_M \psi_i\omega=-\int_{-M}\psi_i\omega$,故最终的结果相差
  一个负号。

  (3) $\omega$ 是正定向的定向形式意味着对于每个正定向的坐标卡 $(U,\varphi)$,
  $(\varphi^{-1})^*\omega$ 是一个正值函数乘以 $\d x^1\wedge\cdots\wedge \d x^n$,
  对于负定向的坐标卡则是负值函数乘以 $\d x^1\wedge\cdots\wedge \d x^n$。因此,
  $\int_M\omega$ 的每一个求和项 $\int_M\psi_i\omega$ 都是正数,所以 $\int_M\omega>0$。

  (4) 只需要假设 $\omega$ 是在单个正定向或者负定向坐标卡中紧支的即可,因为
  $M$ 上的任意紧支 $n$-形式都可以凭借单位分解写成这样的形式的有限和。
  因此,假设 $(U,\varphi)$ 是正定向的坐标卡,其包含 $\supp\omega$。当
  $F$ 保定向的时候,容易验证 $\bigl(F^{-1}(U),\varphi\circ F\bigr)$
  是 $N$ 上的正定向坐标卡,且包含 $\supp F^*\omega$。最后根据
  \autoref{prop:change of variables for top form on open set} 即可。
\end{proof}

虽然通过单位分解定义形式的积分非常便于理论分析,但是在实际计算中基本无法发挥作用。
一个一般性的难点是很难显式地写出一个光滑单位分解,并且即使写出了,我们也必须
计算多个积分项的和。

出于计算的目的,一种更方便的方法是将流形划分为有限多片,其中每一片的边界
都是零测集,并且利用局部参数化计算每一片上的积分。

\begin{proposition}[参数化上的积分]\label{prop:integration on manifold by parametrization}
  令 $M$ 是定向的带边或者无边 $n$-流形,$\omega$ 是 $M$ 上的紧支 $n$-形式。
  假设 $D_1,\dots,D_k$ 是 $\mathbb{R}^n$ 中的开积分区域,并且对于
  $i=1,\dots,k$,有光滑映射 $F_i:\bar D_i\to M$ 使得
  \begin{enumerate}
    \item $F_i$ 限制到 $D_i$ 到某个开子集 $W_i\subseteq M$ 上时是保定向的微分同胚;
    \item 当 $i\neq j$ 时,有 $W_i\cap W_j=\emptyset$;
    \item $\supp\omega\subseteq \wbar W_1\cup\cdots\cup\wbar W_k$。
  \end{enumerate}
  那么
  \begin{equation}
    \int_M\omega=\sum_{i=1}^k \int_{D_i} F_i^*\omega.
  \end{equation}
\end{proposition}
\begin{proof}
  与前一个证明一样,只需要假设 $\omega$ 在某个正定向或者负定向坐标卡
  $(U,\varphi)$ 中紧支即可。实际上,通过充分收缩坐标卡,我们可以假设 $U$
  是预紧的,$Y=\varphi(U)$ 是 $\mathbb{R}^n$ 或者 $\mathbb{H}^n$ 中的积分区域,
  $\varphi$ 延拓为 $\bar U$ 到 $\bar Y$ 的微分同胚。

  对于每个 $i$,定义开子集 $A_i\subseteq D_i$,$B_i\subseteq W_i$ 和 
  $C_i\subseteq Y$ 为 
  \[
    A_i=F_i^{-1}(W_i\cap U),\quad B_i=U\cap W_i=F_i(A_i),
    \quad C_i=\varphi(B_i)=\varphi\bigl(F_i(A_i)\bigr).
  \]
  因为 $\bar D_i$ 是紧的,所以可以直接验证得 $\partial W_i\subseteq F_i(\partial D_i)$,
  因此 $\partial W_i$ 是 $M$ 中的零测集,$\partial C_i=\varphi(\partial B_i)$
  是 $\mathbb{R}^n$ 中的零测集。

  $\bigl(\varphi^{-1}\bigr)^*\omega$ 的支集被 $\bar C_1\cup\cdots\cup \bar C_k$
  包含,并且其中任意两个集合仅在它们的边界上相交,所以是零测集。因此,我们有 
  \[
    \int_M\omega=\int_Y\bigl(\varphi^{-1}\bigr)^*\omega=\sum_{i=1}^k \int_{C_i}\bigl(\varphi^{-1}\bigr)^*\omega.
  \]
  由于 $\varphi\circ F_i:A_i\to C_i$ 是微分同胚,所以
  \[
    \int_{C_i}\bigl(\varphi^{-1}\bigr)^*\omega=\int_{A_i} F_i^*\varphi^*\bigl(\varphi^{-1}\bigr)^*\omega
    =\int_{A_i} F_i^*\omega=\int_{D_i} F_i^*\omega.
  \]
  将这些等式代入前面的求和式中,即可得到所需的结果。
\end{proof}

\begin{example}
  我们计算 $\mathbb{S}^2$ 上 $2$-形式的积分,配备 $\bar{\mathbb{B}}^3$ 的边界定向。
  令 $\omega$ 是 $\mathbb{R}^3$ 上的 $2$-形式
  \[
    \omega=x\d y\wedge \d z+y\d z\wedge \d x+z\d x\wedge \d y.
  \]
  令 $D$ 是开矩形 $(0,\pi)\times (0,2\pi)$,$F:\bar D\to \mathbb{S}^2$
  是球坐标参数化 
  \[ 
    F(\varphi,\theta)=(\sin\varphi\cos\theta,\sin\varphi\sin\theta,\cos\varphi).
  \]
  \autoref{exa:orientation of sphere} 表明 $F|_D$ 是保定向的,所以满足
  \autoref{prop:integration on manifold by parametrization} 的条件。注意到 
  \begin{align*}
    F^*\d x&=\cos\varphi\cos\theta\d \varphi-\sin\varphi\sin\theta\d \theta,\\
    F^*\d y&=\cos\varphi\sin\theta\d \varphi+\sin\varphi\cos\theta\d \theta,\\
    F^*\d z&=-\sin\varphi\d \varphi,
  \end{align*}
  因此,有
  \begin{align*}
    \int_{\mathbb{S}^2}\omega&=\int_D\bigl(
      -\sin^3\varphi\cos^2\theta\d\theta\wedge\d\varphi
      +\sin^3\varphi\sin^2\theta\d\varphi\wedge\d\theta
    \bigr.\\
    &=\bigl.
      +\cos^2\varphi\sin\varphi\cos^2\theta\d\varphi\wedge\d\theta
      -\cos^2\varphi\sin\varphi\sin^2\theta\d\theta\wedge\d\varphi
    \bigr)\\
    &=\int_D\sin\varphi\d\varphi\wedge\d\theta=
    \int_0^{2\pi}\int_0^\pi \sin\varphi\d\varphi\d\theta=4\pi.
  \end{align*}
\end{example} 

\section{Stokes 定理}

\begin{theorem}[Stokes 定理]
  令 $M$ 是定向的带边光滑 $n$-流形,$\omega$ 是 $M$ 上的紧支 $(n-1)$-形式。
  那么 
  \begin{equation}
    \int_M\d\omega=\int_{\partial M}\omega.
  \end{equation}
\end{theorem}




