
\chapter{张量}

\section{多重线性代数}

\subsection{向量空间的张量积}

\begin{proposition}[抽象的张量积 vs. 具体的张量积]
  如果 $V_1,\dots,V_k$ 是有限维向量空间,那么存在典范同构:
  \[
    V_1^*\otimes \cdots\otimes V_k^*\simeq \LL(V_1,\dots,V_k;\mathbb{R}),
  \]
  其中抽象的单张量 $\omega^1\otimes \cdots\otimes\omega^k$ 对应到
  多重线性映射
  \[
    \omega^1\otimes\cdots\otimes\omega^k(v_1,\dots,v_k)=
    \omega^1(v_1)\cdots \omega^k(v^k).  
  \]
\end{proposition}
\begin{proof}
  定义映射 $\varPhi:V_1^*\times\cdots\times V_k^*\to \LL(V_1,\dots,V_k;\mathbb{R})$
  为
  \[
    \varPhi\bigl(\omega^1,\dots,\omega^k\bigr)(v_1,\dots,v_k)
    =\omega^1(v_1)\cdots\omega^k(v_k).
  \]
  不难验证 $\varPhi$ 是多重线性映射,所以诱导了唯一的线性映射 
  $\tilde{\varPhi}$ 满足
  \[
    \tilde\varPhi\bigl(\omega^1\otimes\cdots\otimes\omega^k\bigr)(v_1,\dots,v_k)
    =\omega^1(v_1)\cdots\omega^k(v_k).
  \]
  线性映射 $\tilde{\varPhi}$ 将 $V_1^*\otimes\cdots\otimes V_k^*$ 的基 $\varepsilon_{(1)}^{i_1}\otimes\cdots\otimes \varepsilon_{(k)}^{i_k}$
  送到 $\LL(V_1,\dots,V_k;\mathbb{R})$ 的基  $\varepsilon_{(1)}^{i_1}\otimes\cdots\otimes \varepsilon_{(k)}^{i_k}$,
  所以是同构映射。
\end{proof}

\subsection{向量空间上的协变张量和逆变张量}

$V$ 是有限维向量空间,\emph{$V$ 上的协变 $k$-张量}指的是
$k$ 重张量积 $V^*\otimes\cdots\otimes V^*$ 中的元素,我们可以将其视为
$V$ 上的 $k$-多重线性映射
\[
  \alpha:V\times\cdots\times V\to \mathbb{R},
\]
$k$ 被称为\emph{$\alpha$ 的秩}。出于惯例,将 $0$-张量定义为实数。我们将 $V$ 上的所有协变 $k$-张量构成的
向量空间简记为
\[
  T^k(V^*)=V^*\otimes\cdots\otimes V^*.
\]

\begin{example}[协变张量]
  令 $V$ 是有限维向量空间。
  \begin{enumerate}
    \item 每个线性泛函 $\omega:V\to \mathbb{R}$ 都是一个协变 $1$-张量,
    由于 $\omega$ 可以视为 $T_a^*V\simeq V^*$ 的元素,所以协变 $1$-张量
    就是余向量。
    \item $V$ 上的协变 $2$-张量就是\emph{双线性型},例如 $\mathbb{R}^2$
    上的内积。
    \item 将行列式视为参数为 $n$ 个向量的函数,是 $\mathbb{R}^n$ 上的协变 $n$-张量。
  \end{enumerate}
\end{example}

定义\emph{$V$ 上秩 $k$ 的逆变张量}空间为向量空间
\[
  T^k(V)=V\otimes \cdots\otimes V.  
\]
特别地,$T^1(V)=V$,同样出于惯例,定义 $T^0(V)=\mathbb{R}$。
逆变 $k$-张量可以被视为 $V^*$ 上的 $k$-多重线性映射
\[
  \alpha:V^*\times \cdots\times V^*\to \mathbb{R} . 
\]
但是对于大多数目的来说,将逆变张量视为抽象的张量积空间中的元素更加简单。

更一般的,定义\emph{$V$ 上的 $(k,l)$ 型混合张量}空间为
\[
  T^{(k,l)}(V)=T^k(V)\otimes T^l(V^*)=\underbrace{V\otimes\cdots\otimes V}_{\text{$k$ copies}}  
  \otimes \underbrace{V^*\otimes\cdots\otimes V^*}_{\text{$l$ copies}}.
\]
上述定义的一些特殊情况为:
\begin{align*}
  T^{(0,0)}(V)&=T^0(V)=T^0(V^*)=\mathbb{R},\\
  T^{(0,1)}(V)&=T^1(V^*)=V^*,\\
  T^{(1,0)}(V)&=T^1(V)=V,\\
  T^{(0,k)}(V)&=T^k(V^*),\\
  T^{(k,0)}(V)&=T^k(V).
\end{align*}

在本书中,我们主要关注协变张量。因此,除非我们明确指定,否则张量
将始终被理解为协变张量。然而,重要的是要意识到逆变和混合张量
在微分几何的更高级部分中发挥着重要作用,特别是黎曼几何。

\section{对称和交错张量}

一般来说,重新排列协变张量的参数对其值造成的影响不必有什么规律。
然而,一些特殊的张量(例如内积)在重新排列参数时值保持不变(内积的对称性)。
对于行列式而言,互换两个参数,行列式会改变符号。在本节中,我们描述这两类
重要的张量:对称张量和交错张量。当它们的参数重新排列时,会以最简单的方式发生变化。

\subsection{对称张量}

令 $V$ 是有限维向量空间。$V$ 上的协变 $k$-张量 $\alpha$ 如果交换任意两个参数
的值都保持不变:
\[
  \alpha(v_1,\dots,v_i,\dots,v_j,\dots,v_k)=
  \alpha(v_1,\dots,v_j,\dots,v_i,\dots,v_k),
\]
其中 $1\leq i<j\leq k$,那么我们说 $\alpha$ 是\emph{对称的}。

令 $S_k$ 是 $k$ 阶对称群,任取 $\sigma\in S_k$,我们记
$k$-张量 $\sigma\alpha$ 为
\[
  \sigma\alpha(v_1,\dots,v_k)=\alpha(v_{\sigma(1)},\dots,v_{\sigma(k)}).  
\]
不难验证 $\tau(\sigma\alpha)=(\tau\sigma)\alpha$。由于
$S_k$ 可以由所有的二轮换生成,所以 $\alpha$ 是对称张量当且仅当
任取 $\sigma\in S_k$ 有 $\sigma\alpha=\alpha$。

所有对称 $k$-张量的集合构成 $T^k(V^*)$ 的一个子空间,我们记为
$\Sigma^k(V^*)$。我们可以定义投影映射 $\Sym:T^k(V^*)\to \Sigma^k(V^*)$
为
\[
  \Sym \alpha=\frac{1}{k!}\sum_{\sigma\in S_k}\sigma\alpha.  
\]
这个映射被称为\emph{对称化}。

\begin{proposition}[对称化的性质]
  令 $\alpha$ 是一个协变张量。
  \begin{enumerate}
    \item $\Sym\alpha$ 是对称张量。
    \item $\alpha$ 是对称张量当且仅当 $\Sym\alpha=\alpha$。
  \end{enumerate}
\end{proposition}
\begin{proof}
  任取 $\alpha\in T^k(V^*)$,$\tau\in S_k$,那么
  \begin{align*}
    (\Sym\alpha)(v_{\tau(1)},\dots,v_{\tau(k)})&=
    \frac{1}{k!}\sum_{\sigma\in S_k}\sigma\alpha(v_{\tau(1)},\dots,v_{\tau(k)})\\
    &=\frac{1}{k!}\sum_{\sigma\in S_k}(\sigma\tau)\alpha(v_1,\dots,v_k)\\
    &=\frac{1}{k!}\sum_{\eta\in S_k}\eta\alpha(v_1,\dots,v_k)\\
    &=(\Sym\alpha)(v_1,\dots,v_k),
  \end{align*}
  所以 $\Sym\alpha$ 是对称张量。

  如果 $\alpha$ 是对称张量,那么任取 $\sigma\in S_k$ 有 $\sigma\alpha=\alpha$,
  所以 $\Sym\alpha=\alpha$。反之,如果 $\Sym\alpha=\alpha$,由 (1)
  可知 $\alpha$ 是对称张量。
\end{proof}

如果 $\alpha$ 和 $\beta$ 是 $V$ 上的对称张量,那么 $\alpha\otimes\beta$
一般不是对称张量。但是,使用对称化算子,可以定义一种新的乘法,将
一对对称张量送到一个对称张量。如果 $\alpha\in \Sigma^k(V^*)$,
$\beta\in \Sigma^l(V^*)$,我们定义\emph{对称积} $\alpha\beta$
为 $(k+l)$-张量
\[
  \alpha\beta=\Sym(\alpha\otimes\beta).  
\]

\begin{proposition}[对称积的性质]
  \mbox{}
  \begin{enumerate}
    \item 对称积是对称的且双线性的:对于对称张量 $\alpha,\beta,\gamma$
    和 $a,b\in \mathbb{R}$,有
    \begin{align*}
      \alpha\beta&=\beta\alpha,\\
      (a\alpha+b\beta)\gamma&=a\alpha\gamma+b\beta\gamma=
      \gamma(a\alpha+b\beta).
    \end{align*}
    \item 如果 $\alpha,\beta$ 是余向量,那么
    \[
      \alpha\beta=\frac{1}{2}(\alpha\otimes\beta+\beta\otimes\alpha).  
    \]
  \end{enumerate}
\end{proposition}

\subsection{交错张量}

$V$ 上的协变 $k$-张量 $\alpha$ 如果满足:对于任意不同的指标 $i,j$,有
\[
  \alpha(v_1,\dots,v_i,\dots,v_j,\dots,v_k)=
  -\alpha(v_1,\dots,v_j,\dots,v_i,\dots,v_k),
\]
那么我们说 $\alpha$ 是\emph{交错张量}。所有交错 $k$-张量
构成的子空间记为 $\Lambda^k(V^*)\subseteq T^k(V^*)$。
此时,可以验证对于 $\sigma\in S_k$,有 $\sigma\alpha=(\sgn\sigma)\alpha$。
交错的协变 $k$-张量也被称为\emph{外形式},\emph{多重余向量}或者\emph{$k$-余向量}。

每个 $0$-张量(实数)和 $1$-张量都是对称且交错的。一个交错 $2$-张量
就是反对称的双线性型(交错型)。


\section{流形上的张量和张量场}

令 $M$ 是光滑流形,定义 \emph{$M$ 上的协变 $k$-张量丛}为
\[
  T^kT^*M=\coprod_{p\in M}T^k\bigl(T_p^*M\bigr).  
\] 
类似地,定义 \emph{$M$ 上的逆变 $k$-张量丛}为
\[
  T^kTM=\coprod_{p\in M}T^k\bigl(T_pM\bigr)  .
\]
\emph{$M$ 上的 $(k,l)$ 型混合张量丛}为
\[
  T^{(k,l)}TM=\coprod_{p\in M}  T^{(k,l)}\bigl(T_pM\bigr).
\]

上述这些丛中任意一个都被称为\emph{$M$ 上的张量丛}。张量丛
的截面被称为\emph{$M$ 上的(协变、逆变或者混合)张量场}。
使用上面的定义,我们发现逆变 $1$-张量场就是向量场,
协变 $1$-张量场就是余向量场。$0$-张量场就是连续实值函数。

这些张量场的光滑截面构成的空间 
$\Gamma\bigl(T^kT^*M\bigr),\Gamma\bigl(T^kTM\bigr),\Gamma\bigl(T^{(k,l)}TM\bigr)$
是 $\mathbb{R}$ 上的无限维向量空间,也是 $C^\infty(M)$ 环上的模。
在任意光滑局部坐标 $\left(x^i\right)$ 下,这些丛的截面
可以表示为
\[
  A=\begin{dcases}
    A_{i_1\dots i_k}dx^{i_1}\otimes\cdots\otimes dx^{i_k} & A\in \Gamma\bigl(T^kT^*M\bigr),\\
    A^{i_1\dots i_k}\frac{\partial}{\partial x^{i_1}}\otimes
    \cdots\otimes \frac{\partial}{\partial x^{i_k}} & A\in \Gamma\bigl(T^kTM\bigr),\\
    A_{j_1\dots j_l}^{i_1\dots i_k}\frac{\partial}{\partial x^{i_1}}\otimes
    \cdots\otimes \frac{\partial}{\partial x^{i_k}}\otimes
    dx^{j_1}\otimes\cdots\otimes dx^{j_l} & A\in \Gamma\bigl(T^{(k,l)}TM\bigr).
  \end{dcases}  
\]
函数 $A_{i_1\dots i_k},A^{i_1\dots i_k},A_{j_1\dots j_l}^{i_1\dots i_k}$ 被称为
\emph{$A$ 的分量函数}。由于我们主要关注光滑协变张量场,所以我们对于
光滑协变 $k$-张量场,采用以下简写:
\[
  \mathcal{T}^k(M)=\Gamma\bigl(T^kT^*M\bigr)  .
\]

\begin{proposition}[张量场的光滑性判别]
  $M$ 是光滑流形,$A:M\to T^kT^*M$ 是截面,下面的说法等价。
  \begin{enumerate}
    \item $A$ 光滑。
    \item 在每个光滑坐标卡中,$A$ 的分量函数都是光滑的。
    \item $M$ 的每个点处都存在一个光滑坐标卡使得 $A$ 的分量函数
    在这上面光滑。
    \item 如果 $X_1,\dots,X_k\in \mathfrak{X}(M)$,函数 $A(X_1,\dots,X_k):M\to \mathbb{R}$
    \[
      A(X_1,\dots,X_k)(p)=A_p\bigl(X_1|_p,\dots,X_k|_p\bigr)
    \]
    是光滑的。
    \item 对于任意定义在开集 $U\subseteq M$ 上的光滑向量场 $X_1,\dots,X_k$,
    函数 $A(X_1,\dots,X_k)$ 在 $U$ 上光滑。
  \end{enumerate}
\end{proposition}

\begin{proposition}
  $M$ 是光滑流形,$A\in \mathcal{T}^k(M)$,$B\in \mathcal{T}^l(M)$,
  $f\in C^\infty(M)$。那么 $fA$ 和 $A\otimes B$ 也是光滑张量场,其在
  光滑坐标卡中的分量为
  \begin{align*}
    (fA)_{i_1\dots i_k}&= fA_{i_1\dots i_k},\\
    (A\otimes B)_{i_1\dots i_{k+l}}&=A_{i_1\dots i_k}B_{i_{k+1}\dots i_{k+l}}.
  \end{align*}
\end{proposition}

若 $A$ 是 $M$ 上的光滑协变 $k$-张量场,$X_1,\dots,X_k$ 是光滑向量场,
那么 $A(X_1,\dots,X_k)$ 是 $M$ 上的光滑函数,因此 $A$ 诱导了映射
\[
  \mathfrak{X}(M)\times\cdots\times \mathfrak{X}(M)\to C^\infty(M).  
\]
不难发现这个映射在 $\mathbb{R}$ 上是多重线性的。实际上,它在 
\emph{$C^\infty(M)$ 上是多重线性的},也就是说任取 $f,f'\in C^\infty(M)$,
$X_i,X_i'\in \mathfrak{X}(M)$,我们有
\begin{align*}
  &A(X_1,\dots,fX_i+f'X_i',\dots,X_k)\\
  &=fA(X_1,\dots,X_i,\dots,X_k)+f'A(X_1,\dots,X_i',\dots,X_k).
\end{align*}
这个性质实际上是光滑张量场的特征。

\begin{lemma}[张量表征引理]
  映射
  \begin{align*}
    \mathcal{A}:\underbrace{\mathfrak{X}(M)\times\cdots\times \mathfrak{X}(M)}
    _{\text{$k$ copies}}\to C^\infty(M)
  \end{align*}
  是由一个光滑协变 $k$-张量场诱导的当且仅当其在 $C^\infty(M)$ 上是多重线性的。
\end{lemma}
\begin{proof}
  我们已经说明了必要性,下面说明充分性。假设 $\mathcal{A}$ 在 $C^\infty(M)$ 上是多重线性的。
  我们希望定义一个张量场 $A:M\to T^kT^*M$ 为
  \[
    A_p(v_1,\dots,v_k)=\mathcal{A}(V_1,\dots,V_k)(p),  
  \]
  其中 $p\in M$,$v_1,\dots,v_k\in T_pM$,$V_1,\dots,V_k$ 是 $v_1,\dots,v_k$
  的任意延拓到 $M$ 上的光滑向量场。此时容易验证 $A$ 是光滑的张量场且
  诱导了 $\mathcal{A}$。下面我们只需要说明 $A$ 是良定义的,即与延拓的选取无关。

  首先说明 $\mathcal{A}$ 是局部的映射。如果 $X_i$ 是光滑向量场,其在 $p$
  的某个邻域 $U$ 中为零,那么我们可以选取支在 $U$ 中的鼓包函数 $\psi$
  满足 $\psi(p)=1$,此时 $\psi X_i\equiv 0$,所以
  \[
    0=\mathcal{A}(X_1,\dots,\psi X_i,\dots,X_k)(p)= 
  \psi(p)\mathcal{A}(X_1,\dots,X_i,\dots,X_k)(p),
  \]
  所以 $\mathcal{A}(X_1,\dots,X_k)$ 在 $p$ 处为零,这表明 $\mathcal{A}(X_1,\dots,X_k)$
  在 $p$ 处的取值只与 $X_1,\dots,X_k$ 在 $p$ 的某个邻域上的取值有关。

  然后我们说明 $\mathcal{A}$ 实际上只与 $X_1,\dots,X_k$ 在 $p$ 点处的值有关。
  若 $X_i|_p=0$,那么在 $p$ 处的任意坐标邻域中可以写为 $X_i=X_i^j\partial/\partial x^j$,
  其中分量函数 $X_i^j(p)=0$。我们可以将向量场 $\partial/\partial x^j$
  延拓为 $M$ 上的光滑向量场 $E_j$ 使得 $E_j$ 在这个邻域中有 $E_j=\partial/\partial x^j$。
  同样的,可以将光滑函数 $X_i^j$ 延拓为 $M$ 上的光滑函数 $f_i^j$ 使得
  $f_i^j$ 在这个邻域中有 $f_i^j=X_i^j$。那么在这个邻域中有 $f_i^jE_j=X_i$,
  根据 $\mathcal{A}$ 的多重线性性以及局部性,就有
  \begin{align*}
    \mathcal{A}(X_1,\dots,X_i,\dots,X_k)(p)&=
    \mathcal{A}(X_1,\dots,f_i^jE_j,\dots,X_k)(p)\\
    &=f_i^j(p)\mathcal{A}(X_1,\dots,E_j,\dots,X_k)(p)=0.
  \end{align*} 
  这就表明 $\mathcal{A}$ 只与 $X_1,\dots,X_k$ 在 $p$ 点处的值有关。
\end{proof}

更一般地,若 $F$ 是 $M$ 上的光滑 $(k,l)$-张量场,给定光滑
余向量场 $\omega^1,\dots,\omega^k\in \mathfrak{X}^*(M)$ 以及
光滑向量场 $X_1,\dots,X_l\in \mathfrak{X}(M)$,此时可以定义映射
\[
  \mathcal{F}:\underbrace{\mathfrak{X}^*(M)\times\cdots\times \mathfrak{X}^*(M)}
  _{\text{$k$ copies}}\times \underbrace{\mathfrak{X}(M)\times\cdots\times \mathfrak{X}(M)}
  _{\text{$l$ copies}}\to C^\infty(M)
\]
满足
\[
  \mathcal{F}(\omega^1,\dots,\omega^k,X_1,\dots,X_l)(p)
  =F_p\bigl(\omega^1|_p,\dots,\omega^k|_p,X_1|_p,\dots,X_l|_p\bigr)  .
\]
不难验证 $\mathcal{F}$ 是 $C^\infty(M)$-多重线性的。与上面的证明完全类似的,
可以证明这样的映射 $\mathcal{F}$ 是 $(k,l)$-张量场诱导的当且仅当
$\mathcal{F}$ 是 $C^\infty(M)$-多重线性的。

然后我们再介绍一个张量场的重要的表征。

\begin{lemma}
  映射
  \[
    \mathcal{F}:\underbrace{\mathfrak{X}^*(M)\times\cdots\times \mathfrak{X}^*(M)}
    _{\text{$k$ copies}}\times \underbrace{\mathfrak{X}(M)\times\cdots\times \mathfrak{X}(M)}
    _{\text{$l$ copies}}\to \mathfrak{X}(M)
  \]
  是 $C^\infty(M)$-线性的当且仅当其由一个光滑 $(k+1,l)$-张量场诱导。
\end{lemma}
\begin{proof}
  若上述 $\mathcal{F}$ 是 $C^\infty(M)$-线性的,那么可以定义
  一个映射
  \[
    F: \mathfrak{X}^*(M)\times\underbrace{\mathfrak{X}^*(M)\times\cdots\times \mathfrak{X}^*(M)}
    _{\text{$k$ copies}}\times \underbrace{\mathfrak{X}(M)\times\cdots\times \mathfrak{X}(M)}
    _{\text{$l$ copies}}\to C^\infty(M)
  \]
  为 
  \[
    F\bigl(\alpha,\omega^1,\dots,\omega^k,X_1,\dots,X_l\bigr)  
    =\alpha\bigl(\mathcal{F}\bigl(\omega^1,\dots,\omega^k,X_1,\dots,X_l\bigr)\bigr),
  \]
  不难验证 $F$ 是 $C^\infty(M)$-线性的,那么根据张量表征引理,
  $F$ 由一个光滑 $(k+1,l)$-张量场诱导。

  反之,给定一个光滑 $(k+1,l)$-张量场 $F$,其诱导
  一个 $C^\infty(M)$-线性的映射
  \[
    F:  \underbrace{\mathfrak{X}^*(M)\times\cdots\times \mathfrak{X}^*(M)}
    _{\text{$k+1$ copies}}\times \underbrace{\mathfrak{X}(M)\times\cdots\times \mathfrak{X}(M)}
    _{\text{$l$ copies}}\to C^\infty(M).
  \]
  定义映射
  \[
    \mathcal{F}:\underbrace{\mathfrak{X}^*(M)\times\cdots\times \mathfrak{X}^*(M)}
    _{\text{$k$ copies}}\times \underbrace{\mathfrak{X}(M)\times\cdots\times \mathfrak{X}(M)}
    _{\text{$l$ copies}}\to \mathfrak{X}(M)
  \]
  满足对于任意的 $\alpha\in \mathfrak{X}^*(M)$,有
  \[
    \alpha\bigl(\mathcal{F}\bigl(\omega^1,\dots,\omega^k,X_1,\dots,X_l\bigr)  \bigr)  
    =F\bigl(\alpha,\omega^1,\dots,\omega^k,X_1,\dots,X_l\bigr)  .
  \]
  换句话说,任取 $p\in M$,向量 $\mathcal{F}\bigl(\omega^1,\dots,\omega^k,X_1,\dots,X_l\bigr)\big|_p\in T_pM$
  满足
  \begin{align*}
    \alpha_p\bigl(\mathcal{F}\bigl(\omega^1,\dots,\omega^k,X_1,\dots,X_l\bigr)\big|_p\bigr)  
    &=F\bigl(\alpha,\omega^1,\dots,\omega^k,X_1,\dots,X_l\bigr) (p)\\
    &=F_p\bigl(\alpha_p,\omega^1|_p,\dots,\omega^k|_p,X_1|_p,\dots,X_l|_p\bigr),
  \end{align*}
  这表明 $\mathcal{F}\bigl(\omega^1,\dots,\omega^k,X_1,\dots,X_l\bigr)\big|_p$ 由余切向量
  $\alpha_p\in T_p^*M$ 唯一确定,所以 $\mathcal{F}$ 是良定义的。
  不难验证 $\mathcal{F}$ 是 $C^\infty(M)$-线性的。下面我们只需要说明 
  $\mathcal{F}\bigl(\omega^1,\dots,\omega^k,X_1,\dots,X_l\bigr)$ 是光滑向量场即可。
  
  假设在局部坐标 $\bigl(x^i\bigr)$ 中,有 $\alpha=\alpha_i\d x^i$,
  $\omega^j=\omega^j_i\d x^i$,$X_j=X_j^i\partial/\partial x^i$,以及
  \[
    F=F_{j_1\dots j_l}^{i_1\dots i_{k+1}}
    \frac{\partial}{\partial x^{i_1}}\otimes  \cdots
    \otimes \frac{\partial}{\partial x^{i_{k+1}}}\otimes \d x^{j_1}
    \otimes\cdots\otimes \d x^{j_{l}},
  \]
  那么
  \[
    F\bigl(\alpha,\omega^1,\dots\omega^k,X_1,\dots,X_l\bigr)  
    =\alpha_{i_1}\omega^1_{i_2}\cdots\omega^{k}_{i_{k+1}}
    X_1^{j_1}\cdots X_l^{j_l}F_{j_1\dots j_l}^{i_1\dots i_{k+1}},
  \]
  这表明
  \[
    \mathcal{F}\bigl(\omega^1,\dots,\omega^k,X_1,\dots,X_l\bigr)=
    \omega^1_{i_2}\cdots\omega^{k}_{i_{k+1}}
    X_1^{j_1}\cdots X_l^{j_l}F_{j_1\dots j_l}^{i_1i_2\dots i_{k+1}}
    \frac{\partial}{\partial x^{i_1}},
  \]
  这就表明 $\mathcal{F}\bigl(\omega^1,\dots,\omega^k,X_1,\dots,X_l\bigr)$
  是光滑向量场。
\end{proof}

流形上的\emph{对称张量场}指的是一个协变张量场并且其在每个点处的取值都是
对称张量。两个或多个张量场的对称积由逐点张量的对称积定义。例如,
如果 $A,B$ 是两个光滑余向量场,那么它们的对称积指的是一个光滑 $2$-张量场
$AB$,其计算为
\[
  AB=\frac{1}{2}(A\otimes B+B\otimes A).  
\]

交错张量场被称为\emph{微分形式},我们将在后面深入研究。


\subsection{张量场的拉回}

与余向量场类似,协变张量场可以通过光滑映射拉回产生定义域上的张量场。
注意拉回构造仅限于协变张量场,这也是我们更多关注协变张量的原因之一。

设 $F:M\to N$ 是光滑映射。对于任意点 $p\in M$ 和 $k$-张量
$\alpha\in T^k\bigl(T_{F(p)}^*N\bigr)$,我们定义张量
$dF_p^*(\alpha)\in T^k\bigl(T_p^*M\bigr)$ 为
\[
  dF_p^*(\alpha)(v_1,\dots,v_k)=\alpha\bigl(dF_p(v_1),\dots,dF_p(v_k)\bigr)  ,
\]
其中 $v_1,\dots,v_k\in T_pM$,这个 $dF_p^*(\alpha)$ 被称为
\emph{$\alpha$ 通过 $F$ 在 $p$ 处的拉回}。如果 $A$ 是一个
$N$ 上的协变 $k$-张量场,我们定义 $M$ 上的 $k$-张量场
$F^*A$ 为
\[
  (F^*A)  _p=dF_p^*(A_{F(p)}).
\]
这被称为\emph{$A$ 通过 $F$ 的拉回}。这个张量场在
$v_1,\dots,v_k\in T_pM$ 上的作用为
\[
  (F^*A)  _p(v_1,\dots,v_k)=A_{F(p)}\bigl(dF_p(v_1),\dots,dF_p(v_k)\bigr).
\]

\begin{proposition}[拉回张量的性质]\label{prop:property of pullback of tensor}
  设 $F:M\to N$ 和 $G:N\to P$ 是光滑映射,$A,B$ 是 $N$ 上的协变张量场,
  $f$ 是 $N$ 上的实值函数。
  \begin{enumerate}
    \item $F^*(fB)=(f\circ F)F^*B$。
    \item $F^*(A\otimes B)=F^*A\otimes F^*B$。
    \item $F^*(A+B)=F^*A+F^*B$。
    \item $F^*B$ 是张量场并且 $F^*B$ 光滑当且仅当 $B$ 光滑。
    \item $(G\circ F)^*B=F^*(G^*B)$。
    \item $(\Id_N)^*B=B$。
  \end{enumerate}
\end{proposition}
 
如果 $f$ 是连续实值函数($0$-张量场),$B$ 是 $k$-张量场,那么我们定义的
$f\otimes B$ 和 $fB$ 可以等同,$F^*f$ 和 $f\circ F$ 可以等同。

下面的命题是 \autoref{prop:property of pullback of tensor} 的直接结果。
与余向量场拉回的计算在形式上是一致的。

\begin{corollary}
  令 $F:M\to N$ 是光滑映射,$B$ 是 $N$ 上的协变 $k$-张量场。如果 $p\in M$,
  $(y^i)$ 是 $F(p)$ 处的一个光滑坐标,那么 $F^*B$ 在 $p$ 的这个邻域可以表示为
  \begin{align*}
    &F^*\bigl(B_{i_1\dots i_k}\d y^{i_1}\otimes\cdots\otimes \d y^{i_k}\bigr)\\
    &=\bigl(B_{i_1\dots i_k}\circ F\bigr)
    \d\bigl(y^{i_1}\circ F\bigr)\otimes\cdots\otimes
    \d\bigl(y^{i_k}\circ F\bigr).
  \end{align*}
\end{corollary}

\begin{example}
  令 $M=\{(r,\theta)\,|\, r>0,|\theta|<\pi/2\}$,$N=\{(x,y)\,|\, x>0\}$,
  $F:M\to \mathbb{R}^2$ 是光滑映射 $F(r,\theta)=(r\cos\theta,r\sin\theta)$。
  张量场 $A=x^{-2}\d y\otimes\d y$ 通过 $F$ 的拉回可以计算为:
  \begin{align*}
    F^*A&=(r\cos\theta)^{-2}\d(r\sin\theta)\otimes\d (r\sin\theta)\\
    &=(r\cos\theta)^{-2}(\sin\theta\d r+r\cos\theta\d\theta)
    \otimes (\sin\theta\d r+r\cos\theta\d\theta)\\
    &=r^{-2}\tan^2\theta\d r\otimes\d r+
    r^{-1}\tan\theta(\d\theta\otimes\d r+\d r\otimes\d \theta)
    +\d\theta\otimes\d\theta.
  \end{align*}
\end{example}

\section{张量场的李导数}

李导数算符可以延拓到任意秩的张量场上。通常来说,我们仅关注协变张量,对应的逆变或者混合张量的
结果只需要由镜像的操作得到。

假设 $M$ 是光滑流形,$V$ 是 $M$ 上的光滑向量场,$\theta$ 是 $V$ 的流。对于任意 $p\in M$,如果
$t$ 充分接近零,那么 $\theta_t$ 是从 $p$ 的一个邻域到 $\theta_t(p)$ 的一个邻域的微分同胚,所以
$d(\theta_t)_p^*$ 把 $\theta_t(p)$ 处的张量拉回到 $p$ 处的张量:
\begin{align*}
  (\theta_t^*A)_p(v_1,\dots,v_k)&=  d(\theta_t)_p^*\bigl(A_{\theta_t(p)}\bigr)(v_1,\dots,v_k)\\
  &=A_{\theta_t(p)}\bigl(
    d(\theta_t)_p(v_1),\dots,d(\theta_t)_p(v_k)
  \bigr).
\end{align*}

给定 $M$ 上的光滑协变张量场 $A$,我们定义 \emph{$A$ 相对于 $V$ 的李导数},记为 $\mathcal L_VA$:
\begin{equation}
  (\mathcal L_VA)_p=\frac{d}{dt}\bigg|_{t=0}(\theta_t^*A)_p=\lim_{t\to 0}
  \frac{d(\theta_t)_p^*(A_{\theta_t(p)})-A_p}{t}.
\end{equation}




