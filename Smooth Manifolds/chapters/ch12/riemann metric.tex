

\chapter{黎曼度量}

\section{黎曼流形}

令 $M$ 是光滑流形,\emph{$M$ 上的黎曼度量}指的是 $M$ 上的
一个光滑的对称协变 $2$-张量场,并且在每个点处是正定的。
一个\emph{黎曼流形}指的是二元组 $(M,g)$,其中 $M$ 是光滑流形,
$g$ 是 $M$ 上的一个黎曼度量。

如果 $g$ 是黎曼度量,那么对于每个 $p\in M$,$2$-张量 $g_p\in T^2(T_p^*M)$
是 $T_pM$ 上的一个内积。出于这一点,对于 $v,w\in T_pM$,我们通常使用 $\langle v,w\rangle_g$
来表示实数 $g_p(v,w)$。

在任意光滑局部坐标 $\left(x^i\right)$ 下,一个黎曼度量可以被写为:
\[
  g=g_{ij}\d x^i\otimes \d x^j.  
\]
其中 $(g_{ij})$ 是光滑函数的对称正定矩阵。$g$ 的对称性允许我们
将 $g$ 写为下述对称积的形式:
\begin{align*}
  g&=g_{ij}\d x^i\otimes \d x^j\\
  &=\frac{1}{2}\bigl(g_{ij}\d x^i\otimes \d x^j+g_{ji}\d x^i\otimes \d x^j\bigr)\\
  &=\frac{1}{2}\bigl(g_{ij}\d x^i\otimes \d x^j+g_{ij}\d x^j\otimes \d x^i\bigr)\\
  &=g_{ij}\d x^i\d x^j.
\end{align*}

\begin{example}[Euclid 度量]
  最简单的黎曼度量的例子是 $\mathbb{R}^2$ 上的\emph{Euclid 度量}
  $\bar g$,在标准坐标下,其表示为
  \[
    \bar g=\delta_{ij}\d x^i\d x^j.  
  \]
  通常我们把张量 $\alpha$ 与自身的对称积记为 $\alpha^2$,所以 Euclid 度量
  可以写为
  \[
    \bar g = \bigl(\d x^1\bigr)^2+\cdots +\bigl(\d x^n\bigr)^2.
  \]
  对于向量 $v,w\in T_p \mathbb{R}^n$,这导致
  \[
    \bar g_p(v,w)=\delta_{ij}v^iw^j=\sum_{i=1}^nv^iw^i=v\cdot w.
  \]
  换句话说,$\bar g$ 是 $2$-张量场,其在每个点处的取值就是 Euclid 内积。
\end{example}

\begin{example}[积度量]
  如果 $(M,g)$ 和 $(\wwtilde{M},\tilde{g})$ 是黎曼流形,我们可以定义
  积流形 $M\times\wwtilde{M}$ 上的黎曼度量 $\hat g=g\oplus\tilde g$,
  这被称为\emph{积度量},定义为:
  \[
    \hat g\Bigl(\bigl(v,\tilde v\bigr),\bigl(w,\tilde w\bigr)\Bigr)  
    =g(v,w)+\tilde g\bigl(\tilde v,\tilde w\bigr).
  \]
  其中 $\bigl(v,\tilde v\bigr),\bigl(w,\tilde w\bigr)\in T_pM\oplus T_q\wwtilde M\simeq T_{(p,q)}\bigl(M\times\wwtilde M\bigr)$。
  给定 $M$ 的局部坐标 $\bigl(x^1,\dots,x^n\bigr)$ 和 $\wwtilde M$
  的局部坐标 $\bigl(y^1,\dots,y^m\bigr)$,可得 $M\times \wwtilde M$
  的一个局部坐标 $\bigl(x^1,\dots,x^n,y^1,\dots,y^m\bigr)$,
  积度量在这个坐标中可以局部地表示为分块对角阵
  \[
    \bigl(\hat g_{ij}\bigr) =\begin{pmatrix}
      g_{ij} & 0 \\
      0 & \tilde g_{ij}
    \end{pmatrix}.
  \]
\end{example}

\begin{proposition}[黎曼度量的存在性]
  每个带边或者无边光滑流形都存在一个黎曼度量。
\end{proposition}
\begin{proof}
  令 $M$ 是带边或者无边光滑流形,选取覆盖 $M$ 的一族光滑坐标卡 $(U_\alpha,\varphi_\alpha)$。
  在每个坐标系中,都存在一个黎曼度量 $g_\alpha=\varphi_\alpha^*\bar g$,其坐标表示为
  $g_\alpha=\delta_{ij}dx^idx^j$。令 $\{\psi\}_\alpha$ 是从属于开覆盖 $\{U_\alpha\}$
  的单位分解,定义
  \[
    g=\sum_\alpha \psi_\alpha g_\alpha,
  \]
  每一个求和项在 $\supp\psi_\alpha$ 外的时候解释为零。根据局部有限性,在每个点的
  邻域中只有有限个求和项,所以上述表达式定义了一个光滑张量场。这显然是对称的,
  下面只需要检查正定性。如果 $v\in T_pM$ 非零,那么 
  \[
    g_p(v,v)=\sum_\alpha\psi_\alpha(p)g_\alpha|_p(v,v),
  \]
  因为每一个求和项都非负,所以 $g_p(v,v)\geq 0$。此时至少有一个 $\psi_\alpha(p)>0$
  (因为所有的 $\psi_\alpha(p)$ 求和为 $1$),所以至少有一个 $\psi_\alpha(p)g_\alpha|_p(v,v)>0$。
  这就表明 $g$ 确实是一个黎曼度量。
\end{proof}

下面是可以定义在黎曼流形 $(M,g)$ 上的一些几何结构。
\begin{itemize}[noitemsep]
  \item 切向量 $v\in T_pM$ 的\emph{长度}定义为
  \[
    |v|_g=\langle v,v\rangle_g^{1/2}=g_p(v,v)^{1/2}.  
  \]
  \item 两个非零切向量 $v,w\in T_pM$ 的\emph{夹角}被定义为
  满足
  \[
    \cos\theta=\frac{\inn{v,w}_g}{|v|_g|w|_g}
  \]
  的唯一的 $\theta\in [0,\pi]$。
  \item 切向量 $v,w\in T_pM$ 如果满足 $\inn{v,w}_g=0$,那么我们说
  它们是\emph{正交的}。这意味着二者至少有一个零向量,或者夹角为 $\pi/2$。
\end{itemize}

研究黎曼流形的一个非常有用的工具是正交标架。我们说开子集 $U\subseteq M$
上的局部标架 $(E_1,\dots,E_n)$ 是\emph{正交标架},如果在每个点 $p\in U$
处的向量组 $(E_1|_p,\dots,E_n|_p)$ 构成 $T_pM$ 的一组正交基,或者等价地说,
有 $\inn{E_i,E_j}_p=\delta_{ij}$。

\begin{example}
  坐标标架 $\bigl(\partial/\partial x^i\bigr)$ 是 $\mathbb{R}^n$
  的关于 Euclid 度量的一个全局正交标架。
\end{example}


\subsection{拉回度量}

设 $M,N$ 是光滑流形,$g$ 是 $N$ 上的黎曼度量,$F:M\to N$ 是光滑映射。
拉回 $F^*g$ 是 $M$ 上的一个光滑 $2$-张量场。如果 $F^*g$ 是正定的,
那么就是 $M$ 上的黎曼度量,被称为由 $F$ 确定的\emph{拉回度量}。

\begin{proposition}[拉回度量判别法]
  设 $F:M\to N$ 是光滑映射,$g$ 是 $N$ 上的黎曼度量,那么 
  $F^*g$ 是 $M$ 上的黎曼度量当且仅当 $F$ 是光滑浸入。
\end{proposition}
\begin{proof}
  若 $F^*g$ 是 $M$ 上的黎曼度量。任取 $p\in M$,$v\in T_pM$,
  若 $dF_p(v)=0$,那么
  \[
    (F^*g)_p(v,v)=g_{F(p)}\bigl(\d F_p(v),\d F_p(v)\bigr)  =0,
  \]
  $(F^*g)_p$ 正定表明 $v=0$,即 $dF_p$ 是单射,即 $F$ 是光滑浸入。

  反之,若 $F$ 是光滑浸入,那么 $\d F_p$ 是单射,若 $(F^*g)_p(v,v)=0$,
  那么 $g_{F(p)}$ 的正定性表明 $\d F_p(v)=0$,所以 $v=0$,即
  $(F^*g)_p$ 是正定的,所以 $F^*g$ 是 $M$ 上的黎曼度量。
\end{proof}

\begin{example}
  考虑光滑映射 $F: \mathbb{R}^2\to \mathbb{R}^3$ 为
  \[
    F(u,v)=(u\cos v,u\sin v,v).
  \]
  这是一个恰当的单射的光滑浸入,因此是一个嵌入。它的图像被称为\emph{螺旋面}。
  此时拉回度量 $F^*\bar g$ 可以计算为:
  \begin{align*}
    F^*\bar g&=d(u\cos v)^2+d(u\sin v)^2+dv^2\\
    &=(\cos v\d u-u\sin v\d v)^2+(\sin v\d u+u\cos v\d v)^2+\d v^2\\
    &=\cos^2 v\d u^2-2u\sin v\cos v\d u\d v+u^2\sin^2 v\d v^2\\
    &\hphantom{{}=}+\sin^2 v\d u^2+2u\sin v\cos v\d u\d v+u^2\cos^2 v\d v^2 
    +\d v^2\\
    &=\d u^2+(1+u^2)\d v^2.
  \end{align*}
  需要注意,当 $u$ 是实值函数的时候,记号 $\d u^2$ 表示对称积 $\d u\d u$
  而不是 $\d (u^2)$。
\end{example}

\begin{example}\label{exa:polar coordinate of euclid metric}
  我们计算 $\mathbb{R}^2$ 上的 Euclid 度量 $\bar g=\d x^2+\d y^2$
  在极坐标中的表示,那么 $x=r\cos\theta$,$y=r\sin\theta$ 表明
  \begin{align*}
    \bar g&=\d (r\cos\theta)^2+\d (r\sin\theta)^2\\
    &=(\cos\theta\d r-r\sin\theta\d\theta)^2+(\sin\theta\d r+r\cos\theta\d\theta)^2\\
    &=\d r^2+r^2\d\theta^2.
  \end{align*}
\end{example}

如果 $(M,g)$ 和 $(\wwtilde{M},\tilde g)$ 是黎曼流形,
光滑映射 $F:M\to N$ 如果是微分同胚并且使得 $F^*\tilde g=g$,
那么我们说 $F$ 是一个\emph{(黎曼)等距}。如果每个 $p\in M$
有一个邻域 $U$ 使得 $F|_U$ 是 $U$ 到 $\wwtilde M$ 的一个开子集
的等距,那么我们说 $F$ 是一个\emph{局部等距}。等价地说,
$F$ 是一个局部微分同胚且满足 $F^*\tilde{g}=g$。

一个黎曼流形 $(M,g)$ 如果局部等距于 $(\mathbb{R}^n,\bar g)$,那么
$(M,g)$ 被称为\emph{平坦黎曼流形},$g$ 被称为\emph{平坦度量}。

\begin{theorem}\label{thm:flat metric}
  对于一个黎曼流形 $(M,g)$,下面的说法等价:
  \begin{enumerate}
    \item $g$ 是平坦的。
    \item $M$ 的每个点处都存在一个光滑坐标卡,使得 $g$ 的坐标表示为
    $g=\delta_{ij}\d x^i\d x^j$。
    \item $M$ 的每个点处都存在一个光滑坐标卡,使得坐标标架是正交标架。
    \item $M$ 的每个点都被包含在某个可交换正交标架的定义域中。
  \end{enumerate}
\end{theorem}
\begin{proof}
  $(1)\Rightarrow (2)$ 设 $F:M\to \mathbb{R}^n$ 是局部等距,那么
  $F^*\bar g=g$。任取 $p\in M$,$F$ 是局部微分同胚表明存在 $p$ 的邻域 $U$ 使得 $F|_U$
  是 $U\to F(U)$ 的微分同胚,故 $(U,F)$ 是 $p$ 处的一个光滑坐标卡,
  此时 $g$ 的坐标表示为
  \[
    g=(\delta_{ij}\circ F) \d F^i\otimes \d F^j=\sum_{i=1}^n
    \d F^i\otimes \d F^i.
  \]

  $(2)\Rightarrow (3)$ 假设 $p\in M$ 处有坐标卡 $(x^i)$ 使得 $g=\delta_{ij}\d x^i\d x^j$,
  那么
  \[
    \inn{\frac{\partial}{\partial x^i},\frac{\partial}{\partial x^j}}
    =\delta_{kl}\d x^k\d x^l\left(
      \frac{\partial}{\partial x^i},\frac{\partial}{\partial x^j}
    \right)
    =\begin{cases}
      1, & i=j,\\
      0, & i\neq j.
    \end{cases}
  \]
  这就表明 $(\partial/\partial x^i)$ 是正交标架。

  $(3)\Rightarrow (4)$ 坐标标架都是可交换的标架。

  $(4)\Rightarrow (1)$ 需要使用 \autoref{thm:canonical form for vector fields}:如果
  $(E_i)$ 是一个定义在开集 $U\subseteq M$ 上的可交换的正交标架,那么
  每个 $p\in U$ 都被包含在一个光滑坐标卡 $(U,\varphi)$ 中,使得
  坐标标架为 $(E_i)$。注意 
  $\varphi:U\to\varphi(U)$ 是微分同胚,这表明 $\varphi_*E_i=\partial/\partial x^i$,所以
  \[
    \varphi^*\bar g(E_i,E_j)=\bar g(\varphi_*E_i,\varphi_*E_j)
    =\bar g\left(
      \frac{\partial}{\partial x^i},\frac{\partial}{\partial x^j}
    \right)=\delta_{ij}=g(E_i,E_j).
  \]
  这就表明 $\varphi^*\bar g=g$,所以 $\varphi$ 是 $(U,g|_U)$ 和 
  $\varphi(U)$ 配备 Euclid 度量之间的等距。这就表明 $g$ 是平坦的。
\end{proof}

根据定义,存在非平坦的黎曼度量并不显然。实际上,在 $1$-维的情况下,
每个度量都是平坦的。后面我们将利用 \autoref{thm:flat metric}
来说明 $\mathbb R^3$ 中的大部分旋转曲面,包括 $\mathbb S^2$,都不是平坦的。


\subsection{黎曼子流形}

如果 $(M,g)$ 是一个黎曼流形,每个子流形 $S\subseteq M$
都自动继承一个拉回度量 $\iota^*g$,其中 $\iota:S\hookrightarrow M$
是包含映射。根据定义,这意味着对于 $v,w\in T_pS$,有
\[
  (\iota^*g)_p(v,w)=g_{\iota(p)}\bigl(d\iota_p(v),d\iota_p(w)\bigr)
  =g_p(v,w),
\]
注意这里我们通过 $d\iota_p:T_pS\to T_pM$,将 $T_pS$ 视为 $T_pM$
的子空间。所以 $(\iota^*g)_p$ 仅仅是 $g_p$ 在 $T_pS\times T_pS$
上的限制。在这个度量下,$S$ 被称为\emph{$M$ 的黎曼子流形}。

\begin{example}
  $\mathbb{S}^n$ 上由包含映射 $\iota:\mathbb{S}^n\hookrightarrow \mathbb{R}^{n+1}$
  确定的度量 $\mathring{g}=\iota^*\bar g$,这个度量被称为球面上的
  \emph{圆度量}或者\emph{标准度量}。
\end{example}

如果 $(M,g)$ 是黎曼流形,$\iota:S\hookrightarrow M$ 是黎曼子流形,
通常通过局部参数化来计算度量 $\iota^*g$。回顾 \autoref{prop:local parameterization},
存在从开集 $U\subseteq \mathbb{R}^k$ 到 $M$ 的单射浸入 $X$,其像集
为 $S$ 的开子集,并且其逆是 $S$ 的一个坐标映射。因为 $\iota\circ X=X$
(这里我们滥用同一个记号 $X$ 来表示值域为 $S$ 或者 $M$ 的映射),
所以 $\iota^*g$ 的坐标表示就是 $X^*(\iota^*g)=X^*g$。

\begin{example}[图像坐标中的度量]
  令 $U\subseteq \mathbb{R}^n$ 是开集,$S\subseteq \mathbb{R}^{n+1}$
  是光滑函数 $f:U\to \mathbb{R}$ 的图像。映射 $X:U\to \mathbb{R}^{n+1}$
  为 $X(u^1,\dots,u^n)=\bigl(u^1,\dots,u^n,f(u)\bigr)$,那么 $X$
  是 $S$ 的一个全局参数化,所以 $S$ 上的由图像坐标诱导的度量为
  \[
    X^*\bar g=\bigl(\d u^1\bigr)^2+\cdots+\bigl(\d u^n\bigr)^2+\d f^2.
  \]
  例如,$\mathbb{S}^2$ 的上半球面由映射 $X:\mathbb{B}^2\to \mathbb{R}^3$
  \[
    X(u,v)=\left(u,v,\sqrt{1-u^2-v^2}\right)  
  \]
  参数化,那么在这个坐标下,圆度量可以表示为
  \begin{align*}
    \mathring{g}&=X^*\bar g=\d u^2+\d v^2+\left(\frac{-u\d u-v\d v}{\sqrt{1-u^2-v^2}}\right)^2\\
    &=\d u^2+\d v^2+\frac{u^2\d u^2+v^2\d v^2+2uv\d u\d v}{1-u^2-v^2}\\
    &=\frac{(1-v^2)\d u^2+(1-u^2)\d v^2+2uv\d u\d v}{1-u^2-v^2}.
  \end{align*}
\end{example}

\begin{example}[旋转曲面上的诱导度量]\label{exa:metric on surface}
  令 $C$ 是半平面 $\{(r,z)\,|\, r >0\}$ 的 $1$-维嵌入子流形,$S_C$
  是由 $C$ 生成的旋转曲面。为了计算 $S_C$ 上的诱导度量,选取 $C$
  的一个光滑局部参数化 $\gamma(t)=(a(t),b(t))$,记映射
  $X(t,\theta)=\bigl(a(t)\cos\theta,a(t)\sin\theta,b(t)\bigr)$ 是
  $S_C$ 的光滑局部参数化,只要把 $(t,\theta)$ 限制在充分小的开集上即可。
  那么我们可以计算
  \begin{align*}
    X^*\bar g&=\d\bigl(a(t)\cos\theta\bigr)^2+\d\left(a(t)\sin\theta\right)^2
    +\d \bigl(b(t)\bigr)^2\\
    &=\bigl(a'(t)\cos\theta \d t-a(t)\sin\theta \d\theta\bigr)^2\\
    &\hphantom{{}=}+\bigl(a'(t)\sin\theta\d t+a(t)\cos\theta \d\theta\bigr)^2 
    +b'(t)^2\d t^2\\
    &=\bigl(a'(t)^2+b'(t)^2\bigr)\d t^2+a(t)^2\d\theta^2.
  \end{align*}
  特别地,如果 $\gamma$ 是单位速度的曲线,即 $|\gamma'(t)|^2=a'(t)^2+b'(t)^2=1$,
  那么上述结果可以简化为 $\d t^2+a(t)^2\d\theta^2$。

  下面有一些简单的例子。
  \begin{enumerate}
    \item 环面可以视为圆周 $(r-2)^2+z^2=1$ 生成的旋转曲面。使用参数化
    $\gamma(t)=(2+\cos t,\sin t)$,我们得到环面上的诱导度量
    $\d t^2+(2+\cos t)^2\d\theta^2$。
    \item 单位球面(挖去北极点和南极点)是由参数曲线 $\gamma(t)=(\sin t,\cos t)\ (0<t<\pi)$
    生成的旋转曲面,此时诱导度量为 $dt^2+\sin^2 t\d\theta^2$。
    \item 单位圆柱面 $x^2+y^2=1$ 是由参数化直线 $\gamma(t)=(1,t)$ 生成的旋转曲面,
    诱导度量为 $\d t^2+\d\theta^2$。
  \end{enumerate}
\end{example}

观察上例中的最后一条。对于圆柱面的每个局部参数化 $X(t,\theta)=(\cos\theta,\sin\theta,t)$,
诱导度量 $X^*\bar g$ 都是 $(t,\theta)$-平面的 Euclid 度量。换句话说,对于每个
圆柱面中的点 $p$,总有 $X$ 的一个合适的限制给出了 $(\mathbb R^2,\bar g)$ 的某个
开子集和 $p$ 的某个邻域配备诱导度量之间的黎曼等距。因此,圆柱面
上的诱导度量是平坦的。这意味着一个生活在圆柱面上的二维生物不可能通过
局部的几何测量得知其生活的空间是否为 Euclid 空间。这个例子阐述了
确定一个度量是否平坦有时候有一个意外的答案。

要确定什么样的度量是平坦或者不平坦的需要的技术超出了本书的范围。
就像证明两个拓扑空间是否同胚需要寻找拓扑不变量一样,为了证明
两个黎曼流形不是局部等距的,我们必须引入黎曼等距所保持的局部不变量,
然后证明不同的度量有不同的不变量。黎曼度量的一个基本不变量是曲率,
这是一个衡量度量与平坦性相差多远的量。在 Lee 的《黎曼流形导论》中有更深入的介绍。

\begin{proposition}[旋转曲面的平坦性判别]
  令 $C\subseteq H$ 是半平面 $H=\{(r,z)\,|\, r>0\}$ 中的连通的
  $1$-维嵌入子流形,$S_C$ 是 $C$ 生成的旋转曲面。$S_C$ 上的诱导度量
  是平坦的当且仅当 $C$ 是直线的一部分。
\end{proposition}
\begin{proof}
  首先假设 $C$ 是直线的一部分。那么存在常数 $P,Q,K,L$ 且 $P,Q$ 不同时为零,
  使得 $\gamma(t)=(Pt+K,Qt+L)$。通过对 $\gamma$ 进行缩放,我们可以假设 
  $\gamma$ 是单位速度的。如果 $Q=0$,那么 $S_C$ 是平面 $z=L$ 的开子集,因此
  是平坦的。如果 $P=0$,那么 $S_C$ 是圆柱面 $x^2+y^2=K^2$ 的一部分,我们已经
  说明了圆柱面是平坦的。如果 $P,Q\neq 0$,那么 \autoref{exa:metric on surface}
  表明诱导度量为 $\d t^2+(Pt+K)^2\d \theta^2$。在任意点的某个邻域中,
  考虑坐标变换 $(u,v)=\bigl((t+K/P)\cos P\theta,(t+K/P)\sin P\theta\bigr)$,
  这把 Euclid 度量 $\d u^2+\d v^2$ 拉回到 $\d t^2+(Pt+K)^2\d\theta^2$,
  所以这个度量是平坦的。

  反之,假设 $S_C$ 是平坦的,设 $\gamma(t)=(a(t),b(t))$ 是 $C$ 的局部参数化。
  我们可以假设 $\gamma$ 是单位速度的,即 $a'(t)^2+b'(t)^2=1$。此时由 \autoref{exa:metric on surface},
  诱导度量为 $\d t^2+a(t)^2\d\theta^2$。因此,定义局部标架 $(E_1,E_2)$ 为
  \[
    E_1=\frac{\partial}{\partial t},\quad E_2=\frac{1}{a}\frac{\partial}{\partial\theta},
  \]
  这是一组正交标架。任意其他的正交标架 $\bigl(\wtilde{E}_1,\wtilde{E}_2\bigr)$ 可以被表示为
  \begin{align*}
    \wtilde E_1&=uE_1+vE_2=u\frac{\partial}{\partial t}+\frac{v}{a}\frac{\partial}{\partial\theta},\\
    \pm\wtilde E_2&=vE_1-uE_2=v\frac{\partial}{\partial t}-\frac{u}{a}\frac{\partial}{\partial\theta}.
  \end{align*}
  其中 $u,v$ 是关于 $(t,\theta)$ 的光滑函数并且 $u^2+v^2=1$。因为 $S_C$ 是平坦的,根据 
  \autoref{thm:flat metric},存在 $u,v$ 使得 $\bigl(\wtilde E_1,\wtilde E_2\bigr)$
  是可交换的正交标架。根据李括号的坐标公式 \ref{prop:coordinate formula for lie bracket},有
  \begin{align*}
    0=\pm \bigl[\wtilde E_1,\wtilde E_2\bigr]&=
    \left(u\frac{\partial v}{\partial t}+\frac{v}{a}\frac{\partial v}{\partial\theta}
    -v\frac{\partial u}{\partial t}+\frac{u}{a}\frac{\partial u}{\partial\theta}
    \right)\frac{\partial}{\partial t}\\
    &\hphantom{{}=}-\left(
      u\frac{\partial}{\partial t}\left(\frac{u}{a}\right)+\frac{v}{a}\frac{\partial}{\partial \theta}\left(\frac{u}{a}\right)
      +v\frac{\partial}{\partial t}\left(\frac{v}{a}\right)-\frac{u}{a}\frac{\partial}{\partial \theta}\left(\frac{v}{a}\right)
    \right)\frac{\partial}{\partial \theta}.
  \end{align*}
  记 $f_\theta=\partial f/\partial\theta$ 以及 $f_t=\partial f/\partial t$。
  注意到 $u^2+v^2=1$ 表明 $uu_t+vv_t=uu_\theta+vv_\theta=0$,此外,$a$ 只依赖于
  $t$ 表明 $a_\theta=0$ 以及 $a_t=a'$。那么我们得到
  \[
    0=(uv_t-vu_t)\frac{\partial}{\partial t}+\frac{a'-vu_\theta+uv_\theta}{a^2}
    \frac{\partial}{\partial\theta},
  \]
  这表明
  \begin{align}
    uv_t-vu_t&=0,\label{eq:tmp1 in ch metric}\\
    vu_\theta-uv_\theta&=a'\label{eq:tmp2 in ch metric}.
  \end{align}

  因为 $u^2+v^2=1$,所以每个点处都有一个邻域使得 $u$ 或者 $v$ 非零。
  假设 $v\neq 0$,那么 \eqref{eq:tmp1 in ch metric} 表明 $u/v$ 对 $t$ 的偏导数
  是零,所以我们可以假设 $u=fv$,其中 $f$ 是 $\theta$ 的函数。那么 $u^2+v^2=1$
  表明 $v^2(f^2+1)=1$,所以 $v=\pm 1/\sqrt{f^2+1}$ 是 $\theta$ 的函数,所以
  $u=\pm\sqrt{1-v^2}$ 也是 $\theta$ 的函数。当 $u\neq 0$ 的时候有类似的论述。
  但是 \eqref{eq:tmp2 in ch metric} 表明 $a'$ 依赖于 $t$,所以 $a'$ 是常值函数,
  所以 $b'=\pm\sqrt{1-(a')^2}$ 也是常值函数,这表明 $a,b$ 都是 $t$ 的仿射函数。
  所以 $C$ 的每个点都有一个邻域被一个直线包含。因为 $C$ 是连通的,所以 $C$
  被一条直线包含。 
\end{proof}

\begin{corollary}
  $\mathbb{S}^2$ 上的圆度量不是平坦的。
\end{corollary}


\subsection{法丛}

设 $(M,g)$ 是 $n$ 维黎曼流形,$S\subseteq M$ 是 $k$ 维黎曼子流形。
对于任意 $p\in S$,如果 $v\in T_pM$ 在内积 $\langle\cdot,\cdot\rangle_g$
的意义下与 $T_pS$ 中的每个向量都正交,那么我们说 $v$ \emph{正交于 $S$}。
定义 \emph{$S$ 在 $p$ 处的法空间}为子空间 $N_pS\subseteq T_pM$,
由所有 $p$ 处的正交于 $S$ 的向量构成。定义 \emph{$S$ 的法丛}
是子集 $NS\subseteq TM$,由 $S$ 在每个点处的法空间的无交并构成。

\section{黎曼距离函数}

黎曼度量为我们提供的最重要的工具之一是定义曲线长度的能力。
假设 $(M,g)$ 是黎曼流形。$\gamma:[a,b]\to M$ 是分段光滑曲线段,
那么定义 \emph{$\gamma$ 的长度} 为
\[
  \LL_g(\gamma)=\int_a^b|\gamma'(t)|_g \d t.  
\]
因为 $|\gamma'(t)|_g$ 在除开有限多个 $t$ 处都是连续的,并且在这些点处
的左右极限都存在,所以这个积分是良定义的。

一个非常重要的事实是长度与参数化的方式是无关的。在 \ref{sec:line integral}
中我们定义了一个分段光滑曲线段 $\gamma:[a,b]\to M$ 的重参数化为形如
$\tilde\gamma=\gamma\circ\varphi$ 的曲线段,其中 $\varphi:[a,b]\to [c,d]$
是微分同胚。

\begin{proposition}[长度的参数化无关性]
  令 $(M,g)$ 是一个带边或者无边的黎曼流形,$\gamma:[a,b]\to M$ 是分段光滑曲线段。
  如果 $\tilde\gamma$ 是 $\gamma$ 的重参数化,那么 $\LL_g(\tilde \gamma)=\LL_g(\gamma)$。
\end{proposition}
\begin{proof}
  首先假设 $\gamma$ 是光滑的,$\varphi:[c,d]\to [a,b]$ 是微分同胚使得
  $\tilde{\gamma}=\gamma\circ\varphi$。$\varphi$ 是微分同胚表明 $\varphi'>0$
  或者 $\varphi'<0$。假设 $\varphi'>0$,那么
  \begin{align*}
    \LL_g(\tilde\gamma)&=\int_c^d \bigl|\tilde\gamma'(t)\bigr|_g\d t=
    \int_c^d \bigl|\gamma'(\varphi(t))\varphi'(t)\bigr|_g\d t\\
    &=\int_c^d \bigl|\gamma'(\varphi(t))\bigr|_g\varphi'(t)\d t=
    \int_c^d \bigl|\gamma'(\varphi(t))\bigr|_g\d\varphi(t)\\
    &=\int_a^b|\gamma'(s)|_g\d s=\LL_g(\gamma).
  \end{align*}
  若 $\gamma'<0$,那么 $|\varphi'(t)|=-\varphi'(t)$,同时换元积分时
  也会出现一个负号,所以最后的结果也是不变的。若 $\gamma$ 是分段光滑的,
  只需要把每个子区间的结果加起来。
\end{proof}


现在我们可以定义黎曼流形中两点之间的距离。设 $(M,g)$ 是连通的黎曼流形
(当 $\partial M=\emptyset$ 的时候这个理论更加的直接,所以本节我们都做这个假设),
对于任意 $p,q\in M$,定义 \emph{$p$ 到 $q$ 的距离} 为所有从 $p$
到 $q$ 的分段光滑曲线段 $\gamma$ 的长度 $\LL_g(\gamma)$ 的下确界,
记为 $d_g(p,q)$。因为连通光滑流形中任意两点都可以用分段光滑曲线段连接,
所以这个定义是有意义的。

\begin{example}
  在 $(\mathbb{R}^n,\bar g)$ 中,直线段是在其两个端点之间的最短的分段光滑曲线段。
  因此,距离函数 $d_{\bar g}$ 等于通常的 Euclid 距离:
  \[
    d_{\bar g}(x,y)=|x-y|.
  \]
\end{example}

\begin{exercise}
  假设 $(M,g)$ 和 $\bigl(\wtilde M,\tilde g\bigr)$ 是连通的黎曼流形并且
  $F:M\to\wtilde M$ 是黎曼等距,证明:对于任意 $p,q\in M$,有
  $d_{\wtilde g}(F(p),F(q))=d_g(p,q)$。
\end{exercise}
\begin{proof}
  设 $\gamma:[a,b]\to M$ 是分段光滑曲线段,对于任意 $t\in [a,b]$,有
  \begin{align*}
    g_{\gamma(t)}(\gamma'(t),\gamma'(t))&=(F^*\tilde g)_{\gamma(t)}(\gamma'(t),\gamma'(t))\\
    &=\tilde g_{F(\gamma(t))}\bigl((F\circ\gamma)'(t),(F\circ\gamma)'(t)\bigr),
  \end{align*}
  所以
  \[
    \LL_g(\gamma)=\int_a^b |\gamma'(t)|_g \d t
    =\int_a^b \bigl|(F\circ\gamma)'(t)\bigr|_{\tilde g} \d t=
    \LL_{\tilde g}\bigl(F\circ\gamma\bigr).
  \]
  $F$ 是微分同胚表明 $p,q$ 之间的分段光滑曲线段和 $F(p),F(q)$ 之间的分段光滑曲线段
  是一一对应的,所以 $d_{\wtilde g}(F(p),F(q))=d_g(p,q)$。
\end{proof}

我们将看到黎曼距离函数将 $M$ 变成为一个度量空间,其度量诱导的拓扑与给定的流形拓扑相同。
关键在于下面的技术性引理,它表明每个黎曼度量在坐标上都可以局部地与 Euclid 度量相比较。

\begin{lemma}\label{lemma:compare with euclid metric}
  令 $g$ 是开子集 $U\subseteq \mathbb{R}^n$ 上的一个黎曼度量。给定紧子集 
  $K\subseteq U$,存在正常数 $c,C$ 使得对于任意 $x\in K$ 和 $v\in T_x \mathbb{R}^n$,有
  \[
    c|v|_{\bar g}\leq |v|_g\leq C|v|_{\bar g}.  
  \]
\end{lemma}
\begin{proof}
  
\end{proof}

\begin{theorem}[黎曼流形作为度量空间]\label{thm:metric space of riemann manifolds}
  设 $(M,g)$ 是连通的黎曼流形,在黎曼距离函数下,$M$ 成为一个度量空间,
  并且这个度量诱导的拓扑和流形自带的拓扑相同。
\end{theorem}

这个定理表明度量空间上的所有技术都可以搬运到连通的黎曼流形上。因此,
一个连通的黎曼流形 $(M,g)$ 被称为是\emph{完备的},$g$ 被称为是\emph{完备的黎曼度量},
如果 $(M,d_g)$ 是完备的度量空间。子集 $B\subseteq M$ 被称为是\emph{有界的},
如果存在常数 $K$ 使得对于所有 $x,y\in B$ 都有 $d_g(x,y)\leq K$。
实际上每个连通的黎曼流形都有一个完备的黎曼度量。

回顾一个拓扑空间是\emph{可度量化的},如果其存在一个度量使得度量拓扑和原本的拓扑
相同。

\begin{corollary}
  任意带边或者无边光滑流形都是可度量化的。
\end{corollary}
\begin{proof}
  首先假设 $M$ 是无边的。选取 $M$ 上的一个黎曼度量 $g$。如果 $M$
  是连通的,那么 \autoref{thm:metric space of riemann manifolds} 就表明 $M$
  是可度量化的。更一般的,设 $\{M_i\}$ 是 $M$ 的连通分支,在每个 $M_i$ 中选取
  一个点 $p_i$。对于 $x\in M_i,y\in M_j$,若 $i=j$,则定义 $d_g(x,y)$ 为这个连通分支
  上的度量,否则则定义为
  \[
    d_g(x,y)=d_g(x,p_i)+1+d_g(p_j,y).
  \]
  可以直接验证这是一个度量并且诱导了和 $M$ 相同的拓扑。最后,若 $M$
  有非空的边界,只需要将 $M$ 嵌入到它的加倍中,然后注意到度量空间的子空间
  仍然是度量空间即可。
\end{proof}



\section{切丛-余切丛同构}

黎曼度量的另一个作用是它提供了切丛和余切丛之间的自然的同构。给定光滑流形 $M$
上的一个黎曼度量 $g$,我们可以定义一个丛同态 $\hat g:TM\to T^*M$。
对于 $p\in M$,$v\in T_pM$,定义 $\hat g(v)\in T_p^*M$ 满足
\[
  \hat g(v)(w)=g_p(v,w)\quad w\in T_pM.  
\]
$\hat g$ 诱导了 $\mathfrak{X}(M)\to \mathfrak{X}^*(M)$ 的映射:
\[
  \hat g(X)(Y)=g(X,Y),\quad X,Y\in \mathfrak{X}(M).
\]
我们使用同一个记号来表示这个映射。由于 $\hat g(X):\mathfrak{X}(M)\to C^\infty(M)$
是 $C^\infty(M)$-线性的,根据张量表征引理 \ref{lemma:tensor characterization 1},$\hat g(X)$ 是一个光滑余向量场。
又因为 $\hat g$ 是 $C^\infty(M)$-线性的,根据丛同态表征引理 \ref{lemma:bundle homomorphism characterization},
$\hat g$ 是一个光滑丛同态。

设 $v\in T_pM$ 使得 $\hat g(v)=0$,那么
\[
  0=\hat g(v)(v)=g_p(v,v),
\]
所以 $v=0$,所以 $\hat g$ 在每个点处是单射,再根据维数相同,所以 $\hat g$ 是双射,
从而是丛同构。

在任意光滑坐标 $(x^i)$ 中,我们可以写为 $g=g_{ij}\d x^i\d x^j$,
如果 $X,Y$ 是光滑向量场,那么
\[
  \hat g(X)(Y)=g(X,Y)=g_{ij}\d x^i(X)\d x^j(Y)=
  g_{ij}X^iY^j,  
\]
这表明余向量场 $\hat g(X)$ 有坐标表示
\[
  \hat g(X)=g_{ij}X^i\d x^j.  
\]
故
\[
  \hat g\left(\frac{\partial}{\partial x^i}\right)=
  g_{ij} \d x^j=g_{ji}\d x^j,  
\]
换句话说,$\hat g:TM\to T^*M$ 在 $TM$ 的坐标标架和 $T^*M$ 的坐标余标架
下的表示矩阵与 $g$ 的表示矩阵相同。

通常将余向量场 $\hat g(X)$ 的分量记为
\[
  \hat g(X)=X_j\d x^j,\quad \text{where $X_j=g_{ij}X^i$}.  
\]
可以看到,采用这种记号后,通过黎曼度量 $g$,我们将向量场 $X$ 变成了一个余向量场,
且分量 $X^j$ 变成了 $X_j=g_{ij}X^i$,出于这一点,我们说
$\hat g(X)$ 是 $X$ 通过\emph{指标下降}得到的。通常我们使用 
$X^\flat$ 来表示 $\hat g(X)$,因为符号 $\flat$ (``flat") 在乐谱中用来
表示音调的降低。

逆映射 $\hat g^{-1}:T^*M\to TM$ 的表示矩阵就是 $(g_{ij})$ 的逆矩阵。
我们记 $\bigl(g^{ij}\bigr)$ 为实值函数矩阵,其在每个点 $p\in M$
处的取值为 $\bigl(g_{ij}(p)\bigr)$ 的逆矩阵,所以
\[
  g^{ij}g_{jk}=g_{kj}g^{ji}=\delta_k^i.  
\]
因为 $g_{ij}$ 是对称矩阵,所以 $g^{ij}$ 也是对称矩阵。所以对于
余向量场 $\omega\in \mathfrak{X}^*(M)$,向量场 $\hat g^{-1}(\omega)$
有坐标表示
\[
  \hat g^{-1}(\omega)=\omega^i\frac{\partial}{\partial x^i},
  \quad \text{where $\omega^i=g^{ij}\omega_j$}  .
\]
与上面一样,我们使用 $\omega^{\sharp}$ (``sharp'') 来表示
$\hat g^{-1}(\omega)$,我们说 $\omega^\sharp$ 是 $\omega$
通过\emph{指标上升}得到的。由于 $\flat$ 和 $\sharp$ 是从
乐谱中借用的,所以这两个互逆的映射通常被称为\emph{音乐同构}。

上升算符最重要的用途是将梯度恢复为黎曼流形上的向量场。
对于黎曼流形 $(M,g)$ 上的任意光滑实值函数 $f$,我们定义向量场
\emph{$f$ 的梯度}为
\[
  \grad f=(\d f)^\sharp.
\]
解开这个定义,我们看到对于任意 $X\in \mathfrak{X}(M)$,梯度满足
\[
  \inn{\grad f,X}_g=\hat g(\grad f)(X)=
  \d f(X)=Xf.
\]
因此梯度 $\grad f$ 是唯一满足
\[
  \inn{\grad f,X}_g=Xf\quad \forall X\in \mathfrak{X}(M)  
\]
的向量场。或者等价地说,
\[
  \inn{\grad f,\cdot}_g=\d f.  
\]

在光滑坐标中,$\grad f$ 有表示
\[
  \grad f=(\d f)^i\frac{\partial}{\partial x^i}
  =g^{ij}(\d f)_j  \frac{\partial}{\partial x^i}
  =g^{ij}\frac{\partial f}{\partial x^j}\frac{\partial}{\partial x^i}.
\]
特别地,这表明 $\grad f$ 是光滑的。在 $\mathbb{R}^n$ 配备 Euclid 度量的时候,
这导出
\[
  \grad f=\delta^{ij}\frac{\partial f}{\partial x^j}\frac{\partial}{\partial x^i}
  =\sum_{i=1}^n\frac{\partial f}{\partial x^i}\frac{\partial}{\partial x^i}.
\]

\begin{example}
  我们计算函数 $f\in C^\infty(\mathbb{R}^2)$ 相对于 Euclid 度量在极坐标中的梯度。
  在 \autoref{exa:polar coordinate of euclid metric} 中我们知道 $\bar g$
  在极坐标中的矩阵为 $\begin{psmallmatrix}1 & 0 \\ 0 & r^2\end{psmallmatrix}$,
  所以逆矩阵为 $\begin{psmallmatrix}1 & 0 \\ 0 & 1/r^2\end{psmallmatrix}$,
  所以 
  \[
    \grad f=\frac{\partial f}{\partial r}\frac{\partial}{\partial r}+
    \frac{1}{r^2}\frac{\partial f}{\partial\theta}  \frac{\partial}{\partial \theta}.
  \]
\end{example}

黎曼流形上函数 $f$ 的梯度实际上与 Euclid 空间中的梯度有相同的几何性质:
其指向 $f$ 上升最快的方向,正交于 $f$ 的水平集,并且其长度是 $f$ 在任意方向上的
方向导数的最大值。要看出这一点,任取 $p\in M$ 和 $v\in T_pM$,有
\[
  vf=\d f_p(v)=  \inn{\grad f|_p,v}_g,
\]
所以 $v=\grad f|_p$ 的时候 $vf$ 取得最大值,此时最大值为
\[
  vf=\bigl|\grad f|_p\bigr|_g^2.
\]
记水平集 $S=f^{-1}(f(p))$,那么对于任意 $w\in T_pS=\ker \d f_p$,有 $wf=0$,所以
\[
  0=wf=\d f_p(w)=\inn{\grad f|_p,w}_g,
\]
所以 $\grad f|_p$ 正交于 $T_pS$。

指标上升和指标下降算符可以扩展到混合张量场上。如果 $F$ 是 $(k,l)$-张量场,
$i\in\{1,\dots,k+l\}$ 是 $F$ 的某个协变指标,我们可以构造一个新的 $(k+1,l-1)$-张量场
$F^\sharp$ 为
\[
  F^\sharp\bigl(\alpha_1,\dots,\alpha_{k+l}\bigr)=
  F\bigl(\alpha_1,\dots,\alpha_{i-1},\alpha_i^\sharp,\alpha_{i+1},\dots,\alpha_{k+l}\bigr),
\]
其中 $\alpha_1,\dots,\alpha_{k+l}$ 是合适的向量场或者余向量场。
类似地,如果 $i$ 是逆变指标,可以定义 $(k-1,l+1)$-张量场 $F^\flat$
为
\[
  F^\flat\bigl(\alpha_1,\dots,\alpha_{k+l}\bigr)=
  F\bigl(\alpha_1,\dots,\alpha_{i-1},\alpha_i^\flat,\alpha_{i+1},\dots,\alpha_{k+l}\bigr).
\]
例如,如果 $A$ 在某个局部坐标中表示为 $(1,2)$-张量场
\[
  A={{A_{i}}^j}_k\d x^i\otimes \frac{\partial}{\partial x^j}\otimes \d x^k,
\]
那么 $A^\flat$ 是一个 $(0,3)$-张量场
\[
  A^\flat =A_{ijk}\d x^i\otimes \d x^j\otimes \d x^k,  
\]
其中
\[
  A_{ijk}= g_{jl}  {{A_{i}}^l}_k.
\]

