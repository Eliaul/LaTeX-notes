\chapter{微分形式}

\section{交错张量代数}

我们定义投影 $\Alt:T^k(V^*)\to \Lambda^k(V^*)$,称为\emph{交错化}:
\[
  \Alt\alpha=\frac{1}{k!}\sum_{\sigma\in S_k}(\sgn\sigma)  (\sigma\alpha).
\]

\begin{example}
  如果 $\alpha$ 是 $1$-张量,那么 $\Alt\alpha=\alpha$。如果 $\beta$
  是 $2$-张量,那么
  \[
    (\Alt\beta) (v,w)=\frac{1}{2}\bigl(\beta(v,w)-\beta(w,v)\bigr).
  \]
\end{example}

\begin{proposition}[交错化的性质]
  令 $\alpha$ 是有限维向量空间上的协变张量。
  \begin{enumerate}
    \item $\Alt\alpha$ 是交错的。
    \item $\Alt\alpha=\alpha$ 当且仅当 $\alpha$ 是交错的。
  \end{enumerate}
\end{proposition}
\begin{proof}
  (1) 任取 $\tau\in S_k$,有
  \begin{align*}
    \tau(\Alt\alpha)&=\frac{1}{k!}\sum_{\sigma\in S_k}(\sgn\sigma)(\tau\sigma\alpha)\\
    &=\frac{1}{k!}\sum_{\sigma\in S_k}(\sgn\tau)(\sgn\tau\sigma)(\tau\sigma\alpha)\\
    &=(\sgn\tau)\frac{1}{k!}\sum_{\eta\in S_k}(\sgn\eta)\eta\alpha\\
    &=(\sgn\tau)(\Alt\alpha).
  \end{align*}
  这就表明 $\Alt \alpha$ 是交错张量。
  
  (2) 若 $\alpha=\Alt\alpha$,则由 (1),$\alpha$ 是交错的。反之,若
  $\alpha$ 是交错的,那么 
  \[
    \Alt\alpha=  \frac{1}{k!}\sum_{\sigma\in S_k}(\sgn\sigma)  (\sigma\alpha)
    =\frac{1}{k!}\sum_{\sigma\in S_k}(\sgn\sigma)^2\alpha=
    \alpha.\qedhere
  \]
\end{proof}

\subsection{基本交错张量}

给定正整数 $k$,一个有序的正整数的 $k$-元组 $I=(i_1,\dots,i_k)$
被称为\emph{长度 $k$ 的多重指标}。如果 $I$ 是这样一个多重指标,$\sigma\in S_k$,
我们记
\[
  I_\sigma=\bigl(i_{\sigma(1)},\dots,i_{\sigma(k)}\bigr).  
\]
注意到我们有 $I_{\sigma\tau}=(I_\sigma)_\tau$。

令 $V$ 是 $n$-维向量空间,$\bigl(\varepsilon^1,\dots,\varepsilon^n \bigr)$
是 $V^*$ 的一组基。我们定义 $V$ 上的一族 $k$-余向量来推广行列式函数的概念。
对于每个多重指标 $I=(i_1,\dots,i_k)$ 且 $1\leq i_1,\dots,i_k\leq n$,定义
协变 $k$-张量 $\varepsilon^I=\varepsilon^{i_1\dots i_k}$ 为
\[
  \varepsilon^I(v_1,\dots,v_k)=\det \begin{pmatrix}
    \varepsilon^{i_1}(v_1) & \cdots & \varepsilon^{i_1}(v_k)\\
    \vdots & \ddots & \vdots \\
    \varepsilon^{i_k}(v_1) & \cdots & \varepsilon^{i_k}(v_k)\\
  \end{pmatrix}  =\det\begin{pmatrix}
    v_1^{i_1} & \cdots & v_k^{i_1} \\
    \vdots & \ddots & \vdots \\
    v_1^{i_k} & \cdots & v_k^{i_k} \\
  \end{pmatrix}.
\]
换句话说,如果 $\mathbb{v}$ 是由 $v_1,\dots,v_k$ 作为列向量构成的 $n\times k$ 矩阵,
每一列是 $v_i$ 在对偶于 $\bigl(\varepsilon^i\bigr)$ 的基 $(E_i)$ 下的分量,
那么 $\varepsilon^I(v_1,\dots,v_k)$ 相当于 $\mathbb{v}$ 的由 $i_1,\dots,i_k$ 行组成的
$k\times k$ 子矩阵的行列式。因为行列式交换两列符号相反,所以 $\varepsilon^I$
是一个交错 $k$-张量。我们说 $\varepsilon^I$ 是\emph{基本交错张量}
或者\emph{基本 $k$-余向量}。

\begin{example}
  设 $\bigl(\mathbb{R}^3\bigr)^*$ 的标准对偶基为 $\bigl(e^1,e^2,e^3\bigr)$,那么
  \begin{gather*}
    e^{13}(v,w)=\det
    \begin{pmatrix}
      v^1 & w^1 \\
      v^3 & w^3 
    \end{pmatrix}=v^1w^3-w^1v^3  ,\\
    e^{123}(v,w,x)=\det(v,w,x).
  \end{gather*}
\end{example}

为了简化计算,我们扩展 Kronecker 符号的定义。如果 $I,J$ 都是长度为 $k$
的多重指标,我们定义
\[
  \delta_J^I=\det\begin{pmatrix}
    \delta_{j_1}^{i_1} & \cdots & \delta_{j_k}^{i_1} \\
    \vdots & \ddots & \vdots \\
    \delta_{j_1}^{i_k} & \cdots & \delta_{j_k}^{i_k} \\
  \end{pmatrix}  .
\]
可以证明,在 $I$ 或者 $J$ 有重复的指标或者 $J$ 不是 $I$ 的一个置换的时候,有
$\delta_J^I=0$。在 $I,J$ 都没有重复指标且 $J=I_\sigma$ 的时候,有 $\delta_J^I=\sgn\sigma$。

\begin{proposition}[基本 $k$-余向量的性质]
  令 $(E_i)$ 是 $V$ 的一组基,$\bigl(\varepsilon^i\bigr)$ 是 $V^*$ 的对偶基。
  \begin{enumerate}
    \item 如果 $I$ 有重复指标,那么 $\varepsilon^I=0$。
    \item 如果 $J=I_\sigma$,那么 $\varepsilon^I=(\sgn\sigma)\varepsilon^J$。
    \item $\varepsilon^I$ 在基向量上的值为
    \[
      \varepsilon^I(E_{j_1},\dots,E_{j_k})=\delta_J^I.  
    \]
  \end{enumerate}
\end{proposition}
\begin{proof}
  (1) $I$ 有重复指标表明 $\varepsilon^I$ 有两行相同,所以为零。
  (2) 考虑 $\sigma$ 是对换的情况即可,此时 $\varepsilon^J$ 是 $\varepsilon^I$
  交换两行得到的,所以符号相反。(3) 按照定义即得。
\end{proof}

基本 $k$-余向量的重要性在于其提供了 $\Lambda^k(V^*)$ 的一组方便的基。
当然,所有的 $\varepsilon^I$ 并不是线性无关的,因为其中一些是零,一些是相差
一个符号。但是,下面的命题表明,如果我们限制多重指标 $I=(i_1,\dots,i_k)$
是\emph{递增的},即 $i_1<\cdots<i_k$,那么这些多重指标对应的基本 $k$-余向量
构成基。我们定义一个加撇的求和记号用于表示仅对递增的多重指标求和,例如
\[
  \sideset{}{'}\sum_{I}\alpha_I\varepsilon^I
  =\sum_{\{I\,|\, i_1<\cdots<i_k\}}  \alpha_I\varepsilon^I.
\]

\begin{proposition}[$\Lambda^k(V^*)$ 的一组基]
  令 $V$ 是有限维向量空间,$\bigl(\varepsilon^i\bigr)$ 是 $V^*$
  的任意一组基,那么对于每个正整数 $k\leq n$,$k$-余向量的集合
  \[
    \mathcal{E}=\bigl\{\varepsilon^I\,|\, \text{$I$ 是长度为 $k$ 的多重递增指标}\bigr\}  
  \]
  构成 $\Lambda^k(V^*)$ 的一组基,因此
  \[
    \dim\Lambda^k(V^*)=\binom{n}{k}=\frac{n!}{k!(n-k)!}.  
  \]
  如果 $k> n$,那么 $\dim\Lambda^k(V^*)=0$。
\end{proposition}
\begin{proof}
  当 $k>n$ 的时候,此时任意的 $\alpha\in \Lambda^k(V^*)$ 的参数一定线性相关,
  所以 $\alpha=0$,故 $\Lambda^k(V^*)$ 为平凡空间。对于 $k\le n$ 的时候,
  我们证明 $\mathcal{E}$ 线性无关且张成 $\Lambda^k(V^*)$。令 $(E_i)$
  是对偶于 $\bigl(\varepsilon^i\bigr)$ 的 $V$ 的基。

  首先说明 $\mathcal{E}$ 张成 $\Lambda^k(V^*)$。任取 $\alpha\in \Lambda^k(V^*)$。
  对于任意多重指标 $I=(i_1,\dots,i_k)$,定义
  \[
    \alpha_I=\alpha(E_{i_1},\dots,E_{i_k}).  
  \]
  那么任取多重指标 $J$,如果 $J=I_\sigma$,那么 $\alpha_J=(\sgn\sigma)\alpha_I$,所以有
  \[
    \alpha(E_{j_1},\dots,E_{j_k})=\alpha_J=\sideset{}{'}\sum_{I}\alpha_I\delta_J^I
    =  \sideset{}{'}\sum_I\alpha_I\varepsilon^I(E_{j_1},\dots,E_{j_k}),
  \]
  故 $\alpha=\sideset{}{'}{\textstyle\sum_I}\alpha_I\varepsilon^I$,这就说明  $\mathcal{E}$ 张成 $\Lambda^k(V^*)$。

  然后说明 $\mathcal{E}$ 线性无关。设 $\sideset{}{'}{\textstyle\sum_I}\alpha_I\varepsilon^I=0$。
  令 $J$ 是任意递增的多重指标,将两边作用在 $(E_{j_1},\dots,E_{j_k})$ 上,有
  \[
    \alpha_J=\sideset{}{'}\sum_I\alpha_I\delta_J^I=0  
  \]
  所以系数 $\alpha_I=0$。
\end{proof}

特别地,对于 $n$ 维向量空间 $V$,$\Lambda^n(V^*)$ 是 $1$-维的并且由
$\varepsilon^{1\dots n}$ 张成。根据定义,$\varepsilon^{1\dots n}$
就是行列式函数。

\begin{proposition}
  设 $V$ 是 $n$ 维向量空间,$\omega\in \Lambda^n(V^*)$。如果 $T:V\to V$
  是线性映射,$v_1,\dots,v_n\in V$,那么
  \begin{equation}
    \omega(Tv_1,\dots,Tv_n)=(\det T)\omega(v_1,\dots,v_n).
  \end{equation}
\end{proposition}
\begin{proof}
  任取 $V$ 的一组基 $(E_i)$,设 $\bigl(\varepsilon^i\bigr)$ 是对偶基。
  此时 $\omega=c\varepsilon^{1\dots n}$。记 $T$ 在 $(E_i)$ 下的表示矩阵为
  $(T_i^j)$,即 $T_i=TE_i=T_i^jE_j$。我们只需要证明结论对基向量成立,此时右边为
  \[
    (\det T)c\varepsilon^{1\dots n}(E_1,\dots,E_n)=c(\det T).
  \]
  左边为
  \[
    \omega(TE_1,\dots,E_n)=c\varepsilon^{1\dots n}
    (T_1,\dots,T_n)=c\det (T_i^j).  
  \]
  所以二者相等。
\end{proof}

\subsection{楔积}

给定 $\omega\in \Lambda^k(V^*)$ 和 $\eta\in \Lambda^l(V^*)$,我们定义\emph{楔积}
或者 \emph{外积} 为一个 $(k+l)$-余向量:
\begin{equation}
  \omega\wedge\eta=\frac{(k+l)!}{k!l!}\Alt(\omega\otimes\eta).
\end{equation}
交错化前面的系数是为了下面的引理。

\begin{lemma}
  $V$ 是有限维向量空间,$\bigl(\varepsilon^1,\dots,\varepsilon^n\bigr)$
  是 $V^*$ 的一组基。对于任意多重指标 $I=(i_1,\dots,i_k)$ 和
  $J=(j_1,\dots,j_l)$,有
  \[
    \varepsilon^I\wedge\varepsilon^J=\varepsilon^{IJ},  
  \]
  其中 $IJ=(i_1,\dots,i_k,j_1,\dots,j_l)$ 由 $I$ 和 $J$ 拼接得到。
\end{lemma} 
\begin{proof}
  根据多重线性性,只需要证明对于任意基向量 $(E_{p_1},\dots,E_{p_{k+l}})$,
  有 
  \begin{equation}
    \varepsilon^I\wedge\varepsilon^J(E_{p_1},\dots,E_{p_{k+l}})
    =\varepsilon^{IJ}(E_{p_1},\dots,E_{p_{k+l}}).
  \end{equation}

  \textsc{Case 1:} 如果 $P=(p_1,\dots,p_{k+l})$ 有重复指标,那么
  两边均为零,所以相等。

  \textsc{Case 2:} $P$ 包含一个指标且这个指标不在 $I$ 或者 $J$ 中。
  此时等式右端为 $\delta_P^{IJ}=0$。类似的,左端的每一项都是 $\varepsilon^I$
  和 $\varepsilon^J$ 的某个置换之积,但是参数有一个指标不在 $I$ 或者 $J$
  中,所以每一项都为零,所以此时两端都为零。

  \textsc{Case 3:} $P=IJ$ 且 $P$ 没有重复指标。此时右端为
  $\delta_P^{IJ}=1$。左端根据定义,有
  \begin{align*}
    &\hphantom{{}={}}\varepsilon^I\wedge \varepsilon^J(E_{p_1},\dots,E_{p_{k+l}})\\
    &=\frac{(k+l)!}{k!l!}\Alt\bigl(\varepsilon^I\otimes\varepsilon^J\bigr)
    (E_{p_1},\dots,E_{p_{k+l}})\\
    &=\frac{1}{k!l!}\sum_{\sigma\in S_{k+l}}(\sgn\sigma)
    \varepsilon^I\bigl(E_{p_{\sigma(1)}},\dots,E_{p_{\sigma(k)}}\bigr)
    \varepsilon^J\bigl(E_{p_{\sigma(k+1)}},\dots,E_{p_{\sigma(k+l)}}\bigr),
  \end{align*}
  由于 $P=IJ$,所以只有 $\sigma=\tau\eta$ 的时候上述求和非零,其中
  $\tau\in S_k$ 置换 $\{1,\dots,k\}$,$\eta\in S_l$ 置换 $\{k+1,\dots,k+l\}$。
  所以
  \begin{align*}
    &\hphantom{{}={}}\varepsilon^I\wedge \varepsilon^J(E_{p_1},\dots,E_{p_{k+l}})\\
    &=\frac{1}{k!l!}\sum_{\substack{\tau\in S_k\\\eta\in S_l}}
    (\sgn\tau)(\sgn\eta) \varepsilon^I\bigl(E_{p_{\tau(1)}},\dots,E_{p_{\tau(k)}}\bigr)
    \varepsilon^J\bigl(E_{p_{k+\eta(1)}},\dots,E_{p_{k+\eta(l)}}\bigr)\\
    &=\left(
      \frac{1}{k!}\sum_{\tau\in S_k}(\sgn\tau)
      \varepsilon^I\bigl(E_{p_{\tau(1)}},\dots,E_{p_{\tau(k)}}\bigr)
    \right)\left(
      \frac{1}{l!}\sum_{\eta\in S_l}\varepsilon^J\bigl(E_{p_{k+\eta(1)}},\dots,E_{p_{k+\eta(l)}}\bigr)
    \right)\\
    &=\bigl(\Alt \varepsilon^I\bigr)\bigl(E_{p_{1}},\dots,E_{p_{k}}\bigr)
    \bigl(\Alt \varepsilon^J\bigr)\bigl(E_{p_{k+1}},\dots,E_{p_{k+l}}\bigr)\\
    &=\varepsilon^I\bigl(E_{p_{1}},\dots,E_{p_{k}}\bigr)
    \varepsilon^J\bigl(E_{p_{k+1}},\dots,E_{p_{k+l}}\bigr)=1.
  \end{align*}

  \textsc{Case 4:} $P$ 是 $IJ$ 的一个置换且没有重复指标。此时等式两端同时用
  置换作用使得参数变为与 $IJ$ 的指标一致即可。
\end{proof} 

\begin{proposition}[楔积的性质]
  令 $\omega,\omega',\eta,\eta',\xi$ 是有限维向量空间 $V$ 上的多重余向量。
  \begin{enumerate}
    \item 对于 $a,a'\in \mathbb{R}$,有
    \begin{align*}
      (a\omega+a'\omega')\wedge\eta&=a(\omega\wedge\eta)+a'(\omega'\wedge\eta),\\
      \eta\wedge(a\omega+a'\omega')&=a(\eta\wedge\omega)+a'(\eta\wedge\omega').
    \end{align*}
  \end{enumerate}
\end{proposition}

