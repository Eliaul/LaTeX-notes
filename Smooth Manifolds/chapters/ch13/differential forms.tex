\chapter{微分形式}

\section{交错张量代数}

我们定义投影 $\Alt:T^k(V^*)\to \Lambda^k(V^*)$,称为\emph{交错化}:
\[
  \Alt\alpha=\frac{1}{k!}\sum_{\sigma\in S_k}(\sgn\sigma)  (\sigma\alpha).
\]

\begin{example}
  如果 $\alpha$ 是 $1$-张量,那么 $\Alt\alpha=\alpha$。如果 $\beta$
  是 $2$-张量,那么
  \[
    (\Alt\beta) (v,w)=\frac{1}{2}\bigl(\beta(v,w)-\beta(w,v)\bigr).
  \]
\end{example}

\begin{proposition}[交错化的性质]
  令 $\alpha$ 是有限维向量空间上的协变张量。
  \begin{enumerate}
    \item $\Alt\alpha$ 是交错的。
    \item $\Alt\alpha=\alpha$ 当且仅当 $\alpha$ 是交错的。
  \end{enumerate}
\end{proposition}
\begin{proof}
  (1) 任取 $\tau\in S_k$,有
  \begin{align*}
    \tau(\Alt\alpha)&=\frac{1}{k!}\sum_{\sigma\in S_k}(\sgn\sigma)(\tau\sigma\alpha)\\
    &=\frac{1}{k!}\sum_{\sigma\in S_k}(\sgn\tau)(\sgn\tau\sigma)(\tau\sigma\alpha)\\
    &=(\sgn\tau)\frac{1}{k!}\sum_{\eta\in S_k}(\sgn\eta)\eta\alpha\\
    &=(\sgn\tau)(\Alt\alpha).
  \end{align*}
  这就表明 $\Alt \alpha$ 是交错张量。
  
  (2) 若 $\alpha=\Alt\alpha$,则由 (1),$\alpha$ 是交错的。反之,若
  $\alpha$ 是交错的,那么 
  \[
    \Alt\alpha=  \frac{1}{k!}\sum_{\sigma\in S_k}(\sgn\sigma)  (\sigma\alpha)
    =\frac{1}{k!}\sum_{\sigma\in S_k}(\sgn\sigma)^2\alpha=
    \alpha.\qedhere
  \]
\end{proof}

\subsection{基本交错张量}

给定正整数 $k$,一个有序的正整数的 $k$-元组 $I=(i_1,\dots,i_k)$
被称为\emph{长度 $k$ 的多重指标}。如果 $I$ 是这样一个多重指标,$\sigma\in S_k$,
我们记
\[
  I_\sigma=\bigl(i_{\sigma(1)},\dots,i_{\sigma(k)}\bigr).  
\]
注意到我们有 $I_{\sigma\tau}=(I_\sigma)_\tau$。

令 $V$ 是 $n$-维向量空间,$\bigl(\varepsilon^1,\dots,\varepsilon^n \bigr)$
是 $V^*$ 的一组基。我们定义 $V$ 上的一族 $k$-余向量来推广行列式函数的概念。
对于每个多重指标 $I=(i_1,\dots,i_k)$ 且 $1\leq i_1,\dots,i_k\leq n$,定义
协变 $k$-张量 $\varepsilon^I=\varepsilon^{i_1\dots i_k}$ 为
\[
  \varepsilon^I(v_1,\dots,v_k)=\det \begin{pmatrix}
    \varepsilon^{i_1}(v_1) & \cdots & \varepsilon^{i_1}(v_k)\\
    \vdots & \ddots & \vdots \\
    \varepsilon^{i_k}(v_1) & \cdots & \varepsilon^{i_k}(v_k)\\
  \end{pmatrix}  =\det\begin{pmatrix}
    v_1^{i_1} & \cdots & v_k^{i_1} \\
    \vdots & \ddots & \vdots \\
    v_1^{i_k} & \cdots & v_k^{i_k} \\
  \end{pmatrix}.
\]
换句话说,如果 $\mathbb{v}$ 是由 $v_1,\dots,v_k$ 作为列向量构成的 $n\times k$ 矩阵,
每一列是 $v_i$ 在对偶于 $\bigl(\varepsilon^i\bigr)$ 的基 $(E_i)$ 下的分量,
那么 $\varepsilon^I(v_1,\dots,v_k)$ 相当于 $\mathbb{v}$ 的由 $i_1,\dots,i_k$ 行组成的
$k\times k$ 子矩阵的行列式。因为行列式交换两列符号相反,所以 $\varepsilon^I$
是一个交错 $k$-张量。我们说 $\varepsilon^I$ 是\emph{基本交错张量}
或者\emph{基本 $k$-余向量}。

\begin{example}
  设 $\bigl(\mathbb{R}^3\bigr)^*$ 的标准对偶基为 $\bigl(e^1,e^2,e^3\bigr)$,那么
  \begin{gather*}
    e^{13}(v,w)=\det
    \begin{pmatrix}
      v^1 & w^1 \\
      v^3 & w^3 
    \end{pmatrix}=v^1w^3-w^1v^3  ,\\
    e^{123}(v,w,x)=\det(v,w,x).
  \end{gather*}
\end{example}

为了简化计算,我们扩展 Kronecker 符号的定义。如果 $I,J$ 都是长度为 $k$
的多重指标,我们定义
\[
  \delta_J^I=\det\begin{pmatrix}
    \delta_{j_1}^{i_1} & \cdots & \delta_{j_k}^{i_1} \\
    \vdots & \ddots & \vdots \\
    \delta_{j_1}^{i_k} & \cdots & \delta_{j_k}^{i_k} \\
  \end{pmatrix}  .
\]
可以证明,在 $I$ 或者 $J$ 有重复的指标或者 $J$ 不是 $I$ 的一个置换的时候,有
$\delta_J^I=0$。在 $I,J$ 都没有重复指标且 $J=I_\sigma$ 的时候,有 $\delta_J^I=\sgn\sigma$。

\begin{proposition}[基本 $k$-余向量的性质]
  令 $(E_i)$ 是 $V$ 的一组基,$\bigl(\varepsilon^i\bigr)$ 是 $V^*$ 的对偶基。
  \begin{enumerate}
    \item 如果 $I$ 有重复指标,那么 $\varepsilon^I=0$。
    \item 如果 $J=I_\sigma$,那么 $\varepsilon^I=(\sgn\sigma)\varepsilon^J$。
    \item $\varepsilon^I$ 在基向量上的值为
    \[
      \varepsilon^I(E_{j_1},\dots,E_{j_k})=\delta_J^I.  
    \]
  \end{enumerate}
\end{proposition}
\begin{proof}
  (1) $I$ 有重复指标表明 $\varepsilon^I$ 有两行相同,所以为零。
  (2) 考虑 $\sigma$ 是对换的情况即可,此时 $\varepsilon^J$ 是 $\varepsilon^I$
  交换两行得到的,所以符号相反。(3) 按照定义即得。
\end{proof}

基本 $k$-余向量的重要性在于其提供了 $\Lambda^k(V^*)$ 的一组方便的基。
当然,所有的 $\varepsilon^I$ 并不是线性无关的,因为其中一些是零,一些是相差
一个符号。但是,下面的命题表明,如果我们限制多重指标 $I=(i_1,\dots,i_k)$
是\emph{递增的},即 $i_1<\cdots<i_k$,那么这些多重指标对应的基本 $k$-余向量
构成基。我们定义一个加撇的求和记号用于表示仅对递增的多重指标求和,例如
\[
  \sideset{}{'}\sum_{I}\alpha_I\varepsilon^I
  =\sum_{\{I\,|\, i_1<\cdots<i_k\}}  \alpha_I\varepsilon^I.
\]

\begin{proposition}[$\Lambda^k(V^*)$ 的一组基]\label{prop:basis of Lambda}
  令 $V$ 是有限维向量空间,$\bigl(\varepsilon^i\bigr)$ 是 $V^*$
  的任意一组基,那么对于每个正整数 $k\leq n$,$k$-余向量的集合
  \[
    \mathcal{E}=\bigl\{\varepsilon^I\,|\, \text{$I$ 是长度为 $k$ 的多重递增指标}\bigr\}  
  \]
  构成 $\Lambda^k(V^*)$ 的一组基,因此
  \[
    \dim\Lambda^k(V^*)=\binom{n}{k}=\frac{n!}{k!(n-k)!}.  
  \]
  如果 $k> n$,那么 $\dim\Lambda^k(V^*)=0$。
\end{proposition}
\begin{proof}
  当 $k>n$ 的时候,此时任意的 $\alpha\in \Lambda^k(V^*)$ 的参数一定线性相关,
  所以 $\alpha=0$,故 $\Lambda^k(V^*)$ 为平凡空间。对于 $k\le n$ 的时候,
  我们证明 $\mathcal{E}$ 线性无关且张成 $\Lambda^k(V^*)$。令 $(E_i)$
  是对偶于 $\bigl(\varepsilon^i\bigr)$ 的 $V$ 的基。

  首先说明 $\mathcal{E}$ 张成 $\Lambda^k(V^*)$。任取 $\alpha\in \Lambda^k(V^*)$。
  对于任意多重指标 $I=(i_1,\dots,i_k)$,定义
  \[
    \alpha_I=\alpha(E_{i_1},\dots,E_{i_k}).  
  \]
  那么任取多重指标 $J$,如果 $J=I_\sigma$,那么 $\alpha_J=(\sgn\sigma)\alpha_I$,所以有
  \[
    \alpha(E_{j_1},\dots,E_{j_k})=\alpha_J=\sideset{}{'}\sum_{I}\alpha_I\delta_J^I
    =  \sideset{}{'}\sum_I\alpha_I\varepsilon^I(E_{j_1},\dots,E_{j_k}),
  \]
  故 $\alpha=\sideset{}{'}{\textstyle\sum_I}\alpha_I\varepsilon^I$,这就说明  $\mathcal{E}$ 张成 $\Lambda^k(V^*)$。

  然后说明 $\mathcal{E}$ 线性无关。设 $\sideset{}{'}{\textstyle\sum_I}\alpha_I\varepsilon^I=0$。
  令 $J$ 是任意递增的多重指标,将两边作用在 $(E_{j_1},\dots,E_{j_k})$ 上,有
  \[
    \alpha_J=\sideset{}{'}\sum_I\alpha_I\delta_J^I=0  
  \]
  所以系数 $\alpha_I=0$。
\end{proof}

特别地,对于 $n$ 维向量空间 $V$,$\Lambda^n(V^*)$ 是 $1$-维的并且由
$\varepsilon^{1\dots n}$ 张成。根据定义,$\varepsilon^{1\dots n}$
就是行列式函数。

\begin{proposition}\label{prop:property of alt}
  设 $V$ 是 $n$ 维向量空间,$\omega\in \Lambda^n(V^*)$。如果 $T:V\to V$
  是线性映射,$v_1,\dots,v_n\in V$,那么
  \begin{equation}
    \omega(Tv_1,\dots,Tv_n)=(\det T)\omega(v_1,\dots,v_n).
  \end{equation}
\end{proposition}
\begin{proof}
  任取 $V$ 的一组基 $(E_i)$,设 $\bigl(\varepsilon^i\bigr)$ 是对偶基。
  此时 $\omega=c\varepsilon^{1\dots n}$。记 $T$ 在 $(E_i)$ 下的表示矩阵为
  $(T_j^i)$,即 $T_j=TE_j=T_j^iE_i$。我们只需要证明结论对基向量成立,此时右边为
  \[
    (\det T)c\varepsilon^{1\dots n}(E_1,\dots,E_n)=c(\det T).
  \]
  左边为
  \[
    \omega(TE_1,\dots,E_n)=c\varepsilon^{1\dots n}
    (T_1,\dots,T_n)=c\det (T_j^i).  
  \]
  所以二者相等。
\end{proof}

\subsection{楔积}

给定 $\omega\in \Lambda^k(V^*)$ 和 $\eta\in \Lambda^l(V^*)$,我们定义\emph{楔积}
或者 \emph{外积} 为一个 $(k+l)$-余向量:
\begin{equation}
  \omega\wedge\eta=\frac{(k+l)!}{k!l!}\Alt(\omega\otimes\eta).
\end{equation}
交错化前面的系数是为了下面的引理。

\begin{lemma}\label{lemma:wedge}
  $V$ 是有限维向量空间,$\bigl(\varepsilon^1,\dots,\varepsilon^n\bigr)$
  是 $V^*$ 的一组基。对于任意多重指标 $I=(i_1,\dots,i_k)$ 和
  $J=(j_1,\dots,j_l)$,有
  \[
    \varepsilon^I\wedge\varepsilon^J=\varepsilon^{IJ},  
  \]
  其中 $IJ=(i_1,\dots,i_k,j_1,\dots,j_l)$ 由 $I$ 和 $J$ 拼接得到。
\end{lemma} 
\begin{proof}
  根据多重线性性,只需要证明对于任意基向量 $(E_{p_1},\dots,E_{p_{k+l}})$,
  有 
  \begin{equation}
    \varepsilon^I\wedge\varepsilon^J(E_{p_1},\dots,E_{p_{k+l}})
    =\varepsilon^{IJ}(E_{p_1},\dots,E_{p_{k+l}}).
  \end{equation}

  \textsc{Case 1:} 如果 $P=(p_1,\dots,p_{k+l})$ 有重复指标,那么
  两边均为零,所以相等。

  \textsc{Case 2:} $P$ 包含一个指标且这个指标不在 $I$ 或者 $J$ 中。
  此时等式右端为 $\delta_P^{IJ}=0$。类似的,左端的每一项都是 $\varepsilon^I$
  和 $\varepsilon^J$ 的某个置换之积,但是参数有一个指标不在 $I$ 或者 $J$
  中,所以每一项都为零,所以此时两端都为零。

  \textsc{Case 3:} $P=IJ$ 且 $P$ 没有重复指标。此时右端为
  $\delta_P^{IJ}=1$。左端根据定义,有
  \begin{align*}
    &\hphantom{{}={}}\varepsilon^I\wedge \varepsilon^J(E_{p_1},\dots,E_{p_{k+l}})\\
    &=\frac{(k+l)!}{k!l!}\Alt\bigl(\varepsilon^I\otimes\varepsilon^J\bigr)
    (E_{p_1},\dots,E_{p_{k+l}})\\
    &=\frac{1}{k!l!}\sum_{\sigma\in S_{k+l}}(\sgn\sigma)
    \varepsilon^I\bigl(E_{p_{\sigma(1)}},\dots,E_{p_{\sigma(k)}}\bigr)
    \varepsilon^J\bigl(E_{p_{\sigma(k+1)}},\dots,E_{p_{\sigma(k+l)}}\bigr),
  \end{align*}
  由于 $P=IJ$,所以只有 $\sigma=\tau\eta$ 的时候上述求和非零,其中
  $\tau\in S_k$ 置换 $\{1,\dots,k\}$,$\eta\in S_l$ 置换 $\{k+1,\dots,k+l\}$。
  所以
  \begin{align*}
    &\hphantom{{}={}}\varepsilon^I\wedge \varepsilon^J(E_{p_1},\dots,E_{p_{k+l}})\\
    &=\frac{1}{k!l!}\sum_{\substack{\tau\in S_k\\\eta\in S_l}}
    (\sgn\tau)(\sgn\eta) \varepsilon^I\bigl(E_{p_{\tau(1)}},\dots,E_{p_{\tau(k)}}\bigr)
    \varepsilon^J\bigl(E_{p_{k+\eta(1)}},\dots,E_{p_{k+\eta(l)}}\bigr)\\
    &=\left(
      \frac{1}{k!}\sum_{\tau\in S_k}(\sgn\tau)
      \varepsilon^I\bigl(E_{p_{\tau(1)}},\dots,E_{p_{\tau(k)}}\bigr)
    \right)\left(
      \frac{1}{l!}\sum_{\eta\in S_l}\varepsilon^J\bigl(E_{p_{k+\eta(1)}},\dots,E_{p_{k+\eta(l)}}\bigr)
    \right)\\
    &=\bigl(\Alt \varepsilon^I\bigr)\bigl(E_{p_{1}},\dots,E_{p_{k}}\bigr)
    \bigl(\Alt \varepsilon^J\bigr)\bigl(E_{p_{k+1}},\dots,E_{p_{k+l}}\bigr)\\
    &=\varepsilon^I\bigl(E_{p_{1}},\dots,E_{p_{k}}\bigr)
    \varepsilon^J\bigl(E_{p_{k+1}},\dots,E_{p_{k+l}}\bigr)=1.
  \end{align*}

  \textsc{Case 4:} $P$ 是 $IJ$ 的一个置换且没有重复指标。此时等式两端同时用
  置换作用使得参数变为与 $IJ$ 的指标一致即可。
\end{proof} 

\begin{proposition}[楔积的性质]\label{prop:property of wedge}
  令 $\omega,\omega',\eta,\eta',\xi$ 是有限维向量空间 $V$ 上的多重余向量。
  \begin{enumerate}
    \item \emph{双线性性。} 对于 $a,a'\in \mathbb{R}$,有
    \begin{align*}
      (a\omega+a'\omega')\wedge\eta&=a(\omega\wedge\eta)+a'(\omega'\wedge\eta),\\
      \eta\wedge(a\omega+a'\omega')&=a(\eta\wedge\omega)+a'(\eta\wedge\omega').
    \end{align*}
    \item \emph{结合律。}
    \[
      \omega\wedge(\eta\wedge\xi)=(\omega\wedge\eta)\wedge\xi.  
    \]
    \item \emph{反交换律。} 对于 $\omega\in\Lambda^k(V^*)$ 和
    $\eta\in\Lambda^l(V^*)$,有
    \begin{equation}
      \omega\wedge\eta=(-1)^{kl}\eta\wedge\omega.
    \end{equation}
    \item 如果 $(\varepsilon^i)$ 是 $V$ 的任意一组基,$I=(i_1,\dots,i_k)$
    是多重指标,那么
    \begin{equation}
      \varepsilon^{i_1} \wedge\cdots\wedge\varepsilon^{i_k}=\varepsilon^I. 
    \end{equation}
    \item 对于任意余向量 $\omega^1,\dots,\omega^k$ 和向量 $v_1,\dots,v_k$,有
    \begin{equation}
      \omega^1\wedge\cdots\wedge\omega^k(v_1,\dots,v_k)
      =\det\bigl(\omega^i(v_j)\bigr).
    \end{equation}
  \end{enumerate} 
\end{proposition}
\begin{proof}
  双线性性根据定义是显然的。对于结合律,我们有
  \[
    \bigl(\varepsilon^I\wedge\varepsilon^J\bigr)\wedge\varepsilon^K
    =\varepsilon^{IJ}\wedge \varepsilon^K
    =\varepsilon^{IJK}=\varepsilon^I\wedge\varepsilon^{JK}
    =\varepsilon^I\wedge\bigl(\varepsilon^J\wedge\varepsilon^K\bigr),
  \]
  根据双线性性即可推得一般情况下的结合律。类似的,我们有
  \[
    \varepsilon^{I}\wedge\varepsilon^J=\varepsilon^{IJ}
    =(\sgn\tau)\varepsilon^{JI}=(\sgn\tau)\varepsilon^J\wedge\varepsilon^I,  
  \]
  其中 $\tau$ 将 $IJ$ 置换为 $JI$。可以证明 $\sgn\tau=(-1)^{kl}$,因为
  $\tau$ 可以分解为 $kl$ 个对换($I$ 的每个指标都要移动经过 $J$ 的所有指标)的乘积,
  一般的反交换律由双线性性得到。

  (4) 是 \autoref{lemma:wedge} 的直接推论。对于 (5),同样只需要证明
  $\omega^j$ 取遍基 $\varepsilon^{i_j}$ 的情况即可,此时根据 (4) 即可证明。
\end{proof}

由于上述命题 (4) 的存在,以后我们混用 $\varepsilon^{i_1} \wedge\cdots\wedge\varepsilon^{i_k}$
和 $\varepsilon^I$ 两种记号。

一个 $k$-余向量 $\eta$ 如果可以表示为 $\eta=\omega^1\wedge\cdots\wedge\omega^k$
的形式,那么我们说 $\eta$ 是\emph{可分解的}。值得注意的是当 $k>1$ 时不是所有 $k$-余向量都是可分解的。
但是根据 \autoref{prop:property of wedge} 的 (4) 和 \autoref{prop:basis of Lambda},
任意 $k$-余向量都可以表示为一些可分解的 $k$-余向量的线性组合。

对于 $n$-维向量空间 $V$,定义向量空间 $\Lambda(V^*)$ 为
\[
  \Lambda(V^*)=\bigoplus_{k=0}^n\Lambda^k(V^*).  
\]
那么 $\dim\Lambda(V^*)=\sum_{k=0}^n \binom{n}{k}=2^n$。楔积的性质
表明 $\Lambda(V^*)$ 是一个结合代数,被称为\emph{$V$ 的外代数}。
一个代数 $A$ 被称为\emph{分次的},如果其有一个直和分解 $A=\bigoplus_{k\in \mathbb{Z}}A^k$
使得 $\bigl(A^k\bigr)\bigl(A^l\bigr)\subseteq A^{k+l}$。如果乘积满足
$ab=(-1)^{kl}ba$ (其中 $a\in A^k$,$b\in A^l$),那么我们说这个分次代数
是\emph{反交换的}。我们已经表明 $\Lambda(V^*)$ 是一个反交换的分次代数。

\subsection{内乘法}

有一个重要的算符可以将向量和交错张量联系起来。令 $V$ 是有限维向量空间,
对于每个 $v\in V$,我们定义线性映射 $i_v:\Lambda^k(V^*)\to\Lambda^{k-1}(V^*)$,
称为\emph{通过 $v$ 的内乘法},定义为
\[
  i_v\omega(w_1,\dots,w_{k-1})=\omega(v,w_1,\dots,w_{k-1}).
\]
换句话说,$i_v\omega$ 即将 $\omega$ 的第一个参数固定为 $v$ 得到的
$k-1$ 多重线性映射。根据惯例,当 $\omega$ 是 $0$-余向量(一个数)的时候,
我们将 $i_v\omega$ 解释为零。另一个常用的记号为
\[
  v\into\omega=i_v\omega.  
\]

\begin{lemma}
  令 $V$ 是有限维向量空间, $v\in V$.
  \begin{enumerate}
    \item $i_v\circ i_v=0$。
    \item 如果 $\omega\in \Lambda^k(V^*)$ 和 $\eta\in\Lambda^l(V^*)$,那么
    \begin{equation}\label{eq:interior multiplication}
      i_v(\omega\wedge\eta)=(i_v\omega)\wedge\eta+(-1)^k\omega\wedge(i_v\eta).
    \end{equation}
  \end{enumerate}
\end{lemma}
\begin{proof}
  (1) 当 $k\geq 2$ 的时候,此时 $i_v(i_v\omega)=\omega(v,v,\dots)$,由于
  $\omega$ 是交错的,所以 $i_v\circ i_v=0$。当 $k=1$ 或者 $0$ 的时候,
  显然 $i_v\circ i_v=0$。

  (2) 只需要考虑 $\omega,\eta$ 均为可分解的情况即可。对于余向量
  $\omega^1,\dots,\omega^k$,任取向量 $v_2,\dots,v_k$,记
  $v_1=v$,有 
  \begin{align*}
    \bigl(v\into\bigl(\omega^1\wedge\cdots\wedge\omega^k\bigr)\bigr)
    (v_2,\dots,v_k)&=\bigl(\omega^1\wedge\cdots\wedge\omega^k\bigr)(v,v_2,\dots,v_k)\\
    &=\det\bigl(\omega^i(v_j)\bigr)\\
    &=\sum_{i=1}^k (-1)^{i-1}\omega^i(v_1)
    \det\bigl(\omega^p(v_q)\bigr)_{p\ne i,q\neq 1},
  \end{align*}
  最后一步即将行列式按第一列展开,这就表明
  \begin{equation}\label{eq:i_v}
    v\into\bigl(\omega^1\wedge\cdots\wedge\omega^k\bigr)
    =\sum_{i=1}^k(-1)^{i-1}\omega^i(v)\omega^1\wedge\cdots\wedge
    \hat\omega^i\wedge\cdots\wedge \omega^k,
  \end{equation}
  其中的尖角符号表明 $\omega^i$ 被跳过。
\end{proof}
 
\section{流形上的微分形式}

回到 $n$-维光滑流形 $M$ 上,记 $T^kT^*M$ 的由交错张量构成的子集为
$\Lambda^kT^*M$:
\[
  \Lambda^k T^*M=\coprod_{p\in M}\Lambda^k\bigl(T_p^*M\bigr).
\]
这是 $T^kT^*M$ 的一个光滑子丛,且是 $M$ 上的秩 $\binom{n}{k}$ 的向量丛。

$\Lambda^kT^*M$ 的截面被称为\emph{微分 $k$-形式},或者简称为\emph{$k$-形式}。
记所有光滑 $k$-形式的向量空间为
\[
  \Omega^k(M)=\Gamma\bigl(\Lambda^kT^*M\bigr)  .
\]

两个微分形式的楔积由逐点定义:$(\omega\wedge\eta)_p=\omega_p\wedge\eta_p$。
因此 $k$-形式和 $l$-形式的楔积为 $(k+l)$-形式。如果 $f$ 是 $0$-形式,
$\eta$ 是 $k$-形式,我们可以将楔积 $f\wedge \eta$ 视为通常的乘积
$f\eta$。如果我们定义
\begin{equation}
  \Omega^*(M)=\bigoplus_{k=0}^n\Omega^k(M),
\end{equation}
那么 $\Omega^*(M)$ 是结合的反对称的分次代数。

在任意光滑坐标卡中,一个 $k$-形式 $\omega$ 可以局部表示为
\[
  \omega=\sideset{}{'}\sum_I\omega_I\d x^{i_1}\wedge
  \cdots\wedge \d x^{i_k}=\sideset{}{'}\sum_I\omega_I \d x^I.
\]
根据楔积的性质,我们有
\[
  \d x^{i_1}\wedge
  \cdots\wedge \d x^{i_k}
  \left(\frac{\partial}{\partial x^{j_1}},\dots,\frac{\partial}{\partial x^{j_k}}\right)
  =\delta_J^I.
\]
因此 $\omega$ 的分量 $\omega_I$ 为
\[
  \omega_I=\omega\left(\frac{\partial}{\partial x^{i_1}},\dots,\frac{\partial}{\partial x^{i_k}}\right)  .
\]

\begin{example}
  一个 $0$-形式就是连续的实值函数,一个 $1$-形式即余向量场。
  $\mathbb{R}^3$ 上的每个 $3$-形式都是 $dx\wedge dy\wedge dz$ 的一个连续实值函数倍。
\end{example}

如果 $F:M\to N$ 是光滑映射,$\omega$ 是 $N$ 上的微分形式,拉回
$F^*\omega$ 是 $M$ 上的微分形式,定义为
\[
  (F^*\omega)_p(v_1,\dots,v_k)=\omega_{F(p)}\bigl(\d F_p(v_1),\dots,\d F_p(v_k)\bigr)  .
\]

\begin{lemma}\label{lemma:pullback of form}
  设 $F:M\to N$ 是光滑映射。
  \begin{enumerate}
    \item $F^*:\Omega^k(N)\to \Omega^k(M)$ 是线性映射。
    \item $F^*(\omega\wedge\eta)=(F^*\omega)\wedge(F^*\eta)$。
    \item 在任意光滑坐标卡中,
    \[
      F^*\left(\sideset{}{'}\sum_I\omega_I\d y^{i_1}\wedge\cdots\wedge\d y^{i_k}\right)  
      =\sideset{}{'}\sum_I(\omega_I\circ F)\d\bigl(y^{i_1}\circ F\bigr)\wedge
      \cdots\wedge \d \bigl(y^{i_k}\circ F\bigr).
    \]
  \end{enumerate}
\end{lemma}

这个引理告诉我们微分形式拉回的计算规则,与协变张量场拉回的计算规则是一致的。

\begin{example}
  定义 $F:\mathbb{R}^2\to \mathbb{R}^3$ 为 $F(u,v)=(u,v,u^2-v^2)$,令
  $\omega=y\d x\wedge\d z+x\d y\wedge\d z$ 是 $\mathbb{R}^3$ 上的 $2$-形式,
  那么拉回 $F^*\omega$ 为
  \begin{align*}
    F^*\omega&=v\d u\wedge\d (u^2-v^2)+u\d v\wedge\d (u^2-v^2)\\
    &=v\d u\wedge(2u\d u-2v\d v)+u\d v\wedge(2u\d u-2v\d v)\\
    &=-2v^2\d u\wedge\d v+2u^2\d v\wedge\d u,
  \end{align*}
  由于 $\d u\wedge\d v=-\d v\wedge \d u$,所以
  \[
    F^*\omega=-2(u^2+v^2)\d u\wedge\d v.  
  \]
\end{example}

\begin{example}
  令 $\omega=dx\wedge dy$ 是 $\mathbb{R}^2$ 上的 $2$-形式。考虑极坐标变换
  $x=r\cos\theta,y=r\sin\theta$,在极坐标中,我们有
  \begin{align*}
    dx\wedge dy&=d(r\cos\theta)\wedge d(r\sin\theta)\\
    &=(\cos\theta \d r-r\sin\theta\d \theta)\wedge (\sin\theta\d r+r\cos\theta \d\theta)\\
    &=r\d r\wedge\d \theta.
  \end{align*}
\end{example}

令人惊讶的是,上述公式和二重积分的极坐标换元公式是一致的!下面的命题推广了这一点。

\begin{proposition}[满次形式的拉回公式]\label{prop:pullback of top form}
  令 $F:M\to N$ 是带边或者无边 $n$-流形之间的光滑映射。如果 $(x^i)$ 和 $(y^j)$
  分别是开子集 $U\subseteq M$ 和 $V\subseteq N$ 上的光滑坐标,$u$ 是 $V$
  上的连续实值函数,那么在 $U\cap F^{-1}(V)$ 上有:
  \begin{equation}
    F^*\bigl(u\d y^1\wedge\cdots\wedge \d y^n\bigr)
    =(u\circ F) (\det DF)\d x^1\wedge\cdots\wedge \d x^n,
  \end{equation}
  其中 $DF$ 是 $F$ 在这个坐标中的 Jacobi 矩阵。
\end{proposition}
\begin{proof}
  根据 \autoref{lemma:pullback of form},有
  \[
    F^*\bigl(u\d y^1\wedge\cdots\wedge \d y^n\bigr)
    =(u\circ F) \d F^1\wedge\cdots\wedge \d F^n.
  \]
  我们还有
  \[
    \d F^i=\frac{\partial F^i}{\partial x^j}\d x^j,
  \]
  所以 
  \begin{align*}
        \d F^1\wedge\cdots\wedge\d F^n&=
    \left(\frac{\partial F^1}{\partial x^{j_1}}\d x^{j_1}\right)
    \wedge\cdots\wedge \left(\frac{\partial F^n}{\partial x^{j_n}}\d x^{j_n}\right)\\
    &=\sum_{\sigma\in S_n} \frac{\partial F^1}{\partial x^{j_{\sigma(1)}}}
    \cdots \frac{\partial F^n}{\partial x^{j_{\sigma(n)}}}
    \d x^{j_{\sigma(1)}}\wedge \cdots\wedge \d x^{j_{\sigma(n)}}\\
    &=\left(\sum_{\sigma\in S_n} (\sgn\sigma)\frac{\partial F^1}{\partial x^{j_{\sigma(1)}}}
    \cdots \frac{\partial F^n}{\partial x^{j_{\sigma(n)}}}\right)
    \d x^1\wedge \cdots\wedge \d x^n\\
    &=\det(DF)\d x^1\wedge \cdots\wedge \d x^n.\qedhere
  \end{align*}
\end{proof}

\begin{corollary}
  如果 $\bigl(U,(x^j)\bigr)$ 和 $\bigl(\wtilde U,(\tilde x^i)\bigr)$ 是 $M$
  上的两个重叠的坐标卡,那么在 $U\cap\wtilde U$ 上有
  \begin{equation}
    \d \tilde x^1\wedge\cdots\wedge \d \tilde x^n=
    \det\left(\frac{\partial\tilde x^i}{\partial x^j}\right) 
    \d x^1\wedge\cdots\wedge \d x^n.
  \end{equation}
\end{corollary}

内乘法也可以自然地拓展到向量场和微分形式上,只需要进行逐点的定义:
如果 $X\in \mathfrak{X}(M)$ 和 $\omega\in \Omega^k(M)$,那么定义一个
$(k-1)$-形式 $X\into\omega=i_X \omega$ 为
\[
  (X\into \omega)_p=X_p\into \omega_p.
\]



\section{外微分}

回顾不是所有的 $1$-形式都是函数的微分:给定一个光滑 $1$-形式 $\omega$,
存在光滑函数 $f$ 使得 $\omega=\d f$ 的一个必要条件是 $\omega$ 是\emph{闭的},
即在每个光滑坐标卡中都有
\begin{equation}
  \frac{\partial\omega_j}{\partial x^i}-\frac{\partial\omega_i}{\partial x^j}=0.
\end{equation}
注意到这个等式左边关于指标 $i,j$ 是反对称的,所以可以解释为一个交错张量场
的 $ij$-分量。于是我们可以定义一个 $2$-形式 $\d\omega$,其在局部坐标卡中为
\begin{equation}
  \d\omega=\sum_{i<j}\left(\frac{\partial\omega_j}{\partial x^i}-\frac{\partial\omega_i}{\partial x^j}\right)
  \d x^i\wedge\d x^j,
\end{equation}
此时 $\omega$ 是闭的当且仅当在每个坐标卡中有 $\d\omega=0$。

事实证明 $\d\omega$ 实际上是全局定义的,独立于坐标卡的选择。对于光滑流形
$M$,我们将证明存在一个可微的算子 $\d:\Omega^k(M)\to\Omega^{k+1}(M)$
使得 $\d(\d\omega)=0$。因此,一个光滑 $k$-形式 $\omega$ 等于 $d\eta$($\eta$ 是
$(k-1)$-形式)的一个必要条件是 $\d\omega=0$。

Euclid 空间中 $d$ 的定义是很直接的:如果 $\omega=\sideset{}{'}{\textstyle\sum_J}\omega_J\d x^J$
是开集 $U\subseteq \mathbb{R}^n$ 上的光滑 $k$-形式,那么我们定义\emph{外微分}
$\d\omega$ 为 $(k+1)$-形式:
\begin{equation}\label{eq:exterior derivate}
  \d\left(\sideset{}{'}{\sum}_J\omega_J\d x^J\right)=
  \sideset{}{'}{\sum}_J\d\omega_J\wedge\d x^J,
\end{equation}
其中 $d\omega_J$ 是函数 $\omega_J$ 的微分。写的更详细一点,即
\begin{equation}
  \d\left(\sideset{}{'}{\sum}_J\omega_J\d x^{j_1}\wedge\cdots\wedge\d x^{j_k}\right)=
  \sideset{}{'}{\sum}_J\sum_i\frac{\partial\omega_J}{\partial x^{i}}
  \d x^i\wedge \d x^{j_1}\wedge\cdots\wedge\d x^{j_k}.
\end{equation}
注意到当 $\omega$ 是 $1$-形式的时候,这表明
\begin{align*}
  \d\bigl(\omega_j\d x^j\bigr)&=\sum_{i,j}\frac{\partial\omega_j}{\partial x^i}
  \d x^i\wedge\d x^j\\
  &=\sum_{i<j}\frac{\partial\omega_j}{\partial x^i}
  \d x^i\wedge\d x^j+\sum_{i> j}\frac{\partial\omega_j}{\partial x^i}
  \d x^i\wedge\d x^j\\
  &=\sum_{i<j}\left(\frac{\partial\omega_j}{\partial x^i}-\frac{\partial\omega_i}{\partial x^j}\right)
  \d x^i\wedge\d x^j,
\end{align*}
所以这与前面的定义是一致的。对于光滑 $0$-形式 $f$,则变为 
\[
  \d f=\frac{\partial f}{\partial x^i}\d x^i.  
\]
即函数的微分。

\begin{proposition}[$\mathbb{R}^n$ 上的外微分的性质]\label{prop:property of d}
  \mbox{}
  \begin{enumerate}
    \item $d$ 在 $\mathbb{R}$ 上是线性的。
    \item 如果 $\omega$ 是光滑 $k$-形式,$\eta$ 是光滑 $1$-形式,那么
    \[
      \d(\omega\wedge\eta)=  \d\omega\wedge\eta+(-1)^k\omega\wedge\d \eta.
    \]
    \item $\d\circ\d \equiv 0$。
    \item $d$ 和拉回操作交换:如果 $U\subseteq \mathbb{R}^n$ 和 $V\subseteq \mathbb{R}^m$
    是开集,$F:U\to V$ 是光滑映射,$\omega\in\Omega^k(V)$,那么
    \begin{equation}
      F^*(\d\omega)=\d\bigl(F^*\omega\bigr).
    \end{equation}
  \end{enumerate}
\end{proposition}
\begin{proof}
  线性性是显然的。对于 (2),由于线性性,我们只需要考虑 $\omega=u\d x^I\in\Omega^k(U)$
  和 $\eta=v\d x^J\in\Omega^l(U)$ 的情况即可。首先,我们证明对于任意多重指标 $I$,
  有 $\d(u\d x^I)=\d u\wedge\d x^I$。如果 $I$ 有重复指标,那么 $\d x^I=0$,所以
  $\d(u\d x^I)=0=\d u\wedge\d x^I$。如果 $I$ 没有重复指标,设 $\sigma$
  为将 $I$ 变为递增指标 $J$ 的置换,那么
  \[
    \d\bigl(u\d x^I\bigr)=(\sgn\sigma)\d\bigl(u\d x^J\bigr)
    = (\sgn\sigma)\d u\wedge\d x^J=\d u\wedge\d x^I.
  \]

  利用这一点,我们计算得
  \begin{align*}
    \d(\omega\wedge\eta)&=\d\bigl((u\d x^I)\wedge (v\d x^J)\bigr)\\
    &=\d\bigl(uv\d x^I\wedge\d x^J\bigr)\\
    &=(u\d v+v\d u)\wedge \d x^I\wedge\d x^J\\
    &=\bigl(\d u\wedge \d x^I\bigr)\wedge\bigl(v\d x^J\bigr)
    +(-1)^k\bigl(u\d x^I\bigr)\wedge \bigl(\d v\wedge\d x^J\bigr)\\
    &=\d \omega\wedge \eta+(-1)^k\omega\wedge \d\eta.
  \end{align*}

  对于 (3),首先证明 $0$-形式的情况,此时
  \begin{align*}
    \d(\d u)&=\d \left(\frac{\partial u}{\partial x^j}\d x^j\right)
    =\frac{\partial^2 f}{\partial x^i\partial x^j}\d x^i\wedge\d x^j\\
    &=\sum_{i<j}\left(\frac{\partial^2 f}{\partial x^i\partial x^j}-\frac{\partial^2 f}{\partial x^j\partial x^i}\right)
    \d x^i\wedge\d x^j=0.
  \end{align*}
  对于一般的情况,利用 (2),我们有
  \begin{align*}
    \d(\d\omega)&=\d\left(\sideset{}{'}\sum_J\d\omega_J\wedge\d x^{j_1}\wedge\cdots\wedge\d x^{j_k}\right)\\
    &=\sideset{}{'}\sum_J\d(\d\omega)\wedge\d x^{j_1}\wedge\cdots\wedge\d x^{j_k}+\\
    &\hphantom{{}={}}\sideset{}{'}\sum_J
    \sum_{i=1}^k(-1)^i\d\omega_J\wedge\d x^{j_1}\wedge\cdots\wedge
    \d\bigl(\d x^{j_i}\bigr)\wedge\cdots\wedge \d x^{j_k}=0.
  \end{align*}

  对于 (4),同样只需要考虑 $\omega=u\d x^{i_1}\wedge\cdots\wedge\d x^{i_k}$ 的情况,
  此时
  \begin{align*}
    F^*\Bigl(\d\bigl(u\d x^{i_1}\wedge\cdots\wedge\d x^{i_k}\bigr)\Bigr)&=
    F^*\bigl(\d u\wedge\d x^{i_1}\wedge\cdots\wedge\d x^{i_k} \bigr)\\
    &=\d(u\circ F)\wedge \d\bigl(x^{i_1}\circ F\bigr)\wedge\cdots\wedge 
    \d\bigl(x^{i_k}\circ F \bigr).
  \end{align*}
  另一方面,有
  \begin{align*}
    \d\Bigl(F^*\bigl(u\d x^{i_1}\wedge\cdots\wedge\d x^{i_k}\bigr)\Bigr)
    &=\d\Bigl((u\circ F)\d\bigl(x^{i_1}\circ F\bigr)\wedge\cdots\wedge 
    \d\bigl(x^{i_k}\circ F \Bigr)\\
    &=\d(u\circ F)\wedge \d\bigl(x^{i_1}\circ F\bigr)\wedge\cdots\wedge 
    \d\bigl(x^{i_k}\circ F \bigr),
  \end{align*}
  所以 $\d$ 和 $F^*$ 可交换。
\end{proof}

这些结果允许我们将外微分的定义转移到流形上。

\begin{theorem}[外微分的存在唯一性]
  令 $M$ 是光滑流形,对于所有的 $k$ 存在唯一的算子 $\d:\Omega^k(M)\to\Omega^{k+1}(M)$,
  称为\emph{外微分},其满足下面的四条性质:
  \begin{enumerate}
    \item $d$ 在 $\mathbb{R}$ 上是线性的。
    \item 如果 $\omega\in\Omega^k(M)$ 和 $\eta\in\Omega^l(M)$,那么
    \[
      \d(\omega\wedge\eta)=\d\omega\wedge\eta+(-1)^k\omega\wedge \d\eta.  
    \]
    \item $\d\circ\d \equiv 0$。
    \item 对于 $f\in\Omega^0(M)=C^\infty(M)$,$\d f$ 是 $f$ 的微分,即
    $\d f(X)=Xf$。
  \end{enumerate}
  在任意光滑坐标卡中,$\d$ 由 \eqref{eq:exterior derivate} 给出。
\end{theorem}
\begin{proof}
  首先,我们证明存在性。假设 $\omega\in\Omega^k(M)$。我们希望在每个坐标卡
  中依据 \eqref{eq:exterior derivate} 定义 $\d\omega$。更准确地说,这意味着
  对于每个光滑坐标卡 $(U,\varphi)$,我们希望定义
  \begin{equation}
    \d\omega=\varphi^*\d\left({\varphi^{-1}}^*\omega\right).
  \end{equation}
  为了说明这是良定义的,我们只需要注意到对于任意其他的光滑坐标卡 $(V,\psi)$,
  映射 $\varphi\circ\psi^{-1}$ 是 $\mathbb{R}^n$ 或者 $\mathbb{H}^n$
  的开子集之间的微分同胚,所以 \autoref{prop:property of d} 的 (4) 表明
  \[
    {\psi^{-1}}^*\varphi^*\d\bigl({\varphi^{-1}}^*\omega\bigr)=
    \bigl(\varphi\circ\psi^{-1}\bigr)^*\d\bigl({\varphi^{-1}}^*\omega\bigr)
    =\d\bigl({\psi^{-1}}^*\omega\bigr).
  \]
  这就表明 $\varphi^*\d\bigl({\varphi^{-1}}^*\omega\bigr)=\psi^*\d\bigl({\psi^{-1}}^*\omega\bigr)$,
  所以 $\d\omega$ 是良定义的。
  根据 \autoref{prop:property of d},$\d$ 自然满足性质 (1)--(4)。

  为了证明唯一性,假设 $\d$ 是另一个满足 (1)--(4) 的算子。首先我们说明
  $\d\omega$ 是局部定义的:如果 $k$-形式 $\omega_1$ 和 $\omega_2$ 在某个开集 $U\subseteq M$
  上相等,那么在 $U$ 上有 $\d\omega_1=\d\omega_2$。为了说明这一点,
  任取 $p\in U$,令 $\eta=\omega_1-\omega_2$,$\psi\in C^\infty(M)$
  是 $p$ 的某个邻域上为 $1$ 的支在 $U$ 中的鼓包函数。那么 $\psi\eta$
  恒等于零,所以性质 (1)--(4) 导出 $0=\d(\psi\eta)=\d\psi\wedge\eta+\psi\d\eta$。
  观察其在 $p$ 处的值,利用 $\psi(p)=1$ 以及 $\d\psi_p=0$,我们得到
  $\d\omega_1|_p-\d\omega_2|_p=\d\eta_p=0$。

  现在令 $\omega\in\Omega^k(M)$ 是任意 $k$-形式,$(U,\varphi)$ 是任意光滑坐标卡。
  在 $U$ 上,$\omega$ 可以写为 $\sideset{}{'}{\sum}_I\omega_I\d x^I$。对于 
  任意 $p\in U$,我们可以利用鼓包函数构造 $M$ 上的全局的光滑函数 $\tilde\omega_I$
  和 $\tilde{x}^i$ 使得他们在 $p$ 的某个邻域上和 $\omega_I$ 以及 $x^i$ 相等。
  根据性质 (1)--(4) 以及上一段的叙述,在 $p$ 处我们有 \eqref{eq:exterior derivate}
  式。因为 $p$ 的任意性,所以 $\d$ 必须等于我们最开始定义的外微分。
\end{proof}

如果 $A=\bigoplus_k A^k$ 是分次代数,线性映射 $T:A\to A$ 如果对于每个 $k$,
都有 $T(A^k)\subseteq A^{k+m}$,那么称 $T$ 是\emph{$m$ 次映射}。
如果 $x\in A^k,y\in A^l$ 时有 $T(xy)=(Tx)y+(-1)^kx(Ty)$,那么称
$T$ 是\emph{反导子}。前面的定理可以总结为函数的微分可以唯一地延拓为
$\Omega^*(M)$ 上的一个 $1$ 次的反导子且平方为零。

\begin{proposition}[外微分的自然性]
  如果 $F:M\to N$ 是光滑映射,那么对于每个 $k$,拉回映射
  $F^*:\Omega^k(N)\to\Omega^k(M)$ 与 $d$ 交换:对于任意 $\omega\in\Omega^k(N)$,
  有
  \begin{equation}\label{eq:naturality of d}
    F^*(\d\omega)=\d (F^*\omega).
  \end{equation}
\end{proposition}
\begin{proof}
  如果 $(U,\varphi)$ 和 $(V,\psi)$ 分别是 $M$ 和 $N$ 上的光滑坐标卡,
  我们将 \autoref{prop:property of d} 的 (4) 应用于坐标表示 $\psi\circ F\circ\varphi^{-1}$。
  在 $U\cap F^{-1}(V)$ 上,我们有
  \begin{align*}
    F^*(\d\omega)&=F^*\psi^*\d\bigl({\psi^{-1}}^*\omega\bigr)\\
    &=\varphi^*(\psi\circ F\circ\varphi^{-1})^*\d\bigl({\psi^{-1}}^*\omega\bigr)\\
    &=\varphi^*\d\bigl((\psi\circ F\circ\varphi^{-1})^*{\psi^{-1}}^*\omega\bigr)\\
    &=\varphi^*\d\bigl({\varphi^{-1}}^*F^*\omega\bigr)\\
    &=\d(F^*\omega).\qedhere
  \end{align*}
\end{proof}

现在我们可以把余向量场的术语进行一个延申。一个微分形式 $\omega\in\Omega^k(M)$
如果使得 $\d\omega=0$,那么我们说 $\omega$ 是\emph{闭形式}。如果存在
$(k-1)$-形式 $\eta$ 使得 $\omega=\d\eta$,那么我们说 $\omega$ 是\emph{恰当形式}。

$\d\circ\d =0$ 表明恰当形式都是闭形式。我们已经看到闭形式不一定都是恰当形式。
换句话说,序列
\[
  \begin{tikzcd}
    \Omega^k(M)\arrow[r,"\d"] & \Omega^{k+1}(M)\arrow[r,"\d"]
    & \Omega^{k+2}(M)
  \end{tikzcd}  
\]
不是正合列。

\subsection{$\mathbb{R}^3$ 中的外微分和向量微积分}

\begin{example}\label{exa:caculus on R3}
  我们来计算 $\mathbb{R}^3$ 上任意 $1$-形式和 $2$-形式的外微分。
  任意光滑 $1$-形式都形如
  \[
    \omega=P\d x+Q\d y+R\d z,  
  \]
  其中 $P,Q,R$ 是光滑函数。那么
  \begin{align*}
    \d\omega&=\d P\wedge\d x+\d Q\wedge\d y+\d R\wedge \d z\\
    &=
    \left(\frac{\partial P}{\partial x}\d x+\frac{\partial P}{\partial y}\d y+\frac{\partial P}{\partial z}\d z\right)
    \wedge\d x+ 
    \left(\frac{\partial Q}{\partial x}\d x+\frac{\partial Q}{\partial y}\d y+\frac{\partial Q}{\partial z}\d z\right)
    \wedge\d y\\
    &\hphantom{{}={}}+
    \left(\frac{\partial R}{\partial x}\d x+\frac{\partial R}{\partial y}\d y+\frac{\partial R}{\partial z}\d z\right)
    \wedge\d z\\
    &=\left(\frac{\partial Q}{\partial x}-\frac{\partial P}{\partial y}\right)\d x\wedge\d y
    +\left(\frac{\partial R}{\partial x}-\frac{\partial P}{\partial z}\right)\d x\wedge\d z
    +\left(\frac{\partial R}{\partial y}-\frac{\partial Q}{\partial z}\right)\d y\wedge\d z.
  \end{align*}
  任意光滑 $2$-形式可以写为
  \[
    \eta=u\d x\wedge\d y+v\d x\wedge\d z+w\d y\wedge\d z.  
  \]
  那么
  \begin{align*}
    \d \eta&=\d u\wedge\d x\wedge \d y+\d v\wedge\d x\wedge\d z+\d w\wedge\d y\wedge\d z\\
    &=\left(\frac{\partial u}{\partial x}\d x+\frac{\partial u}{\partial y}\d y+\frac{\partial u}{\partial z}\d z\right)
    \wedge \d x\wedge \d y\\
    &\hphantom{{}={}}+\left(\frac{\partial v}{\partial x}\d x+\frac{\partial v}{\partial y}\d y+\frac{\partial v}{\partial z}\d z\right)
    \wedge \d x\wedge \d z\\
    &\hphantom{{}={}}+\left(\frac{\partial w}{\partial x}\d x+\frac{\partial w}{\partial y}\d y+\frac{\partial w}{\partial z}\d z\right)
    \wedge \d y\wedge \d z\\
    &=\left(\frac{\partial u}{\partial z}-\frac{\partial v}{\partial y}+\frac{\partial w}{\partial x}\right)
    \d x\wedge\d y\wedge \d z.
  \end{align*}
\end{example}

回顾 $\mathbb{R}^n$ 上的向量微积分算符:函数 $f\in C^\infty(\mathbb{R^n})$
的 (Euclid) 梯度和向量场 $X\in \mathfrak{X}(\mathbb{R}^n)$ 的\emph{散度}
分别定义为
\begin{equation}
  \grad f=\sum_{i=1}^n\frac{\partial f}{\partial x^i}\frac{\partial}{\partial x^i},\quad 
  \dive X=\sum_{i=1}^n\frac{\partial X^i}{\partial x^i}.
\end{equation}
此外,在 $n=3$ 的时候,向量场 $X\in \mathfrak{X}(\mathbb{R}^3)$ 的\emph{旋度}
被定义为
\[
  \curl X=\left(\frac{\partial X^3}{\partial y}-\frac{\partial X^2}{\partial z}\right)
  \frac{\partial}{\partial x}+  \left(\frac{\partial X^1}{\partial z}-\frac{\partial X^3}{\partial x}\right)
  \frac{\partial}{\partial y}+
  \left(\frac{\partial X^2}{\partial x}-\frac{\partial X^1}{\partial y}\right)
  \frac{\partial}{\partial z}.
\]
一个有趣的事实是,前面例子中 $2$-形式 $\d\omega$ 的分量正好是分量函数为
$(P,Q,R)$ 的向量场的旋度的分量(除开可能相差一个符号之外)。类似的,
除开符号差异之外,$\d\eta$ 的分量和向量场散度的分量也非常相似。
这些相似性可以由下面的方式精确叙述。

$\mathbb{R}^3$ 上的 Euclid 度量导出指标下降同构:
$\flat:\mathfrak{X}(\mathbb{R}^3)\to \Omega^1(\mathbb{R}^3)$。
即将向量场 $X=X^i\partial/\partial x^i$ 送到
\[
  X^\flat=X_j\d x^j=g_{ij}X^i\d x^j=\sum_{i=1}^3 X^i\d x^i.
\]
内乘法导出了另一个映射 $\beta:\mathfrak{X}(\mathbb{R}^3)\to \Omega^2(\mathbb{R}^3)$
为
\begin{equation}
  \beta(X)=X\into (\d x\wedge\d y\wedge \d z).
\end{equation}
不难验证 $\beta$ 是 $C^\infty(\mathbb{R}^3)$-线性的,所以其对应一个
光滑丛同态 $T \mathbb{R}^3\to \Lambda^2T^*\mathbb{R}^3$。这是一个单射且
$T \mathbb{R}^3$ 和 $\Lambda^2T^*\mathbb{R}^3$ 的秩都是 $3$,所以这是一个丛同构。
类似的,我们可以定义光滑丛同构 $*:C^\infty(\mathbb{R}^3)\to \Omega^3(\mathbb{R}^3)$
为
\begin{equation}
  *(f)=f\d x\wedge\d y\wedge\d z.
\end{equation}
那么上述所有算符的关系可以总结为如下交换图:
\begin{equation}
  \begin{tikzcd}[row sep=large]
    C^\infty(\mathbb{R}^3)\arrow[r,"\grad"]\arrow[d,"\Id"]
    & \mathfrak{X}(\mathbb{R}^3)\arrow[r,"\curl"]\arrow[d,"\flat"]
    & \mathfrak{X}(\mathbb{R}^3)\arrow[r,"\dive"]\arrow[d,"\beta"]
    & C^\infty(\mathbb{R}^3)\arrow[d,"*"]
    \\
    \Omega^0(\mathbb{R}^3)\arrow[r,"\d"] & 
    \Omega^1(\mathbb{R}^3)\arrow[r,"\d"] & 
    \Omega^2(\mathbb{R}^3)\arrow[r,"\d"] & 
    \Omega^3(\mathbb{R}^3)
  \end{tikzcd}
\end{equation}

\begin{proof}
  我们证明上面的图确实是交换的。首先任取 $f\in C^\infty(\mathbb{R}^3)$,
  有 $(\grad f)^\flat =\d f$,这就表明左半部分是交换的。
  然后任取 $X=P\partial/\partial x+Q\partial/\partial y+R\partial/\partial z\in \mathfrak{X}(\mathbb{R}^3)$,
  我们有
  \begin{align*}
    \beta(\curl X)&=
    \left(\frac{\partial R}{\partial y}-\frac{\partial Q}{\partial z}\right)
    \frac{\partial}{\partial x}\into 
    (\d x\wedge\d y\wedge\d z)\\
    &\hphantom{{}={}}
    +\left(\frac{\partial P}{\partial z}-\frac{\partial R}{\partial x}\right)
    \frac{\partial}{\partial y}\into 
    (\d x\wedge\d y\wedge\d z)\\
    &\hphantom{{}={}}
    +\left(\frac{\partial Q}{\partial x}-\frac{\partial P}{\partial y}\right)
    \frac{\partial}{\partial z}\into 
    (\d x\wedge\d y\wedge\d z),
  \end{align*}
  由于
  \begin{equation*}
    \frac{\partial}{\partial x}\into\d x=1,\quad
    \frac{\partial}{\partial x}\into\d y=0,\quad
    \frac{\partial}{\partial x}\into\d z=0,
  \end{equation*}
  所以
  \begin{equation*}
    \frac{\partial}{\partial x}\into 
    (\d x\wedge\d y\wedge\d z)=
    \left(\frac{\partial}{\partial x}\into\d x\right)
    \wedge\d y\wedge\d z=\d y\wedge \d z,
  \end{equation*}
  类似的有
  \begin{align*}
    \frac{\partial}{\partial y}\into 
    (\d x\wedge\d y\wedge\d z)&=-\d x\wedge \d z,\\
    \frac{\partial}{\partial z}\into 
    (\d x\wedge\d y\wedge\d z)&=\d x\wedge \d y,
  \end{align*}
  所以 
  \[
    \beta(\curl X)= 
    \left(\frac{\partial R}{\partial y}-\frac{\partial Q}{\partial z}\right)
    \d y\wedge\d z
    -\left(\frac{\partial P}{\partial z}-\frac{\partial R}{\partial x}\right)
    \d x\wedge\d z
    +\left(\frac{\partial Q}{\partial x}-\frac{\partial P}{\partial y}\right)
    \d x\wedge\d y.
  \]
  结合 \autoref{exa:caculus on R3},所以 $\beta(\curl X)=\d(X^\flat)$。
  这表明中间部分是交换的。

  最后,我们有
  \begin{align*}
    *(\dive X)&=\left(\frac{\partial P}{\partial x}+\frac{\partial Q}{\partial y}+\frac{\partial R}{\partial z}\right)
    \d x\wedge\d y\wedge\d z.
  \end{align*}
  另一方面,结合 \autoref{exa:caculus on R3},有
  \begin{align*}
    \d(\beta(X))&=\d\bigl(P\d y\wedge\d z-Q\d x\wedge\d z+R\d x\wedge\d y\bigr)\\
    &=\left(\frac{\partial P}{\partial x}+\frac{\partial Q}{\partial y}+\frac{\partial R}{\partial z}\right)
    \d x\wedge\d y\wedge\d z\\
    &=*(\dive X).
  \end{align*}
  这就表明右半部分是交换的。
\end{proof}

\begin{remark}
  实际上,该图的左半部分和右半部分替换为 $\mathbb{R}^n$ 仍然是交换的。
  将 $\mathbb{R}^3$ 上的向量微积分推广到更高维度也是微分形式理论发展
  的主要动机之一。特别地,旋度仅仅在 $3$ 维空间的向量场上有意义,但是
  外微分是不受维数限制的。
\end{remark}

\subsection{外微分的形式不变性}

除开我们使用的 $\d$ 的坐标定义 \eqref{eq:exterior derivate} 式,
还存在常用的 $\d$ 的另一个公式,这个公式是坐标无关的。我们从 $1$-形式的
外微分开始。

\begin{proposition}[$1$-形式的外微分]
  对于任意光滑 $1$-形式 $\omega$ 和任意光滑向量场 $X,Y$,有
  \begin{equation}
    \d\omega(X,Y)=X(\omega(Y))-Y(\omega(X))-\omega([X,Y]).
  \end{equation}
\end{proposition}




