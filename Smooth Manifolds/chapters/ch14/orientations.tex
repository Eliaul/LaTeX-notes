\chapter{定向}

\section{向量空间的定向}

令 $V$ 是维数 $n\ge 1$ 的向量空间。设
$(E_1,\dots,E_n)$ 和 $\bigl(\wtilde E_1,\dots,\wtilde E_n\bigr)$
是 $V$ 的两组有序基,
如果过渡矩阵 $\bigl(B_j^i\bigr)$ 的行列式为正,那么我们说这两组基
是\emph{一致定向的},其中过渡矩阵定义为
\[
  E_j=B^i_j\wtilde E_i.  
\]
显然一致定向是 $V$ 的所有有序基集合上的一个等价关系。而且这个等价关系恰好
划分了两个等价类,
任取一组有序基 $(E_1,\dots,E_n)$,显然 $(-E_1,E_2,\dots,E_n)$
和它不是一致定向的,所以这两个有序基处于不同的等价类。
对于任意一组有序基 $\bigl(\wtilde E_1,\dots,\wtilde E_n\bigr)$,
如果其不与 $(E_1,\dots,E_n)$ 一致定向,即 $(E_1,\dots,E_n)$
到 $\bigl(\wtilde E_1,\dots,\wtilde E_n\bigr)$ 的过渡矩阵行列式为负,
那么 $(-E_1,E_2,\dots,E_n)$ 到 $\bigl(\wtilde E_1,\dots,\wtilde E_n\bigr)$ 的过渡矩阵
行列式为正,所以 $\bigl(\wtilde E_1,\dots,\wtilde E_n\bigr)$ 和
$(-E_1,E_2,\dots,E_n)$ 一致定向,故这个等价关系划分了两个等价类。

如果 $\dim V=n\ge 1$,定义 \emph{$V$ 的定向} 是有序基的一个等价类。
如果 $(E_1,\dots,E_n)$ 是 $V$ 的任意一组有序基,我们记其确定的定向为
$[E_1,\dots,E_n]$,以及其相反的定向为 $-[E_1,\dots,E_n]$。
一个附带了定向的向量空间被称为\emph{定向向量空间}。如果 $V$
是定向的,那么在给定定向中的任一有序基 $(E_1,\dots,E_n)$ 都被称为是
\emph{定向的}或者\emph{正定向的}。不在给定定向中的任意基都被称为
是\emph{负定向的}。

对于零维向量空间 $V$ 的特殊情况,我们定义 $V$ 的定向在数 $\pm 1$
中选取。

\begin{example}
  $\mathbb{R}^n$ 的由标准基确定的定向 $[e_1,\dots,e_n]$ 被称为\emph{标准定向}。
  $\mathbb{R}^0$ 的标准定向被定义为 $+1$。
\end{example}

利用交错张量对参数顺序敏感的特性,定向和交错张量之间存在密切联系。

\begin{proposition}
  令 $V$ 是 $n$ 维向量空间。每个非零张量 $\omega\in\Lambda^n(V^*)$
  都确定了一个定向 $\mathcal{O}_\omega$ 为:如果 $n\ge 1$,那么 
  $\mathcal{O}_\omega$ 是有序基 $(E_1,\dots,E_n)$ 的集合,
  其中 $\omega(E_1,\dots,E_n)>0$。如果 $n=0$,那么 $\omega>0$
  的时候 $\mathcal{O}_\omega=+1$,$\omega<0$ 的时候 $\mathcal{O}_\omega=-1$。
  两个非零 $n$-余向量确定了同一个定向当且仅当其中一个是另一个的正的倍数。 
\end{proposition}
\begin{proof}
  $0$-维的情况是显然的,因为 $\Lambda^0(V^*)=\mathbb{R}$。
  假设 $n\ge 1$,对于非零 $\omega\in \Lambda^n(V^*)$,
  我们证明 $\mathcal{O}_\omega$ 确实是一个等价类。

  假设 $(E_i)$ 和 $\bigl(\wtilde E_j\bigr)$ 是 $V$ 的两组有序基,
  $B:V\to V$ 是将 $E_j$ 送到 $\wtilde E_j$ 的线性映射。
  也就是说 $\wtilde E_j=BE_j=B^i_jE_i$,那么根据
  \autoref{prop:property of alt},有
  \[
    \omega\bigl(\wtilde E_1,\dots,\wtilde E_n\bigr)  =
    (\det B)\omega(E_1,\dots,E_n).
  \]
  这表明 $\bigl(\wtilde E_j\bigr)$ 和 $(E_i)$ 是一致定向的当且仅当
  $\omega\bigl(\wtilde E_1,\dots,\wtilde E_n\bigr)$ 和 $\omega(E_1,\dots,E_n)$
  的符号相同,这就表明 $\mathcal{O}_\omega$ 确实是一个等价类。
\end{proof}

如果 $V$ 是定向的 $n$-维向量空间,$\omega$ 是上述确定了 $V$ 的定向的
$n$-余向量,那么我们说 $\omega$ 是\emph{(正)定向的 $n$-余向量}。
例如,$n$-余向量 $e^1\wedge\cdots\wedge e^n$ 是关于 $\mathbb{R}^n$
的标准定向的正定向余向量。

注意到 $\Lambda^n(V^*)$ 是 $1$ 维向量空间,所以选取 $V$ 的定向等价于选取
$\Lambda^n(V^*) \smallsetminus\{0\}$ 的分支。

\section{流形的定向}

令 $M$ 是光滑流形。我们定义\emph{$M$ 上的逐点定向}是在每个切空间中的
一个定向的选择。但是这并不是一个非常有用的概念,因为邻近点的定向可能
没有任何关系。为了使得定向和光滑结构产生联系,我们需要额外的条件
确保邻近切空间的定向是彼此一致的。

令 $M$ 是光滑 $n$-流形,配备一个逐点定向。如果 $(E_i)$ 是 $TM$
的一个局部标架,且对于每个 $p\in U$,$(E_1|_p,\dots,E_n|_p)$
都是 $T_pM$ 的正定向的基,那么我们说 $(E_i)$ 是\emph{(正)定向的}。
\emph{负定向}的概念做类似的定义。
 
如果 $M$ 的每个点都在一个定向的局部标架的定义域中,那么我们说
这个逐点定向是\emph{连续的}。定义\emph{$M$ 的定向}为一个连续的逐点定向。
如果 $M$ 存在一个定向,那么我们说 $M$ 是\emph{可定向的},否则称
$M$ 是\emph{不可定向的}。一个\emph{定向流形}指的是有序对 $(M,\mathcal{O})$,
其中 $M$ 是可定向的光滑流形,$\mathcal{O}$ 是 $M$ 的一个定向。
对于每个 $p\in M$,由 $\mathcal{O}$ 确定的 $T_pM$ 的定向记为 $\mathcal{O}_p$。

  





