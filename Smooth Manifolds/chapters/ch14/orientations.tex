\chapter{定向}

\section{向量空间的定向}

令 $V$ 是维数 $n\ge 1$ 的向量空间。设
$(E_1,\dots,E_n)$ 和 $\bigl(\wtilde E_1,\dots,\wtilde E_n\bigr)$
是 $V$ 的两组有序基,
如果过渡矩阵 $\bigl(B_j^i\bigr)$ 的行列式为正,那么我们说这两组基
是\emph{一致定向的},其中过渡矩阵定义为
\[
  E_j=B^i_j\wtilde E_i.  
\]
显然一致定向是 $V$ 的所有有序基集合上的一个等价关系。而且这个等价关系恰好
划分了两个等价类,
任取一组有序基 $(E_1,\dots,E_n)$,显然 $(-E_1,E_2,\dots,E_n)$
和它不是一致定向的,所以这两个有序基处于不同的等价类。
对于任意一组有序基 $\bigl(\wtilde E_1,\dots,\wtilde E_n\bigr)$,
如果其不与 $(E_1,\dots,E_n)$ 一致定向,即 $(E_1,\dots,E_n)$
到 $\bigl(\wtilde E_1,\dots,\wtilde E_n\bigr)$ 的过渡矩阵行列式为负,
那么 $(-E_1,E_2,\dots,E_n)$ 到 $\bigl(\wtilde E_1,\dots,\wtilde E_n\bigr)$ 的过渡矩阵
行列式为正,所以 $\bigl(\wtilde E_1,\dots,\wtilde E_n\bigr)$ 和
$(-E_1,E_2,\dots,E_n)$ 一致定向,故这个等价关系划分了两个等价类。

如果 $\dim V=n\ge 1$,定义 \emph{$V$ 的定向} 是有序基的一个等价类。
如果 $(E_1,\dots,E_n)$ 是 $V$ 的任意一组有序基,我们记其确定的定向为
$[E_1,\dots,E_n]$,以及其相反的定向为 $-[E_1,\dots,E_n]$。
一个附带了定向的向量空间被称为\emph{定向向量空间}。如果 $V$
是定向的,那么在给定定向中的任一有序基 $(E_1,\dots,E_n)$ 都被称为是
\emph{定向的}或者\emph{正定向的}。不在给定定向中的任意基都被称为
是\emph{负定向的}。

对于零维向量空间 $V$ 的特殊情况,我们定义 $V$ 的定向在数 $\pm 1$
中选取。

\begin{example}
  $\mathbb{R}^n$ 的由标准基确定的定向 $[e_1,\dots,e_n]$ 被称为\emph{标准定向}。
  $\mathbb{R}^0$ 的标准定向被定义为 $+1$。
\end{example}

利用交错张量对参数顺序敏感的特性,定向和交错张量之间存在密切联系。

\begin{proposition}
  令 $V$ 是 $n$ 维向量空间。每个非零张量 $\omega\in\Lambda^n(V^*)$
  都确定了一个定向 $\mathcal{O}_\omega$ 为:如果 $n\ge 1$,那么 
  $\mathcal{O}_\omega$ 是有序基 $(E_1,\dots,E_n)$ 的集合,
  其中 $\omega(E_1,\dots,E_n)>0$。如果 $n=0$,那么 $\omega>0$
  的时候 $\mathcal{O}_\omega=+1$,$\omega<0$ 的时候 $\mathcal{O}_\omega=-1$。
  两个非零 $n$-余向量确定了同一个定向当且仅当其中一个是另一个的正的倍数。 
\end{proposition}
\begin{proof}
  $0$-维的情况是显然的,因为 $\Lambda^0(V^*)=\mathbb{R}$。
  假设 $n\ge 1$,对于非零 $\omega\in \Lambda^n(V^*)$,
  我们证明 $\mathcal{O}_\omega$ 确实是一个等价类。

  假设 $(E_i)$ 和 $\bigl(\wtilde E_j\bigr)$ 是 $V$ 的两组有序基,
  $B:V\to V$ 是将 $E_j$ 送到 $\wtilde E_j$ 的线性映射。
  也就是说 $\wtilde E_j=BE_j=B^i_jE_i$,那么根据
  \autoref{prop:property of alt},有
  \[
    \omega\bigl(\wtilde E_1,\dots,\wtilde E_n\bigr)  =
    (\det B)\omega(E_1,\dots,E_n).
  \]
  这表明 $\bigl(\wtilde E_j\bigr)$ 和 $(E_i)$ 是一致定向的当且仅当
  $\omega\bigl(\wtilde E_1,\dots,\wtilde E_n\bigr)$ 和 $\omega(E_1,\dots,E_n)$
  的符号相同,这就表明 $\mathcal{O}_\omega$ 确实是一个等价类。
\end{proof}

如果 $V$ 是定向的 $n$-维向量空间,$\omega$ 是上述确定了 $V$ 的定向的
$n$-余向量,那么我们说 $\omega$ 是\emph{(正)定向的 $n$-余向量}。
例如,$n$-余向量 $e^1\wedge\cdots\wedge e^n$ 是关于 $\mathbb{R}^n$
的标准定向的正定向余向量。

注意到 $\Lambda^n(V^*)$ 是 $1$ 维向量空间,所以选取 $V$ 的定向等价于选取
$\Lambda^n(V^*) \smallsetminus\{0\}$ 的分支。

\section{流形的定向}

令 $M$ 是光滑流形。我们定义\emph{$M$ 上的逐点定向}是在每个切空间中的
一个定向的选择。但是这并不是一个非常有用的概念,因为邻近点的定向可能
没有任何关系。为了使得定向和光滑结构产生联系,我们需要额外的条件
确保邻近切空间的定向是彼此一致的。

令 $M$ 是光滑 $n$-流形,配备一个逐点定向。如果 $(E_i)$ 是 $TM$
的一个局部标架,且对于每个 $p\in U$,$(E_1|_p,\dots,E_n|_p)$
都是 $T_pM$ 的正定向的基,那么我们说 $(E_i)$ 是\emph{(正)定向的}。
\emph{负定向}的概念做类似的定义。
 
如果 $M$ 的每个点都在一个定向的局部标架的定义域中,那么我们说
这个逐点定向是\emph{连续的}。定义\emph{$M$ 的定向}为一个连续的逐点定向。
如果 $M$ 存在一个定向,那么我们说 $M$ 是\emph{可定向的},否则称
$M$ 是\emph{不可定向的}。一个\emph{定向流形}指的是有序对 $(M,\mathcal{O})$,
其中 $M$ 是可定向的光滑流形,$\mathcal{O}$ 是 $M$ 的一个定向。
对于每个 $p\in M$,由 $\mathcal{O}$ 确定的 $T_pM$ 的定向记为 $\mathcal{O}_p$。

如果 $M$ 是零维的,那么定义 $M$ 的定向是在每个点处选取 $\pm 1$ 中的一个数。
这种情况下连续性的概念是无意义的。

\begin{proposition}[由 $n$-形式确定的定向]
  令 $M$ 是一个光滑 $n$-流形,$M$ 上的任意非消失的 $n$-形式 $\omega$ 都确定了唯一一个
  $M$ 上的定向使得 $\omega$ 在每个点处是正定向的。反之,如果 $M$
  有一个给定的定向,那么存在一个光滑的非消失 $n$-形式使得在每个点处都是正定向的。
\end{proposition}
\begin{remark}
  这里的非消失指的是处处非零。此时 $M$ 上的任意非消失 $n$-形式都被称为\emph{定向形式}。如果 $M$ 是定向的,
  $\omega$ 是确定了这个定向的定向形式,那么我们说 $\omega$ 是\emph{(正)定向的}。
  如果 $\omega$ 和 $\tilde\omega$ 是两个正定向形式,那么显然有 
  $\tilde\omega=f\omega$,其中 $f$ 是严格正值的光滑函数。
\end{remark}
\begin{proof}
  令 $\omega$ 是非消失 $n$-形式。那么 $\omega$ 定义了一个逐点定向,
  我们只需要验证这是连续的。在 $n=0$ 时是平凡的。假设 $n\ge 1$,
  给定 $p\in M$,设 $(E_i)$ 是 $p$ 的某个连通邻域 $U$ 上的局部标架,
  $\bigl(\varepsilon^i\bigr)$ 是其对偶余标架。那么 $\omega$
  在这个标架中表示为 $\omega=f\varepsilon^1\wedge\cdots\wedge\varepsilon^n$,
  其中 $f$ 是连续函数。$\omega$ 非消失表示 $f$ 非处处非零,所以
  对于所有 $U$ 中的点有
  \[
    \omega(E_1,\dots,E_n)=f\neq 0.
  \]
  因为 $U$ 连通,所以上述表达式一定恒正或者恒负,因此给定的标架
  在 $U$ 上要么正定向要么负定向。如果是负定向的,我们可以将 $E_1$
  替换为 $-E_1$ 得到一组新的正定向的标架。因此,由 $\omega$ 确定
  的逐点定向是连续的。

  反之,假设 $M$ 是定向的,令 $\Lambda_+^nT^*M\subseteq \Lambda^nT^*M$
  是由所有点处的正定向的 $n$-余向量构成的开子集。在任意点 $p\in M$
  处,$\Lambda_+^nT^*M$ 与纤维 $\Lambda^n(T_p^*M)$ 的交是上半平面,
  因此是凸集。根据通常的单位分解,存在 $\Lambda_+^nT^*M$ 的一个光滑
  全局截面,即一个正定向的光滑 $n$-形式。
\end{proof}

对于一个定向光滑流形上的光滑坐标卡,如果坐标标架 $\bigl(\partial/\partial x^i\bigr)$
是正定向的,那么我们说这个坐标卡是\emph{正定向的},如果
坐标标架是负定向的,则称这个坐标卡是\emph{负定向的}。
如果一个光滑图册 $\{(U_\alpha,\varphi_\alpha)\}$ 的任意转移映射
$\varphi_\beta\circ\varphi_\alpha^{-1}$ 在 $\varphi_\alpha(U_\alpha\cap U_\beta)$
中的每个点处都有正的 Jacobi 行列式,那么我们说这个图册是\emph{一致定向的}。

\begin{proposition}[通过坐标图册确定的定向]
  令 $M$ 是光滑带边或者无边流形。给定 $M$ 的一个一致定向的光滑图册,
  存在 $M$ 的一个定向使得给定图册中的每个坐标卡都是正定向的。反之,
  如果 $M$ 是定向的并且 $\partial M=\emptyset$ 或者 $\dim M>1$,
  那么所有定向的光滑坐标卡的集合是 $M$ 的一个一致定向图册。
\end{proposition}
\begin{proof}
  首先假设 $M$ 有一个一致定向的光滑图册。那么图册中的每个坐标卡
  都在定义域中的每个点处确定了一个定向。图册是一致定向的保证
  在两个坐标卡重叠的区域中,两个坐标标架的转移矩阵是行列式为
  正的 Jacobi 矩阵,所以这两个坐标卡在重叠区域确定了相同的定向。
  由于每个点都被一个坐标卡包含,所以这个逐点定向是连续的。

  反之,假设 $M$ 是定向的且 $\partial M=\emptyset$ 或者 $\dim M>1$。
  根据连续性,每个点 $p\in M$ 都被一个定向的局部标架 $(E_1,\dots,E_n)$
  包含,此时取 $p$ 处的一个坐标标架 $\bigl(\partial/\partial x^1,\dots,\partial/\partial x^n\bigr)$,
  有 $E_j=B^i_j\partial/\partial x^i$,由于 $(B_j^i)$ 是连续函数,
  假设这个局部坐标卡是连通的,所以 $\det(B_j^i)$ 恒正或者恒负,所以
  坐标标架要么正定向要么负定向,如果负定向,则可以将坐标 $x^1$ 替换为 
  $-x^1$ 得到一个正定向的坐标标架。于是所有正定向的光滑坐标卡覆盖 $M$,
  且这些坐标卡的转移映射都有正的 Jacobi 行列式,所以构成一个一致定向图册。
  (当 $\dim M=1$ 时对带边流形不适用,因为我们约定边界坐标卡中的最后一个坐标是非负的)
\end{proof}

\begin{proposition}[积定向]
  假设 $M_1,\dots,M_k$ 是定向的光滑流形。那么 $M_1\times\cdots\times M_k$
  上存在唯一的定向,被称为积定向,使得:如果 $\omega_i$ 是 $M_i$ 上给定定向的定向形式,那么
  $\pi_1^*\omega_1\wedge\cdots\wedge\pi_k^*\omega_k$ 是积定向的定向形式。
\end{proposition}
\begin{proof}
  此时 $M_i$ 的一致定向图册之积构成 $M_1\times\cdots\times M_k$
  的一个一致定向图册,所以给出 $M_1\times\cdots\times M_k$ 上的一个定向。
  % 任取 $(p_1,\dots,p_k)\in M_1\times\cdots\times M_k$,那么
  % $\omega_i|_{p_i}$ 是 $T_{p_i}M_i$ 上的定向 $n_i$-余向量 ($\dim M_i=n_i$),
  % 取 $T_{p_i}M_i$ 的一组定向基 $\bigl(E_1^{(i)},\dots,E_{n_i}^{(i)}\bigr)$,
  % 那么
  % \[
  %   \omega_i|_p\bigl(E_1^{(i)},\dots,E_{n_i}^{(i)}\bigr)>0,
  % \]
  % 显然 $\bigl(E_1^{(1)},\dots,E_{n_1}^{(1)},\dots,E_1^{(k)},\dots,E_{n_k}^{(k)}\bigr)$
  % 是 $T_{(p_1,\dots,p_k)}(M_1\times\cdots\times M_k)$ 的一组定向基,
  % 由于 $\pi_i^*\omega_i$ 的参数存在 
  % 我们有
  % \[
  %   \pi_1^*\omega_1\wedge\cdots\wedge \pi_k^*\omega_k
  %   \bigl(E_1^{(1)},\dots,E_{n_1}^{(1)},\dots,E_1^{(k)},\dots,E_{n_k}^{(k)}\bigr)
  %   =\prod_{i=1}^k
  %   \pi_i^*\omega_i\bigl(E^{(i)}_{1},\dots,E_{n_i}^{(i)}\bigr)>0,
  % \]
\end{proof}

\begin{proposition}
  令 $M$ 是连通的定向光滑流形,那么 $M$ 恰有两个定向。如果 $M$
  的两个定向在某个点处相同,那么它们相等。
\end{proposition}
\begin{proof}
  设 $\mathcal{O}_1$ 和 $\mathcal{O}_2$ 是 $M$ 的两个定向。
  对于任意的 $p\in M$,$\mathcal{O}_1|_p$ 和 $\mathcal{O}_2|_p$
  要么相同要么相反,定义函数 $f:M\to \{\pm 1\}$ 为
  \[
    f(p)=\begin{cases}
      1 & \mathcal{O}_1|_p=\mathcal{O}_2|_p,\\
      -1 & \mathcal{O}_1|_p=-\mathcal{O}_2|_p
    \end{cases}
  \]
\end{proof}


