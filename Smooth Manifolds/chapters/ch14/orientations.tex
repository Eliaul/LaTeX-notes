\chapter{定向}

\section{向量空间的定向}

令 $V$ 是维数 $n\ge 1$ 的向量空间。设
$(E_1,\dots,E_n)$ 和 $\bigl(\wtilde E_1,\dots,\wtilde E_n\bigr)$
是 $V$ 的两组有序基,
如果过渡矩阵 $\bigl(B_j^i\bigr)$ 的行列式为正,那么我们说这两组基
是\emph{一致定向的},其中过渡矩阵定义为
\[
  E_j=B^i_j\wtilde E_i.  
\]
显然一致定向是 $V$ 的所有有序基集合上的一个等价关系。而且这个等价关系恰好
划分了两个等价类,
任取一组有序基 $(E_1,\dots,E_n)$,显然 $(-E_1,E_2,\dots,E_n)$
和它不是一致定向的,所以这两个有序基处于不同的等价类。
对于任意一组有序基 $\bigl(\wtilde E_1,\dots,\wtilde E_n\bigr)$,
如果其不与 $(E_1,\dots,E_n)$ 一致定向,即 $(E_1,\dots,E_n)$
到 $\bigl(\wtilde E_1,\dots,\wtilde E_n\bigr)$ 的过渡矩阵行列式为负,
那么 $(-E_1,E_2,\dots,E_n)$ 到 $\bigl(\wtilde E_1,\dots,\wtilde E_n\bigr)$ 的过渡矩阵
行列式为正,所以 $\bigl(\wtilde E_1,\dots,\wtilde E_n\bigr)$ 和
$(-E_1,E_2,\dots,E_n)$ 一致定向,故这个等价关系划分了两个等价类。

如果 $\dim V=n\ge 1$,定义 \emph{$V$ 的定向} 是有序基的一个等价类。
如果 $(E_1,\dots,E_n)$ 是 $V$ 的任意一组有序基,我们记其确定的定向为
$[E_1,\dots,E_n]$,以及其相反的定向为 $-[E_1,\dots,E_n]$。
一个附带了定向的向量空间被称为\emph{定向向量空间}。如果 $V$
是定向的,那么在给定定向中的任一有序基 $(E_1,\dots,E_n)$ 都被称为是
\emph{定向的}或者\emph{正定向的}。不在给定定向中的任意基都被称为
是\emph{负定向的}。

对于零维向量空间 $V$ 的特殊情况,我们定义 $V$ 的定向在数 $\pm 1$
中选取。

\begin{example}
  $\mathbb{R}^n$ 的由标准基确定的定向 $[e_1,\dots,e_n]$ 被称为\emph{标准定向}。
  $\mathbb{R}^0$ 的标准定向被定义为 $+1$。
\end{example}

利用交错张量对参数顺序敏感的特性,定向和交错张量之间存在密切联系。

\begin{proposition}
  令 $V$ 是 $n$ 维向量空间。每个非零张量 $\omega\in\Lambda^n(V^*)$
  都确定了一个定向 $\mathcal{O}_\omega$ 为:如果 $n\ge 1$,那么 
  $\mathcal{O}_\omega$ 是有序基 $(E_1,\dots,E_n)$ 的集合,
  其中 $\omega(E_1,\dots,E_n)>0$。如果 $n=0$,那么 $\omega>0$
  的时候 $\mathcal{O}_\omega=+1$,$\omega<0$ 的时候 $\mathcal{O}_\omega=-1$。
  两个非零 $n$-余向量确定了同一个定向当且仅当其中一个是另一个的正的倍数。 
\end{proposition}
\begin{proof}
  $0$-维的情况是显然的,因为 $\Lambda^0(V^*)=\mathbb{R}$。
  假设 $n\ge 1$,对于非零 $\omega\in \Lambda^n(V^*)$,
  我们证明 $\mathcal{O}_\omega$ 确实是一个等价类。

  假设 $(E_i)$ 和 $\bigl(\wtilde E_j\bigr)$ 是 $V$ 的两组有序基,
  $B:V\to V$ 是将 $E_j$ 送到 $\wtilde E_j$ 的线性映射。
  也就是说 $\wtilde E_j=BE_j=B^i_jE_i$,那么根据
  \autoref{prop:property of alt},有
  \[
    \omega\bigl(\wtilde E_1,\dots,\wtilde E_n\bigr)  =
    (\det B)\omega(E_1,\dots,E_n).
  \]
  这表明 $\bigl(\wtilde E_j\bigr)$ 和 $(E_i)$ 是一致定向的当且仅当
  $\omega\bigl(\wtilde E_1,\dots,\wtilde E_n\bigr)$ 和 $\omega(E_1,\dots,E_n)$
  的符号相同,这就表明 $\mathcal{O}_\omega$ 确实是一个等价类。
\end{proof}

如果 $V$ 是定向的 $n$-维向量空间,$\omega$ 是上述确定了 $V$ 的定向的
$n$-余向量,那么我们说 $\omega$ 是\emph{(正)定向的 $n$-余向量}。
例如,$n$-余向量 $e^1\wedge\cdots\wedge e^n$ 是关于 $\mathbb{R}^n$
的标准定向的正定向余向量。

注意到 $\Lambda^n(V^*)$ 是 $1$ 维向量空间,所以选取 $V$ 的定向等价于选取
$\Lambda^n(V^*) \smallsetminus\{0\}$ 的分支。

\section{流形的定向}

令 $M$ 是光滑流形。我们定义\emph{$M$ 上的逐点定向}是在每个切空间中的
一个定向的选择。但是这并不是一个非常有用的概念,因为邻近点的定向可能
没有任何关系。为了使得定向和光滑结构产生联系,我们需要额外的条件
确保邻近切空间的定向是彼此一致的。

令 $M$ 是光滑 $n$-流形,配备一个逐点定向。如果 $(E_i)$ 是 $TM$
的一个局部标架,且对于每个 $p\in U$,$(E_1|_p,\dots,E_n|_p)$
都是 $T_pM$ 的正定向的基,那么我们说 $(E_i)$ 是\emph{(正)定向的}。
\emph{负定向}的概念做类似的定义。
 
如果 $M$ 的每个点都在一个定向的局部标架的定义域中,那么我们说
这个逐点定向是\emph{连续的}。定义\emph{$M$ 的定向}为一个连续的逐点定向。
如果 $M$ 存在一个定向,那么我们说 $M$ 是\emph{可定向的},否则称
$M$ 是\emph{不可定向的}。一个\emph{定向流形}指的是有序对 $(M,\mathcal{O})$,
其中 $M$ 是可定向的光滑流形,$\mathcal{O}$ 是 $M$ 的一个定向。
对于每个 $p\in M$,由 $\mathcal{O}$ 确定的 $T_pM$ 的定向记为 $\mathcal{O}_p$。

如果 $M$ 是零维的,那么定义 $M$ 的定向是在每个点处选取 $\pm 1$ 中的一个数。
这种情况下连续性的概念是无意义的。

\begin{proposition}[由 $n$-形式确定的定向]
  令 $M$ 是一个光滑 $n$-流形,$M$ 上的任意非消失的 $n$-形式 $\omega$ 都确定了唯一一个
  $M$ 上的定向使得 $\omega$ 在每个点处是正定向的。反之,如果 $M$
  有一个给定的定向,那么存在一个光滑的非消失 $n$-形式使得在每个点处都是正定向的。
\end{proposition}
\begin{remark}
  这里的非消失指的是处处非零。此时 $M$ 上的任意非消失 $n$-形式都被称为\emph{定向形式}。如果 $M$ 是定向的,
  $\omega$ 是确定了这个定向的定向形式,那么我们说 $\omega$ 是\emph{(正)定向的}。
  如果 $\omega$ 和 $\tilde\omega$ 是两个正定向形式,那么显然有 
  $\tilde\omega=f\omega$,其中 $f$ 是严格正值的光滑函数。
\end{remark}
\begin{proof}
  令 $\omega$ 是非消失 $n$-形式。那么 $\omega$ 定义了一个逐点定向,
  我们只需要验证这是连续的。在 $n=0$ 时是平凡的。假设 $n\ge 1$,
  给定 $p\in M$,设 $(E_i)$ 是 $p$ 的某个连通邻域 $U$ 上的局部标架,
  $\bigl(\varepsilon^i\bigr)$ 是其对偶余标架。那么 $\omega$
  在这个标架中表示为 $\omega=f\varepsilon^1\wedge\cdots\wedge\varepsilon^n$,
  其中 $f$ 是连续函数。$\omega$ 非消失表示 $f$ 非处处非零,所以
  对于所有 $U$ 中的点有
  \[
    \omega(E_1,\dots,E_n)=f\neq 0.
  \]
  因为 $U$ 连通,所以上述表达式一定恒正或者恒负,因此给定的标架
  在 $U$ 上要么正定向要么负定向。如果是负定向的,我们可以将 $E_1$
  替换为 $-E_1$ 得到一组新的正定向的标架。因此,由 $\omega$ 确定
  的逐点定向是连续的。

  反之,假设 $M$ 是定向的,令 $\Lambda_+^nT^*M\subseteq \Lambda^nT^*M$
  是由所有点处的正定向的 $n$-余向量构成的开子集。在任意点 $p\in M$
  处,$\Lambda_+^nT^*M$ 与纤维 $\Lambda^n(T_p^*M)$ 的交是上半平面,
  因此是凸集。根据通常的单位分解,存在 $\Lambda_+^nT^*M$ 的一个光滑
  全局截面,即一个正定向的光滑 $n$-形式。
\end{proof}

对于一个定向光滑流形上的光滑坐标卡,如果坐标标架 $\bigl(\partial/\partial x^i\bigr)$
是正定向的,那么我们说这个坐标卡是\emph{正定向的},如果
坐标标架是负定向的,则称这个坐标卡是\emph{负定向的}。
如果一个光滑图册 $\{(U_\alpha,\varphi_\alpha)\}$ 的任意转移映射
$\varphi_\beta\circ\varphi_\alpha^{-1}$ 在 $\varphi_\alpha(U_\alpha\cap U_\beta)$
中的每个点处都有正的 Jacobi 行列式,那么我们说这个图册是\emph{一致定向的}。

\begin{proposition}[通过坐标图册确定的定向]
  令 $M$ 是光滑带边或者无边流形。给定 $M$ 的一个一致定向的光滑图册,
  存在 $M$ 的一个定向使得给定图册中的每个坐标卡都是正定向的。反之,
  如果 $M$ 是定向的并且 $\partial M=\emptyset$ 或者 $\dim M>1$,
  那么所有定向的光滑坐标卡的集合是 $M$ 的一个一致定向图册。
\end{proposition}
\begin{proof}
  首先假设 $M$ 有一个一致定向的光滑图册。那么图册中的每个坐标卡
  都在定义域中的每个点处确定了一个定向。图册是一致定向的保证
  在两个坐标卡重叠的区域中,两个坐标标架的转移矩阵是行列式为
  正的 Jacobi 矩阵,所以这两个坐标卡在重叠区域确定了相同的定向。
  由于每个点都被一个坐标卡包含,所以这个逐点定向是连续的。

  反之,假设 $M$ 是定向的且 $\partial M=\emptyset$ 或者 $\dim M>1$。
  根据连续性,每个点 $p\in M$ 都被一个定向的局部标架 $(E_1,\dots,E_n)$
  包含,此时取 $p$ 处的一个坐标标架 $\bigl(\partial/\partial x^1,\dots,\partial/\partial x^n\bigr)$,
  有 $E_j=B^i_j\partial/\partial x^i$,由于 $(B_j^i)$ 是连续函数,
  假设这个局部坐标卡是连通的,所以 $\det(B_j^i)$ 恒正或者恒负,所以
  坐标标架要么正定向要么负定向,如果负定向,则可以将坐标 $x^1$ 替换为 
  $-x^1$ 得到一个正定向的坐标标架。于是所有正定向的光滑坐标卡覆盖 $M$,
  且这些坐标卡的转移映射都有正的 Jacobi 行列式,所以构成一个一致定向图册。
  (当 $\dim M=1$ 时对带边流形不适用,因为我们约定边界坐标卡中的最后一个坐标是非负的)
\end{proof}

\begin{proposition}[积定向]
  假设 $M_1,\dots,M_k$ 是定向的光滑流形。那么 $M_1\times\cdots\times M_k$
  上存在唯一的定向,被称为积定向,使得:如果 $\omega_i$ 是 $M_i$ 上给定定向的定向形式,那么
  $\pi_1^*\omega_1\wedge\cdots\wedge\pi_k^*\omega_k$ 是积定向的定向形式。
\end{proposition}
\begin{proof}
  此时 $M_i$ 的一致定向图册之积构成 $M_1\times\cdots\times M_k$
  的一个一致定向图册,所以给出 $M_1\times\cdots\times M_k$ 上的一个定向。
  % 任取 $(p_1,\dots,p_k)\in M_1\times\cdots\times M_k$,那么
  % $\omega_i|_{p_i}$ 是 $T_{p_i}M_i$ 上的定向 $n_i$-余向量 ($\dim M_i=n_i$),
  % 取 $T_{p_i}M_i$ 的一组定向基 $\bigl(E_1^{(i)},\dots,E_{n_i}^{(i)}\bigr)$,
  % 那么
  % \[
  %   \omega_i|_p\bigl(E_1^{(i)},\dots,E_{n_i}^{(i)}\bigr)>0,
  % \]
  % 显然 $\bigl(E_1^{(1)},\dots,E_{n_1}^{(1)},\dots,E_1^{(k)},\dots,E_{n_k}^{(k)}\bigr)$
  % 是 $T_{(p_1,\dots,p_k)}(M_1\times\cdots\times M_k)$ 的一组定向基,
  % 由于 $\pi_i^*\omega_i$ 的参数存在 
  % 我们有
  % \[
  %   \pi_1^*\omega_1\wedge\cdots\wedge \pi_k^*\omega_k
  %   \bigl(E_1^{(1)},\dots,E_{n_1}^{(1)},\dots,E_1^{(k)},\dots,E_{n_k}^{(k)}\bigr)
  %   =\prod_{i=1}^k
  %   \pi_i^*\omega_i\bigl(E^{(i)}_{1},\dots,E_{n_i}^{(i)}\bigr)>0,
  % \]
\end{proof}

\begin{proposition}
  令 $M$ 是连通的定向光滑流形,那么 $M$ 恰有两个定向。如果 $M$
  的两个定向在某个点处相同,那么它们相等。
\end{proposition}
\begin{proof}
  设 $\mathcal{O}$ 和 $\wtilde{\mathcal{O}}$ 是 $M$ 的两个定向。
  对于任意的 $p\in M$,$\mathcal{O}_p$ 和 $\wtilde{\mathcal{O}}_p$
  要么相同要么相反,定义函数 $f:M\to \{\pm 1\}$ 为
  \[
    f(p)=\begin{cases}
      1 & \mathcal{O}_p=\wtilde{\mathcal{O}}_p,\\
      -1 & \mathcal{O}_p=-\wtilde{\mathcal{O}}_p,
    \end{cases}
  \]
  根据连续性,存在 $p$ 处的连通邻域 $U$ 上的定向局部标架
  $\bigl(E_1,\dots,E_n\bigr)$ 和 $\bigl(\wtilde E_1,\dots,\wtilde E_n\bigr)$,
  分别对应定向 $\mathcal{O}$ 和 $\wtilde{\mathcal{O}}$。
  那么 $E_j=A_j^i\wtilde E_i$,其中 $(A_j^i):U\to \GL(n,\mathbb{R})$
  是连续函数,所以行列式 $\det(A_j^i):U\to \mathbb{R}^\times$ 是连续函数,
  $U$ 连通表明 $\det(A^i_j)$ 恒正或者恒负,这就表明在 
  $U$ 上 $\mathcal{O}=\pm\wtilde{\mathcal{O}}$,所以 $f$ 是局部常值函数。
  $M$ 连通表明 $f$ 是常值函数,这就说明 $M$ 恰有两个定向
  且由其在任意点处的定向决定。
\end{proof}

\begin{proposition}[余维数 $0$ 的子流形的定向]
  设 $M$ 是一个定向的带边或者无边光滑流形,$D\subseteq M$
  是余维数为 $0$ 的带边或者无边光滑子流形。那么 $M$ 的定向限制到
  $D$ 上是 $D$ 的一个定向。如果 $\omega$ 是 $M$ 上的定向形式,那么
  $\iota_D^*\omega$ 是 $D$ 上的定向形式。
\end{proposition}
\begin{proof}
  设 $\mathcal{O}$ 是 $M$ 上的定向,$\mathcal{O}|_D$ 是 $\mathcal{O}$
  在 $D$ 上的限制。由于 $D$ 是余维数为 $0$ 的子流形,所以是嵌入子流形。
  任取 $p\in D$,$T_pD$ 可以等同为 $T_pM$,所以 $\mathcal{O}|_D$
  可以视为 $D$ 上的定向。
\end{proof}

令 $M,N$ 是一个定向的带边或者无边光滑流形,$F:M\to N$
是局部微分同胚。如果对于每个 $p\in M$,同构 $\d F_p$
将 $T_pM$ 的定向基送到 $T_{F(p)}N$ 的定向基,那么我们说 $F$
是 \emph{保定向的}。如果 $\d F_p$ 将 $T_pM$ 的定向基送到 $T_{F(p)}N$ 的负定向基,
那么我们说 $F$ 是 \emph{反定向的}。

\begin{proposition}[拉回定向]
  设 $M,N$ 是带边或者无边光滑流形。如果 $F:M\to N$ 是局部微分同胚且
  $N$ 是定向的,那么 $M$ 有唯一的定向使得 $F$ 是保定向的映射,
  这个定向被称为\emph{拉回定向}。
\end{proposition}
\begin{proof}
  对于每个 $p\in M$,存在唯一的 $T_pM$ 上的定向使得同构 $\d F_p:T_pM\to T_{F(p)}N$
  是保定向的,这定义了 $M$ 上的一个逐点定向。选取 $N$ 的一个定向形式 $\omega$,
  那么 $F^*\omega$ 是 $M$ 上的确定了这个逐点定向的微分形式,所以这个逐点定向是连续的。
\end{proof}

在这种情况下,如果 $\mathcal{O}$ 是 $N$ 上的定向,那么 $M$ 上的拉回定向
记为 $F^*\mathcal{O}$。

回顾一个有光滑全局标架的光滑流形被称为\emph{可平行化的}。

\begin{proposition}
  每个可平行化的光滑流形都是可定向的。
\end{proposition}
\begin{proof}
  设 $M$ 是光滑流形,$(E_1,\dots,E_n)$ 是 $M$ 的一个全局标架,
  那么在每个点 $p\in M$ 处都确定了一个定向基 $\bigl(E_1|_p,\dots,E_n|_p\bigr)$,
  这给出了 $M$ 的一个逐点定向。由于每个点都被这个全局标架包含,所以
  这个逐点定向自然是连续的。
\end{proof}

特别地,这表明李群都是可定向的,因为李群有一个左不变向量场构成的全局标架。

在李群的情况下,如果 $G$ 是李群,对于每个 $g\in G$,$L_g$
都是保定向的,那么我们说这个定向是\emph{左不变的}。

\begin{proposition}
  每个李群恰有两个左不变定向,对应于其李代数的两个定向。
\end{proposition}
\begin{proof}
  设 $[v_1,\dots,v_n]$ 和 $-[v_1,\dots,v_n]$ 是 $T_eG$ 的两个定向,
  那么 $v_1^\LL,\dots,v_n^\LL$ 是 $G$ 的一个光滑全局标架且确定了
  $G$ 上的一个定向 $\mathcal{O}$。我们说明这个定向是左不变的。任取 $g\in G$,
  设 $\omega$ 是 $\mathcal{O}$ 的定向形式,那么对于任意
  $g'\in G$,由于 $v_i^\LL|_{gg'}=\d (L_{gg'})_e(v_i)$,所以
  \[
    \omega_{gg'}\bigl(\d (L_{gg'})_e(v_1),\dots,\d (L_{gg'})_e(v_n)\bigr)>0,
  \]
  即
  \[
    (L_g^*\omega)_{g'}\bigl(\d (L_{g'})_e(v_1),\dots,\d (L_{g'})_e(v_n)\bigr)>0.
  \]
  任取定向基 $w_1,\dots,w_n\in T_{g'}G$,所以 $(w_1,\dots,w_n)$ 和 $\bigl(\d (L_{g'})_e(v_1),\dots,\d (L_{g'})_e(v_n)\bigr)$
  是一致定向的,即有正行列式的过渡矩阵 $(B^i_j)$,所以
  \[
    (L_g^*\omega)_{g'}(w_1,\dots,w_n)=\det(B^i_j)\cdot (L_g^*\omega)_{g'}\bigl(\d (L_{g'})_e(v_1),\dots,\d (L_{g'})_e(v_n)\bigr)>0.
  \]
  这就表明 $L_g^*\omega$ 也是 $\mathcal{O}$ 的定向形式,即
  $L_g^* \mathcal{O}=\mathcal{O}$,$\mathcal{O}$ 是左不变定向。
  类似的,$-[v_1,\dots,v_n]$ 确定了另一个左不变定向。
\end{proof}

\subsection{超曲面的定向}

如果 $M$ 是定向光滑流形,$S$ 是 $M$ 的一个光滑子流形(带边或者无边),
那么 $S$ 也可能没有从 $M$ 继承的定向,即使 $S$ 是嵌入的。显然,
简单的将 $M$ 的定向形式限制到 $S$ 上是不行的,因为 $n$-形式限制到
低维上必然变成零。一个例子是 M\"obius 带,即使其可以嵌入到 $\mathbb{R}^3$
中,其也不可定向。

\section{黎曼体积形式}

令 $(M,g)$ 是定向的黎曼流形。我们知道在 $M$ 的每个点的邻域中都存在
一个光滑的正交标架 $(E_1,\dots,E_n)$。通过将 $E_1$ 替换为 $-E_1$,
我们可以在每个点的邻域中找到一个定向的正交标架。

\begin{proposition}
  设 $(M,g)$ 是一个定向的黎曼 $n$-流形,且 $n\geq 1$。那么存在唯一的
  光滑定向形式 $\omega_g\in\Omega^n(M)$,满足对于每个定向的局部正交标架
  $(E_i)$ 都有
  \begin{equation}\label{eq:riemann volumn form}
    \omega_g(E_1,\dots,E_n)=1,
  \end{equation}
  这个定向形式被称为\emph{黎曼体积形式}。
\end{proposition}
\begin{proof}
  首先说明 $\omega_g$ 的唯一性。如果 $(E_1,\dots,E_n)$ 是开集 $U\subseteq M$ 上的任意定向的
  局部正交标架,$(\varepsilon^1,\dots,\varepsilon^n)$ 是其对偶余标架。
  那么我们有局部表示 $\omega_g=f\varepsilon^1\wedge\cdots\wedge \varepsilon^n$,
  \eqref{eq:riemann volumn form} 表明 $f=1$,所以
  \begin{equation}\label{eq:riemann volumn form2}
    \omega_g=\varepsilon^1\wedge\cdots\wedge\varepsilon^n.
  \end{equation}
  这表明 $\omega_g$ 是唯一确定的。

  下面说明存在性,我们希望通过 \eqref{eq:riemann volumn form2} 式来在每个点的
  邻域上定义 $\omega_g$,所以我们说明这个定义和定向正交标架的选取无关。
  如果 $\bigl(\wtilde E_1,\dots,\wtilde E_n\bigr)$ 是另一组定向正交标架,
  有对偶余标架 $\bigl(\tilde\varepsilon^1,\dots,\tilde\varepsilon^n\bigr)$,
  令
  \[
    \wtilde\omega_g=\tilde\varepsilon^1\wedge\cdots\wedge \tilde\varepsilon^n.
  \]
  这两组正交标架之间有关系 $\wtilde E_j=A_j^iE_i$,正交性表明系数矩阵
  $(A_j^i)$ 是正交矩阵,所以
  \[
    \omega_g\bigl(\wtilde E_1,\dots,\wtilde E_n\bigr)
    =\det\bigl(\varepsilon^i(\wtilde E_j)\bigr)=
    \det(A_j^i)=1=\wtilde\omega_g\bigl(
      \wtilde E_1,\dots,\wtilde E_n
    \bigr).
  \]
  这表明 $\omega_g=\wtilde\omega_g$,所以 \eqref{eq:riemann volumn form2} 式
  是良好定义的,这给出了一个全局 $n$-形式且满足要求。
\end{proof}

\begin{proposition}
  设 $(M,g)$ 是一个定向的黎曼 $n$-流形,且 $n\geq 1$。在任意定向光滑坐标 
  $\bigl(x^i\bigr)$ 中,黎曼体积形式有局部表示
  \[
    \omega_g=\sqrt{\det(g_{ij})}\d x^1\wedge\cdots\wedge \d x^n.
  \]
\end{proposition}
\begin{proof}
  设 $\bigl(U,\bigl(x^i\bigr)\bigr)$ 是一个定向的光滑坐标卡,
  令 $p\in M$。此时 $\omega_g$ 有局部表示 $\omega_g=f\d x^1\wedge\cdots\wedge \d x^n$,
  其中 $f$ 是正值函数。令 $(E_i)$ 是定义在 $p$ 的某个邻域上的
  定向正交标架,$\bigl(\varepsilon^i\bigr)$ 是其对偶余标架。
  此时我们可以把坐标标架表示为
  \[
    \frac{\partial}{\partial x^j}=A^i_jE_i ,
  \]
  所以我们可以计算
  \begin{equation*}
    f=\omega_g\left( \frac{\partial}{\partial x^1},\dots, \frac{\partial}{\partial x^n}\right)
    =\det(A^i_j)\ \omega_g(E_1,\dots,E_n)=\det(A_j^i).
  \end{equation*}
  另一方面,注意到
  \[
    g_{ij}=\inn{\frac{\partial}{\partial x^i},\frac{\partial}{\partial x^j}}_g =
    \inn{A^k_iE_k,A^l_jE_l}_g=A_i^kA_j^l\inn{E_k,E_l}_g
    =\sum_{k=1}^n A_i^kA_j^k.
  \]
  于是 $g_{ij}$ 是 $A^TA$ 的第 $(i,j)$-元,所以
  \[
    \det(g_{ij})=\det(A^TA)=(\det A)^2,
  \]
  故 $f=\det A=\sqrt{\det(g_{ij})}$。
\end{proof}






