
\chapter{子流形}

\section{嵌入子流形}

设 $M$ 是带边或者无边光滑流形,$M$ 的\emph{嵌入子流形}
指的是子集 $S\subseteq M$,其配备子空间拓扑使得 $S$ 成为一个(无边)流形,
配备使得包含映射 $S\hookrightarrow M$
为光滑嵌入的光滑结构。嵌入子流形也被称为\emph{正则子流形}。

如果 $S$ 是 $M$ 的嵌入子流形,那么 $\dim M-\dim S$ 被称为 $S$ 在 $M$
中的\emph{余维数}。余维数为 $1$ 的嵌入子流形被称为一个\emph{嵌入超曲面}。

\begin{proposition}[开子流形]\label{prop:open submanifold}
  设 $M$ 是光滑流形,那么 $M$ 的余维数为 $0$ 的嵌入子流形恰好为开子流形。
\end{proposition}
\begin{proof}
  设 $U\subseteq M$ 是开子流形,$\iota:U\hookrightarrow M$ 为包含映射。
  此时 $\iota$ 的坐标表示为恒等映射,所以 $\iota$ 为光滑浸入,又因为
  $U$ 有子空间拓扑,所以 $\iota$ 是光滑嵌入。故 $U$ 为余维数
  $\dim M-\dim U=0$ 的嵌入子流形。反之,设 $U$ 是余维数为 $0$ 的嵌入子流形。
  根据定义,包含映射 $\iota:U\hookrightarrow M$ 是光滑嵌入,根据
  \autoref{prop:local diffeomorphism},$\iota$ 是局部微分同胚,从而是开映射,
  所以 $U$ 是 $M$ 的开子集,故 $U$ 是开子流形。
\end{proof}

\begin{proposition}[嵌入的像作为子流形]\label{prop:image of embedding}
  设 $M$ 是带边或者无边光滑流形,$N$ 是光滑流形,$F:N\to M$ 是光滑嵌入。
  令 $S=F(N)$,在子空间拓扑下,$S$ 是拓扑流形,并且有唯一的光滑结构使得
  其为 $M$ 的嵌入子流形以及 $F$ 为 $N\to F(N)$ 的微分同胚。 
\end{proposition}
\begin{proof}
  $F$ 是嵌入表明 $F$ 可以视为 $N\to S$ 的同胚,所以 $S$ 是一个拓扑流形。
  定义 $S$ 上的光滑坐标卡形如 $\left(F(U),\varphi\circ F^{-1}\right)$,
  其中 $(U,\varphi)$ 为 $N$ 的任意光滑坐标卡。$N$ 的光滑坐标卡的相容性
  导出了 $S$ 的光滑坐标卡的相容性,所以这给出了 $S$ 上的光滑结构。
  在这个光滑结构下,$F:N\to S$ 显然是微分同胚。包含映射 $S\hookrightarrow M$
  是光滑嵌入的复合:
  \[
    S\xlongrightarrow{F^{-1}}N\xlongrightarrow{F}M,  
  \]
  所以包含映射 $S\hookrightarrow M$ 是光滑嵌入,即 $S$ 是 $M$ 的嵌入子流形。
\end{proof}

\begin{proposition}[积流形的切片]
  令 $M,N$ 是光滑流形。对于每个 $p\in N$,子集 $M\times\{p\}$ (被称为积流形的一个\emph{切片})
  是 $M\times N$ 的一个微分同胚于 $M$ 的嵌入子流形。
\end{proposition}
\begin{proof}
  考虑光滑嵌入 $x\mapsto (x,p)$,那么 $M\times\{p\}$ 作为该光滑嵌入的像集
  是 $M\times N$ 的同胚于 $M$ 的嵌入子流形。
\end{proof}

\begin{proposition}[图像作为子流形]\label{prop:graph is submanifold}
  设 $M$ 是光滑(无边) $m$-流形,$N$ 是光滑带边或者无边 $n$-流形,$U\subseteq M$
  是开集,$f:U\to N$ 是光滑映射。令 $\Gamma(f)\subseteq M\times N$ 为 $f$
  的图像:
  \[
    \Gamma(f)=\left\{(x,y)\in M\times N\,\middle|\, x\in U,y=f(x)\right\}  .
  \]
  那么 $\Gamma(f)$ 是 $M\times N$ 的嵌入 $m$-维子流形。
\end{proposition}
\begin{proof}
  定义映射 $\gamma_f:U\to M\times N$ 为
  \[
    \gamma_f(x)=(x,f(x)),
  \]
  那么 $\gamma_f$ 是像集为 $\Gamma(f)$ 的光滑映射。投影映射 $\pi_M:M\times N\to M$
  满足 $\pi_M\circ\gamma_f=\Id_U$,所以对于任意的 $x\in U$,有 
  $d(\pi_M)_{(x,f(x))}\circ d(\gamma_f)_x=\Id_{T_xM}$,所以 $d(\gamma_f)_x$
  是单射,故 $\gamma_f$ 是光滑浸入。$\gamma_f$ 也是拓扑嵌入,因为
  $\pi_M|_{\Gamma(f)}$ 为 $\gamma_f$ 的逆映射。所以 $\Gamma(f)$
  是同胚于 $U$ 的嵌入子流形。
\end{proof}

有时候仅仅是嵌入还不够,我们定义更强的嵌入。如果嵌入子流形 $S\subseteq M$ 使得
包含映射 $S\hookrightarrow M$ 是恰当映射,那么我们说 $S$ 是\emph{恰当嵌入的}。

\begin{proposition}
  假设 $M$ 是带边或者无边光滑流形,$S\subseteq M$ 是嵌入子流形,那么 $S$ 是恰当嵌入
  当且仅当其是 $M$ 的闭子集。
\end{proposition} 
\begin{proof}
  如果 $S$ 是恰当嵌入的,利用结论:到局部紧的 Hausdorff 空间的恰当映射是闭映射,所以 $S$
  是闭子集。如果 $S$ 是闭子集,任取 $M$ 的紧子集 $K$,此时 $K\cap S$ 是 $S$ 的闭子集,
  从而是 $M$ 的闭子集,所以 $K\cap S$ 是紧子集 $K$ 的闭子集,所以 $K\cap S$ 是紧子集,所以
  $S\hookrightarrow M$ 是恰当映射。
\end{proof}

\begin{corollary}
  任意紧的嵌入子流形是恰当嵌入的。
\end{corollary}


\subsection{嵌入子流形的切片坐标卡}

下面的定理将表明嵌入子流形可以局部地建模在 $\mathbb{R}^k$ 到 $\mathbb{R}^n$
的标准嵌入上,我们把 $\mathbb{R}^k$ 视为子空间
\[
  \left\{\left(x^1,\dots,x^k,x^{k+1},\dots,x^n\right)\,\middle|\,x^{k+1}=\cdots=x^n=0\right\}  
  \subseteq\mathbb{R}^n.
\]
更一般地,如果 $U$ 是 $\mathbb{R}^n$ 的开子集,定义\emph{$U$ 的 $k$-维切片}
(简称 $k$-切片)是
形如
\[
  S=\left\{\left(x^1,\dots,x^k,x^{k+1},\dots,x^n\right)\in U\,\middle|\,
  x^{k+1}=c^{k+1},\dots,x^n=c^n\right\}  
\]
的任意子集,其中 $c^{k+1},\dots,c^n$ 是常数。显然,每个 $k$-切片
都同胚于 $\mathbb{R}^k$ 的一个开子集。

令 $M$ 是光滑 $n$-流形,$(U,\varphi)$ 是一个光滑坐标卡。如果 $S$
是 $U$ 的子集并且使得 $\varphi(S)$ 是 $\varphi(U)$ 的 $k$-切片,那么我们说
$S$ 是\emph{$U$ 的 $k$-切片}。(虽然一般来说我们允许切片由任意常数
$c^{k+1},\dots,c^n$ 定义,但是有时候将这些常数定义为零是更有用的,这只需要
将每个坐标函数减去一个常数即可做到。)给定一个子集 $S\subseteq M$,
如果 $S$ 的每个点都被包含在 $M$ 的某个光滑坐标卡 $(U,\varphi)$ 中且使得
$S\cap U$ 是 $U$ 的 $k$-切片,那么我们说 $S$ 满足\emph{局部 $k$-切片条件}。
任意这样的坐标卡被称为\emph{$S$ 在 $M$ 中的切片坐标卡},对应的坐标 
$\left(x^1,\dots,x^n\right)$ 被称为\emph{切片坐标}。

\begin{theorem}[嵌入子流形的局部切片判别法]
  令 $M$ 是光滑 $n$-流形,如果 $S\subseteq M$ 是嵌入 $k$-维子流形,那么
  $S$ 满足局部 $k$-切片条件。反之,如果 $S\subseteq M$ 满足局部 $k$-切片条件,
  那么在子空间拓扑下,$S$ 是 $k$-维拓扑流形,并且有一个光滑结构使其成为
  $M$ 的 $k$-维嵌入子流形。
\end{theorem}
\begin{proof}
  设 $S\subseteq M$ 是嵌入 $k$-维子流形,那么包含映射 $\iota:S\hookrightarrow M$
  是光滑嵌入,从而是光滑浸入。任取 $p\in S$,根据秩定理,存在 $S$
  的 $p$ 为中心的光滑坐标卡 $(U,\varphi)$ 和 $M$ 的 $p$ 为中心的光滑坐标卡
  $(V,\psi)$ 使得 $U=\iota(U)\subseteq V$ 并且 $\iota$ 的坐标表示
  $\psi\circ\iota|_U\circ\varphi^{-1}:\varphi(U)\to \psi(V)$ 为
  \[
    \left(x^1,\dots,x^k\right)  \to\left(x^1,\dots,x^k,0,\dots,0\right).
  \]
  由于 $U$ 是 $S$ 的开集,所以存在 $M$ 的开集 $W$ 使得 $U=W\cap S$,
  令 $V_0=W\cap V$,那么 $S\cap V_0=U\cap V=U$ 是 $V_0$
  的 $k$-切片,所以 $(V_0,\psi|_{V_0})$ 是 $S$ 在 $M$
  中的切片坐标卡。故 $S$ 满足局部 $k$-切片条件。

  反之,设 $S$ 满足局部 $k$-切片条件。在子空间拓扑下,$S$ 是 Hausdorff 的
  和第二可数的,下面我们构造 $S$ 的坐标卡。

  记 $\pi:\mathbb{R}^n\to\mathbb{R}^k$ 是前 $k$ 个坐标的投影。
  设 $(U,\varphi)$ 是 $S$ 在 $M$ 中的切片坐标卡,定义
  $V=U\cap S$,$\psi=\pi\circ\varphi|_V$。由于 $\varphi(V)$
  是 $\varphi(U)$ 的 $k$-切片,即 $\varphi(V)=\varphi(U)\cap A$,其中
  \[
    A=\left\{\left(x^1,\dots,x^k,x^{k+1},\dots,x^n\right)\in\mathbb{R}^n\,\middle|\,
      x^{k+1}=c^{k+1},\dots,x^n=c^n
    \right\},
  \]
  显然 $A$ 微分同胚于 $\mathbb{R}^k$。由于 $\varphi(U)$ 是 $\mathbb{R}^n$
  的开集,所以 $\varphi(V)=\varphi(U)\cap A$ 是 $A$ 的开集,所以
  $\psi(V)=\pi\circ\varphi(V)$ 是 $\pi(A)=\mathbb{R}^k$ 的开集。此外,
  $\psi:V\to\psi(V)$ 是同胚,因为其有连续逆映射 $\varphi^{-1}\circ j|_{\psi(V)}$,
  其中 $j:\mathbb{R}^k\to\mathbb{R}^n$ 为 
  \[
    j\left(x^1,\dots,x^k\right)=\left(x^1,\dots,x^k,c^{k+1},\dots,c^n\right).
  \]
  这表明 $S$ 是一个拓扑 $k$-流形,并且包含映射 $\iota:S\hookrightarrow M$ 是拓扑嵌入。

  下面我们验证上述坐标卡之间的相容的。设 $(U,\varphi)$ 和 $(U',\varphi')$ 是
  $S$ 在 $M$ 中的两个切片坐标卡,令 $(V,\psi)$ 和 $(V',\psi')$ 为对应的
  $S$ 的坐标卡。转移映射为 $\psi'\circ\psi^{-1}=\pi\circ\varphi'\circ\varphi^{-1}\circ j$,
  这是光滑映射的复合,所以 $\psi'\circ\psi^{-1}$ 是光滑的。这定义了 $S$
  上的光滑结构。选取 $S$ 在 $M$ 中的切片坐标卡 $(U,\varphi)$ 和对应的 $S$
  的坐标卡 $(V,\psi)$,包含映射 $\iota:S\hookrightarrow M$ 的坐标表示为
  \[
    \left(x^1,\dots,x^k\right)\mapsto\left(
      x^1,\dots,x^k,c^{k+1},\dots,c^n
    \right),
  \]
  这是一个光滑浸入。所以 $\iota$ 是光滑嵌入,$S$ 是 $M$ 的嵌入子流形。
\end{proof}

\begin{example}[球面作为子流形]
  $\mathbb{S}^n$ 是 $\mathbb{R}^{n+1}$ 的嵌入子流形。
  令 $U_i^+=\{x\in\mathbb{R}^{n+1}\,|\, x^i>0\}$,我们知道 $\mathbb{S}^n\cap U_i^+$
  是光滑函数
  \[
    x^i=f\left(x^1,\dots,x^{i-1},x^{i+1},\dots,x^{n+1}\right)  
  \]
  的图像,其中 $f:\mathbb{B}^n\to\mathbb{R}$ 为 $f(u)=\sqrt{1-|u|^2}$。
  类似地,令 $U_i^-=\{x\in\mathbb{R}^{n+1}\,|\, x^i<0\}$,$\mathbb{S}^n\cap U_i^-$
  是 $-f$ 的图像。根据 \autoref{prop:graph is submanifold},每个
  $\mathbb{S}^n\cap U_i^\pm$ 都是 $\mathbb{R}^{n+1}$ 的 $n$ 维嵌入子流形,所以
  满足局部 $n$-切片条件,所以 $\mathbb{S}^n$ 满足局部 $n$-切片条件,
  因此 $\mathbb{S}^n$ 是嵌入子流形。

  我们也可以直接写出 $\mathbb{S}^n$ 在 $\mathbb{R}^{n+1}$ 中的切片坐标卡。
  记
  \[
    V_i^+=\left\{\left(x^1,\dots,x^{n+1}\right)\,\middle|\,
    \left(x^1,\dots,x^{i-1},x^{i+1},\dots,x^{n+1}\right)\in\mathbb{B}^n,x^i>0\right\} , 
  \]
  类似地,记 $V_i^-$ 为上述集合,其中 $x^i<0$。定义
  $\varphi_i^\pm:V_i^\pm\to\mathbb{R}^{n+1}$ 为
  \[
    \left(x^1,\dots,x^{n+1}\right)\mapsto\left(
      u,x^i\mp f(u)
    \right)  ,
  \]
  其中 $u=\left(x^1,\dots,x^{i-1},x^{i+1},\dots,x^{n+1}\right)$。不难验证
  $\varphi_i^\pm\left(V_i^\pm\right)=\mathbb{B}^{n}\times\mathbb{R}$。
  此时 $\left(V_i^\pm,\varphi_i^\pm\right)$ 就是 $\mathbb{S}^n$ 在 $\mathbb{R}^{n+1}$
  中的切片坐标卡,因为
  \[
    \varphi_i^\pm\left(V_i^\pm\cap\mathbb{S}^n\right)=\mathbb{B}^n
  \]
  是 $\varphi_i^\pm\left(V_i^\pm\right)$ 的 $n$-切片。
\end{example}

如果 $M$ 是带边的光滑流形并且 $S\subseteq M$ 是嵌入子流形,那么 $S$ 可能通过非常复杂的方式与 $\partial M$
相交,所以我们不会证明一般情况下 $S$ 在 $M$ 中的切片坐标卡。然而,在子流形本身就是 $M$ 的边界的情况下,
$M$ 的边界坐标卡就扮演了 $\partial M$ 在 $M$ 中的切片坐标卡的角色。

\begin{theorem}
  如果 $M$ 是带边的光滑 $n$-流形,那么在子空间拓扑下,$\partial M$ 是(无边的)拓扑 $(n-1)$-流形,
  并且有一个光滑结构使得其是 $M$ 的恰当嵌入子流形。
\end{theorem}
\begin{proof}
  $\partial M$ 是拓扑 $(n-1)$-流形是第一章的结论。任取 $p\in \partial M$,那么存在包含 $p$ 的边界
  坐标卡 $(U,\varphi)$,满足 $\varphi(U)$ 是 $\mathbb{H}^n$ 的开集并且 $\varphi(p)\in\partial \mathbb{H}^n$。
  此时 $U\cap\partial M$ 是 $\partial M$ 的开集,并且 $\varphi(U\cap\partial M)=\varphi(U)\cap \partial \mathbb{H}^n$ 
  是 $\varphi(U)$ 的 $(n-1)$-切片,所以 $(U,\varphi)$ 是 $\partial M$ 在 $M$ 中的切片坐标卡,
  所以 $\partial M$ 是 $M$ 的嵌入子流形。又因为 $\partial M$ 是 $M$ 的闭子集,
  所以 $\partial M$ 是恰当嵌入。
\end{proof}


\subsection{水平集}

如果 $\varPhi:M\to N$ 是任意映射,$c\in N$,我们说 $\varPhi^{-1}(c)$
是\emph{$\varPhi$ 的水平集}。在 $N=\mathbb{R}^k$ 和 $c=0$ 的时候,
水平集 $\varPhi^{-1}(0)$ 通常被称为\emph{零点集}。

\begin{theorem}[常秩水平集定理]
  令 $M,N$ 是光滑流形,$\varPhi:M\to N$ 是光滑映射,并且有常秩 $r$。
  $\varPhi$ 的每个水平集都是 $M$ 的余维数为 $r$ 的嵌入子流形。
\end{theorem}
\begin{proof}
  记 $m=\dim M$,$n=\dim N$。任取 $p\in \varPhi^{-1}(c)$,根据秩定理,
  存在以 $p$ 为中心的 $M$ 的光滑坐标卡 $(U,\varphi)$ 和以 $\varPhi(p)$
  为中心的 $N$ 的光滑坐标卡 $(V,\psi)$,使得 $\varPhi(U)\subseteq V$
  并且 $\varPhi$ 的坐标表示为
  \[
    \left(x^1,\dots,x^m\right)\mapsto\left(x^1,\dots,x^r,0,\dots,0\right),
  \]
  那么 
  \[ 
    \varphi\left(\varPhi^{-1}(c)\cap U\right)=
    \varphi\left(\varPhi^{-1}(c)\right)\cap\varphi(U)=
    \varphi\left(\varPhi^{-1}\left(\psi^{-1}(0)\right)\right)\cap\varphi(U)
  \]
  是 $\varphi(U)$ 的 $(m-r)$-切片
  \[
    \left\{\left(x^1,\dots,x^m\right)\in\varphi(U)\,\middle|\, x^1=\cdots=x^r=0\right\}  .
  \]
  这表明 $\varPhi^{-1}(c)$ 满足局部 $(m-r)$-切片条件,所以是 $M$ 的 $(m-r)$-维
  嵌入子流形。
\end{proof}

\begin{corollary}[浸没水平集定理]
  如果 $M,N$ 是光滑流形,$\varPhi:M\to N$ 是光滑浸没,那么 $\varPhi$
  的水平集是 $M$ 的余维数为 $\dim N$ 的嵌入子流形。
\end{corollary}

如果 $\varPhi:M\to N$ 是光滑映射,点 $p\in M$ 使得 $d\varPhi_p:T_pM\to T_{\varPhi(p)}N$
是满射,那么 $p$ 被称为\emph{$\varPhi$ 的正则点},否则被称为\emph{$\varPhi$ 的临界点}。
我们有以下两个简单的观察:如果 $\dim M<\dim N$,那么 $M$ 的每个点都是临界点;
$M$ 的每个点都是正则点当且仅当 $\varPhi$ 是浸没。
根据 \autoref{prop:regular points is open},我们知道 $\varPhi$ 的正则点的集合
是 $M$ 的开集。一个点 $c\in N$ 被称为\emph{$\varPhi$ 的正则值},如果水平集
$\varPhi^{-1}(c)$ 中的每个点都是正则点,否则被称为\emph{$\varPhi$ 的临界值}。
特别地,如果 $\varPhi^{-1}(c)=\emptyset$,那么 $c$ 是正则值。
如果 $c$ 是正则值,那么水平集 $\varPhi^{-1}(c)$ 被称为\emph{正则水平集}。

\begin{corollary}[正则水平集定理]
  令 $M,N$ 是光滑流形,$\varPhi:M\to N$ 是光滑映射,$c\in N$ 是正则值,那么
  正则水平集 $\varPhi^{-1}(c)$ 是 $M$ 的余维数为 $\dim N$ 的嵌入子流形。
\end{corollary}
\begin{proof}
  令 $U$ 为 $\varPhi$ 的所有正则值的集合,根据 \autoref{prop:regular points is open},
  $U$ 是 $M$ 的开子集。显然 $\varPhi^{-1}(c)\subseteq U$。此时
  $\varPhi|_U:U\to N$ 是光滑浸没,根据浸没水平集定理,$\varPhi^{-1}(c)$
  是 $U$ 的嵌入子流形。因为光滑嵌入的复合 $\varPhi^{-1}(c)\hookrightarrow U\hookrightarrow M$
  是光滑嵌入,所以 $\varPhi^{-1}(c)$ 是 $M$ 的嵌入子流形。
\end{proof}

\begin{example}[球面]\label{exa:sphere}
  现在我们可以更简单地证明 $\mathbb{S}^n$ 是 $\mathbb{R}^{n+1}$ 的嵌入子流形。
  定义光滑函数 $f:\mathbb{R}^{n+1}\to\mathbb{R}$ 为 $f(x)=\abs{x}^2$,因为
  $df_x(v)=2\sum_i x^iv^i$,所以 $x\neq 0$ 时 $df_x$ 都是满射,所以
  $\mathbb{S}^n=f^{-1}(1)$ 是 $f$ 的正则水平集,故 $\mathbb{S}^n$
  是 $\mathbb{R}^{n+1}$ 的余维数为 $1$ 的嵌入子流形。
\end{example}

\begin{proposition}
  令 $S$ 为光滑 $m$-流形 $M$ 的子集,那么 $S$ 是 $M$ 的嵌入 $k$-子流形当且仅当
  $S$ 的每个点都有 $M$ 中的邻域 $U$ 使得 $U\cap S$ 是某个光滑浸没 $\varPhi:U\to \mathbb{R}^{m-k}$
  的水平集。
\end{proposition}
\begin{proof}
  若 $S$ 是嵌入 $k$-子流形,任取 $p\in S$,存在 $S$ 在 $M$ 中的切片坐标卡 $(U,\varphi)$,
  使得 $\varphi(U\cap S)$ 是 $\varphi(U)$ 的 $k$-切片,令 $\varPhi:U\to \mathbb{R}^{m-k}$
  为 $\varPhi=\pi\circ\varphi$,其中 $\pi:\mathbb{R}^m\to \mathbb{R}^{m-k}$
  是后 $m-k$ 个坐标的投影,显然 $\varPhi$ 是光滑浸没,且 
  $U\cap S$ 是 $\varPhi$ 的某个水平集。反之,若任取 $p\in S$
  都存在 $M$ 中的邻域 $U$ 使得 $U\cap S$ 是光滑浸没 $\varPhi:U\to \mathbb{R}^{m-k}$
  的水平集,根据浸没水平集定理,$U\cap S$ 是 $U$ 的嵌入 $k$-子流形,从而是
  $M$ 的嵌入 $k$-子流形,这表明 $S$ 满足局部 $k$-切片条件,所以
  $S$ 是 $M$ 的嵌入 $k$-子流形。
\end{proof}

若 $S\subseteq M$ 是嵌入子流形,光滑映射 $\varPhi:M\to N$ 使得 $S$ 是 $\varPhi$
的正则水平集,那么 $\varPhi$ 被称为 $S$ 的\emph{定义映射}。特别地,在 $N=\mathbb{R}^{m-k}$
的情况下,$\varPhi$ 被称为 $S$ 的\emph{定义函数}。\autoref{exa:sphere} 表明
$f(x)=|x|^2$ 是球面的定义函数。若 $U$ 是 $M$ 的开子集,
$\varPhi:U\to N$ 是使得 $S\cap U$ 为正则水平集的光滑映射,那么 $\varPhi$
被称为 $S$ 的\emph{局部定义映射}。

\begin{example}[旋转曲面]
  令 $H$ 是半平面 $\{(r,z)\,|\, r>0\}$,$C\subseteq H$ 是嵌入的 $1$-维子流形。
  定义 $C$ 确定的\emph{旋转曲面}为子集
  \[
    S_C=\left\{(x,y,z)\middle|\left(\sqrt{x^2+y^2},z\right)\in C\right\}\subseteq \mathbb{R}^3.
  \]
  $C$ 被称为\emph{生成曲线}。如果 $\varphi:U\to \mathbb{R}$ 是任意关于 $C$ 在 $H$ 中的局部定义函数,
  那么我们可以得到关于 $S_C$ 的定义函数 $\varPhi$ 为
  \[
    \varPhi(x,y,z)=\varphi\left(\sqrt{x^2+y^2},z\right).
  \]
  $\varPhi$ 定义在开集
  \[
    \wtilde U=\left\{(x,y,z)\middle|\left(\sqrt{x^2+y^2},z\right)\in U\right\}
  \]
  上。计算可得 $\varPhi$ 的 Jacobi 矩阵为
  \[
    D\varPhi(x,y,z)=\left(\frac{x}{r}\frac{\partial\varphi}{\partial r}(r,z),
    \frac{y}{r}\frac{\partial\varphi}{\partial r}(r,z),
    \frac{\partial\varphi}{\partial z}(r,z)\right),
  \]
  其中 $r=\sqrt{x^2+y^2}$。由于 $\varphi$ 是局部定义函数,所以
  $D\varPhi(x,y,z)$ 满秩,所以 $S_C$ 是 $\varPhi$ 的正则水平集,
  也是 $\mathbb{R}^3$ 的 $2$-维嵌入子流形。
\end{example}


\section{浸入子流形}

设 $M$ 是带边或者无边光滑流形,一个子集 $S\subseteq M$ 如果
配备了一个拓扑(不一定是子空间拓扑),在这个拓扑下成为一个拓扑流形,
并且有一个光滑结构使得包含映射 $S\hookrightarrow M$ 是光滑浸入,那么
我们说 $S$ 是 $M$ 的\emph{浸入子流形}。与嵌入子流形类似,
我们定义 $S$ 在 $M$ 中的\emph{余维数}是 $\dim M-\dim S$。

显然,每个嵌入子流形都是浸入子流形,因此浸入子流形是一种更一般的子流形。

\begin{proposition}[浸入的像集作为子流形]\label{prop:image of immersion}
  设 $M$ 是带边或者无边光滑流形,$N$ 是光滑流形,$F:N\to M$
  是单射的光滑浸入。令 $S=F(N)$,那么 $S$ 有唯一的拓扑结构和光滑结构
  使得 $S$ 是 $M$ 的一个光滑子流形并且 $F:N\to S$ 是微分同胚。
\end{proposition}
\begin{proof}
  证明和 \autoref{prop:image of embedding} 基本类似,只不过我们现在凭借
  $N$ 来给予 $S$ 的拓扑结构。定义 $U\subseteq S$ 是开集当且仅当
  $F^{-1}(U)$ 是 $N$ 的开集。定义 $S$ 的坐标卡形如 $\left(F(U),\varphi\circ F^{-1}\right)$,
  其中 $(U,\varphi)$ 是 $N$ 的光滑坐标卡。这些坐标卡是光滑相容的,
  所以给出了 $S$ 上的一个光滑结构。显然在这个光滑结构下,$F:N\to S$
  是微分同胚。包含映射 $S\hookrightarrow M$ 是微分同胚和光滑浸入的复合
  \[
    S\xlongrightarrow{F^{-1}} N\xlongrightarrow{F} M,  
  \]
  所以 $S\hookrightarrow M$ 是光滑浸入,即 $S$ 是 $M$ 的光滑子流形。
\end{proof}

\begin{exercise}{}{}
  设 $M$ 是光滑流形,$S\subseteq M$ 是浸入子流形。证明:
  $S$ 的子空间拓扑下的开集在上述子流形拓扑下也是开的。
  此外,$S$ 在子流形拓扑下的开集是子空间拓扑下的开集当且仅当
  $S$ 是嵌入子流形。
\end{exercise}
\begin{proof}
  设 $U$ 是 $S$ 在子空间拓扑下的开子集,那么存在 $M$ 的开子集 $W$
  使得 $U=S\cap W$,此时 $U=\iota(\iota^{-1}(W))$,由于 $\iota^{-1}(W)$
  是 $S$ 的开子集,所以 $U$ 是 $S$ 在子流形拓扑下的开子集。 

  若 $S$ 在子流形拓扑下的开集是子空间拓扑下的开集,此时表明 $S$
  的子流形拓扑就是子空间拓扑。所以包含映射 $\iota:S\hookrightarrow M$
  是拓扑嵌入,从而是光滑嵌入,即 $S$ 是嵌入子流形。反之,若 $S$
  是嵌入子流形,那么 $\iota:S\hookrightarrow M$ 是拓扑嵌入,
  所以若 $U\subseteq S$ 是开集,那么 $U=\iota(U)$ 是 $\iota(S)=S$
  的开集,即 $U$ 是 $S$ 在子空间拓扑下的开集。
\end{proof}

\begin{proposition}
  设 $M$ 是带边或者无边光滑流形,$S\subseteq M$ 是浸入子流形。
  如果下面三个条件之一被满足,那么 $S$ 是嵌入子流形。
  \begin{enumerate}
    \item $S$ 在 $M$ 中的余维数是 $0$。
    \item 包含映射 $S\hookrightarrow M$ 是恰当映射。
    \item $S$ 是紧子集。
  \end{enumerate}
\end{proposition}
\begin{proof}
  (1) 若 $S$ 在 $M$ 中的余维数是 $0$,由于包含映射 $\iota:S\hookrightarrow M$
  是光滑浸入且 $\dim S=\dim M$,根据 \autoref{prop:local diffeomorphism},
  $\iota$ 是局部微分同胚,所以是开映射,所以 $S$ 是 $M$ 的开子集,
  故 $\iota$ 为拓扑嵌入,所以 $S$ 是嵌入子流形。
\end{proof}

\begin{proposition}[浸入子流形是局部的嵌入]
  如果 $M$ 是带边或者无边光滑流形,$S\subseteq M$ 是浸入子流形,那么对于每个
  $p\in S$,都存在 $p$ 在 $S$ 中的邻域 $U$ 使得 $U$ 是 $M$ 的嵌入子流形。
\end{proposition}
\begin{proof}
  包含映射 $\iota:S\hookrightarrow M$ 是光滑浸入,任取 $p\in S$,根据
  \autoref{thm:local embedding},存在 $p$ 在 $S$ 中的邻域 $U$ 使得
  $\iota|_U:U\hookrightarrow M$ 是光滑嵌入,即 $U$ 是 $M$ 的嵌入子流形。
\end{proof}

设 $S\subseteq M$ 是浸入 $k$-维子流形。设 $U\subseteq \mathbb{R}^k$ 是开集,
如果连续映射 $X:U\to M$ 的像集是 $S$ 的开子集并且 $X:U\to S$ 是拓扑嵌入,
那么我们说 $X$ 是\emph{$S$ 的局部参数化}。如果 $X$
是 $U\to X(U)$ 的微分同胚(相对于 $S$ 的光滑流形结构),那么我们说
$X$ 是\emph{光滑局部参数化}。如果 $X$ 的像集是整个 $S$,那么 $X$
被称为\emph{全局参数化}。

\begin{proposition}\label{prop:local parameterization}
  设 $M$ 是光滑流形,$S\subseteq M$ 是 $k$ 维浸入子流形,$\iota:S\hookrightarrow M$
  是包含映射,$U$ 是 $\mathbb{R}^k$ 的开子集。映射 $X:U\to M$ 是 $S$
  的光滑局部参数化当且仅当存在 $S$ 的一个光滑坐标卡 $(V,\varphi)$
  使得 $X=\iota\circ\varphi^{-1}$。因此,这表明 $S$ 的每个点都在
  某个局部参数化的像集中。
\end{proposition}
\begin{proof}
  若 $X:U\to M$ 是 $S$ 的光滑局部参数化。令 $V=X(U)$,$\varphi=X^{-1}:V\to U$,
  根据定义,$(V,\varphi)$ 是一个坐标卡。任取 $S$ 的光滑坐标卡 $(W,\psi)$,
  那么 $\psi\circ\varphi^{-1}=\psi\circ X$ 是光滑映射,所以
  $(V,\varphi)$ 是 $S$ 的光滑坐标卡。

  反之,若存在 $S$ 的光滑坐标卡 $(V,\varphi)$ 使得 $X=\iota\circ\varphi^{-1}$,
  那么取 $U=\varphi(V)$,此时 $(X:U\to S)=\varphi^{-1}$ 是拓扑嵌入。
  并且 $\varphi$ 是微分同胚表明 $X:U\to V$ 是微分同胚,所以 $X:U\to M$
  是 $S$ 的光滑局部参数化。
\end{proof}

\begin{example}[图像参数化]
  设 $U\subseteq \mathbb{R}^n$ 是开子集,$f:U\to \mathbb{R}^k$ 是光滑函数。
  定义映射 $\gamma_f:U\to \mathbb{R}^n\times \mathbb{R}^k$ 为
  $\gamma_f(u)=(u,f(u))$,那么 $\gamma_f$ 是图像 $\Gamma(f)$ 的光滑全局参数化。
  例如,映射 $F:\mathbb{B}^2\to \mathbb{R}^3$ 为
  \[
    F(u,v)=\left(u,v,\sqrt{1-u^2-v^2}\right)  
  \]
  是 $\mathbb{S}^2$ 的光滑局部参数化,其像集为上半开球面。
\end{example} 

\begin{example}[八字曲线的参数化]
  令 $S\subseteq \mathbb{R}^2$ 是 \autoref{exa:eight-curve} 中的八字曲线,
  将其视为 $\mathbb{R}^2$ 的浸入子流形。那么 \autoref{exa:eight-curve} 中的
  $\beta:(-\pi,\pi)\to \mathbb{R}^2$ 是 $S$ 的光滑全局参数化。
\end{example}



\section{将映射限制在子流形上}

\begin{theorem}[光滑映射限制定义域]
  如果 $M,N$ 是带边或者无边光滑流形,$F:M\to N$ 是光滑映射,$S\subseteq M$
  是浸入或者嵌入子流形,那么 $F|_S:S\to N$ 是光滑映射。
\end{theorem}
\begin{proof}
  注意到 $F|_S=F\circ\iota$,其中 $\iota:S\hookrightarrow M$ 是包含映射。
  那么 $F|_S$ 作为光滑映射的复合是光滑的。
\end{proof}

当光滑映射限制值域的时候,得到的映射并不总是光滑的,我们有下面的例子。

\begin{example}
  令 $S\subseteq \mathbb{R}^2$ 是八字曲线,其拓扑和光滑结构由 \autoref{exa:eight-curve}
  中的 $\beta$ 诱导。定义映射 $G:\mathbb{R}\to \mathbb{R}^2$ 为
  \[
    G(t)=(\sin 2t,\sin t).
  \]
  这与 $\beta$ 的形式一致,但是定义域扩大到了整个实数。显然 $G$ 的像集在 $S$
  中,但是 $G$ 视为 $\mathbb{R}\to S$ 的映射不是光滑的,甚至不是连续的。
  这是因为 $-\pi+2n\pi<t<\pi+2n\pi$ 时 $\beta^{-1}\circ G(t)=t-2n\pi$,
  $t=\pm\pi+2n\pi$ 时 $\beta^{-1}\circ G(t)=0$。
\end{example}

\begin{theorem}[光滑映射限制值域]
  设 $M$ 是光滑流形,$S\subseteq M$ 是浸入子流形,$F:N\to M$
  是光滑映射并且 $F(N)\subseteq S$,如果 $F:N\to S$ 是连续映射,那么
  $F:N\to S$ 光滑。
\end{theorem}
\begin{proof}
  任取 $p\in N$,记 $q=F(p)\in S$。根据局部嵌入定理,存在 $q$ 在 $S$ 中的某个邻域 $V$
  使得 $V$ 是 $M$ 的嵌入子流形,所以存在 $V$ 在 $M$ 中的切片坐标卡 $(W,\psi)$,
  令 $V_0=W\cap V$,设 $\psi(V_0)$ 是 $\psi(W)$ 的 $k$-切片。令 $\pi:\mathbb{R}^n\to \mathbb{R}^k$
  为前 $k$ 个分量的投影,$\tilde{\psi}=\pi\circ\psi$。由于 $V_0$ 是 $V$ 中的开集,
  $V$ 是 $S$ 的开集,所以 $V_0$ 是 $S$ 的开集,所以 $(V_0,\tilde\psi)$ 是 $S$ 的光滑坐标卡。

  令 $U=F^{-1}(V_0)$,$F:N\to S$ 连续表明 $U$ 是 $N$ 的开集,取 $p$ 处的光滑坐标卡 $(U_0,\varphi)$
  使得 $U_0\subseteq U$,那么 $F:N\to S$ 的坐标表示为
  \[
    \tilde{\psi}\circ F\circ \varphi^{-1}=\pi\circ(\psi\circ F\circ\varphi^{-1}),
  \] 
  右端的 $F:N\to M$,所以这是光滑映射,故 $F:N\to S$ 光滑。
\end{proof}

\begin{corollary}
  令 $M$ 是光滑流形,$S\subseteq M$ 是嵌入子流形,那么光滑映射 $F:N\to M$ 只要 $F(N)\subseteq S$,
  那么 $F:N\to S$ 就是光滑映射。
\end{corollary}
\begin{proof}
  $S$ 配备子空间拓扑,所以 $F:N\to S$ 作为连续映射的限制是连续映射。
\end{proof}

\section{子流形的切空间}

令 $M$ 是带边或者无边光滑流形,$S\subseteq M$ 是嵌入或者浸入子流形,由于 $\iota:S\hookrightarrow M$
是光滑浸入,所以任取 $p\in S$ 微分 $d\iota_p:T_pS\to T_pM$ 是单射,这意味着我们可以把 $T_pS$ 视为
$T_pM$ 的子空间,此时对于任意切向量 $v\in T_pS$,它的像 $\tilde v=d\iota_p(v)\in T_pM$ 为
\[
  \tilde{v}f=d\iota_p(v)(f)=v(f\circ\iota)=v(f|_S),
\]
其中 $f\in C^\infty(M)$,我们将 $v$ 和 $\tilde v$ 等同。注意这种等同与嵌入或者浸入无关。

\begin{proposition}
  设 $M$ 是带边或者无边光滑流形,$S\subseteq M$ 是浸入或者嵌入子流形,$p\in S$。
  切向量 $v\in T_pM$ 在 $T_pS$ 中当且仅当存在光滑曲线 $\gamma:J\to M$,
  其像集 $\gamma(J)\subseteq S$,并且作为 $J\to S$ 的映射是光滑映射,使得
  $0\in J$,$\gamma(0)=p$ 以及 $\gamma'(0)=v$。
\end{proposition}
\begin{proof}
  $v\in T_pS$ 意味着存在 $w\in T_pS$ 使得 $v=d\iota_p(w)$,
  根据 \autoref{prop:velocity},存在光滑曲线 $\tilde\gamma:J\to S$
  使得 $\tilde\gamma(0)=p$,$\tilde\gamma'(0)=w$,令
  $\gamma=\iota\circ\tilde{\gamma}$,那么 $\gamma'(0)=d\iota_p(\gamma'(0))=v$。
  反之,由于 $\tilde{\gamma}:J\to S$ 是光滑映射,所以
  $\gamma=\iota\circ\tilde{\gamma}$,所以
  \[
    v=\gamma'(0)=d\iota_p(\tilde \gamma'(0))  \in \im d\iota_p.\qedhere
  \]
\end{proof}

\begin{proposition}
  设 $M$ 是光滑流形,$S\subseteq M$ 是嵌入子流形,$p\in S$。
  切空间 $T_pS$ 作为 $T_pM$ 的子空间可以刻画为
  \[
    T_pS=\bigl\{v\in T_pM\,|\, \forall f\in C^\infty(M), f|_S=0,vf=0\bigr\}  .
  \]
\end{proposition}
\begin{proof}
  任取 $v=d\iota_p(w)\in T_p S\subseteq T_pM$,那么对于任意 $f\in C^\infty(M)$ 且 $f|_S=0$,有
  \[
    vf=d\iota_p(w)f=w(f\circ\iota_p)=w(f|_S)=0.
  \]

  反之,若 $v\in T_pM$ 满足 $f|_S=0$ 的时候 $vf=0$。
  假设 $(U,\varphi=(x^1,\dots,x^n))$ 是 $S$ 在 $M$ 中的切片坐标卡,
  那么 $T_pM$ 的一组基为 $\partial/\partial x^i|_p$。
  设 $\varphi(S\cap U)$ 是 $\varphi(U)$ 的 $k$-切片,那么
  $\iota:S\cap U\hookrightarrow M$ 的坐标表示为
  \[
    \varphi\circ\iota\circ\tilde\varphi^{-1}:\left(x^1,\dots,x^k\right)\mapsto
    \left(x^1,\dots,x^k,0,\dots,0\right)  ,
  \]
  这表明
  \[
    d\iota_p\left(\left.\frac{\partial}{\partial x^i}\right|_p\right)
    =  \left.\frac{\partial}{\partial x^i}\right|_p,\quad i=1,\dots,k.
  \]
  所以 $d\iota_p(T_pS)$ 是由 $\partial/\partial x^1|_p,\dots,\partial/\partial x^k|_p$
  张成的 $T_pM$ 的子空间。设
  \[
    v=v^i \left.\frac{\partial}{\partial x^i}\right|_p\in T_pM,  
  \]
  所以 $v\in T_pS$ 当且仅当 $v^j=0\ (j>k)$。

  令 $\psi:M\to \mathbb{R}$ 是关于 $\{p\}$ 的支在 $U$ 中的光滑鼓包函数,
  对于每个 $j>k$,考虑函数 $f(q)=\psi(q)x^j$,其中 $q\in M$。
  显然 $f|_S=0$,所以
  \[
    0=vf= v^i\frac{\partial (f\circ\varphi^{-1})}{\partial x^i}(\varphi(p))
    =v^j.\qedhere  
  \]
\end{proof}

\begin{proposition}
  假设 $M$ 是光滑流形,$S\subseteq M$ 是嵌入子流形,如果 $\varPhi:U\to N$
  是 $S$ 的局部定义映射,那么 $T_pS=\ker d\varPhi_p:T_pM\to T_{\varPhi(p)}N$,
  其中 $p\in S\cap U$。
\end{proposition}
\begin{proof}
  根据定义,$S\cap U$ 是 $\varPhi$ 的正则水平集,令 $\iota:S\cap U\hookrightarrow U$,
  那么 $\varPhi\circ\iota:S\cap U\to N$ 是常值映射,所以
  $d(\varPhi\circ\iota)_p:T_pS\to T_{\varPhi(p)}N$ 是零映射,所以
  \[
    d\varPhi_p\circ d\iota_p=d(\varPhi\circ\iota)_p=0,
  \]
  这表明 $T_pS=\im d\iota_p\subseteq \ker d\varPhi_p$。另一方面,$\varPhi$ 是定义映射表明
  $d\varPhi_p$ 是满射,故
  \[
    \dim\ker d\varPhi_p=\dim T_pM-\dim T_{\varPhi(p)}N=\dim S\cap U=
    \dim T_pS,
  \]
  这就表明 $T_pS=\ker d\varPhi_p$。
\end{proof}

\begin{corollary}
  设 $S\subseteq M$ 是光滑浸没 $\varPhi=(\varPhi^1,\dots,\varPhi^k):M\to \mathbb{R}^k$
  的水平集,切向量 $v\in T_pM$ 在 $T_pS$ 中当且仅当 $v\varPhi^1=\cdots=v\varPhi^k=0$。
\end{corollary}

\begin{exercise}\label{exer:tangent space of submanifold}
  设 $S\subseteq M$ 是常秩映射 $\varPhi:M\to N$ 的水平集,证明对于每个 $p\in S$
  有 $T_pS=\ker d\varPhi_p$。
\end{exercise}
\begin{proof}
  根据常秩水平集定理,有 $\dim S=\dim M-\dim N$,故
  \[
    \dim T_pS=\dim T_pM-\dim T_{\varPhi(p)}N=\dim \ker d\varPhi_p.
  \]
  记 $\iota:S\hookrightarrow M$,那么 $\varPhi\circ\iota:S\to N$ 是常值映射,
  所以 $d(\varPhi\circ \iota)_p=0$,这就表明 $T_pS=\ker d\varPhi_p$。
\end{proof}
