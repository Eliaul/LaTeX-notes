

\chapter{李群}

\section{基本定义}

\begin{example}[李群]
  \mbox{}
  \begin{enumerate}
    \item $n\times n$ 可逆实矩阵集合 $\GL(n,\mathbb{R})$ 被称为\emph{一般线性群}。
    其是 $M(n,\mathbb{R})$ 的开子流形,由于矩阵乘法 $AB$ 是多项式,所以
    是光滑的。矩阵求逆的光滑性由 Cramer 法则保证。
    \item 令 $\GL^+(n,\mathbb{R})$ 表示 $\GL(n,\mathbb{R})$ 中行列式为正的
    可逆矩阵集合。显然 $\GL^+(n,\mathbb{R})$ 是 $\GL(n,\mathbb{R})$ 的子群
    同时是开子集,所以其群运算作为 $\GL(n,\mathbb{R})$ 的群运算的限制是光滑的,
    所以 $\GL^+(n,\mathbb{R})$ 是李群。
    \item 设 $G$ 的任意李群,$H\subseteq G$ 是\emph{开子群}(子群同时是开子集)。
    类似 (2) 中的叙述,$H$ 是李群,其群结构和光滑结构都继承于 $G$。
    \item \emph{复一般线性群} $\GL(n,\mathbb{C})$ 是 $M(n,\mathbb{C})$
    的开子流形,从而是 $2n^2$-维光滑流形,由于其矩阵乘法和求逆都是光滑的,
    所以是一个李群。
    \item 如果 $V$ 是实或者复向量空间,$\GL(V)$ 表示 $V$ 上可逆线性变换的集合。
    如果 $V$ 的维数是 $n$,那么 $V$ 的任意一组基都确定了 $\GL(V)$ 到 $\GL(n,\mathbb{R})$
    或者 $\GL(n,\mathbb{C})$ 的一个同构,所以 $\GL(V)$ 是李群。任意两个这样的
    同构之间的转移映射形如 $A\mapsto BAB^{-1}$ (其中 $B$ 是两组基之间的过渡矩阵),
    这是光滑的,所以 $\GL(V)$ 的光滑结构独立于基的选取。
    \item Euclid 空间 $\mathbb{R}^n$ 在加法下是李群。类似的,$\mathbb{C}$
    和 $\mathbb{C}^n$ 也是李群。
    \item 圆周 $\mathbb{S}^1\subseteq \mathbb{C^*}$ 是光滑流形并且在
    复数乘法下构成群。利用 $\mathbb{S}^1$ 的开子集上的适当的角度函数
    作为局部坐标,乘法和求逆是光滑的,它们分别有表示 $(\theta_1,\theta_2)\mapsto\theta_1+\theta_2$
    和 $\theta\mapsto -\theta$,因此 $\mathbb{S}^1$ 是李群,称为\emph{圆周群}。
    \item 给定李群 $G_1,\dots,G_k$,定义\emph{直积}是积流形 $G_1\times\cdots\times G_k$
    同时配备直积群的群结构,这是一个李群。
  \end{enumerate}
\end{example}

\section{李群同态}

如果 $G,H$ 是李群,如果 $F:G\to H$ 同时是群同态和光滑映射,那么我们说
$F$ 是\emph{李群同态}。如果 $F$ 还是微分同胚,那么我们说 $F$
是\emph{李群同构}。

\begin{theorem}
  李群同态是常秩映射。
\end{theorem} 
\begin{proof}
  设 $F:G\to H$ 是李群同态,记 $G$ 的单位元为 $e$,$H$ 的单位元为 $\tilde{e}$。
  任取 $g_0\in G$,对于任意 $g\in G$,有
  \[
    F(L_{g_0}(g))=F(g_0g)=F(g_0)F(g)=L_{F(g_0)}(F(g)),  
  \] 
  所以 $F\circ L_{g_0}=L_{F(g_0)}\circ F$,所以
  \[
    dF_{g_0}\circ d\left(L_{g_0}\right)_e=d\left(L_{F(g_0)}\right)_{\tilde{e}}\circ dF_e,  
  \]
  由于 $L_{g_0}$ 和 $L_{F(g_0)}$ 是微分同胚,所以 $dF_{g_0}$ 和 $dF_e$
  有相同的秩,故 $F$ 是常秩映射。
\end{proof}

\begin{corollary}
  一个李群同态是李群同构当且仅当其是双射。
\end{corollary}
\begin{proof}
  全局秩定理表明双射的常秩映射是微分同胚。
\end{proof}

\section{李子群}

设 $G$ 的李群,$G$ 的\emph{李子群}指的是 $G$ 的一个子群,配备了一个拓扑
和光滑结构使得其成为一个李群以及 $G$ 的浸入子流形。

\begin{proposition}\label{prop:embedding and subgroup is lie subgroup}
  令 $G$ 是李群,设 $H\subseteq G$ 是子群同时是嵌入子流形,那么 $H$ 是李子群。
\end{proposition}
\begin{proof}
  乘法 $G\times G\to G$ 是光滑的,$H$ 是嵌入子流形表明 $H\times H\to G$
  光滑(这一步实际上只需要浸入子流形),将值域限制在嵌入子流形不改变光滑性,
  所以 $H\times H\to H$ 是光滑的。类似地,求逆也是光滑映射,所以 $H$
  是李群。
\end{proof}

\begin{lemma}
  设 $G$ 是李群,$H\subseteq G$ 是开子群。那么 $H$ 是嵌入李子群并且是闭集,故 $H$
  是 $G$ 的连通分支的并。
\end{lemma}
\begin{proof}
  $H$ 是开子流形表明 $H$ 是嵌入李子群。对于任意 $g\in G$,陪集 $gH=L_g(H)$ 是开集,所以
  $G\smallsetminus H$ 作为陪集的并是开集,所以 $H$ 是闭集。$H$ 既开又闭表明 $H$
  是连通分支的并集。
\end{proof}


\begin{proposition}
  设 $G$ 是李群,$W\subseteq G$ 是单位元处的任意邻域。
  \begin{enumerate}
    \item $W$ 生成 $G$ 的一个开子群。
    \item 如果 $W$ 是连通的,那么其生成 $G$ 的一个连通开子群。
    \item 如果 $G$ 是连通的,那么 $W$ 生成 $G$。
  \end{enumerate}
\end{proposition}
\begin{proof}
  (1) 记 $W_1=W\cup W^{-1}$,对于 $k\geq 2$,递归地定义 $W_k=W_1W_{k-1}$,那么
  \[
    \langle W\rangle=\bigcup_{k=1}^\infty W_k,
  \]
  由于求逆映射 $g\mapsto g^{-1}$ 是微分同胚,所以 $W^{-1}$ 是开集,所以 $W_1$ 是开集。
  假设 $W_{k-1}$ 是开集,那么
  \[
    W_k=\bigcup_{g\in W_1}L_g(W_{k-1})
  \]
  是开集的并,所以 $W_k$ 是开集。故 $\langle W\rangle$ 是开集。

  (2) 由于 $W$ 和 $W^{-1}$ 都是连通的,且 $e\in W\cap W^{-1}$,所以 $W_1$ 是连通的。
  假设 $W_{k-1}$ 是连通的,那么 $W_k=m(W_1\times W_{k-1})$ 是连通空间在连续映射下的像,所以是连通的。
  又因为 $e\in W_k$,所以 $\langle W\rangle$ 是连通的。

  (3) $\langle W\rangle$ 是开子群表明 $\langle W\rangle$ 是连通分支的并,$G$ 连通表明 $\langle W\rangle=G$。
\end{proof}

若 $G$ 是李群,则 $G$ 的包含单位元的连通分支被称为 $G$ 的\emph{单位分支}。

\begin{proposition}
  令 $G$ 是李群,$G_0$ 是单位分支。那么 $G_0$ 是 $G$ 的正规子群,并且是唯一的连通开子群。
  $G$ 的任意连通分支都微分同胚于 $G_0$。
\end{proposition}
\begin{proof}
  任取 $g\in G$,那么 $gG_0g^{-1}=L_g\left(R_{g^{-1}}(G_0)\right)$ 是连通的,并且 $e\in gG_0g^{-1}$,所以
  $gG_0g^{-1}\subseteq G_0$,这就表明 $G_0$ 是正规子群。设 $G_0'$ 是连通开子群,那么 $G_0'$ 是开子群表明
  $G_0'$ 是连通分支的并集,$G_0'$ 是连通的表明 $G_0'$ 是某一个连通分支,又因为 $e\in G_0'$,所以 $G_0'=G_0$
  是单位分支。设 $G_0'$ 是任意连通分支,任取 $g\in G_0'$,$g^{-1}G_0'=L_{g^{-1}}(G_0')$ 是连通的且包含
  单位元,故 $g^{-1}G_0'$ 被单位分支 $G_0$ 包含,那么 $L_{g}(G_0)=gG_0\supseteq G_0'$ 是连通的,所以
  $G_0'=gG_0=L_g(G_0)$ 微分同胚于 $G_0$。
\end{proof}
 
\begin{proposition}
  令 $F:G\to H$ 是李群同态,那么 $\ker F$ 是 $G$ 的恰当嵌入李子群,其余维数
  为 $\rk F$。
\end{proposition}
\begin{proof}
  根据常秩水平集定理,$\ker F$ 是 $G$ 的余维数为 $\rk F$ 的嵌入子流形。
  再根据 \autoref{prop:embedding and subgroup is lie subgroup},$\ker F$
  是李子群。
\end{proof}

\begin{proposition}\label{prop:image of injective lie group homomorphism}
  如果 $F:G\to H$ 是单射的李群同态,那么 $F$ 的像集有唯一的光滑流形结构使得
  $F(G)$ 是 $H$ 的李子群并且 $F:G\to F(G)$ 是李群同构。
\end{proposition}
\begin{proof}
  因为李群同态是常秩映射,所以根据全局秩定理,此时 $F$ 是浸入。
  根据 \autoref{prop:image of immersion},所以 $F(G)$ 有唯一的光滑结构
  使得 $F(G)$ 是 $H$ 的浸入子流形且 $F:G\to F(G)$ 是微分同胚。
\end{proof}

\begin{example}[嵌入李子群]
  \mbox{}
  \begin{enumerate}
    \item 子群 $\GL^+(n,\mathbb{R})\subseteq \GL(n,\mathbb{R})$ 是开子群,
    从而是嵌入李子群。
    \item 圆周 $\mathbb{S}^1$ 是 $\mathbb{C}^*$ 的嵌入李子群,因为其是嵌入子流形。
    \item 行列式为 $1$ 的实矩阵集合 $\SL(n,\mathbb{R})$ 被称为\emph{特殊线性群}。
    因为 $\SL(n,\mathbb{R})$ 是李群满同态 $\det:\GL(n,\mathbb{R})\to \mathbb{R}^*$
    的核,根据全局秩定理,$\det$ 是光滑浸没,所以 $\SL(n,\mathbb{R})$ 是维数
    $n^2-1$ 的嵌入李子群。
  \end{enumerate}
\end{example}

一般情况下,光滑子流形可以同时是非嵌入的以及闭的,例如八字曲线
\ref{exa:eight-curve},但是下面的定理表明对于李子群而言,
闭和嵌入性并不是独立的,即嵌入性和闭等价。

\begin{theorem}
  设 $G$ 是李群,$H\subseteq G$ 是李子群,那么 $H$ 是闭集当且仅当
  $H$ 是嵌入李子群。
\end{theorem}

\section{群作用和等变映射}

如果 $M$ 是光滑流形,$G$ 是李群,群作用 $G\times M\to M$ 是光滑映射,
那么这个群作用被称为\emph{光滑作用}。
对于光滑作用 $\theta:G\times M\to M$,任取 $g\in G$,记 
$\theta_g:M\to M$ 为 $\theta_g(p)=g\cdot p$。由于 $\theta_{g^{-1}}$
是其光滑逆映射,所以 $\theta_g$ 一定是微分同胚。

设 $G$ 是李群,$M,N$ 是带边或者无边光滑流形并且都配备一个光滑
$G$-作用,映射 $F:M\to N$ 被称为\emph{等变的},如果对于每个 $g\in G$
都有
\[
  F(g\cdot p)=g\cdot F(p) \ (\text{left actions})  ,\quad
  F(p\cdot g)=F(p)\cdot g \ (\text{right actions})  .
\]
等价地说,如果 $\theta$ 是 $G$ 在 $M$ 上的光滑作用,$\varphi$
是 $G$ 在 $N$ 上的光滑作用,那么 $F$ 是等变的当且仅当
\[
  F\circ \theta_g=\varphi_g\circ F.  
\]

\begin{theorem}[等变秩定理]
  设 $G$ 是李群,$M,N$ 是带边或者无边光滑流形,
  $F:M\to N$ 是光滑映射,相对于 $M$ 上的传递的光滑 $G$-作用 $\theta$ 和
  $N$ 上的任意光滑 $G$-作用 $\varphi$ 是等变的。那么 $F$ 是常值映射。
  进而,若 $F$ 是满射,那么是一个光滑浸没;若 $F$ 是单射,那么
  是一个光滑浸入;若 $F$ 是双射,那么是一个微分同胚。
\end{theorem}
\begin{proof}
  任取 $p,q\in M$,$\theta$ 是传递的表明存在 $g\in G$ 使得
  $q=\theta_g(p)$,那么 $F\circ\theta_g=\varphi_g\circ F$ 表明
  \[
    dF_{q}\circ d(\theta_g)_p=d(\varphi_g)_{F(p)}\circ dF_p,  
  \]
  $\theta_g$ 和 $\varphi_g$ 都是微分同胚表明 $dF_q$ 和 $dF_p$
  的秩相同,所以 $F$ 是常秩映射。
\end{proof}

假设 $G$ 是李群,$M$ 是光滑流形,$\theta:G\times M\to M$ 是光滑左作用,
对于每个 $p\in M$,定义\emph{轨道映射} $\theta^{(p)}:G\to M$ 为
\[
  \theta^{(p)}(g)=g\cdot p.  
\]
显然 $\theta^{(p)}$ 的像集就是 $p$ 所在的轨道。

\begin{proposition}[轨道映射的性质]
  假设 $\theta$ 是李群 $G$ 在光滑流形 $M$ 上的光滑左作用,对于每个
  $p\in M$,轨道映射 $\theta^{(p)}:G\to M$ 是光滑常秩映射,
  所以稳定化子 $G_p=\left(\theta^{(p)}\right)^{-1}(p)$
  是 $G$ 的恰当嵌入李子群。如果 $G_p=\{e\}$,那么
  $\theta^{(p)}$ 是单射的光滑浸入,所以轨道 $G\cdot p$
  是 $M$ 的浸入子流形。
\end{proposition}
\begin{proof}
  映射 $\theta^{(p)}$ 可以视为光滑映射的复合
  \[
    G\to G\times M\to M,  
  \]
  所以 $\theta^{(p)}$ 是光滑映射。考虑 $G$ 在自身上的作用为
  $g\cdot g'=gg'$,这是一个传递的光滑作用,那么
  \[
    \theta^{(p)}(g\cdot g')=(gg')\cdot p=g\cdot \theta^{(p)}(g'),  
  \]
  所以 $\theta^{(p)}$ 是等变映射,从而是常秩映射。根据常秩水平集定理,
  稳定化子 $G_p$ 是嵌入子流形,根据 \autoref{prop:embedding and subgroup is lie subgroup},
  $G_p$ 是恰当嵌入李子群。若 $G_p=\{e\}$,那么
  \[
    \theta^{(p)}(g)=\theta^{(p)}(g')\Rightarrow
    p=(g^{-1}g')\cdot p\Rightarrow g^{-1}  g'\in G_p
    \Rightarrow g=g',
  \]
  所以 $\theta^{(p)}$ 为单射。再根据全局秩定理,$\theta^{(p)}$
  是浸入。根据 \autoref{prop:image of immersion},轨道 $G\cdot p$
  是浸入子流形。
\end{proof}

\begin{example}[正交群]\label{exa:orthogonal group}
  由 $n\times n$ 正交矩阵构成的 $\GL(n,\mathbb{R})$ 的子群 
  $\Orth(n)$ 被称为正交群。定义光滑映射 $\varPhi:\GL(n,\mathbb{R})\to M(n,\mathbb{R})$
  为 $\varPhi(A)=A^TA$,那么 $\Orth(n)$ 是 $\varPhi$ 的水平集 $\varPhi^{-1}(I_n)$。
  我们只要能说明 $\varPhi$ 是常秩映射就能够表明 $\Orth(n)$ 是 
  $\GL(n,\mathbb{R})$ 的嵌入李子群。下面我们构造合适的群作用使得
  $\varPhi$ 是等变映射。定义 $\theta$ 为 $\GL(n.\mathbb{R})$
  在自身上的右乘作用,即 $A\cdot B=AB$。$\varphi$
  为 $\GL(n,\mathbb{R})$ 在 $M(n,\mathbb{R})$ 上的右作用为:
  \[
    X\cdot B=B^TXB,  
  \]
  那么
  \[
    \varPhi(A\cdot B)=(AB)^T(AB)=B^TA^TAB=\varPhi(A)\cdot B ,
  \]
  这就表明 $\varPhi$ 是等变映射,根据等变秩定理,$\varPhi$
  确实是常秩映射。因此,$\Orth(n)$ 是 $\GL(n,\mathbb{R})$
  的嵌入李子群。此外,不难发现 $\Orth(n)$ 是紧的。

  为了确定 $\Orth(n)$ 的维数,还需要确定 $\varPhi$ 的秩,即
  微分 $d\varPhi_{I_n}$ 的像空间的维数。对于 $B\in T_{I_n}\GL(n,\mathbb{R})=M(n,\mathbb{R})$,
  设 $\gamma:(-\varepsilon,\varepsilon)\to \GL(n,\mathbb{R})$ 
  光滑曲线 $\gamma(t)=I_n+tB$,那么
  \[
    d\varPhi_{I_n}(B)=\left.\frac{d}{dt}\right|_{t=0}\varPhi(\gamma(t))
    =\left.\frac{d}{dt}\right|_{t=0}(I_n+tB^T)(I_n+tB)
    =B+B^T,
  \]
  所以 $\im d\varPhi_{I_n}$ 被对称矩阵空间包含。反之,任取对称矩阵 $B$,
  有 $d\varPhi_{I_n}(B/2)=B$,所以 $\im d\varPhi_{I_n}$ 就是对称矩阵空间,
  维数为 $(n^2+n)/2$,这就表明 $\rk \varPhi=(n^2+n)/2$,
  故
  \[
    \dim\Orth(n)=\dim\GL(n,\mathbb{R})-\rk \varPhi
    =\frac{n^2-n}{2}.
  \]
\end{example}

\begin{example}[特殊正交群]
  $n\times n$ 特殊正交群被定义为行列式为 $1$ 的正交矩阵构成的集合,
  记为 $\SO(n)$。由于 $\det:\Orth(n)\to \mathbb{R}$ 为连续映射
  且 $\SO(n)=\det^{-1}(0,\infty)$,所以 $\SO(n)$ 是 $\Orth(n)$
  的开子集,故 $\SO(n)$ 是 $\Orth(n)$ 的嵌入李子群并且
  $\dim \SO(n)=\dim\Orth(n)=(n^2-n)/2$。同时,$\SO(n)$
  也是 $\Orth(n)$ 的闭子集,所以也是紧的。
\end{example}

\subsection{表示}

目前我们看到的大部分李群都可以视为 $\GL(n,\mathbb{R})$ 或者 $\GL(n,\mathbb{C})$
的李子群。一个自然的问题就是是否所有的李群都有这种形式。研究这个问题的关键是
群表示论。

如果 $V$ 是有限维实或者复向量空间,我们用 $\GL(V)$ 表示 $V$ 上的可逆线性变换
群,这是一个同构于 $\GL(n,\mathbb{R})$ 或 $\GL(n,\mathbb{C})$
的李群,其中 $n=\dim V$。如果 $G$ 是李群,我们说 $G$ 的\emph{(有限维)表示}
指的是一个 $G$ 到某个 $\GL(V)$ 的李群同态。

如果表示 $\rho:G\to \GL(V)$ 是单射,那么我们说这个表示是\emph{忠实的}。
此时根据 \autoref{prop:image of injective lie group homomorphism},
$\rho(G)$ 是 $\GL(V)$ 的李子群,且 $\rho:G\to \rho(G)$ 是李群同构。
因此,一个李群有一个忠实表示当且仅当其同构于 $\GL(n,\mathbb{R})$ 或 $\GL(n,\mathbb{C})$
的某个李子群。不是所有的李群都有这样的表示,但是我们还没有构造反例的技术。

\begin{example}[李群表示]
  \mbox{}
  \begin{enumerate}
    \item 包含映射 $\mathbb{S}^1\hookrightarrow \mathbb{C}^*\simeq \GL(1,\mathbb{C})$
    是圆周群的忠实表示。更一般地,映射 $\rho:\mathbb{T}^n\to \GL(n,\mathbb{C})$
    \[
      \rho\bigl(z^1,\dots,z^n\bigr)  =\begin{pmatrix}
        z^1 & 0 & \cdots & 0 \\
        0 & z^2 & \cdots & 0 \\
        \vdots & \vdots & \ddots & \vdots \\
        0  & 0 & \cdots & z^n 
      \end{pmatrix}
    \]
    是 $\mathbb{T}^n$ 的忠实表示。
    \item 令 $\sigma:\mathbb{R}^n\to \GL(n+1,\mathbb{R})$ 为
    \[
      \sigma(x)=\begin{pmatrix}
        I_n & x \\
        0 & 1
      \end{pmatrix}  ,
    \]
    此时 $\sigma$ 是 $\mathbb{R}^n$ 的忠实表示。
    \item 令 $\Euc(n)=\mathbb{R}^n\rtimes \Orth(n)$ 是 Euclid 群,
    $\Euc(n)$ 的一个忠实表示 $\rho:\Euc(n)\to \GL(n,\mathbb{R})$ 为
    \[
      \rho(b,A)=\begin{pmatrix}
        A & b \\
        0 & 1
      \end{pmatrix}  .
    \]
  \end{enumerate}
\end{example}

\section{问题}

\begin{problem}{}{}
  令 $\mathbb{H}=\mathbb{C}\times \mathbb{C}$(视为实向量空间),定义
  双线性的乘法 $\mathbb{H}\times \mathbb{H}\to \mathbb{H}$ 为
  \[
    (a,b)(c,d)=\bigl(ac-b\bar d,ad+b\bar c\bigr),\quad a,b,c,d\in \mathbb{C}.  
  \]
  在这个乘法下,$\mathbb{H}$ 是 $\mathbb{R}$ 上的 $4$-维代数,
  被称为\emph{四元数代数}。对于每个 $p=(a,b)\in \mathbb{H}$,定义
  $p^*=\bigl(\bar a,-b\bigr)$。定义 $\mathbb{H}$ 的一组基为
  $(\mathbb{1},\mathbb{i},\mathbb{j},\mathbb{k})$ 为
  \[
    \mathbb{1}=(1,0),\mathbb{i}=(i,0),\mathbb{j}=(0,1),\mathbb{k}=(0,i)  ,
  \]
  那么可以验证它们满足    
  \begin{gather*}
    \mathbb{i}^2=\mathbb{j}^2=\mathbb{k}^2=-1,\quad
    \mathbb{1}q=q \mathbb{1}=q\ \forall q\in \mathbb{H},\\
    \mathbb{ij}=-\mathbb{ji}=\mathbb{k},\quad \mathbb{jk}=-\mathbb{kj}=\mathbb{i},
    \quad \mathbb{ki}=-\mathbb{ik}=\mathbb{j},\\ 
    \mathbb{1}^*=\mathbb{1},\quad \mathbb{i}^*=-\mathbb{i},\quad \mathbb{j}^*=-\mathbb{j},
    \quad \mathbb{k}^*=-\mathbb{k}.
  \end{gather*}
  如果 $p^*=p$,那么称 $p$ 为实四元数,如果 $p^*=-p$,那么称 $p$ 为纯虚四元数。
  实四元数可以通过对应 $x\leftrightarrow x\mathbb{1}$ 等同为实数。
  \begin{enumerate}
    \item 证明四元数乘法是结合的但不是交换的。
    \item 证明 $(pq)^*=q^*p^*$。
    \item 证明 $\langle p,q\rangle=\frac{1}{2}(p^*q+q^*p)$ 是 $\mathbb{H}$
    上的一个内积,其导出的范数满足 $|pq|=|p||q|$。
    \item 证明任意非零四元数都有乘法逆元 $p^{-1}=p^*/|p|^2$。
    \item 证明非零四元数集合 $\mathbb{H}^*$ 在四元数乘法下构成李群。
  \end{enumerate} 
\end{problem}
\begin{proof}
  (5) $\mathbb{H}^*$ 是乘法群。范数映射 $|\cdot|:\mathbb{H}\to \mathbb{R}$
  是连续映射,所以 $\mathbb{H}^*$ 作为 $\mathbb{R}\smallsetminus\{0\}$
  的原像是 $\mathbb{H}$ 的开子集,故 $\mathbb{H}^*$ 是开子流形,
  且容易验证 $\mathbb{H}^*$ 的乘法的求逆都是光滑映射,所以 
  $\mathbb{H}^*$ 是李群,维数为 $4$。
\end{proof}

\begin{problem}{}{}
  令 $\mathbb{H}^*$ 是非零四元数李群,$\mathcal{S}\subseteq \mathbb{H}^*$
  是单位四元数的集合,证明 $\mathcal{S}$ 是 $\mathbb{H}^*$ 的恰当嵌入李子群,
  并且同构于 $\SU(2)$。
\end{problem}
\begin{proof}
  记 $N:\mathbb{H}^*\to \mathbb{R}^*$ 为范数映射,显然这是一个李群同态,
  所以是常秩映射,所以 $\mathcal{S}=\ker N$ 是恰当嵌入李子群。
\end{proof}
