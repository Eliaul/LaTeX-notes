
\chapter{浸没、浸入和嵌入}

\section{常秩映射}

设 $M,N$ 是带边或者无边光滑流形。给定一个光滑映射 $F:M\to N$
和点 $p\in M$,定义\emph{$F$ 在 $p$ 处的秩}为线性映射 $dF_p:T_pM\to T_{F(p)}N$
的秩。显然其等价定义有 $F$ 在任意光滑坐标卡下的 Jacobi 矩阵的秩以及
$\im dF_p\subseteq T_{F(p)}N$ 的维数。如果 $F$ 在任意点处的秩
都是 $r$,那么我们说 $F$ 是\emph{常秩}的,记为 $\rk F=r$。
显然 $F$ 在任意点处的秩都小于等于 $\min\{\dim M,\dim N\}$,如果
$dF_p$ 的秩等于 $\min\{\dim M,\dim N\}$,那么我们说\emph{$F$ 在 $p$ 处满秩}。
如果 $F$ 在任意点处都满秩,那么我们说\emph{$F$ 满秩}。

如果光滑映射 $F:M\to N$ 在任意点处的微分都是满射(等价地说,$\rk F=\dim N$),那么我们说 $F$ 是\emph{光滑浸没}。
如果 $F$ 在任意点处的微分都是单射(等价地说,$\rk F=\dim M$),那么我们说 $F$ 是\emph{光滑浸入}。

\begin{proposition}\label{prop:regular points is open}
  设 $F:M\to N$ 是光滑映射,$p\in M$。如果 $dF_p$ 是满射,那么存在 $p$ 
  的邻域 $U$ 使得 $F|_U$ 是浸没。如果 $dF_p$ 是单射,那么存在 $p$ 
  的邻域 $U$ 使得 $F|_U$ 是浸入。
\end{proposition}
\begin{proof}
  任取 $p$ 处的光滑坐标卡 $(W,\varphi)$ 和 $F(p)$ 处的光滑坐标卡 $(V,\psi)$,那么 $F$
  的坐标表示 $\psi\circ F\circ\varphi^{-1}$ 在 $\hat p=\varphi(p)$ 处的 Jacobi 矩阵是满秩矩阵,由于满秩矩阵的集合是 $\dim N\times\dim M$
  矩阵空间中的开集,所以存在 $\hat p$ 的邻域 $U\subseteq W$ 使得 $F$ 
  在 $U$ 中任意点都满秩,此时 $F|_U$ 为浸没或者浸入。
\end{proof}

\begin{example}[浸没和浸入]\label{exa:submersion and immersion}
  \mbox{}
  \begin{enumerate}
    \item 设 $M_1,\dots,M_k$ 是光滑流形,那么每个投影映射 $\pi_i:M_1\times\cdots\times M_k\to M_i$
    是光滑浸没。这是因为 $\pi_i$ 的某个坐标表示为
    $\hat \pi(x_1,\dots,x_k)=x_i$,其 Jacobi 矩阵满秩。
    \item 如果 $\gamma:J\to M$ 是光滑曲线,那么 $\gamma$ 是光滑浸入当且仅当
    对于任意的 $t\in J$ 有 $\gamma'(t)\neq 0$。
    \item 如果 $M$ 是光滑流形,赋予切丛 $TM$ \autoref{prop:smooth structure of tangent bundle}
    中的光滑结构,那么投影 $\pi:TM\to M$ 是光滑浸没。对于 $M$ 的任意光滑坐标卡
    $(U,(x^i))$ 和对应的 $TM$ 的光滑坐标卡 $(\pi^{-1}(U),(x^i,v^i))$,$\pi$
    的坐标表示为 $\hat \pi(x,v)=x$,其 Jacobi 矩阵是行满秩矩阵。
    \item 定义光滑映射 $X:\mathbb{R}^2\to \mathbb{R}^3$ 为
    \[
      X(u,v)=\bigl(
        (2+\cos2\pi u)\cos2\pi v,(2+\cos2\pi u)\sin 2\pi v,
        \sin 2\pi u
      \bigr),
    \]
    那么 $X$ 的像集是 $yz$-平面的圆 $(y-2)^2+z^2=1$ 绕 $z$-轴旋转一圈得到的
    环面。容易计算 $X$ 在 $(u,v)$ 处的 Jacobi 矩阵为
    \[
      DX(u,v)=2\pi\begin{pmatrix}
        -\cos 2\pi v\sin2\pi u & -\sin2\pi v(2+\cos2\pi u)\\
        -\sin 2\pi v\sin2\pi u & \cos2\pi v(2+\cos2\pi u) \\
        \cos2\pi u & 0
      \end{pmatrix},
    \]
    显然 $\cos2\pi u\neq 0$ 的时候,$DX(u,v)$ 的秩为 $2$。
    当 $\cos2\pi u=0$ 的时候,有
    \[
      DX(u,v)=2\pi\begin{pmatrix}
        -\cos2\pi v & -2\sin 2\pi v\\
        -\sin 2\pi v & 2\cos2\pi v \\
        0 & 0
      \end{pmatrix},
    \]
    此时 $DX(u,v)$ 的一个 2-阶子式不为零,所以秩也为 $2$。
    所以 $DX(u,v)$ 的秩始终为 $2$,即 $X$ 是光滑浸入。
  \end{enumerate}
\end{example}

\subsection{局部微分同胚}

令 $M,N$ 是带边或者无边光滑流形,映射 $F:M\to N$。如果每个 $p\in M$
处都有一个邻域 $U$ 使得 $F(U)$ 是开集并且 $F|_U:U\to F(U)$ 是微分同胚,那么
我们说 $F$ 是\emph{局部微分同胚}。由于 $F$ 满足局部的光滑性,根据
\autoref{prop:smoothness is local},局部微分同胚是光滑映射。

\begin{theorem}[流形上的反函数定理]
  设 $M,N$ 是光滑流形,$F:M\to N$ 是光滑映射。如果 $p\in M$ 使得
  $dF_p$ 可逆,那么存在 $p$ 的连通邻域 $U_0$ 和 $F(p)$ 的连通邻域 $V_0$
  使得 $F|_{U_0}:U_0\to V_0$ 是微分同胚。
\end{theorem}
\begin{proof}
  $dF_p$ 可逆表明 $\dim M=\dim N=n$。取 $p$ 处的光滑坐标卡 $(U,\varphi)$
  和 $F(p)$ 处的光滑坐标卡 $(V,\psi)$ 并且 $F(U)\subseteq V$,
  $\hat F=\psi\circ F\circ\varphi^{-1}:\hat U=\varphi(U)\to \psi(V)=\hat V$
  是光滑函数。微分 $d\hat F_{\hat p}=d\psi_{F(p)}\circ dF_p\circ d(\varphi^{-1})_{\hat p}$
  可逆,Euclid 空间中的反函数定理告诉我们存在 $\hat p$ 的连通邻域 $\hat U_0\subseteq\hat U$
  以及 $\psi(F(p))$ 的连通邻域 $\hat V_0\subseteq \hat V$ 使得 
  $\hat F|_{\hat U_0}:\hat U_0\to\hat V_0$ 是微分同胚,那么 $U_0=\varphi^{-1}(\hat U_0)$
  和 $V_0=\psi^{-1}(\hat V_0)$ 即为我们所需要的。
\end{proof}

\begin{proposition}[局部微分同胚的基本性质]
  \mbox{}
  \begin{enumerate}
    \item 局部微分同胚的复合是局部微分同胚。
    \item 光滑流形之间局部微分同胚的有限积是局部微分同胚。
    \item 局部微分同胚是局部同胚并且是开映射。
    \item 局部微分同胚限制在带边或者无边开子流形上是局部微分同胚。
    \item 微分同胚是局部微分同胚。
    \item 双射的局部微分同胚是微分同胚。
  \end{enumerate}
\end{proposition}
\begin{proof}
  (1) 设 $F:M\to N$ 和 $G:N\to P$ 是局部微分同胚。任取 $p\in M$,
  那么存在 $F(p)$ 的邻域 $V$ 使得 $G(V)$ 是 $P$ 的开集,并且 
  $G|_V:V\to G(V)$ 是微分同胚。$F$ 是局部微分同胚表明存在 $p$
  的邻域 $U$ 使得 $F(U)$ 为 $N$ 的开集并且 $F|_U:U\to F(U)$ 是微分同胚,
  用 $U\cap F^{-1}(V)$ 替代 $U$,可以假设 $U\subseteq F^{-1}(V)$,即
  $F(U)\subseteq V$。那么 $G|_{F(U)}:F(U)\to G(F(U))$ 是微分同胚,故
  $(G\circ F)|_U:U\to G(F(U))$ 是微分同胚,即 $G\circ F$ 是局部微分同胚。
\end{proof}


\begin{proposition}\label{prop:local diffeomorphism}
  设 $M,N$ 是光滑流形,$F:M\to N$ 是映射。
  \begin{enumerate}
    \item $F$ 是局部微分同胚当且仅当 $F$ 同时是光滑浸没以及光滑浸入。
    \item 如果 $\dim M=\dim N$ 并且 $F$ 为光滑浸没或者光滑浸入,那么
    $F$ 是局部微分同胚。
  \end{enumerate}
\end{proposition}
\begin{proof}
  (1) 设 $F$ 是局部微分同胚。任取 $p\in M$,存在 $p$ 的邻域 $U$ 使得 $F(U)$
  为开集并且 $F|_U:U\to F(U)$ 是微分同胚,那么微分 $dF_p:T_pM\to T_{F(p)}N$
  是同构,所以 $\rk F=\dim M=\dim N$,即 $F$ 是光滑浸没以及光滑浸入。

  反之,若 $F$ 是光滑浸没以及光滑浸入,任取 $p\in M$,那么 $dF_p:T_pM\to T_{F(p)}N$
  是同构,根据反函数定理,存在 $p$ 的邻域 $U$ 和 $F(p)$ 的邻域 $V$ 使得
  $F|_U:U\to V$ 是微分同胚,即 $F$ 是局部微分同胚。

  (2) 只需注意到对于相同维数的向量空间之间的线性映射,单射或者满射即可推出同构。
\end{proof}

\begin{example}[局部微分同胚]
  定义映射 $\varepsilon:\mathbb{R}\to\mathbb{S}^1$ 为 $\varepsilon(t)=e^{2\pi it}$,
  $\varepsilon$ 是局部微分同胚,因为其坐标表示为 $\hat\varepsilon(t)=2\pi t+c$
  是微分同胚。  
\end{example}





\subsection{秩定理}

\begin{theorem}[秩定理]
  设 $M,N$ 分别是 $m$ 维和 $n$ 维光滑流形,$F:M\to N$ 是秩 $r$ 的光滑映射。
  对于每个 $p\in M$,存在以 $p$ 为中心的光滑坐标卡 $(U,\varphi)$ 和以
  $F(p)$ 为中心的光滑坐标卡 $(V,\psi)$ 使得 $F(U)\subseteq V$,并且
  $F$ 的坐标表示满足
  \[
    \hat F\left(x^1,\dots,x^r,x^{r+1},\dots,x^m\right) =
    \left(x^1,\dots,x^r,0,\dots,0\right).
  \]
  特别地,如果 $F$ 是光滑浸没,那么
  \[
    \hat F\left(x^1,\dots,x^n,x^{n+1},\dots,x^m\right) =
    \left(x^1,\dots,x^n\right),
  \]
  如果 $F$ 是光滑浸入,那么
  \[
    \hat F\left(x^1,\dots,x^m\right) =
    \left(x^1,\dots,x^m,0,\dots,0\right).
  \]
\end{theorem}
\begin{proof}
  该定理的叙述是局部的,所以在选定光滑坐标卡后,我们可以将 $M,N$
  替换为开集 $U\subseteq\mathbb{R}^m$ 和开集 $V\subseteq\mathbb{R}^n$。
  $DF(p)$ 的秩为 $r$ 表明其存在某个 $r\times r$ 子矩阵的行列式不为零。
  通过调整坐标的顺序,我们假定其左上角的子矩阵的行列式不为零,即
  $\left(\partial F^i/\partial x^j\right)$,其中 $1\leq i,j\leq r$。
  记 $\mathbb{R}^m$ 中的标准坐标为 $(x,y)=(x^1,\dots,x^r,y^1,\dots,y^{m-r})$,
  $\mathbb{R}^n$ 中的标准坐标为 $(v,w)=(v^1,\dots,v^r,w^1,\dots,w^{n-r})$。
  通过平移坐标系,我们可以假设 $p=(0,0)$ 以及 $F(p)=(0,0)$。将 
  $F(x,y)$ 分解为 $F(x,y)=\left(Q(x,y),R(x,y)\right)$,其中
  $Q:U\to\mathbb{R}^r$ 和 $R:U\to\mathbb{R}^{n-r}$ 是光滑函数。
  此时 $\left(\partial Q^i/\partial x^j(0,0)\right)$ 是可逆矩阵。

  定义 $\varphi:U\to\mathbb{R}^m$ 为 $\varphi(x,y)=\left(Q(x,y),y\right)$,那么
  \[
    D\varphi(0,0)=\begin{pmatrix}
      \dfrac{\partial Q^i}{\partial x^j}(0,0) &  \dfrac{\partial Q^i}{\partial y^j}(0,0)\\[4mm]
      0 & I_{m-r}
    \end{pmatrix}  .
  \]
  显然 $D\varphi(0,0)$ 可逆。根据反函数定理,存在 $(0,0)$ 的连通邻域 $U_0$
  和 $\varphi(0,0)=(0,0)$ 的连通邻域 $\tilde{U}_0$ 使得 $\varphi:U_0\to\tilde{U}_0$
  是微分同胚。通过缩小 $U_0$ 和 $\tilde{U}_0$,我们假设 $\tilde{U}_0$ 是开立方体。
  记逆映射 $\varphi^{-1}:\tilde{U}_0\to U_0$ 为
  $\varphi^{-1}(x,y)=\left(A(x,y),B(x,y)\right)$,其中 $A:\tilde{U}_0\to\mathbb{R}^r$
  和 $B:\tilde{U}_0\to\mathbb{R}^{m-r}$ 是光滑函数,那么
  \[
    (x,y)=\varphi\left(A(x,y),B(x,y)\right)=\left(Q\bigl(A(x,y),B(x,y)\bigr),B(x,y)\right),
  \]
  故 $B(x,y)=y$。因此 $\varphi^{-1}$ 满足
  \[
    \varphi^{-1}(x,y)=\left(A(x,y),y\right)  .
  \]
  另一方面,对比分量 $x$,有 $Q\left(A(x,y),y\right)=x$,因此
  \[
    F\circ\varphi^{-1}(x,y)=\left(x,\tilde{R}(x,y)\right)  ,
  \]
  其中 $\tilde{R}:\tilde{U}_0\to\mathbb{R}^{n-r}$ 满足
  $\tilde{R}(x,y)=R\left(A(x,y),y\right)$。$F\circ\varphi^{-1}$ 在
  任意 $(x,y)\in\tilde{U}_0$ 处的 Jacobi 矩阵为
  \[
    D\left(F\circ\varphi^{-1}\right)(x,y)=\begin{pmatrix}
      I_r & 0 \\[2mm]
      \dfrac{\partial \tilde{R}^i}{\partial x^j}(x,y) & 
      \dfrac{\partial \tilde{R}^i}{\partial y^j}(x,y)
    \end{pmatrix}.
  \]
  由于微分同胚不改变映射的秩,所以上述矩阵的秩为 $r$,这表明
  $\partial\tilde{R}^i/\partial y^j(x,y)$ 为零矩阵,所以 $\tilde{R}$
  的取值实际上与 $y$ 无关,不妨设 $S(x)=\tilde{R}(x,0)$ (这里 $\tilde{U}_0$ 是开立方体保证 $\tilde{R}(x,0)$ 有定义),
  那么
  \[
    F\circ\varphi^{-1}(x,y)=\left(x,S(x)\right)  .
  \] 

  令 $V_0\subseteq V$ 为 $V_0=\left\{(v,w)\in V\,\middle|\, (v,0)\in\tilde{U}_0\right\}$,
  那么 $V_0$ 是包含 $(0,0)$ 的开集。此时 $F\circ\varphi^{-1}\left(\tilde{U}_0\right)\subseteq V_0$。
  定义 $\psi:V_0\to\mathbb{R}^n$ 为 $\psi(v,w)=(v,w-S(v))$,这是到其像集的微分同胚,
  因为有逆映射 $\psi^{-1}(s,t)=(s,t+S(s))$,所以 $(V_0,\psi)$ 是光滑坐标卡。
  那么
  \[
    \psi\circ F\circ\varphi^{-1}(x,y)=\psi(x,S(x))=(x,S(x)-S(x))=(x,0) .
  \]
  这就完成了证明。
\end{proof}

\begin{corollary}
  令 $M,N$ 是光滑流形,$F:M\to N$ 是光滑映射,且 $M$ 是连通空间。那么下面的说法等价:
  \begin{enumerate}
    \item 对于每个 $p\in M$ 都存在包含 $p$ 的光滑坐标卡和包含 $F(p)$ 的光滑坐标卡,
    使得 $F$ 的坐标表示是线性映射。
    \item $F$ 是常秩的。
  \end{enumerate}
\end{corollary}
\begin{proof}
  $(2)\Rightarrow (1)$ 即秩定理。
  $(1)\Rightarrow (2)$ 由于线性映射是常秩的,所以 $F$ 在每个点 $p$ 处都有一个邻域
  使得 $F$ 在这个邻域上为常秩,$M$ 的连通性表明 $F$ 在 $M$ 上为常秩。
\end{proof}

\begin{theorem}[全局秩定理]
  令 $M,N$ 是光滑流形,$F:M\to N$ 是常秩光滑映射。
  \begin{enumerate}
    \item 若 $F$ 是满射,那么 $F$ 是光滑浸没。
    \item 若 $F$ 是单射,那么 $F$ 是光滑浸入。
    \item 若 $F$ 是双射,那么 $F$ 是微分同胚。
  \end{enumerate}
\end{theorem}
\begin{proof}
  记 $m=\dim M$,$n=\dim N$,设 $F$ 有常秩 $r$。(a)
  若 $F$ 是满射,假设 $F$ 不是光滑浸没,那么 $r<n$。根据秩定理,存在
  以 $p$ 为中心的光滑坐标卡 $(U,\varphi)$ 和以 $F(p)$ 为中心的光滑坐标卡
  $(V,\psi)$ 使得 $F(U)\subseteq V$ 并且 $F$ 的坐标表示为
  \[
    \hat F\left(x^1,\dots,x^r,x^{r+1},\dots,x^m\right) =
    \left(x^1,\dots,x^r,0,\dots,0\right).
  \]
  通过适当缩小 $U$,我们假设 $U$ 是一个正则坐标球并且 $F(\bar U)\subseteq V$。
  这表明 $F(\bar U)$ 是集合 $\left\{y\in V\,\middle|\, y^{r+1}=\cdots=y^n=0\right\}$
  ($N$ 的闭子集且不包含 $N$ 的任意非空开子集) 的紧子集,所以 $F(\bar U)$ 是 $N$ 的闭集且不包含 $N$ 的任何非空开子集,
  所以
  \[
    \overline{N\smallsetminus F(\bar U)}=N\smallsetminus \Int F(\bar U)=
    N\smallsetminus\emptyset=N,
  \]
  故 $F(\bar U)$ 在 $N$ 中无处稠密。因为流形的任意开覆盖都有可数子覆盖,所以我们可以
  选取可数个这样的坐标卡 $\{(U_i,\varphi_i)\}$ 覆盖 $M$,其对应的坐标卡 $\{(V_i,\psi_i)\}$
  覆盖 $F(M)$。因为
  \[
    F(M)=\bigcup_{i=1}^\infty F(\bar U),  
  \]
  根据 Baire 纲定理,所以 $F(M)$ 在 $N$ 中的内部为空集,这与 $F$ 是满射矛盾。

  (b) 若 $F$ 是单射,假设 $F$ 不是光滑浸入,那么 $r<m$。根据秩定理,
  存在
  以 $p$ 为中心的光滑坐标卡 $(U,\varphi)$ 和以 $F(p)$ 为中心的光滑坐标卡
  $(V,\psi)$ 使得 $F(U)\subseteq V$ 并且 $F$ 的坐标表示为
  \[
    \hat F\left(x^1,\dots,x^r,x^{r+1},\dots,x^m\right) =
    \left(x^1,\dots,x^r,0,\dots,0\right),
  \]
  那么对于任意小的 $\varepsilon$ 都有 $\hat F(0,\dots,0,\varepsilon)=(0,\dots,0,0)$,
  这与 $F$ 是单射矛盾。

  (c) 根据 (a) 和 (b),$F$ 同时是光滑浸没以及光滑浸入,根据 \autoref{prop:local diffeomorphism},
  $F$ 是局部微分同胚,双射的局部微分同胚是微分同胚。
\end{proof}

\section{嵌入}

如果 $M,N$ 是带边或者无边光滑流形,若 $F:M\to N$ 同时是光滑浸入以及
拓扑嵌入,那么我们说 $F$ 是\emph{$M$ 到 $N$ 的光滑嵌入}。

\begin{example}[光滑嵌入]
  \mbox{}
  \begin{enumerate}
    \item $M$ 是带边或者无边光滑流形,$U\subseteq M$ 是开子流形,那么
    包含映射 $\iota:U\hookrightarrow M$ 是光滑嵌入。
    \item 如果 $M_1,\dots,M_k$ 是光滑流形,$p_i\in M_i$ 是任意点,
    定义 $\iota_j:M_j\to M_1\times\cdots\times M_k$ 为
    \[
      \iota_j(q)=(p_1,\dots,p_{j-1},q,p_{j+1},\dots,p_k),  
    \]
    那么 $\iota_j$ 是光滑嵌入。
    \item \autoref{exa:submersion and immersion} 的 (4) 定义的映射
    $X:\mathbb{R}^2\to \mathbb{R}^3$ 可以下降为一个 $\mathbb{S}^1\times \mathbb{S}^1$
    到 $\mathbb{R}^3$ 的光滑嵌入。
  \end{enumerate}
\end{example}

\begin{example}[光滑的拓扑嵌入]
  定义映射 $\gamma:\mathbb{R}\to\mathbb{R}^2$ 为 $\gamma(t)=(t^3,0)$ 是光滑映射
  并且是拓扑嵌入,但是其不是光滑嵌入,因为 $\gamma'(0)=(0,0)$ 不满秩。
\end{example}

\begin{example}[八字曲线]\label{exa:eight-curve}
  考虑曲线 $\beta:(-\pi,\pi)\to\mathbb{R}^2$ 为
  \[
    \beta(t)=(\sin 2t,\sin t) . 
  \]
  $\beta$ 的像集也被称为\emph{双扭线}($x^2=4y^2(1-y^2)$)。
  由于 $\beta'(t)=(2\cos 2t,\cos t)\neq (0,0)$,所以 $\beta$ 是单射的光滑浸入。
  但是 $\beta$ 不是拓扑嵌入,因为其像集在子空间拓扑下是紧集,但是
  $(-\pi,\pi)$ 不是紧集。
\end{example}

\begin{example}[环面上的稠密曲线]\label{exa:dense curve on torus}
  令 $\mathbb{T}^2=\mathbb{S}^1\times\mathbb{S}^1\subseteq\mathbb{C}^2$ 是环面,
  $\alpha$ 是任意无理数。定义映射 $\gamma:\mathbb{R}\to\mathbb{T}^2$ 为
  \[
    \gamma(t)=\left(e^{2\pi it},e^{2\pi i\alpha t}\right),
  \]
  由于 $\gamma'(t)$ 始终不为零,所以 $\gamma$ 是光滑浸入。若 $\gamma(t_1)=\gamma(t_2)$,
  那么 $t_1-t_2$ 和 $\alpha(t_1-t_2)$ 同时为整数,这只能表明 $t_1=t_2$,所以 $\gamma$ 是单射。

  考虑集合 $\gamma(\mathbb{Z})$。根据 Dirichlet 逼近定理,对于任意的 $\varepsilon>0$,
  存在整数 $n,m$ 使得 $\abs{\alpha n-m}<\varepsilon$。使用不等式
  $\abs{e^{it_1}-e^{it_2}}\leq\abs{t_1-t_2}$,其中 $t_1,t_2\in\mathbb{R}$ (这是因为从 $e^{it_1}$ 到 $e^{it_2}$ 的线段长度小于等于圆弧长度),
  我们有 $\abs{e^{2\pi i\alpha n}-1}=\abs{e^{2\pi i\alpha n}-e^{2\pi i m}}\leq \abs{2\pi(\alpha n-m)}<2\pi \varepsilon$,
  因此,
  \[
    \abs{\gamma(n)-\gamma(0)}=\abs{\left(e^{2\pi i n},e^{2\pi i \alpha n}\right)-(1,1)}
    =\abs{\left(1,e^{2\pi i\alpha n}\right)-(1,1)}<2\pi\varepsilon , 
  \]
  所以 $\gamma(0)$ 是 $\gamma(\mathbb{Z})$ 的极限点。这意味着 $\gamma$ 并不同胚于它的像,
  因为 $\mathbb{Z}$ 在 $\mathbb{R}$ 中没有任何极限点。实际上,可以证明像集 $\gamma(\mathbb{R})$
  在 $\mathbb{T}^2$ 中稠密。这表明单射的光滑浸入也不一定是光滑嵌入。
\end{example}

下面的命题给出了一个判断单射的浸入为嵌入的充分条件。

\begin{proposition}
  设 $F:M\to N$ 是带边或者无边光滑流形,$F:M\to N$ 是单射的光滑浸入,如果 $F$
  满足下列条件之一,那么 $F$ 是光滑嵌入。
  \begin{enumerate}
    \item $F$ 是开映射或者闭映射。
    \item $F$ 是恰当映射。
    \item $M$ 是紧空间。
    \item $M$ 有空的边界并且 $\dim M=\dim N$。
  \end{enumerate}
\end{proposition}
\begin{proof}
  若 $F$ 为开映射或者闭映射,那么 $F$ 是拓扑嵌入,进而是光滑嵌入。
  (2) 和 (3) 都能推出 $F$ 是闭映射。对于 (4),$\dim M=\dim N$ 表明
  $dF_p$ 可逆,$M$ 边界为空表明 $F(M)\subseteq\Int N$,
  \autoref{prop:local diffeomorphism} 表明 $F:M\to\Int N$ 是局部微分同胚,
  从而是开映射。$F:M\to N$ 是开映射的复合 $M\to\Int N\hookrightarrow N$,
  所以 $F$ 是光滑嵌入。
\end{proof}

\begin{theorem}[局部嵌入定理]\label{thm:local embedding}
  设 $M,N$ 是带边或者无边光滑流形,$F:M\to N$ 是光滑映射。那么 $F$
  是光滑浸入当且仅当在 $M$ 的每个点处,都存在一个邻域 $U$ 使得
  $F|_U:U\to N$ 是光滑嵌入。
\end{theorem}
\begin{proof}
  若 $F$ 是光滑浸入。任取点 $p\in M$,根据秩定理,存在包含 $p$
  的光滑坐标卡 $(U_1,\varphi)$ 和包含 $F(p)$ 的光滑坐标卡 $(V,\psi)$,
  使得 $F(U_1)\subseteq V$ 以及 $F$ 的坐标表示为
  \[
    \psi\circ F\circ\varphi^{-1}\left(x^1,\dots,x^m\right)=
    \left(x^1,\dots,x^m,0,\dots,0\right),
  \]
  此时 $\psi\circ F\circ\varphi^{-1}$ 是单射,所以 $F|_{U_1}$ 是单射。
  取 $p$ 的一个预紧的邻域 $U$,并且 $U$ 满足 $\bar U\subseteq U_1$。
  那么 $F|_{\bar U}:\bar U\to N$ 是紧空间到 Hausdorff 空间的连续映射,从而
  是闭映射,又因为 $F|_{\bar U}$ 是单射,所以 $F|_{\bar U}$ 是拓扑嵌入,
  于是 $F|_U$ 作为 $F|_{\bar U}$ 的限制也是拓扑嵌入。显然 $F|_U$
  是光滑浸入,所以 $F|_U$ 是光滑嵌入。

  反之,若在 $M$ 的每个点 $p$ 处,都存在一个邻域 $U$ 使得 $F|_U:U\to N$ 是光滑嵌入。
  这表明 $d(F|_U)_p=d(F\circ\iota)_p=dF_{p}\circ d\iota_p$ 是单射,而
  $d\iota_p:T_pU\to T_pM$ 是同构,所以 $dF_p$ 是单射,所以 $F$ 是光滑浸入。 
\end{proof}


\section{浸没}

如果 $\pi:M\to N$ 是连续映射,定义\emph{$\pi$ 的截面}是 $\pi$ 的连续右逆,
即一个连续映射 $\sigma:N\to M$ 使得 $\pi\circ\sigma=\Id_N$。
定义\emph{$\pi$ 的局部截面}是连续映射 $\sigma:U\to M$,其中 $U\subseteq N$
是开集并且满足 $\pi\circ\sigma=\Id_U$。下面的定理表明,对于光滑浸没而言,
其在值域上的局部行为类似满射。

\begin{theorem}[局部截面定理]
  设 $M,N$ 是光滑流形,$\pi:M\to N$ 是光滑映射。那么 $\pi$ 是光滑浸没当且仅当
  $M$ 的每个点都在 $\pi$ 的某个光滑局部截面的像集中。
\end{theorem}
\begin{proof}
  若 $\pi$ 是光滑浸没。任取 $p\in M$,根据秩定理,存在以 $p$ 为中心的光滑坐标卡
  $(U,\varphi)$ 和以 $\pi(p)$ 为中心的光滑坐标卡 $(V,\psi)$ 使得
  $\pi(U)\subseteq V$ 并且有坐标表示
  \[
    \psi\circ\pi\circ\varphi^{-1}\left(x^1,\dots,x^m\right)=\left(x^1,\dots,x^n\right).
  \]
  对于 $\varepsilon>0$,记 $\mathbb{R}^m$ 中的开立方体
  \[
    \hat C_\varepsilon=\bigl\{\,x\bigm| \abs{x^i}<\varepsilon,i=1,\dots,m\,\bigr\},
  \]
  我们可以让 $\varepsilon$ 足够小使得 $\hat C_\varepsilon\subseteq \varphi(U)$,
  记 $C_\varepsilon=\varphi^{-1}\left(\hat C_\varepsilon\right)\subseteq U$。
  于是 
  \[ 
    \psi\circ\pi(C_\varepsilon)=\bigl\{\,y\bigm| \abs{y^i}<\varepsilon,i=1,\dots,n\,\bigr\}
  \]
  是 $\mathbb{R}^n$ 中的开立方体。记 $C_\varepsilon'=\psi^{-1}\left(\psi\circ \pi(C_\varepsilon)\right)$
  是 $N$ 的开集。令 $\sigma:C_\varepsilon'\to C_\varepsilon$,其坐标表示满足
  \[
    \varphi\circ\sigma\circ\psi^{-1}\left(x^1,\dots,x^n\right)  
    =\left(x^1,\dots,x^n,0,\dots,0\right),
  \]
  那么 $\pi\circ\sigma=\Id_{C_\varepsilon'}$,且 $p\in C_\varepsilon$ 表明
  $p\in\im\sigma$。

  反之,若任取 $p\in M$,存在 $\pi$ 的光滑局部截面 $\sigma:U\to M$ 使得 $p\in\im\sigma$。
  设 $q\in U$ 使得 $p=\sigma(q)$。那么 $\pi\circ\sigma=\Id_U$ 表明
  \[
    \Id_{T_qN}=d\left(\Id_U\right)_q= d\pi_{p}\circ d\sigma_q,
  \]
  即 $d\pi_p$ 是满射,所以 $\pi$ 是光滑浸没。
\end{proof}

\begin{proposition}[光滑浸没的性质]
  令 $M,N$ 是光滑流形,$\pi:M\to N$ 是光滑浸没。那么 $\pi$ 是开映射,
  进一步的,如果 $\pi$ 是满射,那么 $\pi$ 是商映射。
\end{proposition}
\begin{proof}
  设 $W\subseteq M$ 是开集,任取 $q=\pi(p)\in \pi(W)$,即 $p\in W$,
  根据局部截面定理,存在 $\pi$ 的光滑局部截面 $\sigma:U\to M$
  使得 $p\in \im\sigma$,设 $q'\in U$ 使得 $p=\sigma(q')$,那么
  $q=\pi(p)=\pi(\sigma(q'))=q'$,所以 $\sigma(q)=p\in W$,所以 $q\in \sigma^{-1}(W)$。
  任取 $y\in\sigma^{-1}(W)$,那么 $\sigma(y)\in W$,即 $y=\pi(\sigma(y))\in \pi(W)$,
  所以 $q\in\sigma^{-1}(W)\subseteq \pi(W)$,这表明 $\pi(W)$
  是开集,即 $\pi$ 是开映射。
\end{proof}

下面的定理反映了光滑浸没的重要特征,即满射的光滑浸没在光滑流形中
的行为类似于商映射在拓扑学中的行为。

\begin{theorem}[满射光滑浸没的刻画]\label{thm:character of surjective submersion}
  设 $M,N$ 是光滑流形,$\pi:M\to N$ 是满射的光滑浸没。对于任意
  带边或者无边光滑流形 $P$,映射 $F:N\to P$ 光滑当且仅当 $F\circ\pi$
  光滑:
  \[
    \begin{tikzcd}[sep=large]
      M\arrow[dr,"F\circ\pi"]\arrow[d,"\pi"'] & \\
      N\arrow[r,"F"'] & P.
    \end{tikzcd}
  \]
\end{theorem}
\begin{proof}
  如果 $F$ 光滑,那么 $F\circ\pi$ 作为光滑映射的复合也光滑。
  如果 $F\circ\pi$ 光滑,任取 $q\in N$,设 $p\in M$ 使得 $\pi(p)=q$,
  根据局部截面定理,存在 $\pi$ 的光滑局部截面 $\sigma:U\to M$
  使得 $p\in \sigma(U)$,此时 $F|_U=F\circ \pi\circ\sigma$
  是光滑映射,这表明 $F$ 是局部光滑的,所以 $F$ 光滑。
\end{proof}

\begin{theorem}\label{thm:pass to the quotient}
  设 $M,N$ 是光滑流形并且 $\pi:M\to N$ 是满的光滑浸没,如果 $P$
  是带边或者无边光滑流形且 $F:M\to P$ 是在 $\pi$ 的每个纤维上为常值
  的光滑映射,那么存在唯一的光滑映射 $\wtilde F:N\to P$ 使得
  $\wtilde F\circ\pi=F$:
  \[
    \begin{tikzcd}[sep=large]
      M\arrow[dr,"F"]\arrow[d,"\pi"'] & \\
      N\arrow[r,"\wtilde F"',dashed] & P.
    \end{tikzcd}
  \]
\end{theorem}
\begin{proof}
  由于满的光滑浸没是商映射,所以存在唯一的连续映射 $\wtilde F:N\to P$
  使得 $\wtilde F\circ\pi=F$。根据 \autoref{thm:character of surjective submersion},
  $F$ 光滑表明 $\wtilde F$ 光滑。
\end{proof}

\begin{theorem}
  设 $M,N_1,N_2$ 是光滑流形,$\pi_1:M\to N_1$ 和 $\pi_2:M\to N_2$
  是满的光滑浸没,并且 $\pi_1,\pi_2$ 在对方的每个纤维上是常值映射,
  那么存在唯一的微分同胚 $F:N_1\to N_2$ 使得 $F\circ \pi_1=\pi_2$。
\end{theorem}
\begin{proof}
  根据 \autoref{thm:pass to the quotient},存在光滑映射 $F:N_1\to N_2$
  满足 $F\circ\pi_1=\pi_2$ 以及光滑映射 $G:N_2\to N_1$ 满足
  $G\circ\pi_2=\pi_1$,所以 $F\circ G\circ\pi_2=\pi_2$,
  那么 $F\circ G=\Id_{N_2}$,类似的有 $G\circ F=\Id_{N_1}$,所以
  $F$ 是微分同胚。
\end{proof}

\section{光滑覆盖映射}

回顾拓扑空间之间的覆盖映射。即一个连通的、局部道路连通空间之间的连续满射 $\pi:E\to M$,
同时使得 $M$ 的每个点都有一个邻域 $U$ 使得 $U$ 是均匀覆盖,即 $\pi^{-1}(U)$ 的每个连通分支
都通过 $\pi$ 同胚于 $U$。

在流形中,如果 $E,M$ 都是连通的带边或者无边光滑流形,如果 $\pi:E\to M$ 是光滑满射,并且 $M$
中的每个点都有一个邻域 $U$ 使得 $\pi^{-1}(U)$ 的每个连通分支都通过 $\pi$ 微分同胚于 $U$,
那么我们说 $\pi$ 是\emph{光滑覆盖映射},同时也说 $U$ 是均匀覆盖。$M$ 被称为\emph{覆盖的底空间},
$E$ 被称为\emph{$M$ 的覆盖流形}。如果 $E$ 是单连通的,那么 $E$ 被称为\emph{$M$ 的万有覆盖流形}。





