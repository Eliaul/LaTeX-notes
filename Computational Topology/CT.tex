\documentclass[fontset=none]{Notes}

\makeatletter
\DeclareRobustCommand{\em}{%
  \@nomath\em \if b\expandafter\@car\f@series\@nil
  \normalfont \else \bfseries \fi}
\makeatother

\usepackage{tikz-cd,wrapstuff}
\usepackage{siunitx,tikz,nicematrix,subcaption}
\usetikzlibrary{matrix,calc}
\usetikzlibrary{intersections}
\usetikzlibrary{arrows.meta}
\usetikzlibrary{decorations.markings}

\ProvidesFile{font.def}

\setCJKmainfont{Source Han Serif SC}[
  UprightFont=*-Regular,
  BoldFont=*-Bold,
  ItalicFont=HYKaiTi S,
  ItalicFeatures={Scale=1.1}
]
\newCJKfontfamily[zhsong]\songti{Source Han Serif SC}[
  UprightFont=*-Regular,
  BoldFont=*-Bold,
  ItalicFont=HYKaiTi S,
  ItalicFeatures={Scale=1.1}
]
\setCJKsansfont{Source Han Sans SC}[
  UprightFont=*-Regular,
  BoldFont=*-Bold
]
\newCJKfontfamily[zhhei]\heiti{Source Han Sans SC}[
  UprightFont=*-Regular,
  BoldFont=*-Bold
]
\setCJKmonofont{HYFangSong S}[
  BoldFont=*,
  ItalicFont=*,
  BoldItalicFont=*
]
\newCJKfontfamily[zhfs]\fangsong{HYFangSong S}[
  BoldFont=*,
  ItalicFont=*,
  BoldItalicFont=*
]
\newCJKfontfamily[zhkai]\kaishu{HYKaiTi S}[
  BoldFont=*,
  ItalicFont=*,
  BoldItalicFont=*
]

\setmainfont{texgyretermes}[
  Extension=.otf,
  UprightFont=*-regular,
  BoldFont=*-bold,
  ItalicFont=*-italic,
  BoldItalicFont=*-bolditalic,
  SlantedFont=*-italic
]
%\setmathrm{texgyretermes}[
%  Extension=.otf,
%  UprightFont=*-regular,
%  BoldFont=*-bold,
%  ItalicFont=*-italic,
%  BoldItalicFont=*-bolditalic,
%  SlantedFont=*-italic
%]
\setsansfont{Cantarell}[
  UprightFont=* Regular,
  ItalicFont=* Italic,
  BoldFont=* Bold,
  BoldItalicFont=* Bold Italic,
  SmallCapsFont=Alegreya Sans SC
]
\setmonofont{Ubuntu Mono}[
  UprightFont=*,
  ItalicFont=* Italic,
  BoldFont=* Bold,
  BoldItalicFont=* Bold Italic
]
%\setmathfont{texgyretermes-math.otf}
%\setmathfont[range={\mathcal,\mathbfcal,\mathfrak},StylisticSet=1]{XITSMath-Regular.otf}
%\setmathfont[range={\mathbb}]{KpMath-Sans.otf}



\usepackage[subscriptcorrection,nofontinfo,mtpbb,mtpfrak]{mtpro2}
\usepackage[normal]{fixdif}

\tikzcdset{
  arrow style=tikz,
  diagrams={>={Straight Barb[scale=0.8]}}
}

\tikzset{
  every picture/.style={
    thick,
    >={Latex[width=6pt, length=8pt]}
  },
  point/.style={
    circle, fill, minimum width=5pt,
    inner sep=0pt
  },
  straight arrow/.style={
    Straight Barb[scale=0.8]
  }
}

\allowdisplaybreaks[1]

\newlength{\mymathln}
\newcommand{\aligninside}[2]{
  \settowidth{\mymathln}{#2}
  \mathmakebox[\mymathln]{#1}
}

\DeclareMathOperator\Spec{Spec}
\DeclareMathOperator\im{im}
\DeclareMathOperator\sgn{sgn}
\DeclareMathOperator\rad{rad}
\DeclareMathOperator\Alt{Alt}
\DeclareMathOperator\Max{Max}
\DeclareMathOperator\card{card}
\DeclareMathOperator\GL{GL}
\DeclareMathOperator\Orth{O}
\DeclareMathOperator\SO{SO}
\DeclareMathOperator\SU{SU}
\DeclareMathOperator\cls{cls}
\DeclareMathOperator\Lie{Lie}
\DeclareMathOperator\End{End}
\DeclareMathOperator\Int{Int}
\DeclareMathOperator\Sym{Sym}
\DeclareMathOperator\tr{tr}
\DeclareMathOperator\Hom{Hom}
\DeclareMathOperator\supp{supp}
\DeclareMathOperator\Id{Id}
\DeclareMathOperator\rk{rank}
\DeclareMathOperator\grad{grad}
\DeclareMathOperator\rank{rank}
\DeclareMathOperator\Euc{E}
\DeclareMathOperator\ob{ob}
\DeclareMathOperator\diam{diam}
\DeclareMathOperator\rel{rel}
\DeclareMathOperator\inte{int}
\DeclareMathOperator\bd{bd}
\DeclareMathOperator\St{St}
\DeclareMathOperator\Lk{Lk}
\DeclareMathOperator\Ver{Vert}
\newcommand{\LL}{{\mathrm{L}}}

\newcommand{\norm}[1]{\left\lVert#1\right\rVert}
\newcommand{\mat}[1]{\mathbold{#1}}
\newcommand{\cat}[1]{\mathsf{#1}}
\newcommand{\uline}{\underline{\hphantom{X}}}
\newcommand{\abs}[1]{\left|#1\right|}
\newcommand{\lie}[1]{\mathfrak{#1}}
\newcommand{\inn}[1]{\left\langle #1\right\rangle}
\newcommand{\partI}{\partial I}
\newcommand{\relhomo}{\rel\partI}

\usepackage{enumitem}

\setlist[enumerate]{nosep}

%\DeclareMathAlphabet\mathcal{OMS}{cmsy}{m}{n}

\newlength\stextwidth
\newcommand\makesamewidth[3][c]{%
  \settowidth{\stextwidth}{#2}%
  \makebox[\stextwidth][#1]{#3}%
}



\begin{document}

\frontmatter

\tableofcontents

\mainmatter

\chapter{复形}

\section{单纯复形}

\paragraph{单纯形}
令 $u_0,u_1,\dots,u_k$ 是 $\mathbb{R}^d$ 中的点。一个点 $x=\sum_{i=0}^k\lambda_i u_i$
被称为 $u_i$ 的\emph{仿射组合},如果 $\sum \lambda_i=1$。仿射组合的几何被称为 \emph{仿射包}。
$k+1$ 个点如果满足 $u_i-u_0\ (1\leq i\leq k)$ 是线性无关的,那么说它们是\emph{仿射无关}的。
在 $\mathbb{R}^d$ 中最多有 $d$ 个线性无关的向量,所以最多有 $d+1$ 个放射无关的点。

仿射组合 $x=\sum\lambda_i u_i$ 的所有系数如果满足 $\lambda_i\geq 0$,那么说这是一个\emph{凸组合}。
凸组合的集合被称为\emph{凸包}。 $k+1$ 个仿射无关点的凸包被称为 \emph{$k$-单纯形},记为
$\sigma=[u_0,u_1,\dots,u_k]$。$\sigma$ 的\emph{面}指的是 $\{u_0,\dots,u_k\}$
的某个非空子集的凸包。如果 $\tau$ 是 $\sigma$ 的面,我们记作 $\tau\leq\sigma$,
如果 $\tau$ 是恰当的,那么记作 $\tau<\sigma$。显然,$\sigma$ 有 $2^{k+1}-1$
个面。$\sigma$ 的所有恰当面的并集被称为 $\sigma$ 的\emph{边界},
记为 $\bd\sigma$。$\sigma$ 的内部定义为 $\inte\sigma=\sigma\setminus \bd\sigma$。
点 $x\in\sigma$ 在 $\sigma$ 的内部当且仅当所有的系数 $\lambda_i$ 均为正数。
可以发现每个点 $x\in\sigma$ 都属于某个面的内部,即正系数 $\lambda_i$ 对应的所有 $u_i$
张成的凸包。

\paragraph{单纯复形}
一个\emph{单纯复形} $K$ 指的是有限个单纯形的集合,其满足:
若 $\sigma\in K$ 和 $\tau\leq\sigma$,则 $\tau\in K$;并且
$\sigma,\sigma_0\in K$ 表明 $\sigma\cap\sigma_0$ 要么是空集要么
是 $\sigma$ 和 $\sigma_0$ 的公共面。 

\begin{figure}[htb]
  \centering
  \includegraphics[width=.3\linewidth]{figures/Simplicial_complex_example.png}
  \caption{一个 $3$-维的单纯复形。}
\end{figure}


$K$ 的\emph{维数}被定义为 $K$ 中单纯形的最大维数。
$K$ 的\emph{底空间} $|K|$ 定义为 $K$ 中所有单纯形的并集,
并且继承 $\mathbb{R}^d$ 的子空间拓扑。一个\emph{多面体}指的是
对于一个拓扑空间 $X$,如果 $X$ 和 $|K|$ 同胚,那么我们说 $X$
的\emph{三角化}是单纯复形 $K$ 附带这个同胚。如果拓扑空间有一个三角化,
那么说这个空间是\emph{可三角化的}。 $K$ 的\emph{子复形}
指的是一个单纯复形 $L\subseteq K$。如果 $L$ 包含了 $L$ 的顶点
在 $K$ 中张成的所有单纯形,那么说 $L$ 是\emph{满}的。$K$ 的\emph{$j$-骨架}
指的是由所有维数小于等于 $j$ 的单纯形构成的子复形,即
$K^{(j)}=\{\sigma\in K\,|\, \dim\sigma\leq j\}$。$0$-骨架也被称为\emph{顶点集}。
对于 $K$ 中的单纯形 $\tau$,定义所有以 $\tau$ 作为面的单纯形的集合,
称为 $\tau$ 的\emph{星形},记为 $\St\tau=\{\sigma\in K\,|\,\tau\leq\sigma\}$。
一般来说,星形在取面的时候不一定封闭,我们可以把丢失的面加进来使其称为一个复形。
这个结果被称为\emph{闭星形},记为 $\overline{\St}\tau$,是包含星形的最小的子复形。
一般的,对于 $K$ 的一个子集 $S$,总可以定义 $S$ 的闭包 $\wbar{S}$ 是包含 $S$
的最小的 $K$ 的子复形。$\tau$ 的\emph{链环}定义为 $\Lk\tau=\{\nu\in\overline{\St}\,\tau\,|\,\nu\cap\tau=\emptyset\}$,
等价的说,也有 $\Lk\tau=\overline{\St}\,\tau\setminus \St\bar\tau$。

\begin{figure}[htb]
  \centering
  \subcaptionbox{单纯形集合的闭包}[.33\linewidth]{
    \includegraphics[width=\linewidth]{figures/closure of simplices.png}
  }%
  \subcaptionbox{一个顶点的星形}[.33\linewidth]{
    \includegraphics[width=\linewidth]{figures/star of vertex.png}
  }%
  \subcaptionbox{一个顶点的链环}[.33\linewidth]{
    \includegraphics[width=\linewidth]{figures/link of vertex.png}
  }
  \caption{闭包、星形和链环。}
\end{figure}

\paragraph{抽象的单纯复形}
通常来说更容易去抽象地构造一个单纯复形,而不用担心如何把它放进
欧式空间。

一个\emph{抽象单纯复形}指的是一个由有限个集合组成的集合族 $A$,
满足:$\alpha\in A$ 和 $\beta\subseteq A$ 能够推出 $\beta\in A$。

$A$ 中的集合被称为\emph{单纯形}。单纯形 $\alpha\in A$ 的\emph{维数}
定义为 $\dim\alpha=\operatorname{card}\alpha-1$,单纯复形的维数定义为
其中单纯形维数的最大值。$\alpha$ 的一个\emph{面}指的是一个非空子集 $\beta\subseteq\alpha$,
如果 $\beta\neq\alpha$ 则称 $\beta$ 是恰当的。\emph{顶点集}定义为
所有单纯形的并集,记为 $\Ver A=\bigcup_{\alpha\in A}\alpha$。一个
\emph{子复形}定义为某个抽象的单纯复形 $B\subseteq A$。两个抽象单纯复形
之间如果存在双射 $b:\Ver A\to\Ver B$ 使得 $\alpha\in A$ 当且仅当 $b(\alpha)\in B$,那么
说它们是\emph{同构的}。大小为 $n$ 的顶点集能够构成的最大的抽象单纯复形
具有基数 $2^n-1$。给定一个(几何)单纯复形 $K$,我们可以把所有单纯形
都丢掉,仅仅保留它们的顶点集,从而得到一个抽象单纯复形 $A$。我们说 $A$
是 $K$ 的一个\emph{顶点概形}。对称地,我们说 $K$ 是 $A$ 以及任意同构于 $A$
的抽象单纯复形的一个\emph{几何实现}。如果环境空间的维数足够高,构造几何实现
是十分简单的。

\begin{theorem}[几何实现定理]
  维数 $d$ 的抽象单纯复形在 $\mathbb{R}^{2d+1}$ 中有一个几何实现。
\end{theorem}





\end{document}