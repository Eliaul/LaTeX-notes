
\section{域扩张}

首先回顾关于环同态的一个重要定理:
\begin{theorem}
  环 $R\subseteq S$ 是两个有相同单位元的环,任取 $u\in S$,那么环同态
   $\sigma:R\to S$ 诱导出一个同态 $\sigma_u:R[x]\to S$,
  满足 $\sigma_u(x)=u$ 以及 $\sigma_u|_R=\sigma$。
\end{theorem}
证明直接验证同态的条件即可。该定理的重要性在于:如果我们取 $\sigma$ 为单位映射,即对于 $a\in R$
有 $\sigma(a)=a$,那么对于 $f(x)=a_0+a_1x+\cdots+a_nx^n\in R[x]$,有
\[
  \sigma_u(f(x))=a_0+a_1u+\cdots+a_nu^n,  
\]
此时 $\sigma_u$ 也叫 $R[x]$ 的求值同态,看上去就像将 $x$ 替换成了 $u$,得到一个值 $f(u)\in S$。
使用同态基本定理,那么 $R[x]/\ker \sigma_u\simeq R[u]$,$R[u]$ 表示 $S$ 中包含 $R$ 和 $u$ 的最小的子环。
如果我们对 $R$ 和 $S$ 再限制一些条件,比如令 $R,S$ 都是域,那么 $R[x]$ 是 PID,
所以 $\ker\sigma_u$ 是一个主理想,此时就会有两种情况:
\begin{enumerate}
  \item $\ker\sigma_u=(0)$,那么 $R[u]\simeq R[x]$ 是一个整环,此时称 $u$ 是 $R$ 上的超越元。
  \item $\ker\sigma_u=(f(x))$,那么 $R[u]\simeq R[x]/(f(x))$,但是 $R[u]$ 作为域 $S$ 的子环,
  必然也是一个整环,再结合 $R[x]$ 是 PID,那么 $(f(x))$ 是一个素(极大)理想,从而
  $f(x)$ 是不可约多项式,且 $R[u]$ 也是一个域,此时称 $u$ 是 $R$ 上的代数元,规定 $f(x)$ 首一,
  称 $f(x)$ 是 $u$ 在 $R$ 上的极小多项式,我们记为 $\min(R,u)$。
\end{enumerate}

若 $K$ 是域 $F$ 的一个扩域,对于 $\alpha\in K$,我们记 $F(\alpha)$ 为 $K$ 中包含 $F$ 和 $\alpha$ 的最小的
子域,不难发现 $F(\alpha)$ 就是 $F[\alpha]$ 的分式域。通过上面的叙述,我们实际上证明了:
\begin{proposition}\label{prop:property of algebraic element}
  若 $K$ 是域 $F$ 的一个扩域,$\alpha\in K$ 是 $F$ 上的代数元,那么
  \begin{enumerate}
    \item 多项式 $\min(F,\alpha)\in F[x]$ 是不可约多项式。
    \item 若 $g(x)\in F[x]$,那么 $g(\alpha)=0$ 当且仅当 $\min(F,\alpha)$ 整除 $g(x)$。
    \item $F(\alpha)=F[\alpha]$。此外,若 $n=\deg(\min(F,\alpha))$,那么 $\{1,\alpha,\dots,\alpha^{n-1}\}$
    组成了 $F$-向量空间 $F(\alpha)$ 的一组基,这表明 $[F(\alpha):F]=n$。
  \end{enumerate}
\end{proposition}
\begin{proof}
  我们只需要证明 3 即可。由于 $F[\alpha]$ 是域,且同时包含 $F$ 和 $\alpha$,所以
   $F(\alpha)\subseteq F[\alpha]$。另一方面,根据域对运算的封闭性,自然有 $F[\alpha]\subseteq F(\alpha)$。
  所以 $F(\alpha)=F[\alpha]$。

  要说明 $\{1,\alpha,\dots,\alpha^{n-1}\}$ 是一组基,即证明它们线性无关且张成 $F(\alpha)=F[\alpha]$。任取
  $f(\alpha)= a_0+a_1\alpha+\cdots +a_m\alpha^m \in F[\alpha]$,若 $m\geq n$,记 $p(x)=\min(F,\alpha)$,
  作带余除法,那么存在 $q(x),r(x)\in F[x]$ 且 $\deg r<\deg p=n$,使得
  \[
    f(x)=q(x)p(x)  +r(x),
  \]
  于是 $f(\alpha)=q(\alpha)p(\alpha)+r(\alpha)=r(\alpha)\in \spa\{1,\alpha,\dots,\alpha^{n-1}\}$。
  令
  \[
    c_0+c_1\alpha+\cdots+c_{n-1}\alpha^{n-1}=0,\quad c_i\in F.
  \]
  那么 $h(x)=c_0+c_1x+\cdots +c_{n-1}x^{n-1}$ 满足 $h(\alpha)=0$,所以 $p(x) \mid h(x)$,但是
  $\deg p>\deg h$,所以 $h(x)=0$,即 $c_i=0$,所以 $\{1,\alpha,\dots,\alpha^{n-1}\}$ 线性无关。
\end{proof}

$K$ 是域 $F$ 的一个扩域,令 $\alpha_1,\dots,\alpha_n\in K$,我们定义
$F[\alpha_1,\dots,\alpha_n]$ 和 $F(\alpha_1,\dots,\alpha_n)$ 分别为 $K$ 的包含
$F$ 和 $\alpha_1,\dots,\alpha_n$ 的最小的子环和子域。此时 $F(\alpha_1,\dots,\alpha_n)$
依然是 $F[\alpha_1,\dots,\alpha_n]$ 的分式域。

更一般地,令 $X\subseteq K$ 是任意子集,定义 $F(X)$ 为 $K$ 的包含 $F$ 和 $X$ 的最小子域,
我们称 $F(X)$ 是 \emph{由 $F$ 和 $X$ 生成的子域}。同样的,定义 $F[X]$ 为 $K$ 的包含 $F$ 和 $X$ 的
最小子环,我们称 $F[x]$ 是 \emph{由 $F$ 和 $X$ 生成的子环}。
对于任意域扩张 $K/F$,总是有 $K=F(K)$,所以 $K$ 总是由 $F$ 和 $K$ 的某个子集生成。
对于 $F(X)$,我们可以有如下刻画:
\begin{proposition}\label{prop:generated by set}
  $K$ 是域 $F$ 的一个扩域,$X\subseteq K$ 是任意子集。如果 $\alpha\in F(X)$,那么存在
  一些 $a_1,\dots,a_n\in X$ 使得 $\alpha\in F(a_1,\dots,a_n)$。因此,
  \[
    F(X)=\bigcup \bigl\{F(a_1,\dots,a_n)\mid a_1,\dots,a_n\in X\bigr\}  ,
  \]
  其中并集取遍 $X$ 的所有有限子集。
\end{proposition}
\begin{proof}
  记 $S=\bigcup \bigl\{F(a_1,\dots,a_n)\mid a_1,\dots,a_n\in X\bigr\}$,由于每个
  $F(a_1,\dots,a_n)\subseteq F(X)$,所以 $S\subseteq F(X)$。显然 $S$ 包含 $F$ 和 $X$,如果我们
  能证明 $S$ 是一个域,那么根据 $F(X)$ 的最小性,就有 $F(X)\subseteq S$,从而推出 $F(X)=S$。
  现在我们证明 $S$ 确实是一个域。任取 $\alpha,\beta\in S$,那么存在
  $a_1,\dots,a_n\in X$ 和 $b_1,\dots,b_m\in X$,使得 $\alpha\in F(a_1,\dots,a_n)$ 以及 
  $\beta\in F(b_1,\dots,b_m)$。此时 $\alpha,\beta\in F(a_1,\dots,a_n,b_1,\dots,b_m)$,
  所以 $\alpha\pm \beta,\alpha\beta$ 和 $\alpha/\beta$ 都在 $S$ 中,故 $S$
  是域。
\end{proof}

\begin{definition}
  $K$ 是域 $F$ 的一个扩域,如果每个 $\alpha\in K$ 都是 $F$ 上的代数元,即存在一个非零多项式
  $f(x)\in F[x]$ 在 $K$ 中满足 $f(\alpha)=0$,那么我们说 $K/F$ 是代数扩张,否则称
  $K/F$ 是超越扩张。
  如果 $K$ 作为 $F$-向量空间是有限维的,那么我们称 $K/F$ 是有限扩张。
\end{definition}

由于 $\pi$ 不是任何有理系数多项式的零点,所以 $\mathbb{Q}(\pi)/\mathbb{Q}$ 是超越扩张。
由于 $\sqrt{2}$ 是多项式 $x^2-2\in\mathbb{Q}[x]$ 的零点,所以 $\mathbb{Q}(\sqrt{2})/\mathbb{Q}$
是有限扩张。由于 $x^2-2$ 在 $\mathbb{Q}$ 上不可约,根据 \autoref{prop:property of algebraic element},
所以 $[\mathbb{Q}(\sqrt{2}):\mathbb{Q}]=2$。

接下来我们证明如果 $K/F$ 是有限扩张,那么 $K$ 在 $F$ 上是有限生成的,即存在 $\alpha_1,\dots,\alpha_n\in K$
使得 $K=F(\alpha_1,\dots,\alpha_n)$。

\begin{proposition}
  若 $K/F$ 是有限扩张,那么 $K$ 在 $F$ 上是有限生成的且 $K/F$ 是代数扩张。
\end{proposition}
\begin{proof}
  设 $[K:F]=n$,也就是说存在 $\alpha_1,\dots,\alpha_n\in K$,使得其成为 $K$ 的一组基,由于
  $K$ 中的任意元素都可以唯一地表示成 $\alpha_1,\dots,\alpha_n$ 的 $F$-线性组合,
  所以 $K=F(\alpha_1,\dots,\alpha_n)$,这就证明了第一句话。任取 $\alpha\in K$,
  那么 $n+1$ 个向量 $1,\alpha,\dots,\alpha^n$ 一定 $F$-线性相关,即存在不全为零的
  $c_i\in F$,使得 $c_0+c_1\alpha+\cdots+c_n\alpha^n=0$,这表明 $\alpha$
  是多项式 $c_0+c_1x+\cdots c_nx^n$ 的根,即 $\alpha$ 在 $F$ 上代数,从而 $K/F$
  是代数扩张。
\end{proof}

结合 \autoref{prop:property of algebraic element},该命题告诉
我们 $\alpha\in K$ 在 $F$ 上代数当且仅当 $[F(\alpha):F]<\infty$,这是后面我们判断一个元素是否为
代数元的主要方法。

该命题的逆命题也成立:若 $K/F$ 是代数扩张且 $K=F(\alpha_1,\dots,\alpha_n)$,那么
$K/F$ 是有限扩张。为了证明这一点,我们需要说明 $[K:F]$ 有限,根据
\autoref{prop:property of algebraic element},我们知道当 $\alpha$ 是 $F$ 上的代数元的
时候,$F(\alpha)/F$ 是有限的,注意到
$F\subseteq F(\alpha_1)\subseteq F(\alpha_1,\alpha_2)$,所以我们需要考虑 $F\subseteq L\subseteq K$ 的时候,
$K/F$ 有限和 $K/L,L/F$ 有限的关系。

\begin{proposition}\label{prop:property of finit extension}
  若 $F\subseteq L\subseteq K$,那么 $K/F$ 是有限扩张当且仅当 $K/L,L/F$ 都是有限扩张,并且
  \[
    [K:F]=[K:L][L:F].  
  \]
\end{proposition}
\begin{proof}
  设 $\{\alpha_i\}_{1\leq i\leq m}$ 是 $L/F$ 的一组基,$\{\beta_j\}_{1\leq j\leq n}$
  是 $K/L$ 的一组基,我们证明 $\{\alpha_i\beta_j\}$ 是 $K/F$ 的一组基。
  设 $\sum_j\sum_i c_{ij}\alpha_i\beta_j=0,c_{ij}\in F$,先对 $j$ 求和,由于
  $\{\beta_j\}$ $L$-线性无关,所以对于每个 $j$ 有 $\sum_i c_{ij}\alpha_i=0$,
  又因为 $\{\alpha_i\}$ $F$-线性无关,所以 $c_{ij}=0$。
  任取 $x\in K$,那么 $x$ 可以表示成 $\{\beta_j\}$ 的 $L$-线性组合,每一项系数又可以
  表示成 $\{\alpha_i\}$ 的 $F$-线性组合,所以 $x$ 可以表示成 $\{\alpha_i\beta_j\}$
  的 $F$-线性组合,即 $\{\alpha_i\beta_j\}$ 张成 $K$。
\end{proof}

\begin{proposition}\label{prop:property of finitly generated extension}
  $K/F$ 是域扩张,若 $\alpha_1,\dots,\alpha_n\in K$ 都是 $F$ 上的代数元且
  $K=F(\alpha_1,\dots,\alpha_n)$,那么 $F[\alpha_1,\dots,\alpha_n]=F(\alpha_1,\dots,\alpha_n)$
  且
  \[
    [K:F]\leq \prod_{i=1}^n [F(\alpha_i):F].  
  \]
  这表明 $[K:F]$ 是有限扩张。
\end{proposition}
\begin{proof}
  对 $n$ 归纳。$n=1$ 的时候,根据 \autoref{prop:property of algebraic element},结论成立。
  假设结论在 $n-1$ 的时候成立,对于 $n$ 的时候,记 $L=F(\alpha_1,\dots,\alpha_{n-1})$,
  根据假设 $F(\alpha_1,\dots,\alpha_{n-1})=F[\alpha_1,\dots,\alpha_{n-1}]$,此时
  $\alpha_n$ 在 $L$ 上代数,所以 $K=L(\alpha_n)=L[\alpha_n]$。
  在 $L[x]$ 中,有 $\min(L,\alpha_n)\mid \min(F,\alpha_n)$,所以
  $[K:L]\leq [F(\alpha_n):F]$,再根据假设和 \autoref{prop:property of finit extension},就有
  \[
    [K:F] =[K:L][L:F]\leq [F(\alpha_n):F]\prod_{i=1}^{n-1}[F(\alpha_i):F]
    =\prod_{i=1}^n [F(\alpha_i):F].\qedhere
  \]
\end{proof}

\begin{theorem}
  $F\subseteq L\subseteq K$,如果 $K/L$ 和 $L/F$ 都是代数扩张,那么 $K/F$ 是代数扩张。
\end{theorem} 
\begin{proof}
  任取 $\alpha\in K$,那么 $\alpha$ 在 $L$ 上是代数的,设
  \[
    \min(L,\alpha)=a_0+a_1x+\cdots+a_{n-1}x^{n-1}+x^n\quad a_i\in L,  
  \]
  考虑 $E=F(a_0,\dots,a_{n-1})$,由于 $a_i$ 在 $F$ 上代数,根据
  \autoref{prop:property of finitly generated extension},所以 $E/F$ 是有限扩张,
  又因为 $\alpha$ 是 $E$ 上多项式的一个根,所以 $E(\alpha)/E$ 是有限扩张,所以
  \[
    [E(\alpha):F]=[E(\alpha):E][E:F]  <\infty,
  \]
  注意到 $F(\alpha)\subseteq E(\alpha)$,所以 $F(\alpha)/F$ 是有限扩张,即 $\alpha$ 是
  $F$ 上的代数元,这就说明了 $K/F$ 是代数扩张。
\end{proof}

反过来,如果 $K/F$ 是代数扩张,很容易说明 $K/L$ 和 $L/F$ 都是代数扩张。所以,对于本节中的概念,
如果有域塔 $F\subseteq L\subseteq K$,那么:
\begin{itemize}
  \item $K/F$ 是代数扩张 $\Leftrightarrow$ $K/L,L/F$ 是代数扩张。
  \item $K/F$ 是有限扩张 $\Leftrightarrow$ $K/L,L/F$ 是有限扩张。
\end{itemize}

接下来,我们给出 $\alpha\in K$ 在 $F$ 上代数当且仅当 $[F(\alpha):F]<\infty$ 的一个应用作为本节的结尾。

\begin{definition}\label{def:closure of extension}
  令 $K/F$ 是一个域扩张。定义集合
  \[
    \{a\in K\mid \text{$a$ is algebraic over $F$}\}  
  \]
  为 $F$ 在 $K$ 中的\emph{代数闭包}。
\end{definition}

对于扩张 $\mathbb{C}/\mathbb{Q}$,$\mathbb{Q}$ 在 $\mathbb{C}$ 中的代数闭包被称为代数数域,我们记为
$\mathbb{A}$,下面我们证明 $F$ 在 $K$ 中的代数闭包确实是一个域,这意味着其是 $K$ 中 $F$ 的最大的代数扩域。

\begin{proposition}
  令 $K/F$ 是一个域扩张,$L$ 是 $F$ 在 $K$ 中的代数闭包,那么 $L$ 是一个域,因此 $L$
  是 $K$ 中 $F$ 的最大的代数扩域。
\end{proposition}
\begin{proof}
  任取 $a,b\in L$,那么 $a,b$ 在 $F$ 上代数,所以 $F(a)/F$ 和 $F(b)/F$ 都是有限扩张,
  根据 \autoref{prop:property of finitly generated extension},我们有
  \[
    [F(a,b):F]\leq [F(a):F][F(b):F]<\infty,  
  \]
  所以 $F(a,b)/F$ 是有限扩张,从而是代数扩张,而 $a\pm b,ab$ 和 $a/b$ 都在 $F(a,b)$
  中,所以 $a\pm b,ab$ 和 $a/b$ 都在 $L$ 中,所以 $L$ 确实是一个域。
\end{proof}

\subsubsection{域的复合}

令 $F$ 是域,$K$ 是 $F$ 的扩张,$L_1,L_2$ 是 $K$ 的包含 $F$ 的两个子域。那么
\emph{复合} $L_1L_2$ 指的是 $K$ 的由 $L_1$ 和 $L_2$ 生成的子域,即
$L_1L_2=L_1(L_2)=L_2(L_1)$。






