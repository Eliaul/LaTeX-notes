\section{自同构}

Galois 理论的核心思想在于将域扩张和群论联系起来。$K,L$ 是域 $F$ 的两个扩张,
一个 \emph{$F$-同态} $\tau:K\to L$ 指的是一个环同态,并且使得 $\tau|_F=\id$。
由于 $\tau$ 是从域出发的同态,并且不是零映射,所以 $\tau$ 必是单同态。
如果 $\tau$ 是同构,那么称 $\tau$ 是 $F$-同构。$K\to K$ 的 $F$-同构被称为
$F$-自同构。

注意到 $F$-同态 $\tau:K\to L$ 同时是 $F$-线性映射,所以如果 $[K:F]=[L:F]<\infty$
的时候,由于 $\tau$ 是单射,所以 $\tau$ 必然是满射。也就是说,如果 $K,L$
都是 $F$ 上的相同维数的有限维向量空间,那么 $F$-同态 $\tau:K\to L$ 必然为 $F$-同构。
特别地,$K\to K$ 的 $F$-同态必为 $F$-自同构。 

\begin{definition}
  $K/F$ 是域扩张,定义 $K/F$ 的 \emph{Galois 群} 为 $K$ 的所有 $F$-自同构构成的群,
  记为 $\Gal(K/F)$。
\end{definition}

\begin{lemma}\label{lemma:tau determined by X}
  令 $K=F(X)$,那么 $K$ 的 $F$-自同构完全由其在 $X$ 上的行为决定,更准确地说,
  如果 $\sigma,\tau\in\Gal(K/F)$ 并且 $\sigma|_X=\tau|_X$,那么 $\sigma=\tau$。
\end{lemma} 
\begin{proof}
  任取 $\alpha\in K$,根据 \autoref{prop:generated by set},存在 $a_1,\dots,a_n\in X$
  使得 $\alpha\in F(a_1,\dots,a_n)$,即存在 $f,g\in F[x_1,\dots,x_n]$ 使得
  $\alpha=f(a_1,\dots,a_n)/g(a_1,\dots,a_n)$,设
  \[
    f(x_1,\dots,x_n)  =\sum b_{i_1,\dots,i_n} x_1^{i_1}\cdots x_n^{i_n},\quad
    g(x_1,\dots,x_n)  =\sum c_{i_1,\dots,i_n} x_1^{i_1}\cdots x_n^{i_n},
  \]
  其中 $b_{i_1,\dots,i_n},c_{i_1,\dots,i_n}\in F$。如果 $\sigma|_X=\tau|_X$,那么
  \begin{align*}
    \sigma(\alpha)&=
    \frac{\sum \sigma(b_{i_1,\dots,i_n}) \sigma(x_1)^{i_1}\cdots \sigma(x_n)^{i_n}}
    {\sum \sigma(c_{i_1,\dots,i_n})\sigma(x_1)^{i_1}\cdots \sigma(x_n)^{i_n}}\\
    &=
    \frac{\sum b_{i_1,\dots,i_n} \tau(x_1)^{i_1}\cdots \tau(x_n)^{i_n}}
    {\sum c_{i_1,\dots,i_n}\tau(x_1)^{i_1}\cdots \tau(x_n)^{i_n}}\\
    &=\tau(\alpha).\qedhere
  \end{align*}
\end{proof}

\begin{lemma}\label{lemma:tau alpha is zero}
  令 $\tau:K\to L$ 是 $F$-同态,$\alpha\in K$ 是 $F$ 上的代数元。如果 $f(x)\in F[x]$
  使得 $f(\alpha)=0$,那么 $f(\tau(\alpha))=0$。特别地,若 $\tau\in\Gal(K/F)$,那么 $\tau$
  一定是 $\min(F,\alpha)$ 的所有根的一个置换,此外我们还有 $\min(F,\alpha)=\min (F,\tau(\alpha))$。
\end{lemma}
\begin{proof}
  设 $f(x)=a_0+a_1x+\cdots+a_nx^n\in F[x]$ 使得 $f(\alpha)=0$,那么
  \begin{align*}
    f(\tau(\alpha))&=a_0+a_1\tau(\alpha)+\cdots+a_n\tau(\alpha)^n  \\
    &=\tau(a_0+a_1\alpha+\cdots+a_n\alpha^n) \\
    &=\tau(f(\alpha))=\tau(0)=0.
  \end{align*}
  若 $\tau\in\Gal(K/F)$,那么 $\tau(\alpha)$ 也是 $\min(F,\alpha)$ 的零点。
  设 $p(x)=\min(F,\tau(\alpha))$,那么有 $p(x)\mid \min(F,\alpha)$,
  由于 $\min (F,\alpha)$ 是不可约多项式,所以 $p(x)=\min(F,\alpha)$。
\end{proof}

\begin{corollary}
  如果 $[K:F]<\infty$,那么 $\lvert\Gal(K/F)\rvert<\infty$。
\end{corollary}
\begin{proof}
  $[K:F]<\infty$ 表明 $K=F(\alpha_1,\dots,\alpha_n)$,其中 $\alpha_1,\dots,\alpha_n$
  都是 $F$ 上的代数元。根据前面的两个引理,那么 $\tau\in\Gal(K/F)$ 完全由 $\tau(\alpha_i)$
  决定,而 $\tau(\alpha_i)$ 又只可能是 $\min(F,\alpha_i)$ 的零点,所以 $\tau(\alpha_i)$
  的取值只可能是有限个,故 $\tau$ 也只有有限种可能。
\end{proof}

令 $K$ 是域,$S\subseteq \Aut(K)$ 是自同构群 $\Aut(K)$ 的子集,定义 $S$ 
的\emph{不动域}为
\[
  \Fix(S)=\{\alpha\in K\,|\, \forall\tau\in S,\tau(\alpha)=\alpha\}  .
\]
不难验证 $\Fix(S)$ 确实是 $K$ 的一个子域。满足 $F\subseteq L\subseteq K$
的域 $L$ 被称为 $K/F$ 的\emph{中间域},此时,如果 $S\subseteq \Gal(K/F)$,那么
$\Fix(S)$ 是 $K/F$ 的中间域。

记 $\mathcal{L}$ 为 $K$ 的所有子域的集合,$\mathcal{S}$ 为 $\Aut(K)$ 的所有子集的集合。
那么 $\Gal$ 可以视为 $\mathcal{L}\to\mathcal{S}$ 的映射,将 $L\subseteq K$
送到 $\Gal(K/L)$。反之,$\Fix$ 可以视为 $\mathcal{S}\to\mathcal{L}$ 的映射,将
$S\subseteq\Aut (K)$ 送到 $\Fix(S)$。

\begin{lemma}\label{lemma:property of Gal and Fix}
  $K$ 是一个域。
  \begin{enumerate}
    \item 如果 $L_1\subseteq L_2\subseteq K$,那么 $\Gal(K/L_1)\supseteq\Gal(K/L_2)$。
    \item 如果 $S_1\subseteq S_2\subseteq\Aut(K)$,那么 $\Fix(S_1)\supseteq \Fix(S_2)$。
    \item 如果 $L\subseteq K$,那么 $\Fix(\Gal(K/L))\supseteq L$。
    \item 如果 $S\subseteq \Aut(K)$,那么 $\Gal(K/\Fix(S))\supseteq S$。
    \item 如果 $L\subseteq K$,那么 $\Gal\bigl(K/\Fix(\Gal(K/L))\bigr)=\Gal(K/L)$。
    也就是说,如果 $H<\Aut(K)$ 是 $K$ 的某个子域 $L$ 对应的 Galois 群,即
    存在 $L\subseteq K$ 使得 $H=\Gal(K/L)$,那么 $\Gal(K/\Fix(H))=H$。
    \item 如果 $S\subseteq\Aut(K)$,那么 $\Fix\bigl(\Gal(K/\Fix(S))\bigr)=\Fix(S)$。
    也就是说,如果 $L\subseteq K$ 是 $\Aut(K)$ 的某个子集对应的不动域,即
    存在 $S\subseteq \Aut(K)$ 使得 $L=\Fix(S)$,那么 $\Fix(\Gal(K/L))=L$。
  \end{enumerate}
\end{lemma}
\begin{proof}
  前四点根据定义可以立即得到。
  
  (5) 若 $H=\Gal(K/L)$,由 (4),总是有 $\Gal(K/\Fix(H))\supseteq H$。
  另一方面,我们有 $L\subseteq \Fix(\Gal(K/L))=\Fix(H)$,所以 $H=\Gal(K/L)\supseteq \Gal(K/\Fix(H))$。

  (6) 若 $L=\Fix(S)$,由 (3),总是有 $\Fix(\Gal(K/L))\supseteq L$。另一方面,我们有
  $S\subseteq \Gal(K/\Fix(S))=\Gal(K/L)$,所以 $L=\Fix(S)\supseteq\Fix(\Gal(K/L))$。
\end{proof}

\begin{corollary}
  记 $\mathcal{L}'$ 为 $K$ 的子域的集合,满足 $L\in\mathcal{L}'$ 当且仅当存在
  子群 $H<\Aut(K)$ 使得 $L=\Fix(H)$。记 $\mathcal{H}'$ 为 $\Aut(K)$ 的子群的集合,
  满足 $H\in\mathcal{H}'$ 当且仅当存在子域 $L\subseteq K$ 使得 $H=\Gal(K/L)$。
  那么 $\mathcal{L}'$ 和 $\mathcal{H}'$ 之间存在一一对应,对应关系为
  $L\mapsto \Gal(K/L)$,$\Fix(H)\mapsto H$。
\end{corollary}
\begin{proof}
  由 \autoref{lemma:property of Gal and Fix} 的 (5) 和 (6) 即得。
\end{proof}

\begin{corollary}\label{coro:corresponce of Gal and Fix}
  $K/F$ 是域扩张。记 $\mathcal{L}'$ 为 $K/L$ 的中间域的集合,满足 $L\in\mathcal{L}'$ 当且仅当存在
  子群 $H<\Gal(K/F)$ 使得 $L=\Fix(H)$。记 $\mathcal{H}'$ 为 $\Gal(K/F)$ 的子群的集合,
  满足 $H\in\mathcal{H}'$ 当且仅当存在子中间域 $F\subseteq L\subseteq K$ 使得 $H=\Gal(K/L)$。
  那么 $\mathcal{L}'$ 和 $\mathcal{H}'$ 之间存在一一对应,对应关系为
  $L\mapsto \Gal(K/L)$,$\Fix(H)\mapsto H$。
\end{corollary}
\begin{proof}
  由 \autoref{lemma:property of Gal and Fix} 的 (5) 和 (6) 即得。
\end{proof}

若 $K/F$ 是有限扩张,\autoref{coro:corresponce of Gal and Fix} 能否推广到一般的情况是后面重点研究的问题。
也就是说,有没有一种域扩张 $K/F$,使得 $K/F$ 的所有中间域的集合 $\mathcal{L}$ 与
$\Gal(K/F)$ 的所有子群的集合 $\mathcal{H}$ 之间一一对应?如果存在一一对应,
那么我们便可以将寻找中间域转化为寻找有限群的子群,这通常要容易的多。Galois 理论的核心内容便在于
研究这一类扩张。现在,我们可以通过两个例子来说明上面的情况可能存在也可能不存在。

\begin{example}\label{exa:Galois theory 1}
  考虑 $\mathbb{Q}(\sqrt[3]{2})/\mathbb{Q}$。根据 \autoref{lemma:tau determined by X} 和
  \autoref{lemma:tau alpha is zero},$\tau\in\Gal(\mathbb{Q}(\sqrt[3]{2})/\mathbb{Q})$ 完全
  由 $\tau(\sqrt[3]{2})$ 确定,并且 $\tau(\sqrt[3]{2})$ 为 $\min(\mathbb{Q},\sqrt[3]{2})=x^3-2$
  的零点,但是 $x^3-2$ 在 $\mathbb{Q}(\sqrt[3]{2})$ 中只有一个根,所以 $\tau(\sqrt[3]{2})=\sqrt[3]{2}$,
  故 $\tau=\id$,所以 $\Gal(\mathbb{Q}(\sqrt[3]{2})/\mathbb{Q})$ 为平凡群。
  显然 $\Fix(\Gal(\mathbb{Q}(\sqrt[3]{2})/\mathbb{Q}))=\mathbb{Q}(\sqrt[3]{2})$。
  所以在这种情况下,$\mathbb{Q}$ 并不是 $\Gal(\mathbb{Q}(\sqrt[3]{2})/\mathbb{Q})$ 的某个子群
  的不动域。
\end{example}

\begin{example}
  考虑 $\mathbb{Q}(\sqrt{2})/\mathbb{Q}$。同样的,$\tau\in\Gal(\mathbb{Q}(\sqrt{2})/\mathbb{Q})$
  完全由 $\tau(\sqrt{2})$ 确定,并且 $\tau(\sqrt{2})$ 为 $\min(\mathbb{Q},\sqrt{2})=x^2-2$
  的零点,故 $\tau(\sqrt{2})=\pm\sqrt{2}$。容易验证 $\tau(\sqrt{2})=-\sqrt{2}$ 时 $\tau$
  确实是 $\mathbb{Q}(\sqrt{2})$ 的自同构,所以 $\Gal(\mathbb{Q}(\sqrt{2})/\mathbb{Q})=\{\id,\tau\}$。
  若 $a+b\sqrt{2}\in\Fix(\Gal(\mathbb{Q}(\sqrt{2})/\mathbb{Q}))$,那么
  $a+b\sqrt{2}=\tau(a+b\sqrt{2})=a-b\sqrt{2}$,即 $b=0$,所以 
  $\mathbb{Q}=\Fix(\Gal(\mathbb{Q}(\sqrt{2})/\mathbb{Q}))$。另一边,假设
  $L$ 是 $\mathbb{Q}(\sqrt{2})/\mathbb{Q}$ 的中间域,那么 
  $[L:\mathbb{Q}]\mid [\mathbb{Q}(\sqrt{2}):\mathbb{Q}]=2$,所以 $[L:\mathbb{Q}]=1$ 或者
  $[L:\mathbb{Q}]=2$,即 $L=\mathbb{Q}$ 或者 $L=\mathbb{Q}(\sqrt{2})$。
  所以在这种情况下,$\mathbb{Q}(\sqrt{2})/\mathbb{Q}$ 的中间域
  和 $\Gal(\mathbb{Q}(\sqrt{2})/\mathbb{Q})$ 的子群是一一对应的。
\end{example}

为了研究上面的问题,仅仅知道 $\Gal(K/F)$ 是有限群还不够,我们需要进一步研究 $\Gal(K/F)$ 的大小。

\begin{lemma}\label{lemma:K-independence}
  $K/F$ 是有限扩张,令 $\tau_1,\dots,\tau_n\in\Gal(K/F)$,并且 $\tau_i$ 之间互不相同,
  注意到 $\tau_i$ 可以视为 $K$-线性映射,我们断言 $\tau_1,\dots,\tau_n$ 是 $K$-线性无关的。
\end{lemma}
\begin{proof}
  对 $k=n$ 归纳。显然 $k=1$ 时结论成立。假设 $k=m-1$ 时结论成立。
  考虑 $k=m$ 的情况。令 $a_1,\dots,a_m\in K$ 使得
  \begin{equation}\label{eq:K-independence eq 1}
    a_1\tau_1+\cdots+a_m\tau_m=0,  
  \end{equation}
  对于 $1\leq i\leq m-1$,由于 $\tau_{i}\neq \tau_m$,
  所以存在非零的 $\alpha_i\in K$ 使得 $\tau_{i}(\alpha_i)\neq\tau_m(\alpha_i)$。
  由于 $\tau_j(\alpha x)=\tau_j(\alpha)\tau_j(x)$,所以
  \begin{equation}\label{eq:K-independence eq 2}
    a_1\tau_1(\alpha_i)\tau_1+\cdots+a_m\tau_m(\alpha_i)\tau_m=0,  
  \end{equation}
  将 \eqref{eq:K-independence eq 1} 式乘以 $\tau_m(\alpha_i)$ 减去
  \eqref{eq:K-independence eq 2} 式,所以
  \[
    \sum_{j=1}^{m-1} a_j(\tau_m(\alpha_i)-\tau_j(\alpha_i))\tau_j=0,
  \]
  根据假设,$\tau_1,\dots,\tau_{m-1}$ 线性无关,
  由于 $\tau_m(\alpha_i)-\tau_{i}(\alpha_i)\neq 0$,
  所以 $a_{i}=0$。所以 $a_1=\cdots=a_{m-1}=a_m=0$,
  即 $\tau_1,\dots,\tau_m$ 线性无关。
\end{proof}

\begin{proposition}\label{prop:Gal leq order of extension}
  若 $K/F$ 是有限扩张,那么 $\lvert\Gal(K/F)\rvert\leq [K:F]$。
\end{proposition}
\begin{proof}
  设 $[K:F]=n$。令 $\tau_1,\dots,\tau_m\in\Gal(K/F)$,假设 $m>n$。
  设 $\alpha_1,\dots,\alpha_n$ 为 $K$ 的一组基。考虑 $K$ 上的矩阵
  \[
    A=(\tau_i(\alpha_j))_{m\times n}=\begin{pmatrix}
      \tau_1(\alpha_1) & \tau_1(\alpha_2) & \cdots & \tau_1(\alpha_n) \\
      \tau_2(\alpha_1) & \tau_2(\alpha_2) & \cdots & \tau_2(\alpha_n) \\
      \vdots & \vdots & \ddots & \vdots \\
      \tau_m(\alpha_1) & \tau_m(\alpha_2) & \cdots & \tau_m(\alpha_n) \\
    \end{pmatrix}  ,
  \]
  $m>n$ 表明 $\rank(A)\leq n<m$,所以 $A$ 的行向量是 $K$-线性相关的,
  所以存在不全为零的 $c_1,\dots,c_m\in K$,使得对于任意的 $1\leq j\leq n$ 有
  \[
    c_1\tau_1(\alpha_j)+\cdots+c_m\tau_m(\alpha_j)=0,  
  \]
  任取 $\alpha\in K$,那么 $\alpha=a_1\alpha_1+\cdots+a_n\alpha_n$,其中
  $a_i\in F$,所以
  \begin{equation*}
    \sum_i c_i\tau_i(\alpha)=\sum_ic_i\tau_i\left(\sum_j a_j\alpha_j\right)
    =\sum_i\sum_j a_jc_i\tau_i(\alpha_j)=\sum_j a_j\left(\sum_i c_i\tau_i(\alpha_j)\right)=0,
  \end{equation*}
  这表明 $c_1\tau_1+\cdots+c_m\tau_m=0$,且 $c_1,\dots,c_m$ 不全为零,即
  $\tau_1,\dots,\tau_m$ 是 $K$-线性相关的,与 \autoref{lemma:K-independence}
  矛盾。所以有 $m\leq n$。
\end{proof}

\autoref{exa:Galois theory 1} 表明 \autoref{prop:Gal leq order of extension} 中的等号可能取不到。
那么什么时候有 $\lvert\Gal(K/F)\rvert=[K:F]$?下面的命题告诉我们,对于自同构群的有限子群而言,
\autoref{lemma:property of Gal and Fix} 的 (5) 总是成立的。

\begin{proposition}\label{prop:when Gal equal to K/F}
  令 $G$ 是 $\Aut(K)$ 的有限子群,$F=\Fix(G)$,那么 $|G|=[K:F]$,进而有 $G=\Gal(K/F)$。
\end{proposition}
\begin{proof}
  由于 $G\subseteq \Gal(K/F)$,若 $[K:F]<\infty$,根据 \autoref{prop:Gal leq order of extension},
  有 $|G|\leq \lvert\Gal(K/F)\rvert\leq [K:F]$,所以 $|G|\leq [K:F]$。若 $[K:F]=\infty$,同样有
  $|G|\leq [K:F]$。假设 $|G|<[K:F]$。设 $|G|=n$,那么可以取 $\alpha_1,\dots,\alpha_{n+1}\in K$
  使得它们 $F$-线性无关。设 $G=\{\tau_1,\dots,\tau_n\}$,考虑 $K$ 上的矩阵
  \[
    A=(\tau_i(\alpha_j))_{n\times (n+1)}=\begin{pmatrix}
      \tau_1(\alpha_1) & \tau_1(\alpha_2) & \cdots & \tau_1(\alpha_{n+1}) \\
      \tau_2(\alpha_1) & \tau_2(\alpha_2) & \cdots & \tau_2(\alpha_{n+1}) \\
      \vdots & \vdots & \ddots & \vdots \\
      \tau_{n}(\alpha_1) & \tau_n(\alpha_2) & \cdots & \tau_n(\alpha_{n+1}) \\
    \end{pmatrix}  ,
  \]
  那么 $\rank(A)\leq n<n+1$,所以 $A$ 的列向量 $K$-线性相关。将 $A$ 的列向量依次记为
  $A_1,\dots,A_{n+1}\in K^n$,选取使得 $A_1,\dots.A_m$ 线性相关的最小的 $m$。
  根据 $m$ 的最小性,存在全不为零的 $c_1,\dots,c_m\in K$,使得 $\sum_i c_iA_i=0$,即
  $\sum_i c_i\tau_j(\alpha_i)=0$,其中 $1\leq j\leq n$。通过等式两边同时除以 $c_m$,我们可以假设
  $c_m=1$。
  
  下面我们断言:系数 $c_1,\dots,c_m\in F$。
  任取 $\sigma\in G$,那么 $\sigma G=G$,所以
  \[
    0=\sum_i \sigma(c_i)\sigma\tau_j(\alpha_i)=\sum_i\sigma(c_i)\tau_j'(\alpha_i),
  \]
  其中随着 $j$ 取遍 $1$ 到 $n$,$\tau_j'$ 取遍 $G$,所以
  对于所有的 $j$ 还是有 $\sum_i\sigma(c_i)\tau_j(\alpha_i)=0$。
  那么
  \[
    \sum_i(c_i-\sigma(c_i))\tau_j(\alpha_i)=0,  
  \]
  由于 $c_m=1$,所以 $\sum_{i=1}^{m-1}(c_i-\sigma(c_i))\tau_j(\alpha_i)=0$,$m$
  的最小性表明 $c_i-\sigma(c_i)=0$。根据 $\sigma$ 的任意性,所以 $c_i\in\Fix(G)=F$。
  
  回到最开始的证明,由于 $\sum_i c_i\tau_j(\alpha_i)=0$ 对于所有的 $j$ 都成立并且 $c_i\in F$,所以
  $\tau_j(\sum_i c_i\alpha_i)=\sum_i c_i\tau_j(\alpha_i)=0$,所以 $\sum_i c_i\alpha_i=0$,
  而 $c_1,\dots,c_m$ 全不为零,$\alpha_1,\dots,\alpha_m$ 线性无关,这是矛盾的。
  所以 $|G|\geq [K:F]$,所以 $K/F$ 是有限扩张并且 $|G|=[K:F]$。

  我们总是有 $G\subseteq \Gal(K/F)$,现在又有 $|G|=[K:F]\geq \lvert\Gal(K/F)\rvert$,所以
  $G=\Gal(K/F)$。
\end{proof}

\autoref{prop:when Gal equal to K/F} 告诉我们,对于有限扩张 $K/F$,$\Gal(K/F)$ 的子群
必然是 $K/F$ 的某个中间域对应的 Galois 群,即 $\Gal:\mathcal{L}\to\mathcal{H}$ 是满射,并且
这个时候扩张次数刚好等于 Galois 群的阶数。我们可以用一个例子验证这一点。

\begin{example}
  考虑 $\mathbb{Q}(\sqrt[4]{2})/\mathbb{Q}$。那么 $\tau\in\Gal(\mathbb{Q}(\sqrt[4]{2})/\mathbb{Q})$
  将 $\sqrt[4]{2}$ 送到 $x^4-2$ 的零点,$x^4-2$ 在 $\mathbb{Q}(\sqrt[4]{2})$ 中只有两个零点,即
  $\tau(\sqrt[4]{2})=\pm\sqrt[4]{2}$,容易验证这两种情况都是自同构,
  所以 $\Gal(\mathbb{Q}(\sqrt[4]{2})/\mathbb{Q})=\{\id,\tau\}$。
  考虑 $L=\Fix(\Gal(\mathbb{Q}(\sqrt[4]{2})/\mathbb{Q}))$,注意到
  \[
    \tau(\sqrt{2})=\tau(\sqrt[4]{2})  ^2=\sqrt{2},
  \] 
  所以 $\sqrt{2}\in L$,即 $\mathbb{Q}(\sqrt{2})\subseteq L$。
  同时 $[L:\mathbb{Q}(\sqrt{2})]\mid [\mathbb{Q}(\sqrt[4]{2}):\mathbb{Q}(\sqrt{2})]=2$,
  显然 $L\neq \mathbb{Q}(\sqrt[4]{2})$,所以 $L=\mathbb{Q}(\sqrt{2})$。不难验证
  $\Gal(\mathbb{Q}(\sqrt[4]{2})/L)=\Gal(\mathbb{Q}(\sqrt[4]{2})/\mathbb{Q})$,
  所以此时有 
  $[\mathbb{Q}(\sqrt[4]{2}):L]=2=\lvert\Gal(\mathbb{Q}(\sqrt[4]{2})/L)\rvert$。这与
  \autoref{prop:when Gal equal to K/F} 的结论是相符的。
\end{example}

自然地,我们会考虑什么时候 $\Gal:\mathcal{L}\to\mathcal{H}$ 是双射,即 $K/F$
的任意中间域都是 $\Gal(K/F)$ 的某个子群的不动域。当然,这也相当于
$\Fix$ 是 $\Gal$ 的左逆。为此,我们做出以下定义。

\begin{definition}
  若 $K/F$ 是代数扩张,如果 $F=\Fix(\Gal(K/F))$,那么我们说 $K/F$ 是\emph{Galois 扩张}。
\end{definition}

\begin{corollary}\label{coro:finite Galois extension}
  $K/F$ 是有限扩张,则 $K/F$ 是 Galois 扩张当且仅当 $\lvert \Gal(K/F)\rvert=[K:F]$。
\end{corollary}
\begin{proof}
  若 $K/F$ 是 Galois 扩张,也就是说 $F=\Fix(\Gal(K/F))$,根据 \autoref{prop:when Gal equal to K/F}
  ($G$ 取为 $\Gal(K/F)$),我们有 $\lvert \Gal(K/F)\rvert=[K:F]$。
  若 $\lvert \Gal(K/F)\rvert=[K:F]$,令 $L=\Fix(\Gal(K/F))$,那么总是有 $L\supseteq F$。
  根据 \autoref{prop:when Gal equal to K/F},有 $\Gal(K/F)=\Gal(K/L)$,
  所以 $[K:F]=\lvert\Gal(K/L)\rvert\leq [K:L]$,结合 $L\supseteq F$,只可能 $L=F$,
  即 $F$ 是 Galois 扩张。
\end{proof}

实际上,对于有限 Galois 扩张而言,$\Gal:\mathcal{L}\to\mathcal{H}$ 是双射,逆映射
就是 $\Fix:\mathcal{H}\to\mathcal{L}$,即 $K/F$ 的中间域和 $\Gal(K/F)$ 的子群是一一对应的。
我们留到第五章来证明这一点。\autoref{coro:finite Galois extension} 十分重要,
提供了判断 Galois 扩张的一种数值方法。对于单扩张而言,我们有更方便的判别方法。

\begin{corollary}\label{coro:simple extension is Galois}
  $K/F$ 是域扩张,令 $a\in K$ 是 $F$ 上的代数元。那么 $\lvert\Gal(F(a)/F)\rvert$
  等于 $\min (F,a)$ 在 $F(a)$ 中的零点的个数。因此,$F(a)/F$ 是 Galois 扩张当且仅当
  $\min(F,a)$ 在 $F(a)$ 中有 $n=\deg(\min(F,a))$ 个不同的零点。
\end{corollary}
\begin{proof}
  如果 $\tau\in\Gal(F(a)/F)$,那么 $\tau(a)$ 是 $\min(F,a)$ 的零点,又因为 $\tau(a)$
  完全决定了 $\tau$,所以 $\tau$ 最多也只有 $n=\deg(\min(F,a))$ 个选择,即
  $\lvert\Gal(F(a)/F)\rvert\leq n$。反之,若 $b$ 是 $\min(F,a)$ 在 $F(a)$ 中的零点,
  我们说明确实存在 $F(a)$ 的 $F$-自同构将 $a$ 送到 $b$。定义
  $\tau:F(a)\to F(a)$ 满足 $\tau(f(a))=f(b)$,若 $f(a)=g(a)$,记 $p(x)=\min (F,a)$,
  那么 $p(x)\mid f(x)-g(x)$,由于 $p(b)=0$,所以 $f(b)=g(b)$,即
  $\tau(f(a))=\tau(g(a))$,所以 $\tau$ 是良定义的。容易验证 $\tau$ 是 $F$-同构。
  所以 $\tau\in\Gal(F(a)/F)$。这就表明 $\Gal(F(a)/F)$ 的元素一一对应到
  $\min (F,a)$ 在 $F(a)$ 中的零点。又因为 $[F(a):F]=n$,根据 \autoref{coro:finite Galois extension},
  $F(a)/F$ 是 Galois 扩张当且仅当 $\lvert\Gal(F(a)/F)\rvert=n$,当且仅当
  $\min (F,a)$ 在 $F(a)$ 中有 $n$ 个不同的零点。
\end{proof}

\autoref{coro:simple extension is Galois} 表明,阻止单扩张 $F(a)/F$ 成为 Galois 扩张
可能有两个原因:
\begin{enumerate}
  \item $p(x)=\min(F,a)$ 在 $F(a)$ 中没有全部的根,例如 $\mathbb{Q}(\sqrt[3]{2})/\mathbb{Q}$,
  此时 $\min(\mathbb{Q},\sqrt[3]{2})=x^3-2$ 在 $\mathbb{Q}(\sqrt[3]{2})$ 中只有一个零点。
  \item $p(x)=\min(F,a)$ 的所有根都在 $F(a)$ 中,但是有重根。这样的例子较为有趣,
  可以见 \autoref{exa:nonGalois has repeat roots}。
\end{enumerate}
上面两种情况可以推广到一般的域扩张进行研究,分别对应着第三章的正规扩张和第四章的可分扩张。
实际上,通过单扩张的例子,可以直观地感受到这两种扩张同时决定着一个域扩张是否为 Galois 扩张。
最后我们以一些例子结束本节。

\begin{example}\label{exa:nonGalois has repeat roots}
  令 $k$ 是特征 $p>0$ 的域,$k(t)$ 为 $k$ 上的有理函数域。考虑扩张 $k(t)/k(t^p)$。
  注意到 $t$ 满足方程 $x^p-t^p\in k(t^p)[x]$,所以 $\min(k(t^p),t) \mid (x^p-t^p)$,
  但是 $x^p-t^p=(x-t)^p$,所以 $\min(k(t^p),t)$ 在 $k(t)$ 中只有唯一的多重零点 $t$,
  因此 $\Gal(k(t)/k(t^p))=\{\id\}$,所以 $k(t)/k(t^p)$ 不是 Galois 扩张。
\end{example}

\begin{example}\label{exa:sqrt 2 and omega on Q}
  记 $\omega=e^{2\pi i/3}$,考虑 $\mathbb{Q}(\sqrt[3]{2},\omega)/\mathbb{Q}$。
  由于 $\omega$ 是多项式 $x^2+x+1\in\mathbb{Q}(\sqrt[3]{2})[x]$ 的根,
  并且 $\omega\notin\mathbb{Q}(\sqrt[3]{2})$,所以 
  $[\mathbb{Q}(\sqrt[3]{2},\omega):\mathbb{Q}(\sqrt[3]{2})]=2$。
  又因为 $[\mathbb{Q}(\sqrt[3]{2}):\mathbb{Q}]=3$,所以 $[\mathbb{Q}(\sqrt[3]{2},\omega):\mathbb{Q}]=6$。
  $\tau\in\Gal(\mathbb{Q}(\sqrt[3]{2},\omega)/\mathbb{Q})$ 完全由 $\tau(\sqrt[3]{2})$ 和
  $\tau(\omega)$ 确定,所以可能的取值为
  \begin{align*}
    \id&:\sqrt[3]{2}\mapsto\sqrt[3]{2},\ \omega\mapsto \omega,\\
    \tau_1&:\sqrt[3]{2}\mapsto\sqrt[3]{2},\ \omega\mapsto \omega^2,\\
    \tau_2&:\sqrt[3]{2}\mapsto \omega\sqrt[3]{2},\ \omega\mapsto \omega,\\
    \tau_3&:\sqrt[3]{2}\mapsto \omega\sqrt[3]{2},\ \omega\mapsto \omega^2,\\
    \tau_4&:\sqrt[3]{2}\mapsto \omega^2\sqrt[3]{2},\ \omega\mapsto \omega,\\
    \tau_5&:\sqrt[3]{2}\mapsto \omega^2\sqrt[3]{2},\ \omega\mapsto \omega^2.
  \end{align*}
  当然,可以逐一验证这些确实都是 $\mathbb{Q}(\sqrt[3]{2},\omega)$ 的 $\mathbb{Q}$-自同构,
  所以 $\mathbb{Q}(\sqrt[3]{2},\omega)/\mathbb{Q}$ 是 Galois 扩张。当然,检验上述映射是自同构
  是一个繁琐且乏味的过程,后面我们有更方便的判断 $\mathbb{Q}(\sqrt[3]{2},\omega)/\mathbb{Q}$ 
  是 Galois 扩张的方法,一旦我们知道 $\mathbb{Q}(\sqrt[3]{2},\omega)/\mathbb{Q}$ 是 Galois 扩张,
  那么上述映射就必然成为自同构。
\end{example}

\begin{example}\label{exa:symmetric function}
  这是一个非常有趣的例子。令 $k$ 是域,$K=k(x_1,\dots,x_n)$ 是 $k$ 上 $n$ 个变量的有理函数域。
  对于置换 $\sigma\in S_n$,定义 $\sigma(x_i)=x_{\sigma(i)}$。这诱导了一个环同构,
  我们仍记为 $\sigma:k[x_1,\dots,x_n]\to k[x_1,\dots,x_n]$,满足
  \[
    \sigma(f(x_1,\dots,x_n))  =f(x_{\sigma(1)},\dots,x_{\sigma(n)}).
  \]
  这又诱导了分式域的同构 $\sigma:K\to K$,满足
  \[
    \sigma\left(\frac{f(x_1,\dots,x_n)}{g(x_1,\dots,x_n)}\right) =
    \frac{f(x_{\sigma(1)},\dots,x_{\sigma(n)})}{g(x_{\sigma(1)},\dots,x_{\sigma(n)})}.
  \]
  所以我们可以将 $S_n$ 视为 $\Aut(K)$ 的子群。令 $F=\Fix(S_n)$,根据 \autoref{prop:when Gal equal to K/F},
  $K/F$ 是 Galois 扩张,并且 $\Gal(K/F)=S_n$。域 $F$ 被称为\emph{关于 $x_1,\dots,x_n$ 的对称函数域},
  这是因为 $f(x_1,\dots,x_n)/g(x_1,\dots,x_n)\in F$ 当且仅当
  \[
    \frac{f(x_{\sigma(1)},\dots,x_{\sigma(n)})}{g(x_{\sigma(1)},\dots,x_{\sigma(n)})}=
    \frac{f(x_1,\dots,x_n)}{g(x_1,\dots,x_n)}\quad \forall\sigma\in S_n.
  \]
  对于 $1\leq k\leq n$,令
  \[
    s_k=\sum_{1\leq i_1<\cdots<i_k\leq n} x_{i_1}\cdots x_{i_k}, 
  \]
  即
  \begin{align*}
    s_1&=x_1+x_2+\cdots+x_n,\\
    s_2&=x_1x_2+\cdots+x_1x_n+x_2x_3+\cdots+x_{n-1}x_n,\\
    &\vdots\\
    s_n&= x_1x_2\cdots x_n.
  \end{align*}
  我们将 $s_1,\dots,s_n$ 称为\emph{基本对称多项式}。显然 $s_1,\dots,s_n\in F$,
  所以 $k(s_1,\dots,s_n)\subseteq F$。我们将在下一节看到实际上有
  $F=k(s_1,\dots,s_n)$,也就是说每个对称函数实际上都是基本对称多项式的加减乘除构成的。
\end{example}
