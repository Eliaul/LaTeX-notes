\documentclass[fontset=none,zihao=-4]{Notes}

\makeatletter
\DeclareRobustCommand{\em}{%
  \@nomath\em \if b\expandafter\@car\f@series\@nil
  \normalfont \else \bfseries \fi}
\makeatother

\usepackage{tikz-cd,wrapstuff}
\usepackage{fixdif,siunitx,tikz,nicematrix}

\usetikzlibrary{hobby,calc,arrows}
\usetikzlibrary{positioning}
\usetikzlibrary{decorations.markings}
\usetikzlibrary{decorations.pathreplacing}

\ProvidesFile{font.def}

\setCJKmainfont{Source Han Serif SC}[
  UprightFont=*-Regular,
  BoldFont=*-Bold,
  ItalicFont=HYKaiTi S,
  ItalicFeatures={Scale=1.1}
]
\newCJKfontfamily[zhsong]\songti{Source Han Serif SC}[
  UprightFont=*-Regular,
  BoldFont=*-Bold,
  ItalicFont=HYKaiTi S,
  ItalicFeatures={Scale=1.1}
]
\setCJKsansfont{Source Han Sans SC}[
  UprightFont=*-Regular,
  BoldFont=*-Bold
]
\newCJKfontfamily[zhhei]\heiti{Source Han Sans SC}[
  UprightFont=*-Regular,
  BoldFont=*-Bold
]
\setCJKmonofont{HYFangSong S}[
  BoldFont=*,
  ItalicFont=*,
  BoldItalicFont=*
]
\newCJKfontfamily[zhfs]\fangsong{HYFangSong S}[
  BoldFont=*,
  ItalicFont=*,
  BoldItalicFont=*
]
\newCJKfontfamily[zhkai]\kaishu{HYKaiTi S}[
  BoldFont=*,
  ItalicFont=*,
  BoldItalicFont=*
]

\setmainfont{texgyretermes}[
  Extension=.otf,
  UprightFont=*-regular,
  BoldFont=*-bold,
  ItalicFont=*-italic,
  BoldItalicFont=*-bolditalic,
  SlantedFont=*-italic
]
%\setmathrm{texgyretermes}[
%  Extension=.otf,
%  UprightFont=*-regular,
%  BoldFont=*-bold,
%  ItalicFont=*-italic,
%  BoldItalicFont=*-bolditalic,
%  SlantedFont=*-italic
%]
\setsansfont{Cantarell}[
  UprightFont=* Regular,
  ItalicFont=* Italic,
  BoldFont=* Bold,
  BoldItalicFont=* Bold Italic,
  SmallCapsFont=Alegreya Sans SC
]
\setmonofont{Ubuntu Mono}[
  UprightFont=*,
  ItalicFont=* Italic,
  BoldFont=* Bold,
  BoldItalicFont=* Bold Italic
]
%\setmathfont{texgyretermes-math.otf}
%\setmathfont[range={\mathcal,\mathbfcal,\mathfrak},StylisticSet=1]{XITSMath-Regular.otf}
%\setmathfont[range={\mathbb}]{KpMath-Sans.otf}



\DeclareMathOperator\Int{Int}
\DeclareMathOperator\supp{supp}
\DeclareMathOperator\im{im}
\DeclareMathOperator\End{End}
\DeclareMathOperator\Ann{Ann}
\DeclareMathOperator\Hom{Hom}
\DeclareMathOperator\diag{diag}
\DeclareMathOperator\Or{O}
\DeclareMathOperator\rank{rank}
\DeclareMathOperator\Ob{Ob}
\DeclareMathOperator\id{id}
\DeclareMathOperator\SO{SO}
\DeclareMathOperator\SU{SU}
\DeclareMathOperator\GL{GL}
\DeclareMathOperator\SL{SL}

\newcommand{\cat}[1]{\mathsf{#1}}

\newcommand{\inn}[1]{\left\langle#1\right\rangle}
\newcommand{\norm}[1]{\left\lVert#1\right\rVert}
\newcommand{\spa}[1]{\operatorname{span}\left(#1\right)}


\makeatletter
\DeclareRobustCommand\bigop[2][1]{%
  \mathop{\vphantom{\sum}\mathpalette\bigop@{{#1}{#2}}}\slimits@
}
\newcommand{\bigop@}[2]{\bigop@@#1#2}
\newcommand{\bigop@@}[3]{%
  \vcenter{%
    \sbox\z@{$#1\sum$}%
    \hbox{\resizebox{\ifx#1\displaystyle#2\fi\dimexpr\ht\z@+\dp\z@}{!}{$\m@th#3$}}%
  }%
}
\makeatother

\newcommand{\fprod}{\DOTSB\bigop[1.05]{\ast}}

\tikzcdset{
  arrow style=tikz,
  diagrams={>={Straight Barb[scale=0.8]}}
}

\tikzset{
  point/.style={
    circle,fill,inner sep=0pt,minimum width=5pt
  }
}

\usepackage[subscriptcorrection,nofontinfo,mtpbb,mtpfrak]{mtpro2}



\setlist[enumerate]{nosep,label=(\arabic*)}
\setlist[itemize]{nosep}

\title{\sffamily 球谐函数笔记}
\author{Eliauk}


\begin{document}

\maketitle

\tableofcontents

\section{圆上的调和函数}

Joseph Fourier (1768--1830) 在研究热传导问题时引入了 Fourier 级数的概念。使用 Fourier 级数,
每个平方可积的周期函数 $f$ 都可以唯一地表示为级数
\[
  f(\theta)=a_0+\sum_{k=1}^\infty\left(a_k\cos k\theta+b_k\sin k\theta\right),
\]
其中 Fourier 系数 $a_k,b_k$ 由下式给出:
\[
  a_0=\frac{1}{2\pi}\int_{-\pi}^{\pi}f(\theta) \d\theta,\quad
  a_k=\frac{1}{\pi}\int_{-\pi}^{\pi}f(\theta)\cos k\theta \d\theta,\quad
  b_k=\frac{1}{\pi}\int_{-\pi}^{\pi}f(\theta)\sin k\theta \d\theta.
\]

这个惊人的发现在物理学、信号处理、工程学等领域有着广泛的理论和实际应用,
但是 Fourier 级数的数学内涵远不止于此。
我们引入平方可积函数的内积
\[
  \langle f,g\rangle=\int_{-\pi}^{\pi}f(\theta)g(\theta)\d\theta,
\]
也记为
\[
  \langle f,g\rangle=\int_{\mathbb{S}^1}f(\theta)g(\theta)\d\theta,
\]
其中 $\mathbb{S}^1$ 是单位圆。实际上上述记法就是在 $\mathbb{S}^1$ 流形
上的积分。通过这个内积,空间 $L^2(\mathbb{S}^1)$ 是一个完备的赋范向量空间,
也即 Hilbert 空间。进一步的,我们定义 $L^2(\mathbb{S}^1)$ 的子空间 
$\mathcal{H}_k(\mathbb{S}^1)$ 为:$\mathcal{H}_0(\mathbb{S}^1)=\mathbb{R}$
是所有常值函数的空间,$\mathcal{H}_k(\mathbb{S}^1)$ 是由函数 $\cos k\theta$ 
和 $\sin k\theta$ 张成的二维空间。那么,Fourier 级数告诉我们有正交直和分解
\[
  L^2(\mathbb{S}^1)=\bigoplus_{k=0}^\infty \mathcal{H}_k(\mathbb{S}^1).
\]

空间 $\mathcal{H}_k(\mathbb{S}^1)$ 会自然地出现在 $\mathbb{R}^2$ 上的 Laplace 方程中,
即 $\Delta f=0$。大致来说,$\mathbb{R}^2$ 中的齐次函数 $f:\mathbb{R}^2\to \mathbb{R}$
在极坐标 $(r,\theta)$ 中有形式
\[
  f(r,\theta)=r^kg(\theta).
\]
极坐标中的 Laplace 算子为
\[
  \Delta f=\frac{1}{r}\frac{\partial}{\partial r}\left(r\frac{\partial f}{\partial r}\right)
  +\frac{1}{r^2}\frac{\partial^2 f}{\partial \theta^2}.
\]
当然,我们要求 $f\in C^2$ 是二阶连续可微的。如果我们把 $f$ 限制在单位圆 $\mathbb{S}^1$ 上,
那么 $\mathbb{S}^1$ 上的 Laplace 算子为
\[
  \Delta_{\mathbb{S}^1} f=\frac{\partial^2 f}{\partial \theta^2}.
\]
下面我们将看到空间 $\mathcal{H}_k(\mathbb{S}^1)$ 实际上是 $\Delta_{\mathbb{S}^1}:L^2(\mathbb{S}^1)\to L^2(\mathbb{S}^1)$ 的特征值
$-k^2$ 的特征函数空间。

对于方程 $\Delta f=0$,采用分离变量法,设 $f(r,\theta)=F(r)g(\theta)$。
假设 $F(r)=r^k$,其中 $k\geq 0$ 是整数,这意味着 $f$ 是 $k$ 次齐次函数。
其中 $g:\mathbb{S}^1\to \mathbb{R}$ 是圆上的函数。现在,将 $f$ 代入 Laplace 算子,有
\[
  \Delta f=r^{k-2}\left(k^2g(\theta)+g''(\theta)\right)
  =r^{k-2}\left(-\Delta_{\mathbb{S}^1}g(\theta)+k^2g(\theta)\right).
\]
于是我们得到 
\[
  \Delta f=0\Longleftrightarrow \Delta_{\mathbb{S}^1}g=-k^2g.
\]
即,$g$ 是 $\Delta_{\mathbb{S}^1}$ 的特征值 $-k^2$ 的特征函数。这等价于一个
二阶微分方程:
\[
  \frac{\d^2 g}{\d\theta^2}+k^2g=0,
\]
有通解
\[
  g(\theta)=a_k\cos k\theta+b_k\sin k\theta.
\]
因此,我们发现 $0,-1,-4,\dots,-k^2,\dots$ 是 $\Delta_{\mathbb{S}^1}$ 的特征值,
并且函数 $\cos k\theta$ 和 $\sin k\theta$ 是对应的特征函数。所以,
$k=0$ 的时候特征空间是 $1$ 维的,而 $k\geq 1$ 的时候特征空间是 $2$ 维的。
可以证明 $\Delta_{\mathbb{S}^1}$ 没有其他的特征值。

回到齐次调和函数 $f(r,\theta)=r^kg(\theta)$ 上,我们注意到这些函数构成的空间由
\[
  u_k=r^k\cos k\theta,\quad v_k=r^k\sin k\theta
\]
张成。因为 $(x+iy)^k=r^k(\cos k\theta+i\sin k\theta)=u_k+i v_k$,
由于复可微函数的实部和虚部都是调和函数,所以 $u_k$ 和 $v_k$ 都是 $\mathbb{R}^2$ 上的调和函数。
因此,$u_k,v_k$ 实际上还给出了 $\mathbb{R}^2$ 上所有 $k$ 次齐次调和多项式的一个基。
其中一些低次的齐次调和多项式如下:
\begin{alignat*}{2}
  k&=0 &\qquad& 1 \\
  k&=1 &\qquad& x,\,y \\
  k&=2 &\qquad& x^2-y^2,\,xy \\
  k&=3 &\qquad& x^3-3xy^2,\,3x^2y-y^3 \\
  k&=4 &\qquad& x^4-6x^2y^2+y^4,\,x^3 y-xy^3.
\end{alignat*}
于是,我们发现 $\mathbb{S}^1$ 上的 Laplace 算子的特征函数是 $\mathbb{R}^2$ 上
调和多项式在 $\mathbb{S}^1$ 上的限制,并且有 Hilbert 空间分解
$L^2(\mathbb{S}^1)=\bigoplus_{k=0}^\infty \mathcal{H}_k(\mathbb{S}^1)$。
这意味着我们可以在球面 $\mathbb{S}^2$ 上寻找类似的分解。

\section{球面上的球谐函数}

我们的目标是寻找 $\mathbb{R}^3$ 中 Laplace 方程 $\Delta f=0$ 的齐次解,并且
证明它们对应于 $2$-球面上的 Laplace 算子 $\Delta_{\mathbb{S}^2}$ 的特征函数空间 $\mathcal{H}_k(\mathbb{S}^2)$。
那么空间 $\mathcal{H}_k(\mathbb{S}^2)$ 将给出 $L^2(\mathbb{S}^2)$ 的 Hilbert 直和分解。
这是 Fourier 级数在球面上的推广,$\mathcal{H}_k(\mathbb{S}^2)$ 中的函数被称为\emph{球谐函数}。

在 $\mathbb{R}^3$ 中,采用球坐标 $(r,\theta,\varphi)$,那么
\[
  x=r\sin\theta\cos\varphi,\quad y=r\sin\theta\sin\varphi,\quad z=r\cos\theta,
\]
并且要求 $0\leq \theta<\pi$,$0\leq\varphi<2\pi$ 以及 $r>0$。那么球坐标中
Laplace 算子为
\[
  \Delta f=\frac{1}{r^2}\frac{\partial}{\partial r}\left(r^2\frac{\partial f}{\partial r}\right)
  +\frac{1}{r^2}\Delta_{\mathbb{S}^2}f,
\]
其中 $\mathbb{S}^2$ 上的 Laplace 算子为
\[
  \Delta_{\mathbb{S}^2}f=\frac{1}{\sin\theta}\frac{\partial}{\partial \theta}
  \left(\sin\theta\frac{\partial f}{\partial \theta}\right)
  +\frac{1}{\sin^2\theta}\frac{\partial^2 f}{\partial \varphi^2}.
\]
同样,我们寻找齐次调和函数 $f(r,\theta,\varphi)=r^kg(\theta,\varphi)$。
将 $f$ 代入 Laplace 方程,有
\begin{align*}
  \Delta f&=\frac{1}{r^2}\frac{\partial}{\partial r}\left(r^2\frac{\partial (r^k g)}{\partial r}\right)
  +\frac{1}{r^2}\Delta_{\mathbb{S}^2}(r^k g) \\
  &=\frac{1}{r^2}\frac{\partial}{\partial r}\left(r^2 k r^{k-1} g\right)
  +\frac{1}{r^2}r^k \Delta_{\mathbb{S}^2} g \\
  &=r^{k-2}\left(k(k+1)g+\Delta_{\mathbb{S}^2}g\right).
\end{align*}
因此
\[
  \Delta f=0\Longleftrightarrow \Delta_{\mathbb{S}^2}g=-k(k+1)g.
\]
也即,$g$ 是 $\Delta_{\mathbb{S}^2}$ 的特征值 $-k(k+1)$ 的特征函数。

我们采用分离变量法寻找上述方程的解。令 $g(\theta,\varphi)=\Theta(\theta)\Phi(\varphi)$,那么得到方程
\[
  \frac{\Phi}{\sin\theta}\frac{\d}{\d \theta}\left(\sin\theta\frac{\d \Theta}{\d \theta}\right)
  +\frac{\Theta}{\sin^2\theta}\frac{\d^2 \Phi}{\d \varphi^2}=-k(k+1)\Theta\Phi,
\]
两边除以 $\Theta\Phi$ 并且乘以 $\sin^2\theta$,得到
\[
  \frac{\sin\theta}{\Theta}\frac{\d}{\d \theta}\left(\sin\theta\frac{\d \Theta}{\d \theta}\right)
  +k(k+1)\sin^2\theta=-\frac{1}{\Phi}\frac{\d^2 \Phi}{\d \varphi^2}.
\]
因为左边只含有 $\theta$,右边只含有 $\varphi$,所以两边只可能都等于一个常数,
记为 $\mu$。那么我们得到两个常微分方程:
\begin{align*}
  &\frac{\d^2 \Phi}{\d\varphi^2}+\mu \Phi=0, \\
  &\sin\theta\frac{\d}{\d \theta}\left(\sin\theta\frac{\d \Theta}{\d \theta}\right)
  +\left(k(k+1)\sin^2\theta-\mu\right)\Theta=0.
\end{align*}
因为我们考虑的是球面 $\mathbb{S}^2$ 上的函数,所以 $\Phi$ 是 $2\pi$-周期的,
所以必须有 $\mu=m^2$,其中 $m$ 是非负整数。那么,我们知道方程
\[
  \frac{\d^2\Phi}{\d\varphi^2}+m^2\Phi=0
\]
的解是 $2$ 维的,由下面两个函数张成:
\[
  \Phi(\varphi)=\cos m\varphi,\quad \Phi(\varphi)=\sin m\varphi.
\]
接下来,我们考虑 $\Theta$ 的方程:
\[
  \sin\theta\frac{\d}{\d \theta}\left(\sin\theta\frac{\d \Theta}{\d \theta}\right)
  +\left(k(k+1)\sin^2\theta - m^2\right)\Theta=0. 
\]
这等价于
\[
  \sin^2\theta \Theta''+\sin\theta\cos\theta \Theta'
  +\left(k(k+1)\sin^2\theta - m^2\right)\Theta=0.
\]
这是 Legendre 方程的一个变体。令 $t=\cos\theta$,考虑函数 $u(\cos\theta)=\Theta(\theta)$,
注意 $0\leq\theta<\pi$,那么我们得到二阶微分方程
\[
  (1-t^2)u''-2tu'+\left(k(k+1)-\frac{m^2}{1-t^2}\right)u=0.
\]
这被称为\emph{连带 Legendre 方程}。解这个方程的技巧是采用代换
$u(t)=(1-t^2)^{m/2}v(t)$,那么得到
\[
  (1-t^2)v''-2(m+1)tv'+(k(k+1)-m(m+1))v=0.
\]
当 $m=0$ 的时候,我们得到\emph{Legendre 方程}:
\[
  (1-t^2)v''-2tv'+k(k+1)v=0.
\]
这个方程有两个基本解 $P_k(t)$ 和 $Q_k(t)$,被称为\emph{第一类和第二类 Legendre 函数}。
$P_k(t)$ 实际上是多项式,$Q_k(t)$ 由一个在 $t=1$ 处发散的幂级数给出,
所以我们也把 $P_k(t)$ 称作\emph{Legendre 多项式}。Legendre 多项式有许多种定义形式,
其中一种定义由 \emph{Rodrigues 公式} 给出:
\[
  P_n(t)=\frac{1}{2^nn!}\frac{\d^n}{\d t^n}(t^2-1)^n.
\]
Legendre 多项式满足 $P_n(1)=1$ 以及递推关系
\begin{align*}
  P_0&=1,\quad P_1=t, \\
  (n+1)P_{n+1}&=(2n+1)tP_n - n P_{n-1} \quad n\geq 1.
\end{align*}
前六个 Legendre 多项式如下:
\[
  1,\, t,\, \frac{1}{2}(3t^2-1),\, \frac{1}{2}(5t^3-3t),\, \frac{1}{8}(35t^4-30t^2+3),\, \frac{1}{8}(63t^5-70t^3+15t).
\]

现在我们回到微分方程
\begin{equation}\label{eq:associated-Legendre}
  (1-t^2)v''-2(m+1)tv'+(k(k+1)-m(m+1))v=0.
\end{equation}
对 $t$ 微分,得到
\[
  (1-t^2)v'''-2(m+2)tv''+(k(k+1)-(m+1)(m+2))v'=0.
\]
这表明如果 $v$ 是 \eqref{eq:associated-Legendre} 式在 $k,m$ 情况下的解,
那么 $v'$ 是 \eqref{eq:associated-Legendre} 式在 $k,m+1$ 情况下的解。
因此,如果 $P_k(t)$ 在 $k$ 和 $m=0$ 的情况下是 \eqref{eq:associated-Legendre} 式的解,
那么 $P_k'(t)$ 在 $k$ 和 $m=1$ 的情况下是解,$P_k''(t)$ 在 $k$ 和 $m=2$ 的情况下是解,依此类推,
$\d^m/\d t^m \bigl(P_k(t)\bigr)$ 在 $k$ 和 $m$ 的情况下是解。于是,对于最开始的方程
\begin{equation}\label{eq:associated-Legendre-original}
  (1-t^2)u''-2tu'+\left(k(k+1)-\frac{m^2}{1-t^2}\right)u=0,
\end{equation}
有解 
\[
  u(t)=(-1)^m(1-t^2)^{m/2}\frac{\d^m}{\d t^m}\bigl(P_k(t)\bigr).
\]
这个函数 $u(t)$ 通常记为 $P_k^m(t)$,被称为\emph{连带 Legendre 函数}。
因为 $P_k(t)$ 是 $k$ 次多项式,所以在 $m>k$ 的情况下有 $P_k^m(t)\equiv 0$。
由于 $(1-t^2)^{m/2}$ 的存在,所以连带 Legendre 函数 $P_k^m(t)$ 一般来说
不是多项式并且只定义在闭区间 $[-1,1]$ 上。

连带 Legendre 函数满足许多的关系式。例如,对于 $m\geq 0$,有递推
\[
  (k-m+1)P_{k+1}^m(t)=(2k+1)tP_k^m(t)-(k+m)P_{k-1}^m(t),\quad k\geq 1.
\]
对于 $k\geq 2$,有
\[
  P_k^{m+2}(t)=-\frac{2(m+1)t}{(1-t^2)^{1/2}}P_k^{m+1}(t)
  -(k-m)(k+m+1)P_k^m(t),\quad 0\leq m\leq k-2.
\]
这些递推式可以用于从初始形式
\begin{align*}
  P_k^0(t)&=P_k(t), \\
  P_k^1(t)&=\frac{kt}{(t^2-1)^{1/2}}P_k(t)-\frac{k}{(t^2-1)^{1/2}}P_{k-1}(t)
\end{align*}
计算任意的连带 Legendre 函数 $P_k^m(t)$。
注意到 $k=m$ 的时候,有
\[
  P_{m+1}^m(t)=(2m+1)tP_m^m(t),
\]
所以还有
\[
  P_m^m(t)=(-1)^m(2m-1)!! (1-t^2)^{m/2}.
\]

再回到球谐函数的求解上,我们需要解方程
\[
  \sin^2\theta\Theta''+\sin\theta\cos\theta \Theta'
  +\left(k(k+1)\sin^2\theta - m^2\right)\Theta=0,
\]
我们使用了替换 $u(\cos\theta)=\Theta(\theta)$,并且得到了连带 Legendre 方程的解
为 $u(t)=P_k^m(t)$,所以我们有解
\[
  \Theta(\theta)=P_k^m(\cos\theta).
\]
将所有的结果综合起来,我们得到齐次函数 $f(r,\theta,\varphi)=r^k\Theta(\theta)\Phi(\varphi)$
为
\[
  f(r,\theta,\varphi)=r^k \cos m\varphi P_k^m(\cos\theta),\quad
  f(r,\theta,\varphi)=r^k \sin m\varphi P_k^m(\cos\theta),
\]
它们都是 $\mathbb{R}^3$ 上的调和函数,并且函数
\[
  \cos m\varphi P_k^m(\cos\theta),\quad \sin m\varphi P_k^m(\cos\theta)
\]
是球面上 Laplace 算子 $\Delta_{\mathbb{S}^2}$ 的特征值 $-k(k+1)$ 的特征函数。
对于固定的 $k$,取遍 $0\leq m\leq k$,我们得到 $2k+1$ 个线性无关的特征函数。

处于不同的用途,我们通常对球谐函数进行归一化处理。我们使用记号 $Y_l^m$,
其中 $l$ 是非负整数,$m$ 是整数且满足 $-l\leq m\leq l$。定义
\[
  Y_{lm}(\theta,\varphi)=\begin{cases}
    N_l^0 P_l^0(\cos\theta), & m=0, \\
    (-1)^m\sqrt{2} N_l^m \cos m\varphi P_l^m(\cos\theta), & m>0, \\
    (-1)^m\sqrt{2} N_l^{-m} \sin (-m\varphi) P_l^{-m}(\cos\theta), & m<0,
  \end{cases}
\]
其中 $N_l^m$ 是标量因子,通常被选为
\[
  N_l^m=\sqrt{\frac{(2l+1)}{4\pi}\frac{(l-m)!}{(l+m)!}}.
\]
函数 $Y_{lm}$ 被称为\emph{$l$ 次 $m$ 阶实球谐函数}。

函数 $Y_l^m$ 有许多非常漂亮的性质,为了解释它们,我们需要回顾 $L^2(\mathbb{S}^2)$
空间上的结构。$L^2(\mathbb{S}^2)$ 上的内积定义为
\[
  \langle f,g\rangle=\int_{\mathbb{S}^2} fg \d \Omega_2
  =\int_0^{2\pi}\int_0^\pi f(\theta,\varphi)g(\theta,\varphi)\sin\theta \d\theta \d\varphi.
\]
其中 $f,g\in L^2(\mathbb{S}^2)$,$\d\Omega_2$ 是 $\mathbb{S}^2$ 上的体积形式。
使用这个内积,空间 $L^2(\mathbb{S}^2)$ 是一个 Hilbert 空间。
可以证明 Laplace 算子 $\Delta_{\mathbb{S}^2}$ 相对于这个内积是自伴算子。
因为当 $l_1\neq l_2$ 的时候,函数 $Y_{l_1m_1}$ 和 $Y_{l_2m_2}$
是对应于不同特征值 $-l_1(l_1+1)$ 和 $-l_2(l_2+1)$ 的特征函数,所以它们
是正交的,也即
\[
  \left\langle Y_{l_1m_1},Y_{l_2m_2}\right\rangle=0,\quad l_1\neq l_2.
\]
同样不难证明对于固定的 $l$,有
\[
  \left\langle Y_{lm_1},Y_{lm_2}\right\rangle=\delta_{m_1 m_2}.
\]
也就是说,函数 $Y_{lm}$ 构成了一个正交函数系,我们记 $\mathcal{H}_l(\mathbb{S}^2)$
表示由这些函数张成的 $(2l+1)$-维空间。总的来说,$\Delta_{\mathbb{S}^2}$
的特征值仅有 $-l(l+1)$,并且对应的特征空间为 $E_l=\mathcal{H}_l(\mathbb{S}^2)$,
由基函数 $Y_l^m$ 张成,其中 $-l\leq m\leq l$。与圆上的情况类似,我们有
\[
  L^2(\mathbb{S}^2)=\bigoplus_{l=0}^\infty \mathcal{H}_l(\mathbb{S}^2).
\]
即任取函数 $f\in L^2(\mathbb{S}^2)$,都可以唯一表示为
\[
  f=\sum_{l=0}^\infty\sum_{-l\leq m\leq l}a_{lm} Y_{lm},
\]
这被称为\emph{球面上的 Fourier 分解}。此时 Fourier 系数计算为
\[
  a_{lm}=\langle f,Y_{lm}\rangle=\int_{\mathbb{S}^2} f(\theta,\varphi) Y_{lm}(\theta,\varphi) \d \Omega_2.
\]

当然,我们也有对应的 $\mathbb{R}^3$ 上的齐次球谐函数 $H_l^m(r,\theta,\varphi)$:
\[
  H_l^m(r,\theta,\varphi)=r^l Y_{lm}(\theta,\varphi).
\]
可以证明 $H_l^m$ 在直角坐标 $xyz$ 下总可以表示为齐次多项式!这表明:
Laplace 算子 $\Delta_{\mathbb{S}^2}$ 的特征函数(球谐函数)是 $\mathbb{R}^3$
上齐次调和多项式的限制。下面是一些低次的齐次调和多项式:
\begin{alignat*}{2}
  k&=0 &\qquad& 1 \\
  k&=1 &\qquad& x,\, y,\, z \\
  k&=2 &\qquad& x^2-y^2,\, x^2-z^2,\, xy,\, xz,\, yz \\
  k&=3 &\qquad& x^3-3xy^2,\, 3x^2y - y^3,\, x^3-3xz^2,\, 3x^2z - z^3,\, y^3-3yz^2,\, 3y^2z - z^3,\, xyz\\
\end{alignat*}

我们也可以定义对应的复球谐函数 $Y_l^m:\mathbb{S}^2\to \mathbb{C}$,其中 $-l\leq m\leq l$:
\[
  Y_l^m(\theta,\varphi)=N_l^m P_l^m(\cos\theta) e^{i m \varphi},
\]
其与实球谐函数的关系为
\[
  Y_l^m=\begin{dcases}
    \frac{(-1)^m}{\sqrt{2}}\left(Y_{l|m|}+i Y_{l,-|m|}\right), & m>0, \\
    Y_{l0}, & m=0, \\
    \frac{1}{\sqrt{2}}\left(Y_{l|m|}-i Y_{l,-|m|}\right), & m<0.
  \end{dcases}
\]
考虑复 Hilbert 空间 $L^2_{\mathbb{C}}(\mathbb{S}^2)$,内积由 Hermite 内积定义:
\[
  \langle f,g\rangle=\int_{\mathbb{S}^2} f\bar g \d \Omega_2.
\]
此时,复球谐函数 $Y_l^m$ 也是 $L_{\mathbb{C}}^2(\mathbb{S}^2)$ 上 Laplace 算子 $\Delta_{\mathbb{S}^2}$ 的特征值 $-l(l+1)$
的特征函数,并且是正交归一的。

\section{一点李群表示论}

\begin{definition}
  李群 $G$ 的一个表示指的是一个向量空间 $V$ 附带一个同态 $\rho:G\to\GL(V)$。
  李代数 $\mathfrak{g}$ 的一个表示指的是一个向量空间 $V$ 附带一个同态
  $\rho:\mathfrak{g}\to \mathfrak{gl}(V)$。
\end{definition}

李群 $G$ 的两个表示 $V,W$ 之间的 $G$-线性映射指的是一个线性映射 $f:V\to W$
与 $G$ 的作用交换:即对于任意 $g\in G$,都有 $f\circ \rho_V(g)=\rho_W(g)\circ f$。
对李代数的表示同样有类似的定义。$V$ 和 $W$ 之间的所有 $G$-线性映射构成一个向量空间,记为 $\Hom_G(V,W)$。
类似的,李代数的表示之间的线性映射构成向量空间 $\Hom_{\mathfrak{g}}(V,W)$。

不加说明的情况下,总是假设 $V$ 是复向量空间,此时 $G\to\GL(V)$ 理解为
一个光滑映射,$\GL(V)$ 是 $2n^2$-维光滑流形。

李群表示和李代数表示之间有如下重要联系。

\begin{theorem}
  令 $G$ 是李群,有李代数 $\mathfrak{g}$。
  \begin{enumerate}
    \item 每个表示 $\rho:G\to\GL(V)$ 都定义了一个李代数表示
      $\rho_*:\mathfrak{g}\to \mathfrak{gl}(V)$,并且每个 $G$-线性映射
      也是 $\mathfrak{g}$-线性映射。
    \item 如果 $G$ 是连通并且单连通的,那么 $\rho\mapsto \rho_*$ 给出了
      李群表示与李代数表示之间的一一对应,特别的,$\mathfrak{g}$
      的每个表示都可以唯一提升为 $G$ 的一个表示。
  \end{enumerate}
\end{theorem}

这是一个非常重要的定理,因为李代数作为有限维向量空间往往更容易处理。
例如,这个定理表明 $\SU(2)$ 的表示和 $\mathfrak{su}(2)$ 的表示相同,
而 $\mathfrak{su}(2)$ 是一个 $3$ 维的实向量空间。

这个定理也可以用于研究连通但是不是单连通的李群的表示。实际上,对于这样的
李群 $G$,存在一个单连通的李群 $\tilde G$ 使得 $G=\tilde G/Z$,
其中 $Z\subseteq G$ 是离散的中心子群。因此,$G$ 的表示和 $\tilde{G}$
的满足 $\rho(Z)=\mathrm{id}$ 的表示相同。一个非常重要的例子
就是二重覆盖 $\SU(2)\to \SO(3)$。

\begin{lemma}
  令 $\mathfrak{g}$ 是实李代数,$\mathfrak{g}_{ \mathbb{C}}$ 是它的复化。
  那么 $\mathfrak{g}$ 的任意复表示都可以提升为 $\mathfrak{g}_{\mathbb{C}}$ 的表示,
  并且 $\Hom_{\mathfrak{g}}(V,W)=\Hom_{\mathfrak{g}_{\mathbb{C}}}(V,W)$。
\end{lemma}
\begin{proof}
  令 $\rho:\mathfrak{g}\to \mathfrak{gl}(V)$ 是 $\mathfrak{g}$ 的一个复表示。
  延拓为 $\rho(x+iy)=\rho(x)+i\rho(y)$,这成为 $\mathfrak{g}_{\mathbb{C}}$ 的一个表示。
\end{proof}

我们重点关注 $\mathfrak{su}(2)$ 和 $\mathfrak{sl}(2,\mathbb{C})$。
$\mathfrak{su}(2)$ 的基(作为实向量空间)由 Pauli 矩阵给出:
\[
  i\sigma_1=\begin{pmatrix}
    0 & i \\ i & 0
  \end{pmatrix},\quad
  i\sigma_2=\begin{pmatrix}
    0 & 1 \\ -1 & 0
  \end{pmatrix},\quad
  i\sigma_3=\begin{pmatrix}
    i & 0 \\ 0 & -i
  \end{pmatrix}.
\]
满足对易关系
\[
  [i\sigma_1,i\sigma_2]=2i\sigma_3  ,\quad
  [i\sigma_2,i\sigma_3]=2i\sigma_1 ,\quad
  [i\sigma_3,i\sigma_1]=2i\sigma_2.
\]
$\mathfrak{sl}(2,\mathbb{C})$ 的基(作为复向量空间)为
\[
  E=\begin{pmatrix}
    0 & 1 \\ 0 & 0
  \end{pmatrix},\quad
  F=\begin{pmatrix}
    0 & 0 \\ 1 & 0
  \end{pmatrix},\quad
  H=\begin{pmatrix}
    1 & 0 \\ 0 & -1
  \end{pmatrix}.
\]
满足对易关系
\[
  [H,E]=2E,\quad [H,F]=-2F,\quad [E,F]=H.
\]
由于 $E=\frac{1}{2}(i\sigma_2-i(i\sigma_1))$,$F=\frac{1}{2}(-i\sigma_2-i(i\sigma_1))$
以及 $H=-i(i\sigma_3)$,所以 $\mathfrak{sl}(2,\mathbb{C})$ 是 $\mathfrak{su}(2)$ 的复化。
根据前面的结论,$\mathfrak{su}(2)$ 的表示和 $\mathfrak{sl}(2,\mathbb{C})$ 的表示是一样的。
同时 $\SU(2)$ 和 $\SL(2,\mathbb{C})$ 都是连通且单连通的李群,所以它们的表示
也和 $\mathfrak{su}(2)$ 的表示是一一对应的。也就是说,研究 $\SU(2)$,$\SL(2,\mathbb{C})$,
$\mathfrak{su}(2)$ 和 $\mathfrak{sl}(2,\mathbb{C})$ 的表示是等价的,
我们只需要选择其中一个来研究即可。

\begin{definition}
  令 $V$ 是 $G$(或者 $\mathfrak{g}$)的一个表示,如果子空间 $W\subseteq V$
  满足:对于任意 $g\in G$,有 $\rho(g)W\subseteq W$(或者对于任意 $x\in \mathfrak{g}$,有 $\rho(x)W\subseteq W$),
  那么称 $W$ 是 $V$ 的一个子表示。如果 $V$ 没有非平凡的子表示(即 $V$ 的子表示只有 $0$ 和 $V$),那么称 $V$ 是不可约的。
\end{definition}

例如 $\mathbb{C}^n$ 考虑为 $\SL(n,\mathbb{C})$ 的表示的时候,是不可约的。
因为可以证明,对于任意非零向量 $v\in \mathbb{C}^n$,都存在一个行列式为 $1$ 的矩阵 $A\in \SL(n,\mathbb{C})$,
使得 $Av$ 是任意一个标准基向量。因此,任意非零子空间 $W\subseteq \mathbb{C}^n$ 都包含所有的标准基向量,
所以 $W=\mathbb{C}^n$。

对于 $G$ 或者 $\mathfrak{g}$ 的两个表示 $V$ 和 $W$,它们的直和 $V\oplus W$ 也是一个表示,
只需要定义 $G$ 在 $V\oplus W$ 上的作用为
\[
  \rho_{V\oplus W}(g)(v,w)=(\rho_V  (g)v,\rho_W(g)w).
\]
  
\begin{definition}
  一个表示如果同构于若干不可约表示的直和,即存在不可约表示 $V_i$
  使得 $V\simeq\bigoplus_i V_i$,那么称它是完全可约的。
\end{definition}

通常来说,我们可以把直和项中同构的空间合并,记为 $V\simeq\bigoplus n_iV_i$,
其中 $V_i$ 是两两不同构的不可约表示,$n_i$ 是它们的重数。

令 $G=\mathbb{R}$,那么 $\mathfrak{g}=\mathbb{R}$。此时 $\mathfrak{g}$
的表示就是一个向量空间 $V$ 附带线性映射 $\mathbb{R}\to\End(V)$,
显然,这样的映射一定形如 $t\mapsto tA$,其中 $A\in \End(V)$。
所以对应的 $\mathbb{R}$ 的表示一定形如 $t\mapsto \exp(tA)$。也就是说,
分类 $\mathbb{R}$ 的表示等价于分类线性映射 $V\to V$。这样的分类
可以通过 Jordan 标准型完成。如果 $v$
是 $A$ 的一个特征向量,那么一维空间 $\mathbb{C}v\subseteq V$
在 $A$ 的作用下是不变的,所以 $\mathbb{C}v$ 是 $V$ 的一个子表示。
因此,$\mathbb{R}$ 的任意大于 $1$ 维的表示都有非平凡的子表示,
也即 $\mathbb{R}$ 的不可约表示都是 $1$ 维的。
因此,把表示 $t\mapsto \exp(tA)$ 写成一些不可约表示的直和等价于把矩阵 $A$ 对角化。
所以 $\mathbb{R}$ 的表示完全可约当且仅当矩阵 $A$ 可对角化。

\begin{lemma}
  令 $\rho:G\to\GL(V)$ 是 $G$ (或者 $\mathfrak{g}$)的表示,
  $A:V\to V$ 是可对角化的 $G$-线性映射。令 $V_\lambda\subseteq V$
  是 $A$ 的特征值 $\lambda$ 的特征空间,那么 $V_\lambda$ 是一个子表示,
  所以 $V=\bigoplus V_\lambda$ 是表示的一个直和分解。
\end{lemma}
\begin{proof}
  任取 $v\in V_\lambda$,那么 $Av=\lambda v$。对于任意 $g\in G$,有
  \[ 
    A(\rho(g)v)=\rho(g)(Av)=\rho(g)(\lambda v)=\lambda (\rho(g)v),
  \]
  所以 $\rho(g)v\in V_\lambda$,也即 $V_\lambda$ 是一个子表示。
  $A$ 可对角化意味着 $V=\bigoplus V_\lambda$,所以这是一个直和分解。
\end{proof}

\begin{lemma}[Schur 引理]
  \mbox{}
  \begin{enumerate}
    \item 令 $V$ 是 $G$ 的一个不可约复表示,那么 $\Hom_G(V,V)=\mathbb{C}\mathrm{id}$,
    即 $G$ 的不可约表示的自同态只有数乘同态。
    \item 如果 $V$ 和 $W$ 是两个不同构的不可约复表示,那么 $\Hom_G(V,W)=0$。
  \end{enumerate}
  类似的结果对于李代数 $\mathfrak{g}$ 也成立。
\end{lemma}
\begin{proof}
  如果 $\Phi:V\to W$ 是 $G$-线性映射,那么 $\ker\Phi$ 和 $\mathrm{im}\,\Phi$
  分别是 $V$ 和 $W$ 的子表示。如果 $V$ 不可约,那么 $\ker\Phi$ 是 $V$ 或者 $0$,
  所以 $\Phi$ 要么是 $0$,要么是单射。同样地,如果 $W$ 不可约,那么 $\mathrm{im}\,\Phi$
  是 $0$ 或者 $W$,所以 $\Phi$ 要么是 $0$,要么是满射。
  
  因此,如果 $V$ 和 $W$ 不同构并且都不可约,那么 $\Phi$ 只能是 $0$,所以 $\Hom_G(V,W)=0$。
  如果 $V$ 不可约,那么 $\Phi:V\to V$ 要么是 $0$,要么是同构。设 $\lambda$ 是 $\Phi$ 的一个特征值,
  那么 $\Phi-\lambda \mathrm{id}$ 不是同构,所以只能是 $0$,也即 $\Phi=\lambda \mathrm{id}$。
\end{proof}

Schur 引理有下述重要推论。

\begin{proposition}
  如果 $G$ 是交换群,那么 $G$ 的任意不可约复表示都是 $1$ 维的。
  类似地,如果 $\mathfrak{g}$ 是交换李代数,那么 $\mathfrak{g}$ 的任意不可约复表示都是 $1$ 维的。
\end{proposition}
\begin{proof}
  设 $\rho:G\to\GL(V)$ 是不可约复表示,任取 $g\in G$,那么 $\rho(g):V\to V$
  是 $G$-线性映射。因为对于任意 $h\in G$ 和 $v\in V$,
  都有 $\rho(g)(\rho(h)v)=\rho(gh)v=\rho(hg)v=\rho(h)(\rho(g)v)$。
  根据 Schur 引理,$\rho(g)=\lambda_g \mathrm{id}$,其中 $\lambda_g\in \mathbb{C}$。
  而 $V$ 不可约,所以 $V$ 是 $1$ 维的。
\end{proof}

还是考虑 $G=\mathbb{R}$。我们已经知道 $\mathbb{R}$ 的不可约表示都是 $1$ 维的。
实际上,任意 $1$ 维表示对应的李代数表示都是形如 $a\mapsto \lambda a$ 的映射,
其中 $\lambda\in \mathbb{C}$。因此,$\mathbb{R}$ 的任意不可约表示都形如 $V_\lambda$,
其中 $\lambda\in \mathbb{C}$ 并且 $V_\lambda$ 是 $1$ 维的复向量空间,
附带表示作用 $\rho(a)=e^{\lambda a}$。

类似地,我们可以描述 $\mathbb{S}^1=\mathbb{R}/\mathbb{Z}$ 的不可约表示。
此时,任意表示 $\tau:\mathbb{S}^1\to \GL(V)$ 都可以通过商映射
$\pi:\mathbb{R}\to \mathbb{S}^1$ 提升为 $\mathbb{R}$ 的表示 $\tau\circ\pi$。
并且对于任意 $a\in \mathbb{Z}$ 有 $\tau\circ\pi(a)=\tau(1)=\mathrm{id}$。
所以说 $\mathbb{S}^1$ 的表示必须是 $\mathbb{R}$ 的满足 $\rho(a)=\mathrm{id}$ 的表示。
因此,$\mathbb{S}^1$ 的不可约表示都是 $1$ 维的,形如 $V_k$,
其中 $k\in \mathbb{Z}$ 并且 $V_k$ 是 $1$ 维复向量空间,附带表示作用
$\tau\circ\pi(x)=e^{2\pi ikx}$,或者说 $\tau(z)=z^k$。

\section{Peter-Weyl 定理}

\begin{definition}
  一个李群 $G$ 的复表示 $V$ 如果存在一个 $G$-不变的 Hermite 内积,
  即对于任意 $g\in G$ 和 $v,w\in V$,都有
  \[
    \langle \rho(g)v,\rho(g)w\rangle=\langle v,w\rangle,
  \]
  那么称 $V$ 是{酉表示}。类似地,一个李代数 $\mathfrak{g}$ 的复表示 $V$
  如果存在一个 $\mathfrak{g}$-不变的 Hermite 内积
  \[
    \langle \rho(x)v,w\rangle + \langle v,\rho(x)w\rangle=0,\quad \forall x\in \mathfrak{g},\, v,w\in V,
  \]
  那么称 $V$ 是{酉表示}。
\end{definition}

酉表示的重要性体现在下面的定理。

\begin{theorem}
  每个酉表示都是完全可约的。
\end{theorem}
\begin{proof}
  对 $V$ 的维数归纳。对于 $\dim V=1$ 的情况,显然成立。假设对于 $\dim V<n$ 的情况成立。
  若 $\dim V=n$,如果 $V$ 是不可约的,那么结论成立。否则,存在非平凡子表示 $W\subseteq V$,
  那么 $V=W\oplus W^\bot$。如果 $v\in W^\bot$,那么任取 $w\in W$,有
  $\langle gv,w\rangle=\langle v,g^{-1}w\rangle=0$,所以 $gv\in W^\bot$,
  所以 $W^\bot$ 也是 $V$ 的子表示。根据归纳假设,$W$ 和 $W^\bot$ 都是完全可约的,
  所以 $V$ 也是完全可约的。对于李代数的情况同理。
\end{proof}

对于有限群而言,可以证明有限群的任意表示都是酉表示。其中需要对 $G$
上的函数平均化处理得到一个 $G$-不变的内积,从而得到酉表示结构。
然而,对于无限群,这种平均并不能通过求和完成,所以我们需要引入 Haar 测度。

\begin{definition}
  李群 $G$ 上的一个右 Haar 测度指的是 $G$ 上的一个 Borel 测度 $\d g$,
  其在右平移作用下不变:即对于任意 Borel 集合 $E\subseteq G$ 和任意 $h\in G$,
  都有 $\d g(Eh)=\d g(E)$。
\end{definition}

显然右 Haar 测度等价于对于任意 $h\in G$ 和可积函数 $f$,有
$\int f(gh)\d g=\int f(g)\d g$。类似地可以定义左 Haar 测度。

对于紧李群而言,从微分几何中的知识可知,$G$ 上存在唯一一个
双不变(同时左不变和右不变)的体积形式 $\omega$,使得 $\int_G\omega=1$。
这个体积形式可以构造一个双不变的 Haar 测度 $\d g$,满足对于任意连续函数 $f$,都有
$\int_G f\d g=\int_G f\omega$。

\begin{theorem}
  令 $G$ 是一个紧李群,那么 $G$ 的任意表示都是酉表示并且是完全可约的。
\end{theorem}
\begin{proof}
  令 $B(v,w)$ 是 $V$ 上的一个 Hermite 内积,将其平均化,定义
  \[
    \tilde{B}(v,w)=\int_G B(gv,gw)\d g.
  \]
  根据 Haar 测度的性质,$\tilde{B}$ 是一个 $G$-不变的 Hermite 内积,所以 $V$ 是酉表示。
\end{proof}

利用这个定理,紧李群的任意表示都可以写成不可约表示的直和
$V\simeq \bigoplus n_iV_i$,其中 $n_i\in \mathbb{Z}$,$V_i$ 是两两不同构的不可约表示。
下面我们研究如何构造这个直和分解。

令 $v_i$ 是 $V$ 的一组基,将映射 $\rho(g):V\to V$ 在这组基下
表示为矩阵 $\rho(g)=(\rho_{ij}(g))$。也就是说,我们考虑 $G$
上的标量函数 $\rho_{ij}:G\to \mathbb{C}$,这被称为矩阵系数。

\begin{theorem}
  \mbox{}
  \begin{enumerate}
    \item 令 $V,W$ 是 $G$ 的不同构的不可约表示。选取基 $v_i\in V\ (i=1,\dots,n)$
    和 $w_a\in W\ (a=1,\dots,m)$。那么矩阵系数 $\rho_{ij}^V$ 和 $\rho_{ab}^W$
    是正交的:$\langle \rho_{ij}^V,\rho_{ab}^W\rangle=0$,其中 $C^\infty(G,\mathbb{C})$
    上的内积定义为
    \[
      \langle f,h\rangle=\int_G f(g)\overline{h(g)}\d g.
    \]
    \item 令 $V$ 是 $G$ 的不可约表示。选取基 $v_i\in V$ 是一个
    相对于某个 $G$-不变内积的正交基。那么矩阵系数 $\rho_{ij}^V$ 是两两正交的
    并且每个的模长平方为 $1/{\dim V}$:
    \[
      \langle \rho_{ij}^V,\rho_{ab}^V\rangle=\frac{1}{\dim V}\delta_{ia}\delta_{jb}.
    \]
  \end{enumerate}
\end{theorem}










\end{document}