\documentclass[fontset=none,zihao=-4]{Notes}

\makeatletter
\DeclareRobustCommand{\em}{%
  \@nomath\em \if b\expandafter\@car\f@series\@nil
  \normalfont \else \bfseries \fi}
\makeatother

\usepackage{tikz-cd,wrapstuff}
\usepackage{fixdif,siunitx,tikz,nicematrix,tabularray}

\usetikzlibrary{hobby,calc,arrows}
\usetikzlibrary{positioning}
\usetikzlibrary{decorations.markings}
\usetikzlibrary{decorations.pathreplacing}

\ProvidesFile{font.def}

\setCJKmainfont{Source Han Serif SC}[
  UprightFont=*-Regular,
  BoldFont=*-Bold,
  ItalicFont=HYKaiTi S,
  ItalicFeatures={Scale=1.1}
]
\newCJKfontfamily[zhsong]\songti{Source Han Serif SC}[
  UprightFont=*-Regular,
  BoldFont=*-Bold,
  ItalicFont=HYKaiTi S,
  ItalicFeatures={Scale=1.1}
]
\setCJKsansfont{Source Han Sans SC}[
  UprightFont=*-Regular,
  BoldFont=*-Bold
]
\newCJKfontfamily[zhhei]\heiti{Source Han Sans SC}[
  UprightFont=*-Regular,
  BoldFont=*-Bold
]
\setCJKmonofont{HYFangSong S}[
  BoldFont=*,
  ItalicFont=*,
  BoldItalicFont=*
]
\newCJKfontfamily[zhfs]\fangsong{HYFangSong S}[
  BoldFont=*,
  ItalicFont=*,
  BoldItalicFont=*
]
\newCJKfontfamily[zhkai]\kaishu{HYKaiTi S}[
  BoldFont=*,
  ItalicFont=*,
  BoldItalicFont=*
]

\setmainfont{texgyretermes}[
  Extension=.otf,
  UprightFont=*-regular,
  BoldFont=*-bold,
  ItalicFont=*-italic,
  BoldItalicFont=*-bolditalic,
  SlantedFont=*-italic
]
%\setmathrm{texgyretermes}[
%  Extension=.otf,
%  UprightFont=*-regular,
%  BoldFont=*-bold,
%  ItalicFont=*-italic,
%  BoldItalicFont=*-bolditalic,
%  SlantedFont=*-italic
%]
\setsansfont{Cantarell}[
  UprightFont=* Regular,
  ItalicFont=* Italic,
  BoldFont=* Bold,
  BoldItalicFont=* Bold Italic,
  SmallCapsFont=Alegreya Sans SC
]
\setmonofont{Ubuntu Mono}[
  UprightFont=*,
  ItalicFont=* Italic,
  BoldFont=* Bold,
  BoldItalicFont=* Bold Italic
]
%\setmathfont{texgyretermes-math.otf}
%\setmathfont[range={\mathcal,\mathbfcal,\mathfrak},StylisticSet=1]{XITSMath-Regular.otf}
%\setmathfont[range={\mathbb}]{KpMath-Sans.otf}



\DeclareMathOperator\Int{Int}
\DeclareMathOperator\supp{supp}
\DeclareMathOperator\im{im}
\DeclareMathOperator\End{End}
\DeclareMathOperator\Ann{Ann}
\DeclareMathOperator\Hom{Hom}
\DeclareMathOperator\diag{diag}
\DeclareMathOperator\Or{O}
\DeclareMathOperator\re{Re}
\DeclareMathOperator\tr{tr}

\newcommand{\inn}[1]{\left\langle#1\right\rangle}
\newcommand{\norm}[1]{\left\lVert#1\right\rVert}
\newcommand{\spa}[1]{\operatorname{span}\left(#1\right)}


\tikzcdset{
  arrow style=tikz,
  diagrams={>={Straight Barb[scale=0.8]}}
}

\tikzset{
  point/.style={
    circle,fill,inner sep=0pt,minimum width=5pt
  }
}

\usepackage[subscriptcorrection,nofontinfo,mtpbb]{mtpro2}



\setlist[enumerate]{nosep,label=(\arabic*)}
\setlist[itemize]{nosep}

\title{\sffamily 矩阵范数}
\author{Eliauk}


\begin{document}

\maketitle

\tableofcontents

\section{向量空间的范数}

本文的 $\mathbb{F}$ 均指代域 $\mathbb{R}$ 或者 $\mathbb{C}$。
对于一个 $\mathbb{F}$ 上的向量空间 $V$,如果函数 $\norm{\cdot}:V\to\mathbb{R}$
满足:
\begin{enumerate}
  \item 任取 $v\in V$,有 $\norm{v}\geq 0$,且 $\norm{v}=0$ 当且仅当 $v=0$;
  \item 任取 $c\in\mathbb{F}$,$v\in V$,有 $\norm{cv}=|c|\norm{v}$;
  \item 任取 $u,v\in V$,有 $\norm{u+v}\leq\norm{u}+\norm{v}$,
\end{enumerate}
那么 $\norm{\cdot}$ 被称为 $V$ 上的\emph{范数}。一个配备了范数的向量空间
被称为\emph{赋范向量空间}。Cauchy-Schwarz 不等式告诉我们 $V$ 上的范数可以自然地来源于内积。

\begin{theorem}[Cauchy-Schwarz 不等式]
  设 $V$ 是内积空间,那么
  \[
    \left|\inn{u,v}\right|^2\leq\inn{u,u}\inn{v,v}\quad \forall u,v\in V.
  \]
  等号成立当且仅当 $u,v$ 线性相关,即存在 $c\in\mathbb{F}$ 使得 $u=cv$
  或者 $v=cu$。
\end{theorem}
\begin{proof}
  任取 $u,v\in V$,如果 $v=0$,显然不等式成立。设 $v\neq 0$,
  我们将 $u$ 向 $v$ 做投影,即
  \[
    u=\frac{\inn{u,v}}{\inn{v,v}}v+w,  
  \]
  此时 $\inn{w,v}=0$。故
  \[
    \inn{u,u}=\frac{\left|\inn{u,v}\right|^2}{\inn{v,v}^2}\inn{v,v}+\inn{w,w}
    \geq \frac{\left|\inn{u,v}\right|^2}{\inn{v,v}},
  \]
  即
  \[
    \left|\inn{u,v}\right|^2\leq\inn{u,u}\inn{v,v}.
  \]
  等号成立当且仅当 $\inn{w,w}=0$,当且仅当 $w=0$,当且仅当 $u,v$ 线性相关。
\end{proof}

\begin{corollary}
  设 $V$ 是内积空间,定义 $\norm{\cdot}:V\to\mathbb{R}$ 为
  \[
    \norm{v}=\sqrt{\inn{v,v}} , 
  \]
  那么这是一个范数。
\end{corollary}
\begin{proof}
  任取 $u,v\in V$,由 Cauchy-Schwarz 不等式,有 
  \begin{align*}
    \norm{u+v}^2&=\inn{u+v,u+v}  =\inn{u,u}+\inn{v,v}+2\re\inn{u,v}\\
    &\leq \inn{u,u}+\inn{v,v}+2\left|\inn{u,v}\right|\\
    &\leq \inn{u,u}+\inn{v,v}+2\norm{u}\norm{v}\\
    &=\left(\norm{u}+\norm{v}\right)^2.\qedhere
  \end{align*}
\end{proof}

上述范数 $\norm{v}=\sqrt{\inn{v,v}}$ 被称为\emph{内积诱导的范数},在不另外说明
的情况下,内积空间的范数均指代内积诱导的范数。
那么是否范数都能由某一个内积导出?我们有下面的结论,这个结论的证明比较繁琐,
这里不予说明。

\begin{theorem}\label{thm:parallelogram law}
  对于 $\mathbb{F}$ 上的赋范向量空间 $V$,范数 $\norm{\cdot}$ 是由某个内积诱导的当且仅当
  其满足平行四边形恒等式:
  \[
    \frac{1}{2}\left(\norm{u+v}^2+\norm{u-v}^2\right)=\norm{u}^2+\norm{v}^2
    \quad \forall u,v\in V.
  \]
\end{theorem}
% \begin{proof}
%   $(\Rightarrow)$ 若 $\norm{u}=\sqrt{\inn{u,u}}$,直接代入即可。

%   $(\Leftarrow)$ 若范数满足平行四边形恒等式。我们首先考虑 $\mathbb{F}=\mathbb{R}$
%   的情况,定义
%   \[
%     \inn{u,v}= \frac{1}{2}\left(\norm{u+v}^2-\norm{u}^2-\norm{v}^2\right),  
%   \]
%   % 此时不难验证内积的正定性和对称性。下面我们证明双线性性。利用平行四边形恒等式,我们有
%   % \[
%   %   \frac{1}{2}\left(\norm{u+w+2v}^2+\norm{u+w}^2\right)=\norm{u+w+v}^2+\norm{v}^2,
%   % \]
%   % 所以
%   % \begin{align*}
%   %   \inn{u+w,v}&=\frac{1}{2}\left(\norm{u+w+v}^2-\norm{u+w}^2-\norm{v}^2\right)\\
%   %   &=\frac{1}{4}\left(\norm{u+w+2v}^2-\norm{u+w}^2-4\norm{v}^2\right)\\
%   %   &=\frac{1}{4}\left(\norm{u+w+2v}^2-\norm{u+w}^2-\norm{2v}^2\right)\\
%   %   &=\frac{1}{4}\left(\norm{u+v+w+v}^2-\norm{u+w}^2-\norm{2v}^2\right)
%   % \end{align*}
% \end{proof}

在 $\mathbb{C}^n$ 上最常见的范数是 $l_p$-范数,即
对于 $x=(x_1,\dots,x_n)\in\mathbb{C}^n$,定义
\begin{equation*}
  \norm{x}_p=\left(|x_1|^p+\cdots+|x_n|^p\right)^{1/p},
\end{equation*}
其中 $p\geq 1$,该范数的三角不等式由 Minkowski 不等式保证。
令 $p\to\infty$,可以定义 $l_\infty$-范数:
\[
  \norm{x}_{\infty}=\max\{|x_1|,\dots,|x_n|\}  ,
\]
不难验证其三角不等式成立。值得一提的是,当 $n\geq 2$ 的时候 ($n=1$ 时所有的 $l_p$-范数都是相同的),
考虑 $x=(1,0,0,\dots,0),y=(0,1,0,\dots,0)$,那么
\[
  \frac{1}{2}\left(\norm{x+y}_p^2+\norm{x-y}_p^2\right)=2^{2/p},
  \quad \norm{x}_p^2+\norm{y}_p^2=2,
\]
所以只有 $p=2$ 时,$l_p$-范数才可能满足平行四边形恒等式,根据 \autoref{thm:parallelogram law},
$l_p$-范数只有在 $p=2$ 时才能从内积诱导出来。

对于一个赋范向量空间 $V$,可以定义度量 $d:V\times V\to\mathbb{R}$ 为
\[
  d(u,v)=\norm{u-v},
\]
这使得 $V$ 成为一个度量空间,从而拥有自然地拓扑结构。那么不同的范数诱导出的拓扑结构
是否相同?这就引出了范数等价的概念。对于复向量空间 $V$ 上的两个范数 $\norm{\cdot}_a$
和 $\norm{\cdot}_b$,如果存在正实数 $C_1\leq C_2$ 使得
\[
  C_1\norm{v}_b\leq \norm{v}_a\leq C_2\norm{v}_b\quad \forall v\in V,  
\]
那么我们说范数 $\norm{\cdot}_a$ 和 $\norm{\cdot}_b$ 是\emph{等价的}。
不难验证这个关系满足自反性、对称性和传递性,所以这是一个等价关系。

\begin{proposition}\label{prop:equiv norm give same topology}
  对于复向量空间 $V$ 上的两个范数 $\norm{\cdot}_a$ 和 $\norm{\cdot}_b$,
  这两个范数给出相同的拓扑结构当且仅当它们是等价的。
\end{proposition}
\begin{proof}
  $(\Rightarrow)$ 若 $\norm{\cdot}_a$ 和 $\norm{\cdot}_b$ 给出相同的拓扑结构,记
  这两个范数对应的度量分别为 $d_a$ 和 $d_b$,开球
  \[
    B_r^{(a)}(x_0)=\{x\in V\,|\, d_a(x,x_0)< r\},\quad
    B_r^{(b)}(x_0)=\{x\in V\,|\, d_b(x-x_0)< r\}.
  \]
  $(V,d_a)$ 和 $(V,d_b)$ 是相同的拓扑空间表明
  $B_1^{(a)}(0)$ 是 $(V,d_b)$ 中的开集,$B_1^{(b)}(0)$ 是 $(V,d_a)$ 中的开集,所以
  存在 $r_1>0$ 和 $r_2>0$ 使得
  \[
    B_{r_1}^{(b)}(0)\subseteq B_1^{(a)}(0),\quad B_{r_2}^{(a)}(0)\subseteq B_1^{(b)}(0),
  \]
  任取 $0\neq x\in V$,那么 $y=r_1x/(2\norm{x}_b)$ 满足 $\norm{y}_b=r_1/2<r_1$,
  所以 $y\in B_{r_1}^{(b)}(0)\subseteq B_1^{(a)}(0)$,即
  \[
    \frac{r_1}{2}\frac{\norm{x}_a}{\norm{x}_b}=\norm{y}_a \leq 1,
  \]
  即 $\norm{x}_a\leq 2/r_1 \norm{x}_b$。类似地,可以证明
  \[
    \frac{r_2}{2}\norm{x}_b\le \norm{x}_a\leq \frac{2}{r_1}\norm{x}_b,  
  \]
  总可以令 $r_1,r_2<1$,所以 $\norm{\cdot}_a$ 和 $\norm{\cdot}_b$ 是等价的。

  $(\Leftarrow)$ 若 $\norm{\cdot}_a$ 和 $\norm{\cdot}_b$ 是等价的。即存在正实数 $C_1\leq C_2$ 使得
  \[
    C_1\norm{v}_b\leq \norm{v}_a\leq C_2\norm{v}_b\quad \forall v\in V.
  \]
  若 $X$ 是 $(V,d_a)$ 中的开集,即任取 $x\in X$,都存在 $r>0$ 使得
  \[
    x\in B_r^{(a)}(x)\subseteq X,  
  \]
  令 $r'=r/C_2$,那么
  $\norm{y-x}_b\leq r'$ 就能表明 $\norm{y-x}_a\leq C_2\norm{y-x}_b\leq r$,所以
  \[
    x\in B_{r'}^{(b)}(x)\subseteq   B_r^{(a)}(x)\subseteq X,
  \]
  这表明 $X$ 是 $(V,d_b)$ 中的开集。同理,可证 $(V,d_b)$ 中的开集也是 $(V,d_a)$
  中的开集,故二者的拓扑结构相同。
\end{proof}

\autoref{prop:equiv norm give same topology} 告诉我们等价的范数给出相同的拓扑,
从而在这样的赋范向量空间中,序列的收敛性、函数的连续性等相关概念都是和所用的范数无关的。
下面的定理告诉我们,有限维向量空间上的任意两个范数都是等价的。

\begin{theorem}\label{thm:equivalence of norm}
  $V$ 是有限维实或者复向量空间,$\norm{\cdot}_a$ 和 $\norm{\cdot}_b$ 是 $V$ 上任意两个范数,
  那么 $\norm{\cdot}_a$ 和 $\norm{\cdot}_b$ 是等价的。 
\end{theorem}
\begin{proof}
  设 $V$ 的一组基为 $e_1,\dots,e_n$。任取 $x=x_1e_1+\cdots+x_ne_n$,定义 $l_1$-范数为
  \[
    \norm{x}_1=|x_1|+\cdots+|x_n|,  
  \]
  容易证明这是一个范数。根据范数等价的传递性,我们只需要证明任意范数 $\norm{\cdot}_a$
  都和 $\norm{\cdot}_1$ 等价即可。即存在正数 $C_1\leq C_2$ 使得
  \[
    C_1\norm{x}_1\leq \norm{x}_a\leq C_2\norm{x}_1.  
  \]
  显然 $x=0$ 时成立,下面假设 $x\neq 0$。此时等价于证明
  \[
    C_1\leq \norm{\frac{x}{\norm{x}_1}}_a  \leq C_2,
  \]
  即证明:对于任意的满足 $\norm{u}_1=1$ 的 $u\in V$,存在正数 $C_1\leq C_2$ 使得
  \[
    C_1\leq \norm{u}_a\leq C_2.  
  \]

  现在考虑 $V$ 上的度量 $d_1(x,y)=\norm{x-y}_1$,此时 $(V,d_1)$ 成为度量空间。
  将 $\norm{\cdot}_a$ 视为 $V\to\mathbb{R}$ 的函数,记 $S$ 是 $(V,d_1)$ 中的单位圆:
  \[
    S=  \{x\in V\,|\, \norm{x}_1=1\}.
  \]
  此时 $S$ 是紧集,如果我们能说明 $\norm{\cdot}_a$ 是连续函数,那么就表明
  $\norm{\cdot}_a$ 限制在 $S$ 上是有界的,从而完成证明。所以下面我们说明 $\norm{\cdot}_a$
  是连续函数。

  给定 $x=x_1e_1+\cdots+x_ne_n\in V$,对于任意的 $\epsilon>0$,取 $\delta=\epsilon$,
  令 $y=y_1e_1+\cdots+y_ne_n\in V$,只要 $\norm{y-x}_1\leq\delta$,就有
  \[
    \left|\norm{y}_a-\norm{x}_a\right|  \leq\norm{y-x}_a
    \leq \sum_{i=1}^n |y_i-x_i|\norm{e_i}_a\leq 
    \left(\max_i\norm{e_i}_a\right)\norm{y-x}_1\leq \left(\max_i\norm{e_i}_a\right)\varepsilon,
  \]
  这就证明了 $\norm{\cdot}_a$ 是连续函数。
\end{proof}

这告诉我们对于有限维赋范向量空间而言,无论其采用何种范数,导出的拓扑结构
都是相同的,这一点使得我们在处理一些问题的时候可以选取更容易处理的范数。下面就是一个例子。

\begin{proposition}
  令 $V,W$ 是两个有限维赋范向量空间,令 $\varphi:V\to W$ 是线性映射,
  那么 $\varphi$ 一定连续。
\end{proposition}
\begin{proof}
  设 $e_1,\dots,e_n$ 是 $V$ 的一组基,对于 $x=x_1e_1+\cdots+x_ne_n$,定义 $V$
  上的范数 $\norm{x}=|x_1|+\cdots+|x_n|$。
  根据 \autoref{thm:equivalence of norm} 和 \autoref{prop:equiv norm give same topology},
  我们说明 $\varphi$ 在这个意义下连续即可。任取 $y=y_1e_1+\cdots+y_ne_n$,那么
  \[
    \norm{\varphi(x)-\varphi(y)}=\norm{\sum_{i=1}^n(x_i-y_i)\varphi(e_i)}
    \leq \sum_{i=1}^n|x_i-y_i|\norm{\varphi(e_i)}\leq \norm{x-y}\sum_{i=1}^n\norm{\varphi(e_i)},  
  \]
  这就表明 $\varphi$ 是连续映射。
\end{proof}

\section{矩阵范数}

令 $M_n(\mathbb{F})$ 表示 $\mathbb{F}$ 上的 $n$ 阶矩阵构成的向量空间,
此时其同构于向量空间 $\mathbb{F}^{n^2}$,所以可以自然地采用 $\mathbb{F}^{n^2}$
上的范数,但是 $M_n(\mathbb{F})$ 有重要的乘法结构,所以对于矩阵范数,我们
定义一个额外的条件。

如果函数 $\norm{\cdot}:M_n(\mathbb{F})\to\mathbb{R}$ 满足:
\begin{enumerate}
  \item 任取 $A\in M_n(\mathbb{F})$,有 $\norm{A}\geq 0$,且 $\norm{A}=0$ 当且仅当 $A=0$;
  \item 任取 $c\in\mathbb{F}$,$A\in M_n(\mathbb{F})$,有 $\norm{cA}=|c|\norm{A}$;
  \item 任取 $A,B\in M_n(\mathbb{F})$,有 $\norm{A+B}\leq\norm{A}+\norm{B}$;
  \item 任取 $A,B\in M_n(\mathbb{F})$,有 $\norm{AB}\leq\norm{A}\norm {B}$,
\end{enumerate}
那么 $\norm{\cdot}$ 被称为\emph{矩阵范数}。根据定义,总是有
$\norm{A^2}\leq\norm{A}^2$,所以对于非零幂等矩阵 $A^2=A$ 来说,总是有 $\norm{A}\geq 1$。
特别地,单位矩阵的范数 $\norm{I}\ge 1$。若 $A$ 是可逆矩阵,那么
$\norm{I}\leq\norm{A}\norm{A^{-1}}$,所以我们有逆矩阵范数的一个下界:
\[
  \norm{A^{-1}}\geq\frac{\norm{I}}{\norm{A}}.  
\]

我们将矩阵视为 $n^2$ 维向量,自然可以继承向量的 $l_p$-范数,此时我们只需要验证
矩阵范数的条件 (4) 即可。

\begin{example}
  矩阵 $A\in M_n(\mathbb{F})$ 的 $l_1$-范数定义为
  \[
    \norm{A}_{[1]}=\sum_{i,j=1}^n|a_{ij}|,  
  \]
  此时
  \begin{align*}
    \norm{AB}_{[1]}&=\sum_{i,j=1}^n\left|\sum_{k=1}^n a_{ik}b_{kj}\right|\leq
    \sum_{i,j=1}^n\sum_{k=1}^n|a_{ik}b_{kj}|\\
    &\leq \sum_{i,j=1}^n\sum_{k,\ell=1}^n|a_{ik}b_{\ell j}|=
    \left(\sum_{i,k=1}^n|a_{ik}|\right)\left(\sum_{j,\ell=1}^n|b_{\ell j}|\right)
    \\
    &=\norm{A}_{[1]}\norm{B}_{[1]},
  \end{align*}
  所以 $l_1$-范数是矩阵范数。这里我们使用 $\norm{A}_{[1]}$ 而不是 $\norm{A}_1$ 来表示
  $l_1$-范数是因为将 $\norm{A}_1$ 的记号留给后面的算子范数,算子范数更加常用。
\end{example}

\begin{example}
  矩阵 $A\in M_n(\mathbb{F})$ 的 $l_2$-范数(也被称为 Frobenius 范数)定义为
  \[
    \norm{A}_{F}=\left(\tr (A^*A)\right)^{1/2}=\left(\sum_{i,j=1}^n|a_{ij}|^2\right)^{1/2},  
  \]
  此时
  \begin{align*}
    \norm{AB}_F&=\left(\sum_{i,j=1}^n\left|\sum_{k=1}^n a_{ik}b_{kj}\right|^2\right)^{1/2}
    \leq \left(\sum_{i,j=1}^n\left(\sum_{k=1}^n|a_{ik}|^2\right)\left(\sum_{\ell=1}^n|b_{\ell j}|^2\right)\right)^{1/2}\\
    &=\left(\sum_{i,k=1}^n|a_{ik}|^2\right)^{1/2}\left(\sum_{j,\ell=1}^n|b_{\ell j}|^2\right)^{1/2}
    =\norm{A}_F\norm{B}_F,
  \end{align*}
  所以 $l_2$-范数是矩阵范数。注意到 $A^*A$ 的迹是 $A^*A$ 的特征值之和,所以 $\norm{A}_F$
  是 $A$ 的奇异值的平方和的平方根。此外,容易证明
  \[
    \norm{A}_F=\norm{A^*}_F,\quad \norm{A}_F=\norm{PAQ}_F,  
  \]
  其中 $P,Q$ 是酉(正交)矩阵。
\end{example}

\begin{example}
  向量的 $l_\infty$-范数推广到矩阵为
  \[
    \norm{A}_{[\infty]}=\max_{i,j}|a_{ij}|,  
  \]
  但是这不是一个矩阵范数,考虑
  \[
    A=\begin{pmatrix}
      1 & 1 & \cdots & 1\\
      1 & 1 & \cdots & 1\\
      \vdots & \vdots & \ddots & \vdots \\
      1 & 1 & \cdots & 1
    \end{pmatrix}  ,\quad A^2=nA,
  \]
  但是 $\norm{A^2}_{[\infty]}=n\geq 1=\norm{A}_{[\infty]}^2$,不满足矩阵范数的条件 (4)。
\end{example}

下面我们定义一个非常重要的矩阵范数,通常被称为\emph{算子范数}。令 $\norm{\cdot}$
为 $\mathbb{F}^n$ 上的向量范数,我们定义 $M_n(\mathbb{F})$ 上的矩阵范数为
\[
  \norm{A}=\sup\bigl\{\norm{Ax}\,|\, x\in \mathbb{F}^n,\norm{x}\leq 1\bigr\}.  
\]
% 这个定义目前并不一定是良定义的,因为在无穷集合上不一定存在最大值,所以我们先说明
% 确实存在这样的 $x$ 使得 $\norm{Ax}$ 最大。
由于 $\{x\in\mathbb{F}^n\,|\, \norm{x}\leq 1\}$
是度量空间 $\mathbb{F}^n$ 中的有界闭集,从而是紧集,所以连续函数 $x\mapsto \norm{Ax}$
是有界的,并且能取到最大值,所以定义中的 $\sup$ 可以改为 $\max$。

\begin{proposition}
  矩阵 $A$ 的算子范数的等价形式:
  \[
    \norm{A}=\sup_{\norm{x}\leq 1}\norm{A x}=\sup_{\norm{x}=1}\norm{Ax}
    =\sup_{x\neq 0}\frac{\norm{Ax}}{\norm{x}}=\inf\bigl\{\,c\,|\, \norm{Ax}\leq c\norm x,\forall x\in \mathbb{F}^n\bigr\}.
  \]
\end{proposition}
\begin{proof}
  记右端四个值分别为 $S_1,S_2,S_3,S_4$,显然 $S_2\leq S_1$。因为
  \[
    \frac{\norm{Ax}}{\norm x}=\norm{A\left(\frac{x}{\norm x}\right)}\leq S_2,  
  \]
  所以 $S_3\leq S_2$。当 $\norm{x}\leq 1$ 的时候,有
  \[
    \norm{Ax}\leq \frac{\norm{Ax}}{\norm x}\leq S_3,  
  \]
  所以 $S_1\leq S_3$。这就表明 $S_1=S_2=S_3$。注意到
  \[
    \norm{Ax}\leq S_3\norm{x}\quad \forall x\in \mathbb{F}^n,  
  \]
  所以 $S_4\leq S_3$。根据 $\sup$ 的定义,对于任意的正整数 $n$,都存在
  $x_n\in \mathbb{F}^n$ 使得
  \[
    S_4\geq \norm{Ax_n}/\norm{x_n}\geq S_3-1/n,  
  \]
  这就表明 $S_3=S_4$。
\end{proof}

\begin{theorem}\label{thm:operator norm}
  上述算子范数满足下面的性质:
  \begin{enumerate}
    \item $\norm{I}=1$;
    \item 对于任意的 $A\in M_n(\mathbb{F})$ 和 $v\in\mathbb{F}^n$,有
    $\norm{Av}\leq\norm{A}\norm{v}$;
    \item 算子范数是一种矩阵范数。
  \end{enumerate}
\end{theorem}
\begin{proof}
  (1) 由于
  \[
    \sup_{\norm{x}=1}\norm{Ix}=\sup_{\norm{x}=1}\norm{x}=1,  
  \]
  所以 $\norm{I}=1$。

  (2) $v=0$ 时显然成立。假设 $v\neq 0$,那么
  \[
    \frac{1}{\norm{v}}\norm{Av}=\norm{A\frac{v}{\norm{v}}}\leq\max_{\norm{x}=1}\norm{Ax}=\norm{A},  
  \]
  所以 $\norm{Av}\leq\norm{A}\norm{v}$。

  (3) 矩阵范数显然满足正定性和齐次性。任取单位向量 $x\in\mathbb{F}^n$,那么
  \[
    \norm{(A+B)x}=\norm{Ax+Bx}\leq\norm{Ax}+\norm{Bx}\leq\norm{A}+\norm{B},  
  \]
  所以 $\norm{A+B}\leq\norm{A}+\norm{B}$。此外,根据 (2),还有
  \[
    \norm{(AB)x}=\norm{A(Bx)}\leq \norm{A}\norm{Bx}\leq\norm{A}\norm{B},  
  \]
  所以 $\norm{AB}\leq\norm{A}\norm{B}$。这就表明算子范数是一种矩阵范数。
\end{proof}

所以算子范数 $\norm{A}$ 通常被称为向量空间 $\mathbb{F}^n$ 上的范数 $\norm{\cdot}$ 诱导的矩阵范数。
\autoref{thm:operator norm} 表明向量空间上任意范数都可以诱导出一个矩阵范数。
如果矩阵范数和向量范数满足 \autoref{thm:operator norm} 的 (2),那么我们说
这个矩阵范数和向量范数是相容的。对于任意向量范数,都存在与之相容的矩阵范数
(即诱导的矩阵范数)。


\begin{example}
  我们考察 $\mathbb{F}^n$ 上的 $l_1$-范数诱导的矩阵范数。
  对于矩阵 $A\in M_n(\mathbb{F})$,
  设其列向量为 $A=[\alpha_1,\dots,\alpha_n]$,任取 $x=(x_1,\dots,x_n)$,那么
  \[
    \norm{Ax}_1=\norm{\sum_{i=1}^n x_i\alpha_i}_1\leq \sum_{i=1}^n|x_i|\norm{\alpha_i}_1
    \leq \left(\max_{1\leq i\leq n}\norm{\alpha_i}_1\right)\sum_{i=1}^n|x_i|
    =\norm{x}_1\left(\max_{1\leq i\leq n}\norm{\alpha_i}_1\right),
  \]
  另一方面,我们取 $x=e_i$ 是标准基,那么
  \[
    \sup_{x\neq 0}\frac{\norm{Ax}_1}{\norm{x}_1}  \geq \norm{Ae_i}_1=
    \norm{\alpha_i}_1,
  \]
  这表明
  \[
    \sup_{x\neq 0}\frac{\norm{Ax}_1}{\norm{x}_1}  \geq\max_{1\leq i\leq n}\norm{\alpha_i}_1.
  \]
  故 $A$ 的 $1$-范数为:
  \[
    \norm{A} _1=\sup_{x\neq 0}\frac{\norm{Ax}_1}{\norm{x}_1}=\max_{1\leq i\leq n}\norm{\alpha_i}_1
    =\max_{1\leq i\leq n}\sum_{j=1}^n|a_{ji}|.
  \]
  即 $\norm{A}_1$ 为 $A$ 的列向量 $l_1$-范数的最大值。
\end{example}

\begin{example}
  考察 $\mathbb{F}^n$ 上的 $l_\infty$-范数诱导的矩阵范数。任取
  $x=(x_1,\dots,x_n)$,那么
  \[
    \norm{Ax}_\infty=\max_{1\leq i\leq n}\left|\sum_{j=1}^n a_{ij}x_j\right|  
    \leq \max_{1\leq i\leq n}\sum_{j=1}^n|a_{ij}||x_j|\leq \norm{x}_\infty\left(\max_{1\leq i\leq n}\sum_{j=1}^n|a_{ij}|\right),
  \]
  这表明
  \[
    \norm{A}_\infty\leq  \max_{1\leq i\leq n}\sum_{j=1}^n|a_{ij}|. 
  \]
  现在假设 $A\ne 0$,那么设 $a_{ij}\neq 0$。令 $y=(y_1,\dots,y_n)$
  满足
  \[
    y_k=\begin{cases}
      \bar a_{ik}/|a_{ik}| & a_{ik}\neq 0,\\
      1 & a_{ik}=0.
    \end{cases}
  \]
  那么 $\norm y_{\infty}=1$,并且 $a_{ik}y_k=|a_{ik}|$,所以
  \[
    \norm{A}_\infty=\sup_{\norm{x}_\infty=1}\norm{Ax}_\infty\geq \norm{Ay}_\infty
    =\max_{1\leq i\leq n}\left|\sum_{k=1}^na_{ik}y_k\right|
    \geq \left|\sum_{k=1}^na_{ik}y_k\right|\geq \sum_{k=1}^n|a_{ik}|,
  \]
  所以
  \[
    \norm{A}_\infty \geq   \max_{1\leq i\leq n}\sum_{j=1}^n|a_{ij}|. 
  \]
  故 $A$ 的 $\infty$-范数为
  \[
    \norm{A}_\infty=\max_{1\leq i\leq n}\sum_{j=1}^n|a_{ij}|.  
  \]
  即 $\norm{A}_\infty$ 为 $A$ 的行向量 $l_1$-范数的最大值。
\end{example}

\begin{example}\label{exa:matrix 2-norm}
  考察 $\mathbb{F}^n$ 上的 $l_2$-范数诱导的矩阵范数,这个范数被称为\emph{谱范数}。
  设 $s_1,\dots,s_k$ 是 $A$ 的非零奇异值(从大到小排列),根据奇异值分解,那么存在 $\mathbb{F}^n$ 的正交向量组
  $e_1,\dots,e_k$ 和 $f_1,\dots,f_k$ 使得
  \[
    Ax=s_1\inn{x,e_1}f_1+\cdots+s_k\inn{x,e_k}f_k,  
  \]
  所以
  \begin{align*}
    \norm{Ax}_2^2&=s_1^2\left|\inn{x,e_1}\right|^2+\cdots+s_k^2\left|\inn{x,e_k}\right|^2\\
    &\leq s_1^2\left(\left|\inn{x,e_1}\right|^2+\cdots+\left|\inn{x,e_k}\right|^2\right)\leq s_1^2 \norm{x}_2^2,
  \end{align*}
  这表明 $\norm{A}_2\leq s_1$。另一方面,$\norm{Ae_1}_2=\norm{s_1f_1}_2=s_1$,所以
  $\norm{A}_2\geq s_1$,故
  \[
    \norm{A}_2=s_1.  
  \]
  即 $\norm{A}_2$ 为 $A$ 的奇异值的最大值。
\end{example}


\section{谱半径和矩阵级数}

矩阵范数的第一个重要应用在于提供了矩阵谱半径的界。矩阵 $A$ 的谱半径为
其特征值模长的最大值,记为 $\rho(A)$。

\begin{proposition}\label{prop:matrix 2-norm of normal matrix}
  若 $A\in M_n(\mathbb{F})$ 是正规矩阵,那么
  \[
    \rho(A)=\norm{A}_2=s_1,
  \]
  其中 $s_1$ 是 $A$ 的最大的奇异值。
\end{proposition}
\begin{proof}
  将 $A$ 始终看作复正规矩阵,根据谱定理,存在正交矩阵 $P$ 使得
  \[
    P^{-1}AP=\diag(\lambda_1,\dots,\lambda_n),  
  \]
  其中 $|\lambda_1|\geq|\lambda_2|\geq\cdots\geq|\lambda_n|$ 是 $A$ 的所有特征值(注意正规矩阵的特征值都是实数),
  于是 $\rho(A)=|\lambda_1|$。$A$ 和 $P^{-1}AP$ 的奇异值相同,
  而 $P^{-1}AP$ 的奇异值显然为 $|\lambda_1|,\dots,|\lambda_n|$,所以
  $\rho(A)=\norm{A}_2$。
\end{proof}

设 $\lambda$ 是 $A$ 的一个特征值,
$x$ 是 $\lambda$ 的特征向量,考虑矩阵 $X=(x,\dots,x)\in M_n(\mathbb{F})$,
此时有 $AX=\lambda X$。如果 $\norm{\cdot}$ 是一个矩阵范数,那么
\[
  |\lambda|\norm{X}=\norm{\lambda X}=\norm{AX}\leq \norm{A}\norm{X},  
\]
因此 $|\lambda|\leq\norm{A}$,这就表明 $\rho(A)\leq\norm{A}$。
若 $A$ 是可逆矩阵,那么 $\lambda^{-1}$ 是 $A^{-1}$ 的特征值,所以
$|\lambda^{-1}|\leq\norm{A^{-1}}$,即 $|\lambda|\geq 1/\norm{A^{-1}}$。
于是我们得到了下面的结论。

\begin{theorem}\label{thm:upper bound of rho}
  令 $\norm{\cdot}$ 是矩阵范数,$A\in M_n(\mathbb{F})$,$\lambda$ 是 $A$
  的特征值,那么
  \[
    |\lambda|\leq\rho(A)\leq\norm{A},  
  \]
  如果 $A$ 可逆,那么
  \[
    \rho(A)\geq |\lambda|\geq1/\norm{A^{-1}}.  
  \]
\end{theorem}

\begin{theorem}\label{thm:matrix norm can close to rho}
  给定 $A\in M_n(\mathbb{F})$ 和 $\epsilon>0$,存在一个矩阵范数 $\norm{\cdot}$
  使得
  \[
    \norm{A}\leq \rho(A)+\epsilon.  
  \]
\end{theorem}
\begin{proof}
  设 $A$ 的 Jordan 标准型为
  \[
    P^{-1}A{P}=\begin{pmatrix}
      J_{n_1}(\lambda_1) \\
      & J_{n_2}(\lambda_2) \\
      & & \ddots \\
      & & & J_{n_k}(\lambda_k)
    \end{pmatrix} =J ,
  \]
  其中 $\lambda_1,\dots,\lambda_k$ 是 $A$ 的所有特征值,$n_1+\cdots+n_k=n$。令
  \[
    D_{n_i}(\epsilon)=\diag(\epsilon,\epsilon^2,\dots,\epsilon^{n_i}),  
    \quad D=\diag\bigl(D_{n_1}(\epsilon),\dots,D_{n_k}(\epsilon)\bigr),
  \]
  定义 $\mathbb{F}^n$ 上的向量范数
  \[
    \norm{x}_D=\norm{D^{-1}P^{-1}x}_{1},  
  \]
  不难验证这确实是一个范数。考虑其诱导的矩阵范数,有
  \begin{align*}
    \norm{A}_D&=\max_{x\neq 0}\frac{\norm{Ax}_D}{\norm{x}_D}
    =\max_{x\neq 0}\frac{\norm{D^{-1}P^{-1}Ax}_1}{\norm{D^{-1}P^{-1}x}_1}\\
    &=\max_{y\neq 0}\frac{\norm{D^{-1}P^{-1}APDy}_1}{\norm{y}_1}=\norm{D^{-1}JD}_1.
  \end{align*}
  计算得
  \[
    D^{-1}JD= \diag\bigl(B_{n_1}(\lambda_1,\epsilon),\dots,B_{n_k}(\lambda_k,\epsilon)\bigr),
  \]
  其中
  \[
    B_{n_i}(\lambda_i,\epsilon)=\begin{pmatrix}
      \lambda_i & \epsilon \\
      & \lambda_i & \epsilon \\
      & & \ddots & \ddots \\
      & & & \lambda_i & \epsilon \\
      & & & & \lambda_i
    \end{pmatrix}  ,
  \]
  这就表明 $\norm{A}_D\leq \rho(A)+\epsilon$。
\end{proof}

\autoref{thm:upper bound of rho} 和 \autoref{thm:matrix norm can close to rho} 共同表明
\begin{equation}\label{eq:rho equal to inf of matrix norm}
  \rho(A)=\inf\{\norm{A}\,|\, \text{$\norm{\cdot}$ is a matrix norm}\}  .
\end{equation}
这给出了谱半径和矩阵范数的一个重要关系,而谱半径是研究矩阵级数的一个重要工具。

\begin{lemma}\label{lemma:power of A convergent to 0}
  令 $A\in M_n(\mathbb{F})$,如果存在矩阵范数 $\norm{\cdot}$ 使得 $\norm{A}< 1$,
  那么 $\lim_{k\to\infty} A^k=0$,这表明 $A^k$ 的每个元素都趋于 $0$。
\end{lemma}
\begin{proof}
  我们有 $\norm{A^k}\leq\norm{A}^k\to 0$,所以 $A^k$ 在范数 $\norm{\cdot}$
  的意义下收敛到 $0$,\autoref{thm:equivalence of norm} 表明任意两个矩阵范数都是
  等价的,所以 $A^k$ 在任意范数意义下收敛到 $0$,特别地,$A^k\to 0$ 在
  无穷范数 $\norm{\cdot}_\infty$ 意义下成立,即 $A^k$ 的行向量模长的最大值趋于零,
  即 $A^k$ 的每个元素都趋于 $0$。
\end{proof}

\begin{theorem}\label{thm:rho less than 1}
  令 $A\in M_n(\mathbb{F})$,那么 $\lim_{k\to\infty} A^k=0$ 当且仅当
  $\rho(A)<1$。
\end{theorem}
\begin{proof}
  $(\Rightarrow)$ 若 $A^k\to 0$,设 $x$ 是 $A$ 的特征向量,即 $Ax=\lambda x$,
  那么 $\lambda^k x=A^kx\to 0$,故 $\lambda^k\to 0$,这表明 $|\lambda|<1$,故
  $\rho(A)<1$。

  $(\Leftarrow)$ 若 $\rho(A)<1$,\eqref{eq:rho equal to inf of matrix norm} 式表明存在某个矩阵范数 $\norm{\cdot}$ 使得
  $\norm{A}<1$,\autoref{lemma:power of A convergent to 0} 表明 
  $A^k\to 0$。
\end{proof}

\begin{corollary}[Gelfand]
  令 $A\in M_n(\mathbb{F})$,$\norm{\cdot}$ 是一个矩阵范数,那么
  \[
    \rho(A)=\lim_{k\to\infty}\norm{A^k}^{1/k}.  
  \]
\end{corollary}
\begin{proof}
  因为 $\rho(A)^k=\rho(A^k)\leq \norm{A^k}$,所以 $\rho(A)\leq \norm{A^k}^{1/k}$。
  任取 $\epsilon>0$,记 $B=(\rho(A)+\epsilon)^{-1}A$,那么 $\rho(B)=\rho(A)/(\rho(A)+\epsilon)<1$,
  根据 \autoref{thm:rho less than 1},所以 $B^k\to 0$,故 $\norm{B^k}\to 0$,所以存在
  $N$ 使得 $k\geq N$ 时 $\norm{B^k}<1$,此时
  \[
    \norm{A^k}^{1/k}=\norm{(\rho(A)+\epsilon)^kB^k}^{1/k}
    =(\rho(A)+\epsilon)\norm{B^k}^{1/k} <\rho(A)+\epsilon,
  \]
  所以 $k\geq N$ 时有 $\rho(A)\leq\norm{A^k}^{1/k}<\rho(A)+\epsilon$,故
  \[
    \rho(A)=\lim_{k\to\infty}\norm{A^k}^{1/k}.  \qedhere
  \]
\end{proof}

回顾幂级数,对于一个复幂级数 $\sum_{k=0}^\infty a_kz^k$,我们知道其有一个收敛半径
$R$,当 $|z|<R$ 时该幂级数是绝对收敛的,$|z|>R$ 时幂级数发散。下面我们考虑
对应的矩阵级数 $\sum_{k=0}^\infty a_kA^k$,注意到
\begin{align*}
  \norm{\sum_{k=0}^n a_kA^k}&\leq\sum_{k=0}^n|a_k|\norm{A^k} \leq
  \sum_{k=0}^n|a_k|\norm{A}^k,
\end{align*}
所以当 $\norm{A}<R$ 时,$\sum_{k=0}^\infty a_kA^k$ 是绝对收敛的,从而收敛。
注意正这个矩阵范数是可以任意选取的,\eqref{eq:rho equal to inf of matrix norm}
式告诉我们存在使得 $\norm{A}<R$ 的矩阵范数当且仅当 $\rho(A)<R$。于是我们得到了下面的定理。

\begin{theorem}
  设 $R$ 是幂级数 $\sum_{k=0}^\infty a_k z^k$ 的收敛半径,矩阵 $A\in M_n(\mathbb{C})$,
  那么 $\rho(A)<R$ 时矩阵级数 $\sum_{k=0}^\infty a_kA^k$ 收敛。等价地说,如果存在
  矩阵范数 $\norm{\cdot}$ 使得 $\norm{A}<R$,那么 $\sum_{k=0}^\infty a_kA^k$ 收敛。
\end{theorem}

由于 $e^z=\sum_{k=0}^\infty z^k/k!$ 的收敛半径是无穷大,所以对于
任意的复矩阵 $A$,$e^A=\sum_{k=0}^\infty A^k/k!$ 总是有意义的。
类似地,$\sin A,\cos A$ 也都是良好定义的。$\log(1+z)=\sum_{k=1}^\infty(-1)^{k-1} z^k/k$
在 $|z|<1$ 时收敛,故 $\rho(A)<1$ 时,$\log(I+A)$ 是良好定义的。

\begin{corollary}
  令 $A\in M_n(\mathbb{F})$,如果存在矩阵范数 $\norm{\cdot}$ 使得 $\norm{I-A}<1$,
  那么 $A$ 可逆,此时
  \[
    A^{-1}=\sum_{k=0}^\infty (I-A)^k.  
  \]
\end{corollary}
\begin{proof}
  如果 $\norm{I-A}<1$,那么 $\sum_{k=0}^\infty (I-A)^k$ 收敛,并且
  \[
    A\sum_{k=0}^n(I-A)^k=(I-(I-A))\sum_{k=0}^n(I-A)^k=
    I-(I-A)^{n+1},  
  \]
  令 $n\to\infty$,便有
  \[
    A  \sum_{k=0}^\infty (I-A)^k=I.\qedhere 
  \]
\end{proof}

\section{条件数和线性方程组}

给定一个可逆矩阵 $A\in M_n(\mathbb{F})$,计算机计算 $A^{-1}$ 时存在舍入误差和截断误差,
此外 $A$ 的元素也可能是实验得到的,存在不确定性。所以我们要研究这些误差如何影响
逆矩阵的计算。

给定一个矩阵范数 $\norm{\cdot}$。我们希望计算 $A$ 的逆矩阵,但是实际情况中
我们需要处理的可能是矩阵 $B=A+\Delta A$。当然,我们需要保证 $B$ 仍然是可逆的,
注意到 $B=A(I+A^{-1}\Delta A)$,所以我们要求
\begin{equation}\label{eq:condition}
  \norm{A^{-1}\Delta A}<1,  
\end{equation}
此时 $\rho(A^{-1}\Delta A)\leq\norm{A^{-1}\Delta A}<1$,这表明 $-1$ 不是 $A^{-1}\Delta A$
的特征值,所以 $0$ 不是 $I+A^{-1}\Delta A$ 的特征值,从而 $I+A^{-1}\Delta A$
可逆,从而保证 $B$ 是可逆的。从直观上来说,这要求扰动 $\Delta A$ 不能太大
从而造成 $B$ 不可逆。从拓扑视角来看,由于可逆矩阵的集合是矩阵空间的开集,
所以只要 $\Delta A$ 足够小,总能使得 $B$ 可逆。

我们有 $A^{-1}(\Delta A) B^{-1}=A^{-1}(B-A)B^{-1}=A^{-1}-B^{-1}$,所以
\begin{equation}\label{eq:tmpa}
  \norm{A^{-1}-B^{-1}}=\norm{A^{-1}(\Delta A)B^{-1}}  \leq
  \norm{A^{-1}\Delta A}\norm{B^{-1}}.
\end{equation}
因为 $B^{-1}=A^{-1}-A^{-1}(\Delta A)B^{-1}$,所以
\[
  \norm{B^{-1}}\leq \norm{A^{-1}}+\norm{A^{-1}(\Delta A)B^{-1}}\leq
  \norm{A^{-1}}+\norm{A^{-1}\Delta A}\norm{B^{-1}},  
\]
这表明
\begin{equation}\label{eq:estimate error}
  \norm{B^{-1}}=\norm{(A+\Delta A)^{-1}}\leq\frac{\norm{A^{-1}}}{1-\norm{A^{-1}\Delta A}}  ,
\end{equation}
代入 \eqref{eq:tmpa} 式,得到
\[
  \norm{A^{-1}-B^{-1}}\leq \frac{\norm{A^{-1}}\norm{A^{-1}\Delta A}}{1-\norm{A^{-1}\Delta A}}
  \leq \frac{\norm{A^{-1}}\norm{A^{-1}}\norm{\Delta A}}{1-\norm{A^{-1}\Delta A}}  ,
\]
这表明所求逆矩阵 $B^{-1}$ 和真实逆矩阵 $A^{-1}$ 之间的相对误差满足
\begin{equation}\label{eq:error of inverse}
  \frac{\norm{A^{-1}-B^{-1}}}{\norm{A^{-1}}}\leq 
  \frac{\norm{A^{-1}}\norm{A}}{1-\norm{A^{-1}\Delta A}}\frac{\norm{\Delta A}}{\norm{A}}.
\end{equation}

对于矩阵 $A$,定义\emph{相对于矩阵范数 $\norm{\cdot}$ 的条件数}为
\[
  \kappa(A)=\begin{cases}
    \norm{A^{-1}}\norm{A} & \text{if $A$ is nonsingular},\\
    \infty & \text{if $A$ is singular}.
  \end{cases}
\]
注意到 $\kappa(A)=\norm{A^{-1}}\norm{A}\geq\norm{A^{-1}A}=\norm{I}\geq 1$。

如果我们把 \eqref{eq:condition} 的要求加强一些,即要求
\begin{equation}
  \norm{A^{-1}}\norm{\Delta A}<1,
\end{equation}
此时相对误差 \eqref{eq:error of inverse} 可以进一步变为
\[
  \frac{\norm{A^{-1}-B^{-1}}}{\norm{A^{-1}}}\leq 
  \frac{\norm{A^{-1}}\norm{A}}{1-\norm{A^{-1}}\norm{\Delta A}}\frac{\norm{\Delta A}}{\norm{A}}
  =\frac{\kappa(A)}{1-\kappa(A)\frac{\norm{\Delta A}}{\norm A}}\frac{\norm{\Delta A}}{\norm{A}},
\]
这能进一步看出条件数的意义。可以发现,当条件数不大的时候,逆矩阵 $B^{-1}$ 的相对误差
和 $\norm{\Delta A}/\norm{A}$ 的阶是差不多的,也就是说此时只要扰动 $\Delta A$ 
不大,逆矩阵 $B^{-1}$ 也不会距离真实的 $A^{-1}$ “太远”。如果条件数很大,
那么这个误差的上界就会很大,此时即使扰动 $\Delta A$ 很小,也可能造成
$B^{-1}$ 与 $A^{-1}$ 有很大的误差。所以当条件数 $\kappa(A)$ 越大的时候,
我们说矩阵 $A$ 越\emph{病态}。

\begin{proposition}\label{prop:condition number for matrix 2-norm}
  在采用谱范数(\autoref{exa:matrix 2-norm})的情况下,设可逆矩阵 $A\in M_n(\mathbb{F})$ 的奇异值为 $s_1\geq\cdots\geq s_n$,
  那么
  \[
    \kappa(A)=\frac{s_1}{s_n},  
  \]
  即 $A$ 的相对于谱范数的条件数为最大奇异值与最小奇异值之比。特别地,根据 
  \autoref{prop:matrix 2-norm of normal matrix},如果 $A$ 是正规矩阵,那么
  $\kappa(A)=\rho(A)\rho(A^{-1})$。
\end{proposition}
\begin{proof}
  $A^{-1}$ 的奇异值为 $1/s_n\geq \cdots\geq 1/s_1$,所以
  \[
    \kappa(A)=\norm{A^{-1}}_2\norm{A}_2=\frac{s_1}{s_n}.\qedhere  
  \]
\end{proof}

\begin{proposition}
  对于任意矩阵 $A,B$,任取矩阵范数 $\norm{\cdot}$,都有
  \[
    \kappa(AB)\leq\kappa(A)\kappa(B).  
  \]
\end{proposition}
\begin{proof}
  若 $A,B$ 至少有一个不可逆,结论显然成立。假设 $A,B$ 都可逆,那么
  \[
    \kappa(AB)=\norm{B^{-1}A^{-1}}\norm{AB}\leq\norm{B^{-1}}\norm{A^{-1}}\norm{A}\norm{B}
    =\kappa(A)\kappa(B).\qedhere  
  \]
\end{proof}

利用 $\rho(A)\leq\norm{A}$,所以矩阵的条件数有下届
\[
  \kappa(A)\geq\rho(A)\rho(A^{-1}),  
\]
这表明如果 $A$ 特征值模长的最大值和最小值之比如果很大,那么其条件数也会很大。

\begin{example}[Hilbert 矩阵]
  Hilbert 矩阵被定义为 $H_n=[(i+j-1)^{-1}]\in M_n(\mathbb{F})$,即 $H_n$
  的 $(i,j)$-元为 $(i+j-1)^{-1}$,例如
  \[
    H_5=\begin{pmatrix}
      1 & \frac{1}{2} & \frac{1}{3} & \frac{1}{4} & \frac{1}{5} \\[1.5mm]
      \frac{1}{2} & \frac{1}{3} & \frac{1}{4} & \frac{1}{5} & \frac{1}{6}\\[1.5mm]
      \frac{1}{3} & \frac{1}{4} & \frac{1}{5} & \frac{1}{6} & \frac{1}{7}\\[1.5mm]
      \frac{1}{4} & \frac{1}{5} & \frac{1}{6} & \frac{1}{7} & \frac{1}{8}\\[1.5mm]
      \frac{1}{5} & \frac{1}{6} & \frac{1}{7} & \frac{1}{8} & \frac{1}{9}\\[1.5mm]
    \end{pmatrix}  .
  \]
  显然 $H_n$ 是对称阵。其行列式为
  \[
    \det H_n=
    \frac{(1!2!\cdots(n-1)!)^4}{1!2!\cdots(2n-1)!},
  \]
  故 Hilbert 矩阵总是可逆。Hilbert 矩阵是一个经典的病态矩阵的例子。
  下面是 $n=2,\dots,8$ 的时候 $H_n$ 相对于谱范数的条件数列表:
  \begin{table}[h]
    \centering
    \begin{tblr}{
      colspec={*{4}{X[c,m]}*{4}{X[1.2,c,m]}},
      width=\linewidth,
      cells={mode=math},
      hlines,
      column{1}={r}
    }
      n= & 2 & 3 & 4 & 5 & 6 & 7 & 8 \\
      \kappa(H_n)\approx & 19 & 524 & 15513 & 4.8\times 10^{5} & 1.5\times 10^7 & 
      4.8\times 10^8 & 1.5\times 10^{10}
    \end{tblr}
  \end{table}

  可以发现 $\kappa(H_n)$ 随着 $n$ 的增大增长非常快,实际上,我们有
  \[
    \kappa(H_n)\sim e^{cn}\quad n\to\infty,
  \]
  其中常数 $c$ 大约为 $3.5$。
  根据 \autoref{prop:condition number for matrix 2-norm},我们有
  $\kappa(H_n)=\rho(H_n)\rho(H_n^{-1})$。意外的是,当 $n\to\infty$ 时,其谱半径有渐近
  $\rho(H_n)=\pi+\mathcal{O}(1/\log n)$,所以其谱半径趋于一个不大的常数,但是其条件数
  却增长非常迅速,这意味着 $\rho(H_n^{-1})=\kappa(H_n)/\rho(H_n)$ 增长非常迅速,
  而 $H_n^{-1}$ 的特征值是 $H_n$ 特征值的倒数,所以这表明 $H_n$ 的特征值的最小值
  会随着 $n$ 增大迅速减小。以 $H_5$ 为例,我们有
  \[
    H_5^{-1}=\begin{pmatrix}
      25 & -300 & 1050 & -1400 & 630 \\
      -300 & 4800 & -18900 & 26880 & -12600 \\
      1050 & -18900& 79380& -117600& 56700 \\
      -1400& 26880&-117600& 179200& -88200 \\
      630& -12600& 56700& -88200& 44100
    \end{pmatrix}  .
  \]
  此时我们添加微小的扰动 $\Delta H_5=0.01 I_5$,逆矩阵变为 
  \[
    (H_5+\Delta H_5)^{-1}\approx\begin{pmatrix}
      4.39 & -9.84& 1.14& 2.59& 2.25 \\
      -9.84& 44.49& -29.22& -10.67& 0.50\\
      1.14& -29.22& 70.39& -24.55& -19.37\\
      2.59& -10.67& -24.55& 69.52& -32.38\\
      2.25& 0.50& -19.37& -32.38& 60.1125
    \end{pmatrix},
  \]
  其与真实的 $H_5^{-1}$ 已经差别非常大了,这意味着计算 Hilbert 矩阵的逆矩阵的时候,
  即使微小的误差也会带来结果上巨大的差异,这也是这类矩阵被称为\emph{病态矩阵}
  的原因。
\end{example}

下面我们考虑条件数对线性方程组的影响。对于线性方程组
\[
  Ax=b,  
\]
其中 $A$ 可逆,$b\in\mathbb{F}^n$ 非零。添加扰动 $\Delta A$ 和 $\Delta b$,
此时实际求解方程组
\[
  (A+\Delta A)\tilde{x}=b+\Delta b,  
\]
设 $\tilde{x}=x+\Delta x$,我们分析 $\Delta x$ 的大小。
取一个与向量范数相容的矩阵范数 $\norm{\cdot}$,即其满足 $\norm{Ax}\leq\norm{A}\norm{x}$。
那么
\begin{align*}
  (A+\Delta A)\tilde{x}&=(A+\Delta A)(x+\Delta x)=Ax+(\Delta A)x+(A+\Delta A)\Delta x\\
&=  b+(\Delta A)x+(A+\Delta A)\Delta x=b+\Delta b,
\end{align*}
所以 
\[
  (\Delta A)x+(A+\Delta A)\Delta x=\Delta b.  
\]
这表明 $\Delta x=(A+\Delta A)^{-1}(\Delta b-(\Delta A)x)$,所以
\[
  \norm{\Delta x}\leq\norm{(A+\Delta A)^{-1}}\left(\norm{\Delta b}+\norm{(\Delta A )x}\right),
\]
根据 \eqref{eq:estimate error} 式以及范数的相容性,得到
\[
  \norm{\Delta x}\leq\frac{\norm{A^{-1}}}{1-\norm{A^{-1}\Delta A}}  \left(\norm{\Delta b}+\norm{\Delta A}\norm{x}\right),
\]
所以
\[
  \frac{\norm{\Delta x}}{\norm{x}}\leq\frac{\norm{A^{-1}}\norm{A}}{1-\norm{A^{-1}\Delta A}}
  \left(\frac{\norm{\Delta b }}{\norm{A}\norm{x}}+\frac{\norm{\Delta A}}{\norm{A}}\right)  ,
\]
再利用 $\norm{b}=\norm{Ax}\leq\norm{A}\norm{x}$,所以
\begin{equation}
  \frac{\norm{\Delta x}}{\norm{x}}\leq\frac{\kappa(A)}{1-\norm{A^{-1}\Delta A}}\left(
    \frac{\norm{\Delta b}}{\norm{b}}+\frac{\norm{\Delta A}}{\norm{A}}
  \right).
\end{equation}
同样,如果我们采用更严格的要求 $\norm{A^{-1}}\norm{\Delta A}<1$,那么
\begin{equation}
  \frac{\norm{\Delta x}}{\norm{x}}\leq\frac{\kappa(A)}{1-\kappa(A)\frac{\norm{\Delta A}}{\norm A}}\left(
    \frac{\norm{\Delta b}}{\norm{b}}+\frac{\norm{\Delta A}}{\norm{A}}
  \right).
\end{equation}
这表明对于一个系数矩阵条件数不大的线性方程组而言,其解的相对误差和系数矩阵
与数据 $b$ 的相对误差大致处于同一个水平上。

更进一步的,如果我们已经计算出了 $Ax=b$ 的一个近似解 $\hat x$,我们可以估计
这个近似解和精确解 $x$ 的相对误差。定义残差向量为 $r=b-A\hat x$。此时
$A^{-1}r=A^{-1}b-\hat x=x-\hat x$,所以 $\norm{x-\hat x}\leq\norm{A^{-1}}\norm r$,
进一步考虑 $\norm b=\norm{Ax}\leq\norm{A}\norm{x}$,那么
\begin{align*}
  \norm{x-\hat x}&\leq\norm{A^{-1}}\norm{r}=\norm{A^{-1}}\norm{A}\frac{\norm{x}\norm{r}}{\norm{A}\norm{x}}\\
  &\leq \norm{A^{-1}}\norm{A}\frac{\norm{x}\norm{r}}{\norm{b}},
\end{align*}
所以相对误差满足
\begin{equation}
  \frac{\norm{x-\hat x}}{\norm x}\leq\kappa(A)\frac{\norm r}{\norm b}.
\end{equation}
这表明如果系数矩阵条件数不大,那么计算误差大致和残差是一个水平,否则
即使计算解的残差很小,也可能与精确解相差很远。

\begin{example}
  考虑 Hilbert 矩阵为系数矩阵的线性方程组
  \[
    H_5x=\begin{pmatrix}
      1 &
      \frac{1}{2}&
      \frac{1}{3}&
      \frac{1}{4} &
      \frac{1}{5}
    \end{pmatrix}^T=b,
  \]
  其显然有精确解 $x=(1,0,0,0,0)^T$。添加扰动 $\Delta b=0.01(1,1,1,1,1)^T$,
  解变为了
  \[
    x+\Delta x=H_5^{-1}(b+\Delta b)=(1.05,-1.2,6.3,-11.2,6.3)^T,  
  \]
  此时相对误差为
  \[
    \frac{\norm{\Delta x}_2}{\norm{x}_2}\approx 14.36,\quad
    \frac{\norm{\Delta b}_2}{\norm{b}_2}=  0.018,
  \]
  二者相差了接近 1000 倍!所以病态矩阵作为系数矩阵的线性方程组,其解的数值稳定性
  是非常差的,即使一个微小的误差也可能造成解的巨大差异。
\end{example}



\begin{thebibliography}{99}
  \bibitem{Horn} Horn RA, Johnson CR. Matrix Analysis. Cambridge university press; 2012 Oct 22.
  \bibitem{LADR}  Axler S. Linear Algebra Done Right. Springer Nature; 2024.
\end{thebibliography}




\end{document}