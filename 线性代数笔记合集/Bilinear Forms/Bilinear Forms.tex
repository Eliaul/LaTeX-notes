\documentclass[fontset=none,zihao=-4]{Notes}

\makeatletter
\DeclareRobustCommand{\em}{%
  \@nomath\em \if b\expandafter\@car\f@series\@nil
  \normalfont \else \bfseries \fi}
\makeatother

\usepackage{tikz-cd,wrapstuff}
\usepackage{fixdif,siunitx,tikz,nicematrix}

\usetikzlibrary{hobby,calc,arrows}
\usetikzlibrary{positioning}
\usetikzlibrary{decorations.markings}
\usetikzlibrary{decorations.pathreplacing}

\ProvidesFile{font.def}

\setCJKmainfont{Source Han Serif SC}[
  UprightFont=*-Regular,
  BoldFont=*-Bold,
  ItalicFont=HYKaiTi S,
  ItalicFeatures={Scale=1.1}
]
\newCJKfontfamily[zhsong]\songti{Source Han Serif SC}[
  UprightFont=*-Regular,
  BoldFont=*-Bold,
  ItalicFont=HYKaiTi S,
  ItalicFeatures={Scale=1.1}
]
\setCJKsansfont{Source Han Sans SC}[
  UprightFont=*-Regular,
  BoldFont=*-Bold
]
\newCJKfontfamily[zhhei]\heiti{Source Han Sans SC}[
  UprightFont=*-Regular,
  BoldFont=*-Bold
]
\setCJKmonofont{HYFangSong S}[
  BoldFont=*,
  ItalicFont=*,
  BoldItalicFont=*
]
\newCJKfontfamily[zhfs]\fangsong{HYFangSong S}[
  BoldFont=*,
  ItalicFont=*,
  BoldItalicFont=*
]
\newCJKfontfamily[zhkai]\kaishu{HYKaiTi S}[
  BoldFont=*,
  ItalicFont=*,
  BoldItalicFont=*
]

\setmainfont{texgyretermes}[
  Extension=.otf,
  UprightFont=*-regular,
  BoldFont=*-bold,
  ItalicFont=*-italic,
  BoldItalicFont=*-bolditalic,
  SlantedFont=*-italic
]
%\setmathrm{texgyretermes}[
%  Extension=.otf,
%  UprightFont=*-regular,
%  BoldFont=*-bold,
%  ItalicFont=*-italic,
%  BoldItalicFont=*-bolditalic,
%  SlantedFont=*-italic
%]
\setsansfont{Cantarell}[
  UprightFont=* Regular,
  ItalicFont=* Italic,
  BoldFont=* Bold,
  BoldItalicFont=* Bold Italic,
  SmallCapsFont=Alegreya Sans SC
]
\setmonofont{Ubuntu Mono}[
  UprightFont=*,
  ItalicFont=* Italic,
  BoldFont=* Bold,
  BoldItalicFont=* Bold Italic
]
%\setmathfont{texgyretermes-math.otf}
%\setmathfont[range={\mathcal,\mathbfcal,\mathfrak},StylisticSet=1]{XITSMath-Regular.otf}
%\setmathfont[range={\mathbb}]{KpMath-Sans.otf}



\DeclareMathOperator\Int{Int}
\DeclareMathOperator\supp{supp}
\DeclareMathOperator\im{im}
\DeclareMathOperator\End{End}
\DeclareMathOperator\Ann{Ann}
\DeclareMathOperator\Hom{Hom}
\DeclareMathOperator\diag{diag}
\DeclareMathOperator\Or{O}
\DeclareMathOperator\Sp{Sp}
\DeclareMathOperator\cha{char}
\DeclareMathOperator\rad{rad}
\DeclareMathOperator\rank{rank}

\newcommand{\inn}[1]{\left\langle#1\right\rangle}
\newcommand{\norm}[1]{\left\lVert#1\right\rVert}
\newcommand{\spa}[1]{\operatorname{span}\left(#1\right)}
\DeclareMathSymbol{\bot}{\mathrel}{symbols}{"3F}

\tikzcdset{
  arrow style=tikz,
  diagrams={>={Straight Barb[scale=0.8]}}
}

\tikzset{
  point/.style={
    circle,fill,inner sep=0pt,minimum width=5pt
  }
}

\usepackage[subscriptcorrection,nofontinfo,mtpbb]{mtpro2}



\setlist[enumerate]{nosep,label=(\arabic*)}
\setlist[itemize]{nosep}

\title{\sffamily 双线性型}
\author{Eliauk}


\begin{document}

\maketitle

\tableofcontents

\section{定义与基本理论}

设 $V$ 是域 $\mathbb{F}$ 上的向量空间,映射 $\inn{,}:V\times V\to \mathbb{F}$
如果在每个分量上的都是线性的,即满足
\[
  \inn{ax+by,z}=a\inn{x,z}+b\inn{y,z}  
\]
以及
\[
  \inn{z,ax+by}=a\inn{z,x}+b\inn{z,y},  
\]
那么我们说 $\inn{,}$ 是一个\emph{双线性型}。如果双线性型满足
\[
  \inn{x,y}=\inn{y,x} \quad\forall x,y\in V,  
\]
那么我们说这个双线性型是\emph{对称型}。如果双线性型满足
\[
  \inn{x,y}=-\inn{y,x}  \quad\forall x,y\in V,  
\]
那么我们说这个双线性型是\emph{反对称型}。如果双线性满足
\[
  \inn{x,x}=0\quad \forall x\in V,  
\]
那么我们说这个双线性型是\emph{交错型}。

本文所指的向量空间 $V$ 均表示其附带一个双线性型 $\inn{,}$。
如果 $V$ 配备对称型,那么我们说 $V$ 是 $\mathbb{F}$ 上的\emph{正交几何}。
如果 $V$ 配备交错型,那么我们说 $V$ 是 $\mathbb{F}$ 上的\emph{辛几何}。
显然,Euclid 内积空间是 $\mathbb{R}$ 上的正交几何。
此外,本文出现的“$V$ 的子空间”均配备和 $V$ 一样的双线性型。

\begin{example}
  \emph{Minkowski 空间} $M_4$ 指的就是一个四维的正交几何 $\mathbb{R}^4$,
  其配备的对称型定义为
  \begin{align*}
    \inn{e_1,e_1}&=\inn{e_2,e_2}=\inn{e_3,e_3}=1 ,\\
    \inn{e_4,e_4}&=-1,\\
    \inn{e_i,e_j}&=0  \ (i\neq j).
  \end{align*}
  其中 $e_1,\dots,e_4$ 是 $\mathbb{R}^4$ 的标准基。
\end{example}

上述定义的对称型、反对称型和交错型的概念并不是相互独立的,它们取决于基域
$\mathbb{F}$ 的性质。

\begin{theorem}
  $V$ 是域 $\mathbb{F}$ 上的向量空间。
  \begin{enumerate}
    \item 若 $\cha(\mathbb{F})=2$,那么
    交错型都是对称型,并且对称型和反对称型等价。
    \item 若 $\cha(\mathbb{F})\neq 2$,那么
    交错型和反对称型等价。此时唯一的既是交错型又是对称型的双线性型是
    零型:对于任意的 $x,y\in V$ 都有 $\inn{x,y}=0$。
  \end{enumerate}
\end{theorem}
\begin{proof}
  首先注意到交错型满足
  \[
    0=\inn{x+y,x+y}=\inn{x,y}+\inn{y,x},  
  \]
  即 $\inn{x,y}=-\inn{y,x}$,所以交错型总是为反对称型。

  若 $\cha(\mathbb{F})=2$,那么对于交错型而言,有 $\inn{x,y}=-\inn{y,x}=\inn{y,x}$,
  所以交错型总是为对称型。又因为此时 $-1=1$,所以对称型和反对称型等价。
  若 $\cha(\mathbb{F})\neq 2$,对于反对称型而言,有 $\inn{x,x}=-\inn{x,x}$,
  所以 $\inn{x,x}=0$,故反对称型总是为交错型。
\end{proof}

设 $e_1,\dots,e_n$ 是向量空间 $V$ 的一组基,任取 $x=\sum x_ie_i$ 以及
$y\in\sum y_je_j$,那么
\[
  \inn{x,y}=\sum_{i=1}^n\sum_{j=1}^n x_iy_j\inn{e_i,e_j}
  =
  \begin{pmatrix}
    x_1 & x_2 & \cdots & x_n
  \end{pmatrix}A
  \begin{pmatrix}
    y_1 \\ y_2 \\ \vdots \\ y_n
  \end{pmatrix},
\]
其中矩阵
\[
  A=\bigl(\inn{e_i,e_j}\bigr)=\begin{pmatrix}
    \inn{e_1,e_1} & \inn{e_1,e_2} & \cdots & \inn{e_1,e_n} \\
    \inn{e_2,e_1} & \inn{e_2,e_2} & \cdots & \inn{e_2,e_n} \\
    \vdots & \vdots & \ddots & \vdots \\
    \inn{e_n,e_1} & \inn{e_n,e_2} & \cdots & \inn{e_n,e_n}
  \end{pmatrix}  .
\]
这表明这个双线性型由矩阵 $A$ 唯一确定,所以我们将 $A$ 称为
双线性型相对于这组基下的表示矩阵。我们说矩阵 $A$ 是\emph{交错矩阵}
当且仅当其是反对称矩阵并且对角线全为零(在特征不为 $2$ 的域上显然等价于反对称矩阵)。由此我们可以看出,一个双线性型
是对称型当且仅当其表示矩阵是对称矩阵,是反对称型当且仅当其表示矩阵
是反对称矩阵,是交错型当且仅当其表示矩阵是交错矩阵。

下面我们观察改变基对表示矩阵的影响。设 $f_1,\dots,f_n$ 是另一组基,
记 $e_1,\dots,e_n$ 到 $f_1,\dots,f_n$ 的过渡矩阵为 $P$,即
\[
  (f_1,\dots,f_n)=(e_1,\dots,e_n)P.  
\]
设双线性型在 $f_1,\dots,f_n$ 下的表示矩阵是 $B$。任取 $x=\sum x_if_i$,
$y=\sum y_jf_j$,那么
\[
  \inn{x,y}= \left[P\begin{pmatrix}
    x_1 \\x_2 \\ \vdots \\ x_n
  \end{pmatrix}\right]^T A\left[
    P  \begin{pmatrix}
      y_1 \\ y_2 \\ \vdots \\ y_n
    \end{pmatrix}
  \right]=  \begin{pmatrix}
    x_1 & x_2 & \cdots & x_n
  \end{pmatrix}P^TAP
  \begin{pmatrix}
    y_1 \\ y_2 \\ \vdots \\ y_n
  \end{pmatrix},
\]
所以 $B=P^TAP$。因此,给定矩阵 $A,B\in M_n(\mathbb{F})$,如果存在可逆矩阵
$P$ 使得
\[
  A=P^TBP,  
\]
那么我们说 $A$ 和 $B$ 是\emph{合同的}。容易验证合同是一个等价关系。

设向量空间 $V$ 有一组基 $e_1,\dots,e_n$。给定一个矩阵 $A$,
任取 $x=\sum x_ie_i$ 以及 $y=\sum y_je_j$,那么我们可以定义
\[
  \inn{x,y}=    \begin{pmatrix}
    x_1 & x_2 & \cdots & x_n
  \end{pmatrix}A
  \begin{pmatrix}
    y_1 \\ y_2 \\ \vdots \\ y_n
  \end{pmatrix},
\]
不难验证这给出了 $V$ 上的一个双线性型。所以在给定一组基的前提下,$V$
上的双线性型和 $\mathbb{F}$ 上的 $n$ 阶矩阵是一一对应的。
若 $B$ 和 $A$ 合同,即 $B=P^TAP$,此时考虑 $V$ 的另一组基
$f_1,\dots,f_n$,满足
\[
  (f_1,\dots,f_n)=(e_1,\dots,e_n)P,
\]
那么对于上述 $x,y$,有
\[
  \inn{x,y}=  \left[P^{-1}\begin{pmatrix}
    x_1 \\x_2 \\ \vdots \\ x_n
  \end{pmatrix}\right]^T P^TAP\left[
    P^{-1} \begin{pmatrix}
      y_1 \\ y_2 \\ \vdots \\ y_n
    \end{pmatrix}\right]=\begin{pmatrix}
      x_1 & x_2 & \cdots & x_n
    \end{pmatrix}A
    \begin{pmatrix}
      y_1 \\ y_2 \\ \vdots \\ y_n
    \end{pmatrix},
\]
所以合同的矩阵表示相同的双线性型。综合起来,
两个矩阵合同当且仅当它们表示相同的双线性型。

\section{二次型}

向量空间 $V$ 上的函数 $Q:V\to\mathbb{F}$ 如果满足:
\begin{enumerate}
  \item 对于任意的 $a\in\mathbb{F}$ 和 $x\in V$,有
  \[
    Q(ax)=a^2Q(x),  
  \]
  \item 映射
  \[
    \inn{x,y}_Q=Q(x+y)-Q(x)-Q(y)
  \]
  是一个(对称)双线性型,
\end{enumerate}
那么我们说 $Q$ 是一个\emph{二次型}。根据定义,每个二次型 $Q$ 都定义了一个对应的
对称型 $\inn{x,y}_Q$。反之,如果 $\cha(\mathbb{F})\neq 2$,$\inn{,}$ 是一个对称型,
那么
\[
  Q(x)=\frac{1}{2}\inn{x,x}  
\]
是 $V$ 上的一个二次型。此时 $Q$ 定义的对称型就是 $\inn{,}$。
这表明 $V$ 上的对称双线性型和二次型之间存在一一对应。故知道了 $V$ 上
的二次型就相当于知道了对应的对称双线性型。

假设 $\cha(\mathbb{F})\neq 2$,设 $e_1,\dots,e_n$ 是正交几何 $V$ 上的一组基,
$V$ 上的对称型在这组基下的表示矩阵为 $A=(a_{ij})$,那么对于
$x=\sum x_ie_i$,有
\[
  Q(x)=\frac{1}{2}\inn{x,x}=\frac{1}{2}\sum_{i,j=1}^n a_{ij}x_ix_j.  
\]
所以二次型 $Q$ 是 $x_1,\dots,x_n$ 的二次齐次多项式。

\section{正交性}

下面我们拓宽内积空间的一些定义。令 $V$ 是向量空间,对于 $x,y\in V$,
如果 $\inn{x,y}=0$,那么我们说 $x,y$ \emph{正交},记作 $x\bot y$。
对于 $V$ 的子集 $S$,如果任取 $s\in S$ 有 $x\bot s$,那么我们说
$x$ 正交于子集 $S$,记作 $x\bot S$。对于两个子集 $S,T$,如果任取 $s\in S$
和 $t\in T$ 有 $s\bot t$,那么我们说 $S,T$ 正交,记作 $S\bot T$。
对于子集 $X\subseteq V$,定义\emph{正交补}为
\[
  X^\bot =\{v\in V\,|\, v\bot X\}.  
\]
时刻注意,本文中的 $\inn{,}$ 仅仅表示双线性型而未必是内积。
对于一般的双线性型而言,正交不一定具有对称型,即 $x\bot y$ 并不一定推出
$y\bot x$。对于对称型和交错型而言,$x\bot y$ 和 $y\bot x$ 才是等价的。
这也是为什么我们主要研究对称型和交错型的原因。

对于一个向量 $x\in V$ 而言,有一种退化的情况,即自己与自己正交,此时
我们给出下面的术语。令 $V$ 是一个向量空间,向量 $x\in V$ 如果满足
$\inn{x,x}=0$,那么我们说 $x$ 是\emph{迷向向量},否则称之为\emph{非迷向向量}。
如果 $V$ 至少包含一个非零迷向向量,那么我们说 $V$ 是\emph{迷向空间},否则称之为\emph{非迷向空间}。
如果 $V$ 中每个向量都是迷向向量,那么我们说 $V$ 是\emph{完全迷向空间}(即辛几何)。

注意到,如果 $x\in V$ 是迷向向量,那么对于任意 $a\in\mathbb{F}$,$ax$ 都是迷向向量。
这表明 $V$ 中的迷向向量集合对标量乘法封闭,这样的子集称为 $V$ 中的\emph{锥体}。

进一步的,我们考虑更严重的退化情况,即下面的术语。
令 $V$ 是一个向量空间,向量 $x$ 如果满足 $x\bot V$,那么我们说 $x$ 是\emph{退化向量}。
$V$ 中所有退化向量的集合 $V^\bot$ 我们称之为 $V$ 的\emph{根},记为 $\rad(V)$。
如果 $\rad(V)=\{0\}$,那么我们说 $V$ 是\emph{非退化的},否则我们说
$V$ 是\emph{退化的}。如果 $\rad(V)=V$,那么
我们说 $V$ 是\emph{完全退化的}。

\begin{theorem}
  向量空间 $V$ 是非退化的当且仅当双线性型 $\inn{,}$ 在任意基下的表示矩阵是可逆的。
\end{theorem}
\begin{proof}
  $(\Rightarrow)$ 设 $V$ 的一组基为 $e_1,\dots,e_n$,$\inn{,}$ 在这组基下
  的表示矩阵为 $A$,令 $Ay=0$,设 $y=(y_1,\dots,y_n)$,记 $y'=\sum_i y_ie_i\in V$。
  任取 $x'=\sum x_ie_i\in V$,记 $x=(x_1,\dots,x_n)$,那么 
  \[
    \inn{x',y'}=x^TAy=0,  
  \]
  所以 $y'\bot V$,$V$ 非退化表明 $y'=0$,所以 $y=0$,所以 $A$ 可逆。

  $(\Leftarrow)$ 继续采用上述记号。若 $A$ 可逆,令 $y'=\sum_iy_ie_i\in\rad(V)$,
  那么任取 $x'=\sum x_ie_i\in V$,有 $\inn{x',y'}=x^TAy=0$,特别地,令
  $x=Ay$,那么 $y^TA^TAy=0$,即 $\norm{Ay}^2=(Ay)^T(Ay)=0$,故 $Ay=0$,
  $A$ 可逆表明 $y=0$,故 $y'=0$,所以 $\rad(V)=\{0\}$,即 $V$ 是非退化的。
\end{proof}

\begin{example}
  非退化空间的子空间可能是退化的。考虑有限域 $\mathbb{F}_2$ 上的向量空间
  $\mathbb{F}_2^4$,配备标准内积,即
  \[
    (x_1,\dots,x_4)\cdot(y_1,\dots,y_4)=x_1y_1+\cdots+x_4y_4,  
  \]
  这个双线性型的表示矩阵为单位阵,所以 $\mathbb{F}_2^4$ 是非退化的。
  考虑子空间
  \[
    U=\{(0,0,0,0),(1,1,0,0),(0,0,1,1),(1,1,1,1)\},  
  \]
  不难验证这里面任意向量都是退化向量,所以 $U$ 甚至是完全退化空间。
\end{example}

若 $U\subseteq V$ 是子空间,那么 $\rad(U)$ 表示 $U$ 中与 $U$ 正交的向量集合,
故
\[
  \rad(U)=U\cap U^\bot.  
\]
此外,还可以发现
\[
  \rad(U)=U\cap U^\bot\subseteq (U^\bot)^\bot\cap U^\bot=\rad(U^\bot),  
\]
所以如果 $U$ 是退化的,那么 $U^\bot$ 也是退化的。

\begin{theorem}
  令 $V$ 是向量空间,那么下面的说法是等价的:
  \begin{enumerate}
    \item 两个向量正交满足对称性,即 
    \[
      x\bot y\Rightarrow y\bot x;  
    \]
    \item $V$ 上的双线性型是对称型或者交错型。
  \end{enumerate}
\end{theorem}
\begin{proof}
  $(2)\Rightarrow (1)$ 交错型都是反对称型,此时 $x\bot y$ 表明 $\inn{x,y}=0$,
  故 $\inn{y,x}=\pm\inn{x,y}=0$,故 $y\bot x$。

  $(1)\Rightarrow (2)$ 为了方便起见,定义
  $x\bowtie y$ 表示 $\inn{x,y}=\inn{y,x}$,$x\bowtie V$ 表示
  对于任意的 $y\in V$ 有 $\inn{x,y}=\inn{y,x}$。如果对于所有的 $x\in V$
  都有 $x\bowtie V$,即 $\inn{,}$ 是对称型。所以假设存在 $x\in V$ 使得
  $x\not\bowtie V$。

  我们首先说明 $x\bowtie y$ 表明 $x\bot y$。$x\not\bowtie V$ 表明存在 $z\in V$
  使得 $\inn{x,z}\neq\inn{z,x}$。注意到
  \begin{align*}
    \inn{x,y}\bigl(\inn{x,z}-\inn{z,x}\bigr)&=\inn{x,y}\inn{x,z}-\inn{x,y}\inn{z,x}\\
    &=\inn{y,x}\inn{x,z}-\inn{x,y}\inn{z,x}\\
    &=\inn{x,\inn{y,x}z-\inn{z,x}y}.
  \end{align*}
  又因为
  \[
    \inn{\inn{y,x}z-\inn{z,x}y,x}=\inn{y,x}\inn{z,x}-\inn{z,x}\inn{y,x}=0,
  \]
  根据正交的对称性,所以 
  \[
    \inn{x,y}\bigl(\inn{x,z}-\inn{z,x}\bigr)=  \inn{x,\inn{y,x}z-\inn{z,x}y}=0,
  \]
  故 $\inn{x,y}=0$。由于 $x\bowtie x$,所以 $x\bot x$,这表明 $x$ 是迷向向量。

  假设 $V$ 不是正交几何,我们证明 $V$ 中向量都是迷向向量,从而说明 $V$ 是辛几何。
  $V$ 不是正交几何表明存在 $u,v\in V$ 使得 $u\not\bowtie v$,所以 $u\not\bowtie V$
  以及 $v\not\bowtie V$,根据上面的叙述,$u,v$ 都是迷向向量。
  任取 $w\in V$,若 $w\not\bowtie V$,则 $w$ 是迷向向量。若 $w\bowtie V$,
  那么 $w\bowtie v$ 以及 $w\bowtie u$,根据上面的叙述,有 $w\bot v$ 以及 $w\bot u$。
  注意到 $w-u\bot u$ 以及 $w=(w-u)+u$,所以
  \[
    \inn{w,w}=\inn{w-u,w-u}+\inn{w-u,u}+\inn{u,w-u}+\inn{u,u}=
    \inn{w-u,w-u},  
  \]
  所以我们说明 $w-u$ 是迷向向量。注意到
  \[
    \inn{w-u,v}=-\inn{u,v}\neq-\inn{v,u}=\inn{v,w-u},  
  \]
  所以 $w-u\not\bowtie V$,所以 $w-u$ 是迷向向量,这就表明 $w$ 是迷向向量。

  故正交性是对称的表明 $V$ 必然为正交几何或者辛几何。
\end{proof}

如果一个向量空间既是正交的又是辛的,那么这个双线性型既是对称型又是反对称型,因此
\[
  \inn{x,y}=\inn{y,x}=-\inn{x,y}.
\]
故 $\cha(\mathbb{F})\neq 2$ 的时候,$V$ 既是正交几何又是辛几何当且仅当
$\rad(V)=V$,即 $V$ 是完全退化空间。
但是,当 $\cha(\mathbb{F})=2$ 的时候,存在非退化的正交辛几何。例如,
令 $V=\mathbb{F}_2^2$ 是 $\mathbb{F}_2$ 上的二维向量空间。考虑基 $e_1=(1,0),e_2=(0,1)$,
在这组基下的矩阵
\[
  A=\begin{pmatrix}
    0 & 1 \\
    1 & 0
  \end{pmatrix}  
\]
导出的双线性型既是对称型又是交错型,但是 $A$ 可逆,所以此时 $V$ 非退化。

\section{线性泛函}

对于有限维实或者复的内积空间 $V$,我们知道有 Riesz 表示定理,即若 $f\in V^*$
是一个线性泛函(即 $V\to\mathbb{F}$ 的线性映射),那么存在唯一的 $u\in V$,使得对于任意的
$v\in V$,有
\[
  f(v)=\inn{v,u}.  
\]
下面我们说明,对于非退化的向量空间,有类似的结果。

令 $V$ 是 $\mathbb{F}$ 上的有限维向量空间,给定 $x\in V$,定义线性映射 $\inn{\cdot,x}:V\to\mathbb{F}$
为
\[
  \inn{\cdot,x}v=\inn{v,x}.  
\]
然后定义线性映射 $\tau:V\to V^*$ 为 
\[
  \tau(x)=\inn{\cdot, x}.  
\]
$\inn{,}$ 是双线性型就表明 $\tau$ 是线性映射,并且
\[
  \ker\tau=\bigl\{\,x\in V \bigm| \inn{\cdot,x}=0\,\bigr\}  
  =V^\bot=\rad(V).
\]
所以 $\tau$ 是单射(从而是同构)当且仅当 $V$ 是非退化的。

\begin{theorem}[Riesz 表示定理]
  令 $V$ 是有限维非退化向量空间,那么映射
  \[
    \tau:V\to V^*\quad \tau(x)=\inn{\cdot,x}  
  \]
  是同构。所以对于每个 $f\in V^*$,都存在唯一的 $x\in V$ 使得
  \[
    f(v)=\tau(x)(v)=\inn{v, x}  \quad \forall v\in V.
  \]
\end{theorem}
\begin{proof}
  $V$ 非退化当且仅当 $\tau$ 是单射,所以 $\tau$ 是同构,故存在
  唯一的 $x\in V$ 使得 $f=\tau(x)$。
\end{proof}

\begin{theorem}[关于子空间的 Riesz 表示定理]\label{thm:Riesz for subspace}
  令 $U$ 是有限维向量空间 $V$ 的子空间,如果 $V$ 和 $U$ 至少有一个非退化,那么
  线性映射
  \[
      \tau:V\to U^*\quad \tau(x)=\inn{\cdot, x}|_U
  \]
  是满射并且 $\ker\tau=U^\bot$。因此,对于任意线性泛函 $f\in U^*$,存在(可能不唯一)
  向量 $x\in V$ 使得对于任意 $u\in U$ 有 $f(u)=\inn{u,x}$。此外,如果 $U$
  是非退化的,那么 $x$ 可以唯一地从 $U$ 中选取。
\end{theorem}
\begin{proof}
  若 $V$ 非退化。对于任意的 $f\in U^*$,我们可以将其延拓为 $\bar f\in V^*$,这只需要扩充一组基,
  然后令 $\bar f$ 在扩充基上取值为零即可。根据 Riesz 表示定理,存在 $x\in V$ 使得
  $\bar f=\inn{\cdot,x}$,此时 $f=\inn{\cdot,x}|_U$,所以 $\tau$ 是满射。
  令 $\inn{\cdot,x}|_U=0$,即对于任意的 $u\in U$ 有 $\inn{u,x}=0$,
  所以 $\ker\tau=U^\bot$。

  若 $U$ 非退化,将同构 $\tau:U\to U^*$ 延拓为 $\bar\tau:V\to U^*$ 即可。
\end{proof}


\section{正交补与正交直和}

对于向量空间 $V$ 的子空间 $U,W$,如果 $U\bot W$ 并且 $U\oplus W=V$,那么
我们说 $V$ 是 $U$ 和 $W$ 的\emph{正交直和}。如果 $V$ 是内积空间,那么我们知道
$V=U\oplus U^\bot$。然而,在一般的向量空间中,正交补也不一定是向量空间补,例如,
如果 $v$ 是退化向量,那么 $\spa{v}^\bot=V$,所以 $\spa{v}+\spa{v}^\bot$ 并不是直和。

\begin{theorem}\label{thm:rad decomposition of V}
  $V$ 是向量空间,那么
  \[
    V=\rad(V)\oplus U,  
  \]
  其中 $U$ 非退化,$\rad(V)$ 完全退化。
\end{theorem}
\begin{proof}
  令 $U$ 为 $\rad(V)$ 的向量空间补,显然 $\rad(V)\bot U$。注意到此时有
  $\rad(U)\subseteq\rad (V)$,所以 $\rad(U)=0$,故 $U$ 非退化。
\end{proof}


\begin{theorem}\label{thm:property of direct complements}
  令 $U$ 是有限维向量空间 $V$ 的子空间。
  \begin{enumerate}
    \item 若 $U,V$ 至少有一个非退化,那么
    \[
      \dim U+\dim U^\bot=\dim V,  
    \]
    这表明 $V=U+U^\bot$,所以此时 $V=U\oplus U^\bot$ 当且仅当 $U\cap U^\bot=0$。
    \item 若 $V$ 非退化,那么
    \begin{enumerate}
      \item $(U^\bot)^\bot=U$;
      \item $\rad(U)=\rad(U^\bot)$;
      \item $U$ 非退化当且仅当 $U^\bot$ 非退化。
    \end{enumerate}
  \end{enumerate}
\end{theorem}
\begin{proof}
  (1) 根据 \autoref{thm:Riesz for subspace},映射 $\tau:V\to U^*$ 是满射,且
  $\ker\tau=U^\bot$,故 $V/ U^\bot\simeq U^*\simeq U$,所以 
  \[
    \dim V-\dim U^\bot=\dim V/U^\bot=\dim U.
  \]

  (2) 显然 $U\subseteq (U^\bot)^\bot$。由 (1),有
  \[
    \dim U+\dim U^\bot=\dim V, \quad \dim U^\bot+\dim(U^\bot)^\bot=\dim V,
  \]
  所以 $\dim U=\dim (U^\bot)^\bot$,故 $U=(U^\bot)^\bot$。因此
  \[
    \rad(U)=U\cap U^\bot=(U^\bot)^\bot\cap U^\bot=\rad(U^\bot).\qedhere
  \]
\end{proof}

\section{等距}

令 $V,W$ 是向量空间,我们使用同一个记号 $\inn{,}$ 分别表示 $V$ 和 $W$ 上的双线性型。
一个线性映射 $\varphi:V\to W$ 如果是双射并且满足
\[
  \inn{\varphi(v),\varphi(u)}=\inn{v,u}\quad \forall v,u\in V,
\]
那么 $\varphi$ 被称为\emph{等距同构}。注意等号两边是不同的双线性型。
显然所有 $V\to V$ 的等距同构全体构成一个群。
若 $V$ 是非退化的正交几何,那么 $V$ 上的等距同构被称为\emph{正交变换},
所有正交变换构成的群被称为\emph{$V$ 上的正交群},记为 $\Or(V)$。
若 $V$ 是非退化的辛几何,那么 $V$ 上的等距同构被称为\emph{辛变换},
所有辛变换构成的群被称为\emph{$V$ 上的辛群},记为 $\Sp(V)$。

\begin{theorem}
  设 $\varphi:V\to W$ 是有限维向量空间之间的线性映射。
  \begin{enumerate}
    \item 设 $V$ 的一组基为 $e_1,\dots,e_n$,那么 $\varphi$ 是等距同构当且仅当
    \[
      \inn{\varphi(e_i),\varphi(e_j)}=\inn{e_i,e_j}.
    \]
    \item 若 $V$ 是正交几何并且 $\cha(\mathbb{F})\neq 2$,那么 $\varphi$ 是等距同构
    当且仅当 $\varphi$ 是双射并且
    \[
      \inn{\varphi(v),\varphi(v)}=\inn{v,v}\quad \forall v\in V.  
    \]
    \item 设 $\varphi$ 是等距同构并且
    \[
      V=U\oplus U^\bot,\quad W=T\oplus T^\bot,  
    \]
    如果 $\varphi(U)=T$,那么 $\varphi(U^\bot)=T^\bot$。
  \end{enumerate}
\end{theorem}
\begin{proof}
  (1) 是显然的。(2) 必要性显然,对于充分性,任取 $u,v\in V$,那么
  \begin{align*}
    \inn{u,u}+2\inn{u,v}+\inn{v,v}&=\inn{u+v,u+v}\\
    &=\inn{\varphi(u+v),\varphi(u+v)}  \\
    &=
    \inn{\varphi(u),\varphi(u)}+2\inn{\varphi(u),\varphi(v)}+\inn{\varphi(v),\varphi(v)}\\
    &=\inn{u,u}+2\inn{\varphi(u),\varphi(v)}+\inn{v,v},\\
  \end{align*}
  这就表明 $\inn{\varphi(u),\varphi(v)}=\inn{u,v}$,即 $\varphi$ 是等距同构。

  (3) 任取 $v\in U^\bot$,$t\in T$,那么存在 $u\in U$ 使得 $t=\varphi(u)$,所以
  \[
    \inn{\varphi(v),t}=\inn{\varphi(v),\varphi(u)}=\inn{v,u}=0,  
  \]
  所以 $\varphi(v)\in T^\bot$,故 $\varphi(U^\bot)\subseteq T^\bot$。
  又因为 $\dim V=\dim W$,$\dim U=\dim T$,所以 $\dim\varphi(U^\bot)=\dim U^\bot=\dim T^\bot$,
  所以 $\varphi(U^\bot)=T^\bot$。
\end{proof}

\section{双曲空间}

令 $V$ 是向量空间,两个向量 $u,v\in V$ 如果满足
\[
  \inn{u,u}=\inn{v,v}=0,\quad \inn{u,v}=1,  
\]
那么我们说 $u,v$ 是一个\emph{双曲对}。
注意到如果 $V$ 是正交几何,那么 $\inn{v,u}=1$,如果 $V$ 是辛几何,那么
$\inn{v,u}=-1$。现在假设 $V$ 是正交几何或者辛几何。
双曲对 $u,v$ 张成的子空间 $\spa{u,v}$
被称为\emph{双曲平面}。若子空间
\[
  H=H_1\oplus H_2\oplus\cdots\oplus H_k,  
\]
其中 $H_i$ 是双曲平面并且两两正交,那么我们说 $H$ 是一个\emph{双曲空间}。
如果 $u_i,v_i$ 是 $H_i$ 的双曲对,那么我们说 $u_1,v_1,\dots,u_k,v_k$
是 $H$ 的\emph{双曲基}(在辛几何的情况下一般被称为辛基)。
根据定义,不难验证双曲空间都是非退化的。

\section{子空间的非退化完备化}

令 $U$ 是非退化向量空间 $V$ 的子空间,如果 $U$ 是退化的,那么寻找
$V$ 中包含 $U$ 的最小非退化子空间是有用的。若 $V$ 的子空间 $S$
包含 $U$,那么我们说 $S$ 是 $U$ 的一个\emph{扩张}。$U$ 的\emph{非退化完备化}
是 $U$ 的所有非退化扩张中最小的子空间。

\begin{theorem}
  $V$ 是非退化的有限维向量空间,当 $V$ 是正交几何的时候我们假设
  $\cha(\mathbb{F})\neq 2$。
  \begin{enumerate}
    \item 令 $S$ 是 $V$ 的子空间。如果 $v$ 是迷向向量并且 $\spa{v}\oplus S$
    是正交直和,那么存在双曲平面 $H=\spa{v,z}$ 使得 $H\oplus S$ 是正交直和。
    特别地,如果 $v$ 是迷向向量,那么存在包含 $v$ 的双曲平面。
    \item 令 $U$ 是 $V$ 的子空间,根据 \autoref{thm:rad decomposition of V},
    可设
    \[
      U=\spa{v_1,\dots,v_k}\oplus W,  
    \]
    其中 $v_1,\dots,v_k$ 是 $\rad(U)$ 的一组基,$W$ 是非退化的。那么存在双曲空间
    $H=H_1\oplus\cdots\oplus H_k$,其双曲基为 $v_1,z_1,\dots,v_k,z_k$,
    使得
    \[
      \bar U=H\oplus W
    \]
    是正交直和,并且 $\bar U$ 是 $U$ 的非退化扩张。此时有
    \[
      \dim\bar U=\dim U+\dim(\rad(U)),
    \]
    我们把 $\bar U$ 称为 $U$ 的一个\emph{双曲扩张}。如果 $U$ 是非退化的,
    那么 $\bar U=U$,即 $U$ 是自身的双曲扩张。 
  \end{enumerate}
\end{theorem}
\begin{proof}
  (1) 根据 \autoref{thm:property of direct complements},$V$ 非退化表明 $(S^\bot)^\bot=S$。
  $v\notin S=(S^\bot)^\bot$ 表明存在 $x\in S^\bot$ 使得 $\inn{v,x}\neq 0$。
  如果 $V$ 是辛几何,令 $z=(1/\inn{v,x})x$,此时 $\inn{v,z}=1$,所以 $v,z$
  是双曲对。如果 $V$ 是正交几何,设 $z=rv+sx\in S^\bot$,那么
  \[
    \inn{v,z}=  r\inn{v,v}+s\inn{v,x}=s\inn{v,x}=1,
  \]
  解得 $s=1/\inn{v,x}$。另一方面,
  \[
    \inn{z,z}=r^2\inn{v,v}+2rs\inn{v,x}+s^2\inn{x,x}=2r+s^2\inn{x,x}=0,  
  \]
  表明 $r=-\inn{x,x}/(2\inn{v,x}^2)$,所以此时 $v,z$ 是双曲对。故总存在
  $z\in S^\bot$ 使得 $H=\spa{v,z}$ 是双曲平面。此时 $H\subseteq S^\bot$,所以
  $S=(S^\bot)^\bot\subseteq H^\bot$,因为 $H$ 非退化,所以 
  \[
    H\cap S\subseteq H\cap H^\bot =0,  
  \]
  所以 $H\oplus S$ 是正交直和。

  (2) 对 $k$ 归纳。$k=1$ 时就是 (1)。假设结论对 $1,\dots,k-1$ 都成立。
  因为
  \[
    U=\spa{v_k}\oplus\bigl(\spa{v_1,\dots,v_{k-1}}\oplus W\bigr),  
  \]
  由 (1),存在双曲平面 $H_k=\spa{v_k,z_k}$ 使得 $H_k\oplus \bigl(\spa{v_1,\dots,v_{k-1}}\oplus W\bigr)$
  是正交直和,根据假设,存在双曲空间 $H'=H_1\oplus\cdots\oplus H_{k-1}$,其
  双曲基为 $v_1,z_1,\dots,v_{k-1},z_{k-1}$,使得 $H'\oplus W$ 是正交直和。
  令 $H=H_k\oplus H'$,那么 $H\oplus W$ 就是正交直和。
\end{proof}

\begin{theorem}[非退化完备化定理]
  令 $U$ 是非退化有限维向量空间 $V$ 的子空间,设 $U=\rad(U)\oplus W$ 是正交直和,
  那么下面的说法是等价的:
  \begin{enumerate}
    \item $\bar U=H\oplus W$ 是 $U$ 的双曲扩张;
    \item $\bar U$ 是 $U$ 的最小非退化扩张;
    \item $\bar U$ 是 $U$ 的非退化扩张并且
    \[
      \dim\bar U=\dim U+\dim(\rad(U)).  
    \]
  \end{enumerate}
\end{theorem}
\begin{proof}
  设 $X$ 是 $U$ 的非退化扩张,考虑 $U$ 在 $X$ 中的双曲扩张
  $H'\oplus W$,所以 $U$ 的任意非退化扩张都包含一个双曲扩张。又因为
  \[
    \dim(H'\oplus W)=  \dim U+\dim(\rad(U)),
  \]
  所以 $U$ 的双曲扩张都拥有同样的维数,这表明它们不可能互相恰当包含,故 $\bar U$ 是 $U$ 的最小非退化扩张。
  这表明 $(1)\Rightarrow (2)\Rightarrow (3)$。

  $(3)\Rightarrow (1)$ 若 $\bar U$ 是 $U$ 的非退化扩张并且
  $\dim\bar U=\dim U+\dim(\rad(U))$,那么可以考虑 $U$ 在 $\bar{U}$ 中
  的双曲扩张 $\tilde{U}$,此时 $\dim\tilde{ U}$ 的维数也为
  $\dim U+\dim(\rad(U))$,所以 $\bar U=\tilde{U}$ 是 $U$ 的双曲扩张。
\end{proof}



\begin{thebibliography}{99}
  \bibitem{LADR}  Axler S. Linear Algebra Done Right. Springer Nature; 2024.
  \bibitem{Albert} Albert A. Regression and the Moore-Penrose Pseudoinverse.
  \bibitem{Roman} Roman S, Axler S, Gehring FW. Advanced Linear Algebra. New York: Springer; 2005 Mar 22.
  \bibitem{LTS} 李烔生, 查建国. 线性代数. 中国科学技术大学出版社; 1989.
  \bibitem{XQH} 姚慕生, 吴泉水, 谢启鸿. 高等代数学. 复旦大学出版社; 2014.
\end{thebibliography}




\end{document}