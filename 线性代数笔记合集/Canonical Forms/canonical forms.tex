\documentclass[fontset=none,zihao=-4]{Notes}

\makeatletter
\DeclareRobustCommand{\em}{%
  \@nomath\em \if b\expandafter\@car\f@series\@nil
  \normalfont \else \bfseries \fi}
\makeatother

\usepackage{tikz-cd,wrapstuff}
\usepackage{fixdif,siunitx,tikz,nicematrix}

\usetikzlibrary{hobby,calc,arrows}
\usetikzlibrary{positioning}
\usetikzlibrary{decorations.markings}
\usetikzlibrary{decorations.pathreplacing}

\ProvidesFile{font.def}

\setCJKmainfont{Source Han Serif SC}[
  UprightFont=*-Regular,
  BoldFont=*-Bold,
  ItalicFont=HYKaiTi S,
  ItalicFeatures={Scale=1.1}
]
\newCJKfontfamily[zhsong]\songti{Source Han Serif SC}[
  UprightFont=*-Regular,
  BoldFont=*-Bold,
  ItalicFont=HYKaiTi S,
  ItalicFeatures={Scale=1.1}
]
\setCJKsansfont{Source Han Sans SC}[
  UprightFont=*-Regular,
  BoldFont=*-Bold
]
\newCJKfontfamily[zhhei]\heiti{Source Han Sans SC}[
  UprightFont=*-Regular,
  BoldFont=*-Bold
]
\setCJKmonofont{HYFangSong S}[
  BoldFont=*,
  ItalicFont=*,
  BoldItalicFont=*
]
\newCJKfontfamily[zhfs]\fangsong{HYFangSong S}[
  BoldFont=*,
  ItalicFont=*,
  BoldItalicFont=*
]
\newCJKfontfamily[zhkai]\kaishu{HYKaiTi S}[
  BoldFont=*,
  ItalicFont=*,
  BoldItalicFont=*
]

\setmainfont{texgyretermes}[
  Extension=.otf,
  UprightFont=*-regular,
  BoldFont=*-bold,
  ItalicFont=*-italic,
  BoldItalicFont=*-bolditalic,
  SlantedFont=*-italic
]
%\setmathrm{texgyretermes}[
%  Extension=.otf,
%  UprightFont=*-regular,
%  BoldFont=*-bold,
%  ItalicFont=*-italic,
%  BoldItalicFont=*-bolditalic,
%  SlantedFont=*-italic
%]
\setsansfont{Cantarell}[
  UprightFont=* Regular,
  ItalicFont=* Italic,
  BoldFont=* Bold,
  BoldItalicFont=* Bold Italic,
  SmallCapsFont=Alegreya Sans SC
]
\setmonofont{Ubuntu Mono}[
  UprightFont=*,
  ItalicFont=* Italic,
  BoldFont=* Bold,
  BoldItalicFont=* Bold Italic
]
%\setmathfont{texgyretermes-math.otf}
%\setmathfont[range={\mathcal,\mathbfcal,\mathfrak},StylisticSet=1]{XITSMath-Regular.otf}
%\setmathfont[range={\mathbb}]{KpMath-Sans.otf}



\DeclareMathOperator\Int{Int}
\DeclareMathOperator\supp{supp}
\DeclareMathOperator\End{End}
\DeclareMathOperator\Ann{Ann}


\tikzcdset{
  arrow style=tikz,
  diagrams={>={Straight Barb[scale=0.8]}}
}

\tikzset{
  point/.style={
    circle,fill,inner sep=0pt,minimum width=5pt
  }
}

\usepackage[subscriptcorrection,nofontinfo,mtpbb]{mtpro2}



\setlist[enumerate]{nosep,label=(\arabic*)}
\setlist[itemize]{nosep}

\title{\sffamily 抽象代数观点下的标准型理论}
\author{Eliauk}


\begin{document}

\maketitle

\tableofcontents

\section{向量空间作为 \texorpdfstring{$\mathbb{F}[x]$}{F[x]}-模}

令 $V$ 是域 $\mathbb{F}$ 上的 $n$ 维向量空间,$\varphi:V\to V$ 是线性变换。
对于 $v\in V$,通过定义 $x\cdot v=\varphi(v)$,使得 $V$
成为一个 $\mathbb{F}[x]$-模。也就是说,对于 $f(x)=a_mx^m+\cdots+a_1x+a_0\in\mathbb{F}[x]$,
我们定义
\[
  f(x)\cdot v=f(\varphi)(v)=a_m\varphi^m(v)+\cdots+a_1\varphi(v)+a_0v.  
\]
不难验证这确实满足模公理。注意,$V$ 的模结构是和线性变换 $\varphi$ 有关的,所以线性变换 $\varphi$
的一些相关概念可以在 $V$ 的 $\mathbb{F}[x]$-模结构中有所体现。
从此开始,后面提到的 $V$ 都指代其相关于 $\varphi$ 的模结构。

我们考虑 $V$ 的零化子 $\Ann(V)$。我们知道 $V$ 上线性变换
全体构成 $n^2$ 维向量空间 $\End(V)$,所以 
$\mathbb{1}_V,\varphi,\dots,\varphi^{n^2}$ 一定 $\mathbb{F}$-线性相关。
这就表明存在不全为零的 $a_1,\dots,a_{n^2}\in\mathbb{F}$ 使得
\[
  a_{n^2}\varphi^{n^2}+\cdots+a_1\varphi+a_0\mathbb{1}_V=0,
\]
所以 $\Ann(V)\neq 0$。又因为 $\mathbb{F}[x]$ 是 Euclid 整环,进而是
主理想整环,故存在首一多项式 $m(x)\in\mathbb{F}[x]$ 使得 $\Ann(V)=(m(x))$。
这里的 $m(x)$ 被称为 $\varphi$ 的\emph{最小多项式}。
任意 $f(x)\in\Ann(V)$ 被称为 $\varphi$ 的\emph{零化多项式}。
由此可得:对于任意零化多项式 $f(x)$,一定有 $m(x)\mid f(x)$。
下面我们用模的语言重新证明 Cayley-Hamilton 定理。

\begin{theorem}[Cayley-Hamilton 定理]
  设 $\varphi:V\to V$ 是线性变换,那么特征多项式 $c(x)=\det(x\mathbb{1}_V-\varphi)$
  是 $\varphi$ 的零化多项式,即 $c(x)\in \Ann(V)$,从而 $m(x)\mid c(x)$。
\end{theorem}
\begin{proof}
  令 $e_1,\dots,e_n$ 是 $V$ 的一组基,设 $\varphi$ 在这组基下的表示矩阵
  为 $A=(a_{ij})$。也就是说,$x\cdot e_j=\varphi(e_j)=\sum_i a_{ij}e_i$,
  即 $\sum_{i}(x\delta_{ij}-a_{ij})e_i=0$,其中 $\delta_{ij}$ 表示 Kronecker
  符号。  
  考虑 $\mathbb{F}[x]$ 上的矩阵 $xI_n-A$,设 $xI_n-A$ 的伴随矩阵为 $B$,
  那么 $C=B(xI_n-A)=\det(xI_n-A)I_n$。$C$ 的 $(i,j)$-元为
  \[
    c_{ij}=\sum_{k=1}^n b_{ik}(x\delta_{jk}-a_{jk})=\det(xI_n-A)\delta_{ij},
  \]
  所以
  \[
    \sum_{j=1}^n c_{ij}e_j=\sum_{j=1}^m\sum_{k=1}^n b_{ik}(x\delta_{jk}-a_{jk})e_j
    =0=\det(xI_n-A)e_i,
  \]
  所以对于任意的 $v\in V$,都有 $\det(xI_n-A)v=0$,故 $c(x)=\det(xI_n-A)\in\Ann(V)$。
\end{proof}

\section{Euclid 整环上的有限生成模}

在本节我们暂时离开向量空间,讨论 Euclid 整环上的有限生成模是如何
发挥作用的。

\begin{lemma}\label{lemma:finite generated module}
  $M$ 是有限生成 $R$-模当且仅当对于某个整数 $n$,存在满的 $R$-模同态 
  $f:R^n\to M$。
\end{lemma}
\begin{proof}
  若 $M$ 是有限生成的,设 $x_1,\dots,x_n$ 是一组生成元,考虑映射 $f:R^n\to M$ 为
  \[
    (a_1,\dots,a_n)\mapsto a_1x_1+\cdots+a_nx_n,
  \]
  容易验证这是一个模同态,并且由于 $x_1,\dots,x_n$ 是生成元,所以 $f$ 是满同态。

  反过来,记 $e_i=(0,\dots,0,1,0,\dots,0)\in R^n$,其中第 $i$ 个分量为 $1$。那么
  任取 $m\in M$,存在 $(a_1,\dots,a_n)\in R^n$ 使得 $f(a_1,\dots,a_n)=m$,
  即
  \[
    m=f(a_1,\dots,a_n)=f(a_1e_1+\cdots+a_ne_n)
    =a_1f(e_1)+\cdots+a_nf(e_n),  
  \]
  故 $M$ 由 $f(e_1),\dots,f(e_n)$ 生成。
\end{proof}

本节的主要目标是证明 Euclid 整环上的有限生成模的结构定理:

\begin{theorem}[基本定理:不变因子型]\label{thm:invariant form}
  $R$ 是 Euclid 整环,$M$ 是有限生成 $R$-模,那么 $M$ 同构于有限多个循环模的直和,即
  \[
    M\simeq R^r\oplus R/(a_1)\oplus R/(a_2)\oplus\cdots  \oplus R/(a_m),
  \]
  其中 $r\geq 0$,$a_1,a_2,\dots,a_m$ 是 $R$ 中非零非单位的元素,并且满足关系
  \[
    a_1\mid a_2\mid\cdots\mid a_m.  
  \]
  此时 $r$ 称为 $M$ 的\emph{自由秩},$a_1,a_2,\dots,a_m$ 称为 $M$ 的\emph{不变因子}。
\end{theorem}

实际上上述定理对于主理想整环上的有限生成模也成立,但是主理想整环没有一个好的算法来寻找
$a_1,a_2,\dots,a_m$,我们将看到 Euclid 整环上的 Euclid 算法会发挥重要作用,并且这也是
线性代数中有理标准型和 Jordan 标准型理论的来源。从现在开始,本节所指的
$R$ 均代表 Euclid 整环。

设 $M$ 是有限生成 $R$-模,根据 \autoref{lemma:finite generated module},我们知道
存在满同态 $f:R^n\to M$,即 $M\simeq R^n/\ker f$。由于
$R$ 是 Noether 环,所以 $R^n$ 是 Noether 模,那么 $R^n$
和 $\ker f$ 都是有限生成 $R$-模,设 $x_1,\dots,x_n$ 是 $R^n$ 的基
(与向量空间中基的定义相同,此时代表 $R^n=Rx_1\oplus\cdots\oplus Rx_n$),
$y_1,\dots,y_m$ 是 $\ker f$ 的生成元,注意此时我们考虑生成元的顺序。那么对于每个 $1\leq i\leq m$,
都存在 $a_{i1},\dots,a_{in}\in R$,使得
\[
  y_i=a_{i1}x_1+\cdots+a_{in}x_n  ,
\]
记矩阵(与线性代数中基的过渡矩阵的写法是一致的)
\[
  A=(a_{ji})=
  \begin{pmatrix}
    a_{11} & a_{21} & \cdots & a_{m1} \\
    a_{12} & a_{22} & \cdots & a_{m2} \\
    \vdots & \vdots & \ddots & \vdots \\
    a_{1n} & a_{2n} & \cdots & a_{mn}
  \end{pmatrix}  \in M_{n,m}(R).
\]
那么矩阵 $ A$ 表述了 $R^n$ 的基和 $\ker f$ 的生成元的关系,我们称其为
$R^n$ 到 $\ker f$ 的\emph{关系矩阵}。

下面我们观察对关系矩阵 $A$ 做初等变换会有什么样的后果。下面是初等列变换。
\begin{enumerate}
  \item 如果用 $u\in R^\times$ 乘以 $A$ 的第 $i$ 列,那么相当于将 $\ker f$ 的生成元
  $y_i$ 变为 $uy_i$,即给出了 $\ker f$ 的一组新的生成元
  $y_1,\dots,uy_i,\dots,y_m$。
  \item 如果交换 $A$ 的第 $i$ 列和第 $j$ 列,那么相当于交换 $\ker f$ 的生成元
  $y_i$ 和 $y_j$,即给出了 $\ker f$ 的一组新的生成元
  $y_1,\dots,y_j,\dots ,y_i,\dots,y_m$。
  \item 如果将 $A$ 的第 $i$ 列乘以 $a\in R$ 加到第 $j$ 列,那么相当于将 $\ker f$
  的生成元 $y_j$ 替换为 $y_j+ay_i$,即给出了 $\ker f$ 的一组新的生成元
  $y_1,\dots,y_i,\dots ,y_j+ay_i,\dots,y_m$。
\end{enumerate}
然后是初等行变换。
\begin{enumerate}
  \item 如果用 $u\in R^\times$ 乘以 $A$ 的第 $i$ 行,那么相当于将 $R^n$ 的基
  $x_i$ 变为 $u^{-1}x_i$,即给出了 $R^n$ 的一组新的基
  $x_1,\dots,u^{-1}x_i,\dots,x_n$。
  \item 如果交换 $A$ 的第 $i$ 行和第 $j$ 行,那么相当于交换 $R^n$ 的基
  $x_i$ 和 $x_j$,即给出了 $R^n$ 的一组新的基
  $x_1,\dots,x_j,\dots ,x_i,\dots,x_n$。
  \item 如果将 $A$ 的第 $i$ 行乘以 $a\in R$ 加到第 $j$ 行,那么相当于将 $R^n$
  的基 $x_i$ 替换为 $x_i-ax_j$,即给出了 $R^n$ 的一组新的基
  $x_1,\dots,x_i-ax_j,\dots ,x_j,\dots,x_n$。
\end{enumerate}
下面的定理告诉我们,通过有限次初等变换,总能将矩阵 $A$ 变为对角阵,
这意味着 $x_1,\dots,x_n$ 的某些倍数将成为 $\ker f$ 的生成元,而
基的倍数还是基(这里使用了 $R$ 是整环的性质),这就表明 
$\ker f$ 也可以写为某些循环子模的直和,从而证明
\autoref{thm:invariant form}。

\begin{theorem}[Smith 标准型]\label{thm:Smith form}
  如果关系矩阵为零矩阵,那么 $\ker f=0$,从而
  $M\simeq R^n$。所以我们假设 $\ker f\neq 0$,
  令 $a_1$ 是初始关系矩阵 $A$ 中所有元素的最大公因子。
  那么通过上述六种初等变换,可以将 $A$ 变换为
  \[
    \begin{pmatrix}
      D_k & \\
      & O_{n-k,m-k}
    \end{pmatrix}  ,
  \]
  其中 $D_k$ 是对角阵,对角线为 $a_1,a_2,\dots,a_k\ (k\leq m)$,并且满足
  $a_1\mid a_2\mid\cdots\mid a_k$。上述对角阵被称为 $A$
  的 Smith 标准型。
\end{theorem}
\begin{proof}
  \uppercase\expandafter{\romannumeral 1}. 
  首先注意到初等行或者列变换并不会改变所有元素的最大公因子,因为新矩阵的元素显然被
  原矩阵元素的最大公因子整除,而初等变换的操作是可逆的,所以原矩阵的元素又被新矩阵元素的
  最大公因子整除,所以新矩阵和原矩阵元素的最大公因子相同。

  我们先考虑对第一行的操作:假设第一行的最大公因子是 $d=\gcd(a_{11},\dots,a_{m1})$,
  那么我们可以按照下面的方法将 $d$ 放到第一行第一列:
  \begin{itemize}[nosep]
    \item 首先仅涉及前两列,使用辗转相除法,将 $\gcd(a_{11},a_{21})$ 放到第一行第一列;
    \item 然后仅涉及第一列和第三列,使用辗转相除法,将 $\gcd(\gcd(a_{11},a_{21}),a_{31})
    =\gcd(a_{11},a_{21},a_{31})$ 放到第一行第一列;
    \item 重复上述步骤,直到把 $d$ 放到第一行第一列。
  \end{itemize}
  上述操作可以导出下面的算法:
  \begin{enumerate}[nosep]
    \item 使用上述操作将 $d=\gcd(a_{11},\dots,a_{m1})$ 放到第一行第一列;
    \item 此时 $d$ 整除 $a_{21},\dots,a_{m1}$,那么我们可以将第一行除开 $d$ 之外全变为零;
    \item 如果此时矩阵的所有元素都被 $d$ 整除,那么操作停止。否则,存在元素 $a_{ji}\,(i>1)$,
    使得 $d\nmid a_{ji}$;
    \item 如果 $d\nmid a_{1i}$,那么我们跳过这一步。否则,我们将第 $j$ 列
    加到第 $1$ 列使得 $a\nmid a_{1i}$,注意此时并没有改变第一行;
    \item 计算新的最大公因子 $d'=\gcd(d,a_{1i})$,通过仅涉及第 $1$ 行和第 $i$ 行的操作,
    可以将 $d'$ 放到第一行第一列,注意到通过第四步,有 $d\nmid a_{1i}$,所以
    $d'$ 的次数严格小于 $d$ 和 $a_{1i}$ 的次数;
    \item 回到操作 1。
  \end{enumerate}
  这个算法必然在有限步结束,因为每进行一次第 (5) 步,第一行第一列的元素的次数就会严格变小,
  最多也只能变到零次,从而在第 (3) 步退出操作。最后我们就得到了 $a_{11}$ 整除所有其他元素的新
  矩阵,而根据前面的观察,这个新矩阵元素的最大公因子等于初始矩阵元素的最大公因子,而新矩阵
  元素的最大公因子显然就是 $a_{11}$,这就完成了第 \uppercase\expandafter{\romannumeral 1}. 步。

  \uppercase\expandafter{\romannumeral 2}.  
  在 \uppercase\expandafter{\romannumeral 1}. 的基础上,
  使用初等变换将第一行第一列的非 $(1,1)$-元全变成零即可。

  \uppercase\expandafter{\romannumeral 3}. 
  在 \uppercase\expandafter{\romannumeral 2}.  的基础上,右下角的分块矩阵又回到了 
  \uppercase\expandafter{\romannumeral 1}. 的形式,注意到对该分块
  矩阵的操作不会影响大矩阵的第一行第一列,所以重复 \uppercase\expandafter{\romannumeral 1}. 的操作,
  又可以将这个分块矩阵化为 \uppercase\expandafter{\romannumeral 2}. 的形式,即使得 $a_1\mid a_2$ 并且 $a_2$ 整除其他所有元素。

  \uppercase\expandafter{\romannumeral 4}. 重复前三步的操作,便可以得到形如
  $
  \begin{psmallmatrix}
    D_k & 0 \\ 0 & O_{n-k,m-k}
  \end{psmallmatrix}
  $
  的关系矩阵。
\end{proof}

\begin{proof}[Proof of~~\autoref{thm:invariant form}]
  根据前面的叙述,$M\simeq R^n/\ker f$,通过初等变换将关系矩阵变为
  Smith 标准型后,此时存在 $R^n$ 的一组基 $x_1,\dots,x_n$ 
  和满足 $a_1\mid \cdots\mid a_k$ 的元素 $a_1,\dots,a_k\in R$,
  使得 $a_1x_1,\dots,a_kx_k$ 是 $\ker f$ 的生成元(从而是 $\ker f$
  的基),那么就有
  \begin{align*}
    M&\simeq R^n/\ker f
    =(Rx_1\oplus\cdots\oplus Rx_n)/(Ra_1x_1\oplus \cdots\oplus Ra_kx_k\oplus 0\oplus\cdots \oplus0)\\
    &\simeq R/(a_1)\oplus R/(a_2)\oplus\cdots\oplus R/(a_k)\oplus R^{n-k}  .
    \qedhere
  \end{align*}
\end{proof}

\autoref{thm:invariant form} 的结果还可以进一步改进,设 $a$ 是 $R$ 中的非零非单位的元素,
那么 $a$ 可以唯一分解为 $a=p_1^{\alpha_1}\cdots p_s^{\alpha_s}$,其中 $p_i$ 为互不相同的素元,
那么根据中国剩余定理,有环同构(同时也是 $R$-模同构)
\[
  R/(a)\simeq R/(p_1^{\alpha_1})  \oplus R/(p_2^{\alpha_2})\oplus\cdots\oplus R/(p_s^{\alpha_s}).
\]
\autoref{thm:invariant form} 的右端每一项都可以做类似的分解,于是我们得到了下面定理。

\begin{theorem}[基本定理:初等因子型]\label{thm:element form}
  $R$ 是 Euclid 整环,$M$ 是有限生成 $R$-模,那么 $M$ 同构于有限多个循环模的直和,即
  \[
    M\simeq R^r\oplus R/(p_1^{\alpha_1})\oplus R/(p_2^{\alpha_2})\oplus\cdots  \oplus R/(p_t^{\alpha_t}),
  \]
  其中 $r\geq 0$,$p_1^{\alpha_1},p_2^{\alpha_2},\dots,p_t^{\alpha_t}$ 是 $R$ 中素元的幂次
  (不需要互不相同)。 $p_1^{\alpha_1},p_2^{\alpha_2},\dots,p_t^{\alpha_t}$
  被称为 $M$ 的\emph{初等因子}。
\end{theorem}

虽然我们已经得到了 Euclid 整环上有限生成模的两种分解,但是还有一个问题,这两种分解
是否唯一?答案是肯定的,但是证明略为繁琐,见参考文献 \cite{Dummit}。

\section{有理标准型和 Jordan 标准型}

从名字就可以看出,Euclid 整环上有限生成模的不变因子分解和初等因子分解
与线性代数中的标准型理论密切相关,事实上也确实如此。下面我们回到对向量空间的研究。

给定 $n$ 维 $\mathbb{F}$-向量空间 $V$,设有一组基 $e_1,\dots,e_n$,
$\varphi$ 是 $V$ 上的线性变换,那么 $V$ 可以成为一个
$\mathbb{F}[x]$-模,通过定义 $x\cdot v=\varphi(v)$。
$V$ 当然是一个有限生成 $\mathbb{F}[x]$-模,而 $\mathbb{F}[x]$
为 Euclid 整环,这允许我们使用前一节的所有结果。

根据 \autoref{thm:invariant form},我们知道存在多项式
$a_1(x),\dots,a_m(x)\in \mathbb{F}[x]$,满足
\[
  a_1(x)\mid \cdots\mid a_m(x),  
\]
使得我们有 $\mathbb{F}[x]$-模同构(自然也是 $\mathbb{F}$-向量空间同构)
\begin{equation}\label{eq:decomposition of vector space}
  \sigma:V\simeq \mathbb{F}[x]/(a_1(x))\oplus \cdots\oplus \mathbb{F}[x]/(a_m(x)).
\end{equation}
注意,由于 $V$ 上的任意线性变换都存在非零的最小多项式,所以 $V$ 作为
$\mathbb{F}[x]$-模是扭模,故自由秩部分为零。观察 \eqref{eq:decomposition of vector space}
式的右端,容易发现
\[
  \Ann(V)=\Ann\bigl(\mathbb{F}[x]/(a_1(x))\oplus \cdots\oplus \mathbb{F}[x]/(a_m(x))\bigr)  
  =(a_m(x)).
\]
此时我们进一步规定不变因子 $a_1,\dots,a_m$ 都是首一多项式,那么根据不变因子分解的唯一性,
我们便得到了下面的命题。

\begin{proposition}
  线性变换 $\varphi$ 的最小多项式 $m(x)$ 就是 $V$ 作为 $\mathbb{F}[x]$-模的最大的
  不变因子,并且 $V$ 的所有不变因子都整除 $m(x)$。
\end{proposition}

我们继续发掘 \eqref{eq:decomposition of vector space} 式带给我们的信息。
注意到 \eqref{eq:decomposition of vector space} 式也是向量空间同构,所以
这意味着我们通过 $V$ 的 $\mathbb{F}[x]$-模结构给出了一种新的向量空间的直和分解,
那么我们自然会寻找 $\mathbb{F}[x]/(a_i(x))$ 的一组基,然后将其拼成 $V$
的一组基,观察 $\varphi$ 在这组基下的表示矩阵是什么形态,实际上,这正是
$\varphi$ 的有理标准型。

对于线性变换 $\varphi:V\to V$,$\sigma\circ \varphi\circ \sigma^{-1}$
是向量空间 $\mathbb{F}[x]/(a_1(x))\oplus \cdots\oplus \mathbb{F}[x]/(a_m(x))$
上的线性变换。对于任意 $f(\bar x)\in \mathbb{F}[x]/(a_i(x))$,
那么
\[
  \sigma\circ \varphi\circ \sigma^{-1}\bigl(f(\bar x)\bigr)
  =\sigma\bigl(x\sigma^{-1}\bigl(f(\bar x)\bigr)\bigr)
  =x\sigma\bigl(\sigma^{-1}\bigl(f(\bar x)\bigr)\bigr)
  =x\cdot f(\bar x)=\bar x f(\bar x).
\]
这表明 $\varphi$ 实际上对应着 $\mathbb{F}[x]/(a_i(x))$ 上的
线性变换 $f(\bar x)\mapsto \bar xf(\bar x)$,我们只需要寻找这个线性变换
在给定基下的表示矩阵即可。

对于首一多项式 $a(x)\in\mathbb{F}[x]$,我们考察 $\mathbb{F}[x]$-模
$\mathbb{F}[x]/(a(x))$,这自然是 $\mathbb{F}$-模,即 $\mathbb{F}$-向量空间。
如果 $a(x)=x^k+a_{k-1}x^{k-1}+\cdots+a_0$,读者可以自行验证 $\mathbb{F}[x]/(a(x))$
有一组基 $\bar 1,\bar x,\dots,\bar x^{k-1}$(利用带余除法)。
那么对于线性变换 $\mathbb{F}[x]/(a(x))$ 上的线性变换 $\psi:f(\bar x)\mapsto \bar xf(\bar x)$,
我们有
\[
  \psi(\bar 1)=\bar x, 
  \psi(\bar x)=\bar x^2,\dots,
  \psi(\bar x^{k-1})=\bar x^k=-a_0-a_1\bar x-\cdots-a_{k-1}\bar x^{k-1},
\]
故 $\psi$ 在基 $\bar 1,\bar x,\dots,\bar x^{k-1}$ 下的表示矩阵为
\[
  \mathcal{C}_{a(x)}=\begin{pmatrix}
    0 & 0 & \cdots & 0 & -a_0\\
    1 & 0 & \cdots & 0 & -a_1\\
    0 & 1 & \cdots & 0 & -a_2\\
    \vdots & \vdots & \ddots & \vdots & \vdots \\
    0 & 0 & \cdots & 1 & -a_{k-1}
  \end{pmatrix}  .
\]
上述矩阵 $\mathcal{C}_{a(x)}$ 被称为首一多项式 $a(x)$ 的\emph{友阵},
不难证明 $\mathcal{C}_{a(x)}$ 的特征多项式恰好就是 $a(x)$。

综合上面的叙述,我们得到如下重要的定理:
\begin{theorem}[线性变换的有理标准型]
  $V$ 是域 $\mathbb{F}$ 上的有限维向量空间,$\varphi:V\to V$ 是线性变换,那么
  存在 $V$ 的一组基和不变因子 $a_1(x),\dots,a_m(x)\in\mathbb{F}[x]$,
  使得 $\varphi$ 在这组基下的表示矩阵是有 $m$ 块的分块对角阵,且每个分块是
  首一多项式 $a_i(x)$ 的友阵。这个分块对角阵我们称之为 $\varphi$
  的\emph{有理标准型}。此外,根据不变因子的唯一性,线性变换的有理标准型也是唯一的。
\end{theorem}

“有理”的含义为对于任意域 $\mathbb{F}$,这样的标准型总是可以计算的,我们将马上看到
Jordan 标准型丢失了这种性质。有理标准型的重要性在于其完全解决了线性变换(矩阵)
在相似意义下的分类问题,即下面的定理。

\begin{theorem}\label{thm:similar equiv rational form}
  设 $\varphi,\psi$ 是 $V$ 上的线性变换,那么:
  \begin{enumerate}
    \item $\varphi$ 和 $\psi$ 相似;
    \item 通过 $\varphi$ 得到的 $\mathbb{F}[x]$-模 $V_1$
    和通过 $\psi$ 得到的 $\mathbb{F}[x]$-模 $V_2$ 是同构的;
    \item $\varphi$ 和 $\psi$ 的有理标准型相同。
  \end{enumerate}
\end{theorem}
\begin{proof}
  $(1)\Rightarrow (2)$ 若 $\varphi$ 和 $\psi$ 相似,即存在同构 $\sigma$
  使得 $\varphi=\sigma\circ \psi\circ\sigma^{-1}$。对于
  $\sigma:V_2\to V_1$,我们说明 $\sigma$ 是 $\mathbb{F}[x]$-模同构即可。
  只需验证 $\sigma$ 是模同态,任取 $v\in V$,我们有
  \[
    \sigma(xv)=\sigma(\psi(v))=\varphi(\sigma(v))=x\sigma(v),
  \]
  所以 $\sigma$ 是 $\mathbb{F}[x]$-模同态。

  $(2)\Rightarrow (3)$ 根据不变因子的唯一性即得。

  $(3)\Rightarrow (1)$ 显然。
\end{proof}

由于 $a(x)$ 的友阵的特征多项式就是 $a(x)$,$\varphi$ 的有理标准型又是友阵构成
的分块对角阵,所以 $\varphi$ 的特征多项式就是这些友阵的特征多项式之积,进而是
$\varphi$ 的不变因子之积,而 $\varphi$ 的最小多项式是最大的不变因子,
所以从这一点我们可以再次推出 Cayley-Hamilton 定理。此外,由于不变因子的整除
关系,所以 $\varphi$ 的最小多项式和特征多项式的根一定相同(不考虑重数)。
总结起来,我们便得到了下面的命题。

\begin{proposition}
  $\varphi$ 是向量空间 $V$ 上的线性变换,那么:
  \begin{enumerate}
    \item $\varphi$ 的特征多项式是 $\varphi$ 的不变因子之积。
    \item $\varphi$ 的特征多项式整除极小多项式的某个幂次。特别地,$A$ 的极小多项式
    和特征多项式有相同的根(不考虑重数)。
  \end{enumerate}
\end{proposition}

由前面的讨论可知,如果 $\varphi$ 的特征多项式在中 $\mathbb{F}[x]$ 能完全分裂为
一次多项式,即域 $\mathbb{F}$ 包含 $\varphi$ 的所有特征值,那么其不变因子
便能全部分解为一次多项式的乘积,这就方便我们使用 \autoref{thm:element form}。
为了方便起见,我们直接假设 $\mathbb{F}$ 是代数闭域,即 $\mathbb{F}[x]$ 中的不可约多项式
只有一次多项式。我们应用 \autoref{thm:element form},此时 $V$
的初等因子只可能形如 $(x-\lambda)^k$,于是 \autoref{thm:element form}
告诉我们:存在 $k_1,\dots,k_t$ 和 $\lambda_1,\dots,\lambda_t\in\mathbb{F}$
(其中 $\lambda_1,\dots,\lambda_t$ 包含了 $\varphi$ 的所有特征值,但允许重复出现),
使得
\begin{equation}
  \tau:V\simeq \mathbb{F}[x]/(x-\lambda_1)^{k_1}\oplus\cdots\oplus
  \mathbb{F}[x]/(x-\lambda_t)^{k_t}.
\end{equation}
类似有理标准型的做法,我们需要寻找 $\mathbb{F}[x]/(x-\lambda)^k$ 上线性变换
$\psi:f(\bar x)\mapsto \bar xf(\bar x)$ 在给定基下的表示矩阵。
此时我们取基为 $(\bar x-\lambda)^{k-1},\dots,\bar x-\lambda,1$,那么
\begin{align*}
  \psi(\bar x-\lambda)^{k-1}&=(\bar x-\lambda+\lambda)(\bar x-\lambda)^{k-1}
  =\lambda(\bar x-\lambda)^{k-1},\\
  \psi(\bar x-\lambda)^{k-2}&=(\bar x-\lambda+\lambda)(\bar x-\lambda)^{k-2}
  =\lambda(\bar x-\lambda)^{k-2}+(\bar x-\lambda)^{k-1},\\
  &\cdots\\
  \psi(1)&=\bar x=\lambda\cdot 1+(\bar x-\lambda)  ,
\end{align*}
也就是说,$\psi$ 在基 $(\bar x-\lambda)^{k-1},\dots,\bar x-\lambda,1$ 下
的表示矩阵为
\[
  J_k=
  \begin{pmatrix}
    \lambda & 1 & & & \\
    & \lambda & \ddots & & \\
    & & \ddots & 1 &  \\
    & & & \lambda & 1 \\
    & & & & \lambda
  \end{pmatrix}  .
\]
上述矩阵被称为 $k$ 阶\emph{Jordan 块}。于是我们得到了下面的定理。

\begin{theorem}[线性变换的 Jordan 标准型]
  $V$ 是域 $\mathbb{F}$ 上的有限维向量空间,$\varphi:V\to V$ 是线性变换,如果
  $\mathbb{F}$ 包含 $\varphi$ 的所有特征值,那么
  存在 $V$ 的一组基和初等因子 $(x-\lambda_1)^{k_1},\dots,(x-\lambda_t)^{k_t}\in\mathbb{F}[x]$,
  使得 $\varphi$ 在这组基下的表示矩阵是有 $t$ 块的分块对角阵,且每个分块是
  $k_t$ 阶 Jordan 块。这个分块对角阵我们称之为 $\varphi$
  的\emph{Jordan 标准型}。此外,根据初等因子的唯一性,线性变换的 Jordan 标准型也是唯一的。
\end{theorem}

借助 \autoref{thm:similar equiv rational form},我们知道线性变换
$\varphi,\psi$ 相似当且仅当它们有相同的有理标准型,而 Jordan 标准型由有理标准型
唯一确定,所以 $\varphi,\psi$ 相似当且仅当它们有相同的 Jordan 标准型。


\section{标准型的计算}

在线性代数中,我们知道计算线性变换 $\varphi:V\to V$ 的有理标准型或者 Jordan 标准型的一般
步骤是对特征矩阵 $xI_n-A$ 做初等变换,变换为 Smith 标准型后对角线
便是所有的不变因子,现在我们来揭示线性变换的特征矩阵和不变因子之间的关系。
取 $\mathbb{F}[x]^n$ 的一组基
为 $\xi_1,\dots,\xi_n$,$V$ 的一组基 $e_1,\dots,e_n$。
定义 $f:\mathbb{F}[x]^n\to V$ 满足 $f(\xi_i)=e_i$,那么 $f$ 是满同态。
回顾 \autoref{thm:Smith form},我们说明
$\varphi$ 的特征矩阵实际上就是一个关系矩阵!

\begin{proposition}\label{prop:characteristic matrix as relation matrix}
  如果 $\varphi$ 在上述基 $e_1,\dots,e_n$ 下的表示矩阵为 $A$,特征矩阵
  \[
    xI_n-A=
    \begin{pmatrix}
      x-a_{11} & -a_{12} & \cdots & -a_{1n}\\
      -a_{21} & x-a_{22} & \cdots & -a_{2n}\\
      \vdots & \vdots & \ddots & \vdots \\
      -a_{n1} & -a_{n2} & \cdots & x-a_{nn}
    \end{pmatrix}  ,
  \]
  那么 $xI_n-A$ 是 $\mathbb{F}[x]^n$ 到 $\ker f$ 的关系矩阵。
\end{proposition}
\begin{proof}
  也就是说我们要证明 $1\leq i\leq n$ 时,
  \[
    \mu_i=-a_{1i}\xi_1-\cdots-a_{i-1,i}\xi_{i-1}+(x-a_{ii})\xi_i-a_{i+1,i}\xi_{i+1}
    -\cdots-a_{ni}\xi_n  
  \]
  组成了 $\ker f$ 的一组生成元。

  直接验证可知
  \[
    f(\mu_i)=-(a_{1i}e_1+\cdots+a_{ni}e_n)+\varphi(e_i)=0,  
  \]
  所以 $\mu_i\in\ker f$。

  由于 
  \[
    x\xi_i=\mu_i+\sum_{j=1}^na_{ji}\xi_j\in 
    (\mathbb{F}[x]\mu_1+\cdots \mathbb{F}[x]\mu_n)+(\mathbb{F}\xi_1+\cdots+\mathbb{F}\xi_n),
  \]
  所以
  \[
    x^2\xi_i=x(x\xi_i)=  x\mu_i+\sum_{j=1}^na_{ji}(x\xi_j)
    \in (\mathbb{F}[x]\mu_1+\cdots \mathbb{F}[x]\mu_n)+(\mathbb{F}\xi_1+\cdots+\mathbb{F}\xi_n),
  \]
  以此类推,可知任意的 $k\geq 0$ 有
  \[
    x^k\xi_i\in   (\mathbb{F}[x]\mu_1+\cdots \mathbb{F}[x]\mu_n)+(\mathbb{F}\xi_1+\cdots+\mathbb{F}\xi_n),
  \]
  所以
  \[
    \mathbb{F}[x]^n=\mathbb{F}[x]\xi_1+\cdots+\mathbb{F}[x]\xi_n
    =   (\mathbb{F}[x]\mu_1+\cdots \mathbb{F}[x]\mu_n)+(\mathbb{F}\xi_1+\cdots+\mathbb{F}\xi_n).
  \]
  那么任取 $\sum_{i=1}^ng_i(x)\mu_i+\sum_{i=1}^n a_i\xi_i\in \ker f$,就有
  \[
    \sum_{i=1}^n a_ie_i=\varphi\left(\sum_{i=1}^ng_i(x)\mu_i+\sum_{i=1}^n a_i\xi_i\right) =0,
  \]
  故 $a_i=0$,这就证明了 $\mu_1,\dots,\mu_n$ 是 $\ker f$ 的生成元。
\end{proof}

至此,结合 \autoref{thm:Smith form},便介绍完了模论在两大标准型中的应用以及
标准型的计算方法。下面我们用两个例子结束本文,这两个例子足以体现一般情况下
两种标准型包括变换矩阵的计算方法。需要注意的是,在计算 Jordan 标准型的时候,
需要读者熟悉中国剩余定理中的同构关系如何给出。

\begin{example}
  给定 $\mathbb{R}$ 上的矩阵
  \[
    A=\begin{pmatrix}
      2 & -2 & 14 \\
      0 & 3 & -7 \\
      0 & 0 & 2
    \end{pmatrix} , 
  \]
  计算 $A$ 的有理标准型并给出变换矩阵。
\end{example}
\begin{solution}
  $A$ 的特征矩阵为
  \[
    xI_3-A=\begin{pmatrix}
      x-2 & 2 & -14 \\
      0 & x-3 & 7 \\
      0 & 0 & x-2
    \end{pmatrix}  ,
  \]
  任取 $\mathbb{R}[x]^3$ 的基 $[\xi_1,\xi_2,\xi_3]$,下面我们对 
  $xI_3-A$ 进行初等变换,并记录基的变化。我们用 $R$ 指代行,
  $C$ 指代列。可以验证依次通过下面的初等变换:
  \begin{gather*}
    C_1\leftrightarrow C_2,R_2+\frac{-(x-3)}{2}R_1\mapsto R_2,
    C_2+\frac{-(x-2)}{2}C_1\mapsto C_2,
    C_3+7C_1\mapsto C_3,\\
    C_2\leftrightarrow C_3,R_3-\frac{1}{7}R_2\mapsto R_3,
    C_3+\frac{x-3}{14}C_2\mapsto C_3,
    \frac{1}{2}C_1,\frac{1}{7}C_2,14C_3,
  \end{gather*}
  便可以将 $xI_3-A$ 变为 Smith 标准型
  \[
    \begin{pmatrix}
      1 & 0 & 0 \\
      0  & x-2& 0 \\
      0 & 0 & x^2-5x+6
    \end{pmatrix}.
  \]
  我们仅关注初等行变换,其相当于对基进行了变换:
  \[
    [\xi_1,\xi_2,\xi_3]\to \left[\xi_1+\frac{(x-3)}{2}\xi_2,\xi_2,\xi_3\right]
    \to \left[\xi_1+\frac{(x-3)}{2}\xi_2,\xi_2+\frac{1}{7}\xi_3,\xi_3\right],
  \]
  令 $[\mu_1,\mu_2,\mu_3]$ 为上述变换后的基,此时有同构
  \begin{align*}
    &\hphantom{{}={}}\mathbb{R}[x]^3/\ker f
    \\&=(\mathbb{R}[x]\mu_1\oplus\mathbb{R}[x]\mu_2\oplus\mathbb{R}[x]\mu_3)
    /(\mathbb{R}[x]\mu_1\oplus \mathbb{R}[x](x-2)\mu_2\oplus\mathbb{R}[x](x^2-5x+6)\mu_3)\\
    &\simeq \mathbb{R}[x]/(x-2)\oplus \mathbb{R}[x]/(x^2-5x+6),
  \end{align*}
  同构映射为
  \[
    h_1\mu_1+h_2\mu_2+h_3\mu_3+\ker f\mapsto (h_2+(x-2),h_3+(x^2-5x+6)),
  \]
  另一方面,又有 $\mathbb{R}[x]^3/\ker f\simeq \mathbb{R}^3$,
  取 $\mathbb{R}^3$ 的基 $e_1,e_2,e_3$,此时同构映射为
  \[
    h_1\xi_1+h_2\xi_2+h_3\xi_3+\ker f\mapsto h_1e_1+h_2e_2+h_3e_3.
  \]
  所以 $\mathbb{R}[x]/(x-2)\oplus \mathbb{R}[x]/(x^2-5x+6)\simeq\mathbb{R}^3$ 的同构映射满足
  \begin{align*}
    (1, 0)&\mapsto \mu_2\mapsto e_2+\frac{1}{7}e_3,\\ 
    (0,1)&\mapsto \mu_3\mapsto e_3,\\
    (0,\bar x)&\mapsto x\mu_3\mapsto xe_3=14e_1-7e_2+2e_3.
  \end{align*}
  所以 $A$ 在 $[e_2+\frac{1}{7}e_3,e_3,14e_1-7e_2+2e_3]$ 这组基下变换为
  有理标准型,即
  \[
    P=\begin{pmatrix}
      0 & 0 & 14 \\
      1 & 0 & -7 \\
      \frac{1}{7} & 1 & 2
    \end{pmatrix} ,\quad
    P^{-1}AP=\begin{pmatrix}
      2 & 0 & 0 \\
      0 & 0 & -6 \\
      0 & 1 & 5
    \end{pmatrix}.\qedhere
  \]
\end{solution}

\begin{example}
  将上例中的矩阵 $A$ 变换为 Jordan 标准型。
\end{example}

\begin{solution}
  初等因子为 $x-2,x-2,x-3$。故我们有同构
  \[
    \mathbb{R}[x]/(x-2)\oplus\mathbb{R}[x]/(x-2)\oplus\mathbb{R}[x]/(x-3)
    \simeq\mathbb{R}[x]/(x-2)\oplus \mathbb{R}[x]/(x^2-5x+6)\simeq\mathbb{R}^3,
  \]
  由于 $1\cdot(x-2)+(-1)\cdot (x-3)=1$,所以同构映射满足
  \begin{align*}
    (1,0,0)&\mapsto (1,0)\mapsto \mu_2\mapsto e_2+\frac{1}{7}e_3,\\
    (0,1,0)&\mapsto (0,-\bar x+3)\mapsto (-x+3)\mu_3\mapsto
    (-x+3)e_3=-14e_1+7e_2+e_3,\\
    (0,0,1)&\mapsto (0,\bar x-2)\mapsto (x-2)\mu_3\mapsto (x-2)e_3=14e_1-7e_2,
  \end{align*}
  所以 $A$ 在 $[e_2+\frac{1}{7}e_3,-14e_1+7e_2+e_3,14e_1-7e_2]$ 这组基下变换为 Jordan 标准型,即
  \[
    P=\begin{pmatrix}
      0 & -14 & 14 \\
      1 & 7 & -7 \\
      \frac{1}{7} & 1 & 0
    \end{pmatrix},\quad P^{-1}AP=
    \begin{pmatrix}
      2 & 0 & 0 \\
      0 & 2 & 0 \\
      0 & 0 & 3
    \end{pmatrix}.\qedhere
  \]
\end{solution}

\begin{example}
  给定 $\mathbb{R}$ 上的矩阵
  \[
    A=\begin{pmatrix}
      1 & 0 & 0 & 0 \\
      0 & 1 &  0 & 0 \\
      -2 & -2 & 0 & 1 \\
      -2 & 0 & -1 & -2
    \end{pmatrix}  ,
  \]
  计算 $A$ 的有理标准型和 Jordan 标准型,并给出变换矩阵。
\end{example}
\begin{solution}
  $A$ 的特征矩阵为
  \[
    xI_4-A=\begin{pmatrix}
      x-1 & 0 & 0 & 0 \\
      0 & x-1 & 0 & 0 \\
      2 & 2 & x & -1 \\
      2 & 0 & 1 & x+2
    \end{pmatrix} ,
  \]  
  任取 $\mathbb{R}[x]^4$ 的一组基 $[\xi_1,\xi_2,\xi_3,\xi_4]$。依次通过初等变换
  \begin{gather*}
    R_3\leftrightarrow R_1,C_2-C_1\mapsto C_2,C_3-\frac{x}{2}C_1\mapsto C_3,
    C_4+\frac{1}{2}C_1\mapsto C_4,\\
    R_3-\frac{x-1}{2}R_1\mapsto R_3,R_4-R_1\mapsto R_4,R_4\leftrightarrow R_2,
    R_3+\frac{-x+1}{2}R_2\mapsto R_3,\\
    R_4+\frac{x-1}{2}R_2\mapsto R_4,C_3+\frac{-x+1}{2}C_2\mapsto C_3,
    C_4+\frac{x+3}{2}C_2\mapsto C_4,\\
    R_4-(x-1)R_3\mapsto R_4,C_4-(x+2)C_3\mapsto C_4,\frac{1}{2}C_1,-\frac{1}{2}C_2,
    -2C_3,2C_4,
  \end{gather*}
  将 $xI_4-A$ 变为 Smith 标准型
  \[
    \begin{pmatrix}
      1 & 0 & 0 & 0 \\
      0 & 1 & 0 & 0 \\
      0 & 0 & x-1 & 0 \\
      0 & 0 & 0 & (x-1)(x+1)^2
    \end{pmatrix}  .
  \]
  对应的基变换为
  \begin{align*}
    [\xi_1,\xi_2,\xi_3,\xi_4]&\to \left[\xi_3,\xi_2,\xi_1,\xi_4\right]
    \to \left[\xi_3+\frac{x-1}{2}\xi_1,\xi_2,\xi_1,\xi_4\right]\\
    &\to \left[\xi_3+\frac{x-1}{2}\xi_1+\xi_4,\xi_2,\xi_1,\xi_4\right]
    \to\left[\xi_3+\frac{x-1}{2}\xi_1+\xi_4,\xi_4,\xi_1,\xi_2\right]\\
    &\to\left[\xi_3+\frac{x-1}{2}\xi_1+\xi_4,\xi_4+\frac{x-1}{2}\xi_1,\xi_1,\xi_2\right]\\
    &\to\left[\xi_3+\frac{x-1}{2}\xi_1+\xi_4,\xi_4+\frac{x-1}{2}\xi_1-\frac{x-1}{2}\xi_2,\xi_1,\xi_2\right]\\
    &\to\left[\xi_3+\frac{x-1}{2}\xi_1+\xi_4,\xi_4+\frac{x-1}{2}\xi_1-\frac{x-1}{2}\xi_2,\xi_1+(x-1)\xi_2,\xi_2\right].
  \end{align*}
  这表明 $\mathbb{R}[x]/(x-1)\oplus\mathbb{R}[x]/(x^3+x^2-x-1)\simeq\mathbb{R}^3$
  的同构映射满足:
  \begin{align*}
    (1,0)&\mapsto e_1+(x-1)e_2=e_1-2e_3,\\
    (0,1)&\mapsto e_2,\\
    (0,\bar x)&\mapsto xe_2=e_2-2e_3,\\
    (0,\bar x^2)&\mapsto x^2e_2=e_2-2e_3+2e_4,
  \end{align*}
  所以 $A$ 在 $[e_1-2e_3,e_2,e_2-2e_3,e_2-2e_3+2e_4]$ 这组基下变换为
  有理标准型,即
  \[
    P=
    \begin{pmatrix}
      1 & 0 & 0 & 0 \\
      0 & 1 & 1 & 1 \\
      -2 & 0 & -2 & -2 \\
      0 & 0 & 0 & 2
    \end{pmatrix}  ,\quad
    P^{-1}AP=
    \begin{pmatrix}
      1 & 0 & 0 & 0 \\
      0 & 0 & 0 & 1 \\
      0 & 1 & 0 & 1 \\
      0 & 0 & 1 & -1
    \end{pmatrix}.
  \]
  由于 $\frac{-x-3}{4}(x-1)+\frac{1}{4}(x+1)^2=1$,所以
  \[
    \mathbb{R}[x]/(x-1)\oplus\mathbb{R}[x]/(x-1)\oplus\mathbb{R}[x]/(x+1)^2\simeq
    \mathbb{R}[x]/(x-1)\oplus\mathbb{R}[x]/(x^3+x^2-x-1)\simeq \mathbb{R}^3  
  \]
  的同构映射满足:
  \begin{align*}
    (1,0,0)&\mapsto (1,0)\mapsto e_1-2e_3,\\
    (0,1,0)&\mapsto \left(0,\frac{1}{4}(\bar x+1)^2\right)
    \mapsto \frac{(x+1)^2}{4}e_2=e_2-\frac{3}{2}e_3+\frac{1}{2}e_4,\\
    (0,0,\bar x+1)&\mapsto \left(0,-\frac{1}{4}(\bar x^3+3\bar x^2-\bar x-3)\right)
    \mapsto -\frac{x^3+3x^2-x-3}{4}e_2=e_3-e_4,\\
    (0,0,1)&\mapsto \left(0,-\frac{1}{4}(\bar x^2+2\bar x-3)\right)\mapsto
    -\frac{x^2+2x-3}{4}e_2=\frac{3}{2}e_3-\frac{1}{2}e_4,
  \end{align*}
  所以 $A$ 在 
  $[e_1-2e_3,e_2-\frac{3}{2}e_3+\frac{1}{2}e_4,e_3-e_4,\frac{3}{2}e_3-\frac{1}{2}e_4]$ 这组基下变换为
  Jordan 标准型,即
  \[
    P=
    \begin{pmatrix}
      1 & 0 & 0 & 0 \\
      0 & 1 & 0 & 0 \\[1mm]
      -2 & -\frac{3}{2} & 1 & \frac{3}{2} \\[1mm]
      0 & \frac{1}{2} & -1 & -\frac{1}{2}
    \end{pmatrix}  ,\quad
    P^{-1}AP=
    \begin{pmatrix}
      1 & 0 & 0 & 0 \\
      0 & 1 & 0 & 0 \\
      0 & 0 & -1 & 1 \\
      0 & 0 & 0 & -1
    \end{pmatrix}.\qedhere
  \]
\end{solution}

\begin{thebibliography}{99}
  \bibitem{Dummit}  Dummit DS, Foote RM. Abstract Algebra. Hoboken: Wiley; 2004.
\end{thebibliography}




\end{document}