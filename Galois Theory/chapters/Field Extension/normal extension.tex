\section{正规扩张}

本节我们研究 \autoref{exa:nonGalois has repeat roots} 之前所说的
阻止 $F(a)/F$ 成为 Galois 扩张的第一种情况,即 $\min(F,a)$
不是所有的零点都落在 $F(a)$ 中。

\begin{lemma}
  令 $f(x)\in F[x]$,$\alpha\in F$,那么 $\alpha$ 是 $f$ 的零点当且仅当 $(x-\alpha)\mid f$。
  此外,在 $F$ 的任意扩域中,$f$ 至多也只能有 $\deg f$ 个零点。
\end{lemma}
\begin{proof}
  若 $f(\alpha)=0$,做带余除法,存在 $q(x),r(x)\in F[x]$ 使得
  $f(x)=q(x)(x-\alpha)+r(x)$,其中 $\deg r<\deg (x-\alpha)=1$,所以 $r(\alpha)=0$,
  这表明 $r(x)=0$,即 $(x-\alpha)\mid f(x)$。反之显然成立。

  对 $\deg f=n$ 归纳。当 $n=1$ 时,即 $f(x)=ax+b$,其中 $a,b\in F$。那么
  $f(x)=0$ 当且仅当 $x=-b/a\in F$,结论成立。假设结论在 $n-1$ 的时候成立。
  设 $K$ 是 $F$ 的任意扩域,如果 $f(x)$ 在 $K$ 中没有零点,结论自然成立。
  如果 $f(x)$ 在 $K$ 中有零点 $\alpha\in K$,根据前面的叙述,$f(x)$
  在 $K[x]$ 中分解为 $f(x)=(x-\alpha)g(x)$,其中 $g(x)\in K[x]$。
  由于 $\deg g=n-1$,根据假设,$g(x)$ 在 $K$ 的任意扩域中至多有 $n-1$
  个零点,特别地,$g(x)$ 在 $K$ 中至多有 $n-1$ 个零点,所以 $f(x)$
  在 $K$ 中至多有 $n$ 个零点,结论成立。
\end{proof}

\begin{definition}
  $K/F$ 是域扩张,$f(x)\in F[x]$,如果存在 $a\in F$ 和 $\alpha_1,\dots,\alpha_n\in K$
  使得
  \[
    f(x)=a\prod_{i=1}^n (x-\alpha_i)\in K[x],  
  \]
  那么我们说 $f$ 在 $K$ 上\emph{分裂}。
\end{definition}

实际上,在第一节的最开始,我们已经证明了:对于任意多项式 $f(x)\in F[x]$,
都可以找到 $F$ 的一个扩域使得其至少包含 $f$ 的一个零点。现在我们来严格叙述这一点并且加以推广。

\begin{theorem}\label{thm:exsistence of split field for one polynomial}
  令 $f(x)\in F[x]$ 是 $n$ 次多项式,那么存在 $F$ 的扩域 $K$,使得 $K$ 包含 $f$ 的一个零点,
  并且 $[K:F]\leq n$。进一步地,存在 $F$ 的扩域 $L$,使得 $f$ 在 $L$ 上分裂,并且
  $[L:F]\leq n!$。
\end{theorem}
\begin{proof}
  设 $p(x)$ 为 $f(x)$ 的不可约因子。那么 $K=F[x]/(p(x))$ 是域,通过将 $a\in F$
  视为 $\bar a=a+(p(x))\in K$,我们可以将 $F$ 视为 $K$ 的子域。在 $K$ 中,$\bar x=x+(p(x))$ 使得
  $p(\bar x)=p(x)+(p(x))=0$,所以 $K$ 包含 $p$ 的零点,从而包含 $f$ 的一个零点。
  并且有 $[K:F]=\deg p\leq \deg f=n$。

  对 $n=\deg f$ 归纳。在 $n=1$ 的时候 $L=F$ 即可。假设结论在 $n-1$ 时成立。根据前面的叙述,
  存在 $F$ 的扩域 $K$,使得 $K$ 包含 $f$ 的一个零点 $\alpha\in K$,那么
  $f(x)=(x-\alpha)g(x)$,其中 $g(x)\in K[x]$。由于 $\deg g=n-1$,根据假设,存在
  $K$ 的一个扩域 $L$,使得 $g$ 在 $L$ 上分裂,并且 $[L:K]\leq (n-1)!$,
  那么 $f$ 在 $L$ 上也分裂,并且 $[L:F]=[L:K][K:F]\leq (n-1)!\cdot n=n!$。
\end{proof}

\begin{definition}
  $K/F$ 是域扩张,$f(x)\in F[x]$。
  \begin{enumerate}
    \item 若 $f$ 在 $K$ 上分裂,设 $f$ 在 $K$ 中的所有根为 $\alpha_1,\dots,\alpha_n$,
    并且 $K=F(\alpha_1,\dots,\alpha_n)$,那么我们说 $K$ 是 $f$ 在 $F$ 上的\emph{分裂域}。
    $F$ 明确的情况下有时会简称为 $f$ 的分裂域。
    \item 若 $S$ 是 $F$ 上的一组多项式的集合,每个 $f\in S$ 在 $K$ 上都分裂,
    设 $X$ 是 $S$ 中所有多项式的零点集,并且 $K=F(X)$,那么我们说 $K$
    是 $S$ 在 $F$ 上的\emph{分裂域}。
  \end{enumerate}
\end{definition}

若 $K$ 是 $S$ 在 $F$ 上的分裂域,$L$ 是 $K$ 的子域且使得 $S$ 中的多项式在 $L$
上都分裂,任取 $\alpha\in K$,那么存在 $\alpha_1,\dots,\alpha_n\in X$ 使得
$\alpha\in F(\alpha_1,\dots,\alpha_n)$,其中 $\alpha_1,\dots,\alpha_n$
是一些多项式 $f_1,\dots,f_m\in S$ 的零点,由于 $f_1,\dots,f_m$ 在 $L$ 上分裂,
所以 $\alpha_1,\dots,\alpha_n\in L$,故 $\alpha\in L$,所以 $L=K$。
这表明 $K$ 实际上是使得 $S$ 中多项式都分裂的 $F$ 的最小扩域。

\autoref{thm:exsistence of split field for one polynomial} 实际上保证了
$S$ 是有限集的时候分裂域的存在性。

\begin{corollary}
  若 $f_1,\dots,f_n\in F[x]$,那么存在 $\{f_1,\dots,f_n\}$ 在 $F$ 上的分裂域。
\end{corollary}
\begin{proof}
  令 $f=f_1\cdots f_n$。显然 $\{f_1,\dots,f_n\}$ 在 $F$ 上的分裂域和 $f$
  在 $F$ 上的分裂域是相同的。根据 \autoref{thm:exsistence of split field for one polynomial},
  存在 $F$ 的扩域 $K$,使得 $f$ 在 $K$ 上分裂。令 $\alpha_1,\dots,\alpha_m$ 为 $f$
  在 $K$ 中的所有零点,那么 $F(\alpha_1,\dots,\alpha_m)$ 就是 $f$ 在 $F$ 上的分裂域。
\end{proof}

\begin{example}
  $\mathbb{Q}(\sqrt[3]{2},\omega)$ 是 $x^3-2$ 在 $\mathbb{Q}$ 上的分裂域,因为
  \[
    x^3-2=(x-\sqrt[3]{2})(x-\omega\sqrt[3]{2})(x-\omega^2\sqrt[3]{2})\in\mathbb{Q}(\sqrt[3]{2},\omega)[x],
  \]
  并且容易
  验证 $\mathbb{Q}(\sqrt[3]{2},\omega)=\mathbb{Q}(\sqrt[3]{2},\omega\sqrt[3]{2},\omega^2\sqrt[3]{2})$。
  $\mathbb{C}$ 是 $x^2+1$ 在 $\mathbb{R}$ 上的分裂域。
\end{example}

\begin{example}
  考虑 $K=\mathbb{F}_2[x]/(x^2+x+1)$,那么 $K$ 可以视为 $\mathbb{F}_2(\alpha)$,其中
  $\alpha$ 是 $x^2+x+1$ 的零点,因为
  \[
    x^2+x+1=(x-\alpha)(x-(\alpha+1))\in K[x],  
  \]
  所以 $K$ 是 $x^2+x+1$ 在 $\mathbb{F}_2$ 上的分裂域。
\end{example}

由 \autoref{thm:exsistence of split field for one polynomial} 立即得出下面的推论。

\begin{corollary}\label{coro:degree of split field}
  $F$ 是域,$f(x)\in F[x]$ 是 $n$ 次多项式,如果 $K$ 是 $f$ 在 $F$ 上的分裂域,那么
  $[K:F]\leq n!$。
\end{corollary}


\begin{example}
  我们继续研究 \autoref{exa:symmetric function}。记
  \[
    f(t)=t^n-s_1t^{n-1}+\cdots+(-1)^n s_n\in k(s_1,\dots,s_n)[t],  
  \]
  其在 $K[x]$ 中分裂为
  \[
    f(t)=(t-x_1)\cdots(t-x_n),  
  \]
  又因为 $K=k(x_1,\dots,x_n)$,所以 $K$ 是 $f$ 在 $k(s_1,\dots,s_n)$ 上的分裂域,
  根据 \autoref{coro:degree of split field},所以 $[K:k(s_1,\dots,s_n)]\leq n!$。
  对于 $F=\Fix(S_n)$,\autoref{prop:when Gal equal to K/F} 表明 $[K:F]=|S_n|=n!$。
  又因为 $k(s_1,\dots,s_n)\subseteq F$,所以只可能 $[K:k(s_1,\dots,s_n)]=[K:F]=n!$。
  这就证明了 $F=k(s_1,\dots,s_n)$。
\end{example}

\subsection{代数闭包}

我们还没有证明无限多个多项式集合在 $F$ 上的分裂域的存在性。我们首先研究最极端的情况,即 $F$
上所有非常数多项式在 $F$ 上的分裂域 $K$,一旦证明了 $K$ 存在,那么任意多项式集合的分裂域
都是 $K$ 的某个子域。我们先假设这样的 $K$ 存在,若 $L/K$ 是代数扩张,任取 $a\in L$,
那么 $a$ 在 $F$ 上也是代数的,而 $\min(F,a)$ 在 $K$ 上分裂,所以 $a\in L$,所以 $L=K$。
于是我们发现,这样的 $K$ 如果存在,那么其没有任意恰当的代数扩域。我们先给出这样的域
的一些等价条件。

\begin{lemma}\label{lemma:TFAE of algebraic closed field}
  $K$ 是域,那么下面的说法是等价的。
  \begin{enumerate}
    \item $K$ 的代数扩张只有 $K$ 本身。
    \item $K$ 的有限扩张只有 $K$ 本身。
    \item 如果 $L$ 是 $K$ 的扩域,那么 $K$ 是 $K$ 在 $L$ 中的代数闭包(\autoref{def:closure of extension})。 
    \item 任意 $f(x)\in K[x]$ 在 $K$ 上都分裂。
    \item 任意 $f(x)\in K[x]$ 在 $K$ 中都有一个根。
    \item $K$ 上的不可约多项式只有一次多项式。
  \end{enumerate}
\end{lemma}
\begin{proof}
  $(1)\Rightarrow (2)$ 有限扩张都是代数扩张。

  $(2)\Rightarrow (3)$ 记 $\bar K$ 为 $K$ 在 $L$ 中的代数闭包。任取 $a\in\bar K$,
  那么 $K(a)/K$ 是有限扩张,所以 $K(a)=K$,故 $a\in K$,所以 $K=\bar K$。

  $(3)\Rightarrow (4)$ 任取 $f(x)\in K[x]$,记 $L$ 为 $f$ 在 $K$ 上的分裂域,
  那么 $L/K$ 是代数扩张,所以 $K$ 在 $L$ 中的代数闭包就是 $L$,故 $L=K$,
  $f$ 在 $K$ 上分裂。

  $(4)\Rightarrow (5)$ 显然。

  $(5)\Rightarrow (6)$ 设 $f(x)\in K[x]$ 是不可约多项式,由于 $f$ 在 $K$
  中有零点 $\alpha$,所以 $(x-\alpha)\mid f$,$f$ 不可约表明 $f=a(x-\alpha)$,
  其中 $a\in K$,即 $f$ 是一次多项式。

  $(6)\Rightarrow (1)$ 设 $L/K$ 是代数扩张。任取 $a\in L$,$\min(K,a)$ 是一次多项式,
  这表明 $[K(a):K]=1$,所以 $a\in K$,$L=K$。
\end{proof}

\begin{definition}
  如果 $K$ 满足 \autoref{lemma:TFAE of algebraic closed field} 中的任意一条,
  那么我们说 $K$ 是\emph{代数闭域}。如果 $K/F$ 是代数扩张且 $K$ 是代数闭域,
  那么我们说 $K$ 是 $F$ 的\emph{代数闭包},通常记为 $\bar F$。
\end{definition}

\begin{example}\label{exa:ag closure of Q}
  复数域 $\mathbb{C}$ 是代数闭域,这源于\emph{代数基本定理},我们将在第五节提供一种代数
  证明。$\mathbb{Q}$ 在 $\mathbb{C}$ 中的代数闭包记为 $\mathbb{A}$,任取 
  $f(x)\in \mathbb{A}[x]$,由于 $\mathbb{C}$ 是代数闭域,所以 $f$ 在 $\mathbb{C}$
  中有根 $\alpha$,$\alpha$ 在 $\mathbb{A}$ 上是代数的,$\mathbb{A}/\mathbb{Q}$
  是代数扩张,所以 $\alpha$ 在 $\mathbb{Q}$ 上是代数的,所以 $\alpha\in\mathbb{A}$,
  这表明 $\mathbb{A}$ 是代数闭域。所以 $\mathbb{A}$ 是 $\mathbb{Q}$ 的代数闭包。
  类似地可以证明:对于域扩张 $K/F$,如果 $K$ 是代数闭域,此时
  $F$ 在 $K$ 中的代数闭包就是 $F$ 的代数闭包。此外,$\mathbb{C}$ 是 $\mathbb{R}$ 的代数闭包,
  但注意 $\mathbb{C}$ 不是 $\mathbb{Q}$
  的代数闭包,因为 $\mathbb{C}/\mathbb{Q}$ 不是代数扩张。
\end{example}

\begin{theorem}
  $F$ 是一个域,那么 $F$ 有一个代数闭包。
\end{theorem}
\begin{proof}
  对于每个首一的非常数多项式 $f\in F[x]$,我们记 $x_f$ 是一个未定元,
  记 $X=\{x_f\,|\, f\in F[x]\}$。考虑无穷多个未定元集合 $X$ 上的
  多项式环 $F[X]$。考虑所有 $f(x_f)$ 生成的理想 $I$。我们首先说明 $I$
  是 $F[X]$ 的一个恰当理想。如果 $1\in I$,那么存在 $f_1,\dots,f_n\in F[x]$
  和 $g_1,\dots,g_n\in F[X]$ 使得
  \[
    1=g_1f_1(x_{f_1})+\cdots+g_nf_n(x_{f_n}). 
  \]
  $g_i\in F[X]$ 表明存在 $f_{n+1},\dots,f_m\in F[x]$ 使得
  $g_i\in F[x_{f_1},\dots,x_{f_n},x_{f_{n+1}},\dots,x_{f_m}]$。
  于是 
  \begin{equation}\label{eq:ag closure eq 1}
    1=g_1(x_{f_1},\dots,x_{f_m})f_1(x_{f_1})+\cdots+  
    g_n(x_{f_1},\dots,x_{f_m})f_n(x_{f_n}).
  \end{equation}
  令 $K$ 是 $\{f_1(x),\dots,f_n(x)\}$ 在 $F$ 上的分裂域,$1\leq i\leq n$ 时,
  取 $x_{f_i}$ 为 $\alpha_i\in K$,其中 $\alpha_i$ 是 $f_i(x)$ 的一个根。
  令 $x_{f_{n+1}},\dots,x_{f_m}$ 均取零,那么 \eqref{eq:ag closure eq 1}
  告诉我们在 $K$ 中有 $0=1$,这是不可能的,所以 $I$ 是 $F[X]$ 的一个恰当理想。

  $I$ 是恰当理想表明其被一个极大理想 $J$ 包含,那么考虑域
  \[
    L_1=F[X]/J.  
  \]
  $F$ 可以自然地嵌入到 $L_1$ 中。根据 $I$ 的定义,每个多项式 $f\in F[x]$
  在 $L_1$ 中都有一个根,即 $x_f+J$。然后重复上面的所有操作,可以得到
  域 $L_2\supseteq L_1$ 使得 $L_1[x]$ 中的多项式在 $L_2$ 中都有一个根。
  于是我们可以得到一个域塔:
  \[
    F=L_0\subseteq L_1\subseteq L_2\subseteq \cdots,  
  \]
  其中每个 $L_i[x]$ 中的多项式在 $L_{i+1}$ 中都有一个根。令
  \[
    L=\bigcup_{i\geq 0}L_i.  
  \]
  显然 $L$ 是 $F$ 的一个扩域。任取多项式 $h(x)\in L[x]$,
  那么存在 $N$ 使得 $h(x)\in L_N[x]$,从而在 $L_{N+1}\subseteq L$
  中有一个根,根据 \autoref{lemma:TFAE of algebraic closed field}
  的 (5),$L$ 是代数闭域。令 $\bar F$ 为 $F$ 在 $L$ 中的代数闭包,
  根据 \autoref{exa:ag closure of Q},
  此时 $\bar F$ 就是 $F$ 的代数闭包。
\end{proof}

\begin{corollary}
  令 $S$ 是 $F$ 上某些非常数多项式的集合,那么存在 $S$ 在 $F$ 上的分裂域。
\end{corollary}
\begin{proof}
  令 $K$ 是 $F$ 的代数闭包。那么每个 $f(x)\in S$ 在 $K$ 上分裂,令 $X$
  是所有 $f\in S$ 的零点集。那么 $F(X)\subseteq K$ 就是 $S$ 在 $F$ 上的分裂域。
\end{proof}

\subsection{同构延拓定理}

当我们说某个多项式集合的分裂域的时候,自然就会存在另一个问题,有没有可能存在两个
不同构的分裂域?答案是不可能。现在我们来证明这一点。

如果 $\sigma:F\to F'$ 是域同态,那么自然诱导出环同态 $\sigma:F[x]\to F'[x]$,
我们仍记为 $\sigma$,其满足 $\sigma(\sum a_ix^i)=\sum\sigma(a_i)x^i$。
如果 $f(x)=(x-a_1)\cdots(x-a_n)\in F[x]$,那么 $\sigma(f(x))=(x-\sigma(a_1))\cdots(x-\sigma(a_n))$,
这一点能够帮助我们研究分裂域。

\begin{lemma}[同构延拓定理 1]\label{lemma:isomorphism extension 1}
  令 $\sigma:F\to F'$ 是域同构,$f(x)\in F[x]$ 是不可约多项式,如果 $\alpha$
  是 $f$ 在 $F$ 的某个扩域 $K$ 中的根,$\alpha'$ 是 $\sigma(f)$ 在 $F'$ 的某个
  扩域 $K'$ 中的根,那么存在同构 $\tau:F(\alpha)\to F'(\alpha')$ 满足
  $\tau(\alpha)=\alpha'$ 以及 $\tau|_F=\sigma$。
\end{lemma}
\begin{proof}
  当然,可以直接按照定义验证 $\tau$ 是域同构,但是我们采用更加体现本质的方法。
  记 $f'(x)=\sigma(f(x))$,我们有同构 $\varphi:F[x]/(f(x))\to F(\alpha)$ 以及同构
  $\psi:F'[x]/(f'(x))\to F'(\alpha')$,所以只需要证明 $F[x]/(f(x))$ 
  同构于 $F'[x]/(f'(x))$。我们已经有同构 $\sigma:F[x]\to F'[x]$,
  诱导出满同态 $F[x]\to F'[x]\to F'[x]/(f'(x))$,同态核显然为 $f(x)$,
  所以存在同构 $\nu:F[x]/(f(x))\to F'[x]/(f'(x))$,作用为
  $\nu(g(x)+(f(x)))=\sigma(g(x))+(f'(x))$。所以复合映射
  $\tau=\psi\circ \nu\circ\varphi^{-1}$ 是 $F(\alpha)$ 到 $F'(\alpha')$
  的同构映射。此时 $\alpha\mapsto x+(f(x))\mapsto x+(f'(x))\mapsto \alpha'$。
  若 $a\in F$,那么 $a\mapsto a+(f(x))\mapsto \sigma(a)+(f'(x))\mapsto \sigma(a)$。
\end{proof}

\begin{lemma}\label{lemma:tauK is split field}
  令 $\sigma:F\to F'$ 是域同构,$K$ 是 $F$ 的扩域,$K'$ 是 $F'$ 的扩域。
  假设 $K$ 是 $\{f_i\}$ 在 $F$ 上的分裂域并且 $\tau:K\to K'$ 是使得
  $\tau|_F=\sigma$ 的同态。记 $f_i'=\sigma(f_i)$,那么 $\tau(K)$
  是 $\{f_i'\}$ 在 $F'$ 上的分裂域。
\end{lemma}
\begin{proof}
  由于 $K$ 是 $\{f_i\}$ 在 $F$ 上的的分裂域,所以 $f_i$ 在 $K$ 上分裂,
  即 $f_i=a\prod_j (x-\alpha_j)$,其中 $a\in F,\alpha_j\in K$,那么
  $f_i'=\sigma(f_i)=\tau(f_i)=\tau(a)\prod_j(x-\tau(\alpha_j))$,所以
  $f_i'$ 在 $\tau(K)$ 上分裂。由于 $K=F(X)$,其中 $X$ 为 $\{f_i\}$
  的所有零点,所以 $\tau(K)=F(\tau(X))$,$\tau(X)$ 为 $\{f_i'\}$ 的所有零点, 
  于是 $\tau(K)$ 为 $\{f_i'\}$ 在 $F'$ 上的分裂域。
\end{proof}

\begin{theorem}[同构延拓定理 2]\label{thm:isomorphism extension 2}
  令 $\sigma:F\to F'$ 是域同构,$f(x)\in F[x]$,$\sigma(f)\in F'[x]$。
  记 $K$ 是 $f$ 在 $F$ 上的分裂域,$K'$ 是 $\sigma(f)$ 在 $F'$ 上的分裂域。
  任取 $\alpha\in K$,若 $\alpha'$ 是 $\sigma(\min(F,\alpha))$ 在 $K'$ 中的任意零点,
  那么存在同构 $\tau:K\to K'$ 使得 $\tau|_F=\sigma$ 以及 $\tau(\alpha)=\alpha'$。
\end{theorem}
\begin{proof}
  对 $[K:F]=n$ 归纳。$n=1$ 时 $\tau=\sigma$ 即满足要求。假设在 $[K:F]<n$
  的时候,都存在同构 $\tau:K\to K'$ 使得 $\tau|_F=\sigma$。
  对于 $n$ 的情况,任取 $\alpha\in K$,记 $p(x)=\min(F,\alpha)$。
  
  如果 $\deg p>1$,根据 \autoref{lemma:isomorphism extension 1},存在同构
  $\rho:F(\alpha)\to F'(\alpha')$ 使得 $\rho|_F=\sigma$ 以及 $\rho(\alpha)=\alpha'$。
  此时 $[K:F(\alpha)]<n$,并且 $K$ 为 $f$ 在 $F(\alpha)$ 上的分裂域,$K'$
  为 $\sigma(f)$ 在 $F'(\alpha')$ 上的分裂域,根据假设,存在同构 $\tau:K\to K'$
  使得 $\tau|_{F(\alpha)}=\rho$,此时 $\tau|_F=\rho|_F=\sigma$,以及
  $\tau(\alpha)=\rho(\alpha)=\alpha'$。

  如果 $\deg p=1$,即 $\alpha\in F$,这表明如果 $\tau|_F=\sigma$,
  那么必有 $\tau(\alpha)=\sigma(\alpha)=\alpha'$,所以只需要证明存在
  同构 $\tau:K\to K'$ 使得 $\tau|_F=\sigma$ 即可。此时任取
  $f(x)$ 的一个不可约因子 $q(x)$,然后重复上一段的操作,就可以得到同构
  $\tau:K\to K'$ 使得 $\tau|_F=\sigma$。
\end{proof}

上面的定理十分有用,证明了有限多个多项式的分裂域的唯一性,虽然可以直接证明一般的
情况,但是大多数情况下,我们使用的都是有限个多项式的情况。

\begin{theorem}[同构延拓定理 3]
  令 $\sigma:F\to F'$ 是域同构,令 $S=\{f_i\}$ 是 $F$ 上的一族多项式,
  $S'=\{\sigma(f_i)\}$ 是 $F'$ 上的一族多项式,$K$ 是 $S$ 在 $F$
  上的分裂域,$K'$ 是 $S'$ 在 $F'$ 上的分裂域。
  任取 $\alpha\in K$,若 $\alpha'$ 是 $\sigma(\min(F,\alpha))$ 在 $K'$ 中的任意零点,
  那么存在同构 $\tau:K\to K'$ 使得 $\tau|_F=\sigma$ 以及 $\tau(\alpha)=\alpha'$。
\end{theorem}

\begin{corollary}
  令 $F$ 是域,$S$ 是 $F[x]$ 中多项式的集合,那么任意两个 $S$ 在 $F$ 上的分裂域
  都是 $F$-同构的。特别地,任意两个 $F$ 的代数闭包是 $F$-同构的。
\end{corollary}

\begin{corollary}
  令 $F$ 是域,$K/F$ 是代数扩张,那么 $K$ 同构于 $\bar F$ 的一个子域。
\end{corollary}
\begin{proof}
  $\bar K$ 是代数闭域并且 $\bar K/F$ 是代数扩张,所以 $\bar K$ 也是 $F$
  的代数闭包,于是存在 $F$-同构 $f: \bar K\to \bar F$,所以 $f(K)$
  是 $\bar F$ 的子域,同构于 $K$。
\end{proof}

现在我们来说明,虽然分裂域的定义依赖于一族多项式,但是实际上存在不依赖于
多项式的刻画方式。

\begin{definition}
  $K/F$ 是域扩张,如果 $K$ 是 $F$ 上某一个多项式集合在 $F$ 上的分裂域,那么我们说
  $K/F$ 是\emph{正规扩张}。
\end{definition}

\begin{example}
  如果 $[K:F]=2$,那么 $K/F$ 是正规扩张。任取 $\alpha\in K\smallsetminus F$,那么
  $K=F(\alpha)$。令 $p(x)=\min(F,\alpha)$,那么 $p$ 在 $K$ 中有一个零点 $\alpha$,
  所以 $p(x)=(x-\alpha)q(x)\in K[x]$,所以 $q(x)$ 是一次多项式,所以 $p(x)$ 在 $K$
  上分裂。故 $K=F(\alpha)$ 是 $p(x)$ 在 $F$ 上的分裂域,$K/F$ 是正规扩张。
\end{example}

\begin{theorem}\label{thm:equivalent of normal extension}
  如果 $K/F$ 是代数扩张,那么下面的说法是等价的。
  \begin{enumerate}
    \item $K/F$ 是正规扩张。
    \item 如果 $\tau: K\to\bar K$ 是 $F$-同态,那么 $\tau(K)=K$。
    % \item 如果 $F\subseteq L\subseteq K\subseteq N$ 是域,$\sigma:L\to N$
    % 是 $F$-同态,那么 $\sigma(L)\subseteq K$,并且存在 $\tau\in\Gal(K/F)$
    % 使得 $\tau|_F=\sigma$。
    \item 对于任意不可约多项式 $f(x)\in F[x]$,如果 $f$ 在 $K$ 中有一个根,那么
    $f$ 在 $K$ 上分裂。
  \end{enumerate}
\end{theorem}
\begin{proof}
  $(1)\Rightarrow (2)$ 令 $\tau:K\to\bar K$ 是 $F$-同态,设 $K$ 是 $S$
  在 $F$ 上的分裂域,根据 \autoref{lemma:tauK is split field},
  所以 $\tau(K)$ 也是 $S$ 在 $F$ 上的分裂域,即 $\tau(K)=F(X)=K$,
  其中 $X$ 是 $S$ 的零点集。

  $(2)\Rightarrow (3)$ 若不可约多项式 $f(x)\in F[x]$ 在 $K$ 中有一个根 $\alpha$,
  由于 $\bar K$ 也是 $F$ 的代数闭包,所以 $f(x)$ 在 $\bar K$ 上分裂。
  令 $\beta\in\bar K$ 为 $f$ 的任意一个根,那么存在 $F$-同构 $\sigma:F(\alpha)\to F(\beta)$,
  满足 $\sigma(\alpha)=\beta$。再根据同构延拓定理 3,存在同构 $\tau:\bar K\to\bar K$
  满足 $\tau|_{F(\alpha)}=\sigma$,根据 (2),我们有 $\tau(K)=K$,故
  $\beta=\sigma(\alpha)=\tau(\alpha)\in \tau(K)=K$,所以 $f$ 在 $K$ 上分裂。

  $(3)\Rightarrow (1)$ 对于任意的 $\alpha\in K$,$\min(F,\alpha)$ 在 $K$ 中有根
  $\alpha$,从而在 $K$ 上分裂,记 $S=\{\min(F,\alpha)\,|\, \alpha\in K\}$,那么
  $K$ 为 $S$ 在 $F$ 上的分裂域,所以 $K/F$ 是正规扩张。
\end{proof}

\autoref{thm:equivalent of normal extension} 的 (3) 通常用于判定某些扩张不是正规扩张。

\begin{example}
  $F\subseteq L\subseteq K$ 是域,如果 $K/F$ 是正规扩张,那么 $K/L$ 是正规扩张。
  设 $K$ 是 $S$ 在 $F$ 上的分裂域。那么 $S$ 作为 $L[x]$ 中的多项式集合在 $K$ 上
  自然也是分裂的。且 $K=F(X)$ 自然推出 $K=L(X)$,所以 $K$ 也是 $S$
  在 $L$ 上的分裂域,故 $K/L$ 是正规扩张。但是 $L/F$ 不一定是正规扩张,例如
  $\mathbb{Q}\subseteq \mathbb{Q}(\sqrt[3]{2})\subseteq\mathbb{Q}(\sqrt[3]{2},\omega)$,
  此时 $\mathbb{Q}(\sqrt[3]{2},\omega)$ 是 $x^3-2$ 在 $\mathbb{Q}$ 上的分裂域。
  注意到 $\mathbb{Q}$ 上的不可约多项式 $x^3-2$ 在 $\mathbb{Q}(\sqrt[3]{2})$ 只有一个根
  $\sqrt[3]{2}$,所以 $\mathbb{Q}(\sqrt[3]{2})/\mathbb{Q}$ 不是正规扩张。

  反之,若 $K/L$ 和 $L/F$ 都是正规扩张,$K/F$ 也不一定是正规扩张。例如
  $\mathbb{Q}\subseteq\mathbb{Q}(\sqrt{2})\subseteq\mathbb{Q}(\sqrt[4]{2})$,
  $\mathbb{Q}(\sqrt{2})$ 是 $x^2-2$ 在 $\mathbb{Q}$ 上的分裂域,$\mathbb{Q}(\sqrt[4]{2})$
  是 $x^2-\sqrt{2}$ 在 $\mathbb{Q}(\sqrt{2})$ 上的分裂域。但是
  $\mathbb{Q}$ 上的不可约多项式 $x^4-2$ 在 $\mathbb{Q}(\sqrt[4]{2})$ 中
  只有两个根,所以 $\mathbb{Q}(\sqrt[4]{2})/\mathbb{Q}$ 不是正规扩张。

  总结起来,这表明对于 $F\subseteq L\subseteq K$:
  \begin{itemize}
    \item $K/F$ 是正规扩张 $\Rightarrow$ $K/L$ 是正规扩张。
    \item $K/F$ 是正规扩张 $\not\Rightarrow$ $L/F$ 是正规扩张。
    \item $K/L,L/F$ 是正规扩张 $\not\Rightarrow$ $K/F$ 是正规扩张。
  \end{itemize}
\end{example}
