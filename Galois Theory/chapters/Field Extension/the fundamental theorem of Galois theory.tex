\section{Galois 理论基本定理}

\begin{theorem}[Galois 理论基本定理]
  令 $K/F$ 是有限 Galois 扩张,$G=\Gal(K/F)$,那么在 $K/F$ 的中间域
  和 $G$ 的子群之间存在一一对应,由 $L\mapsto \Gal(K/L)$ 和 $H\mapsto\Fix(H)$
  给出。如果 $L\leftrightarrow H$,那么 $[K:L]=|H|$ 并且 $[L:F]=[G:H]$。
  此外,$H$ 是 $G$ 的正规子群当且仅当 $L/F$ 是 Galois 扩张,此时
  $\Gal(L/F)\simeq G/H$。
\end{theorem}

\begin{center}
  \begin{tikzcd}
    K\arrow[r,leftrightarrow] & \Gal(K/K) \mathmakebox[0pt][l]{{}=0} \\
    L\arrow[u,symbol=\subseteq]\arrow[r,leftrightarrow] & \Gal(K/L)\arrow[u,symbol=>]\mathmakebox[0pt][l]{{}=H} \\
    F\arrow[u,symbol=\subseteq]\arrow[r,leftrightarrow] & \Gal(K/F)\arrow[u,symbol=>]\mathmakebox[0pt][l]{{}=G}
  \end{tikzcd}
\end{center}


\begin{theorem}[本原元定理]\label{thm:primitive element}
  有限扩张 $K/F$ 是单扩张当且仅当只存在有限多个中间域 $L$ 使得 $F\subseteq L\subseteq K$。
\end{theorem}
\begin{proof}
  我们仅对无限域的情况做证明,有限域的情况留到下一节中。
  由于 $[K:F]<\infty$,所以设 $K=F(\alpha_1,\dots,\alpha_n)$。对 $n$ 归纳,
  当 $n=1$ 时结论显然成立。假设 $n-1$ 时成立,对于 $n>1$ 的时候,根据假设,
  有 $\beta\in K$ 使得 $F(\beta)=F(\alpha_1,\dots,\alpha_{n-1})$,
  所以 $K=F(\alpha_n,\beta)$。任取 $a\in F$,考虑 $M_a=F(\alpha_n+a\beta)$,
  显然 $F\subseteq M_a\subseteq K$。
  由于 $F$ 有无限个元素,所以一定存在 $a\neq b$ 使得 $M_a=M_b$。注意到
  \[
    \beta=\frac{(\alpha_n+a\beta)-(\alpha_n+b\beta)}{a-b}\in M_a,
  \]
  所以 $\alpha_n=(\alpha_n+a\beta)-a\beta\in M_a$,所以 $K=M_a$,
  即 $K$ 是 $F$ 的单扩张。

  反之,假设存在某个 $\alpha\in K$ 使得 $K=F(\alpha)$。设 $M$ 使得 $F\subseteq M\subseteq K$,
  那么 $K=M(\alpha)$。记 $q(x)=\min(M,\alpha)\in M[x]$,设 $q(x)=a_0+a_1x+\cdots+x^r$,
  考虑 $M_0=F(a_0,\dots,a_{r-1})$,那么 $M_0\subseteq M$ 且 $q(x)\in M_0[x]$。
  因为 $\min(M_0,\alpha)\mid q(x)$,所以 
  \begin{equation*}
    [K:M]=q\geq \deg (\min(M_0,\alpha))=[K:M_0]=[K:M][M:M_0],
  \end{equation*}
  所以 $M=M_0$,这表明 $M$ 完全由多项式 $q$ 确定。记 $p(x)=\min(F,\alpha)$,
  那么 $q\mid p$,所以这样的 $q$ 只有有限多个,即 $K/F$ 只有有限多个中间域。
\end{proof}

\begin{corollary}
  若 $K/F$ 是有限可分扩张,那么 $K/F$ 是单扩张。
\end{corollary}
\begin{proof}
  设 $K=F(\alpha_1,\dots,\alpha_n)$,令 $N$ 为 $\{\min(F,\alpha_i)\}$
  在 $F$ 上的分裂域,根据 \autoref{thm:normal separable extension},
  $N/F$ 是 Galois 扩张。此时 $F\subseteq K\subseteq N$。根据基本定理,
  $N/F$ 的中间域和有限群 $\Gal(N/F)$ 的子群一一对应,所以 $N/F$ 只有有限多个
  中间域,即 $K/F$ 只有有限多个中间域,所以 $K/F$ 是单扩张。
\end{proof}

\begin{corollary}
  如果 $K/F$ 是有限扩张且 $\cha(F)=0$,那么 $K/F$ 是单扩张。
\end{corollary}
\begin{proof}
  特征零的域都是完全域。
\end{proof}







