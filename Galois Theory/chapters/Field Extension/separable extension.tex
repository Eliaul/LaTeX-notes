\section{可分与不可分扩张}

本节我们研究 \autoref{exa:nonGalois has repeat roots} 之前所说的
阻止 $F(a)/F$ 成为 Galois 扩张的第二种情况,即
$\min(F,a)$ 零点在 $F(a)$ 中有重根。

令 $f(x)\in F[x]$,$f$ 的根 $\alpha$ 如果满足 $(x-\alpha)^m\mid f(x)$
但是 $(x-\alpha)^{m+1}\nmid f(x)$,那么我们说 $\alpha$ 是 $f$ 的 $m$ 重根。
若 $m >1$,那么我们说 $\alpha$ 是 $f$ 的重根。注意,对于 $f(x)\in F[x]$,如果
$f$ 在 $F$ 中有 $m$ 重根 $\alpha$,那么其在 $F$ 的任意扩域 $K$ 中仍然有 $m$ 重根 $\alpha$。
反之,如果 $f$ 在扩域 $K$ 中有 $m$ 重根 $\alpha$,并且 $\alpha\in F$,那么
$f$ 在 $F$ 中也有 $m$ 重根 $\alpha$。这是由域上多项式环的唯一因子分解性质决定的。

\begin{definition}
  令 $F$ 是域,不可约多项式 $f(x)\in F[x]$ 如果在分裂域中没有重根,那么
  我们说 $f$ 在 $F$ 上是\emph{可分的}。如果多项式 $g(x)\in F[x]$ 的所有不可约因子
  在 $F$ 上都是可分的,那么我们说 $g$ 在 $F$ 上是\emph{可分的}。
\end{definition}

根据上面的叙述,不难验证上述定义等价于 $f$ 在 $\bar F$ 中无重根。

\begin{example}\label{exa:separable or not polynomial}
  $x^2-2,(x-1)^9$ 在 $\mathbb{Q}$ 上是可分的。$x^2+x+1$ 在 $\mathbb{F}_2$ 上是可分的。
  下面是一个不可分多项式的例子。令 $k$ 是特征 $p$ 的域,考虑 $k(t^p)[x]$ 中的多项式
  $x^p-t^p$。$x^p-t^p$ 的分裂域为 $k(t)$,因为在 $k(t)[x]$ 中有
  $x^p-t^p=(x-t)^p$。如果 $f(x)\in k(t^p)[x]$ 是 $x^p-t^p$ 的因子,
  那么在 $k(t)[x]$ 中有 $f\mid (x-t)^p$,所以 $f(x)=(x-t)^k\in k(t)[x]$,
  但是 $1<k<p$ 时 $f(x)\notin k(t^p)[x]$,所以只可能 $f(x)=1$ 或者
  $f(x)=x^p-t^p$。这表明 $x^p-t^p$ 是 $k(t^p)$ 上的不可约多项式,
  但是其在分裂域 $k(t)$ 中有重根 $t$,所以这是不可分多项式。

  更一般地,若 $\cha{F}=p$ 且 $a\in F\setminus F^p$,那么 $x^p-a$ 不可约
  并且不可分,因为它在任意 $F$ 的扩域中至多只有一个根:若 $r^p-a=s^p-a$,
  那么 $(r-s)^p=r^p-s^p=0$,所以 $r=s$。
\end{example}

\begin{lemma}
  令 $f(x),g(x)\in F[x]$,那么
  \begin{enumerate}
    \item 如果 $f$ 在分裂域中没有重根,那么 $f$ 在 $F$ 上可分。
    \item 如果 $g\mid f$ 并且 $f$ 在 $F$ 上可分,那么 $g$ 在 $F$ 上可分。
    \item 如果 $f_1,\dots,f_n$ 在 $F$ 上可分,那么 $f_1\cdots f_n$ 在 $F$ 上可分。
    \item 如果 $f$ 在 $F$ 上可分,那么 $f$ 在 $F$ 的任意扩域上可分。
    \item 如果 $f$ 在 $F$ 的某个扩域上可分,那么 $f$ 在 $F$ 上可分。
  \end{enumerate}
\end{lemma}
\begin{proof}
  (1) 对于 $f$ 的不可约因子 $p$,$p$ 的分裂域被 $f$ 的分裂域包含,所以 $p$
  在分裂域中没有重根,即 $p$ 在 $F$ 上可分,所以 $f$ 在 $F$ 上可分。

  (2) $g$ 的不可约因子也是 $f$ 的不可约因子,所以 $g$ 在 $F$ 上可分。

  (3) $f_1\cdots f_n$ 的不可约因子必为 $f_1,\dots,f_n$ 中某个多项式的不可约因子,
  所以 $f_1\cdots f_n$ 在 $F$ 上可分。

  (4) 令 $K$ 是 $F$ 的扩域。设 $p$ 是 $f\in K[x]$ 的不可约因子,令 $\alpha$
  是 $p(x)$ 在 $\bar K$ 中的根,那么在 $K[x]$ 中有 $p(x)\mid \min(F,\alpha)$,
  而 $\min(F,\alpha)$ 在 $F$ 上可分,所以 $\min(F,\alpha)$ 在 $\bar F$
  中没有重根,那么 $\min(F,\alpha)$ 在 $\bar K$ 中也没有重根,所以 $p(x)$
  在 $\bar K$ 中没有重根,故 $p$ 在 $K$ 上可分,所以 $f$ 在 $K$ 上可分。

  (5) 设 $K$ 是 $F$ 的扩域,$f$ 在 $K$ 上可分。设 $p$ 是 $f$ 的不可约因子。
  那么 $p$ 在 $F$ 上的分裂域 $L_1$
  被 $p$ 在 $K$ 上的分裂域 $L_2$ 包含,所以 $p$ 在 $L_1$ 中无重根,
  即 $p$ 在 $F$ 上可分,所以 $f$ 在 $F$ 上可分。
\end{proof}

为了有效地判断多项式的可分性,我们需要引入形式导数。

\begin{definition}
  对于 $f(x)=a_0+a_1x+\cdots+a_nx^n\in F[x]$,$f(x)$ 的\emph{形式导数}定义为
  $f'(x)=a_1+2a_2x+\cdots+na_nx^{n-1}$。
\end{definition}

在特征零的域中,上述定义和微积分中利用极限求得的导数没有区别。但是在特征 $p$
的域中,我们需要注意:$x^p$ 的形式导数为 $px^{p-1}=0$。

形式导数满足与微积分中导数同样的性质:如果 $f(x),g(x)\in F[x]$,那么
\begin{enumerate}
  \item $(af(x)+bg(x))'=af'(x)+bg'(x)$;
  \item $(f(x)g(x))'=f'(x)g(x)+f(x)g'(x)$;
  \item $(f(g(x)))'=f'(g(x))g'(x)$。
\end{enumerate}
利用形式导数,我们可以非常方便地判断多项式是否有重根。若 $f(x)\in F[x]$
在 $f$ 的分裂域 $K$ 中有根 $\alpha$,那么在 $K[x]$ 中可设 $f(x)=(x-\alpha)^mg(x)$,
其中 $m\geq 1$ 以及 $(x-\alpha)\nmid g(x)$。此时 $f'(x)=m(x-\alpha)^{m-1}g(x)+(x-\alpha)^mg'(x)$,
不难注意到 $\alpha$ 是重根当且仅当 $m\geq 2$ 当且仅当 $(x-\alpha)\mid f'(x)$
当且仅当 $f,f'$ 在 $K$ 中有公共根。

\begin{lemma}\label{lemma:gcd is invariant}
  令 $f(x),g(x)\in F[x]$,$K$ 是 $F$ 的任意扩域。
  设 $F[x]$ 中 $\gcd(f,g)=d$, $K[x]$ 中 $\gcd(f,g)=d'$,那么实际上有
  $d'=d\in F[x]$。
\end{lemma}
\begin{proof}
  由于 $d\in F[x]\subseteq K[x]$
  是 $f,g$ 的公因子,所以 $d\mid d'$。另一方面,
  $F[x]$ 中 $\gcd(f,g)=d$ 表明存在 $h_1,h_2\in F[x]$
  使得 $fh_1+gh_2=d$,所以在 $K[x]$ 中有 $d'\mid (fh_1+gh_2)=d$,
  故 $d'=d$。
\end{proof}

\begin{proposition}\label{prop:separable polynomial test}
  令 $f(x)\in F[x]$ 是非常数多项式,那么 $f$ 在分裂域中无重根当且仅当
  $\gcd(f,f')=1$。
\end{proposition}
\begin{proof}
  若 $f$ 无重根,令 $K$ 为 $\{f,f'\}$ 在 $F$ 上的分裂域,那么 $f$
  在 $K$ 上可分,所以 $f,f'$ 在 $K$ 中没有公共根。令 $d$
  是 $f,f'$ 在 $K[x]$ 中的最大公因子。由于 $f,f'$ 在 $K$ 上分裂,所以
  $d$ 也在 $K$ 上分裂。若 $\deg d\geq 1$,那么 $d$ 的任意根都是
  $f,f'$ 的公共根,与上面矛盾。所以 $d=1$。根据 \autoref{lemma:gcd is invariant},
  在 $F[x]$ 中也有 $\gcd(f,f')=1$。

  若 $F[x]$ 中有 $\gcd(f,f')=1$,同样令 $K$ 为 $\{f,f'\}$ 在 $F$ 上的分裂域。
  根据 \autoref{lemma:gcd is invariant},在 $K[x]$ 中有
  $\gcd(f,f')=1$,这表明 $f,f'$ 在 $K$ 中没有公共根,从而在
  $f$ 的分裂域中也没有公共根,所以 $f$ 无重根。
\end{proof}

\begin{proposition}\label{prop:separable of irreducible polynomial}
  $f(x)\in F[x]$ 是不可约多项式。
  \begin{enumerate}
    \item 如果 $\cha(F)=0$,那么 $f$ 在 $F$ 上可分。如果 $\cha(F)=p>0$,
    那么 $f$ 在 $F$ 上可分当且仅当 $f'(x)\neq 0$,当且仅当
    $f(x)\notin F[x^p]$。
    \item 如果 $\cha(F)=p$,那么存在 $m\geq 0$ 和不可约的可分多项式 $g(x)\in F[x]$,
    使得 $f(x)=g(x^{p^m})$。
  \end{enumerate}
\end{proposition}
\begin{proof}
  (1) 因为 $f$ 不可约,所以 $\gcd(f,f')$ 作为 $f$ 的因子只可能等于 $1$ 或者
  $f$ 。如果 $\cha(F)=0$,那么 $\deg f'=\deg f-1$,所以 $f\nmid f'$,所以
  $\gcd(f,f')=1$,根据 \autoref{prop:separable polynomial test},
  $f$ 可分。如果 $\cha(F)=p$,那么 $f$ 不可分当且仅当 $\gcd(f,f')= f$,
  当且仅当 $f\mid f'$,当且仅当 $f'(x)=0$,当且仅当 $f(x)\in F[x^p]$。

  (2) 对于 $r\geq 0$,注意到 $F[x^{p^r}]$ 中非常数多项式至少是 $p^r$ 次的,
  所以只可能存在有限多个 $r_1,\dots,r_k$ 使得 $f(x)\in F[x^{p^{r_i}}]$,由于
  $f(x)\in F[x]$,所以 $k\geq 1$。令 $m=\max\{r_1,\dots,r_k\}$。
  设 $f(x)=g(x^{p^m})$,那么 $g(x)\notin F[x^p]$,否则与 $m$ 的最大性矛盾。
  由 (1),$g$ 在 $F$ 上可分。
\end{proof}

\begin{definition}
  $K/F$ 是域扩张,$\alpha\in K$。如果 $\min(F,\alpha)$ 在 $F$ 上可分,
  那么我们说 $\alpha$ 在 $F$ 上可分。如果每个 $\alpha\in K$ 都在 $F$
  上可分,那么我们说 $K/F$ 是\emph{可分扩张}。
\end{definition}

\begin{example}
  如果 $\cha(F)=0$,根据 \autoref{prop:separable of irreducible polynomial},
  $F$ 的任意代数扩张都是可分扩张。如果 $\cha(F)=p$,根据 \autoref{exa:separable or not polynomial},
  对于域 $k$,扩张 $k(t)/k(t^p)$ 不是可分扩张,因为元素 $t$ 在 $k(t^p)$
  上不可分。
\end{example}

现在我们可以说明 Galois 的代数扩张等价于正规可分扩张,这是我们判断
一个代数扩张是否为 Galois 扩张的一个最普遍的方法。

\begin{theorem}\label{thm:normal separable extension}
  $K/F$ 是代数扩张,那么下面的说法是等价的。
  \begin{enumerate}
    \item $K/F$ 是 Galois 扩张。
    \item $K/F$ 是正规可分扩张。
    \item $K$ 是 $F$ 上某一族可分多项式集合的分裂域。
  \end{enumerate}
\end{theorem}
\begin{proof}
  $(1)\Rightarrow (2)$ 若 $K/F$ 是 Galois 扩张。任取 $\alpha\in K$,
  设 $\min(F,\alpha)$ 在 $K$ 中的所有不同的根为 $\alpha_1=\alpha,\dots,\alpha_n$。
  令 $f(x)=\prod(x-\alpha_i)=x^n+\beta_{n-1}x^{n-1}+\cdots+\beta_0\in K[x]$。
  任取 $\sigma\in\Gal(K/F)$,由于 $\sigma$ 是 $\alpha_1,\dots,\alpha_n$ 的一个置换,
  所以 $\sigma(f)=f$。根据 $\sigma$ 的任意性,有 $\beta_i\in\Fix(\Gal(K/F))=F$,
  所以 $f(x)\in F[x]$。因为 $f(\alpha)=0$,所以 $\min(F,\alpha)\mid f(x)$。
  另一方面,在 $K[x]$ 中有 $f(x)\mid \min(F,\alpha)$,所以 $f(x)=\min(F,\alpha)$。
  这表明 $\min(F,\alpha)$ 无重根且在 $K$ 上分裂,所以 $K/F$ 是可分扩张,
  同时 $K$ 是 $\{\min(F,\alpha)\,|\,\alpha\in K\}$ 在 $F$ 上的分裂域。
  故 $K/F$ 为正规可分扩张。

  $(2)\Rightarrow (3)$ 若 $K/F$ 为正规可分扩张。那么 $K$ 是可分多项式
  集合 $\{\min(F,\alpha)\,|\,\alpha\in K\}$ 在 $F$ 上的分裂域。

  $(3)\Rightarrow (1)$ 首先假设 $[K:F]<\infty$。对 $n=[K:F]$ 归纳。若 $n=1$,
  那么 $K=F$,此时 $K/F$ 当然是 Galois 扩张。现在假设 $n>1$ 且结论对
  所有小于 $n$ 的扩张成立。若 $K$ 是可分多项式集合 $\{f_i(x)\}$ 的分裂域
  且 $[K:F]=n$。$n>1$ 表明存在某个 $f_i$ 的根 $\alpha$ 不在 $F$ 中,令 
  $L=F(\alpha)$,那么 $[L:F]>1$,所以 $[K:L]<n$。此时 $K$ 是 $\{f_i\}$
  在 $L$ 上的分裂域,根据归纳假设,$K/L$ 是 Galois 扩张。令 $H=\Gal(K/L)$
  是 $\Gal(K/F)$ 的子群,$\alpha_1,\dots,\alpha_r\in L$ 是 $\min(F,\alpha)$
  的不同的根。由于 $\alpha$ 在 $F$ 上可分,所以 $[L:F]=r$。根据同构延拓定理,
  存在 $\tau_i\in\Gal(K/F)$ 使得 $\tau_i(\alpha)=\alpha_i$。注意到陪集
  $\tau_iH$ 互不相同,因为若 $\tau_i^{-1}\tau_j\in H$,那么 $\tau_i^{-1}\tau_j(\alpha)=\alpha$,
  即 $\alpha_i=\tau_i(\alpha)=\tau_j(\alpha)=\alpha_j$。令 $G=\Gal(K/F)$,我们有
  \[
    |G|=[G:H]|H|\geq r|H|=[L:F][K:L]=[K:F],
  \]
  又因为总是有 $|G|\leq [K:F]$,所以 $|G|=[K:F]$,即 $K/F$ 是 Galois 扩张。

  现在假设 $K/F$ 是任意代数扩张。
\end{proof}

\begin{corollary}\label{coro:generators are separable}
  令 $L/F$ 是有限扩张。
  \begin{enumerate}
    \item $L/F$ 可分当且仅当 $L$ 被 $F$ 的某个 Galois 扩张包含。
    \item 如果 $L=F(\alpha_1,\dots,\alpha_n)$,其中 $\alpha_i$ 在 $F$ 上可分,
    那么 $L/F$ 可分。
  \end{enumerate}
\end{corollary}
\begin{proof}
  (1) 若 $L\subseteq K$ 且 $K/F$ 是 Galois 扩张,那么 $K/F$ 可分,所以 $L/F$ 可分。
  反之,若 $L/F$ 可分,因为 $[L:F]<\infty$,所以可以假设 $L=F(\alpha_1,\dots,\alpha_n)$,
  其中每个 $\alpha_i$ 都在 $F$ 上可分。令 $K$ 是 $\{\min(F,\alpha_i)\}$ 在 $F$ 上的分裂域,
  那么 $K/F$ 正规可分,所以是 Galois 扩张,所以 $L$ 被 Galois 扩张 $K$ 包含。

  (2) 和 (1) 的证明一致。
\end{proof}

那些所有代数扩张都可分的域是性质良好的,我们现在来确定哪些域有这样的属性。

\begin{definition}
  对于域 $F$,如果 $F$ 的每个代数扩张都是可分扩张,那么说 $F$ 是完全域。
\end{definition}

\begin{example}
  根据 \autoref{prop:separable of irreducible polynomial},特征零的域都是完全域。
  任意代数闭域都是完全域,因为它的代数扩张只有本身。
\end{example}

下面的定理刻画了素特征的完全域。我们在 \autoref{exa:separable or not polynomial} 中已经发现,如果 $a\in F\setminus F^p$,
那么 $x^p-a$ 是不可分的不可约多项式。因此,若 $F$ 是完全域,那么必须有 $F^p=F$。
我们现在证明这其实是一个充分必要条件。

\begin{theorem}
  令 $F$ 是特征 $p$ 的域,那么 $F$ 是完全域当且仅当 $F^p=F$。
\end{theorem}
\begin{proof}
  假设 $F$ 是完全域,若 $F^p\neq F$,取 $a\in F^p\setminus F$,令 $K$
  是 $x^p-a$ 在 $F$ 上的分裂域,那么 $a^{1/p}\in K$ 不可分,因为 $\min(F,a^{1/p})=x^p-a$
  有重根,这与 $F$ 是完全域矛盾,所以 $F^p=F$。

  反之,若 $F^p=F$。令 $K/F$ 是代数扩张,任取 $\alpha\in K$,$p(x)=\min(F,\alpha)$。
  根据 \autoref{prop:separable of irreducible polynomial},存在 $g(x)\in F[x]$
  使得 $p(x)=g(x^{p^m})$,其中 $g$ 不可约且可分。如果 $g(x)=a_0+a_1x+\cdots+x^r$,
  由于 $a_i\in F=F^p$,所以存在 $b_i\in F$ 使得 $a_i=b_i^p$,所以
  \[
    p(x)=b_0^p+b_1^px^{p^m}+\cdots+x^{rp^{m}}=\bigl(b_0+b_1x^{p^{m-1}}+\cdots+x^{rp^{m-1}}\bigr)^p,
  \]
  而 $p$ 不可约,所以 $m=0$,即 $p=g$ 在 $F$ 上可分。这就表明 $K/F$ 是可分扩张。
\end{proof}

\begin{example}
  任意有限域都是完全域。设 $F$ 是有限域,$\varphi:F\to F$ 为 $\varphi(a)=a^p$,
  那么 $\varphi$ 是单射,$F$ 是有限集表明 $\varphi$ 是满射,所以 $F=F^p$,
  即 $F$ 是完全域。
\end{example}

\subsection{纯不可分扩张}

现在我们讨论与可分性完全相反的概念。这种情况只与素特征相关,因为特征 $0$ 的任意代数扩张都是可分扩张。
如果 $F$ 是特征 $p>0$ 的域,$a\in F$,那么 $x^p-a$ 在任意分裂域中只有一个根。这是因为
如果 $\alpha$ 是一个根,那么 $\alpha^p=a$,即 $x^p-a=(x-\alpha)^p$。

\begin{definition}
  令 $K/F$ 是代数扩张,如果 $\alpha\in K$ 使得 $\min (F,\alpha)$ 只有一个零点,那么
  我们说 $\alpha$ 在 $F$ 上\emph{纯不可分}。如果 $K$ 的每个元素在 $F$ 上都纯不可分,
  那么我们说 $K/F$ 是纯不可分扩张。
\end{definition}

\begin{lemma}\label{lemma:polynomial of inseparable element}
  令 $F$ 是特征 $p$ 的域。如果 $\alpha$ 在 $F$ 上是代数的,那么 $\alpha$ 纯不可分当且仅当
  存在某个 $n$ 使得 $\alpha^{p^n}\in F$。此时存在 $m$ 使得 $\min(F,\alpha)=(x-\alpha)^{p^m}$。
\end{lemma}
\begin{proof}
  根据 \autoref{prop:separable of irreducible polynomial},存在不可约的可分多项式
  $g(x)\in F[x]$ 使得 $\min(F,\alpha)=g(x^{p^n})$。若 $g(x)$ 在分裂域中分裂为 
  $g(x)=(x-a_1)\cdots(x-a_r)$,其中 $a_1,\dots,a_r$ 互不相同。
  于是 $\min(F,\alpha)$ 的零点为 $x^{p^n}=a_i$ 的所有解。
  $\alpha$ 纯不可分当且仅当 $\min(F,\alpha)$ 只有 $\alpha$ 一个零点,
  所以 $r=1$ 并且 $\alpha^{p^n}=a_1$,此时 $\min(F,\alpha)=x^{p^n}-a_1\in F[x]$,
  故 $\alpha^{p^n}=a_1\in F$。

  反之,若 $\alpha^{p^n}\in F$,那么 $\alpha$ 是多项式 $x^{p^n}-\alpha^{p^n}\in F[x]$
  的零点,所以 $\min (F,\alpha)\mid (x-\alpha)^{p^n}$,所以 $\min(F,\alpha)$ 只有 $\alpha$
  一个零点,即 $\alpha$ 纯不可分。
\end{proof}

\begin{lemma}\label{lemma:property of purely inseparable}
  令 $K/F$ 是代数扩张。
  \begin{enumerate}
    \item 如果 $\alpha\in K$ 在 $F$ 上同时可分以及纯不可分,那么 $\alpha\in F$。
    \item 如果 $K/F$ 纯不可分,那么 $K/F$ 是正规扩张并且 $\Gal(K/F)=\{\id\}$。
    此外,如果 $[K:F]<\infty$,$\cha(F)=p$,那么 $[K:F]=p^n$。
    \item 如果 $K=F(X)$,其中每个 $\alpha\in X$ 都在 $F$ 上纯不可分,那么 $K/F$
    纯不可分。
    \item 如果 $F\subseteq L\subseteq K$ 是域,那么 $K/F$ 纯不可分当且仅当 $K/L$ 和
    $L/F$ 都纯不可分。
  \end{enumerate}
\end{lemma}
\begin{proof}
  (1) $\alpha$ 同时可分和纯不可分表明 $\min(F,\alpha)$ 无重根且只有 $\alpha$ 一个零点,所以
  $\min(F,\alpha)=x-\alpha$,即 $\alpha\in F$。

  (2) 任取 $\alpha\in K$,根据 \autoref{lemma:polynomial of inseparable element},
  $\alpha$ 纯不可分表明 $\min(F,\alpha)=(x-\alpha)^{p^m}$,所以 $\min(F,\alpha)$
  在 $K$ 上分裂,故 $K/F$ 是正规扩张。任取 $\sigma\in\Gal(K/F)$,那么对于任意 $\alpha\in K$
  $\sigma(\alpha)$ 是 $\min(F,\alpha)$ 的根,所以 $\sigma(\alpha)=\alpha$,即 $\sigma=\id$,
  即 $\Gal(K/F)=\{\id\}$。若 $[K:F]<\infty$,设 $K=F(\alpha_1,\dots,\alpha_n)$。
  由于 $[F(\alpha_1,\dots,\alpha_i):F(\alpha_1,\dots,\alpha_{i-1})]$
  都是 $p$ 的幂次,所以 $[K:F]$ 是 $p$ 的幂次。

  (3) 任取 $a\in K$,那么存在 $\alpha_1,\dots,\alpha_n\in X$ 使得 $a\in F(\alpha_1,\dots,\alpha_n)$。
  对于每个 $\alpha_i$,存在 $m_i$ 使得 $\alpha_i^{p^{m_i}}\in F$,那么 $a$ 作为 $\alpha_1,\dots,\alpha_n$
  的多项式,自然有 $a^{p^m}\in F$,所以 $a$ 在 $F$ 上纯不可分。

  (4) 若 $K/F$ 纯不可分。任取 $\alpha\in K$,那么 $\alpha^{p^n}\in F\subseteq L$,
  所以 $\alpha$ 在 $L$ 上纯不可分。$L/F$ 纯不可分也是显然的。反之,若 $K/L,L/F$ 纯不可分。
  任取 $\alpha\in K$,那么 $\alpha^{p^n}\in L$,进而 $(\alpha^{p^n})^{p^m}\in F$,
  即 $\alpha^{p^{m+n}}\in F$,所以 $\alpha$ 在 $F$ 上纯不可分。
\end{proof}

\begin{example}
  域扩张可能是既不可分又不是纯不可分的。例如,若 $F=\mathbb{F}_2(x)$,$K=F(x^{1/6})$。
  那么 $K=\mathbb{F}_2(x^{1/6})=\mathbb{F}_2(\sqrt{x},\sqrt[3]{x})$。
  此时 $\sqrt{x}$ 在 $F$ 上纯不可分,$\sqrt[3]{x}$ 在 $F$ 上可分。故
  $F(\sqrt{x})/F$ 是纯不可分扩张,$F(\sqrt[3]{x})/F$ 是可分扩张,
  这表明 $K/F$ 既不是纯不可分扩张又不是可分扩张。
\end{example}

\begin{definition}
  令 $K/F$ 是域扩张。那么 $F$ 在 $K$ 中的\emph{可分闭包}定义为集合
  \[ 
    \{a\in K\,|\, \text{$a$ is separable over $F$}\}.
  \]$F$ 在 $K$ 中的\emph{纯不可分闭包}
  定义为集合 
  \[
    \{a\in K\,|\, \text{$a$ is purely inseparable over $F$}\}.
  \]
\end{definition}

\begin{proposition}\label{prop:K/S is purely inseparable}
  $K/F$ 是域扩张。设 $S,I$ 分别是 $F$ 在 $K$ 中的可分闭包与纯不可分闭包。那么 $S,I$
  是 $F$ 的扩域且 $S/F$ 可分,$I/F$ 纯不可分,以及 $S\cap I=F$。如果 $K/F$
  是代数扩张,那么 $K/S$ 是纯不可分扩张。
\end{proposition}
\begin{proof}
  任取 $a,b\in S$,根据 \autoref{coro:generators are separable},$F(a,b)/F$
  是可分扩张,所以 $S$ 是域。任取 $c,d\in I$,那么 $c^{p^n},d^{p^m}\in F$,
  于是 $(c\pm d)^{p^{n+m}}\in F$,以及 $(cd)^{p^{n+m}},(c/d)^{p^{n+m}}\in F$,
  所以 $c\pm d,cd,c/d\in I$,所以 $I$ 也是域。根据定义,$S/F$ 可分,$I/F$ 纯不可分。
  根据 \autoref{lemma:property of purely inseparable},有 $S\cap I=F$。

  若 $K/F$ 是代数扩张。任取 $\alpha\in K$,设 $\min(F,\alpha)=g(x^{p^m})$,其中 $g$
  是不可约的可分多项式。由于 $\alpha^{p^m}$ 是 $g$ 的零点,所以 $g(x)$ 
  是 $\alpha^{p^m}$ 在 $F$ 上的最小多项式,所以 $\alpha^{p^m}$ 在 $F$ 上可分,故
  $\alpha^{p^m}\in S$,所以 $\alpha$ 在 $S$ 上纯不可分。 
\end{proof}

如果 $K/F$ 是代数扩张,那么我们可以把 $K/F$ 分为可分扩张 $S/F$ 以及一个纯不可分扩张 $K/S$。
可分闭包是一个处理可分性问题的非常棒的工具。例如,我们证明可分性是可以传递的。

\begin{proposition}
  如果 $F\subseteq L\subseteq K$ 是域且 $L/F,K/L$ 是可分扩张,那么 $K/F$ 可分。
\end{proposition}
\begin{proof}
  记 $S$ 是 $F$ 在 $K$ 中的可分闭包。那么 $L\subseteq S\subseteq K$。$K/L$
  可分表明 $K/S$ 可分,同时 $K/S$ 又纯不可分,所以 $S=K$,即 $K/F$ 可分。
\end{proof}

\autoref{prop:K/S is purely inseparable} 告诉我们 $K/S$ 纯不可分。一个自然的问题
是 $K/I$ 是否可分。一般情况下这是不对的,但是在 $K/F$ 是正规扩张的情况下是正确的。

\begin{theorem}\label{thm:Gal(K/I)}
  $K/F$ 是正规扩张,$S,I$ 分别是 $F$ 在 $K$ 中的可分闭包与纯不可分闭包。那么 $S/F$
  是 Galois 扩张,$I=\Fix(\Gal(K/F))$,并且 $\Gal(S/F)\simeq \Gal(K/I)$。
  因此,$K/I$ 是 Galois 扩张。此外,还有 $K=SI$。
\end{theorem}

$K/F$ 是域扩张。设 $S,I$ 分别是 $F$ 在 $K$ 中的可分闭包与纯不可分闭包。
我们可以定义 $K/F$ 的\emph{可分次数} $[K:F]_s$ 为 $[S:F]$,以及\emph{不可分次数}
$[K:F]_i$ 为 $[K:S]$。那么我们有 $[K:F]=[K:F]_i[K:F]_s$。\autoref{thm:Gal(K/I)}
还告诉我们在 $K/F$ 正规的时候,有 $[K:I]=[S:F]$,因此有 $[K:S]=[I:F]$。
需要注意一般情况下并没有 $[K:S]=[I:F]$ !将不可分次数定义为 $[K:S]$ 而不是 $[I:F]$
是因为 $[K:S]$ 通常可以更好的衡量 $K/F$ 距离可分扩张有多远。下面的例子表明在 $K/F$
不可分的时候也有可能有 $I=F$。






