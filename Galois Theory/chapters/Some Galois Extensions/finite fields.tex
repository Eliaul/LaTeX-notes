
\section{有限域}

如果 $F$ 是特征 $p$ 的有限域,那么 $F$ 包含 $\mathbb{F}_p$,所以 $F$
可以视为 $\mathbb{F}_p$-向量空间,$F$ 有限表明维数是有限维的,即可以假设
$[F:\mathbb{F}_p]=n$。根据线性代数的基本知识,我们知道 $F$ 同构于
$\mathbb{F}_p^n$,所以 $|F|=p^n$。
为了研究有限域,我们首先研究 $F^\times$ 的群结构,实际上有限域的乘法群的群结构
决定了有限域的很多性质。

首先证明一个群论的结论。

\begin{lemma}
  设 $G$ 是 $n$ 阶有限群,如果对于任意正整数 $m$,方程 $x^m=e$ 在 $G$
  中最多只有 $m$ 个解,那么 $G$ 是循环群。
\end{lemma}
\begin{proof}
  记集合
  \[
    A_d=\{g\in G\,|\, \mathrm{ord}(g)=d\},\quad S_d=\{g\in G\,|\, g^d=e\}.
  \]
  显然 $A_d\subseteq S_d$。若 $d\nmid n$,那么 $A_d=\emptyset$。若 $d\mid n$
  且 $A_d\neq\emptyset$,设 $a\in A_d$,那么 $\langle a\rangle\subseteq S_d$。
  根据条件,我们有 $|S_d|\leq d=|\langle a\rangle|$,所以 $S_d=\langle a\rangle$。
  于是 $A_d\subseteq \langle a\rangle$。$\langle a\rangle$ 的生成元有
  $\varphi(d)$ 个,所以 $|A_d|=\varphi(d)$。这表明 $d\mid n$ 的时候,
  要么 $|A_d|=0$,要么 $|A_d|=\varphi(d)$。

  又因为 $G=\bigcup_{d\mid n}A_d$,所以
  \[
    n=|G|=\sum_{d\mid n}|A_d|\leq\sum_{d\mid n}\varphi(d)=n,
  \]
  所以 $d\mid n$ 的时候必须有 $|A_d|=\varphi(d)$。这表明 $|A_n|\neq 0$,
  即 $G$ 中存在 $n$ 阶元。
\end{proof}

\begin{corollary}
  若 $F$ 是有限域,那么 $F^\times $ 是循环群。
\end{corollary}
\begin{proof}
  对于任意正整数 $m$,方程 $x^m=1$ 在 $F$ 中最多只有 $m$ 个解,所以在
  $F^\times$ 中最多也只有 $m$ 个解,所以 $F^\times$ 是循环群。
\end{proof}

\begin{example}
  $\mathbb{F}_p^\times$ 的生成元被称为\emph{模 $p$ 的原根}。例如,
  $\mathbb{F}_5^\times=\{1,2,3,4\}=\langle 2\rangle$,即 $2$ 是模 $5$
  的原根。又比如 $3$ 是模 $7$ 的原根。一般来说,没有一种简单的方式去寻找
  模 $p$ 的原根。
\end{example}

\autoref{thm:primitive element} 证明了无限域情况下的本原元定理,下面我们证明
有限域情况下的。注意到如果 $K/F$ 是有限域的扩张,那么 $K/F$ 显然只有有限个中间域,
因此本原元定理的假设始终成立。下面的推论完成了有限域情况下的本原元定理。

\begin{corollary}
  如果 $K/F$ 是有限域的扩张,那么 $K$ 是 $F$ 的单扩张。
\end{corollary}
\begin{proof}
  设 $K^\times=\langle\alpha\rangle$,那么 $K$ 的任意非零元都是 $\alpha$
  的幂次,所以 $K=F(\alpha)$。
\end{proof}

\begin{theorem}[有限域的结构定理]\label{thm:structure of finite field}
  令 $F$ 的特征 $p$ 的有限域,设 $|F|=p^n$。那么 $F$ 是可分多项式 $x^{p^n}-x$
  在 $\mathbb{F}_p$ 上的分裂域,所以 $F/\mathbb{F}_p$ 是 Galois 扩张。
  此外,如果定义 $\sigma:F\to F$ 为 $\sigma(a)=a^p$,
  那么 $\sigma$ 生成 $\Gal(F/\mathbb{F}_p)$,故 Galois 群 $\Gal(F/\mathbb{F}_p)$
  是循环群。这个自同构 $\sigma$ 被称为\emph{Frobenius 自同构}。
\end{theorem}
\begin{proof}
  根据导数判别法,$x^{p^n}-x$ 是可分多项式。由于 $|F^\times|=p^n-1$,所以任意 $a\in F^\times$ 满足
  $a^{p^n-1}=1$,也即 $a^{p^n}-a=0$,显然 $0$ 也满足这个方程,所以 $F$ 的元素均为
  $x^{p^n}-x$ 的零点。又因为 $x^{p^n}-x$ 至多只有 $p^n$ 个零点,所以 $F$ 的元素
  恰为 $x^{p^n}-x$ 的全部零点,故 $F$ 是 $x^{p^n}-x$ 的分裂域。根据 \autoref{thm:normal separable extension},
  $F/\mathbb{F}_p$ 是 Galois 扩张。 

  不难验证 $\sigma$ 是一个 $\mathbb{F}_p$-单同态。$F$ 有限表明 $\sigma$ 是满射,
  所以是 $\mathbb{F}_p$-自同构,所以 $\sigma\in \Gal(F/\mathbb{F}_p)$。
  注意到 $\Fix(\langle\sigma\rangle)=\{a\in F\,|\, a^p=a\}\supseteq \mathbb{F}_p$,
  且 $x^p-x$ 最多只有 $p$ 个零点,所以 $\mathbb{F}_p=\Fix(\langle\sigma\rangle)$。
  所以 $\Gal(F/\mathbb{F}_p)=\Gal(F/\Fix(\langle \sigma\rangle))=\langle\sigma\rangle$。 
\end{proof} 

\begin{corollary}
  任意两个相同大小的有限域是同构的。
\end{corollary}
\begin{proof}
  $p^n$ 阶有限域是 $x^{p^n}-x$ 在 $\mathbb{F}_p$ 上的分裂域,所以它们是同构的。
\end{proof}

实际上 \autoref{thm:structure of finite field} 可以刻画任意有限域的扩张,
而不必从 $\mathbb{F}_p$ 开始。

\begin{corollary}\label{coro:extension of finite field}
  如果 $K/F$ 是有限域的域扩张,那么 $K/F$ 是具有循环 Galois 群的 Galois 扩张。
  此外,如果 $\cha(F)=p$ 以及 $|F|=p^n$,那么 $\Gal(K/F)$ 由自同构 $\tau$ 生成,
  其中 $\tau(a)=a^{p^n}$。
\end{corollary}
\begin{proof}
  设 $[K:\mathbb{F}_p]=m$,那么 $K$ 是 $x^{p^m}-x$ 在 $\mathbb{F}_p$ 上的分裂域,
  从而也是 $x^{p^m}-x$ 在 $F$ 上的分裂域,所以 $K/F$ 是 Galois 扩张。
  由于 $\Gal(K/F)$ 是 $\Gal(K/\mathbb{F}_p)$ 的子群,所以 $\Gal(K/F)$ 是循环群。
  记 $\sigma$ 是 Frobenius 自同构,设 $s=\lvert\Gal(K/F)\rvert=[K:F]$,那么
  $m=ns$,所以 $\Gal(K/F)$ 的生成元是 $\sigma^n$,满足 $\sigma^n(a)=a^{p^n}$。
\end{proof}

我们已经描述了有限域作为 $\mathbb{F}_p$ 的扩域时的结构,并且知道 $\mathbb{F}_p$
的任意有限扩张都有 $p^n$ 个元素。但是,我们还没有确定是否对于每个 $n$,
都存在 $p^n$ 个元素的有限域。利用基本定理和有限域的结构定理,我们现在证明对于每个
$n$,确实存在唯一的 $p^n$ 个元素的有限域。

\begin{theorem}
  令 $N$ 是 $\mathbb{F}_p$ 的代数闭包。对于每个正整数 $n$,存在唯一的 $p^n$
  个元素的 $N$ 的子域。如果 $K,L$ 分别是 $p^m,p^n$ 阶子域,那么 $K\subseteq L$
  当且仅当 $m\mid n$。此时,$L/K$ 是 Galois 扩张,并且 $\Gal(L/K)=\langle \tau\rangle$,
  其中 $\tau(a)=a^{p^m}$。
\end{theorem}
\begin{proof}
  对于每个正整数 $n$,多项式 $x^{p^n}-x$ 在 $N$ 中有 $p^n$ 个根,这些根的集合
  构成一个域,即 $p^n$ 个元素的子域。根据 \autoref{thm:structure of finite field},
  $N$ 的任意 $p^n$ 阶子域都由 $x^{p^n}-x$ 的所有根组成,所以是唯一的。

  若 $K,L$ 分别是 $p^m,p^n$ 阶子域。假设 $K\subseteq L$,那么 $\mathbb{F}_p\subseteq K\subseteq L$,所以
  \[
    n=[L:\mathbb{F}_p]=[L:K][K:\mathbb{F}_p]=[L:K]m,
  \]
  所以 $m\mid n$。反之,若 $m\mid n$,任取 $a\in K$,那么 $a^{p^m}=a$,
  从而 $a^{p^n}=a$,所以 $a\in L$,即 $K\subseteq L$。
  根据 \autoref{coro:extension of finite field},$L/K$ 是 Galois 扩张,
  并且 $\Gal(L/K)$ 由 $\tau(a)=a^{p^m}$ 生成。
\end{proof}

\begin{corollary}
  令 $F$ 是有限域,$f(x)$ 是 $F$ 上的 $n$ 次首一不可约多项式。
  \begin{enumerate}
    \item 如果 $a$ 是 $f$ 在 $F$ 的任意扩域中的根,那么 $F(a)$ 是 $f$
    在 $F$ 上的分裂域。因此,如果 $K$ 是 $f$ 在 $F$ 上的分裂域,那么
    $[K:F]=n$。
    \item 如果 $|F|=q$,那么 $f$ 的根的集合为 $\big\{a^{q^r}\,|\, r\geq 1\big\}$。
  \end{enumerate}
\end{corollary}
\begin{proof}
  令 $K$ 是 $f$ 在 $F$ 上的分裂域。若 $a\in K$ 是 $f$ 的根,考虑 $F(a)$,
  此时 $[F(a):F]=n$,所以 $|F(a)|=q^n$。由于 $F(a)/F$ 是 Galois 扩张,所以 
  $f=\min(F,a)$ 在 $F(a)$ 中分裂,所以 $K\subseteq F(a)$,故 $F(a)=K$
  是 $f$ 在 $F$ 上的分裂域以及 $[K:F]=n$。

  由于 $\Gal(K/F)=\langle\tau\rangle$,对于 $x\in K$ 有 $\tau(x)=x^{q}$。由于 $a\in K$
  是 $f$ 的根,所以对于任意 $r\geq 1$,$a^{q^r}=\tau^r(a)$ 都是 $f$ 的根。
  再根据同构延拓定理,对于 $f$ 的任意根 $a'$,都存在某个 $r$ 使得 $\tau^r(a)=a'$,
  故 $f$ 的根的集合恰为 $\big\{a^{q^r}\,|\, r\geq 1\big\}$。
\end{proof}

\begin{example}
  考虑 $K=\mathbb{F}_2(\alpha)$,其中 $\alpha$ 是 $f(x)=x^3+x^2+1$ 的根。
  多项式 $f$ 在 $\mathbb{F}_2$ 中没有零点,直接计算可知 $f$ 是不可约多项式,
  所以 $[K:\mathbb{F}_2]=3$。现在我们知道 $K$ 是 $f$ 在 $\mathbb{F}_2$ 上的分裂域,
  并且 $f$ 的所有根为 $\alpha,\alpha^2,\alpha^4$。我们可以验证这一点。
  因为 $\alpha^3+\alpha^2+1=0$,所以 
  \begin{gather*}
    \alpha^6+\alpha^4+1=(\alpha^3+\alpha^2)^2+1=0,\\
    \alpha^{12}+\alpha^8+1=(\alpha^3+\alpha^2)^4+1=0.
  \end{gather*}
  注意到 $\alpha^4=(\alpha^2+1)\alpha=\alpha^2+\alpha+1$,所以 
  $\{1,\alpha,\alpha^2\}$ 构成 $K/\mathbb{F}_2$ 的一组基,这就表明 $\mathbb{F}_2(\alpha)$
  确实是 $f$ 在 $\mathbb{F}_2$ 上的分裂域。
\end{example}


\section{分圆扩张}

一个域的 $n$ 次单位根指的是满足 $\omega^n=1$ 的元素 $\omega$。例如,复数 $\zeta_n=e^{2\pi i/n}$ 是一个 $n$ 次单位根。
本节我们研究域扩张 $F(\omega)/F$。这个扩张在 Galois 理论的应用中扮演着重要角色,
尤其是在多项式方程的可解性问题上。

\begin{definition}
  如果 $\omega\in F$ 满足 $\omega=1$,那么说 $\omega$ 是 $n$ 次单位根。
  如果 $\zeta_n$ 是乘法群 $F^\times$ 的 $n$ 阶元,那么说 $\zeta_n$ 是 
  $n$ 次本原单位根。对于任意单位根 $\omega$,域扩张 $F(\omega)/F$ 被称为
  分圆扩张。
\end{definition}





