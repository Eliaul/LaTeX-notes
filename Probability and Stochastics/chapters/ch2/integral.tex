\chapter{可测函数的积分}

\section{非负函数的积分}

在本章中,我们考虑配备正测度 $\mu$ 的可测空间 $(E,\mathcal{A})$。

\paragraph{简单函数}
如果可测函数 $f:E\to \mathbb{R}$ 的值域是有限集,那么我们说 $f$ 的\emph{简单函数}。
假设 $f$ 的所有可能的取值为 $\alpha_1,\dots,\alpha_n$,不妨假设 $\alpha_1<\alpha_2<\cdots<\alpha_n$。
那么 $f$ 可以表示为
\[
  f(x)=\sum_{i=1}^n \alpha_i\mathbold 1_{A_i}(x),
\]
其中 $A_i=f^{-1}(\{\alpha_i\})\in \mathcal{A}$。注意到 $E$ 是 $A_1,\dots,A_n$ 的无交并。
上述公式 $f=\sum_{i=1}^n \alpha_i\mathbold 1_{A_i}$ 被称为 $f$ 的标准表示。

\begin{definition}
  令 $f$ 是取值在 $\mathbb{R}_+$ 中的简单函数,标准表示为 $f=\sum_{i=1}^n \alpha_i\mathbold 1_{A_i}$。
  定义\emph{$\mathbold f$ 相对于 $\mathbold\mu$ 的积分}为
  \[
    \int f\d\mu=\sum_{i=1}^n\alpha_i\mu(A_i).
  \]
  在 $\alpha_i=0$ 和 $\mu(A_i)=\infty$ 的情况下,约定 $0\times\infty=0$。
\end{definition}

注意上述定义中 $\sum_{i=1}^n\alpha_i\mu(A_i)$ 的取值为 $[0,\infty]$。所以在上述定义中
我们只考虑非负的简单函数,这是为了避免出现 $\infty-\infty$ 之类的表达式。

值得注意的是,如果简单函数 $f$ 有表达
\[
  f=\sum_{j=1}^m\beta_j\mathbold 1_{B_j},
\]
其中 $B_j$ 仍然构成 $E$ 的一个划分,但是 $\beta_j$ 不再是两两不同的。此时
$f$ 的积分仍然为
\[
  \int f\d\mu=\sum_{j=1}^m\beta_j\mu(B_j).
\]
这是因为对于每个 $A_i$,某些 $B_j$ 构成了 $A_i$ 的划分,即
\[
  A_i=\bigcup_{\{j\,|\,\beta_j=\alpha_i\}}B_j,
\]
那么
\[
  \alpha_i\mu(A_i)=\alpha_i\sum_{\{j\,|\,\beta_j=\alpha_i\}}\mu( B_j)=
  \sum_{\{j\,|\,\beta_j=\alpha_i\}}\beta_j\mu( B_j).
\]

非负简单函数的积分满足下面的一些基本的性质。

\begin{proposition}
  令 $f,g$ 是 $E$ 上的非负简单函数。
  \begin{enumerate}
    \item 对于每个 $a,b\in \mathbb{R}_+$,有
    \[
      \int(af+bg)\d\mu=a\int f\d\mu+b\int g\d\mu.
    \]
    \item 如果 $f\leq g$,那么
    \[
      \int f\d\mu\leq \int g\d\mu.
    \]
  \end{enumerate}
\end{proposition}
\begin{proof}
  (1) 设 $f,g$ 的标准表示分别为
  \[
    f=\sum_{i=1}^n\alpha_i\mathbold 1_{A_i},\quad  
    g=\sum_{j=1}^m\beta_j\mathbold 1_{B_j}.  
  \]
  那么每个 $A_i$ 都是某些 $A_i\cap B_j$ 的无交并,同理,
  每个 $B_j$ 都是某些 $A_i\cap B_j$ 的无交并,于是我们可以
  使用一个新的划分 $\{C_1,\dots,C_p\}$ 使得
  \[
    f=\sum_{k=1}^p \gamma_k\mathbold 1_{C_k} ,\quad
    g=\sum_{k=1}^p \theta_k\mathbold 1_{C_k} ,
  \]
  此时 $\gamma_k$ 不一定互不相同,$\theta_k$ 也不一定互不相同,
  根据命题前面的叙述,我们有
  \begin{align*}
    \int
    (af+bg)\d\mu&=\sum_{k=1}^p(a\gamma_k+b\theta_k)  \mu(C_k)\\&=
    a\sum_{k=1}^p\gamma_k\mu(C_k)+b\sum_{k=1}^p\theta_k\mu(C_k)\\
    &=a\int f\d\mu+b\int g\d\mu.
  \end{align*}

  (2) 由 (1),有
  \[
    \int g\d\mu=\int(g-f)\d\mu+\int f\d\mu\geq \int f\d\mu.\qedhere  
  \]
\end{proof}

我们用 $\mathcal{E}_+$ 来表示 $E$ 上的非负简单函数的集合。

\begin{definition}
  令 $f:E\to[0,\infty]$ 是可测函数,定义\emph{$\mathbold f$ 相对于 $\mathbold\mu$ 的积分}
  为
  \[
    \int f\d\mu=\sup_{h\in \mathcal{E}_+,h\leq f}\int h\d\mu.
  \]
\end{definition}

$f$ 相对于 $\mu$ 的积分通常有很多写法,下面的表达
\[
  \int f\d\mu,\ \int f(x)\d\mu(x),\ \int f(x)\mu(\d x),\ \int\mu(\d x)f(x)
\]
表示的含义是完全相同的。此外,如果 $A$ 是 $E$ 的可测子集,我们定义
\[
  \int_A f\d\mu=\int f\mathbold 1_A\d\mu.
\]

从现在开始,我们用非负可测函数表示 $E\to [0,\infty]$ 的可测函数(值可以为无穷)。
需要注意的是,我们前面定义的非负简单函数值必须有限。

\begin{proposition}\label{prop:elementary property of positive measurable function}
  令 $f,g$ 是 $E$ 上的非负可测函数。
  \begin{enumerate}
    \item 如果 $f\leq g$,那么 $\int f\d\mu\leq \int g\d\mu$。
    \item 如果 $\mu(\{x\in E\,|\, f(x)> 0\})=0$,那么 $\int f\d\mu=0$。
  \end{enumerate}
\end{proposition}
\begin{proof}
  (1) 显然
  \[
      \bigl\{h\in \mathcal{E}_+\,|\, h\leq f\bigr\}\subseteq 
      \bigl\{h\in \mathcal{E}_+\,|\, h\leq g\bigr\},
  \]
  根据定义即得 $\int f\d\mu\leq \int g\d\mu$。

  (2) 设 $h\in \mathcal{E}_+$ 且 $h\leq f$,设 $h$ 的标准表示
  为 $h=\sum_{i=1}^n \alpha_i\mathbold 1_{A_i}$,若 $\alpha_i> 0$,那么
  \[
    \mu(A_i)\leq   \mu (\{x\in E\,|\, h(x)> 0\})\leq \mu(\{x\in E\,|\, f(x)> 0\})=0,
  \]
  所以
  \[
    \int h\d\mu=\sum_{\{i\,|\,\alpha_i=0\}}\alpha_i\mu(A_i)+
    \sum_{\{i\,|\,\alpha_i>0\}}\alpha_i\mu(A_i)=0+0=0,
  \]
  故 $\int f\d\mu=0$。
\end{proof}

下面的单调收敛定理是测度论中的一个极为重要的基本定理,其表明
对于一列递增的非负可测函数,极限和积分可以交换次序。

\begin{theorem}[单调收敛定理]\label{thm:monotone convergence thm}
  令 $(f_n)_{n\in \mathbb{N}}$ 是 $E$ 上的一列递增的非负可测函数,即 $f_n\leq f_{n+1}$,
  记 $f=\ulim f_n$,那么
  \[
    \int f\d\mu=\ulim[n\to\infty]\int f_n\d\mu.
  \]
\end{theorem}
\begin{proof}
  由于 $f_n\leq f$,所以 $\int f_n\d\mu\leq\int f\d\mu$,
  所以 $\ulim \int f_n\d\mu\leq\int f\d\mu$,于是我们只需要证明
  反向的不等式。

  假设非负可测函数 $h=\sum_{i=1}^k \alpha_i\mathbold 1_{A_i}$
  满足 $h\leq f$,任取 $a\in[0,1)$,定义一列可测集
  \[
    E_n=\{x\in E\,|\, ah(x)\leq f_n(x)\}  ,
  \]
  此时对于任意的 $x\in E$,都有 $ah(x)<h(x)\leq f(x)$,而 $f=\ulim f_n$,
  所以总存在足够大的 $n$,使得 $ah(x)\leq f_n(x)$,这表明 $E=\bigcup_{n\in \mathbb{N}}E_n$。
  此外,$f_n\leq f_{n+1}$ 表明 $E_n\subseteq E_{n+1}$。

  显然 $f_n\geq ah\mathbold 1_{E_n}$,所以
  \[
    \int f_n\d\mu\geq a\int_{E_n} h\d\mu=
    a\sum_{i=1}^k\alpha_i\mu(A_i\cap E_n),  
  \]
  由于 $A_i=A_i\cap E=\bigcup_{n\in \mathbb{N}}(A_i\cap E_n)$,
  所以
  \[
    \mu(A_i)=\mu\biggl(\bigcup_{n\in \mathbb{N}}(A_i\cap E_n)\biggr)
    =\ulim[n\to\infty]\mu(A_i\cap E_n),
  \]
  于是
  \[
    \ulim[n\to\infty]\int f_n\d\mu\geq a\sum_{i=1}^k
    \alpha_i\ulim[n\to\infty]\mu(A_i\cap E_n)
    =  a\sum_{i=1}^k\alpha_i\mu(A_i)
    =a\int h\d\mu,
  \]
  由于 $a$ 可以任意接近 $1$,所以
  \[
    \ulim[n\to\infty]\int f_n\d\mu\geq \int h\d\mu,
  \]
  所以
  \[
    \ulim[n\to\infty]\int f_n\d\mu\geq \int f\d\mu=
    \sup_{h\in \mathcal{E}_+,h\leq f}\int h\d\mu.\qedhere
  \]
\end{proof}


\begin{proposition}\label{prop:property of integral of positive function}
  \mbox{}
  \begin{enumerate}
    \item 设 $f$ 是 $E$ 上的非负可测函数,那么存在一列递增的非负简单函数 $(f_n)_{n\in \mathbb{N}}$
    使得 $f=\ulim f_n$。如果 $f$ 有界,那么 $f_n\to f$ 一致收敛。
    \item 令 $f,g$ 是两个 $E$ 上的非负可测函数,$a,b\in \mathbb{R}_+$,那么
    \[
      \int (af+bg)\d\mu=a\int f\d\mu+b\int g\d\mu.
    \]
    \item 令 $(f_n)_{n\in \mathbb{N}}$ 是一列 $E$ 上的非负可测函数,那么
    \[
      \int\biggl(\sum_{n\in \mathbb{N}}f_n\biggr)\d\mu=\sum_{n\in \mathbb{N}}\int f_n\d\mu.
    \]
  \end{enumerate}
\end{proposition}
\begin{proof}
  (1) 令 $d_n:[0,\infty]\to \mathbb{R}_+$ 为
  \[
    d_n=\sum_{k=1}^{n2^n}\frac{k-1}{2^n}  
    \mathbold 1_{\left[\frac{k-1}{2^n},\frac{k}{2^n}\right)}+
    n\mathbold 1_{[n,\infty]},
  \]
  显然 $d_n$ 是非负简单函数。直观上来看,$d_n$ 将区间 $[0,n]$
  等分为了 $n2^n$ 份,即将 $[0,1]$ 等分为了 $2^n$ 份。
  那么对于 $x\in [0,n)$,总存在唯一的 $k_n$ 使得 
  $(k_n-1)/2^n\leq x< k_n/2^n$,此时 $k_{n+1}=2k_n$
  或者 $k_{n+1}=2k_n-1$,所以
  \[
    d_{n+1}(x)  =\frac{k_{n+1}-1}{2^{n+1}}\geq \frac{k_n-1 }{2^n}
    =d_n(x),
  \]
  这表明 $d_n\leq d_{n+1}$。此外,不难看出 $\lim d_n(x)=x$。
  
  令 $f_n=d_n\circ f$,由于 $f_n$ 只有有限多个取值,所以
  $f_n$ 是非负简单函数。$d_n\leq d_{n+1}$ 表明
  $f_n\leq f_{n+1}$。且 $\lim f_n=\lim d_n\circ f=f$,
  所以 $f_n$ 就是一列递增的非负简单函数且 $f=\ulim f_n$。
  $f$ 有界表明在 $n$ 足够大的时候有 $0\leq f-f_n\leq 2^{-n}$,
  即 $f_n\to f$ 一致收敛。

  (2) 由 (1),设 $f=\ulim f_n$,$g=\ulim g_n$,其中 $(f_n),(g_n)$
  均为一列递增的简单函数,那么
  \[
    \int (af_n+bg_n)\d\mu=a\int f_n\d\mu+b\int g_n\d\mu,  
  \]
  令 $n\to\infty$,再根据单调收敛定理,就有
  \[
    \int (af+bg)\d\mu=a\int f\d\mu+b\int g\d\mu.
  \]

  (3) 根据 (2),有
  \[
    \int\biggl(\sum_{n=1}^m f_n\biggr)  \d\mu=
    \sum_{n=1}^m \int f_n\d\mu,
  \]
  令 $m\to\infty$,再根据单调收敛定理,就有
  \[
    \int\biggl(\sum_{n\in \mathbb{N}}f_n\biggr)\d\mu=\sum_{n\in \mathbb{N}}\int f_n\d\mu.
    \qedhere
  \]
\end{proof}

\begin{remark}
  \autoref{prop:property of integral of positive function} 和
  单调收敛定理 \ref{thm:monotone convergence thm} 给出了证明
  关于非负可测函数积分的命题的一种基本范式,即根据
  \autoref{prop:property of integral of positive function} 的 (1),
  假设一列非负简单函数逼近原函数,先证明命题对非负简单函数成立,
  这通常是非常容易的,再使用单调收敛定理证明命题对所有的
  非负可测函数成立。
\end{remark}

下面的推论在概率论中十分有用,其对应于随机变量的概率密度函数。其
证明是上述注释中技巧的一个典型运用。

\begin{corollary}\label{coro:density of measure}
  令 $g$ 是非负可测函数,对于 $A\in \mathcal{A}$,令
  \[
    \nu(A)=\int_A g\d\mu=\int g\mathbold 1_A \d\mu,
  \]
  那么 $\nu$ 是 $E$ 上的正测度,被称为密度 $g$ 相对于 $\mu$
  的测度,记为 $\nu=g\cdot \mu$。此外,对于非负可测函数
  $f$,有
  \[
    \int f\d\nu=\int fg\d\mu.
  \]
\end{corollary}
\begin{proof}
  显然 $\nu(\emptyset)=0$。任取一列不相交的 $A_n\in \mathcal{A}$,
  那么
  \[
    \nu\biggl(\bigcup_{n\in \mathbb{N}}A_n\biggr)  
    =\int \biggl(\sum_{n\in \mathbb{N}}g\mathbold 1_{A_n}\biggr)
    \d\mu=\sum_{n\in \mathbb{N}}\int g\mathbold 1_{A_n}\d\mu
    =\sum_{n\in \mathbb{N}}\nu(A_n),
  \]
  这就表明 $\nu$ 是 $E$ 上的正测度。

  对于任意示性函数 $\mathbold 1_A$,有
  \[
    \int \indicator{A}\d\nu=\nu(A)=\int \indicator{A}g\d\mu,
  \]
  进一步的,令 $f=\ulim f_n$,其中 $f_n$ 是非负简单函数,
  对于每个 $f_n$,根据积分的线性性,都有
  \[
    \int f_n \d\nu=\int f_ng\d\mu, 
  \]
  令 $n\to\infty$,根据单调收敛定理,就有 
  \[
    \int f\d\nu=\int fg\d\mu.\qedhere
  \]
\end{proof}
\begin{remark}
  在实际中,我们通常也会写作 $\nu(\d x)=g(x)\mu(\d x)$,或者
  $g=\d\nu/\d\mu$。
\end{remark}

在测度论中,命题通常在\emph{几乎处处}(almost everywhere)的意义下成立,也就是说,
对于不满足该命题的所有 $x\in E$ 的集合,这个集合的 $\mu$-测度为 $0$,
我们使用简写 $\alev{\mu}$ 来表示这个意思。也就是说,当我们
写到 $f=g,\ \alev{\mu}$ 的时候,我们表示的意思实际上是
\[
  \mu\bigl(\{x\in E\,|\, f(x)\neq g(x)\}\bigr)=0.
\]

\begin{proposition}\label{prop:more properties of integral of positive function}
  令 $f$ 是非负可测函数。
  \begin{enumerate}
    \item 对于每个 $a\in(0,\infty)$,有
    \[
      \mu(\{x\in E\,|\, f(x)\geq a\})\leq \frac{1}{a}\int f\d\mu.
    \]
    \item 我们有
    \[
      \int f\d\mu<\infty\Rightarrow f<\infty,\ \text{$\mu$ a.e.}
    \]
    \item 我们有
    \[
      \int f\d\mu=0\Leftrightarrow f=0,\ \text{$\mu$ a.e.}
    \]
    \item 如果 $g$ 是非负可测函数,
    \[
      f=g,\ \text{$\mu$ a.e.}\Rightarrow \int f\d\mu=\int g\d\mu.
    \]
  \end{enumerate}
\end{proposition}
\begin{proof}
  (1) 令可测集
  $
    A=\{x\in E\,|\, f(x)\geq a\}
  $,
  那么 $f\geq a\indicator{A}$,所以
  \[
    \int f\d\mu\geq a\int \indicator{A}\d\mu  
    =a\mu(A).
  \]

  (2) 令可测集
  $
    A_n=\{x\in E\,|\, f(x)\geq n\}  
  $ 以及 $A_\infty=\{x\in E\,|\, f(x)=\infty\}$,
  那么 $A_{n+1}\subseteq A_n$ 且 $A_\infty=\bigcap_{n\in \mathbb{N}}A_n$。
  根据 (1),有
  \[
    \mu(A_1)\leq \int f\d\mu<\infty,  
  \]
  所以
  \[
    \mu(A_\infty)=\dlim[n\to\infty] \mu(A_n)\leq
    \dlim[n\to\infty]\frac{1}{n}\int f\d\mu=0, 
  \]
  所以 $\mu(A_\infty)=0$,即 $f<\infty,\ \alev{\mu}$。

  (3) 充分性由 \autoref{prop:property of integral of positive function} 的 (2) 保证。
  下证必要性。令可测集
  $
    A_n=\{x\in E\,|\, f(x)\geq 1/n\}  
  $ 以及
  $A_\infty=\{x\in E\,|\, f(x)\neq 0\}$,那么
  $A_n\subseteq A_{n+1}$ 且 $A_\infty=\bigcup_{n\in \mathbb{N}}A_n$。
  根据 (1),有
  \[
    \mu(A_\infty)=\ulim[n\to\infty]\mu(A_n)\leq
    \ulim[n\to\infty]n\int f\d\mu=0,
  \]
  所以 $\mu(A_\infty)=0$。

  (4) 记 $f\wedge g=\min(f,g)$ 及 $f\vee g=\max(f,g)$,那么
  $f=g,\ \alev{\mu}$ 表明  $f\vee g=f\wedge g,\ \alev{\mu}$。
  根据 (3),有
  \[
    \int f\vee g\d\mu=\int f\wedge g\d\mu+\int (f\vee g-f\wedge g)\d\mu
    =\int f\wedge g\d\mu,  
  \]
  又因为 $\int f\wedge g\d\mu\leq \int f\d\mu\leq \int f\vee g\d\mu$,
  对于 $g$ 类似,所以
  \[
    \int f\d\mu=\int g\d\mu.\qedhere  
  \]
\end{proof}

\begin{theorem}[Fatou 引理]
  令 $(f_n)_{n\in \mathbb{N}}$ 是一列非负可测函数,那么
  \[
    \int \liminf f_n\d\mu\leq \liminf \int f_n\d\mu.
  \]
\end{theorem}
\begin{proof}
  只需证明
  \[
    \int\lim_{n\to\infty}\inf_{k\geq n}f_k \d\mu\leq 
    \lim_{n\to\infty}\inf_{k\geq n}\int f_k\d\mu,  
  \]
  令 $g_n=\inf_{k\geq n}f_k$,那么 $g_n\leq g_{n+1}$,根据
  单调收敛定理,有
  \[
    \int \lim_{n\to\infty}g_n\d\mu=\ulim[n\to\infty]\int g_n\d\mu.
  \]
  对于任意 $n$ 和 $k\geq n$,有 $\int g_n\d\mu\leq \int f_k\d\mu$,所以
  \[
    \int \lim_{n\to\infty}g_n\d\mu=\ulim[n\to\infty]\int g_n\d\mu
    \leq\inf_{k\geq n} \int f_k\d\mu,
  \]
  令 $n\to\infty$,即可得到
  \[
    \int \liminf f_n\d\mu\leq \liminf \int f_n\d\mu.\qedhere
  \]
\end{proof}

可以用一个简单的例子理解 Fatou 引理。考虑 $f_n=\mathbold 1_{[n,n+1]}$,
那么对于任意 $x\in \mathbb R$,有 $\liminf f_n(x)=0$,此时显然有
\[
  0=\int \liminf f_n\d\mu \leq \liminf \int f_n\d\mu=1.
\]
直观上来看,就是把函数压扁(取极限)再求和(积分),得到的总和通常会变小,因为
在这个过程中,有些原本存在的面积可能逃逸或者坍缩掉了。


\begin{proposition}\label{prop:change variable}
  令 $(F,\mathcal{B})$ 是可测空间,$\varphi:E\to F$ 是可测映射。令
  $\nu$ 是 $\mu$ 在 $\varphi$ 下的推前。那么,对于任意 $F$ 上的非负可测函数 $h$,
  我们有
  \[
    \int_Eh(\varphi(x))\mu(\d x)=\int_Fh(y)\nu(\d y).
  \]
\end{proposition}
\begin{proof}
  若 $h=\indicator{B}$ 是示性函数,那么
  \[
    \int_E h(\varphi(x))\mu(\d x)=
    \mu(\varphi^{-1}(B)) =\nu(B)=\int_F h(y)\nu(\d y).
  \]
  若 $h=\sum_{i=1}^n\alpha_i\indicator{B_i}$ 是非负简单函数,
  那么根据积分的线性性,结论也成立。
  若 $h$ 是一般的非负可测函数,设 $(h_n)_{n\in \mathbb{N}}$ 是一列递增的非负简单函数
  且 $h=\ulim h_n$,根据单调收敛定理,即可证明结论。
\end{proof}



\section{可积函数}

本节我们讨论可变号的可测函数。
如果 $f:E\to \mathbb{R}$ 是可测函数,记 $f$ 正部分 $f^+=\max(f,0)$,
负部分 $f^-=\max(-f,0)$,需要注意 $f^+$ 和 $f^-$ 此时都是非负可测函数
并且 $f=f^+-f^-$,$|f|=f^++f^-$。

\begin{definition}
  令 $f:E\to \mathbb{R}$ 是可测函数,如果
  \[
    \int|f|\d\mu<\infty,
  \]
  那么我们说 $f$ 相对于 $\mu$ \emph{可积}。在这种情况下,
  我们定义
  \[
    \int f\d\mu=\int f^+\d\mu-\int f^-\d\mu.
  \]
  如果 $A\in \mathcal{A}$,记
  \[
    \int_A f\d\mu=\int f\mathbold 1_A\d\mu.
  \]
\end{definition}

我们使用 $\mathcal{L}^1(E,\mathcal{A},\mu)$ 来表示所有可积函数
$f:E\to \mathbb{R}$ 构成的空间。$\mathcal{L}_+^1(E,\mathcal{A},\mu)$ 来表示
所有非负可积函数构成的空间。

\begin{proposition}[可积函数的性质]
  \mbox{}
  \begin{enumerate}
    \item 对于任意 $f\in \mathcal{L}^1(E,\mathcal{A},\mu)$,有
    $\bigl|\int f\d\mu\bigr|\leq \int |f|\d\mu$。
    \item $\mathcal{L}^1(E,\mathcal{A},\mu)$ 是 $\mathbb{R}$-向量空间。
    \item 如果 $f,g\in \mathcal{L}^1(E,\mathcal{A},\mu)$ 且
    $f\leq g$,那么 $\int f\d\mu\leq \int g\d\mu$。
    \item 如果 $f\in \mathcal{L}^1(E,\mathcal{A},\mu)$,$g:E\to [0,\infty]$
    是非负可测函数使得 $f=g,\ \alev{\mu}$,那么
    $g\in \mathcal{L}^1(E,\mathcal{A},\mu)$ 且 $\int f\d\mu=\int g\d\mu$。
    \item 令 $(F,\mathcal{B})$ 是可测空间,$\varphi:E\to F$ 是可测映射。令
    $\nu$ 是 $\mu$ 在 $\varphi$ 下的推前。那么,对于任意可测函数 $h:F\to \mathbb{R}$,
    $h$ 是 $\nu$-可积的当且仅当 $h\circ\varphi$ 是 $\mu$-可积的,并且我们有
    \[
      \int_Eh(\varphi(x))\mu(\d x)=\int_Fh(y)\nu(\d y).
    \]
  \end{enumerate}
\end{proposition}

\paragraph{延拓到复值函数} 
令 $f:E\to \mathbb{C}$ 是可测函数(这等价于实部 $\Re f$ 和虚部 $\Im f$ 都是可测的),
如果
\[
  \int |f|\d \mu <\infty,
\]
那么我们说 $f$ 是可积的。在这种情况下,我们定义
\[
  \int f\d\mu=\int \Re(f)\d \mu+i\int \Im (f)\d\mu.
\]
记复值可积函数空间为 $\mathcal L_{\mathbb{C}}^1(E,\mathcal{A},\mu)$,
那么上述命题对复值可积函数依然成立。并且 $\mathcal L_{\mathbb{C}}^1(E,\mathcal{A},\mu)$
是一个复向量空间。



\begin{theorem}[控制收敛定理]
  令 $(f_n)_{n\in \mathbb{N}}$ 是 $\mathcal{L}^1(E,\mathcal{A},\mu)$
  中的一列函数(对于 $\mathcal{L}_{\mathbb{C}}^1(E,\mathcal{A},\mu)$ 也成立),如果:
  \begin{enumerate}
    \item 存在可测函数 $f:E\to \mathbb{R}$ 使得
    \[
      f_n(x)\to f(x),\quad\alev{\mu}  
    \]
    \item 存在非负可测函数 $g$ 使得 $\int g\d\mu<\infty$,并且
    对于每个 $n\in \mathbb{N}$,都有
    \[
      |f_n(x)|\leq g(x),\quad \alev{\mu}   
    \]
  \end{enumerate}
  那么 $f\in \mathcal{L}^1(E,\mathcal{A},\mu)$ 且我们有
  \[
    \lim_{n\to\infty}\int f_n\d\mu=\int f\d\mu,\quad
    \lim_{n\to\infty}\int |f_n-f|\d\mu=0.  
  \]
\end{theorem}
\begin{proof}
  我们首先将两个条件中的几乎处处去掉,证明结论成立。
  由于 $|f_n|\leq g$,所以 $|f|=\lim|f_n|\leq g$,所以
  $\int |f|\d\mu\leq \int g\d\mu<\infty$,故
  $f\in L^1(E,\mathcal{A},\mu)$。由于 $|f-f_n|\leq 2g$
  以及 $\lim |f-f_n|=0$,根据 Fatou 引理,有
  \[
    \liminf \int\bigl(2g-|f-f_n|\bigr)\d\mu\geq
    \int \bigl(2g-\limsup|f-f_n|\bigr)\d\mu 
    =\int 2g\d\mu,
  \]
  再根据积分的线性性,有
  \[
    \int 2g\d\mu-\limsup\int|f-f_n|\d\mu\geq \int 2g\d\mu,  
  \]
  这表明
  \[
    \limsup\int |f-f_n|\d\mu = 0,  
  \]
  所以 $\lim \int |f-f_n|\d\mu$ 存在且为 $0$。最后,我们有
  \[
    \left|\int f\d\mu-\int f_n\d\mu\right|\leq \int |f-f_n|\d\mu\to 0,
  \]
  所以 $\int f\d\mu=\lim\int f_n\d\mu$。

  现在我们证明几乎处处的情况。记
  \[
    A=\big\{x\in E\,|\, f_n(x)\to f(x),|f_n(x)|\leq g(x)\big\}  ,
  \]
  那么 $A$ 可测且条件表明 $\mu(A^c)=0$。定义
  \[
    \tilde f_n(x)=\indicator{A}(x)f_n(x),\quad \tilde{f}(x)=\indicator{A}(x)f(x),  
  \]
  于是在几乎处处的意义下有 $f_n=\tilde{f}_n$ 以及 $f=\tilde{f}$,
  所以 $\int f_n\d\mu=\int\tilde{f}_n\d\mu$,
  $\int f\d\mu=\int\tilde f\d\mu$ 以及 $\int|f-f_n|\d\mu=\int|\tilde f-\tilde f_n|\d\mu$。
  对 $\tilde{f}_n$ 和 $\tilde{f}$ 应用上面的结论即可。
\end{proof}

\section{含参积分}

我们考虑带有一个参数的函数的积分。
设 $(U,d)$ 是一个度量空间,参数位于这个空间中。

\begin{theorem}[含参积分的连续性]
  令 $f:U\times E\to \mathbb{R}$ (or $\mathbb{C}$),$u_0\in U$。假设:
  \begin{enumerate}
    \item 对于每个 $u\in U$,函数 $x\mapsto f(u,x)$ 可测;
    \item $\alev{\mu(\d x)}$,函数 $u\mapsto f(u,x)$ 在 $u_0$ 处连续;
    \item 存在函数 $g\in \mathcal{L}_+^1(E,\mathcal{A},\mu)$
    使得任取 $u\in U$ 有
    \[
      |f(u,x)|\leq g(x)\quad \alev{\mu(\d x)}  
    \]
  \end{enumerate}
  那么函数 $F(u)=\int f(u,x)\mu(\d x)$ 是良好定义的且在 $u_0$
  处连续。
\end{theorem}
\begin{proof}
  条件 (1) 保证了 $x\mapsto f(u,x)$ 是可测的,所以 $F(u)$ 是良好定义的。
  设 $(u_n)_{n\in \mathbb{N}}$ 是任意趋于 $u_0$ 的点列,那么 
  对于几乎处处的 $x$,$u\mapsto f(u,x)$ 连续表明 $f(u_n,x)\to f(u,x)$,
  再根据条件 (3) 和控制收敛定理,就有
  \[
    F(u_0)=\int \lim_{n\to\infty} f(u_n,x)\mu(\d x)
    =  \lim_{n\to\infty}\int f(u_n,x)\mu(\d x)=\lim_{n\to\infty}F(u_n),
  \]
  这就表明 $F(u)$ 在 $u_0$ 处连续。
\end{proof}

\begin{example}
  \mbox{}
  \begin{enumerate}
    \item 令 $\mu$ 是 $(\mathbb{R},\mathcal{B}(\mathbb{R}))$ 上的扩散测度并且
    $\varphi\in \mathcal{L}^1(\mathbb{R},\mathcal{B}(\mathbb{R}),\mu)$。
    定义函数 $F:\mathbb{R}\to \mathbb{R}$ 为
    \[
      F(u)=\int_{(\infty,u]}\varphi(x)\mu(\d x)=
      \int \idf_{(-\infty,u]}(x)\varphi(x)\mu(\d x).
    \]
    令 $f(u,x)=\idf_{(-\infty,u]}(x)\varphi(x)$,
    那么任取 $u,x\in  \mathbb{R}$ 有 $|f(u,x)|\leq |\varphi(x)|$。
    此外,对于任意 $u_0\in \mathbb{R}$,当 $x\in \mathbb{R}\setminus\{u_0\}$
    时,函数 $u\mapsto f(u,x)$ 在 $u_0$ 处连续,由于 $\mu$ 是扩散测度,
    即 $\mu(\{u_0\})=0$,所以满足条件 (2)。根据上述定理,$F$ 在 $\mathbb{R}$ 上连续。 
    \item \emph{Fourier 变换}。令 $\lambda$ 表示 $\mathbb{R}$ 上的 Lebesgue 测度。
    如果 $\varphi\in \mathcal{L}^1(\mathbb{R},\mathcal{B}(\mathbb{R}),\lambda)$,
    定义函数 $\hat\varphi:\mathbb{R}\to \mathbb{C}$ 为:
    \[
      \hat\varphi(u)=\int e^{iux}\varphi(x)\lambda(\d x),  
    \]
    根据上面的定理,$\hat\varphi$ 是连续函数。函数 $\hat\varphi$ 被称为 $\varphi$
    的 \emph{Fourier 变换}。在概率论中,我们经常会考虑有限测度的 Fourier 变换。
    如果 $\mu$ 是 $\mathbb{R}$ 上的有限测度,定义 $\mu$ 的 Fourier 变换为
    \[
      \hat{\mu}(u)=\int e^{iux}\mu(\d x)\quad u\in \mathbb{R}.  
    \]
    此时 $|e^{iux}|\leq \indicator{\mathbb{R}}$ 是可积函数,所以
    $\hat\mu$ 是连续函数。
    \item \emph{卷积}。令 $\varphi\in \mathcal{L}^1(\mathbb{R},\mathcal{B}(\mathbb{R}),\lambda)$,
    $h: \mathbb{R}\to \mathbb{R}$ 是有界连续函数,那么定义函数 $h*\varphi$ 为
    \[
      h*\varphi(u)=\int h(u-x)\varphi(x)\lambda(\d x),  
    \]
    这是一个连续函数。
  \end{enumerate}
\end{example}

下面我们叙述含参积分的可微性。令 $I\subseteq \mathbb{R}$ 是开区间。

\begin{theorem}
  考虑函数 $f:I\times E\to \mathbb{R}$,$u_0\in I$,假设
  \begin{enumerate}
    \item 对于每个 $u\in I$,函数 $x\mapsto f(u,x)$ 是可积函数;
    \item $\alev{\mu(\d x)}$,函数 $u\mapsto f(u,x)$ 在 $u_0$ 处可导,
    导数记为
    \[
      \frac{\partial f}{\partial u}(u_0,x);  
    \]
    \item 存在函数 $g\in \mathcal{L}_+^1(E,\mathcal{A},\mu)$ 使得
    对于任意 $u\in I$ 有
    \[
      |f(u,x)-f(u_0,x)|\leq g(x)|u-u_0|,\quad \alev{\mu(\d x)}  
    \]
  \end{enumerate}
  那么函数 $F(u)=\int f(u,x)\mu(\d x)$ 在 $u_0$ 处可导,并且
  \[
    F'(u_0)=\int \frac{\partial f}{\partial u}(u_0,x)\mu(\d x).
  \]
\end{theorem}
\begin{remark}
  (2) 中的 $\partial f/\partial u(u_0,x)$ 只在某个 $\mu$-零可测集 $H$ 的补集
  上有定义。当 $x\in H$ 的时候,我们定义其为 $0$,这样就可以保证
  $x\mapsto \partial f/\partial u(u_0,x)$ 在整个 $E$ 上都有定义。
  利用下面证明中的 $\varphi_n$ 和 \autoref{lemma:pointwise limit measurable},
  $x\mapsto \partial f/\partial u(u_0,x)$ 是可测的。
  再利用条件 (3),这个函数也是可积的,这使得 $F'(u_0)$ 有意义。
  我们也可以把这个定理推广到复数情形,只需要对实部和虚部分别应用该定理即可。
\end{remark}

\begin{proof}
  令 $(u_n)_{n\geq 1}$ 是 $I\setminus\{u_0\}$ 中收敛到 $u_0$ 的序列,令
  \[
    \varphi_n(x)=\frac{f(u_n,x)-f(u_0,x)}{u_n-u_0}.
  \]
  那么对于几乎处处的 $x$,有
  \[
    \lim_{n\to\infty}\varphi_n(x)=\frac{\partial f}{\partial u}(u_0,x).  
  \]
  根据条件 (3),对于任意 $n$,都有
  \[
    |\varphi_n(x)|\leq g(x),\quad \alev{\mu(\d x)}.  
  \]
  根据控制收敛定理,有
  \[
    \lim_{n\to\infty}\int \varphi_n(x)\mu(\d x)=
    \int \frac{\partial f}{\partial u}(u_0,x)\mu(\d x).
  \]
  注意到
  \[
    \int \varphi_n(x)\mu(\d x)=\frac{F(u_n)-F(u_0)}{u_n-u_0},  
  \]
  所以
  \[
    F'(u_0)=\int \frac{\partial f}{\partial u}(u_0,x)\mu(\d x).\qedhere  
  \]
\end{proof}


在许多应用中,条件 (2) 和 (3) 通常以下更易验证的形式出现:
\begin{enumerate}
  \item[(2$'$)] $\alev{\mu(\d x)}$,函数 $u\mapsto f(u,x)$ 在 $I$ 上可微;
  \item[(3$'$)] 存在函数 $g\in \mathcal{L}_+^1(E,\mathcal{A},\mu)$ 使得
  对于任意 $u\in I$ 有
  \[
    \left|\frac{\partial f}{\partial u}(u,x)\right|\leq g(x),\quad \alev{\mu(\d x)}.  
  \]
\end{enumerate}

\begin{example}
  \mbox{}
  \begin{enumerate}
    \item 令 $\varphi\in \mathcal{L}^1(\mathbb{R},\mathcal{B}(\mathbb{R}),\lambda)$ 使得
    \[
      \int |x\varphi(x)|\lambda(\d x)<\infty.
    \]
    那么 Fourier 变换 $\hat\varphi(u)$ 在 $\mathbb{R}$ 中可微,并且
    \[
      \hat\varphi'(u)=\int ixe^{iux}\varphi(x)\lambda(\d x).
    \]
    \item 令 $\varphi\in \mathcal{L}^1(\mathbb{R},\mathcal{B}(\mathbb{R}),\lambda)$,
    $h:\mathbb{R}\to \mathbb{R}$ 是连续可微函数,并且 $h$ 和 $h'$ 都有界,那么
    卷积 $h*\varphi$ 在 $\mathbb{R}$ 上可微,并且
    \[
      (h*\varphi)'= h'*\varphi.
    \]
    这个论证可以递归下去。例如,如果 $h$ 是无穷可微且紧支的,那么 $h*\varphi$
    也是无穷可微的。
  \end{enumerate}
\end{example}


\begin{exercise}
  计算
  \[
    \lim_{n\to\infty}\int_0^n\left(1+\frac{x}{n}\right)^ne^{-2x}\d x.  
  \]
  令 $\alpha\in \mathbb{R}$,证明极限
  \[
    \lim_{n\to\infty}\int_0^n\left(1-\frac{x}{n}\right)^n x^{\alpha-1}\d x  
  \]
  在 $[0,\infty]$ 上存在,且极限值有限当且仅当 $\alpha>0$。
\end{exercise}
\begin{proof}
  令
  \[
    f_n(x)=\idf_{[0,n]}(x)\left(1+\frac{x}{n}\right)^n e^{-2x},
  \]
  那么对于任意 $x\geq 0$,都有 $\lim_{n\to\infty}f_n(x)=e^{-x}$。
  此外,对于任意 $x\geq 0$,还有
  \[
    |f_n(x)|\leq (e^{x/n})^ne^{-2x}=e^{-x},
  \]
  所以可以选控制函数为 $g(x)=e^{-x}$。根据控制收敛定理,有
  \[
    \lim_{n\to\infty}\int_0^n\left(1+\frac{x}{n}\right)^ne^{-2x}\d x=
    \int_0^\infty e^{-x}\d x=1.
  \]
\end{proof}

