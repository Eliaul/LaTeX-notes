
\chapter{概率论基础}

\section{一般定义}

\subsection{概率空间}

令 $(\Omega,\mathcal{A})$ 是可测空间,$\mathbb{P}$ 是 $(\Omega,\mathcal{A})$
上的概率测度,我们说 $(\Omega,\mathcal{A},\mathbb{P})$ 是\emph{概率空间}。

因此,概率空间是测度空间的一个特例。然而,概率论的观点与测度论有很大不同。在概率论中,
我们的目标是一个“随机实验”的数学模型:
\begin{itemize}[nosep]
  \item $\Omega$ 表示实验的所有可能的结果的集合。
  \item $\mathcal{A}$ 是所有“事件”的集合。这里的事件指的是 $\Omega$ 的一个子集,其概率
  可以被计算(也就是可测集)。我们应当把事件 $A$ 视为满足某一属性的所有 $\omega\in\Omega$ 
  构成的子集。
  \item 对于每个 $A\in \mathcal{A}$,$\mathbb{P}(A)$ 表示事件 $A$ 发生的概率。
\end{itemize}

当然,一个自然的疑问是,为什么需要考虑事件域 $\mathcal{A}$?换句话说,为什么不能对
$\Omega$ 的任意子集都计算一个概率?原因在于,一般不可能在 $\Omega$ 的幂集 $\mathcal{P}(\Omega)$
上定义我们感兴趣的概率测度(除开 $\Omega$ 是可数集这一简单情况)。例如,取 $\Omega=[0,1]$,
配备 Borel $\sigma$-域和 Lebesgue 测度,但是,可以证明不可能将 Lebesgue 测度扩展到
$[0,1]$ 的任意子集上使得其仍然满足测度的定义。

\begin{example}\label{exa:dice model}
  一些常见的概率模型。
  \begin{enumerate}
    \item 考虑扔两次骰子这一实验,那么
    \[
      \Omega=\{1,2,\dots,6\}^2,\quad \mathcal{A}=\mathcal{P}(\Omega),\quad
      \mathbb{P}(A)=\frac{\card(A)}{36}.
    \]
    这里概率 $\mathbb{P}$ 的选取意味着让所有结果都有相同的概率。更一般地,如果 $\Omega$
    是有限集,$\mathcal{A}=\mathcal{P}(\Omega)$,概率测度 $\mathbb{P}(\{\omega\})=1/\card(\Omega)$
    被称为 $\Omega$ 上的\emph{均匀概率测度}。
    \item 现在我们考虑实验:扔骰子,直到出现 $6$ 为止。由于得到 $6$ 所需的投掷次数是无界的
    (即使你扔了 $1000$ 次骰子,仍有可能没有得到 $6$),所以 $\Omega$ 的正确选择是想象
    我们扔了无限次骰子:
    \[
      \Omega=\{1,2,\dots,6\}^{\mathbb{N}}.
    \]
    $\Omega$ 上的 $\sigma$-域 $\mathcal{A}$ 被定义为包含形如
    \[
      \{\omega\in\Omega\,|\, \omega_1=i_1,\dots,\omega_n=i_n\}
    \]
    的最小的 $\sigma$-域。这个形式的集合代表了“只观测有限次”的事件,
    例如假设 $n=1,i_1=6$,那么这个形式的集合包含了所有以 $6$ 开头的无限序列。
    虽然 $\omega$ 是无限长的,但是这个集合的性质只由前几项决定。
    这样的 $\sigma$-域被称为\emph{圆柱 $\sigma$-域}。
    最后,令 $\mathbb{P}$ 是有限测度,满足对于每个 $n$ 和 $i_1,\dots,i_n$
    有
    \[
      \mathbb{P}\bigl(
        \bigl\{
          \omega\in\Omega\,|\, \omega_1=i_1,\dots,\omega_n=i_n
        \bigr\}
      \bigr)=\left(\frac{1}{6}\right)^n.
    \]
    
  \end{enumerate}
\end{example}

与测度论类似,零测集也会出现在概率论的很多叙述中,如果某个
命题对于某个概率为 $1$ 的事件中的每个 $\omega\in\Omega$ 都成立,那么
我们说这个命题\emph{几乎肯定}成立,用缩写 a.s. 表示。


\subsection{随机变量}

在本章的剩余部分,我们都考虑一个概率空间 $(\Omega,\mathcal{A},\mathbb{P})$,并且所有
随机变量都将在这个概率空间上定义。

\begin{definition}
  令 $(E,\mathcal{E})$ 是可测空间,值在 $E$ 中的\emph{随机变量}指的是一个可测映射
  $X:\Omega\to E$。
\end{definition}

\begin{example}
  回顾 \eqref{exa:dice model} 中的模型。
  \begin{enumerate}
    \item $X((i,j))=i+j$ 定义了值在 $\{2,3,\dots,12\}$ 中的随机变量。
    \item $X(\omega)=\inf\{j\,|\, \omega_j=6\}$,约定 $\inf \emptyset=\infty$,
    定义了值在 $\bar{\mathbb{N}}=\mathbb{N}\cup\{\infty\}$ 中的随机变量。为了验证
    $X$ 的可测性,只需要注意到
    \[
      X^{-1}(\{k\})=\{\omega\in\Omega\,|\, \omega_1\neq 6,\dots,\omega_{k-1}\neq 6,\omega_k=6\}.
    \]
  \end{enumerate}
\end{example}

\begin{definition}
  令 $X$ 是值在 $(E,\mathcal{E})$ 中的随机变量,定义随机变量 $X$ 的
  \emph{分布律} $\mathbb{P}_X$ 是概率测度 $\mathbb{P}$ 在 $X$ 下的推前。
  也就是说,$\mathbb{P}_X$ 是 $(E,\mathcal{E})$ 上的概率测度,满足
  \[
    \mathbb{P}_X(B)=\mathbb{P}(X^{-1}(B)),\quad \forall B\in \mathcal{E}.
  \]
  两个值在 $(E,\mathcal{E})$ 中的随机变量 $Y,Y'$ 如果有相同的分布
  $\mathbb{P}_Y=\mathbb{P}_{Y'}$,那么我们说 $Y$ 和 $Y'$ 是\emph{同分布}的。
\end{definition}

在概率论中,我们通常将 $\mathbb{P}_X(B)$ 写为 $\mathbb{P}(X\in B)$
而不是 $\mathbb{P}(X^{-1}(B))$。这里 $X\in B$ 是集合
$\{\omega\in\Omega\,|\, X(\omega)\in B\}$ 的简写,这是一个一般性
的简写规则,在概率论中参数 $\omega$ 通常被隐藏。

\paragraph{离散型随机变量}
当 $E$ 是有限或者可数($\mathcal{E}=\mathcal{P}(E)$)的时候,
$X$ 的分布是点测度,这是因为
\[
  \mathbb{P}_X(B)=\mathbb{P}(X\in B)=\mathbb{P}\biggl(\bigcup_{x\in B}\{X=x\}\biggr)
  =\sum_{x\in B}\mathbb{P}(X=x)=\sum_{x\in E}p_x\delta_x(B),
\]
其中 $p_x=\mathbb{P}(X=x)$。这就表明
\[
  \mathbb{P}_X=\sum_{x\in E}p_x\delta_x 
\]
是 $E$ 上的点测度。

\begin{example}
  我们考虑 \eqref{exa:dice model} 中的第二个例子,随机变量
  为 $X(\omega)=\inf\{j\,|\,\omega_j=6\}$。那么
  \begin{align*}
    \mathbb{P}(X=k)&=\mathbb{P}
    \biggl(\bigcup_{1\leq i_1,\dots,i_k\leq 5}\{\omega\,|\, 
    \omega_1=i_1,\dots,\omega_{k-1}=i_{k-1},\omega_k=6\}\biggr)\\
    &=5^{k-1}\left(\frac{1}{6}\right)^k=\frac{1}{6}\left(\frac{5}{6}\right)^{k-1}.
  \end{align*}
  注意到
  \[
    \sum_{k=1}^\infty \mathbb{P}(X=k)=\frac{1}{6}\frac{1}{1-\frac{5}{6}}=1
  \]
  并且 $\{X=\infty\}\cup\bigcup_{k=1}^\infty \{X=k\}=\Omega$,所以
  \[
    \mathbb{P}(X=\infty)=1-\sum_{k=1}^\infty \mathbb{P}(X=k)=0,
  \]  
  但是 $\{X=\infty\}\neq\emptyset$。
\end{example}

\paragraph{具有密度的随机变量} $\mathbb{R}^d$ 上的密度函数是一个非负的 Borel 函数
$p:\mathbb{R}^d\to \mathbb{R}_+$,其满足 
\[
  \int_{\mathbb{R}^d} p(x)\d x=1.
\]
对于一个值在 $\mathbb{R}^d$ 中的随机变量 $X$,如果存在密度 $p$ 使得
\[
  \mathbb{P}_X(B)=\int_B p(x)\d x
\]
对于任意 Borel 子集 $B$ 都成立,那么我们说 $X$ 有密度函数 $p$。
换句话说,$p$ 是 $\mathbb{P}_X$ 相对于 Lebesgue 测度 $\lambda$ 的密度
(\autoref{coro:density of measure}),
也记为 $\mathbb{P}_X(\d x)=p(x)\lambda(\d x)=p(x)\d x$。
根据 Radon-Nikodym 定理,随机变量 $X$ 有密度函数的充分必要条件是
$\mathbb{P}_X$ 相对于 Lebesgue 测度 $\lambda$ 是绝对连续的。此时
$p$ 在相差一个 Lebesgue 零测集的意义下是唯一确定的。

注意到密度 $p$ 实际上是在相差一个 Lebesgue 零测集的意义下由 $\mathbb{P}_X$ 确定的。
在我们遇到的大多数例子中,$p$ 在 $\mathbb{R}^d$ 上连续,在这种情况下,$p$ 由
$\mathbb{P}_X$ 唯一确定。

在 $d=1$ 的时候,我们有
\[
  \mathbb{P}(\alpha\leq X\leq \beta)=\int_{\alpha}^\beta p(x)\d x.
\]

\subsection{数学期望}

\begin{definition}
  令 $X$ 是定义在 $(\Omega,\mathcal{A},\mathbb{P})$ 上的实随机变量,我们定义
  \[
    \mathbb{E}[X]=\int_\Omega X(\omega)\mathbb{P}(\d\omega)=\int X\d \mathbb{P},
  \]
  只要上述积分有意义,我们就说 $\mathbb{E}[X]$ 是 $X$ 的\emph{期望}。
\end{definition}

根据前面的内容,上述积分有意义的条件为下列二者之一:
\begin{itemize}[nosep]
  \item $X\geq 0$,此时 $\mathbb{E}[X]\in [0,\infty]$。
  \item $X$ 符号任意,但是 $\mathbb{E}[|X|]=\int|X| \d \mathbb{P}<\infty$。
\end{itemize}


上面的定义可以拓展到多元随机变量 $X=(X_1,\dots,X_d)\in \mathbb{R}^d$,
此时我们定义 $\mathbb{E}[X]=\bigl(\mathbb{E}[X_1],\dots,\mathbb{E}[X_d]\bigr)$。
类似的,如果 $M$ 是随机矩阵(值在实矩阵空间中的随机变量),我们可以定义
矩阵 $\mathbb{E}[M]$ 为对 $M$ 的每个分量求期望构成的矩阵。

注意到若 $X=\mathbold 1_B$,那么
\[
  \mathbb{E}[X]=\int\mathbold 1_B \d \mathbb{P}=\mathbb{P}(B).
\]

对于一些特殊的随机变量,下面的命题被频繁地使用。

\begin{proposition}
  令 $X$ 是值在 $[0,\infty]$ 中的随机变量,那么
  \[
    \mathbb{E}[X]=\int_0^\infty \mathbb{P}(X\geq x) \d x.
  \]
  令 $Y$ 是值在 $\mathbb{Z}_+$ 中的随机变量,那么
  \[
    \mathbb{E}[Y]=\sum_{k=0}^\infty k \mathbb{P}(X=k)=\sum_{k=1}^\infty \mathbb{P}(Y\geq k).
  \]
\end{proposition}
\begin{proof}
  根据 Fubini 定理,我们有
  \[
    \mathbb{E}[X]=\mathbb{E}\left[
      \int_0^\infty \mathbold 1_{\{x\leq X\}}\d x
    \right]=\int_0^\infty \mathbb{E}[\mathbold 1_{\{x\leq X\}}] \d x
    =\int_0^\infty \mathbb{P}(X\geq x) \d x.
  \]
  对于随机变量 $Y$,我们有
  \[
    \mathbb{E}[Y]=\mathbb{E}\left[
      \sum_{k=0}^\infty k\mathbold 1_{\{Y=k\}}
    \right]=\int \biggl(\sum_{k=0}^\infty k\mathbold 1_{\{Y=k\}}\biggr)\d \mathbb{P}
    =\sum_{k=0}^\infty k \mathbb{P}(Y=k).
  \]
  对于第二个等式,只需注意到
  \[
    Y=\sum_{k=1}^\infty \mathbold 1_{\{Y\geq k\}}.\qedhere
  \]
\end{proof}

下面的命题是 \autoref{prop:change variable} 的特例,由于其结果十分重要,所以我们再次叙述一遍。

\begin{proposition}\label{prop:use law to calculate exception}
  令 $X$ 是值在 $(E,\mathcal{E})$ 中的随机变量,对于任意可测函数 $f:E\to [0,\infty]$,
  我们有
  \[
    \mathbb{E}[f(X)]=\int_\Omega f(X(\omega)) \mathbb{P}(\d\omega)=\int_E f(x) \mathbb{P}_X(\d x).
  \]
\end{proposition}

如果可测函数 $f:E\to \mathbb{R}$,上面的命题在两端有意义的情况下也是成立的,
即 $\mathbb{E}[|f(X)|]<\infty$ 的时候。特别地,如果 $X$ 是实值随机变量
且使得 $\mathbb{E}[|X|]<\infty$,那么有
\[
  \mathbb{E}[X]  =\int_{\Omega} X(\omega)\mathbb{P}(\d\omega)
  =\int_{\mathbb{R}} x \mathbb{P}_X(\d x).
\]
如果 $X$ 有密度 $p$,也就是说 $\mathbb{P}_X=p\cdot \lambda$,那么还有
\[
  \mathbb{E}[X]=\int_{\mathbb{R}} x \mathbb{P}_X(\d x)=\int_{\mathbb{R}} x p(x)\d x.
\]

\autoref{prop:use law to calculate exception} 告诉我们可以使用
分布 $\mathbb{P}_X$ 来计算 $f(X)$ 的期望。实际上这个过程可以倒过来,
如果我们能找到 $E$ 上的测度 $\nu$ 使得
\[
  \mathbb{E}[f(X)]=\int f\d\nu,  
\]
其中 $f:E\to \mathbb{R}$ 是任意示性函数,此时对于任意
$E$ 的可测子集 $A$,有
\[
  \mathbb{P}_X(A)=\int \indicator{A} \d \mathbb{P}_X=\mathbb{E}[\indicator{A}(X)]=  \int \indicator{A}\d\nu=\nu(A),
\]
所以分布 $\mathbb{P}_X=\nu$。下面的命题应用了这样的思想。

\begin{proposition}\label{prop:margin pdf}
  令 $X=(X_1,\dots,X_d)$ 是值在 $\mathbb{R}^d$ 中的随机变量,假设
  $X$ 有密度 $p(x_1,\dots,x_d)$。那么,对于任意 $1\leq j\leq d$,
  $X_j$ 的密度为
  \[
    p_j(x)=\int_{\mathbb{R}^{d-1}}p(x_1,\dots,x_{j-1},x,x_{j+1},\dots,x_d)
    \d x_1\cdots \d x_{j-1}\d x_{j+1}\cdots \d x_d.  
  \]
\end{proposition}
\begin{proof}
  记 $\pi_j$ 是投影函数 $\pi_j(x_1,\dots,x_d)=x_j$。对于任意的
  Borel 函数 $f:\mathbb{R}\to \mathbb{R}_+$,根据 Fubini 定理,有
  \begin{align*}
    \mathbb{E}[f(X_j)]&=\mathbb{E}[f\circ\pi_j(X)]\\
    &=\int_{\mathbb{R}^d}f(\pi_j(x)) \mathbb{P}_X(\d x)\\
    &=\int_{\mathbb{R}^d}f(x_j)p(x_1,\dots,x_d) \d x_1\cdots\d x_d \\
    &=\int_{\mathbb{R}}f(x_j)\left(
      \int_{\mathbb{R}^{d-1}}p(x_1,\dots,x_d)\d x_1\cdots
      \d x_{j-1}\d x_{j+1}\cdots \d x_d
    \right)\d x_j\\
    &=\int_{\mathbb{R}}f(x_j)p_j(x_j)\d x_j
    =\int_{\mathbb{R}}f(x_j) \mathbb{P}_{X_j}(\d x_j),
  \end{align*}
  这就表明对于任意 Borel 子集 $A$ 有
  \[
    \mathbb{P}_{X_j}(A)=\int_A p_j(x_j)\d x_j,  
  \]
  即 $X_j$ 有密度函数 $p_j$。 
\end{proof}

如果 $X=(X_1,\dots,X_d)$ 是值在 $\mathbb{R}^d$ 中的随机变量,那么
概率测度 $\mathbb{P}_{X_j}$ 被称为 $X$ 的\emph{边缘分布},分布律
$\mathbb{P}_{X_j}$ 由 $\mathbb{P}_X$ 完全决定:$\mathbb{P}_{X_j}$
就是 $\mathbb{P}_X$ 在投影 $\pi_j$ 下的推前。需要注意反之不是正确的,
也就是说即使确定了所有的边缘分布 $\mathbb{P}_{X_1},\dots,\mathbb{P}_{X_j}$,
也不能确定 $\mathbb{P}_X$。
 

\subsection{经典分布}

本小节我们列举一些重要的概率分布。

\paragraph{离散分布}
\begin{enumerate}
  \item \emph{均匀分布}。如果 $E$ 是有限集,值在 $E$ 中的随机变量
  $X$ 如果满足
  \[
    \mathbb{P}(X=x)=\frac{1}{\card(E)},\quad \forall x\in E,  
  \]
  那么我们说 $X$ 是 $E$ 上的均匀分布。
  \item \emph{参数 $\mathbold{p\in[0,1]}$ 的 Bernoulli 分布}。如果值在
  $\{0,1\}$ 中的随机变量 $X$ 满足
  \[
    \mathbb{P}(X=1)=p,\quad \mathbb{P}(X=0)=1-p,  
  \]
  那么我们说 $X$ 是 $E$ 上参数 $p$ 的 Bernoulli 分布。
  \item \emph{二项分布 $\mathbold{\mathcal{B}(n,p)\ (n\in \mathbb{N},p\in[0,1])}$}。
  如果值在 $\{0,1,\dots,n\}$ 中的随机变量 $X$ 满足
  \[
    \mathbb{P}(X=k)=\binom{n}{k}p^k(1-p)^{n-k} ,\quad \forall k\in\{0,1,\dots,n\} ,
  \]
  那么我们说 $X$ 是 $E$ 上的二项分布。
  \item \emph{参数 $\mathbold{p\in(0,1)}$ 的几何分布}。如果
  值在 $\mathbb{Z}_+$ 中的随机变量 $X$ 使得
  \[
    \mathbb{P}(X=k)=(1-p)p^k  ,\quad k\in \mathbb{Z}_+,
  \]
  那么我们说 $X$ 是 $E$ 上参数 $p$ 的几何分布。
  \item \emph{参数 $\mathbold{\lambda>0}$ 的 Poisson 分布}。
  如果值在 $\mathbb{Z}_+$ 中的随机变量 $X$ 使得
  \[
    \mathbb{P}(X=k)=\frac{\lambda^k}{k!}e^{-\lambda},\quad\forall k\in \mathbb{Z}_+,  
  \]
  那么我们说 $X$ 是 $E$ 上参数 $\lambda$ 的 Poisson 分布。容易计算
  \[
    \mathbb{E}[X]=\sum_{k=0}^\infty k \mathbb{P}(X=k)=
    \sum_{k=1}  ^\infty \frac{\lambda^k}{(k-1)!}e^{-\lambda}=\lambda,
  \]
  Poisson 分布在实际应用中非常重要,通常被用于建模某个“罕见事件”
  在长时间段内发生的次数。准确的数学叙述是 Poisson 分布是
  二项分布的近似。对于每个 $n\geq 1$,记 $X_n$ 为服从二项分布
  $\mathcal{B}(n,p_n)$ 的随机变量,如果在 $n\to\infty$ 的时候
  有 $np_n\to\lambda$,那么对于每个 $k\in \mathbb{N}$,有
  \[
    \lim_{n\to\infty} \mathbb{P}(X_n=k)=
    \frac{\lambda^k}{k!}e^{-\lambda}.
  \]
  这可以解释为,如果每天有很小的概率 $p_n\approx\lambda/n$
  发生地震,那么地震在 $n$ 天内发生的次数将近似服从泊松分布。
\end{enumerate}

\paragraph{连续分布}
在下面的五个例子中,$X$ 都指的是一个有密度 $p$ 的实值随机变量。
\begin{enumerate}
  \item \emph{$\mathbold{[a,b]}$ 上的均匀分布}:
  \[
    p(x)=\frac{1}{b-a}\indicator{[a,b]}(x).  
  \]
  \item \emph{参数 $\mathbold{\lambda>0}$ 的指数分布}:
  \[
    p(x)=\lambda e^{-\lambda x}\indicator{\mathbb{R}_+}(x),  
  \]
  此时对于 $a\geq 0$,有
  \[
    \mathbb{P}(X\geq a)=\int_a^\infty p(x)\d x=
    e^{-\lambda a}.  
  \]
  这表明指数分布有下面的重要性质:对于 $a,b\geq 0$,有
  \begin{equation}
    \mathbb{P}(X\geq a+b)=\mathbb{P}(X\geq a)\mathbb{P}(X\geq b).
  \end{equation}
  \item \emph{Gamma 分布 $\mathbold{\Gamma(a,\lambda)\ (a>0,\lambda>0)}$}:
  \[
    p(x)=\frac{\lambda^a}{\Gamma(a)}x^{a-1}e^{-\lambda x}\indicator{\mathbb{R}_+}(x),  
  \]
  这是指数分布的推广,$a=1$ 时即指数分布。
  \item \emph{参数 $\mathbold{a>0}$ 的 Cauchy 分布}:
  \[
    p(x)=\frac{1}{\pi}\frac{a}{a^2+x^2} , 
  \]
  注意到服从 Cauchy 分布的随机变量的数学期望是不存在的,因为
  \[
    \mathbb{E}[|X|]=\int_{-\infty}^{\infty}\frac{1}{\pi}\frac{a|x|}{a^2+x^2}\d x=\infty.  
  \]
  \item \emph{正态分布 $\mathbold{\mathcal{N}(m,\sigma^2)\ (m\in \mathbb{R},\sigma>0)}$}:
  \[
    p(x)=\frac{1}{\sigma\sqrt{2\pi}}\exp\left(-\frac{(x-m)^2}{2\sigma^2}\right)  .
  \]
  正态分布与 Poisson 分布一起成为概率论中最重要的两个分布。正态分布
  的密度曲线呈著名的钟形曲线。按定义很容易验证
  \[
    m=\mathbb{E}[X],\quad \sigma^2 =\mathbb{E}[(X-m)^2]. 
  \]

  对于 $a,b\in \mathbb{R}$,考虑随机变量 $Y=aX+b$,那么
  对于任意的 Borel 函数 $f:\mathbb{R}\to \mathbb{R}_+$,有
  \begin{align*}
    \mathbb{E}[f(Y)]&=\mathbb{E}[f(aX+b)]=\int_{\mathbb{R}}f(ax+b)\mathbb{P}_X(\d x)\\
    &=\int_{\mathbb{R}}f(ax+b)p(x)\d x
    =\frac{1}{a}\int_{\mathbb{R}}f(y)p\left(\frac{y-b}{a}\right)\d y\\
    &=\int_{\mathbb{R}}f(y)\frac{1}{a}p\left(\frac{y-b}{a}\right)\d y, 
  \end{align*}
  这表明
  \[
    p(y)=\frac{1}{a\sigma\sqrt{2\pi}}\exp\left(-\frac{(y-(am+b))^2}{2(a\sigma)^2}\right),
  \]
  即 $aX+b$ 服从分布 $\mathcal{N}(am+b,a^2\sigma^2)$。
\end{enumerate}

\subsection{实值随机变量的分布函数}

令 $X$ 是实值随机变量,定义 $X$ 的\emph{分布函数}为 $F_X:\mathbb{R}\to [0,1]$,
其满足
\[
  F_X(t)=\mathbb{P}(X\leq t)=\mathbb{P}_X((-\infty,t]),\quad \forall t\in \mathbb{R}.  
\]
根据 \autoref{coro:uniqueness of measure},$F_X$ 实际上完全刻画了分布 $\mathbb{P}_X$。
确切的说,如果知道了 $F_X$,即相当于知道了所有 $\mathbb{P}_X((-\infty,t])$ 的值,
而所有区间 $(-\infty,t]$ 构成的子集族对有限交封闭,又因为 $\mathbb{P}_X$
为有限测度,所以 $\mathbb{P}_X$ 在所有区间 $(-\infty,t]$ 上的值可以完全
确定 $\mathbb{P}_X$ 在 $\mathcal{B}(\mathbb{R})$ 上的值。

显然函数 $F_X$ 是递增的、右连续的并且在 $-\infty$ 处极限为 $0$、
在 $+\infty$ 处极限为 $1$。反之,如果 $F:\mathbb{R}\to [0,1]$ 满足
上面的性质,\autoref{thm:finite measure on R} 表明存在(唯一的) $\mathbb{R}$ 上的概率测度
$\mu$ 使得 $\mu((-\infty,t])=F(t)$。即这样的函数 $F$ 总能
解释为某个实值随机变量的分布函数。

令 $F_X(a-)$ 表示 $F_X$ 在 $a\in \mathbb{R}$ 处的左极限。那么
容易验证
\begin{align*}
  \mathbb{P}(a\leq X\leq b)&=F_X(b)-F_X(a-),\\
  \mathbb{P}(a<X<b)&=F_X(b-)-F_X(a).
\end{align*}
特别的,$\mathbb{P}(X=a)=F_X(a)-F_X(a-)$。这表明 $F_X$ 的间断点的个数
恰为 $\mathbb{P}_X$ 的原子个数。

\subsection{由随机变量生成的 $\sigma$-域}

\begin{definition}
  令 $X$ 是值在 $(E,\mathcal{E})$ 中的随机变量,定义由 $X$ 生成的
  \emph{$\sigma$-域} 为
  \[
    \sigma(X)=\{X^{-1}(B)\,|\, B\in \mathcal{E}\}\subseteq \mathcal{A}.  
  \]
  换句话说,这是使得 $\omega\mapsto X(\omega)$ 可测的最小 $\sigma$-域,
  记为 $\sigma(X)$。
\end{definition}

$\sigma(X)$ 的定义可以延拓到任意族随机变量 $(X_i)_{i\in I}$ 上,
其中 $X_i$ 是值在 $(E_i,\mathcal{E}_i)$ 中的随机变量。这种情况下,
定义
\[
  \sigma\bigl(
    (X_i)_{i\in I}
  \bigr)=\sigma\left(
    \bigl\{X_i^{-1}(B_i)\bigm| B_i\in \mathcal{E}_i\bigr\}
  \right).  
\]

下面的命题表明一个实值随机变量是 $\sigma(X)$-可测的当且仅当其是 $X$
的一个可测函数。这对于研究条件非常重要。

\begin{proposition}\label{prop:sigmaX-measurable}
  令 $X$ 是值在 $(E,\mathcal{E})$ 中的随机变量,$Y$ 是实值随机变量,
  那么下面的说法等价:
  \begin{enumerate}
    \item $Y$ 是 $\sigma(X)$-可测的;
    \item 存在 $(E,\mathcal{E})$ 到 $(\mathbb{R},\mathcal{B}(\mathbb{R}))$
    的可测函数 $f$ 使得 $Y=f(X)$。
  \end{enumerate}
\end{proposition}
\begin{proof}
  $(2)\Rightarrow (1)$ 是显然的,因为 $X$ 是 $\sigma(X)$-可测的,
  所以 $Y=f(X)$ 也是 $\sigma(X)$-可测的。

  $(1)\Rightarrow (2)$ 假设 $Y$ 是 $\sigma(X)$-可测的。首先考虑 $Y$
  是简单函数,即
  \[
    Y=\sum_{i=1}^n \lambda_i \idf_{A_i},
  \]
  其中 $\lambda_i$ 是不同的实数,$A_i=Y^{-1}(\lambda_i)\in\sigma(X)$。
  那么,对于每个 $i$,存在 $B_i\in \mathcal{E}$ 使得 $A_i=X^{-1}(B_i)$。
  定义函数 $f:E\to \mathbb{R}$ 为
  \[
    f(x)=\sum_{i=1}^n \lambda_i \idf_{B_i}(x).
  \]
  显然 $f$ 是 $\mathcal{E}$-可测的函数,并且
  \[
    Y=\sum_{i=1}^n \lambda_i\idf_{A_i}=\sum_{i=1}^n \lambda_i \idf_{X^{-1}(B_i)}
    = \sum_{i=1}^n \lambda_i \idf_{B_i}\circ X= f\circ X.
  \]

  一般情况下,假设 $Y$ 是一列 $\sigma(X)$-可测的简单函数 $(Y_n)_{n\in \mathbb{N}}$
  的极限。对于每个 $n$,根据上面的讨论,存在 $\mathcal{E}$-可测函数
  $f_n:E\to \mathbb{R}$ 使得 $Y_n=f_n(X)$。对于每个 $x\in E$,定义
  \[
    f(x)=\begin{dcases}
      \lim_{n\to\infty} f_n(x),&\text{如果极限存在;}\\
      0,&\text{否则。}
    \end{dcases}
  \]
  对于每个 $\omega\in\Omega$,由于 $Y(\omega)=\lim Y_n(\omega)=\lim f_n(X(\omega))$,
  所以 $X(\omega)$ 使得上述极限存在,所以 
  $Y(\omega)=f(X(\omega))$。
\end{proof}


\section{随机变量的矩}

\subsection{矩和方差}

令 $X$ 是实值随机变量,$p\in \mathbb{N}$。定义 $X$ 的\emph{$p$-阶矩}
为 $\mathbb{E}[X^p]$,其仅在 $X\geq 0$ 或者 $\mathbb{E}[|X|^p]<\infty$
的时候有定义。如果 $X$ 满足 $\mathbb{E}[X]=0$,那么我们说 $X$
是\emph{中心化}的。

因为期望是相对于测度 $\mathbb{P}_X$ 的一种积分,所以我们有下面的结果。
如果 $X$ 是值在 $[0,\infty]$ 中的随机变量,那么我们有
\begin{itemize}[nosep]
  \item $\mathbb{E}[X]<\infty\Rightarrow X<\infty,\ \alsu{\mathbb{P}_X}$
  \item $\mathbb{E}[X]=0\Rightarrow X=0\,\ \alsu{\mathbb{P}_X}$
\end{itemize}
此外,各种极限与积分交换次序的定理也可以直接改写为期望的形式:
\begin{itemize}[nosep]
  \item \emph{单调收敛定理}。如果 $(X_n)_{n\in \mathbb{N}}$ 是一列
  值在 $[0,\infty]$ 中递增的随机变量,那么
  \[
    \ulim[n\to\infty]  \mathbb{E}[X_n]=\mathbb{E}\left[\ulim[n\to\infty]X_n\right].
  \]
  \item \emph{Fatou 引理}。如果 $(X_n)_{n\in \mathbb{N}}$ 是一列
  值在 $[0,\infty]$ 中的随机变量,那么
  \[
    \mathbb{E}[\liminf X_n]\leq \liminf \mathbb{E}[X_n].  
  \]
  \item \emph{控制收敛定理}。如果 $(X_n)_{n\in \mathbb{N}}$ 是一列
  实值随机变量,并且存在值在 $[0,\infty]$ 中的随机变量 $Z$ 使得
  \[
    |X_n|\leq Z,\quad \mathbb{E}[Z]<\infty, \quad X_n\to X,\ 
    \alsu{\mathbb{P}_X}  
  \]
  那么
  \[
    \lim_{n\to\infty}\mathbb{E}[X_n]=\mathbb{E}\left[\lim_{n\to\infty}X_n\right]=\mathbb{E}[X],
    \quad \lim_{n\to\infty}\mathbb{E}[|X_n-X|]=0.
  \]
\end{itemize}

对于每个 $p\in[1,\infty]$,考虑空间 $L^p(\Omega,\mathcal{A},\mathbb{P})$。
H\"older 不等式表明对于任意实值随机变量 $X,Y$,如果 $p,q\in (1,\infty)$
使得 $1/p+1/q=1$,那么
\[
  \mathbb{E}[|XY|]\leq \mathbb{E}[|X|^p]^{1/p}\mathbb{E}[|Y|^q]^{1/q}. 
\]
取 $Y=1$,我们得到 $\norm{X}_1\leq\norm{X}_p$。此外,
如果 $1\leq p<q\leq \infty$,有 $\norm{X}_p\leq \norm{X}_q$,
这也表明 $L^q(\Omega, \mathcal{A},\mathbb{P})\subseteq L^p(\Omega,\mathcal{A},\mathbb{P})$。

Hilbert 空间 $L^2(\Omega,\mathcal{A},\mathbb{P})$ 上的内积定义为
$\langle X,Y\rangle_{L^2}=\mathbb{E}[XY]$,Cauchy-Schwarz 不等式表明
\[
  \mathbb{E}[|XY|]\leq \mathbb{E}[X^2]  ^{1/2}\mathbb{E}[Y^2]^{1/2}.
\]
特别地,我们有
\[
  \mathbb{E}[|X|]^2\leq \mathbb{E}[X^2].  
\]

如果 $X\in L^1(\Omega,\mathcal{A},\mathbb{P})$,$f:\mathbb{R}\to \mathbb{R}_+$
是凸函数,那么 Jensen 不等式表明 
\[
  \mathbb{E}[f(X)]\geq f(\mathbb{E}[X]).
\]

\begin{definition}
  令 $X\in L^2(\Omega,\mathcal{A},\mathbb{P})$,定义 $X$
  的\emph{方差}为
  \[
    \var(X)=\mathbb{E}\bigl[(X-\mathbb{E}[X])^2\bigr]\geq 0,
  \]
  $X$ 的\emph{标准差}为
  \[
    \sigma_X=\sqrt{\var(X)}.  
  \]
\end{definition}

\begin{proposition}
  令 $X\in L^2(\Omega, \mathcal{A},\mathbb{P})$,方差
  $\var(X)=\mathbb{E}[X^2]-\bigl(\mathbb{E}[X]\bigr)^2$。
  对于任意的 $a\in \mathbb{R}$,有
  \[
    \mathbb{E}[(X-a)^2]=\var(X)+\bigl(\mathbb{E}[X]-a\bigr)  ^2.
  \]
  因此,还有
  \[
    \var(X)=\inf_{a\in \mathbb{R}} \mathbb{E}[(X-a)^2].
  \]
\end{proposition}

下面的两个不等式是非常重要的工具。

\paragraph{Markov 不等式} 如果 $X$ 是非负随机变量并且 $a>0$,那么
根据 \autoref{prop:more properties of integral of positive function},有
\[
  \mathbb{P}(X\geq a)\leq \frac{1}{a}\mathbb{E}[X].
\]

\paragraph{Bienaym\'e-Chebyshev 不等式} 如果 $X\in L^2(\Omega,\mathcal{A},\mathbb{P})$
并且 $a>0$,对 $(X-\mathbb{E}[X])^2$ 和 $a^2$ 应用 Markov 不等式,得到 
\[
  \mathbb{P}\bigl(
    \bigl|X-\mathbb{E}[X]\bigr|\geq a
  \bigr)\leq \frac{1}{a^2}\var(X).
\]

\begin{definition}
  令 $X,Y\in L^2(\Omega,\mathcal{A},\mathbb{P})$。定义 $X$ 和 $Y$ 的\emph{协方差}为
  \[
    \cov(X,Y)=\mathbb{E}\bigl[(X-\mathbb{E}[X])(Y-\mathbb{E}[Y])\bigr]
    =\mathbb{E}[XY]-\mathbb{E}[X]\mathbb{E}[Y].
  \]
  如果 $Z=(Z_1,\dots,Z_d)$ 是值在 $\mathbb{R}^d$ 中的随机变量并且所有分量
  都属于 $L^2(\Omega,\mathcal{A},\mathbb{P})$ (等价的说 $\mathbb{E}[|Z|^2]<\infty$),
  定义 $Z$ 的\emph{协方差矩阵}为
  \[
    K_Z=\Bigl(
      \cov(Z_i,Z_j)
    \Bigr)_{1\leq i,j\leq d}.
  \]
\end{definition}

直觉上,$X$ 和 $Y$ 的协方差衡量了 $X$ 和 $Y$ 之间的相关性。可以注意到
$\cov(X,X)=\var(X)$,还有 Cauchy-Schwarz 不等式表明
\[
  \left|\cov(X,Y)\right|\leq \sqrt{\var(X)}\sqrt{\var(Y)}.
\]
映射 $(X,Y)\mapsto \cov(X,Y)$ 给出了 $L^2(\Omega,\mathcal{A},\mathbb{P})$
上的一个对称的双线性映射。

对于一个随机向量 $Z=(Z_1,\dots,Z_d)$,矩阵 $K_Z$ 是对称的半正定矩阵:
对于每个 $\lambda_1,\dots,\lambda_d\in \mathbb{R}$,有
\begin{align*}
    \sum_{i,j=1}^d \lambda_i \lambda_j K_Z(i,j)
  &=\sum_{i,j=1}^d \lambda_i\lambda_j \left(
    \mathbb{E}[Z_i Z_j]-\mathbb{E}[Z_i]\mathbb{E}[Z_j]
  \right) 
  \\
  &=\mathbb{E}\biggl[
    \sum_{i,j=1}^d \lambda_i\lambda_j Z_i Z_j
  \biggr]-\sum_{i,j=1}^d \lambda_i\lambda_j \mathbb{E}[Z_i]\mathbb{E}[Z_j]\\
  &=
  \mathbb{E}\biggl[
    \biggl(
      \sum_{i=1}^d \lambda_i Z_i
    \biggr)^2
  \biggr]-\left(
    \mathbb{E}\biggl[
      \sum_{i=1}^d \lambda_i Z_i
    \biggr]
  \right)^2\\
  &=
      \var\biggl(
    \sum_{i=1}^d \lambda_i Z_i
  \biggr)\geq 0.
\end{align*}

令 $\wtilde Z=Z-\mathbb{E}[Z]$。如果我们把 $\wtilde Z$
视为列向量,那么我们可以把协方差矩阵写成 $K_Z=\mathbb{E}[\wtilde Z\wtilde Z^T]$。
因此,如果 $A$ 是一个 $n\times d$ 实矩阵,并且 $Z'=AZ$,那么
\begin{equation}
  K_{Z'}=\mathbb{E}\bigl[\wtilde Z'\wtilde Z'^T\bigr]
  =\mathbb{E}\bigl[A\wtilde Z\wtilde ZA^T\bigr]=A K_Z A^T.
\end{equation}
作为特殊情况,如果 $\xi \in \mathbb{R}^d$,点积 $\xi\cdot Z$
可以写为 $\xi^T Z$,于是
\begin{equation}
  \var(\xi\cdot Z)=\xi^T K_Z \xi.
\end{equation}

\section{线性回归}

令 $X,Y_1,\dots,Y_n$ 是 $L^2(\Omega,\mathcal{A},\mathbb{P})$ 中的实值随机变量。
记 $Y=(Y_1,\dots,Y_n)$。我们来寻找最接近 $X$ 的 $Y_1,\dots,Y_n$ 的仿射函数。
准确地说,我们需要寻找实数 $(\beta_0,\beta_1,\dots,\beta_n)$ 最小化
\[
  \mathbb{E}\bigl[
    \bigl(X-(\beta_0+\beta_1 Y_1+\cdots +\beta_n Y_n)\bigr)^2
  \bigr].
\]

\begin{proposition}
  我们有
  \[
    \inf_{\beta_0,\beta_1,\dots,\beta_n\in \mathbb{R}}
    \mathbb{E}\bigl[
      \bigl(X-(\beta_0+\beta_1 Y_1+\cdots +\beta_n Y_n)\bigr)^2
    \bigr]
    =\mathbb{E}\bigl[(X-Z)^2\bigr],
  \]
  其中
  \begin{equation}
    Z=\mathbb{E}[X]+\sum_{j=1}^n \alpha_j (Y_j-\mathbb{E}[Y_j]),
  \end{equation}
  系数 $\alpha_j$ 是下述线性方程组的任意解:
  \[
    \sum_{j=1}^n \alpha_j \cov(Y_j,Y_k)=\cov(X,Y_k),\quad 1\leq k\leq n.
  \]
  特别的,如果 $K_Y$ 是可逆的,那么我们有 $\alpha=\cov(X,Y)K_Y^{-1}$,
  这里 $\cov(X,Y)$ 表示向量 $(\cov(X,Y_j))_{1\leq j\leq n}$。
\end{proposition}

\section{特征函数}

\begin{definition}
  令 $X$ 是值在 $\mathbb{R}^d$ 中的随机变量,$t\in \mathbb{R}^d$。定义
  $X$ 的\emph{特征函数} $\varPhi_X:\mathbb{R}^d\to \mathbb{C}$ 为
  \[
    \varPhi_X(\xi)=\mathbb{E}\left[
      e^{i \xi\cdot X}
    \right]=\int_{\mathbb{R}^d} e^{i \xi\cdot x} \mathbb{P}_X(\d x).
  \]
\end{definition}

由于 $|e^{i\xi\cdot x}|=1$,所以任意随机变量 $X$ 的特征函数 $\varPhi_X$ 都是良定义的。
可以发现 $\varPhi_X$ 就是分布律 $\mathbb{P}_X$ 的 Fourier 变换,
所以我们也时常记作 $\varPhi_X(\xi)=\what{\mathbb{P}}_X(\xi)$。
利用控制收敛定理,可以验证 $\varPhi_X$ 在 $\mathbb{R}^d$ 上连续。

我们的目标是证明特征函数确定了分布律 $\mathbb{P}_X$。这等价于
说明 Fourier 变换 $\mathbb{P}_X\mapsto \what{\mathbb{P}}_X$ 是单射的映射。

\begin{lemma}
  令 $X$ 是服从正态分布 $\mathcal{N}(m,\sigma^2)$ 的实值随机变量,
  那么
  \[
    \varPhi_X(\xi)=\exp\left(
      i m \xi -\frac{\sigma^2 \xi^2}{2}
    \right),\quad \forall \xi\in \mathbb{R}.
  \]
\end{lemma}
\begin{proof}
  不妨设 $\sigma>0$ 且用 $X-m$ 替代 $X$,也即 $m=0$。那么
  \[
    \varPhi_X(\xi)=\int_{\mathbb{R}}e^{i\xi x}\frac{1}{\sigma\sqrt{2\pi}}
    \exp\left(-\frac{x^2}{2\sigma^2}\right)\d x.
  \]
  通过换元,不妨设 $\sigma=1$。并且扔掉奇函数的部分,只需要计算含参积分
  \[
    f(\xi)=\int_{\mathbb{R}}\frac{1}{\sqrt{2\pi}}e^{-x^2/2}\cos(\xi x)\d x.
  \]
  由于 $|xe^{-x^2/2}\sin(\xi x)|\leq |x| e^{-x^2/2}$ 是可积函数,所以
  \[
    f'(\xi)=-\int_{\mathbb{R}}\frac{1}{\sqrt{2\pi}}xe^{-x^2/2}\sin(\xi x)\d x.
  \]
  再通过分部积分,有
  \[
    f'(\xi)=-\xi \int_{\mathbb{R}}\frac{1}{\sqrt{2\pi}}e^{-x^2/2}\cos(\xi x)\d x
    =-\xi f(\xi).
  \]
  且初值条件 $f(0)=1$,解这个微分方程,就得到 $f(\xi)=e^{-\xi^2/2}$。
\end{proof}

\begin{theorem}
  值在 $\mathbb{R}^d$ 中的随机变量 $X$ 的特征函数确定了它的分布律。
  换句话说,$\mathbb{R}^d$ 上概率测度的 Fourier 变换算子是单射。 
\end{theorem}
\begin{proof}
  考虑 $d=1$ 的情况。对于每个 $\sigma>0$,令 $g_\sigma$ 是正态分布 $\mathcal{N}(0,\sigma^2)$
  的密度
  \[
    g_\sigma(x)=\frac{1}{\sigma\sqrt{2\pi}}\exp\left(-\frac{x^2}{2\sigma^2}\right).
  \]
  令 $\mu$ 是 $\mathbb{R}$ 上的概率测度,令 
  \begin{align*}
    f_\sigma(x)&=\int_{\mathbb{R}}g_\sigma(x-y)\mu(\d y)=g_\sigma * \mu (x),\\
    \mu_\sigma(\d x)&=f_\sigma(x)\d x.
  \end{align*}
  记 $C_b(\mathbb{R})$ 是所有有界连续函数 $\varphi:\mathbb{R}\to \mathbb{R}$
  构成的空间。
\end{proof}







