

\chapter{积测度}

\section{积 $\sigma$-域}

令 $(E,\mathcal{A})$ 和 $(F,\mathcal{B})$ 是两个可测空间。
回顾第一章,我们定义 $E\times F$ 上的乘积 $\sigma$-域
\[
  \mathcal{A}\otimes \mathcal{B}=\sigma(A\times B\,|\, A\in \mathcal{A},B\in \mathcal{B}).  
\]
不难验证 $\mathcal{A}\otimes \mathcal{B}$ 是使得两个投影映射
$\pi_1:E\times F\to E$ 和 $\pi_2:E\times F\to F$ 都可测的最小的
$\sigma$-域。

如果 $C\subseteq E\times F$,$x\in E$,记
\[
  C_x=\{y\in F\,|\, (x,y)\in C\}\subseteq F,  
\]
如果 $y\in F$,记
\[
  C^y=\{x\in E\,|\, (x,y)\in C\}  \subseteq E.
\]
如果 $f$ 是 $E\times F$ 上的函数,$x\in E$,我们记 $f_x$ 表示
$F$ 上的函数 $f_x(y)=f(x,y)$。类似地,如果 $y\in F$,我们记
$f^y(x)=f(x,y)$ 表示 $E$ 上的函数。

\begin{proposition}
  \mbox{}
  \begin{enumerate}
    \item 令 $C\in \mathcal{A}\otimes \mathcal{B}$,那么
    对于任意 $x\in E$,$C_x\in \mathcal{B}$,
    对于任意 $y\in F$,$C^y\in \mathcal{A}$。
    \item 令 $(G,\mathcal{G})$ 是可测空间,$f:E\times F\to G$
    是可测函数,那么对于任意 $x\in E$,$f_x:F\to G$ 是可测的,
    对于任意 $y\in F$,$f^y:E\to G$ 是可测的。
  \end{enumerate}
\end{proposition}
\begin{proof}
  (1) 对于 $x\in E$,令
  \[
    \mathcal{C}  =\{C\in \mathcal{A}\otimes \mathcal{B}\,|\, C_x\in \mathcal{B}\},
  \]
  那么不难验证 $\mathcal{C}$ 是一个 $\sigma$-域且包含所有的可测矩形,
  于是 $\mathcal{C}=\mathcal{A}\otimes \mathcal{B}$,
  即表明对于任意 $C\in \mathcal{A}\otimes \mathcal{B}$ 都有 $C_x\in \mathcal{B}$。
  $C^y\in \mathcal{A}$ 同理。

  (2) 对于 $x\in E$,任取 $D\in \mathcal{G}$,有
  \[
    f_x^{-1}(D)=\bigl(f^{-1}(D)\bigr)_x \in \mathcal{B}.\qedhere
  \]
\end{proof}


\section{积测度}

\begin{theorem}\label{thm:product measure}
  令 $\mu$ 和 $\nu$ 分别是 $(E,\mathcal{A})$ 和 $(F,\mathcal{B})$
  上的 $\sigma$-有限测度,那么
  \begin{enumerate}
    \item 存在唯一的 $(E\times F,\mathcal{A}\otimes \mathcal{B})$
    上的测度 $m$,使得对于每个 $A\in \mathcal{A},B\in \mathcal{B}$,都有
    \[
      m(A\times B)=\mu(A)\nu(B),  
    \]
    约定 $0\times \infty=0$。测度 $m$ 也是 $\sigma$-有限的,
    记为 $m=\mu\otimes\nu$。
    \item 对于每个 $C\in \mathcal{A}\otimes \mathcal{B}$,函数
    $x\mapsto \nu(C_x)$ 是 $\mathcal{A}$-可测的,
    $y\mapsto \mu(C^y)$ 是 $\mathcal{B}$-可测的,并且我们有
    \[
      \mu\otimes \nu(C)=\int_E \nu(C_x)\mu(\d x)=\int_F\mu(C^y)\nu(\d y).  
    \]
  \end{enumerate}
\end{theorem}
\begin{proof}
  我们首先说明这样的测度 $m$ 一定是唯一的。我们使用 \autoref{coro:uniqueness of measure}
  说明唯一性。若测度 $m'$ 也满足性质 (1)。首先所有可测矩形对有限交封闭且生成 $\mathcal{A}\otimes \mathcal{B}$,
  并且 $m$ 和 $m'$ 在可测矩形上取值相同。$\mu$ 是 $\sigma$-有限的表明存在
  递增的可测子集 $(A_n)_{n\in \mathbb{N}}$ 使得 $\bigcup_{n\in \mathbb{N}}A_n=E$
  并且 $\mu(A_n)$ 有限。同理存在递增的可测子集 $(B_n)_{n\in \mathbb{N}}$ 
  使得 $\bigcup_{n\in \mathbb{N}}B_n=F$ 并且 $\nu(B_n)$ 有限。
  令 $G_n=A_n\times B_n$,那么 $G_n\subseteq G_{n+1}$ 并且
  $E\times F=\bigcup_{n\in \mathbb{N}}G_n$,此时
  \[
    m'(G_n)=\mu(A_n)\nu(B_n)=m(G_n)<\infty,  
  \] 
  所以 \autoref{coro:uniqueness of measure} 表明 $m=m'$。

  然后我们说明存在性。对于 $C\in \mathcal{A}\otimes \mathcal{B}$,定义
  \begin{equation}\label{eq:def of product measure}
    m(C) =\int_E\nu(C_x)\mu(\d x).
  \end{equation}
  对于任意 $x\in E$,有 $C_x\in \mathcal{B}$,所以 $\nu(C_x)$
  是良好定义的。下面我们证明 $x\mapsto \nu(C_x)$ 是可测函数。
  \begin{enumerate}[label=(\arabic*)]
    \item 首先假设 $\nu$ 是有限测度。
    令 $\mathcal{G}=\{C\in \mathcal{A}\otimes \mathcal{B}\,|\, \text{$x\mapsto \nu(C_x)$ 可测}\}$。
    那么对于可测矩形 $A\times B\in \mathcal{A}\otimes \mathcal{B}$,有
    $\nu((A\times B)_x)=\indicator{A}(x)\nu(B)$,此时 $x\mapsto \indicator{A}(x)\nu(B)$
    当然是可测函数。故 $\mathcal{G}$ 包含所有的可测矩形。
    其次,我们证明 $\mathcal{G}$ 是一个单调类。如果 $C,C'\in \mathcal{G}$
    且 $C\subseteq  C'$,那么利用 $\nu$ 的有限性,就有
    \[
      \nu\bigl((C'\smallsetminus C)_x\bigr)  
      =\nu(C'_x \smallsetminus C_x)=\nu(C_x')-\nu(C_x),
    \]
    所以 $x\mapsto \nu\bigl((C'\smallsetminus C)_x\bigr)$ 是可测函数,
    即 $C' \smallsetminus C\in \mathcal{G}$。如果 $(C_n)_{n\in \mathbb{N}}$
    是 $\mathcal{G}$ 中的一个递增序列,那么
    \[
      \nu\biggl(\biggl(\bigcup_{n\in \mathbb{N}}C_n\biggr)_x\biggr)  
      =\nu\biggl(\bigcup_{n\in \mathbb{N}}(C_n)_x\biggr)
      =\ulim[n\to\infty]\nu((C_n)_x),
    \] 
    而 $x\mapsto \ulim \nu((C_n)_x)$ 是可测函数,所以 $\bigcup_{n\in \mathbb{N}}C_n\in \mathcal{G}$。
    这就表明 $\mathcal{G}$ 是单调类。由于 $\mathcal{G}$ 包含可测矩形,可测矩形对有限交封闭,
    根据单调类定理,所以 $\mathcal{G}$ 是 $\sigma$-域,所以 $\mathcal{G}=\mathcal{A}\otimes \mathcal{B}$,
    这表明对于任意 $C\in \mathcal{A}\otimes \mathcal{B}$,$x\mapsto \nu(C_x)$ 都是可测函数。
    \item 然后假设 $\nu$ 是 $\sigma$-有限测度。此时存在 $\mathcal{B}$ 的一列递增
    子集 $(B_n)_{n\in \mathbb{N}}$ 使得 $F=\bigcup_{n\in \mathbb{N}}B_n$
    以及 $\nu(B_n)<\infty$。令 $\nu_n$ 表示测度 $\nu$ 在 $B_n$ 上的限制,那么
    根据上面的叙述,任取 $C\in \mathcal{A}\otimes \mathcal{B}$,函数 $x\mapsto \nu_n(C_x)$ 是可测函数。
    注意到
    \[
      \nu(C_x)=\nu\biggl(\bigcup_{n\in \mathbb{N}}(C_x\cap B_n)\biggr)  
      =\ulim[n\to\infty] \nu_n(C_x),
    \]
    所以 $x\mapsto \ulim \nu(C_x)$ 是可测函数。
  \end{enumerate}

  于是我们证明了 $x\mapsto \nu(C_x)$ 是可测函数,这表明定义 \eqref{eq:def of product measure}
  式是有意义的。下面我们验证 $m$ 满足测度的条件。显然 $m(\emptyset)=0$。
  任取 $(C_n)_{n\in \mathbb{N}}$ 是 $\mathcal{A}\otimes \mathcal{B}$
  中的一列不相交的子集,那么
  \begin{align*}
    m\biggl(\bigcup_{n\in \mathbb{N}}C_n\biggr)&=\int_E
    \nu\biggl(\bigcup_{n\in \mathbb{N}}(C_n)_x\biggr)\mu(\d x)
    =\int_E\sum_{n\in \mathbb{N}}\nu((C_n)_x)\mu(\d x)\\
    &=\sum_{n\in \mathbb{N}}\int_E\nu((C_n)_x)\mu(\d x)
    =\sum_{n\in \mathbb{N}}m(C_n),
  \end{align*}
  其中第三个等号利用了单调收敛定理。这就表明 $m$ 确实是一个测度。

  对于可测矩形 $A\times B\in \mathcal{A}\otimes \mathcal{B}$,有
  \[
    m(A\times B)=\int_E\nu((A\times B)_x)\mu(\d x)
    =\int_E\nu(B)\indicator{A}(x)\mu(\d x)  
    =\mu(A)\nu(B),
  \]
  所以这样的 $m$ 是唯一的。对于 $C\in \mathcal{A}\otimes \mathcal{B}$,定义
  \begin{equation*}
    m'(C) =\int_F\mu(C^y)\nu(\d y),
  \end{equation*}
  重复上面的过程,可以证明 $m'$ 满足和 $m$ 相同的性质,所以 $m=m'$。
\end{proof}

\begin{example}
  如果 $(E,\mathcal{A})=(F,\mathcal{B})=(\mathbb{R},\mathcal{B}(\mathbb{R}))$,
  并且 $\mu=\nu=\lambda$,可以验证 $\lambda\otimes\lambda$ 就是 $\mathbb{R}^2$
  上的 Lebesgue 测度 (只需要在矩形 $[a,b]\times [c,d]$ 上验证,然后利用 \autoref{coro:uniqueness of measure})。
\end{example}

\section{Fubini 定理}

考虑可测空间 $(E,\mathcal{A})$ 和 $(F,\mathcal{B})$。

\begin{theorem}[Fubini-Tonelli]\label{thm:Fubini-Tonelli}
  令 $\mu$ 和 $\nu$ 分别是 $(E,\mathcal{A})$ 和 $(F,\mathcal{B})$ 上的两个
  $\sigma$-有限的测度。令 $f:E\times F\to [0,\infty]$ 是可测函数。
  \begin{enumerate}
    \item 函数
    \[
      E\ni x\mapsto \int_F f(x,y)\nu(\d y),\quad 
      F\ni y\mapsto \int_E f(x,y)\mu(\d x)  
    \]
    是值在 $[0,\infty]$ 中的可测函数。
    \item 我们有
    \[
      \int_{E\times F}f\d\mu\otimes\nu 
      =\int_E\left(\int_F f(x,y)\nu(\d y)\right)\mu(\d x)=\int_F
      \left(\int_E f(x,y)\mu(\d x)\right)\nu(\d y).
    \]
  \end{enumerate}
\end{theorem}
\begin{proof}
  (1) 设 $f=\ulim f_n$,$(f_n)_{n\in \mathbb{N}}$ 是一列递增的非负简单函数,
  那么根据单调收敛定理,有
  \[
    \int_F f(x,y)\nu(\d y)=\ulim[n\to\infty]\int_Ff_n(x,y)\nu(\d y)  ,
  \]
  所以我们只需要说明对于任意非负简单函数 $h$,$x\mapsto \int_F h(x,y)\nu(\d y)$
  可测即可。对于示性函数 $\indicator{C}$,有
  $\int_F \indicator{C}(x,y) \nu(\d y)=\nu(C_x)$,\autoref{thm:product measure}
  表明 $x\mapsto \nu(C_x)$ 是可测的,再根据线性性,这就说明了
  $x\mapsto \int_F h(x,y)\nu(\d y)$ 可测。
  对于 $x\mapsto\int_E f(x,y)\mu(\d x)$ 同理。

  (2) 设 $f=\ulim f_n$,$(f_n)_{n\in \mathbb{N}}$ 是一列递增的非负简单函数,
  那么根据单调收敛定理,有
  \[
    \int_{E\times F} f\d\mu\otimes \nu
    =\ulim[n\to\infty]\int_{E\times F}f_n(x,y)\nu(\d y),  
  \]
  所以只需要证明结论对于非负简单函数成立即可。根据线性性,只需要证明结论对
  示性函数成立即可。任取示性函数 $\indicator{C}$,根据 \autoref{thm:product measure},有
  \[
    \int_{E\times F} \indicator{C}\d\mu\otimes\nu=
    \mu\otimes \nu(C)=\int_E\nu(C_x)\mu(\d x)  
    =\int_E\left(\int_F \indicator{C}(x,y)\nu(\d y)\right)\mu(\d x),
  \]
  另一个等式同理。
\end{proof}

\autoref{thm:Fubini-Tonelli} 也可以推广到任意符号的版本。

\begin{theorem}[Fubini-Lebesgue]
  令 $f\in \mathcal{L}^1(E\times F,\mathcal{A}\otimes \mathcal{B},\mu\otimes\nu)$,那么
  \begin{enumerate}
    \item $\alev{\mu(\d x)}$,函数 $y\mapsto f(x,y)$ 属于 $\mathcal{L}^1(F,\mathcal{B},\nu)$。
    $\alev{\nu(\d y)}$,函数 $x\mapsto f(x,y)$ 属于 $\mathcal{L}^1(E,\mathcal{A},\mu)$。
    \item 函数 $x\mapsto \int_F f(x,y)\nu(\d y)$ 属于 $\mathcal{L}^1(E,\mathcal{A},\mu)$。
    函数 $y\mapsto \int_E f(x,y)\mu(\d x)$ 属于 $\mathcal{L}^1(F,\mathcal{B},\nu)$。
    \item 我们有
    \[
      \int_{E\times F}f\d\mu\otimes\nu 
      =\int_E\left(\int_F f(x,y)\nu(\d y)\right)\mu(\d x)=\int_F
      \left(\int_E f(x,y)\mu(\d x)\right)\nu(\d y).
    \]
  \end{enumerate}
\end{theorem}
\begin{remark}
  (1) 函数 $x\mapsto \int_F f(x,y)\nu (\d y)$ 仅仅在某个 $\mu$-零测集之外有定义,
  在这个零测集上,我们通常让这个函数为零。
  (2) 对于 $f\in \mathcal{L}_{\mathbb{C}}^1(E\times F,\mathcal{A}\otimes \mathcal{B},\mu\otimes\nu)$
  有同样的结论。
\end{remark}
\begin{proof}
  (1) 将 \autoref{thm:Fubini-Tonelli} 应用于 $|f|$,有
  \[
    \int_E \left(\int_F |f(x,y)| \nu(\d y)\right) \mu(\d x) 
    = \int_{E\times F} |f| \d(\mu\otimes \nu) < \infty.
  \]
  这表明 $\int_F |f(x,y)|\nu(\d y) <\infty,\ \alev{\mu}$。令
  \[
    N=\left\{
      x\in E\,\middle|\, \int_F |f(x,y)|\nu(\d y)=\infty
    \right\}
  \]
  是 $\mathcal{A}$-可测集,那么 $\mu(N)=0$。对于任意 $x\in E\setminus N$,
  函数 $y\mapsto f(x,y)$ 属于 $\mathcal{L}^1(F,\mathcal{B},\nu)$。
  对于 $x\mapsto f(x,y)$ 有同样的论述。

  (2) 如果 $x\in N^c$,那么 
  \begin{align*}
        \int_E\left|
      \int_F f(x,y)\nu(\d y)
    \right|\mu(\d x)&\leq     \int_E \left(\int_F |f(x,y)| \nu(\d y)\right) \mu(\d x) \\
    &= \int_{E\times F} |f| \d(\mu\otimes \nu) < \infty.
  \end{align*}
  如果 $x\in N$,注意我们定义 $\int_F f(x,y)\nu(\d y)=0$。此时
  \[
    x\mapsto \int_F f(x,y)\nu(\d y)=
    \idf_{N^c}(x)\int_F f^+(x,y)\nu(\d y)
    -\idf_{N^c}(x)\int_F f^-(x,y)\nu(\d y),
  \]
  这个函数是 $\mathcal{A}$-可测的,并且同样有上面的不等式,所以 
  $x\mapsto \int_F f(x,y)\nu(\d y)$ 在 $\mathcal{L}^1(E,\mathcal{A},\mu)$
  中。 对于 $y\mapsto \int_E f(x,y)\mu(\d x)$ 有同样的论述。

  (3) 将 \autoref{thm:Fubini-Tonelli} 应用于 $f^+$ 和 $f^-$ 即可。
\end{proof}
