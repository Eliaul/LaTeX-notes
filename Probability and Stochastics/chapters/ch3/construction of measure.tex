\chapter{测度的构造}

\section{外测度}

\begin{definition}
  令 $E$ 是集合,映射 $\mu^*:\mathcal{P}(E)\to [0,\infty]$ 如果满足:
  \begin{enumerate}
    \item  $\mu^*(\emptyset)=0$;
    \item $A\subseteq B\Rightarrow \mu^*(A)\leq\mu^*(B)$;
    \item ($\sigma$-次可加性) 对于 $\mathcal{P}(E)$ 中的一列子集 $(A_k)_{k\in \mathbb{N}}$,
    有
    \[
      \mu^*\biggl(\bigcup_{k\in \mathbb{N}}A_k \biggr)  \leq
      \sum_{k\in \mathbb{N}}\mu^*(A_k).
    \]
  \end{enumerate}
  那么我们说 $\mu^*$ 是一个\emph{外测度}。
\end{definition}

外测度的要求不如测度严格,首先 $\sigma$-可加性被替换为 $\sigma$-次可加性,
其次外测度是在幂集 $\mathcal{P}(E)$ 上定义的,而测度只能在
$\sigma$-域上定义。

我们本节的目标是从外测度 $\mu^*$ 开始,在某个 $\sigma$-域 $\mathcal{M}(\mu^*)$
上构造一个测度。从现在开始,我们固定一个外测度 $\mu^*$。

\begin{definition}
  对于 $E$ 的子集 $B$,如果任取 $A\subseteq E$,都有
  \[
    \mu^*(A)=\mu^*(A\cap B)+\mu^*(A\cap B^c),  
  \]
  那么我们说 $B$ 是\emph{$\mathbold{\mu^*}$-可测的}。用
  $\mathcal{M}(\mu^*)$ 表示所有 $\mu^*$-可测的子集构成的子集族。
\end{definition}

\begin{remark}
  根据 $\sigma$-次可加性,总是有
  \[
    \mu^*(A)\leq \mu^*(A\cap B)+\mu^*(A\cap B^c),
  \]
  所以要验证子集 $B$ 是 $\mu^*$-可测的,只需要说明反向的不等式即可。
\end{remark}

\begin{theorem}\label{thm:outer measure}
  \mbox{}
  \begin{enumerate}
    \item $\mathcal{M}(\mu^*)$ 是 $\sigma$-域,并且其包含
    所有的满足 $\mu^*(B)=0$ 的子集 $B\subseteq E$。
    \item $\mu^*$ 在 $\mathcal{M}(\mu^*)$ 上的限制是一个测度。
  \end{enumerate}
\end{theorem}


\section{Lebesgue 测度}

对于任意子集 $A\subseteq \mathbb{R}$,定义
\[
  \lambda^*(A)=\inf\bigg\{\sum_{i\in \mathbb{N}}(b_i-a_i)\,|\, A\subseteq \bigcup_{i\in \mathbb{N}}(a_i,b_i)\bigg\}  .
\]
注意这个下确界的取值范围为 $[0,\infty]$:如果 $A$ 无界,那么
将会得到 $\infty$。

\begin{theorem}
  \mbox{}
  \begin{enumerate}
    \item $\lambda^*$ 是 $\mathbb{R}$ 上的一个外测度。
    \item $\sigma$-域 $\mathcal{M}(\lambda^*)$ 包含 $\mathcal{B}(\mathbb{R})$。
    \item 对于任意实数 $a\leq b$,$\lambda^*([a,b])=\lambda^*((a,b))=b-a$。
  \end{enumerate}
\end{theorem}
\begin{proof}
  (1) 显然 $\lambda^*(\emptyset)=0$ 并且 $A\subseteq B$ 表明 $\lambda^*(A)\leq\lambda^*(B)$。
  下面证明 $\sigma$-次可加性。任取 $\mathbb{R}$ 的一列子集 $(A_n)_{n\in \mathbb{N}}$,
  不妨假设每个 $\lambda^*(A_n)<\infty$。任取 $\varepsilon>0$,对于每个 $A_n$,都存在
  一列开区间 $\bigl(a_i^{(n)},b_i^{(n)}\bigr)$ 使得
  \[
    \lambda^*(A_n)\leq \sum_{i\in \mathbb{N}}\bigl(b_i^{(n)}-a_i^{(n)}\bigr)
    <\lambda^*(A_n)+\frac{\varepsilon}{2^n},
  \]
  注意到所有的开区间 $\bigl(a_i^{(n)},b_{i}^{(n)}\bigr)\ (i,n\in \mathbb{N})$
  构成了 $\bigcup_{n\in \mathbb{N}}A_n$ 的一个可数开覆盖,所以
  \[
    \lambda^*\biggl(\bigcup_{n\in \mathbb{N}}A_n\biggr)\leq
    \sum_{n\in \mathbb{N}}\sum_{i\in \mathbb{N}}\bigl(b_i^{(n)}-a_i^{(n)}\bigr)
    \leq \sum_{n\in \mathbb{N}}\lambda^*(A_n)+\varepsilon,
  \]
  由于 $\varepsilon$ 的任意性,所以 $\lambda^*$ 满足 $\sigma$-次可加性。

  (2) 
\end{proof}

