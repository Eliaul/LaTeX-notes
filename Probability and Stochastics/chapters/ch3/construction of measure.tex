\chapter{测度的构造}

\section{外测度}

\begin{definition}
  令 $E$ 是集合,映射 $\mu^*:\mathcal{P}(E)\to [0,\infty]$ 如果满足:
  \begin{enumerate}
    \item  $\mu^*(\emptyset)=0$;
    \item $A\subseteq B\Rightarrow \mu^*(A)\leq\mu^*(B)$;
    \item ($\sigma$-次可加性) 对于 $\mathcal{P}(E)$ 中的一列子集 $(A_k)_{k\in \mathbb{N}}$,
    有
    \[
      \mu^*\biggl(\bigcup_{k\in \mathbb{N}}A_k \biggr)  \leq
      \sum_{k\in \mathbb{N}}\mu^*(A_k).
    \]
  \end{enumerate}
  那么我们说 $\mu^*$ 是一个\emph{外测度}。
\end{definition}

外测度的要求不如测度严格,首先 $\sigma$-可加性被替换为 $\sigma$-次可加性,
其次外测度是在幂集 $\mathcal{P}(E)$ 上定义的,而测度只能在
$\sigma$-域上定义。

我们本节的目标是从外测度 $\mu^*$ 开始,在某个 $\sigma$-域 $\mathcal{M}(\mu^*)$
上构造一个测度。从现在开始,我们固定一个外测度 $\mu^*$。

\begin{definition}
  对于 $E$ 的子集 $B$,如果任取 $A\subseteq E$,都有
  \[
    \mu^*(A)=\mu^*(A\cap B)+\mu^*(A\cap B^c),  
  \]
  那么我们说 $B$ 是\emph{$\mathbold{\mu^*}$-可测的}。用
  $\mathcal{M}(\mu^*)$ 表示所有 $\mu^*$-可测的子集构成的子集族。
\end{definition}

\begin{remark}
  根据 $\sigma$-次可加性,总是有
  \[
    \mu^*(A)\leq \mu^*(A\cap B)+\mu^*(A\cap B^c),
  \]
  所以要验证子集 $B$ 是 $\mu^*$-可测的,只需要说明反向的不等式即可。
\end{remark}

外测度的定义来源于 $\mathbb{R}$ 中定义集合长度的想法,也即用一堆开区间
覆盖这个集合,然后取这些开区间长度之和的下确界,所以叫“外”测度,有
从外面逼近的含义。$\mu^*$-可测的定义看似奇怪,其实可以把 $A$ 理解为一个
“测试集”,用集合 $B$ 把 $A$ 切成两部分 $A\cap B$ 和 $A\cap B^c$,如果
这两部分的测度之和等于 $A$ 的测度,那么就说 $B$ 是可测的,也就是比较好的集合。
如果 $B$ 的形状非常复杂,那么切出来的两部分可能会产生很多复杂的碎片,
导致体积之和增大,这时说 $B$ 是不可测的。




\begin{theorem}\label{thm:outer measure}
  \mbox{}
  \begin{enumerate}
    \item $\mathcal{M}(\mu^*)$ 是 $\sigma$-域,并且其包含
    所有的满足 $\mu^*(B)=0$ 的子集 $B\subseteq E$。
    \item $\mu^*$ 在 $\mathcal{M}(\mu^*)$ 上的限制是一个测度。
  \end{enumerate}
\end{theorem}
\begin{proof}
  (1) 如果 $\mu^*(B)=0$,那么对于任意 $A\subseteq E$,由于
  $A\cap B\subseteq B$,所以 $\mu^*(A\cap B)\leq \mu^*(B)=0$,所以
  \[
    \mu^*(A\cap B)+\mu^*(A\cap B^c)\leq \mu^*(A\cap B)+\mu^*(A)=\mu^*(A),
  \]
  这就表明 $B\in \mathcal{M}(\mu^*)$。

  显然 $\emptyset\in \mathcal{M}(\mu^*)$,并且 $B\in \mathcal{M}(\mu^*)$
  蕴含 $B^c\in \mathcal{M}(\mu^*)$。下面证明 $\mathcal{M}(\mu^*)$ 对
  可数并运算封闭。我们首先证明 $\mathcal{M}(\mu^*)$ 对有限并封闭。令
  $B_1,B_2\in \mathcal{M}(\mu^*)$,任取 $A\subseteq E$,则 $B_1\in \mathcal{M}(\mu^*)$
  表明
  \begin{align*}
    \mu^*(A\cap (B_1\cup B_2))&=\mu^*(A\cap(B_1\cup B_2)\cap B_1)+\mu^*(A\cap(B_1\cup B_2)\cap B_1^c) \\
    &=\mu^*(A\cap B_1)+\mu^*(A\cap B_2\cap B_1^c),
  \end{align*}
  再利用 $B_2\in \mathcal{M}(\mu^*)$,就得到
  \begin{align*}
    &\quad \mu^*(A\cap (B_1\cup B_2))+\mu^*(A\cap (B_1\cup B_2)^c)\\
    &=\mu^*(A\cap B_1)+\mu^*(A\cap B_1^c\cap B_2)+\mu^*(A\cap B_1^c\cap B_2^c) \\
    &=\mu^*(A\cap B_1)+\mu^*(A\cap B_1^c) \\
    &=\mu^*(A).
  \end{align*}
  所以 $B_1\cup B_2\in \mathcal{M}(\mu^*)$。因为 $\mathcal{M}(\mu^*)$ 对有限并
  和补运算封闭,所以对有限交也封闭。因此,如果 $B,B'\in \mathcal{M}(\mu^*)$,
  那么 $B'\setminus B=B'\cap B^c\in \mathcal{M}(\mu^*)$。

  由于 $\mathcal{M}(\mu^*)$ 对差集运算封闭,所以要证明可数并封闭,只需要
  证明 $\mathcal{M}(\mu^*)$ 中一列不相交的集合 $(B_k)_{k\in \mathbb{N}}$ 的并在 $\mathcal{M}(\mu^*)$ 中。
  采用归纳法。对于任意整数 $m\in \mathbb{N}$ 和任意 $A\subseteq E$,我们证明 
  \begin{equation}\label{eq:outer measure tmp1}
    \mu^*(A)=\sum_{k=1}^m\mu^*(A\cap B_k)+\mu^*\biggl(A\cap \bigcap_{k=1}^m B_k^c\biggr).
  \end{equation}
  当 $m=1$ 时,这就是 $B_1\in \mathcal{M}(\mu^*)$ 的定义。假设 $m$ 步成立,
  $B_k$ 互不相交表明对于 $1\leq k\leq m$ 有 $B_{m+1}\subseteq B_k^c$,那么
  \begin{align*}
    \mu^*\biggl(
      A\cap \bigcap_{k=1}^m B_k^c
    \biggr)&=\mu^*\biggl(
      A\cap \bigcap_{k=1}^m B_k^c \cap B_{m+1}
    \biggr)+\mu^*\biggl(
      A\cap \bigcap_{k=1}^m B_k^c \cap B_{m+1}^c
    \biggr) \\
    &=\mu^*(A\cap B_{m+1})+\mu^*\biggl(
      A\cap \bigcap_{k=1}^{m+1} B_k^c
    \biggr).
  \end{align*}
  再根据归纳假设,就得到式 \eqref{eq:outer measure tmp1} 对 $m+1$ 也成立。
  利用式 \eqref{eq:outer measure tmp1},对于任意 $A\subseteq E$,都有
  \[
    \mu^*(A)\geq \sum_{k=1}^m \mu^*(A\cap B_k)+
    \mu^*\biggl(
      A\cap \bigcap_{k=1}^\infty B_k^c
    \biggr).
  \]
  然后令 $m\to\infty$,就得到
  \begin{align}
    \mu^*(A)&\geq \sum_{k=1}^\infty \mu^*(A\cap B_k)+
    \mu^*\biggl(
      A\cap \bigcap_{k=1}^\infty B_k^c
    \biggr) \label{eq:outer measure tmp2} \\
    &\geq \mu^*\biggl(
      A\cap \bigcup_{k=1}^\infty B_k
    \biggr)+\mu^*\biggl(
      A\cap \bigcap_{k=1}^\infty B_k^c
    \biggr),\notag
  \end{align}
  这就表明 $\bigcup_{k=1}^\infty B_k\in \mathcal{M}(\mu^*)$。

  (2) 显然 $\mu^*(\emptyset)=0$。下面证明 $\sigma$-可加性。令
  $(B_k)_{k\in \mathbb{N}}$ 是 $\mathcal{M}(\mu^*)$ 中的一列互不相交的集合,
  那么根据 \eqref{eq:outer measure tmp2},取 $A=\bigcup_{k\in \mathbb{N}}B_k$,就得到
  \[
    \mu^*\biggl(\bigcup_{k\in \mathbb{N}}B_k\biggr)\geq
    \sum_{k=1}^\infty\mu^*(B_k).
  \]
  结合外测度的 $\sigma$-次可加性,就得到等号成立,所以 $\mu^*$ 在
  $\mathcal{M}(\mu^*)$ 上是一个测度。
\end{proof}


\section{Lebesgue 测度}

对于任意子集 $A\subseteq \mathbb{R}$,定义
\[
  \lambda^*(A)=\inf\bigg\{\sum_{i\in \mathbb{N}}(b_i-a_i)\,|\, A\subseteq \bigcup_{i\in \mathbb{N}}(a_i,b_i)\bigg\}  .
\]
注意这个下确界的取值范围为 $[0,\infty]$:如果 $A$ 无界,那么
将会得到 $\infty$。

\begin{theorem}\label{thm:lebesgue measure}
  \mbox{}
  \begin{enumerate}
    \item $\lambda^*$ 是 $\mathbb{R}$ 上的一个外测度。
    \item $\sigma$-域 $\mathcal{M}(\lambda^*)$ 包含 $\mathcal{B}(\mathbb{R})$。
    \item 对于任意实数 $a\leq b$,$\lambda^*([a,b])=\lambda^*((a,b))=b-a$。
  \end{enumerate}
\end{theorem}
\begin{remark}
  $\lambda^*$ 在 $\mathcal{B}(\mathbb{R})$ 上的限制被称为 $\mathbb{R}$
  上的\emph{Lebesgue 测度},记为 $\lambda$。根据单调类定理的推论 
  \ref{coro:uniqueness of measure},这是 $\mathcal{B}(\mathbb{R})$ 上
  唯一一个满足 $\lambda([a,b])=b-a$ 的测度。
\end{remark}

\begin{proof}
  (1) 显然 $\lambda^*(\emptyset)=0$ 并且 $A\subseteq B$ 表明 $\lambda^*(A)\leq\lambda^*(B)$。
  下面证明 $\sigma$-次可加性。任取 $\mathbb{R}$ 的一列子集 $(A_n)_{n\in \mathbb{N}}$,
  不妨假设每个 $\lambda^*(A_n)<\infty$。任取 $\varepsilon>0$,对于每个 $A_n$,都存在
  一列开区间 $\bigl(a_i^{(n)},b_i^{(n)}\bigr)$ 使得
  \[
    \lambda^*(A_n)\leq \sum_{i\in \mathbb{N}}\bigl(b_i^{(n)}-a_i^{(n)}\bigr)
    <\lambda^*(A_n)+\frac{\varepsilon}{2^n},
  \]
  注意到所有的开区间 $\bigl(a_i^{(n)},b_{i}^{(n)}\bigr)\ (i,n\in \mathbb{N})$
  构成了 $\bigcup_{n\in \mathbb{N}}A_n$ 的一个可数开覆盖,所以
  \[
    \lambda^*\biggl(\bigcup_{n\in \mathbb{N}}A_n\biggr)\leq
    \sum_{n\in \mathbb{N}}\sum_{i\in \mathbb{N}}\bigl(b_i^{(n)}-a_i^{(n)}\bigr)
    \leq \sum_{n\in \mathbb{N}}\lambda^*(A_n)+\varepsilon,
  \]
  由于 $\varepsilon$ 的任意性,所以 $\lambda^*$ 满足 $\sigma$-次可加性。

  (2) 因为 $\mathcal{M}(\lambda^*)$ 是一个 $\sigma$-域,所以只需要证明其
  包含生成 Borel $\sigma$-域的某个集合族即可,我们选择所有区间 $(-\infty,\alpha]$。
  固定 $\alpha\in \mathbb{R}$,记 $B=(-\infty,\alpha]$。我们需要证明
  对于任意 $A\subseteq \mathbb{R}$,都有
  \[
    \lambda^*(A)\geq \lambda^*(A\cap B)+\lambda^*(A\cap B^c).
  \]
  令 $\bigl((a_i,b_i)\bigr)_{i\in \mathbb{N}}$ 是 $A$ 的一个开覆盖,$\varepsilon>0$。
  那么区间 $\bigl(a_i\wedge\alpha,(b_i\wedge\alpha)+\varepsilon 2^{-i}\bigr)$
  覆盖 $A\cap B$,并且区间 $(a_i\vee \alpha,b_i\vee\alpha)$ 覆盖 $A\cap B^c$,所以
  \begin{align*}
    \lambda^*(A\cap B)&\leq \sum_{i\in \mathbb{N}}
    \bigl((b_i\wedge\alpha)-(a_i\wedge\alpha)\bigr)+\varepsilon,\\
    \lambda^*(A\cap B^c)&\leq \sum_{i\in \mathbb{N}}
    \bigl((b_i\vee\alpha)-(a_i\vee\alpha)\bigr).
  \end{align*}
  将上面两式相加,就有
  \[
    \lambda^*(A\cap B)+\lambda^*(A\cap B^c)\leq
    \sum_{i\in \mathbb{N}}(b_i-a_i)+\varepsilon.
  \]
  因为 $\varepsilon$ 的任意性,就有
    \[
    \lambda^*(A\cap B)+\lambda^*(A\cap B^c)\leq
    \sum_{i\in \mathbb{N}}(b_i-a_i).
  \]
  由于 $\lambda^*(A)$ 是所有开覆盖长度之和的下确界,所以就得到需要的结论。

  (3) 根据定义,立马得到
  \[
    \lambda^*([a,b])\leq b-a.
  \]
  假设 $[a,b]\subseteq \bigcup_{i\in \mathbb{N}}(a_i,b_i)$。根据紧性,
  存在 $N$ 使得 $[a,b]\subseteq \bigcup_{i=1}^N (a_i,b_i)$。
  那么
  \[
    b-a\leq \sum_{i=1}^N (b_i-a_i)\leq \sum_{i\in \mathbb{N}}(b_i-a_i).
  \]
  这就表明 $b-a\leq \lambda^*([a,b])$。最后,注意到 $\lambda^*(\{a\})=\lambda^*(\{b\})=0$
  就得到 $\lambda^*((a,b))=\lambda^*([a,b])=b-a$。
\end{proof}

下面我们将 Lebesgue 测度推广到 $\mathbb{R}^d$ 上。我们说 $\mathbb{R}^d$
中的一个开矩形 $P$ 是形如以下形式的集合:
\[
  P=(a_1,b_1)\times (a_2,b_2)\times \cdots \times (a_d,b_d),
\]
对应的,如果每个区间都是闭区间,那么我们说 $P$ 是一个闭矩形。
定义 $P$ 的体积为
\[
  \vol(P)=\prod_{i=1}^d (b_i-a_i).
\]
然后,对于每个子集 $A\subseteq \mathbb{R}^d$,定义
\[
  \lambda^*(A)=\inf\bigg\{\sum_{i\in \mathbb{N}}\vol(P_i)\,|\, A\subseteq \bigcup_{i\in \mathbb{N}}P_i\bigg\},
\]
即考虑所有覆盖 $A$ 的开矩形体积之和的下确界。
现在我们可以推广 \autoref{thm:lebesgue measure} 的结论到 $\mathbb{R}^d$ 上。

\begin{theorem}
  \mbox{}
  \begin{enumerate}
    \item $\lambda^*$ 是 $\mathbb{R}^d$ 上的一个外测度。
    \item $\sigma$-域 $\mathcal{M}(\lambda^*)$ 包含 $\mathcal{B}(\mathbb{R}^d)$。
    \item 对于任意开或者闭矩形 $P$,有 $\lambda^*(P)=\vol(P)$。
  \end{enumerate}
\end{theorem}
\begin{proof}
  (1) 这一点和 $d=1$ 时的证明完全相同。

  (2) 只需要证明,如果 $A\subseteq \mathbb{R}^d$ 形如
  \[
    A=\mathbb{R}\times \cdots\times \mathbb{R}\times (-\infty,a]
    \times \mathbb{R}\times \cdots\times \mathbb{R},
  \]
  其中 $a\in \mathbb{R}$,那么 $A\in \mathcal{M}(\lambda^*)$。证明和
  $d=1$ 时的做法也完全类似。

  (3) 根据定义,立马得到 $\lambda^*(P)\leq \vol(P)$。下面证明反向不等式。
  同样利用 $P$ 的紧性,只需要证明:如果 $P$ 是闭矩形,并且
  $ P\subseteq \bigcup_{i=1}^n P_i$,
  其中 $P_i$ 是开矩形,那么 
  \[
    \vol(P)\leq \sum_{i=1}^n\vol(P_i).
  \]
  记
  \[
    C_{n}^{k_1\dots k_n}=[k_12^{-n},(k_1+1)2^{-n}]\times \cdots \times
    [k_d2^{-n},(k_d+1)2^{-n}],\quad k_1,\ldots,k_d\in \mathbb{Z}.
  \]
  那么 $\{C_n^{k_1\dots k_d}\,|\, k_1,\ldots,k_d\in \mathbb{Z}\}$ 构成了
  $\mathbb{R}^d$ 的一个划分,并且每个 $C_n^{k_1\dots k_d}$ 的体积为 $2^{-nd}$。
  固定 $n$,假设 $P$ 和 $c_n$ 个 $C_n^{k_1\dots k_d}$ 相交,
  $\bigcup_{i=1}^n P_i$ 和 $c_n'$ 个 $C_n^{k_1\dots k_d}$ 相交,那么
  $c_n\leq c_n'$,所以
  \[
    c_n 2^{-nd}\leq c_n' 2^{-nd}.
  \]
  取 $n\to\infty$,就得到 $\vol(P)\leq \sum_{i=1}^n\vol(P_i)$。
\end{proof}

通常,我们用
\[
  \int_{\mathbb{R}^d} f(x)\d x =\int_{\mathbb{R}^d} f(x)\lambda(\d x)
\]
表示 $f$ 相对于 Lebesgue 测度的积分。当 $d=1$ 以及 $a\leq b$ 时,
我们也用
\[
  \int_a^b f(x)\d x =\int_{[a,b]} f(x)\lambda(\d x).
\]

一个自然的问题是 $\mathcal{M}(\lambda^*)$ 比 $\mathcal{B}(\mathbb{R}^d)$
大多少。从某种意义上,这两个 $\sigma$-域的差距不算很大。

\begin{proposition}
  令 $(E,\mathcal{A},\mu)$ 是测度空间。定义 $\mu$-可忽略集的集合为
  \[
    \mathcal{N}=\bigl\{A\in \mathcal{P}(E)\bigm| \exists B\in \mathcal{A},A\subseteq B\text{\ and\ }\mu(B)=0\bigr\}.
  \]
  我们把包含 $\mathcal{A}$ 和 $\mathcal{N}$ 的最小 $\sigma$-域称为 $\mathcal{A}$
  相对于 $\mu$ 的\emph{完备化},记作 $\bar{\mathcal{A}}$。
  此时,$(E,\bar{\mathcal{A}})$ 上存在唯一的测度使得其在 $\mathcal{A}$ 上的限制等于 $\mu$。
\end{proposition}
\begin{proof}
  令
  \[
    \mathcal{B}=\bigl\{
      A\in \mathcal{P}(E)\bigm| \exists B,B'\in \mathcal{A},
      B\subseteq A\subseteq B',\ \mu(B'\setminus B)=0
    \bigr\}.
  \]
  首先我们证明 $\bar{\mathcal{A}}=\mathcal{B}$。
  首先,直接验证可知 $\mathcal{B}$ 是一个 $\sigma$-域。显然还有 $\mathcal{A}\subseteq \mathcal{B}$
  以及 $\mathcal{N}\subseteq \mathcal{B}$,所以 $\bar{\mathcal{A}}\subseteq \mathcal{B}$。
  反之,任取 $A\in \mathcal{B}$,那么存在 $B,B'\in \mathcal{A}$ 使得
  $B\subseteq A\subseteq B'$ 且 $\mu(B'\setminus B)=0$。
  此时 $A\setminus B\subseteq B'\setminus B$ 且 $\mu(B'\setminus B)=0$,
  所以 $A\setminus B\in \mathcal{N}$,所以 $A=B\cup (A\setminus B)\in \bar{\mathcal{A}}$。

  下面我们把 $\mu$ 延拓到 $\mathcal{B}$ 上。如果 $A\in \bar{\mathcal{A}}=\mathcal{B}$,
  且 $B,B'\in \mathcal{A}$ 是满足 $B\subseteq A\subseteq B'$ 且 $\mu(B'\setminus B)=0$ 的集合,
  那么定义 $\mu(A)=\mu(B)=\mu(B')$。这种定义不依赖于 $B$ 和 $B'$ 的选取:
  如果 $C,C'$ 也有这样的性质,那么同时有 $\mu(C)\leq \mu(B')$ 以及 $\mu(B)\leq \mu(C')$,
  所以必须有 $\mu(B)=\mu(B')=\mu(C)=\mu(C')$。最后只需要验证 $\mu$ 在 $\mathcal{B}$
  上确实是一个测度。假设 $(A_n)$ 是 $\mathcal{B}$ 中一列不相交的集合,那么对于每个 $n$
  都可以选取一个 $B_n\in \mathcal{A}$ 使得 $B_n\subseteq A_n$ 并且 $A_n\setminus B_n$
  是可忽略集。此时,有
  \[
    \mu\biggl(
      \bigcup_{n\in \mathbb{N}} A_n
    \biggr)=\mu\biggl(
      \bigcup_{n\in \mathbb{N}} B_n
    \biggr)=\sum_{n\in \mathbb{N}}\mu(B_n)=\sum_{n\in \mathbb{N}}\mu(A_n).\qedhere
  \]
\end{proof}

\begin{remark}
  令 $f,g$ 是两个定义在 $E$ 上的实值函数。假设 $g$ 是 $\mathcal{A}$-可测的
  并且 $f=g,\ \alev{\mu}$,这里指的是集合 $\{x\in E\,|\, f(x)\neq g(x)\}$
  是 $\mu$-可忽略集。那么 $f$ 是 $\bar{\mathcal{A}}$-可测的。
  根据定义,存在 $C\in \mathcal{A}$ 使得 $\mu(C)=0$ 并且 $f(x)=g(x)$ 
  对于 $x\notin C$。此时,任取 Borel 集 $H\subseteq \mathbb{R}$,有
  $g^{-1}(H)\setminus C\subseteq f^{-1}(H)\subseteq g^{-1}(H)\cup C$,
  所以 $f^{-1}(H)\in \bar{\mathcal{A}}$。 
\end{remark}

\begin{proposition}
  $\sigma$-域 $\mathcal{M}(\lambda^*)$ 等于 $\mathcal{B}(\mathbb{R}^d)$
  相对于 Lebesgue 测度的完备化 $\overline{\mathcal{B}}(\mathbb{R}^d)$。
\end{proposition}
\begin{proof}
  $\overline{\mathcal{B}}(\mathbb{R}^d)\subseteq \mathcal{M}(\lambda^*)$
  是简单的。如果 $A$ 是可忽略集,那么存在 $B\in \mathcal{B}(\mathbb{R}^d)$
  使得 $\lambda(B)=0$ 且 $A\subseteq B$。于是 $\lambda^*(A)\leq \lambda^*(B)=0$,
  根据 \autoref{thm:outer measure},$A\in \mathcal{M}(\lambda^*)$。

  反之,令 $A\in \mathcal{M}(\lambda^*)$,我们需要证明 $A\in \overline{\mathcal{B}}(\mathbb{R}^d)$。
  不失一般性,我们假设存在某个 $K>0$ 使得 $A\subseteq (-K,K)^d$,
  否则我们可以把 $A$ 写为可数个集合 $A\cap (-n,n)^d$ 的并集。
  那么 $\lambda^*(A)<\infty$,并且对于每个 $n\geq 1$,我们可以找到可数个
  开矩形 $(P_i^n)_{i\in \mathbb{N}}$ 使得
  \[
    A\subseteq \bigcup_{i\in \mathbb{N}} P_i^n,\quad
    \sum_{i\in \mathbb{N}} \vol(P_i^n)\leq \lambda^*(A)+\frac{1}{n}.
  \]
  我们还可以假设每个 $P_i^n\subseteq (-K,K)^d$,否则
  将 $P_i^n$ 替换为 $P_i^n\cap (-K,K)^d$ (两个开矩形的交依然是开矩形)即可。
  令
  \[
    B_n=\bigcup_{i\in \mathbb{N}} P_i^n,\quad B=\bigcap_{n=1}^\infty B_n.
  \]
  那么 $B\in \mathcal{B}(\mathbb{R}^d)$ 并且 $A\subseteq B$。对于每个 $n$,还有
  \[
    \lambda(B)\leq \lambda(B_n)\leq \sum_{i\in \mathbb{N}} \vol(P_i^n)
    \leq \lambda^*(A)+\frac{1}{n},
  \]
  这就表明 $\lambda(B)\leq \lambda^*(A)$,同时还有 $\lambda(B)=\lambda^*(B)\geq \lambda^*(A)$,
  所以 $\lambda(B)=\lambda^*(A)$。然后把 $A$ 替换为 $(-K,K)^d\setminus A$,
  同样的论证表明存在 $B'\in \mathcal{B}(\mathbb{R}^d)$ 使得
  $(-K,K)^d\setminus A\subseteq B'$ 且 $\lambda(B')=\lambda^*((-K,K)^d\setminus A)$。
  再令 $C=(-K,K)^d\setminus B'$,那么 $C\subseteq A\subseteq B$ 且
  $\lambda(C)=\lambda^*(A)=\lambda(B)$。由于 $\lambda(B\setminus C)=0$,
  所以 $A\in \overline{\mathcal{B}}(\mathbb{R}^d)$。
\end{proof}


\begin{proposition}
  $\mathbb{R}^d$ 上的 Lebesgue 测度是平移不变的:对于任意 $A\in \mathcal{B}(\mathbb{R}^d)$
  和 $x\in \mathbb{R}^d$,有 $\lambda(A+x)=\lambda(A)$。
  反过来,如果 $\mu$ 是 $\mathcal{B}(\mathbb{R}^d)$ 上的一个测度,
  其在有界集上的值有限并且平移不变,那么存在常数 $c\geq 0$ 使得
  $\mu=c\lambda$。
\end{proposition}
\begin{proof}
  对于 $x\in \mathbb{R}^d$,记 $\sigma_x$ 是平移映射 $\sigma_x(y)=y-x$。
  那么推前测度 $\sigma_x(\lambda)$ 满足
  \[
    \forall A\in \mathcal{B}(\mathbb{R}^d),\quad
    \sigma_x(\lambda)(A)=\lambda(x+A).
  \]
  首先,对于任意矩形 $A$,肯定有 $\sigma_x(\lambda)(A)=\lambda(A)$,所以
  根据单调类定理的推论 \ref{coro:uniqueness of measure},对于任意 $A\in \mathcal{B}(\mathbb{R}^d)$,
  都有 $\sigma_x(\lambda)(A)=\lambda(A)$。


\end{proof}


\section{$\mathbb{R}$ 上的有限测度和 Stieltjes 积分}

下面的定理描述了 $(\mathbb{R},\mathcal{B}(\mathbb{R}))$ 上的所有有限测度。

\begin{theorem}
  \mbox{}
  \begin{enumerate}
    \item 令 $\mu$ 是 $(\mathbb{R},\mathcal{B}(\mathbb{R}))$ 上的一个有限测度。
    对于每个 $x\in \mathbb{R}$,令
    \[
      F_\mu(x)=\mu((-\infty,x]).
    \]
    函数 $F_\mu$ 是递增的,有界的,右连续的,并且使得 $F_\mu(-\infty)=0$
    (这意味着 $F_\mu(x)$ 在 $x\to-\infty$ 时为零)。
    \item 反之,若函数 $F:\mathbb{R}\to \mathbb{R}_+$ 是递增的,有界的,右连续的,并且
    使得 $F(-\infty)=0$,那么唯一存在 $(\mathbb{R},\mathcal{B}(\mathbb{R}))$ 上的一个有限测度 $\mu$ 使得
    $F=F_\mu$。
  \end{enumerate}
\end{theorem}
\begin{proof}
  (1) 显然 $F_\mu$ 是递增的并且有界的。下面证明右连续性。
  令 $(x_n)$ 是一个递减到 $x$ 的数列,那么
  \[
    F_\mu(x_n)=\mu((-\infty,x_n])\xrightarrow[n\to\infty]{} \mu\biggl(
      \bigcap_{n=1}^\infty (-\infty,x_n]
    \biggr)=\mu((-\infty,x])=F_\mu(x).
  \]
  类似地,如果 $(x_n)$ 是递减趋于 $-\infty$ 的数列,由于 $\bigcap_{n}(-\infty,x_n]=\emptyset$,
  所以 $F_\mu(x_n)\to 0$。

  (2) 因为集合族 $\mathcal{C}=\{(-\infty,x]\,|\, x\in \mathbb{R}\}$
  对有限交封闭并且生成了 $\mathcal{B}(\mathbb{R})$,
  所以 $\mu$ 的唯一性由单调类定理的推论 \ref{coro:uniqueness of measure} 保证。
  下面证明存在性。对于每个 $A\subseteq \mathbb{R}$,令
  \[
    \mu^*(A)=\inf\left\{
      \sum_{i\in \mathbb{N}} \bigl(
        F(b_i)-F(a_i)
      \bigr)\,\middle|\, A\subseteq \bigcup_{i\in \mathbb{N}}(a_i,b_i]
    \right\}.
  \]
  注意这里使用左开右闭区间覆盖更加方便。类似于 Lebesgue 测度的证明,可以验证
  $\mu^*$ 是 $\mathbb{R}$ 上的一个外测度并且区间 $(-\infty,\alpha]$
  在 $\mathcal{M}(\mu^*)$ 中,所以 $\mathcal{B}(\mathbb{R})\subseteq \mathcal{M}(\mu^*)$。
  于是我们可以把 $\mu^*$ 限制到 $\mathcal{B}(\mathbb{R})$ 上,得到一个测度 $\mu$。
  显然,$\mu$ 是有限测度。

  最后只需要验证 $\mu((-\infty,x])=F(x)$。只需要证明 $a<b$ 的时候有 $\mu((a,b])=F(b)-F(a)$。
  根据 $\mu^*$ 的定义,立马得到 $\mu((a,b])\leq F(b)-F(a)$。

  反过来,令 $((x_i,y_i])$ 是 $(a,b]$ 的可数覆盖,$\varepsilon\in (0,b-a)$。
  对于每个 $i\in \mathbb{N}$,我们可以找到 $y_i'>y_i$ 使得 $F(y_i')\leq F(y_i)+\varepsilon 2^{-i}$。
  根据紧性,我们可以找到足够大的 $N_\varepsilon$ 使得 $[a+\varepsilon,b]$ 可以被
  有限个 $((x_i,y_i'))_{i\in \{1,\dots,N_\varepsilon\}}$ 覆盖。 我们有
  \begin{align*}
    F(b)-F(a+\varepsilon)&\leq \sum_{i=1}^{N_\varepsilon}(F(y_i')-F(x_i))
    \leq \sum_{i=1}^\infty (F(y_i')-F(x_i))\\
    &\leq \sum_{i=1}^\infty (F(y_i)-F(x_i))+\varepsilon.
  \end{align*}
  因为 $\varepsilon$ 是任意的并且 $\varepsilon\to 0$ 的时候有
  $F(a+\varepsilon)\to F(a)$,所以就得到
  \[
    F(b)-F(a)\leq \sum_{i=1}^\infty (F(y_i)-F(x_i)).
  \]
  这就表明 $\mu((a,b])\geq F(b)-F(a)$。
\end{proof}










