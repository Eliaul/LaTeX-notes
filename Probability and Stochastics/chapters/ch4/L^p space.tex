
\chapter{$L^p$ 空间}

\section{定义与 H\"older 不等式}

在本章中,我们考虑测度空间 $(E,\mathcal{A},\mu)$。对于实数 $p\geq 1$,
我们令 $\mathcal{L}^p(E,\mathcal{A},\mu)$ 表示所有满足
\[
  \int |f|^p\d\mu<\infty  
\]
的可测函数 $f:E\to \mathbb{R}$ 构成的空间。
此外,我们引入 $\mathcal{L}^\infty(E,\mathcal{A},\mu)$ 表示所有
几乎处处有界的可测函数 $f:E\to \mathbb{R}$ 构成的空间,即存在
常数 $C\in \mathbb{R}_+$ 使得
\[
  |f|\leq C,\ \alev{\mu}  
\]
同样,我们可以引入复值函数空间 $\mathcal{L}_{\mathbb{C}}^p(E,\mathcal{A},\mu)$
以及 $\mathcal{L}_{\mathbb{C}}^\infty(E,\mathcal{A},\mu)$。

对于每个 $p\in [1,\infty]$,我们可以定义 $\mathcal{L}^p$ 上
的一个等价关系:
\[
  f\sim g\Leftrightarrow f=g,\ \alev{\mu}  
\]
于是我们可以考虑商空间
\[
  L^p(E,\mathcal{A},\mu)=\mathcal{L}^p(E,\mathcal{A},\mu)/\sim.  
\]
也就是说,我们只考虑几乎处处相等意义上的函数,如果两个函数几乎处处
相等,那么我们认为这是同一个函数。通常情况下,我们把 $L^p(E,\mathcal{A},\mu)$
中的元素视为其在 $\mathcal{L}^p(E,\mathcal{A},\mu)$ 中的任意代表元:
当我们谈及 $L^p(E,\mathcal{A},\mu)$ 中的函数时,这意味着我们在 $\mathcal{L}^p(E,\mathcal{A},\mu)$
中选取了一个代表元,并且得到的结论和这个代表元的选取无关。

在没有歧义的情况下,我们使用 $L^p(\mu)$ 或者 $L^p$ 表示
$L^p(E,\mathcal{A},\mu)$。注意到 $L^1$ 就是所有可积函数
构成的空间。

对于可测函数 $f:E\to \mathbb{R}$ 和 $p\in [1,\infty)$,我们定义
\[
  \norm{f}_p=\left(\int |f|^p\d\mu\right)^{1/p}  .
\]
约定 $\infty^{1/p}=\infty$。定义
\[
  \norm{f}_\infty=\inf\big\{C\in[0,\infty]\,\big|\, |f|\leq C,\ \alev{\mu}\big\}  .
\]
显然有 $|f|\leq \norm{f}_\infty$ 并且 $\norm{f}_{\infty}$ 是 $[0,\infty]$
中满足这个性质的最小值。
如果 $f,g$ 是两个几乎处处相等的可测函数,那么有 $\norm{f}_p=\norm{g}_p$,
所以我们可以针对 $f\in L^p(E,\mathcal{A},\mu)$ 良好的定义 $\norm{f}_p$。

对于 $p,q\in [1,\infty]$,我们说 $p$ 和 $q$ 是\emph{共轭指数},如果
\[
  \frac{1}{p}+\frac{1}{q}=1.  
\]
特别地,$1$ 和 $\infty$ 是共轭的。

\begin{theorem}[H\"older 不等式]
  令 $p,q$ 是共轭指数,$f,g$ 是两个 $E\to \mathbb{R}$ 的可测函数,那么
  \[
    \int|fg|\d\mu\leq \norm{f}_p\norm{g}_q.  
  \]
  特别地,如果 $f\in L^p$ 以及 $g\in L^q$,那么
  $fg\in L^1$。
\end{theorem}
\begin{proof}
  若 $\norm{f}_p=0$,那么 $|f|=0\,\ \alev{\mu}$,这表明 
  $\int |fg|\d\mu=0$,结论显然成立,所以我们不妨假设
  $\norm{f}_p>0$ 以及 $\norm{g}_p>0$。进一步的,我们还可以假设
  $f\in L^p$ 以及 $g\in L^q$,否则右边为 $\infty$ 显然成立。

  先假设 $p=1$ 和 $q=\infty$,那么
  \[
    \int|fg|\d\mu\leq \norm{g}_{\infty}\int |f|\d\mu
    =  \norm{f}_1\norm{g}_\infty.
  \]
  下面假设 $1<p,q<\infty$。

  设 $\alpha\in (0,1)$,那么对于 $x\in [0,\infty)$ 有不等式
  \[
    x^\alpha-\alpha x\leq 1-\alpha,  
  \]
  取 $x=u/v\ (u\geq 0,v>0)$,我们有
  \[
    u^\alpha v^{1-\alpha}\leq \alpha u+(1-\alpha) v,  
  \]
  该不等式在 $v=0$ 时也成立。取 $\alpha=1/p$,$1-\alpha=1/q$,
  以及
  \[
    u=\frac{|f|^p}{\norm{f}_p^p} ,\quad v=\frac{|g|^q}{\norm{g}_q^q},
  \]
  那么
  \[
  \frac{|fg|}{\norm{f}_p^p\norm{g}_q^q}\leq
  \frac{1}{p}\frac{|f|^p}{\norm{f}_p^p}+
  \frac{1}{q}\frac{|g|^q}{\norm{g}_q^q},
  \]
  两边积分,即得
  \[
    \int|fg|\d\mu\leq   \norm{f}_p^p\norm{g}_q^q.\qedhere
  \]
\end{proof}

\begin{corollary}[Cauchy-Schwarz 不等式]
  取 $p=q=2$,即得
  \[
    \int |fg|\d\mu\leq \left(\int |f|^2 \d\mu\right)^{1/2}  
    \left(\int |g|^2 \d\mu\right)^{1/2}  .
  \]
\end{corollary}
 
\begin{corollary}
  假设 $\mu$ 是有限测度,$p,q$ 是共轭指数且 $p>1$,那么对于
  任意可测函数 $f:E\to \mathbb{R}$,有
  \[
    \norm{f}_1\leq \mu(E)^{1/q}\norm{f}_p,  
  \]
  因此,对于任意 $p\in (1,\infty]$,有 $L^p\subseteq L^1$。
  更一般地,对于任意 $1\leq r< r'<\infty$,有
  \[
    \norm{f}_r\leq \mu(E)^{\frac{1}{r}-\frac{1}{r'}}  
    \norm{f}_{r'},
  \]
  因此,对于任意 $1\leq p<q\leq \infty$,有 $L^q\subseteq L^p$,
  特别地,当 $\mu$ 是概率测度的时候,还有 $\norm{f}_p\leq\norm{f}_q$。
\end{corollary}
\begin{proof}
  取 $g=\indicator{E}$,即得
  \[
    \int|f|\d\mu=\int|f \indicator{E}|\d\mu\leq
    \norm{f}_p \norm{\indicator{E}}_q=\mu(E)^{1/q}\norm{f}_p. 
  \]
  用 $f^r$ 替代 $f$,取 $p=r'/r$,$1/q=1-r/r'$,那么
  \[
    \norm{f}_r\leq \mu(E)^{1/r-1/r'}\norm{f^r}_{r'/r}^{1/r}  
    =\mu(E)^{1/r-1/r'}\norm{f}_{r'}.\qedhere
  \]
\end{proof}


\begin{theorem}[Jensen 不等式]
  假设 $\mu$ 是概率测度,$\varphi:\mathbb{R}\to \mathbb{R}_+$
  是凸函数,那么对于每个 $f\in L^1(E,\mathcal{A},\mu)$,有
  \[
    \int \varphi\circ f\d\mu\geq \varphi\left(\int f\d\mu\right).
  \]
\end{theorem}





\section{Banach 空间 $L^p(E,\mathcal{A},\mu)$}

\begin{theorem}[Minkowski 不等式]
  令 $p\in [1,\infty]$,$f,g\in L^p(E,\mathcal{A},\mu)$,那么
  $f+g\in L^p(E,\mathcal{A},\mu)$ 并且
  \[
    \norm{f+g}_p\leq \norm{f}_p+\norm{g}_p.
  \]
\end{theorem}
\begin{proof}
  $p=1$ 时就是绝对值不等式 $|f+g|\leq |f|+|g|$。$p=\infty$ 时也是显然的。
  假设 $1<p<\infty$。由于
  \[
    |f+g|^p\leq (|f|+|g|)^p\leq \bigl(2\max(|f|,|g|)\bigr)^p
    \leq 2^p\bigl(|f|^p+|g|^p\bigr),
  \]
  所以 $f+g\in L^p(E,\mathcal{A},\mu)$。然后,考虑不等式
  \[
    |f+g|^p=|f+g|\times |f+g|^{p-1}\leq |f||f+g|^{p-1}+|g||f+g|^{p-1},
  \]
  两边积分,并且利用 H\"older 不等式,取 $p$ 和 $q=p/(p-1)$,即得
  \[
    \int|f+g|^p\d \mu\leq 
    \norm{f}_p\left(\int |f+g|^p\d\mu\right)^{\frac{p-1}{p}}+
    \norm{g}_p\left(\int |f+g|^p\d\mu\right)^{\frac{p-1}{p}}.
  \]
  如果 $\int |f+g|^p\d\mu=0$,那么定理自然成立,否则两边同时除以
  $\left(\int |f+g|^p\d\mu\right)^{(p-1)/p}$,即得所需不等式。
\end{proof}

\begin{theorem}[Reisz]
  对于每个 $p\in[1,\infty]$,空间 $L^p(E,\mathcal{A},\mu)$ 配备
  范数 $f\mapsto \norm{f}_p$ 是一个 Banach 空间(即完备的赋范向量空间)。
\end{theorem}
\begin{proof}
  首先考虑 $1\leq p<\infty$。先验证 $f\mapsto \norm f_p$ 是一个范数。
  我们有
  \[
    \norm f_p=0\Rightarrow \int |f|^p\d\mu=0\Rightarrow |f|=0,\ \alev{\mu}
  \]
  这意味着在 $L^p$ 中有 $f=0$。线性性 $\norm{\lambda f}_p=|\lambda| \norm{f}_p$
  是显然的。三角不等式由 Minkowski 不等式保证。

  然后我们说明 $L^p$ 空间是完备的。令 $(f_n)_{n\geq 1}$ 是 $L^p$ 中的一个
  Cauchy 列。我们可以找到一个严格递增的序列 $(k_n)$,使得对于每个 $n\geq 1$,有
  \[
    \norm{f_{k_{n+1}}-f_{k_n}}_p<2^{-n}.
  \]
  令 $g_n=f_{k_n}$,那么有 $\norm{g_{n+1}-g_n}_p <2^{-n}$。利用单调
  收敛定理和 Minkowski 不等式,我们有
  \begin{align*}
    \int\left(
      \sum_{n=1}^\infty |g_{n+1}-g_n|
    \right)^p\d\mu&=\ulim[N\to\infty]
    \int\left(
      \sum_{n=1}^N |g_{n+1}-g_n|
    \right)^p\d\mu\\
    &=\ulim[N\to\infty]
    \norm{\sum_{n=1}^N (g_{n+1}-g_n)}_p^p\\
    &\leq \ulim[N\to\infty]\left(\sum_{n=1}^N\norm{g_{n+1}-g_n}_p\right)^p\\
    &<\infty,
  \end{align*}
  这表明
  \[
    \sum_{n=1}^\infty |g_{n+1}-g_n|<\infty,\ \alev{\mu},
  \]
  对于每个使得 $\sum_{n=1}^\infty |g_{n+1}(x)-g_n(x)|<\infty$
  的 $x$,我们有
  \[
    h(x)=g_1(x)+\sum_{n=1}^\infty\bigl(g_{n+1}(x)-g_n(x)\bigr)
    =\lim_{n\to\infty} g_n(x).
  \]
  否则令 $h(x)=0$。根据 \autoref{lemma:pointwise limit measurable},
  $h$ 是可测函数。因为 $g_n$ 几乎处处收敛到 $h$,所以几乎处处有
  $|h|=\liminf |g_n|$,利用 Fatou 引理,我们有
  \[
    \int |h|^p\d\mu\leq \liminf \int|g_n|^p\d\mu
    \leq \sup_{n\geq 1}\int |g_n|^p\d\mu<\infty,
  \]
  最后一个小于号是因为 Cauchy 列 $(f_n)$ 一定有界。因此,$h\in L^p$。
  再次利用 Fatou 引理,对于每个 $n\geq 1$,有
  \begin{align*}
    \norm{h-g_n}_p^p&=\int |h-g_n|^p\d\mu\leq 
    \liminf_{m\to\infty} \int|g_m-g_n|^p\d \mu\\
    &=\liminf_{m\to\infty}\norm{g_m-g_n}_p^p\leq (2^{-n+1})^p,
  \end{align*}
  最后一个不等号使用了估计
  \[
    \norm{g_m-g_n}_p\leq \norm{g_{n+1}-g_n}_p+\cdots +\norm{g_m-g_{m-1}}_p<2^{-n+1}.
  \]
  于是,序列 $(g_n)$ 在 $L^p$ 中收敛到 $h$。由于 Cauchy 列有收敛子列则一定收敛,
  所以 $(f_n)$ 也收敛到 $h$,这就完成了 $1\leq p<\infty$ 时的证明。

  当 $p=\infty$ 时,验证 $f\mapsto\norm{f}_\infty$ 是一个范数的方法是一样的。
  令 $(f_n)_{n\geq 1}$ 是 $L^\infty$ 中的 Cauchy 列。
  根据 $L^\infty$-范数的定义,对于每个 $m>n\geq 1$,存在一个零可测集 $N_{m,n}$
  使得对于 $x\in E\setminus N_{m,n}$,有 $|f_m(x)-f_n(x)|\leq \norm{f_m-f_n}_\infty$。
  令 $N$ 是所有 $N_{m,n}$ 的可数并,那么仍然有 $\mu(N)=0$,并且
  对于每个 $x\in E\setminus N$ 有 $|f_n(x)-f_m(x)|\leq \norm{f_n-f_m}_\infty$。
  此时,对于每个 $x\in E\setminus N$,序列 $(f_n(x))_{n\geq 1}$ 是 $\mathbb{R}$ 中的 Cauchy 列,
  因此有一个极限,记为 $g(x)$。若 $x\in N$ 则令 $g(x)=0$。此时 $g$ 是可测函数,
  并且
  \[
    \sup_{x\in E\setminus N}|f_n(x)-g(x)|\leq \sup_{m\in\{n+1,n+2,\dots\}}
    \norm{f_n-f_m}_\infty,
  \]
  右端在 $n\to\infty$ 时趋于 $0$,因此 $f_n$ 在 $L^\infty$ 中收敛到 $g$。
\end{proof}
 
\begin{example}
  如果 $E=\mathbb{N}$ 并且 $\mu$ 是计数测度,那么对于每个 $p\in [1,\infty)$,
  空间 $L^p(\mathbb{N},\mathcal{P}(\mathbb{N}),\mu)$ 就是所有实数列
  $a=(a_n)_{n\in \mathbb{N}}$ 的空间,并且满足
  \[
    \norm{a}_p=\left(\sum_{n=0}^\infty |a_n|^p\right)^{1/p}<\infty.
  \]
  空间 $L^\infty$ 则是所有有界实数列构成的空间,并且范数 $\norm{a}_\infty=\sup_{n\in \mathbb{N}}|a_n|$。
  注意到这个测度空间中没有非空的零测集,所以此时 $L^p$ 和 $\mathcal{L}^p$ 是重合的。
  这个空间通常记为 $\ell^p=\ell^p(\mathbb{N})$,是一个非常重要的 Banach 空间。
\end{example}

从上述定理的证明中我们还可以得到以下命题。

\begin{proposition}\label{prop:subsequence pointwise convergence in Lp}
  令 $p\in [1,\infty)$,$(f_n)_{n\in \mathbb{N}}$ 是 $L^p(E,\mathcal{A},\mu)$
  中的收敛序列,有极限 $f$。那么存在一个子列 $(f_{k_n})_{n\in \mathbb{N}}$
  几乎处处逐点收敛到 $f$。
\end{proposition}
\begin{remark}
  这个命题对 $p=\infty$ 时也成立,但是这种情况下不需要提取子列了,
  因为 $L^\infty$ 中的范数在除开零测集的意义上等价于一致收敛。
\end{remark}

需要注意的是,函数空间 $L^p$ 中的收敛并不意味着逐点收敛。
下面的例子说明了这两种收敛之间的区别。我们构造一个在 $L^p\bigl([0,1],\mathcal{B}([0,1]),\lambda\bigr)$
中收敛到 $0$ 但是逐点处处发散的序列 $(f_n)$。
首先,我们把区间 $[0,1]$ 不断二等分、四等分、八等分、……,按照下面的顺序依次编号为 $I_n$:
\[
  [0,1],\left[0,\frac{1}{2}\right],\left[\frac{1}{2},1\right],
  \left[0,\frac{1}{4}\right],\left[\frac{1}{4},\frac{1}{2}\right],
  \left[\frac{1}{2},\frac{3}{4}\right],\left[\frac{3}{4},1\right],\dots.
\] 
令 $f_n=\idf_{I_n}$。对于任意 $p\in [1,\infty)$,有
\[
  \norm{f_n-0}_p=\lambda(I_n)^{1/p}\to 0,
\]
因此 $f_n\xlongrightarrow{L^p}0$。但是任取 $x\in [0,1]$,
总存在无穷多个 $n$ 使得 $x\in I_n$,于是有无穷多个 $f_n(x)=1$,
因此序列 $(f_n(x))$ 发散。所以为了构造 $(f_n)$ 在 $L^p$ 中的收敛函数,
直接取逐点的极限是不行的,必须要从中挑选收敛行为一致的子列。
例如,在这个例子中,我们可以取 $g_n=f_{2^n}=\idf_{[0,1/2^n]}$,那么
对于任意 $x\in [0,1]$ 有
\[
  \sum_{n=1}^\infty |g_{n+1}(x)-g_n(x)|<\infty.
\]
此时再取极限
\[
  h(x)=g_1(x)+\sum_{n=1}^\infty(g_{n+1}(x)-g_n(x))=\lim_{n\to\infty} g_n(x)=\begin{cases}
    1 & x=0,\\
    0 & x\in (0,1],
  \end{cases}
\]
就能得到 $L^p$ 空间中的极限函数。

不过,根据 \autoref{prop:subsequence pointwise convergence in Lp},如果已知
$L^p(E,\mathcal{A},\mu)$ 空间中的函数列 $(f_n)$ 有极限 $f$,并且几乎处处
有逐点收敛 $f_n(x)\to g(x)$,那么可以得到 $f=g,\,\alev{\mu}$。

对于 $p\in [1,\infty)$,我们可能会问什么时候反过来成立:如果 $L^p(E,\mathcal{A},\mu)$
中的函数列 $(f_n)$ 几乎处处逐点收敛到 $f$,那么是否在 $L^p$ 中收敛?
也即函数空间 $L^p$ 中的逐点收敛是否意味着 $L^p$ 中的收敛?
一般来说这也是错误的,但是根据控制收敛定理,如果满足下面的条件,
\begin{enumerate}
  \item $f_n\to f,\,\alev{\mu}$,
  \item 存在一个非负可测函数 $g$ 使得 $\int g^p\d\mu<\infty$ 并且对于每个 $n$
  有 $|f_n|\leq g,\,\alev{\mu}$,
\end{enumerate}
那么函数 $f\in L^p$ 且 $f_n\xrightarrow{L^p}f$。

$p=2$ 的时候尤其重要,因为 $L^2$ 空间有一个 Hilbert 空间的结构。

\begin{theorem}
  空间 $L^2(E,\mathcal{A},\mu)$ 配备内积
  \[
    \langle f,g\rangle=\int f g\d\mu  
  \]
  是一个实 Hilbert 空间。
\end{theorem}
\begin{proof}
  Cauchy-Schwarz 不等式保证了 $fg$ 是可积的,因此内积是良定义的。 
  在这个内积中有 $\langle f,f\rangle=\norm f_{2}^2$,完备性在上面已经证明。
\end{proof}

经典的 Hilbert 空间理论可以应用到 $L^2$ 空间中。例如,如果
$\varPhi:L^2(E,\mathcal{A},\mu)\to \mathbb{R}$ 是连续线性映射,
那么存在唯一的 $g\in L^2(E,\mathcal{A},\mu)$,使得对于每个
$f\in L^2(E,\mathcal{A},\mu)$,都有 $\varPhi(f)=\langle f,g\rangle$。
这个结果非常有用。

\section{$L^p$ 空间中的稠密性定理}
 
令 $(E,d)$ 是度量空间。回顾一个函数 $f:E\to \mathbb{R}$ 被称为 Lipschitz
的,如果存在常数 $K>0$ 使得
\[
  \forall x,y\in E,\quad |f(x)-f(y)|\leq K d(x,y).
\]
$(E,\mathcal{B}(E))$ 上的测度 $\mu$ 被称为外正则的,如果对于每个 $A\in \mathcal{B}(E)$,
都有
\[
  \mu(A)=\inf \big\{\mu(U)\,\big|\, U\supseteq A,\ U\text{ 是开集}\big\}.
\]  
当 $\mu$ 是有限测度时,外正则性总是成立的。

考虑一般的可测空间 $(E,\mathcal{A},\mu)$,我们在前面的章节引入了简单函数。
如果 $\mu$ 是 $(E,\mathcal{A})$ 上的测度,对于可积的简单函数和任意 $p\in [1,\infty)$,
它们也都是 $L^p(\mu)$ 中的元素。

\begin{theorem}\label{thm:density in Lp space}
  对于 $p\in [1,\infty)$。
  \begin{enumerate}
    \item 如果 $(E,\mathcal{A},\mu)$ 是测度空间,那么所有可积的简单函数
    集合在 $L^p(E,\mathcal{A},\mu)$ 中稠密。
    \item 如果 $(E,d)$ 是度量空间,$\mu$ 是 $(E,\mathcal{B}(E))$ 上的外正则测度,
    那么所有有界 Lipschitz 函数的集合是 $L^p(E,\mathcal{B}(E),\mu)$ 的稠密子集。
    \item 如果 $(E,d)$ 是可分的局部紧的度量空间,$\mu$ 是 $E$ 上的 Radon 测度,
    那么所有紧支的 Lipschitz 函数的集合在 $L^p(E,\mathcal{B}(E),\mu)$ 中稠密。
  \end{enumerate}
\end{theorem}
\begin{proof}
  (1) 任取函数 $f\in L^p$,有分解 $f=f^+-f^-$,所以不妨假设 $f$ 是非负函数。
  此时,根据 \autoref{prop:property of integral of positive function},
  $f$ 是一列递增简单函数 $(\varphi_n)$ 的逐点极限,记作
  \[
    f=\ulim[n\to\infty] \varphi_n.
  \]
  那么 $\int |\varphi_n|^p\d\mu \leq \int|f|^p\d\mu<\infty$,
  所以 $\varphi_n\in L^p$ (对于简单函数而言,这等价于 $\varphi_n\in L^1$)。
  因为 $|f-\varphi_n|^p\leq f^p$,根据控制收敛定理,有
  \[
    \lim_{n\to\infty}\int |f-\varphi_n|^p\d\mu=0.
  \]
  这就表明 $\varphi_n\xrightarrow{L^p} f$。

  (2) 令 $f\in L^p(E,\mathcal{B}(E),\mu)$,根据 (1),存在一列
\end{proof}

\section{Radon-Nikodym 定理}

\begin{definition}
  令 $\mu$ 和 $\nu$ 是 $(E,\mathcal{A})$ 上的两个测度。
  \begin{enumerate}
    \item 如果
    \[
      \forall A\in \mathcal{A},\quad \mu(A)=0\Rightarrow \nu(A)=0,
    \]
    那么说 $\nu$ 相对于 $\mu$ 是\emph{绝对连续}的,记作 $\nu\ll \mu$。
    \item 如果存在 $N\in \mathcal{A}$ 使得 $\mu(N)=0$ 并且 $\nu(N^c)=0$,
    那么说 $\nu$ 相对于 $\mu$ 是\emph{奇异}的,记作 $\nu\perp \mu$。
  \end{enumerate}
\end{definition}

\begin{example}
  令 $f$ 是 $E$ 上的非负可测函数。那么测度 $\nu =f\cdot \mu$ 相对于 $\mu$
  是绝对连续的。实际上,如果 $A\in \mathcal{A}$ 且 $\mu(A)=0$,那么
  \[
    \nu(A)=\int_A f\d\mu=\int \idf_A f\d\mu=0.
  \]
\end{example}

回顾 $\mu$ 是 $\sigma$-有限的,如果 $E$ 是可数个 $\mu$-有限测度集的并。

\begin{theorem}[Radon-Nikodym]
  假设 $\mu$ 是 $\sigma$-有限的,并且 $\nu$ 是另一个 $\sigma$-有限测度。
  那么,存在唯一的 $\sigma$-有限测度对 $(\nu_a,\nu_s)$,使得
  \begin{enumerate}
    \item $\nu=\nu_a+\nu_s$。
    \item $\nu_a\ll \mu$ 并且 $\nu_s\perp \mu$。
  \end{enumerate}
  此外,存在非负可测函数 $g:E\to \mathbb{R}_+$ 使得 $\nu_a=g\cdot\mu$,
  也即
  \[
    \forall A\in \mathcal{A},\quad \nu_a(A)=\int_A g\d\mu.
  \]
  并且 $g$ 在以下意义下唯一:如果 $\tilde g$ 也满足上述性质,那么
  有 $g=\tilde g,\,\alev{\mu}$。
\end{theorem}
\begin{remark}
  这个定理的第一部分,是说 $\nu$ 可以分解为绝对连续部分和奇异部分的和,
  被称为\emph{Lebesgue 分解}。如果 $\nu\ll \mu$,这个定理表明存在
  非负可测函数 $g$ 使得 $\nu=g\cdot \mu$,这个函数 $g$ 被称为
  $\nu$ 相对于 $\mu$ 的\emph{Radon-Nikodym 导数}。
\end{remark}
\begin{proof}
  我们证明 $\mu$ 和 $\nu$ 都是有限测度的情况,最后解释如何推广到
  $\sigma$-有限测度的情况。

  \noindent\textit{Step 1.} 假设 $\mu$ 和 $\nu$ 是有限的且 $\nu\leq\mu$。
  也就是说,我们假设对于每个 $A\in \mathcal{A}$,都有 $\nu(A)\leq \mu(A)$,
  特别的,这当然表明 $\nu\ll \mu$。并且,对于任意非负可测函数 $g$,有
  $\int g\d \nu\leq \int g\d\mu$。考虑线性映射 $\varPhi:L^2(E,\mathcal{A},\mu)\to \mathbb{R}$
  为
  \[
    \varPhi(f)=\int f\d\nu.
  \]
  注意到这个积分是良定义的,因为 $\int |f|\d\nu\leq \int |f|\d\mu$,并且
  $\mu$ 是有限测度表明 $L^2(\mu)\subseteq L^1(\mu)$。此外 $\varPhi$
  也不依赖于代表元 $f$ 的选取,这是因为
  \[
    f=\tilde f,\ \alev{\mu}\Rightarrow f=\tilde f,\ \alev{\nu}
    \Rightarrow \int f\d\nu=\int \tilde f\d\nu.
  \]
  那么 Cauchy-Schwarz 不等式表明
  \[
    |\varPhi(f)|\leq \left(
      \int |f|^2\d\nu
    \right)^{1/2}\nu(E)^{1/2}\leq 
    \left(
      \int |f|^2\d\mu
    \right)^{1/2}\nu(E)^{1/2}=\nu(E)^{1/2}\norm{f}_{L^2(\mu)}.
  \]
  这表明 $\varPhi$ 是连续线性映射,因此根据 Riesz 表示定理,
  存在唯一的 $g\in L^2(E,\mathcal{A},\mu)$,使得对于每个
  $f\in L^2(E,\mathcal{A},\mu)$,都有
  \[
    \varPhi(f)=\langle f,g\rangle=\int f g\d\mu.
  \]
  特别的,取 $f=\idf_{A}$,就有
  \[
    \forall A\in \mathcal{A},\quad \nu(A)=\int_A g\d\mu.
  \]
  我们还可以注意到 $0\leq g\leq 1,\ \alev{\mu}$。这是因为,任取 $\varepsilon>0$,
  有
  \begin{align*}
    \mu\bigl(\{x\in E\,|\, g(x)\geq 1+\varepsilon\}\bigr)&\geq 
    \nu\bigl(\{x\in E\,|\, g(x)\geq 1+\varepsilon\}\bigr)\\
    &=\int_{\{x|g(x)\geq 1+\varepsilon\}} g\d\mu \\
    &\geq (1+\varepsilon)\mu\bigl(\{x\in E\,|\, g(x)\geq 1+\varepsilon\}\bigr),
  \end{align*}
  这表明 $\mu(\{x|g(x)\geq 1+\varepsilon\})=0$,由于 $\varepsilon$ 是任意的,
  所以 $g\leq 1,\ \alev{\mu}$。类似的论述可以得到 $g\geq 0,\ \alev{\mu}$。
  将 $g$ 替换为 $(g\vee 0)\wedge 1$ 之后,我们可以假设对于
  任意 $x\in E$ 有 $0\leq g(x)\leq 1$。于是,我们得到了定理在
  $\nu_a=g\cdot \mu$ 以及 $\nu_s=0$ 的情况下的结论。$g$ 的唯一性
  在第二步证明。

  \noindent\textit{Step 2.} 假设 $\mu$ 和 $\nu$ 是有限测度。首先我们用 $\mu+\nu$
  替代 $\mu$,根据第一步的结论,存在满足 $0\leq h\leq 1$ 的可测函数 $h$,
  使得对于任意 $f\in L^2(\mu+\nu)$,都有
  \[
    \int f\d\nu =\int fh\d(\mu+\nu).
  \]
  特别的,对于任意有界可测函数 $f$,都有 
  \[
    \int f\d\nu=\int fh\d\mu+\int fh\d\nu.
  \]
  这表明
  \begin{equation}\label{eq:radon-nikodym step2}
    \int f(1-h)\d\nu=\int fh\d\mu.
  \end{equation}
  使用单调收敛定理,我们可以将上式推广到任意非负可测函数 $f$。

  令 $N=\{x\in E\,|\, h(x)=1\}$。那么,在 \eqref{eq:radon-nikodym step2} 中取
  $f=\idf_N$,我们得到 $\mu(N)=0$。考虑测度 $\nu_s=\idf_N\cdot \nu$,
  那么对于任意 $A\in \mathcal{A}$,都有
  \[
    \nu_s(A)=\nu(A\cap N),
  \]
  于是 $\nu_s(N^c)=0$,因此 $\nu_s$ 相对于 $\mu$ 是奇异的。另一方面,
  在 \eqref{eq:radon-nikodym step2} 中取 $f=\idf_{N^c}(1-h)^{-1}f$,我们得到,
  对于每个非负可测函数 $f$,都有
  \[
    \int_{N^c}f\d\nu=\int_{N^c}f\frac{h}{1-h}\d\mu=\int fg\d\mu,
  \]
  其中 $g=\idf_{N^c}\frac{h}{1-h}$。令
  \[
    \nu_a=\idf_{N^c}\cdot\nu =g\cdot \mu.
  \]
  此时定理的 (1) 和 (2) 就得到了满足。注意到 $\int g\d\mu=\nu_a(E)<\infty$。

  下面验证 $(\nu_a,\nu_s)$ 的唯一性。假设 $(\tilde\nu_a,\tilde \nu_s)$
  是另一个满足 (1) 和 (2) 的测度对。那么对于每个 $A\in \mathcal{A}$,都有
  \[
    \nu_s(A)-\tilde \nu_s(A)=\tilde \nu_a(A)-\nu_a(A).
  \]
  由于 $\nu_s\perp \mu$ 并且 $\tilde\nu_s\perp \mu$,所以可以找到两个
  可测集 $N$ 和 $\tilde N$ 使得 $\nu_s(N^c)=0$ 并且 $\tilde \nu_s(\tilde N^c)=0$。
  那么,对于每个 $A\in \mathcal{A}$,就有
  \begin{align*}
    \nu_s(A)-\tilde \nu_s(A)&=\nu_s(A\cap (N\cup \tilde N))-\tilde \nu_s
    (A\cap (N\cup \tilde N))\\
    &=\tilde \nu_a(A\cap (N\cup \tilde N))-\nu_a(A\cap (N\cup \tilde N))=0.
  \end{align*}
  第一个等号是因为 $\nu_s(N^c\cap \tilde N^c)=0$,所以 $\nu_s(N\cup \tilde N)=\nu_s(E)$。
  第三个等号是因为 $\mu(N\cup\tilde N)=0$,同时 $\nu_a\ll \mu$ 以及 $\tilde\nu_a\ll\mu$。
  所以我们证明了 $\nu_s=\tilde \nu_s$,那么自然也有 $\nu_a=\tilde\nu_a$。

  最后,再说明 $g$ 的唯一性。假设存在另一个函数 $\tilde g$ 使得 $\nu_a=\tilde g\cdot\mu$。
  令 $\{\tilde g\geq g\}$ 表示集合 $\{x\in E\,|\, \tilde g(x)\geq g(x)\}$。
  那么
  \[
    \int_{\{\tilde g\geq g\}} \tilde g\d\mu=\nu_a(\{\tilde g\geq g\})
    =\int_{\{\tilde g\geq g\}} g\d\mu,
  \]
  这表明
  \[
    \int_{\{\tilde g\geq g\}} (\tilde g -g)\d\mu=0,
  \]
  即 $\idf_{\{\tilde g\geq g\}}(\tilde g-g)=0,\ \alev{\mu}$,即
  $\tilde g\leq g,\ \alev{\mu}$。交换 $g$ 和 $\tilde g$ 的角色,我们也可以得到
  $\tilde g\geq g,\ \alev{\mu}$。综上所述,$g=\tilde g,\ \alev{\mu}$。

  \noindent\textit{Step 3.} 现在考虑一般情况。如果 $\mu$ 和 $\nu$
  都是 $\sigma$-有限的,那么可以找到 $E$ 的一列不相交的可测集 $(E_n)_{n\in \mathbb{N}}$
  使得每个 $\mu(E_n)<\infty$ 和 $\nu(E_n)<\infty$,并且 $E=\bigcup_{n}E_n$。
  令 $\mu_n$ 是 $\mu$ 在 $E_n$ 上的限制,$\nu_n$ 是 $\nu$ 在 $E_n$ 上的限制。
  根据第二步的结论,对于每个 $n\in \mathbb{N}$,有分解
  \[
    \nu_n=\nu_{a}^n+\nu_{s}^n,
  \]
  其中 $\nu_s^n\perp \mu_n$ 并且 $\nu_a^n=g_n\cdot\mu_n$,我们还可以假设
  $g_n$ 在 $E_n^c$ 上为零。令
  \[
    \nu_a=\sum_{n\in \mathbb{N}}\nu_a^n,\quad
    \nu_s=\sum_{n\in \mathbb{N}}\nu_s^n,\quad
    g=\sum_{n\in \mathbb{N}} g_n.
  \]
  对于每个 $x\in E$,由于 $E_n$ 互不相交,所以只有一个 $n$ 使得 $g_n(x)>0$。
  $(\nu_a,\nu_s,g)$ 的唯一性与有限测度的情况类似。
\end{proof} 

\begin{example}
  \mbox{}
  \begin{enumerate}
    \item 取 $(E,\mathcal{A})=(\mathbb{R},\mathcal{B}(\mathbb{R}))$,
    假设 $\mu=f\cdot\lambda$,其中 $\lambda$ 是 Lebesgue 测度。
    假设 $f$ 在 $(0,\infty)$ 上是正的,在 $(-\infty,0]$ 上为零。令 $\nu=g\cdot \lambda$
    是另一个相对于 $\lambda$ 绝对连续的测度,此时 $\nu$ 相对于 $\mu$
    的 Lebesgue 分解为
    \[
      \nu= h\cdot\mu +\theta,
    \]
    其中 $h(x)=\idf_{(0,\infty)}\cdot g(x)/f(x)$ 并且 $\theta(\d x)=\idf_{(-\infty,0]}(x)\nu(\d x)$。
    \item 取 $E=[0,1)$ 和 $\mathcal{A}=\mathcal{B}([0,1))$,对于每个 $n\in \mathbb{N}$,
    令 $\sigma$-域
    \[
      \mathcal{F}_n=\sigma\left(
        \left\{
          \Bigl[
            \frac{i-1}{2^n},\frac{i}{2^n}
          \Bigr)\,\middle|\, i\in\{1,2,\dots,2^n\}
        \right\}
      \right),
    \]
    不难发现任意 $A\in \mathcal{F}_n$ 都是区间 $\bigl[(i-1)2^{-n},i2^{-n}\bigr)$
    的有限并。记 $\lambda$ 是 $[0,1)$ 上的 Lebesgue 测度,$\nu$ 是 $\mathcal{B}([0,1))$
    上的有限测度。把 $\lambda$ 和 $\nu$ 都限制到 $\mathcal{F}_n$ 上,可以把
    $\lambda$ 和 $\nu$ 都视为 $([0,1),\mathcal{F}_n)$ 上的测度。显然,此时
    $\nu$ 相对于 $\lambda$ 是绝对连续的,因为只有 $\emptyset$ 使得 $\lambda(\emptyset)=0$。
    此外,还可以验证 $\nu$ 相对于 $\lambda$ 的 Radon-Nikodym 导数为
    \[
      f_n(x)=\sum_{i=1}^{2^n}\frac{\nu\bigl(\bigl[(i-1)2^{-n},i2^{-n}\bigr)\bigr)}{2^{-n}}
      \idf_{[(i-1)2^{-n},i2^{-n})}(x).
    \]
    后面,我们将使用鞅论证明存在一个 Borel 可测函数 $f$ 使得
    $f_n(x)\to f(x),\ \alev{\lambda}$,并且将 $\nu$ 和 $\lambda$ 视为
    $([0,1),\mathcal{B}([0,1)))$ 上的测度时,$\nu$ 相对于 $\lambda$ 的
    Lebesgue 分解的绝对连续部分正是 $f\cdot\lambda$。
  \end{enumerate}
\end{example}





