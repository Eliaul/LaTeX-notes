\chapter{可测空间}

\section{可测集}

\begin{definition}
  集合 $E$ 上的 $\sigma$-域 $\mathcal{A}$ 指的是 $E$ 的一个子集族,
  其满足下面的性质:
  \begin{enumerate}
    \item $E\in \mathcal{A}$;
    \item $A\in \mathcal{A}\Rightarrow A^c\in \mathcal{A}$;
    \item 如果一列子集 $A_n\in \mathcal{A}$,那么
    $\bigcup_{n\in \mathbb{N}} A_n\in \mathcal{A}$。
  \end{enumerate}
\end{definition}

$\mathcal A$ 的元素被称为\emph{可测集},$(E,\mathcal{A})$ 被称为\emph{可测空间}。
根据定义,我们很容易得出下面的结果:
\begin{itemize}[nosep]
  \item $\emptyset=E^c\in \mathcal{A}$。
  \item 如果一列子集 $A_n\in \mathcal{A}$,那么
  \[
    \bigcap_{n\in \mathbb{N}}A_n=\biggl(\bigcup_{n\in \mathbb{N}}A_n\biggr) ^c\in \mathcal{A}.
  \]
  \item $\mathcal{A}$ 对有限并和有限交也是封闭的,只需要从某一项 $A_n$ 开始
  全部取空集即可。
\end{itemize}

\begin{example}
  根据可测集的定义,很容易构造出一些最简单的例子:
  \begin{enumerate}
    \item $\mathcal{A}=\mathcal{P}(E)$,当 $E$ 是有限集或者可数集的时候
    我们通常会使用这样的 $\sigma$-域,其他情况则很少使用。
    \item $\mathcal{A}=\{\emptyset,E\}$,平凡 $\sigma$-域。
    \item $E$ 的所有至多可数的子集以及所有补集至多可数的子集构成
    $E$ 上的一个 $\sigma$-域。
  \end{enumerate}
\end{example}

为了产生更多的例子,我们注意到 $E$ 上任意 $\sigma$-域的交集仍然是
$\sigma$-域,这导出了下面的定义。

\begin{definition}
  令 $\mathcal{C}$ 是 $\mathcal{P}(E)$ 的子集,$E$ 上包含 $\mathcal{C}$
  的最小的 $\sigma$-域被记为 $\sigma(\mathcal{C})$,不难看出其是所有包含 $\mathcal{C}$
  的 $\sigma$-域的交集。我们称 $\sigma(\mathcal{C})$ 是由 $\mathcal{C}$ 生成的
  $\sigma$-域。
\end{definition}

\begin{definition}
  设 $(E,\mathcal{O})$ 是拓扑空间,所有开集 $\mathcal{O}$ 生成的 $\sigma$-域
  $\sigma(\mathcal{O})$ 被称为 $E$ 上的 Borel $\sigma$-域,记为 $\mathcal{B}(E)$。
\end{definition}

$E$ 上的 Borel $\sigma$-域是包含所有开集的最小的 $\sigma$-域。
$\mathcal{B}(E)$ 的元素被称为 $E$ 的\emph{Borel 子集}。显然,
$E$ 中的闭集也都是 Borel 子集。

\begin{example}[$\mathbb{R}$ 上的 Borel $\sigma$-域]
  记 $\mathcal{C}_1$ 为 $\mathbb{R}$ 中开区间的集合:
  \[
    \mathcal{C}_1=\{(a,b)\,|\, a,b\in \mathbb{R},a<b\}  ,
  \]
  显然有 $\mathcal{C}_1\subseteq \mathcal{B}(\mathbb{R})$,于是
  $\sigma(\mathcal{C}_1)\subseteq \mathcal{B}(\mathbb{R})$。
  下面我们说明 $\mathcal{B}(\mathbb{R})\subseteq \sigma(\mathcal{C}_1)$。
  我们不加证明地使用一个结论(Lindel\"of 定理):$\mathbb{R}$ 的任意开子集 $U$ 都是开区间的可数并。
  那么根据 $\sigma$-域的定义,任意开区间都在 $\sigma(\mathcal{C}_1)$ 中,
  故 $\mathcal{B}(\mathbb{R})\subseteq \sigma(\mathcal{C}_1)$。
  这表明 $\mathcal{B}(\mathbb{R})$ 可以由所有开区间生成。

  此外,如果注意到
  \[
    (a,b)=(-\infty,b)\cap (-\infty,a)^c,  
  \]
  还可以证明 $\mathcal{B}(\mathbb{R})$ 由 $\mathcal{C}_2$ 生成,其中
  \[
    \mathcal{C}_2=\{(-\infty,a)\,|\, a\in \mathbb{R}\}.
  \]
  最后,不难证明这里的开区间都可以换成闭区间。
\end{example}

在后文中,每当我们考虑拓扑空间(例如 $\mathbb{R}$ 或者 $\mathbb{R}^d$)时,
除非有特别说明,否则我们总是假设它们配备 Borel $\sigma$-域。

下一个非常重要的 $\sigma$-域是乘积 $\sigma$-域。

\begin{definition}
  令 $(E_1,\mathcal{A}_1)$ 和 $(E_2,\mathcal{A}_2)$ 是可测空间,定义
  $E_1\times E_2$ 上的 $\sigma$-域 $\mathcal{A}_1\otimes \mathcal{A}_2$ 为
  \[
    \mathcal{A}_1\otimes \mathcal{A}_2=\sigma\bigl(\{A_1\times A_2\,|\, A_1\in \mathcal{A}_1,A_2\in \mathcal{A}_2\}\bigr).
  \]
\end{definition}

\begin{lemma}
  设 $E$ 和 $F$ 是可分(有可数的稠密子集)的拓扑空间,$E\times F$ 配备积拓扑,那么
  $\mathcal{B}(E\times F)=\mathcal{B}(E)\otimes \mathcal{B}(F)$。
\end{lemma}


\section{正测度}

令 $(E,\mathcal{A})$ 是可测空间。

\begin{definition}
  $(E,\mathcal{A})$ 上的正测度指的是一个映射 $\mu:\mathcal{A}\to [0,\infty]$,
  其满足下面的性质:
  \begin{enumerate}
    \item $\mu(\emptyset)=0$;
    \item ($\sigma$-可加性) 对于任意可数个不相交的可测集序列 $(A_n)_{n\in \mathbb{N}}$,有
    \[
      \mu\biggl(\bigcup_{n\in \mathbb{N}}A_n\biggr) =\sum_{n\in \mathbb{N}}\mu(A_n).
    \]
  \end{enumerate}
  此时,三元组 $(E,\mathcal{A},\mu)$ 被称为\emph{测度空间}。
  值 $\mu(E)$ 被称为测度 $\mu$ 的总质量。
\end{definition}

需要注意的是,我们允许 $\mu$ 的值为 $+\infty$,此时级数
$\sum_{n\in \mathbb{N}}\mu(A_n)$ 作为正向级数在 $[0,\infty]$
中总是有意义的。
根据 $\sigma$-可加性,如果我们令 $n>n_0$ 开始 $A_n=\emptyset$,
便可以得到有限可加性。

\begin{proposition}[测度的性质]
  根据定义,测度 $\mu$ 满足下面的性质:
  \begin{enumerate}
    \item 如果 $A\subseteq B$,那么 $\mu(A)\leq\mu(B)$。此外,如果
    还满足 $\mu(A)<\infty$,那么
    \[
        \mu(B \smallsetminus A)=\mu(B)-\mu(A).
    \]
    \item 如果 $A,B\in \mathcal{A}$,那么
    \[
      \mu(A)+\mu(B)=\mu(A\cup B)+\mu(A\cap B).  
    \]
    \item 如果 $A_n\in \mathcal{A}$ 且 $A_n\subseteq A_{n+1}$,那么
    \[
      \mu\biggl(\bigcup_{n\in \mathbb{N}}A_n\biggr)  
      =\lim_{n\to\infty}\mu(A_n).
    \]
    \item 如果 $B_n\in \mathcal{A}$ 且 $B_{n+1}\subseteq B_n$,
    $\mu(B_1)<\infty$,那么
    \[
      \mu\biggl(\bigcap_{n\in \mathbb{N}}B_n\biggr)  
      =\lim_{n\to\infty}\mu(B_n).
    \]
    \item 如果 $A_n\in \mathcal{A}$,那么
    \[
      \mu\biggl(\bigcup_{n\in \mathbb{N}}A_n\biggr)\leq \sum_{n\in \mathbb{N}}\mu(A_n).
    \]
  \end{enumerate}
\end{proposition}
\begin{proof}
  (1) 若 $A\subseteq B$,那么 $B=A\bigcup (B \smallsetminus A)$ 是无交并,所以
  \[
    \mu(B)=\mu(A)+\mu(B \smallsetminus A)\geq \mu(A).  
  \]

  (2) 若 $\mu(A),\mu(B)$ 中有至少一个为无穷,那么根据 (1),
  $\mu(A\cup B)$ 为无穷,所以结论成立。下面假设
  $\mu(A),\mu(B)$ 均有限,记 $C=A\cap B$,那么
  $A\cup B=(A \smallsetminus C)\cup C\cup(B \smallsetminus C)$ 是无交并,
  所以
  \[
    \mu(A\cup B)=\mu(A \smallsetminus C)+\mu(C)+\mu(B \smallsetminus C)
    =\mu(A)+\mu(B)-\mu(C),
  \]
  结论 (2) 成立。

  (3) 令 $C_1=A_1$,对于 $n\geq 2$ 的时候,令
  \[
    C_n=A_{n}  \smallsetminus A_{n-1},
  \]
  那么 $A_n=\bigcup_{k\leq n}C_k$ 是无交并,所以
  \[
    \mu\biggl(\bigcup_{n\in \mathbb{N}}A_n\biggr)  
    =\mu\biggl(\bigcup_{n\in \mathbb{N}}C_n \biggr)
    =\sum_{n\in \mathbb{N}}\mu(C_n)=\lim_{n\to\infty}
    \sum_{k=1}^n \mu(C_k)=\lim_{n\to\infty} \mu(A_n).
  \]

  (4) 令 $A_n=B_1 \smallsetminus B_n$,那么 $A_n\subseteq A_{n+1}$,此时
  \[
    \mu\biggl(\bigcap_{n\in \mathbb{N}}B_n\biggr)  =
    \mu(B_1)-\mu\biggl(B_1 \smallsetminus\bigcap_{n\in \mathbb{N}}B_n\biggr)
    = \mu(B_1)-\mu\biggl(\bigcup_{n\in \mathbb{N}}A_n\biggr),
  \]
  再根据 (3),就有
  \[
    \mu\biggl(\bigcap_{n\in \mathbb{N}}B_n\biggr)  =\mu(B_1)-\lim_{n\to\infty}\mu(A_n)
    =\lim_{n\to\infty}\mu(B_1 \smallsetminus A_n)=\lim_{n\to\infty}\mu(B_n).
  \]

  (5) 令 $C_1=A_1$,对于 $n\geq 2$ 的时候,令
  \[
    C_n=A_n \smallsetminus\bigcup_{k=1}^{n-1}A_k,
  \]
  那么 $C_n$ 之间互不相交,所以
  \[
    \mu\biggl(\bigcup_{n\in \mathbb{N}}A_n\biggr)=\mu\biggl(\bigcup_{n\in \mathbb{N}}C_n\biggr)
    =\sum_{n\in \mathbb{N}}\mu(C_n)\leq \sum_{n\in \mathbb{N}}\mu(A_n).\qedhere
  \]
\end{proof}

\begin{example}[常见的测度]
  \mbox{}
  \begin{enumerate}
    \item 令 $E=\mathbb{N}$,$A=\mathcal{P}(\mathbb{N})$,定义%
    \emph{计数测度}为
    \[
      \mu(A)=\card(A).  
    \]
    \item 如果 $A$ 是 $E$ 的子集,定义 $A$ 的示性函数 $\mathbf{1}_A:E\to\{0,1\}$
    为
    \[
      \mathbf{1}_A(x)=\begin{cases}
        1 & x\in A,\\
        0 & x\notin A.
      \end{cases}  
    \]
    令 $(E,\mathcal{A})$ 是可测空间,固定 $x\in E$。对于每个 $A\in \mathcal{A}$,令
    $\delta_x(A)=\mathbf{1}_A(x)$,这给出了 $(E,\mathcal{A})$ 上的一个测度,被称为
    \emph{$\mathbold x$ 处的 Dirac 测度}。更一般的,如果 $(x_n)_{n\in \mathbb{N}}$
    是 $E$ 中的点列,$(\alpha_n)_{n\in \mathbb{N}}$ 是 $[0,\infty]$ 中的点列,
    我们可以考虑测度 $\sum_{n\in \mathbb{N}}\alpha_n\delta_{x_n}$ 为
    \[
      \biggl(\sum_{n\in \mathbb{N}}\alpha_n\delta_{x_n}\biggr)
      (A)=\sum_{n\in \mathbb{N}}\alpha_n\delta_{x_n}(A)=
      \sum_{n\in \mathbb{N}}\alpha_n\mathbold 1_{A}(x_n),
    \]
    这个测度被称为 $E$ 上的\emph{点测度}。
    \item 可以证明,在 $(\mathbb{R},\mathcal{B}(\mathbb{R}))$ 上存在唯一的正测度 $\lambda$
    使得:对于每个闭区间 $[a,b]$,有 $\lambda\bigl([a,b]\bigr)=b-a$。这个
    测度 $\lambda$ 被称为\emph{Lebesgue 测度}。
    Lebesgue 测度的唯一性可以由 \autoref{coro:uniqueness of measure} 保证,
    存在性由 ?保证。
  \end{enumerate}
\end{example}

如果 $\mu$ 是 $(E,\mathcal{A})$ 上的正测度,$C\in \mathcal{A}$,
那么可以定义 $\mu$ 在 $C$ 上的\emph{限制} $\nu$ 为:
\[
  \nu(A)=\mu(A\cap C),\quad \forall A\in \mathcal{A}.
\]
不难验证 $\nu$ 还是 $(E,\mathcal{A})$ 上的正测度。

\begin{definition}
  \mbox{}
  \begin{itemize}[nosep]
    \item 如果 $\mu(E)<\infty$,那么我们说测度 $\mu$ 是\emph{有限的}。
    \item 如果 $\mu(E)=1$,那么我们说测度 $\mu$ 是\emph{概率测度},$(E,\mathcal{A},\mu)$ 是\emph{概率空间}。
    \item 如果存在一列可测集 $(E_n)_{n\in \mathbb{N}}$ 使得 $E=\bigcup_n E_n$ 以及每个
    $\mu(E_n)<\infty$,那么我们说测度 $\mu$ 是\emph{$\mathbold\sigma$-有限的}。
    \item 如果 $x\in E$ 使得单点集 $\{x\}\in \mathcal{A}$ 并且 $\mu(\{x\})>0$,那么我们说
    $x$ 是测度 $\mu$ 的一个\emph{原子}。
    \item 如果测度 $\mu$ 没有原子,那么我们说 $\mu$ 是\emph{扩散测度}。
  \end{itemize}
\end{definition}

如果 $(A_n)_{n\in \mathbb{N}}$ 是一列可测集,类比数列的上下极限,我们可以定义
集合列的上下极限分别为:
\[
  \limsup A_n=\bigcap_{n=1}^\infty\biggl(\bigcup_{k=n}^\infty A_k\biggr),\quad
  \liminf A_n=\bigcup_{n=1}^\infty\biggl(\bigcap_{k=n}^\infty A_k\biggr).
\]
注意到对于任意 $m$,都有
\[
  \bigcup_{n=1}^m\biggl(\bigcap_{k=n}^\infty A_k\biggr)
  =\bigcap_{k=m}^\infty A_k,\quad 
  \bigcap_{n=1}^m\biggl(\bigcup_{k=n}^\infty A_k\biggr)=\bigcup_{k=m}^\infty A_k,
\]
所以显然有 $\liminf A_n\subseteq \limsup A_n$。

\begin{lemma}\label{lemma:liminf and limsup ineq}
  令 $\mu$ 是 $(E,\mathcal{A})$ 上的测度,那么
  \[
    \mu(\liminf A_n)\leq \liminf \mu(A_n).
  \]
  如果 $\mu$ 是有限测度,或者更一般地,$\mu\left(\bigcup_{n=1}^\infty A_n\right)<\infty$,
  那么
  \[
    \mu(\limsup A_n)\geq\limsup\mu(A_n).
  \]
\end{lemma}
\begin{proof}
  对于任意的 $n$,有 
  \[
    \mu\biggl(\bigcap_{k=n}^\infty A_k\biggr)\leq \inf_{k\geq n}\mu(A_k),
  \]
  所以
  \[
    \mu(\liminf A_n)=\lim_{n\to\infty}\mu\biggl(\bigcap_{k=n}^\infty A_k\biggr)
    \leq \lim_{n\to\infty}\inf_{k\geq n}\mu(A_k)=\liminf \mu(A_n).
  \]
  第二个结论同理。
\end{proof}

\section{可测函数}

\begin{definition}
  令 $(E,\mathcal{A})$ 和 $(F,\mathcal{B})$ 是两个可测空间,如果映射 $f:E\to F$ 满足:
  \[
    \forall B\in \mathcal{B},\  f^{-1}(B)\in \mathcal{A},
  \]
  那么我们说 $f$ 是\emph{可测映射}。当 $E,F$ 是两个配备了 Borel $\sigma$-域的拓扑空间时,
  我们说 $f$ 是\emph{Borel 可测的}。
\end{definition}

显然,可测映射的复合是可测映射。

\begin{proposition}
  令 $(E,\mathcal{A})$ 和 $(F,\mathcal{B})$ 是两个可测空间,映射 $f:E\to F$。$f$ 可测
  当且仅当对于某个生成 $\mathcal{B}$ 的子集族 $\mathcal{C}$ (即 $\mathcal{B}=\sigma(\mathcal{C})$),
  有 $f^{-1}(B)\in \mathcal{A}\ (\forall B\in \mathcal{C})$。
\end{proposition}
\begin{proof}
  只需证明充分性。记
  \[
    \mathcal{G}=\{B\in \mathcal{B}\,|\, f^{-1}(B)\in \mathcal{A}\},
  \]
  直接验证可知 $\mathcal{G}$ 是一个 $\sigma$-域,又因为 $\mathcal{C}\subseteq \mathcal{G}$,
  所以 $\mathcal{B}=\sigma(\mathcal{C})\subseteq \mathcal{G}\subseteq \mathcal{B}$,
  所以 $\mathcal{G}=\mathcal{B}$,这就表明 $f$ 是可测的。
\end{proof}

\begin{example}
  若 $(F,\mathcal{B})=(\mathbb{R},\mathcal{B}(\mathbb{R}))$,要证明 $f$ 是可测的,只需说明
  集合 $f^{-1}((a,b))$ 是可测的,或者 $f^{-1}((-\infty,a))$ 是可测的。
\end{example}

\begin{corollary}
  设 $E,F$ 是两个配备 Borel $\sigma$-域的拓扑空间,那么连续映射 $f:E\to F$ 都是可测的。
\end{corollary}

\begin{lemma}
  令 $(E,\mathcal{A}),(F_1,\mathcal{B}_1)$ 和 $(F_2,\mathcal{B}_2)$ 是可测空间,乘积
  $F_1\times F_2$ 配备乘积 $\sigma$-域 $\mathcal{B}_1\otimes \mathcal{B}_2$,令映射
  $f_1:E\to F_1$ 和 $F_2:E\to F_2$,定义 $f:E\to F_1\times F_2$ 为 $f(x)=(f_1(x),f_2(x))$,
  那么 $f$ 可测当且仅当 $f_1,f_2$ 都可测。
\end{lemma}

\begin{corollary}
  令 $(E,\mathcal{A})$ 是可测空间,$f,g$ 是从 $E$ 到 $\mathbb{R}$ 的可测函数,那么函数
  \[ 
    f+g,fg,\min(f,g),\max(f,g)
  \] 
  都是可测的。
\end{corollary}

记扩充实数 $\bar{\mathbb{R}}=\mathbb{R}\cup\{-\infty,+\infty\}$,其拓扑为序拓扑。
与 $\mathbb{R}$ 类似,$\bar{\mathbb{R}}$ 的 Borel $\sigma$-域由区间 $[-\infty,a)$
生成。

\begin{proposition}
  令 $(f_n)_{n\in \mathbb{N}}$ 是 $E\to\bar{\mathbb{R}}$ 的可测函数列,那么
  \[
    \sup f_n,\quad \inf f_n,\quad \limsup_{n\to\infty}f_n,\quad \liminf_{n\to\infty} f_n
  \]
  都是可测函数。特别地,如果 $(f_n)$ 逐点收敛,那么极限 $\lim f_n$ 是可测函数。
\end{proposition}

下述技巧性引理通常是有用的。

\begin{lemma}\label{lemma:pointwise limit measurable}
  令 $(f_n)$ 是一列从 $E$ 到 $\mathbb{R}$ 的可测函数。所有使得 $f_n(x)\ (n\to\infty)$
  收敛的 $x\in \mathbb{R}$ 的集合 $A$ 是可测的。此外,定义函数 $h:E\to \mathbb{R}$ 为
  \[
    h(x)=\begin{cases}
      \lim_{n\to\infty}f_n(x) & x\in A,\\
      0 & x\notin A.
    \end{cases}
  \]
  那么 $h$ 是可测的。
\end{lemma}

\begin{definition}
  令 $(E,\mathcal{A})$ 和 $(F,\mathcal{B})$ 是可测空间,$\varphi:E\to F$ 是可测映射, 
  $\mu$ 是 $(E,\mathcal{A})$ 上的测度,定义 $(F,\mathcal{B})$ 上的测度 $\nu$ 为
  \[
    \nu(B)=\mu(\varphi^{-1}(B)),\quad \forall B\in \mathcal{B}.
  \]
  $\nu$ 被称为\emph{$\mathbold\mu$ 在 $\mathbold\varphi$ 下的推前},记为
  $\varphi(\mu)$,有时也记为 $\varphi_*\mu$。
\end{definition}

\section{单调类}

本节我们陈述单调类定理,这是测度论甚至概率论中的一个基本工具。

\begin{definition}
  $\mathcal{P}(E)$ 的一个子集 $\mathcal{M}$ 如果满足:
  \begin{enumerate}
    \item $E\in \mathcal{M}$;
    \item 对于任意 $A,B\in \mathcal{M}$ 且 $A\subseteq B$,有
    $B \smallsetminus A\in \mathcal{M}$;
    \item 如果一列子集 $A_n\subseteq \mathcal{M}$ 且 $A_n\subseteq A_{n+1}$,
    那么 $\bigcup_{n\in \mathbb{N}}A_n\in \mathcal{M}$,
  \end{enumerate}
  那么我们说 $\mathcal{M}$ 是一个\emph{单调类}。
\end{definition}

显然,$\sigma$-域都是单调类。反之,一个单调类是 $\sigma$-域当且仅当
其对有限交封闭。这很容易证明,若单调类 $\mathcal{M}$ 对有限交封闭,那么
任取一列子集 $A_n\subseteq \mathcal{M}$,对于任意的 $n$,有 
\[
  \bigcup_{k=1}^n  A_k=E \smallsetminus\bigcap_{k=1}^n A_k^c\in \mathcal{M},
\]
所以
\[
  \bigcup_{n\in \mathbb{N}}A_n=\bigcup_{n\in \mathbb{N}}
  \biggl(\bigcup_{k=1}^n A_k\biggr)  \in \mathcal{M},
\]
这就表明 $\mathcal{M}$ 是一个 $\sigma$-域。

与 $\sigma$-域类似,显然单调类的任意交仍然是单调类。如果
$\mathcal{C}\subseteq \mathcal{P}(E)$,那么我们可以定义
由 $\mathcal{C}$ 生成的单调类 $\mathcal{M}(\mathcal{C})$,
即包含 $\mathcal{C}$ 的最小的单调类,其可以通过对所有
包含 $\mathcal{C}$ 的单调类取交集得到。

\begin{theorem}[单调类定理]
  令 $\mathcal{C}\subseteq \mathcal{P}(E)$ 对有限交封闭,
  那么 $\mathcal{M}(\mathcal{C})=\sigma(\mathcal{C})$。
  因此,如果 $\mathcal{M}$ 是包含 $\mathcal{C}$ 的任意单调类,
  那么 $\sigma(\mathcal{C})\subseteq \mathcal{M}$。
\end{theorem}
\begin{proof}
  显然有 $\mathcal{M}(\mathcal{C})\subseteq \sigma(\mathcal{C})$。
  要证明 $\sigma(\mathcal{C})\subseteq \mathcal{M}(\mathcal{C})$,
  只需要说明 $\mathcal{M}(\mathcal{C})$ 是 $\sigma$-域。
  根据上面的叙述,这只需要说明 $\mathcal{M}(\mathcal{C})$ 对
  有限交封闭。

  对于 $A\in \mathcal{P}{(E)}$,记
  \[
    \mathcal{M}_A=\{B\in \mathcal{M}(\mathcal{C})\,|\,
    A\cap B\in \mathcal{M}(\mathcal{C})\}.  
  \]
  直接验证可知 $\mathcal{M}_A$ 是一个单调类。
  下面任取 $A\in \mathcal{C}$,由于 $\mathcal{C}$
  对有限交封闭,所以 $\mathcal{C}\subseteq \mathcal{M}_A$,这就表明
  $\mathcal{M}(\mathcal{C})\subseteq \mathcal{M}_A$。

  接下来任取 $D\in \mathcal{M}(\mathcal{C})$,上面的叙述告诉我们
  $\mathcal{C}\subseteq \mathcal{M}_D$,所以 $\mathcal{M}(\mathcal{C})\subseteq \mathcal{M}_D$。
  这就表明 $\mathcal{M}(\mathcal{C})$ 对有限交封闭,所以
  $\mathcal{M}(\mathcal{C})$ 是 $\sigma$-域。
\end{proof}

单调类定理最重要的应用是证明某些测度的唯一性。

\begin{corollary}\label{coro:uniqueness of measure}
  令 $\mu,\nu$ 是 $(E,\mathcal{A})$ 上的两个测度。假设存在一个
  子集族 $\mathcal{C}\subseteq \mathcal{A}$ 满足 $\mathcal{C}$
  对有限交封闭且 $\mathcal{A}=\sigma(\mathcal{C})$,并且对于
  任意 $A\in \mathcal{C}$ 都有 $\mu(A)=\nu(A)$。
  \begin{enumerate}
    \item 如果 $\mu(E)=\nu(E)<\infty$,那么 $\mu=\nu$。
    \item 如果存在一列 $\mathcal{C}$ 中的递增序列 $(E_n)_{n\in \mathbb{N}}$ 
    使得 $E=\bigcup_{n\in \mathbb{N}}E_n$,并且
    $\mu(E_n)=\nu(E_n)<\infty$,那么 $\mu=\nu$。
  \end{enumerate}
\end{corollary}
\begin{proof}
  (1) 令
  \[
    \mathcal{G}=\{A\in \mathcal{A}\,|\,\mu(A)=\nu(A)\},
  \]
  那么 $\mathcal{C}\subseteq \mathcal{G}$ 且不难验证 $\mathcal{G}$
  是单调类,根据单调类定理,有 $\mathcal{A}=\sigma(\mathcal{C})\subseteq \mathcal{G}$,
  即 $\mu=\nu$。

  (2) 记 $\mu_n$ 为 $\mu$ 在 $E_n$ 上的限制,$\nu_n$ 同理。
  那么
  \[
    \mu_n(E)=\mu(E\cap E_n)=\mu(E_n)=\nu(E_n)=\nu(E\cap E_n)  
    =\nu_n(E),
  \]
  根据 (1),有 $\mu_n=\nu_n$。于是任取 $A\in \mathcal{A}$,有
  \begin{align*}
    \mu (A)&=\mu(A\cap E)=\mu\biggl(\bigcup_{n\in \mathbb{N}}(A\cap E_n)\biggr)
    =\ulim[n\to\infty]\mu(A\cap E_n)\\
    &=\ulim[n\to\infty]\mu_n(A)=\ulim[n\to\infty]\nu_n(A)
    =\ulim[n\to\infty]\nu(A\cap E_n)\\
    &=\nu\biggl(\bigcup_{n\in \mathbb{N}}(A\cap E_n)\biggr)
    =\nu(A\cap E)=\nu(A),
  \end{align*}
  这就表明 $\mu=\nu$。
\end{proof}

\autoref{coro:uniqueness of measure} 表明了 Lebesgue 测度的唯一性。
即若 $\lambda$ 是 $(\mathbb{R},\mathcal{B}(\mathbb{R}))$ 上的 
正测度,且使得 $\lambda\bigl([a,b]\bigr)=b-a$,那么这样的测度 $\lambda$
是唯一的。这是因为我们可以取
\[
  \mathcal{C}=\big\{[a,b]\,|\, a,b\in \mathbb{R},a<b\big\}  ,
\]
此时 $\mathcal{C}$ 对有限交封闭并且 $\mathcal{B}(\mathbb{R})=\sigma(\mathcal{C})$。
取 $E_n=[-n,n]\in \mathcal{C}$,那么 $\mathbb{R}=\bigcup_{n\in \mathbb{N}} E_n$
且 $\lambda(E_n)<\infty$,应用 \autoref{coro:uniqueness of measure}
的 (2) 即可表明唯一性。

