\documentclass[fontset=none]{Notes}

\makeatletter
\DeclareRobustCommand{\em}{%
  \@nomath\em \if b\expandafter\@car\f@series\@nil
  \normalfont \else \bfseries \fi}
\makeatother

\usepackage{tikz-cd,wrapstuff}
\usepackage{fixdif,siunitx,tikz,nicematrix}
\usetikzlibrary{matrix,calc}

\ProvidesFile{font.def}

\setCJKmainfont{Source Han Serif SC}[
  UprightFont=*-Regular,
  BoldFont=*-Bold,
  ItalicFont=HYKaiTi S,
  ItalicFeatures={Scale=1.1}
]
\newCJKfontfamily[zhsong]\songti{Source Han Serif SC}[
  UprightFont=*-Regular,
  BoldFont=*-Bold,
  ItalicFont=HYKaiTi S,
  ItalicFeatures={Scale=1.1}
]
\setCJKsansfont{Source Han Sans SC}[
  UprightFont=*-Regular,
  BoldFont=*-Bold
]
\newCJKfontfamily[zhhei]\heiti{Source Han Sans SC}[
  UprightFont=*-Regular,
  BoldFont=*-Bold
]
\setCJKmonofont{HYFangSong S}[
  BoldFont=*,
  ItalicFont=*,
  BoldItalicFont=*
]
\newCJKfontfamily[zhfs]\fangsong{HYFangSong S}[
  BoldFont=*,
  ItalicFont=*,
  BoldItalicFont=*
]
\newCJKfontfamily[zhkai]\kaishu{HYKaiTi S}[
  BoldFont=*,
  ItalicFont=*,
  BoldItalicFont=*
]

\setmainfont{texgyretermes}[
  Extension=.otf,
  UprightFont=*-regular,
  BoldFont=*-bold,
  ItalicFont=*-italic,
  BoldItalicFont=*-bolditalic,
  SlantedFont=*-italic
]
%\setmathrm{texgyretermes}[
%  Extension=.otf,
%  UprightFont=*-regular,
%  BoldFont=*-bold,
%  ItalicFont=*-italic,
%  BoldItalicFont=*-bolditalic,
%  SlantedFont=*-italic
%]
\setsansfont{Cantarell}[
  UprightFont=* Regular,
  ItalicFont=* Italic,
  BoldFont=* Bold,
  BoldItalicFont=* Bold Italic,
  SmallCapsFont=Alegreya Sans SC
]
\setmonofont{Ubuntu Mono}[
  UprightFont=*,
  ItalicFont=* Italic,
  BoldFont=* Bold,
  BoldItalicFont=* Bold Italic
]
%\setmathfont{texgyretermes-math.otf}
%\setmathfont[range={\mathcal,\mathbfcal,\mathfrak},StylisticSet=1]{XITSMath-Regular.otf}
%\setmathfont[range={\mathbb}]{KpMath-Sans.otf}



\usepackage[subscriptcorrection,nofontinfo,mtpbb,mtpfrak]{mtpro2}

\tikzcdset{
  arrow style=tikz,
  diagrams={>={Straight Barb[scale=0.8]}}
}

\allowdisplaybreaks[1]

\newlength{\mymathln}
\newcommand{\aligninside}[2]{
  \settowidth{\mymathln}{#2}
  \mathmakebox[\mymathln]{#1}
}

\DeclareMathOperator\Spec{Spec}
\DeclareMathOperator\im{im}
\DeclareMathOperator\nil{nil}
\DeclareMathOperator\rad{rad}
\DeclareMathOperator\Ann{Ann}
\DeclareMathOperator\Max{Max}
\DeclareMathOperator\GL{GL}
\DeclareMathOperator\End{End}
\DeclareMathOperator\Int{Int}
\DeclareMathOperator\Tor{Tor}
\DeclareMathOperator\Frac{Frac}
\DeclareMathOperator\Tr{Tr}
\DeclareMathOperator\Hom{Hom}
\DeclareMathOperator\Leb{Leb}
\DeclareMathOperator\supp{supp}
\DeclareMathOperator\Id{Id}
\DeclareMathOperator\rk{rank}
\DeclareMathOperator\coker{coker}

\newcommand{\ideal}[1]{\mathfrak{#1}}
\newcommand{\mat}[1]{\mathbold{#1}}
\newcommand{\uline}{\underline{\hphantom{X}}}
\newcommand{\abs}[1]{\left|#1\right|}

\usepackage{enumitem}

\setlist[enumerate]{nosep}

%\DeclareMathAlphabet\mathcal{OMS}{cmsy}{m}{n}

\newlength\stextwidth
\newcommand\makesamewidth[3][c]{%
  \settowidth{\stextwidth}{#2}%
  \makebox[\stextwidth][#1]{#3}%
}



\begin{document}

\frontmatter

\tableofcontents

\mainmatter

\chapter{测度和积分}

\section{可测空间}

令 $E$ 是集合,$\mathcal{E}$ 是 $E$ 的一个子集族。若对于任意 $A,B\in\mathcal{E}$ 有
$A\cap B\in\mathcal{E}$,那么我们说 $\mathcal{E}$ \emph{对交封闭}。如果 $\mathcal{E}$
中任意可数个集合的交还在 $\mathcal{E}$ 中,那么我们说 $\mathcal{E}$ 对可数交封闭。
类似地,我们可以定义对补封闭、对并封闭和对可数并封闭的概念。

\subsubsection{$\sigma$-代数}

如果 $E$ 的非空子集族 $\mathcal{E}$ 对补和有限并封闭,那么我们说 $\mathcal{E}$
是 $E$ 上的\emph{代数}。如果其对补和可数并封闭,那么我们说 $\mathcal{E}$
是 $E$ 上的\emph{$\sigma$-代数},即:
\begin{alphenum}
  \item $A\in\mathcal{E}\Rightarrow E\smallsetminus A\in\mathcal{E}$,
  \item $A_1,A_2,\dotsc\in \mathcal{E}\Rightarrow\bigcup_n A_n\in\mathcal{E}$。
\end{alphenum}
由于 $\left(\bigcup_n A_n\right)^c=\bigcap_n A_n^c\in\mathcal{E}$,所以对补和可数并封闭
可以自然导出对可数交封闭,即 $\sigma$-代数对可数交也封闭。

任取 $A\in \mathcal{E}$,那么 $E=A\cup(E \smallsetminus A)\in \mathcal{E}$,所以 $E$ 上
任意 $\sigma$-代数都至少包含 $E$ 和 $\emptyset$。事实上,$\mathcal{E}=\{E,\emptyset\}$
是 $E$ 上的最简单的 $\sigma$-代数,被称为\emph{平凡 $\sigma$-代数}。$E$ 上最大的
$\sigma$-代数当然是 $\mathcal{E}=2^E$,即 $\mathcal{E}$ 就是 $E$ 的幂集,被
称为\emph{离散 $\sigma$-代数}。

不难看出,$E$ 上一族 $\sigma$-代数的任意交(不一定可数)还是 $E$ 上的 $\sigma$-代数。
给定 $E$ 的一个子集族 $\mathcal{C}$,我们可以考虑所有包含 $\mathcal{C}$ 
的 $\sigma$-代数(总是存在至少一个这样的 $\sigma$-代数,即 $2^E$),将这些
$\sigma$-代数取交集,我们便得到了包含 $\mathcal{C}$ 的最小的 $\sigma$-代数,
被称为\emph{由 $\mathcal{C}$ 生成的} $\sigma$-代数,记为 $\sigma\mathcal{C}$。

如果 $E$ 是拓扑空间,由 $E$ 的所有开集族生成的 $\sigma$-代数被称为 $E$ 上的
\emph{Borel $\sigma$-代数},记为 $\mathcal{B}(E)$ 或者 $\mathcal{B}_E$,
其元素被称为\emph{Borel 集}。

\subsubsection{p-系和 d-系}

对于 $E$ 的子集族 $\mathcal{C}$,如果其对交封闭,那么我们说 $\mathcal{C}$
是一个 p-系,这里 p 代表 product,是“交”的另一种说法。
$E$ 的子集族 $\mathcal{D}$ 被称为 d-系,如果其满足:
\begin{alphenum}
  \item $E\in\mathcal{D}$,
  \item $\text{$A,B\in \mathcal{D}$ and $A\supseteq B\Rightarrow A \smallsetminus B\in \mathcal{D}$}$,
  \item $\text{$(A_n)\subseteq \mathcal{D}$ and $A_n\nearrow A\Rightarrow A\in \mathcal{D}$}$。
\end{alphenum}
其中 $(A_n)\subseteq D$ 表明 $(A_n)$ 是 $D$ 中的集合序列,$A_n\nearrow A$ 表明这个序列
递增于极限 $A$:
\[
  A_1\subseteq A_2\subseteq\cdots,\quad \bigcup_{n=1}^\infty A_n=A.
\]

显然一个 $\sigma$-代数既是 p-系又是 d-系,其反面也是成立的。所以 p-系和 d-系是产生
$\sigma$-代数的原始结构。

\begin{proposition}\label{prop:equivalent condition of sigma algebra}
  $E$ 的子集族是 $\sigma$-代数当且仅当其既是 p-系又是 d-系。
\end{proposition}
\begin{proof}
  $(\Rightarrow)$ 若 $\mathcal{E}$ 是 $\sigma$-代数,其显然是 p-系并且满足 d-系的条件 (a) 和 (c)。
  下面我们验证其满足 d-系的条件 (b)。任取 $A,B\in\mathcal{E}$ 且 $A\supseteq B$,
  那么 $A \smallsetminus B=A\cap (E \smallsetminus B)\in \mathcal{E}$,
  所以 $\mathcal{E}$ 是 d-系。

  $(\Leftarrow)$ 若 $\mathcal{E}$ 既是 p-系又是 d-系。任取 $A\in \mathcal{E}$,
  根据 d-系的 (a) 和 (b),我们有 $E \smallsetminus A\in \mathcal{E}$。
  所以 $\mathcal{E}$ 对补封闭。然后我们说明对并封闭。任取 $A,B\in \mathcal{E}$, 
  由于
  \[
    A\cup B=E \smallsetminus (A\cup B)^c=E \smallsetminus (A^c\cap B^c),
  \]
  结合 p-系对交封闭,所以 $A\cup B\in \mathcal{E}$。最后我们说明对可数并封闭。
  如果 $(A_n)\subseteq \mathcal{E}$,令 $B_n=A_1\cup\cdots\cup A_n$,
  那么 $(B_n)\subseteq \mathcal{E}$ 且 $B_n\nearrow A$,根据 d-系的 (c),所以
  $A\in \mathcal{E}$,故 $\mathcal{E}$ 对可数并封闭。
\end{proof}

下面的引理为本节的主要定理做准备。

\begin{lemma}\label{lemma:hat D}
  令 $\mathcal{D}$ 是 $E$ 上的 d-系,固定 $D\in \mathcal{D}$,令
  \[
    \hat{\mathcal{D}}=\{A\in \mathcal{D}: A\cap D\in \mathcal{D}\},
  \]
  那么 $\hat{\mathcal{D}}$ 仍然是 d-系。
\end{lemma}

\subsubsection{单调类定理}

这是一个非常有用的工具来证明某些集族是 $\sigma$-代数。

\begin{theorem}\label{thm:monotone class theorem}
  如果一个 d-系包含一个 p-系,那么其包含这个 p-系生成的 $\sigma$-代数。
\end{theorem}
\begin{proof}
  设 $\mathcal{C}$ 是一个 p-系。令 $\mathcal{D}$ 是包含 $\mathcal{C}$
  的最小的 d-系,即包含 $\mathcal{C}$ 的所有 d-系的交(不难看出 d-系的任意交是 d-系)。
  我们证明 $\mathcal{D}$ 实际上是一个 $\sigma$-代数,这样包含 $\mathcal{C}$
  的任意 d-系都包含 $\mathcal{D}$,而 $\mathcal{D}$ 作为包含 $\mathcal{C}$
  的 $\sigma$-代数,其包含 $\sigma \mathcal{C}$。
  根据 \autoref{prop:equivalent condition of sigma algebra},只需要说明 
  $\mathcal{D}$ 既是 p-系又是 d-系,而 $\mathcal{D}$ 已经是 d-系,所以只需要
  说明 $\mathcal{D}$ 是 p-系。

  我们首先说明对于任意的 $D\in \mathcal{D}$ 和 $C\in \mathcal{C}$,
  有 $D\cap C\in \mathcal{D}$。令
  \[
    \mathcal{D}_1=\{A\in \mathcal{D}: A\cap C\in \mathcal{D}\},  
  \]
  根据 \autoref{lemma:hat D},$\mathcal{D}_1$ 是 d-系。由于 $\mathcal{C}$
  是 p-系,所以 $\mathcal{C}\subseteq \mathcal{D}_1$,即 $\mathcal{D}_1$
  是包含 $\mathcal{C}$ 的 d-系,所以 $\mathcal{D}\subseteq \mathcal{D}_1$。
  这就表明 $D\in \mathcal{D}_1$,即 $D\cap C\in \mathcal{D}$。

  下面说明对于任意的 $D,B\in \mathcal{D}$,有 $D\cap B\in \mathcal{D}$。
  令
  \[
    \mathcal{D}_2=\{A\in \mathcal{D}: A\cap D\in \mathcal{D}\} .
  \]
  同样根据 \autoref{lemma:hat D},$\mathcal{D}_2$ 是 d-系。根据上面的叙述,
  有 $\mathcal{C}\subseteq \mathcal{D}_2$,即 $\mathcal{D}_2$ 是包含 
  $\mathcal{C}$ 的 d-系,所以 $\mathcal{D}\subseteq \mathcal{D}_2$,
  这就表明 $B\in \mathcal{D}_2$,即 $D\cap B\in \mathcal{D}$。
  这就证明了 $\mathcal{D}$ 是 p-系。
\end{proof}

\subsubsection{可测空间}

一个\emph{可测空间}指的是二元组 $(E,\mathcal{E})$,其中 $E$ 是集合,
$\mathcal{E}$ 是 $E$ 上的 $\sigma$-代数。此时,$\mathcal{E}$
的元素被称为\emph{可测集}。当 $E$ 是拓扑空间,$\mathcal{E}=\mathcal{B}_E$
的时候,可测集也被称为\emph{Borel 集}。

\subsubsection{可测空间的积}

令 $(E,\mathcal{E})$ 和 $(F,\mathcal{F})$ 是可测空间。
如果 $A\in \mathcal{E}$ 和 $B\in \mathcal{F}$,那么 $A\times B$
被称为\emph{可测矩形}。我们用 $\mathcal{E}\otimes \mathcal{F}$
表示 $E\times F$ 上的由可测矩形集族生成的 $\sigma$-代数,
被称为\emph{乘积 $\sigma$-代数}。可测空间 $(E\times F,\mathcal{E}\otimes \mathcal{F})$
被称为 $(E,\mathcal{E})$ 和 $(F,\mathcal{F})$ 的积,我们通常使用
$(E,\mathcal{E})\times (F,\mathcal{F})$ 来表示。

\subsubsection{Exercises}

\begin{exercise}{划分生成 $\sigma$-代数}{}
  \begin{alphenum}[nosep]
    \item 令 $\mathcal{C}=\{A,B,C\}$ 是 $E$ 的一个划分,列出
    $\sigma \mathcal{C}$ 的元素。
    \item 令 $\mathcal{C}$ 是 $E$ 的(可数)划分。证明 $\sigma \mathcal{C}$
    的每个元素都是 $\mathcal{C}$ 中元素的可数并。
    \item 令 $E=\mathbb{R}$,$\mathcal{C}$ 是 $\mathbb{R}$ 的所有单点子集
    构成的子集族。证明 $\sigma \mathcal{C}$ 的元素要么是可数集要么
    是可数集的补集。这表明从直观上来看,$\sigma \mathcal{C}$ 要比 $\mathcal{B}(\mathbb{R})$
    小得多,例如开区间 $(0,1)$ 属于后者但是不属于前者。
  \end{alphenum}
\end{exercise}
\begin{solution}
  (a) 令
  \[
    \mathcal{E}=\{A,B,C,A\cup B,A\cup C,B\cup C,E\},
  \]
  显然 $\mathcal{E}$ 是一个 $\sigma$-代数。对于任意包含 $\mathcal{C}$
  的 $\sigma$-代数,由于其对并封闭,所以其必须包含 $\mathcal{E}$,
  所以 $\mathcal{E}=\sigma \mathcal{C}$。

  (b) 令 $\mathcal{E}$ 为 $\mathcal{C}$ 中元素的所有可数并构成的
  集族。根据 $\sigma$-代数对可数并的封闭性,所以 $\sigma \mathcal{C}\supseteq \mathcal{E}$。
  设 $(A_n)$ 构成 $E$ 的可数划分,即 $(A_n)$ 两两不相交且 $E=\bigcup_n A_n$。
  任取 $\bigcup_k A_{n_k}\in \mathcal{E}$,那么
  $E \smallsetminus \left(\bigcup_k A_{n_k}\right)$ 依然是某些 $(A_n)$
  的可数并,所以 $E \smallsetminus \left(\bigcup_k A_{n_k}\right)\in \mathcal{E}$,
  即 $\mathcal{E}$ 对补封闭,所以 $\mathcal{E}$ 是 $\sigma$-代数,
  所以 $\mathcal{E}=\sigma \mathcal{C}$。

  (c) 令 $\mathcal{E}$ 为 $\mathbb{R}$ 的可数子集和补集可数的子集构成的子集族。
  显然 $\mathcal{E}\subseteq \sigma \mathcal{C}$ 且不难验证 $\mathcal{E}$
  是一个 $\sigma$-代数(可数个可数集的并是可数集),所以 $\mathcal{E}=\sigma \mathcal{C}$。
\end{solution}

\begin{exercise}{$\mathbb{R}$ 上的 Borel $\sigma$-代数}{}
  $\mathbb{R}=(-\infty,+\infty)$ 的任意开子集都是开区间的可数并。使用这一事实证明
  $\mathcal{B}(\mathbb{R})$ 由所有开区间构成的子集族生成。
\end{exercise}
\begin{proof}
  设 $\mathcal{C}$ 为所有开区间构成的子集族,$\mathcal{T}$ 为所有开集构成的子集族(即 $\mathbb{R}$ 上的拓扑)。
  显然 $\mathcal{C}\subseteq \mathcal{T}$,
  所以 $\sigma \mathcal{C}\subseteq \sigma \mathcal{T}=\mathcal{B}(\mathbb{R})$。由于
  $\mathcal{T}$ 中集合都是 $\mathcal{C}$ 中集合的可数并,所以 $\mathcal{T}\subseteq \sigma \mathcal{C}$,
  这表明 $\mathcal{B}(\mathbb{R})=\sigma \mathcal{T}\subseteq \sigma \mathcal{C}$。
  所以 $\mathcal{B}(\mathbb{R})=\sigma \mathcal{C}$ 由所有开区间构成的子集族生成。
\end{proof}

\begin{exercise}{$\mathbb{R}$ 上的 Borel $\sigma$-代数}{}
  证明:$\mathbb{R}$ 中的任意区间都是 Borel 集。特别的,$(-\infty,x),(-\infty,x],(x,y],[x,y]$ 都是 Borel 集。
  对于每个 $x$,单点集 $\{x\}$ 也是 Borel 集。 
\end{exercise}
\begin{proof}
  只需注意到
  \begin{gather*}
    (-\infty,x)=\bigcup_{n=1}^\infty \left(-n+x,x\right), 
    (-\infty, x]=\bigcap_{n=1}^\infty \left(-\infty,x+\frac{1}{n}\right),\\
    (x,y]=\bigcap_{n=1}^\infty \left(x,y+\frac{1}{n}\right),
    [x,y]=\bigcap_{n=1}^\infty\left(x-\frac{1}{n},y\right],
    \{x\}=\bigcap_{n=1}^\infty\left(x-\frac{1}{n},x\right].
  \end{gather*}
  所以上述集合都是 Borel 集,对于其他的区间同理。
\end{proof}

\begin{exercise}{$\mathbb{R}$ 上的 Borel $\sigma$-代数}{}
  证明 $\mathcal{B}(\mathbb{R})$ 可以由以下任意一种集族生成(实际上还有很多可能):
  \begin{alphenum}[nosep]
    \item 所有形如 $(-\infty, x]$ 的区间构成的子集族。
    \item 所有形如 $(x,y]$ 的区间构成的子集族。
    \item 所有形如 $[x,y]$ 的区间构成的子集族。
    \item 所有形如 $(x,+\infty)$ 的区间构成的子集族。
  \end{alphenum}
  此外,在每种情况中 $x,y$ 可以被限制为有理数。
\end{exercise}
\begin{proof}
  (a) 记该集族为 $\mathcal{C}$,由上题,这样的区间已经是 Borel 集,
  所以 $\sigma\mathcal{C}\subseteq \mathcal{B}(\mathbb{R})$。任取 $\mathbb{R}$
  的开区间 $(x,y)$,有
  \[
    (x,y)=(-\infty,x]^c\cap (-\infty, y)=
    (-\infty,x]^c\cap\bigcup_{n=1}^\infty \left(-\infty,y-\frac{1}{n}\right]\in\sigma \mathcal{C},
  \]
  而 $\mathcal{B}(\mathbb{R})$ 由所有开区间生成,所以 $\mathcal{B}(\mathbb{R})\subseteq \sigma \mathcal{C}$。
  所以 $\mathcal{B}(\mathbb{R})=\sigma \mathcal{C}$。

  (b) (c) (d) 完全同理。
\end{proof}

\begin{exercise}{迹空间}{}
  令 $(E,\mathcal{E})$ 是可测空间,固定 $D\subseteq E$,令
  \[
    \mathcal{D}=\mathcal{E}\cap D=\{A\cap D:A\in \mathcal{E}\}.  
  \]
  证明 $\mathcal{D}$ 是 $D$ 上的 $\sigma$-代数,被称为 $\mathcal{E}$
  在 $D$ 上的\emph{迹}。$(D,\mathcal{D})$ 也被称为 $(E,\mathcal{E})$
  在 $D$ 上的迹。
\end{exercise}
\begin{proof}
  任取 $A\cap D\in \mathcal{D}$,其中 $A\in \mathcal{E}$,那么
  \[
    D \smallsetminus (A\cap D)= (E \smallsetminus A)\cap D\in \mathcal{D},
  \]
  所以 $\mathcal{D}$ 对补封闭。任取 $(A_n\cap D)\subseteq \mathcal{D}$,那么
  \[
    \bigcup_{n=1}^\infty  (A_n\cap D)=\left(\bigcup_{n=1}^\infty A_n\right)\cap D\in \mathcal{D},
  \]
  所以 $\mathcal{D}$ 对可数并封闭。
\end{proof}

\begin{exercise}{子集的 Borel $\sigma$-代数是迹}{}
  设 $(E,\mathcal{T})$ 是拓扑空间,$(D,\mathcal{T}_D)$ 是
  子空间。证明 $D$ 上的 Borel $\sigma$-代数 $\mathcal{B}_D$
  与 $D$ 在 $E$ 上的迹 $\mathcal{B}_E\cap D$ 相同。
\end{exercise}
\begin{proof}
  由于 $\mathcal{T}_D=\mathcal{T}\cap D\subseteq \mathcal{B}_E\cap D$,
  由上题 $\mathcal{B}_E\cap D$ 是 $\sigma$-代数,所以 $\mathcal{B}_D\subseteq\mathcal{B}_E\cap D$。
  记
  \[
    \mathcal{C}=\{A\subseteq E: A\cap D\in \mathcal{B}_D\},  
  \]
  那么 $\mathcal{T}\subseteq \mathcal{C}$。我们只需要证明 $\mathcal{C}$
  是 $E$ 上的 $\sigma$-代数,那么就有 $\mathcal{C}\supseteq \sigma \mathcal{T}=\mathcal{B}_E$,即
  $\mathcal{B}_D\supseteq \mathcal{C}\cap D\supseteq \mathcal{B}_E\cap D$。
  任取 $A\in \mathcal{C}$,那么 $(E \smallsetminus A)\cap D=D \smallsetminus (A\cap D)\in \mathcal{B}_D$,
  所以 $E \smallsetminus A\in \mathcal{C}$。任取 $(A_n)\subseteq \mathcal{C}$,那么
  \[
    \left(\bigcup_{n=1}^\infty A_n\right)\cap D=\bigcup_{n=1}^\infty (A_n\cap D)
    \in \mathcal{B}_D,  
  \]
  所以 $\bigcup_n A_n\in \mathcal{C}$。这就表明 $\mathcal{C}$
  是 $E$ 上的 $\sigma$-代数。
\end{proof}


\section{可测函数}

\subsubsection{可测函数}

令 $(E,\mathcal{E})$ 和 $(F,\mathcal{F})$ 是可测空间,映射 $f:E\to F$ 如果使得任取 $B\in \mathcal{F}$,有
$f^{-1}B\in \mathcal{E}$,那么我们说 $f$ 相对于 $\mathcal{E}$ 和 $\mathcal{F}$ \emph{可测}。

\begin{proposition}\label{prop:equivalent condition of measurable}
  映射 $f:E\to F$ 相对于 $\mathcal{E}$ 和 $\mathcal{F}$ 可测当且仅当对于任意生成 $\mathcal{F}$ 的子集族
  $\mathcal{F}_0$,任取 $B\in \mathcal{F}_0$,有 $f^{-1}B\in \mathcal{E}$。
\end{proposition}
\begin{proof}
  必要性显然。下证充分性。设 $\mathcal{F}_0$ 使得 $\mathcal{F}=\sigma \mathcal{F}_0$,且
  对于任意的 $B\in \mathcal{F}_0$ 有 $f^{-1}B\in \mathcal{E}$。记
  \[
    \mathcal{F}_1=\{A\in \mathcal{F}: f^{-1}A\in \mathcal{E}\},
  \]
  显然 $\mathcal{F}_0\subseteq \mathcal{F}_1\subseteq \mathcal{F}$。由于
  \[
    f^{-1}\left(F \smallsetminus A\right)=E \smallsetminus(f^{-1}A),\quad
    f^{-1}\left(\bigcup_{i\in I}A_i\right)=\bigcup_{i\in I} f^{-1}A_i,
  \]
  所以 $\mathcal{F}_1$ 是 $\sigma$-代数,所以 $\mathcal{F}=\mathcal{F}_1$,即 $f$ 
  相对于 $\mathcal{E}$ 和 $\mathcal{F}$ 可测。
\end{proof}

\begin{proposition}
  给定可测空间 $(E,\mathcal{E}),(F,\mathcal{F}),(G,\mathcal{G})$,如果 $f$ 相对于 $\mathcal{E}$ 和 
  $\mathcal{F}$ 可测,$g$ 相对于 $\mathcal{F}$ 和 $\mathcal{G}$ 可测,那么复合
  $g\circ f$ 相对于 $\mathcal{E}$ 和 $\mathcal{G}$ 可测。
\end{proposition}
\begin{proof}
  任取 $C\in \mathcal{G}$,有
  \[
    (g\circ f)^{-1}(C)=f^{-1}\left(g^{-1}(C)\right),  
  \]
  $g$ 可测表明 $g^{-1}(C)\in \mathcal{F}$,$f$ 可测表明 $f^{-1}\left(g^{-1}(C)\right)\in \mathcal{E}$,
  所以 $g\circ f$ 相对于 $\mathcal{E}$ 和 $\mathcal{G}$ 可测。
\end{proof}

\subsubsection{数值函数}

令 $(E,\mathcal{E})$ 是可测空间。回顾实数及扩充实数 $\mathbb{R}=(-\infty,+\infty)$,
$\bar{\mathbb{R}}=[-\infty,+\infty]$,$\mathbb{R}_+=[0,+\infty)$,
$\bar{\mathbb{R}}_+=[0,+\infty]$。$E$ 上的\emph{数值函数}指的是
从 $E$ 到 $\bar{\mathbb{R}}$ 或者 $\bar{\mathbb{R}}$ 的子集的映射。
如果这个映射的值在 $\mathbb{R}$ 中,那么我们一般称其为\emph{实值函数}。

$E$ 上的数值函数如果相对于 $\mathcal{E}$ 和 $\mathcal{B}(\bar{\mathbb{R}})$
可测,那么我们说其是 $\mathcal{E}$-可测的。如果 $E$
是拓扑空间且 $\mathcal{E}=\mathcal{B}(E)$,那么 $\mathcal{E}$-可测函数
被称为\emph{Borel 函数}。
下面的命题是 \autoref{prop:equivalent condition of measurable} 的直接结果。

\begin{proposition}
  映射 $f:E\to\bar{\mathbb{R}}$ 是 $\mathcal{E}$-可测的当且仅当
  对于每个 $r\in \mathbb{R}$,$f^{-1}[-\infty,r]\in \mathcal{E}$。
\end{proposition}

上述命题中的 $[-\infty,r]$ 可以改为 $[-\infty,r)$,$[r,\infty]$,
$(r,\infty]$ 中的任意一种。

\subsubsection{函数的正部分和负部分}

对于 $a,b\in\bar{\mathbb{R}}$,我们记 $a\vee b$ 为 $a$ 和 $b$
中的最大者,$a\wedge b$ 为 $a$ 和 $b$ 中的最小者。
对于函数 $f,g$,用 $f\vee g$ 表示函数 $x\mapsto f(x)\vee g(x)$。
令 $(E,\mathcal{E})$ 是可测空间,$f$ 是 $E$ 上的数值函数。
那么
\[
  f^+=f\vee 0,\quad f^-=-(f\wedge 0)  
\]
都是正值函数并且 $f=f^+-f^-$。函数 $f^+$ 被称为 $f$ 的\emph{正部分},
$f^-$ 被称为 $f$ 的\emph{负部分}。

\begin{proposition}
  函数 $f$ 是 $\mathcal{E}$-可测的当且仅当 $f^+$ 和 $f^-$
  都是 $\mathcal{E}$-可测的。
\end{proposition}
\begin{proof}
  若 $f$ 是 $\mathcal{E}$-可测的。任取 $r\in \mathbb{R}$,
若 $r< 0$,则 $\left(f^+\right)^{-1}[-\infty,r]=\emptyset\in \mathcal{E}$。
  若 $r\geq 0$,则
  \[
    \left(f^+\right)^{-1}[-\infty,r]=E \smallsetminus \left(f^+\right)^{-1}(r,\infty]
    =E \smallsetminus f^{-1}(r,\infty],
  \]
  由于 $(r,\infty]$ 是 Borel 集,所以 $\left(f^+\right)^{-1}[-\infty,r]\in \mathcal{E}$。
  综合起来,$f^+$ 是 $\mathcal{E}$-可测的。同理可证 $f^-$ 是 $\mathcal{E}$-可测的。

  若 $f^+$ 和 $f^-$ 都是 $\mathcal{E}$-可测的。任取 $r\in \mathbb{R}$,若 $r<0$,那么
  \[
    f^{-1}[-\infty,r]=  \left(f^-\right)^{-1}[-r,\infty]\in \mathcal{E}.
  \]
  若 $r\geq 0$,那么
  \[
    f^{-1}[-\infty,r]= E \smallsetminus f^{-1}(r,\infty]
    =E \smallsetminus \left(f^+\right)^{-1}(r,\infty]\in \mathcal{E}.
  \]
  所以 $f$ 是 $\mathcal{E}$-可测的。
\end{proof}

\subsubsection{指示函数和简单函数}

令 $A\subseteq E$,定义 $A$ 的指示函数为 $1_A$:
\[
  1_A(x)=\begin{cases}
    1 & x\in A,\\
    0 & x\notin A.
  \end{cases}  
\]
对于 $1_E$,我们简记为 $1$。显然,$1_A$ 是 $\mathcal{E}$-可测的当且仅当
$A\in \mathcal{E}$。

$E$ 上的函数 $f$ 如果形如
\[
  f=\sum_{i=1}^n a_i 1_{A_i},  
\]
其中 $n\geq 1$,$a_1,\dots,a_n\in \mathbb{R}$,$A_1,\dots,A_n$ 是可测集,
那么我们说 $f$ 是\emph{简单函数}。
在这个定义中,若 $A_i\cap A_j\ne\emptyset$,那么我们可以将
$a_i1_{A_i}+a_j1_{A_j}$ 拆为
\[ 
  a_i1_{A_i \smallsetminus (A_i\cap A_j)}+(a_i+a_j)1_{A_i\cap A_j}+a_j 1_{A_j \smallsetminus (A_i\cap A_j)},
\]
所以我们可以假设 $A_i$ 两两不相交。此外,如果 $\bigcup_i A_i\neq E$,记
$B=E \smallsetminus\bigcup_i A_i\in \mathcal{E}$,那么
\[
  f=  \sum_{i=1}^n a_i 1_{A_i}+0\cdot 1_{B},
\]
所以我们还可以假设 $\bigcup_i A_i=E$。这意味着对于一个简单函数 $f$,
总存在整数 $m$,不同的实数 $b_1,\dots,b_m$ 和 $E$ 的可测划分
$\{B_1,\dots,B_m\}$ 使得 $f=\sum_{i=1}^m b_i 1_{B_i}$,
这种表示被称为简单函数 $f$ 的\emph{标准型}。

利用简单函数的标准型,很容易验证简单函数都是 $\mathcal{E}$-可测的。
反之,若 $f$ 是 $\mathcal{E}$-可测的,只有有限个取值且值为实数,那么
$f$ 为简单函数。特别地,任意常值函数是简单函数。最后,如果 $f,g$
是简单函数,那么
\[
  f+g,\quad f-g,\quad fg,\quad f/g,\quad f\vee g,\quad f\wedge g  
\]
都是简单函数,其中 $f/g$ 要求 $g$ 的值始终非零。

\subsubsection{函数列的极限}

令 $(f_n)$ 是 $E$ 上的一列数值函数,我们可以逐点定义
\begin{equation}\label{eq:inf and sup}
  \inf f_n,\quad \sup f_n,\quad \liminf f_n,\quad \limsup f_n,
\end{equation}
例如,$\inf f_n$ 将 $x\in E$ 送到实数列 $(f_n(x))$ 的下确界。
如果
\[
  \liminf f_n=\limsup f_n=f,
\]
那么我们说 $(f_n)$ 有逐点极限 $f$,记为 $f=\lim f_n$ 或者 $f_n\to f$。

如果 $(f_n)$ 单调递增,即 $f_1\leq f_2\leq \cdots$,那么根据单调有界定理,$\lim f_n$
存在且等于 $\sup f_n$。此时我们用 $f_n\nearrow f$ 来表示 $(f_n)$ 单调递增且有极限 $f$。
类似地,用 $f_n\searrow f$ 来表示 $(f_n)$ 单调递减且有极限 $f$。

下面的定理表明可测函数类对极限操作是封闭的。
\begin{theorem}\label{thm:lim of measurable function is measurable}
  令 $(f_n)$ 是一列 $\mathcal{E}$-可测函数,那么 \eqref{eq:inf and sup} 中的四个函数都是
  $\mathcal{E}$-可测的。此外,如果 $\lim f_n$ 存在,那么 $\lim f_n$ 也是 $\mathcal{E}$-可测的。
\end{theorem}
\begin{proof}
  记 $g=\sup f_n$。任取 $r\in \mathbb{R}$,注意到 $g(x)\leq r$ 当且仅当对于所有 $n$ 有 $f_n(x)\leq r$。
  所以
  \[
    g^{-1}[-\infty,r]=\bigcap_{n=1}^\infty f_n^{-1}[-\infty,r],
  \]
  $f_n$ 可测表明 $f_n^{-1}[-\infty,r]\in \mathcal{E}$,所以 $g^{-1}[-\infty,r]\in \mathcal{E}$,即
  $g$ 可测。

  对于 $\inf f_n$,我们有 $\inf f_n=-\sup (-f_n)$,所以 $\inf f_n$ 也可测。
  最后,注意到
  \[
    \liminf f_n=\sup\limits_m \mathop{\vphantom{\sup}\inf}_{n\geq m} f_n,\quad
    \limsup f_n=\mathop{\vphantom{\sup}\inf}_{m}\sup_{n\geq m} f_n,
  \] 
  所以 $\liminf f_n$ 和 $\limsup f_n$ 可测。若二者相等,那么 $\lim f_n$ 也可测。
\end{proof}

\subsubsection{可测函数的逼近}

\begin{lemma}
  对于 $n\in \mathbb{N}^*$,令
  \[
    d_n(r)=\sum_{k=1}^{n2^n}\frac{k-1}{2^n}1_{\left[\frac{k-1}{2^n},\frac{k}{2^n}\right)}(r)
    +n1_{[n,\infty]}(r),\quad r\in \bar{\mathbb{R}}_+.
  \]
  那么,$d_n(r)$ 是 $\bar{\mathbb{R}}_+$ 上单调递增的简单函数,并且对于每个 $r\in\bar{\mathbb{R}}_+$,
  $d_n(r)$ 随着 $n$ 的增大是的单调递增的。
\end{lemma}
\begin{proof}
  显然 $d_n(r)$ 是单调递增的简单函数,我们证明任取 $r\in \bar{\mathbb{R}}_+$,$d_n(r)$
  是单调递增的。若 $r=\infty$,那么 $d_n(r)=n$ 是单调递增的。现在假设 $r\in \mathbb{R}_+$,
  那么存在正整数 $m$ 使得 $m\leq r<m+1$,所以当 $n\leq m$ 的时候,$d_n(r)=n$ 单调递增。
  当 $n>m$ 的时候,直观来看,$d_n$ 将区间 $[0,n]$ 等分为 $n2^{n}$ 份,$r\in [0,n]$
  表明一定存在唯一的 $k_n$ 使得 $(k_n-1)/2^n\leq r <k_n/2^n$,可以发现 $k_{n}$ 满足 
  递推关系 $k_{n+1}=2k_n-1$ 或者 $k_{n+1}=2k_n$,这表明
  \[
    d_{n+1}(r)= \frac{k_{n+1}-1}{2^{n+1}}\geq\frac{2k_n-2}{2^{n+1}}=\frac{k_n-1}{2^n}=d_n(r).
  \]
  综上,$d_n(r)$ 随着 $n$ 的增大是的单调递增的。
\end{proof}

\begin{theorem}\label{thm:approximation of measurable function}
  $E$ 上的正值函数是 $\mathcal{E}$-可测的当且仅当其是一列单调递增的正值简单函数序列
  的极限。
\end{theorem}
\begin{proof}
  充分性来源于 \autoref{thm:lim of measurable function is measurable}。
  对于必要性,设 $f:E\to\bar{\mathbb{R}}_+$ 是 $\mathcal{E}$-可测的正值函数。
  记 $d_n$ 为上述引理中的函数,令 $f_n=d_n\circ f$。那么 $f_n$ 是正值的 $\mathcal{E}$-可测函数,
  并且其取值只有有限个,所以是简单函数。由于 $(d_n)$ 单调递增,所以 $(f_n)$ 单调递增。
  对于任意 $x\in E$,由于 $f_n(x)=d_n(f(x))$,所以 $n\to\infty$ 的时候
  $f_n(x)\to f(x)$,故 $f=\lim f_n$。
\end{proof}

\subsubsection{函数的单调类}

令 $\mathcal{M}$ 为 $E$ 上数值函数的一个集合,记 $\mathcal{M}_+$ 为 $\mathcal{M}$
中正值函数组成的子集,$\mathcal{M}_b$ 为 $\mathcal{M}$ 中有界函数组成的子集。

如果 $\mathcal{M}$ 包含常值函数 $1$,$\mathcal{M}_b$ 构成 $\mathbb{R}$ 上的向量空间
以及 $\mathcal{M}_+$ 在递增极限下封闭,那么我们说 $\mathcal{M}$ 是一个\emph{单调类}。
更准确地说,$\mathcal{M}$ 是单调类当且仅当:
\begin{alphenum}
  \item $1\in \mathcal{M}$,
  \item 若 $f,g\in \mathcal{M}_b$ 且 $a,b\in \mathbb{R}$,则 $af+bg\in \mathcal{M}$,
  \item 若 $(f_n)\subseteq \mathcal{M}_+$ 且 $f_n\nearrow f$,那么
  $f\in \mathcal{M}$。
\end{alphenum}

下面的定理通常被用于证明所有 $\mathcal{E}$-可测函数拥有的某一性质。

\begin{theorem}
  令 $\mathcal{M}$ 是 $E$ 上函数的单调类。假设对于某个生成 $\mathcal{E}$
  的 p-系 $\mathcal{C}$,任取 $A\in \mathcal{C}$,有 $1_A\in \mathcal{M}$。那么,
  $\mathcal{M}$ 包含所有的正值 $\mathcal{E}$-可测函数以及所有的有界 $\mathcal{E}$-可测函数。
\end{theorem} 
\begin{proof}
  首先证明对于任意的 $A\in \mathcal{E}$ 有 $1_A\in \mathcal{M}$。记
  \[
    \mathcal{D}=\{A\in \mathcal{E}:1_A\in \mathcal{M}\}.
  \]
  由于 $1=1_E\in \mathcal{M}$,所以 $E\in \mathcal{D}$。任取 $A,B\in \mathcal{D}$ 且 $A\supseteq B$,
  那么 $1_{A \smallsetminus B}=1_A-1_B\in \mathcal{M}$,这表明 $A \smallsetminus B\in \mathcal{D}$。
  设 $(A_n)\subseteq \mathcal{D}$ 且 $A_n\nearrow A$,那么 $(1_{A_n})\subseteq \mathcal{M}_+$
  且 $1_{A_n}\nearrow 1_A$,所以 $1_A\in \mathcal{M}$,这表明 $A\in \mathcal{D}$。
  这表明 $\mathcal{D}$ 是 d-系。由于 $\mathcal{C}$ 是 p-系且 $\mathcal{C}\subseteq \mathcal{D}$,
  根据单调类定理 \ref{thm:monotone class theorem},$\mathcal{D}$ 包含 $\sigma \mathcal{C}=\mathcal{E}$,
  所以对于任意的 $A\in \mathcal{E}$ 有 $A\in \mathcal{D}$,即 $1_A\in \mathcal{M}$。

  再根据单调类的定义 (b),$\mathcal{M}$ 包含所有的简单函数。

  令 $f$ 是正值 $\mathcal{E}$-可测函数,根据 \autoref{thm:approximation of measurable function},
  $f$ 是函数序列 $(f_n)$ 的极限,其中 $f_n$ 是递增的正值简单函数,即 $(f_n)\subseteq \mathcal{M}_+$。
  根据单调类的定义 (c),有 $f\in \mathcal{M}$。

  令 $g$ 是有界 $\mathcal{E}$-可测函数,那么 $g^+$ 和 $g^-$ 都是正值 $\mathcal{E}$-可测函数,所以
  $g^+,g^-\in \mathcal{M}$。显然 $g^+,g^-$ 也都是有界的,根据单调类的定义 (b),所以
  $g=g^+-g^-\in \mathcal{M}$。
\end{proof}

\subsubsection{标准可测空间}

令 $(E,\mathcal{E})$ 和 $(F,\mathcal{F})$ 是可测空间。如果 $f:E\to F$ 是双射的
相对于 $\mathcal{E}$ 和 $\mathcal{F}$ 的可测函数,并且其逆映射 $f^{-1}:F\to E$
是相对于 $\mathcal{F}$ 和 $\mathcal{E}$ 的可测函数,那么我们说 $f$ 是\emph{同构}。

如果可测空间 $(E,\mathcal{E})$ 同构于 $(F,\mathcal{B}_F)$,其中 $F$ 是 $\mathbb{R}$
的某个 Borel 子集,那么我们说 $(E,\mathcal{E})$ 是\emph{标准可测空间}。
标准可测空间有非常多。如果 $E$ 是完备度量空间,那么 $(E,\mathcal{B}_E)$
是标准可测空间。如果 $E$ 是波兰空间,即可分的可完备度量化的拓扑空间,那么
$(E,\mathcal{B}_E)$ 是标准可测空间。如果 $E$ 是可分的 Banach 空间,
那么 $(E,\mathcal{B}_E)$ 是标准可测空间。

显然,$[0,1]$ 和它的 Borel $\sigma$-代数构成标准可测空间。
$\{1,2,\dots,n\}$ 和它的离散 $\sigma$-代数构成标准可测空间。
$\mathbb{N}=\{0,1,2,\dots\}$ 和它的离散 $\sigma$-代数构成标准可测空间。
一个深刻的结果是,任意标准可测空间都同构于上述三者之一。


\section{测度}

令 $(E,\mathcal{E})$ 是可测空间,$(E,\mathcal{E})$ 上的\emph{测度}
指的是一个映射 $\mu:\mathcal{E}\to\bar{\mathbb{R}}_+$,其满足:
\begin{alphenum}
  \item $\mu(\emptyset)=0$,
  \item 对于不相交的子集列 $(A_n)\subseteq \mathcal{E}$,有 
  $\mu(\bigcup_n A_n)=\sum_n \mu(A_n)$。
\end{alphenum}
条件 (b) 被称为\emph{可列可加性}。需要注意 $\mu(A)$ 总是为正数且可以
为 $+\infty$。数 $\mu(A)$ 被称为 $A$ 的\emph{测度},也简记为 $\mu A$。

一个\emph{测度空间}指的是三元组 $(E,\mathcal{E},\mu)$,其中 $(E,\mathcal{E})$
是可测空间,$\mu$ 是 $(E,\mathcal{E})$ 上的测度。

\subsubsection{例子}

\begin{example}[Dirac 测度]
  令 $(E,\mathcal{E})$ 是可测空间,固定 $x\in E$。对于每个 $A\in \mathcal{E}$,
  令
  \[
    \delta_x(A)=\begin{cases}
      1 & x\in A,\\
      0 & x\notin A.
    \end{cases}  
  \]
  那么 $\delta_x$ 是 $(E,\mathcal{E})$ 上的测度,被称为\emph{Dirac 测度}。
  直观来看,其基于一个集合 $A$ 是否含有特定元素 $x$ 来给出这个集合的“大小”。
\end{example}

\begin{example}[计数测度]
  令 $(E,\mathcal{E})$ 是可测空间,固定 $D\subseteq E$。对于每个
  $A\in \mathcal{E}$,令 $\nu(A)$ 是 $A\cap D$ 中点的个数,
  此时 $\nu$ 是 $(E,\mathcal{E})$ 上的测度,被称为\emph{计数测度}。
  通常,集合 $D$ 被选取为可数集,在这种情况下
  \[
    \nu(A)=\sum_{x\in D}\delta_x(A),\quad A\in \mathcal{E}.  
  \]
\end{example}

\begin{example}[离散测度]
  令 $(E,\mathcal{E})$ 是可测空间,固定可数子集 $D\subseteq E$。
  对于每个 $x\in D$,分配一个正数 $m(x)$。定义
  \[
    \mu(A)=\sum_{x\in D} m(x)\delta_x(A),\quad A\in \mathcal{E}.  
  \]
  那么 $\mu$ 是 $(E,\mathcal{E})$ 上的测度,被称为\emph{离散测度}。
  我们可能会把 $m(x)$ 理解为点 $x$ 的质量,那么 $\mu(A)$ 就是集合 $A$
  的质量。特别地,如果 $(E,\mathcal{E})$ 是离散可测空间,那么
  每个测度 $\mu$ 都有这种形式。
\end{example}

\begin{example}[Lebesgue 测度]
  $(\mathbb{R},\mathcal{B}_{\mathbb{R}})$ 上的测度 $\mu$ 如果对于每个
  区间 $A$ 都满足 $\mu(A)$ 为 $A$ 的长度,那么我们说 $\mu$ 是\emph{Lebesgue 测度}。
  类似地,$\mathbb{R}^2$ 上的 Lebesgue 测度是“面积”测度,
  $\mathbb{R}^3$ 上的 Lebesgue 测度是“体积”测度等等。
  我们将它们记作 $\Leb$。
\end{example}

\subsubsection{测度的性质}

\begin{proposition}
  令 $\mu$ 是可测空间 $(E,\mathcal{E})$ 上的测度,那么对于任意
  可测集 $A,B$ 和 $A_1,A_2,\dots$,有:
  \begin{description}[nosep,font=\sffamily\mdseries,itemindent=0pt]
    \item[有限可加性] $A\cap B=\emptyset\Rightarrow \mu(A\cup B)=\mu(A)+\mu(B)$。
    \item[单调性] $A\subseteq B\Rightarrow \mu(A)\leq \mu(B)$。
    \item[连续性] $A_n\nearrow A\Rightarrow \mu(A_n)\nearrow \mu(A)$。
    \item[Boole 不等式]  $\mu(\bigcup_n A_n)\leq \sum_n \mu(A_n)$。  
  \end{description}
\end{proposition}
\begin{proof}
  有限可加性是可列可加性的特殊情况,取 $A_1=A,A_2=B,A_3=A_4=\cdots=\emptyset$
  即可。若 $A\subseteq B$,由于 $\mathcal{E}$ 是 d-系,所以 $B \smallsetminus A\in \mathcal{E}$,
  所以
  \[
    \mu(B)=\mu(A\cup(B \smallsetminus A))= \mu(A)+\mu(B \smallsetminus A)
    \geq \mu(A).
  \]
  若 $A_n\nearrow A$,令 $B_1=A_1$,$B_n=A_{n}\smallsetminus A_{n-1}$,那么
  $B_n$ 互不相交且 $\bigcup_{k=1}^n B_k=A_n$,所以
  \[
    \lim \mu(A_n)=\lim\mu\left(\bigcup_{k=1}^n B_k\right)  
    =\lim \sum_{k=1}^n \mu(B_k)=\sum_{k=1}^\infty \mu(B_k)
    =\mu(A).
  \]
  对于 Boole 不等式,注意到
  \[
    \mu(A\cup B)=\mu(A\cup (B \smallsetminus A))=\mu(A)+\mu(B \smallsetminus A) 
    \leq \mu(A)+\mu (B), 
  \]
  所以归纳可得
  \[
    \mu\left(\bigcup_{k=1}^n A_k\right)\leq \sum_{k=1}^n\mu(A_k),  
  \]
  令 $n\to\infty$,左边根据连续性即可得到 Boole 不等式。
\end{proof}

\subsubsection{有限测度}

令 $\mu$ 是可测空间 $(E,\mathcal{E})$ 上的测度,如果 $\mu(E)<\infty$,
那么 $\mu$ 被称为\emph{有限测度},根据单调性,此时对于任意 $A\in \mathcal{E}$,
都有 $\mu(A)<\infty$。如果 $\mu(E)=1$,那么 $\mu$ 被称为\emph{概率测度}。
如果存在 $E$ 的可测划分 $(E_n)$ 使得 $\mu(E_n)<\infty$,那么
$\mu$ 被称为\emph{$\sigma$-有限测度}。如果存在一列有限测度 $\mu_n$
使得 $\mu=\sum_n \mu_n$,那么 $\mu$ 被称为\emph{$\Sigma$-有限测度}。
有限测度都是 $\sigma$-有限的,$\sigma$-有限测度都是 $\Sigma$-有限的。

\begin{proposition}
  令 $(E,\mathcal{E})$ 是可测空间,$\mu,\nu$ 是两个有限测度且 $\mu(E)=\nu(E)$,如果
  $\mu,\nu$ 在生成 $\mathcal{E}$ 的某个 p-系上取值相同,那么
  $\mu=\nu$。
\end{proposition}
\begin{proof}
  设 $\mathcal{C}$ 是 p-系且 $\mathcal{E}=\sigma \mathcal{C}$,
  任取 $A\in \mathcal{C}$ 有 $\mu(A)=\nu(A)$。令
  \[
    \mathcal{D}=\{A\in \mathcal{E}:\mu(A)=\nu(A)\},  
  \]
  那么 $\mathcal{C}\subseteq \mathcal{D}$。如果我们证明 $\mathcal{D}$
  是 d-系,那么根据单调类定理,就有 $\mathcal{E}=\sigma \mathcal{C}\subseteq \mathcal{D}$,
  即任取 $A\in \mathcal{E}$ 有 $\mu(A)=\nu(A)$。下面我们证明 $\mathcal{D}$
  是 d-系。由于 $\mu(E)=\nu(E)$,所以 $E\in \mathcal{D}$。
  若 $A,B\in \mathcal{D}$ 且 $A\supseteq B$,那么
  \[
    \mu(B)+\mu(A \smallsetminus B)= \mu(A)=\nu(A)=\nu(B)+\mu(A \smallsetminus B),
  \]
  由于 $\mu(B)=\nu(B)$,所以 $\mu(A \smallsetminus B)=\nu (A \smallsetminus B)$,
  即 $A \smallsetminus B\in \mathcal{D}$。
  任取 $(A_n)\subseteq \mathcal{D}$ 且 $A_n\nearrow A$,根据连续性,所以
  $\mu(A_n)\nearrow \mu(A)$ 以及 $\nu(A_n)\nearrow \nu(A)$,所以
  \[
    \mu(A)=\lim \mu(A_n)=\lim \nu(A_n)=\nu(A),  
  \]
  所以 $A\in \mathcal{D}$。这就证明了 $\mathcal{D}$ 是 d-系。
\end{proof}

\begin{corollary}
  令 $\mu,\nu$ 是 $\left(\bar{\mathbb{R}},\mathcal{B}\left(\bar{\mathbb{R}}\right)\right)$
  上的概率测度,那么 $\mu=\nu$ 当且仅当对于任意的 $r\in \mathbb{R}$ 有
  $\mu[-\infty,r]=\nu[-\infty, r]$。
\end{corollary}

\subsubsection{原子,纯原子测度和非原子测度}

令 $(E,\mathcal{E})$ 是可测空间,假设对于每个 $x\in E$,单点集 $\{x\}\in \mathcal{E}$,
这一点对于所有的标准可测空间都是成立的。令 $\mu$ 是 $(E,\mathcal{E})$ 上的测度,
如果点 $x$ 使得 $\mu\{x\}>0$,那么 $x$ 被称为 $\mu$ 的一个\emph{原子}。如果 $\mu$
没有任何原子,那么 $\mu$ 被称为\emph{非原子测度}。如果 $\mu$ 的原子的集合 $D$
是可数集并且 $\mu(E \smallsetminus D)=0$,那么 $\mu$ 被称为\emph{纯原子测度}。
例如,Lebesgue 测度是非原子测度,Dirac 测度是纯原子测度(其只有一个原子),离散测度
是纯原子测度。

\begin{proposition}
  令 $\mu$ 是 $(E,\mathcal{E})$ 上的 $\Sigma$-有限测度,那么
  \[
    \mu=\lambda+\nu,
  \]
  其中 $\lambda$ 是非原子测度,$\nu$ 是纯原子测度。
\end{proposition}


\subsubsection{完备性,零测集}

令 $(E,\mathcal{E},\mu)$ 是测度空间,如果可测集 $B$ 使得 $\mu(B)=0$,那么 $B$
被称为\emph{零测集}。$E$ 的任意子集如果被一个可测的零测集包含,那么也被称为\emph{零测集}。
如果 $E$ 的每个零测集都是可测集,那么我们说这个测度空间是\emph{完备的}。对于不完备的
测度空间,下面的结果表明可以通过包含所有的零测集来扩大 $\mathcal{E}$ 以及
$\mu$ 来得到一个完备测度空间。测度空间 $(E,\bar{\mathcal{E}},\bar\mu)$ 被称为
$(E,\mathcal{E},\mu)$ 的\emph{完备化}。当 $E=\mathbb{R}$,$\mathcal{E}=\mathcal{B}_{\mathbb{R}}$
和 $\mu=\Leb$ 的时候,$\bar{\mathcal{E}}$ 的元素被称为\emph{Lebesgue 可测集}。

\begin{proposition}
  令 $\mathcal{N}$ 是 $E$ 的所有零测子集的集合族,$\bar{\mathcal{E}}$ 为 $\mathcal{E}\cup \mathcal{N}$
  生成的 $\sigma$-代数,那么
  \begin{alphenum}[nosep]
    \item 每个 $B\in\bar{\mathcal{E}}$ 都形如 $B=A\cup N$,其中 $A\in \mathcal{E}$ 以及 $N\in \mathcal{N}$,
    \item 定义 $\bar\mu(A\cup N)=\mu(A)$,这给出了 $\bar{\mathcal{E}}$ 上的测度 $\bar\mu$,并且
    是唯一的满足 $\bar\mu(A)=\mu(A)\ (A\in \mathcal{E})$ 的测度,此时测度空间 $(E,\mathcal{E},\mu)$
    是完备的。
  \end{alphenum}
\end{proposition}

\begin{exercise}{限制和迹}{}
  令 $(E,\mathcal{E})$ 是可测空间,$\mu$ 是测度。令 $D\in \mathcal{E}$。
  \begin{alphenum}[nosep]
    \item 定义 $\nu(A)=\mu(A\cap D)$,证明 $\nu$ 是 $(E,\mathcal{E})$ 上的测度,
    被称为 $\mu$ 在 $D$ 上的迹。
    \item 令 $\mathcal{D}$ 为 $\mathcal{E}$ 在 $D$ 上的迹,对于 $A\in \mathcal{D}$,
    定义 $\nu(A)=\mu(A)$,证明 $\nu$ 是 $(D,\mathcal{D})$ 上的测度,被称为
    $\mu$ 在 $D$ 上的限制。
  \end{alphenum}
\end{exercise}





\end{document}
