\documentclass[fontset=none]{Notes}

\makeatletter
\DeclareRobustCommand{\em}{%
  \@nomath\em \if b\expandafter\@car\f@series\@nil
  \normalfont \else \bfseries \fi}
\makeatother

\usepackage{tikz-cd,wrapstuff}
\usepackage{fixdif,siunitx,tikz,nicematrix}
\usetikzlibrary{matrix,calc}

\ProvidesFile{font.def}

\setCJKmainfont{Source Han Serif SC}[
  UprightFont=*-Regular,
  BoldFont=*-Bold,
  ItalicFont=HYKaiTi S,
  ItalicFeatures={Scale=1.1}
]
\newCJKfontfamily[zhsong]\songti{Source Han Serif SC}[
  UprightFont=*-Regular,
  BoldFont=*-Bold,
  ItalicFont=HYKaiTi S,
  ItalicFeatures={Scale=1.1}
]
\setCJKsansfont{Source Han Sans SC}[
  UprightFont=*-Regular,
  BoldFont=*-Bold
]
\newCJKfontfamily[zhhei]\heiti{Source Han Sans SC}[
  UprightFont=*-Regular,
  BoldFont=*-Bold
]
\setCJKmonofont{HYFangSong S}[
  BoldFont=*,
  ItalicFont=*,
  BoldItalicFont=*
]
\newCJKfontfamily[zhfs]\fangsong{HYFangSong S}[
  BoldFont=*,
  ItalicFont=*,
  BoldItalicFont=*
]
\newCJKfontfamily[zhkai]\kaishu{HYKaiTi S}[
  BoldFont=*,
  ItalicFont=*,
  BoldItalicFont=*
]

\setmainfont{texgyretermes}[
  Extension=.otf,
  UprightFont=*-regular,
  BoldFont=*-bold,
  ItalicFont=*-italic,
  BoldItalicFont=*-bolditalic,
  SlantedFont=*-italic
]
%\setmathrm{texgyretermes}[
%  Extension=.otf,
%  UprightFont=*-regular,
%  BoldFont=*-bold,
%  ItalicFont=*-italic,
%  BoldItalicFont=*-bolditalic,
%  SlantedFont=*-italic
%]
\setsansfont{Cantarell}[
  UprightFont=* Regular,
  ItalicFont=* Italic,
  BoldFont=* Bold,
  BoldItalicFont=* Bold Italic,
  SmallCapsFont=Alegreya Sans SC
]
\setmonofont{Ubuntu Mono}[
  UprightFont=*,
  ItalicFont=* Italic,
  BoldFont=* Bold,
  BoldItalicFont=* Bold Italic
]
%\setmathfont{texgyretermes-math.otf}
%\setmathfont[range={\mathcal,\mathbfcal,\mathfrak},StylisticSet=1]{XITSMath-Regular.otf}
%\setmathfont[range={\mathbb}]{KpMath-Sans.otf}



\usepackage[subscriptcorrection,nofontinfo,mtpbb,mtpfrak]{mtpro2}

\tikzcdset{
  arrow style=tikz,
  diagrams={>={Straight Barb[scale=0.8]}}
}

\allowdisplaybreaks[1]

\newlength{\mymathln}
\newcommand{\aligninside}[2]{
  \settowidth{\mymathln}{#2}
  \mathmakebox[\mymathln]{#1}
}

\DeclareMathOperator\Spec{Spec}
\DeclareMathOperator\im{im}
\DeclareMathOperator\nil{nil}
\DeclareMathOperator\rad{rad}
\DeclareMathOperator\Ann{Ann}
\DeclareMathOperator\Max{Max}
\DeclareMathOperator\GL{GL}
\DeclareMathOperator\End{End}
\DeclareMathOperator\Int{Int}
\DeclareMathOperator\Tor{Tor}
\DeclareMathOperator\Frac{Frac}
\DeclareMathOperator\Tr{Tr}
\DeclareMathOperator\Hom{Hom}
\DeclareMathOperator\Leb{Leb}
\DeclareMathOperator\supp{supp}
\DeclareMathOperator\Id{Id}
\DeclareMathOperator\rk{rank}
\DeclareMathOperator\var{var}
\DeclareMathOperator\card{card}
\DeclareMathOperator\coker{coker}

\newcommand{\ideal}[1]{\mathfrak{#1}}
\newcommand{\mat}[1]{\mathbold{#1}}
\newcommand{\uline}{\underline{\hphantom{X}}}
\newcommand{\abs}[1]{\left|#1\right|}
\newcommand{\ulim}[1][]{\lim_{#1}\mathrel{\uparrow}}
\newcommand{\dlim}[1][]{\lim_{#1}\mathrel{\downarrow}}
\newcommand{\indicator}[1]{\mathbold 1_{#1}}
\newcommand{\alev}[1]{\text{$#1$ a.e.}}
\newcommand{\alsu}[1]{\text{$#1$ a.s.}}
\newcommand{\norm}[1]{\left\Vert#1\right\Vert}

\usepackage{enumitem}

\setlist[enumerate]{nosep}

%\DeclareMathAlphabet\mathcal{OMS}{cmsy}{m}{n}

\newlength\stextwidth
\newcommand\makesamewidth[3][c]{%
  \settowidth{\stextwidth}{#2}%
  \makebox[\stextwidth][#1]{#3}%
}



\begin{document}

\frontmatter

\tableofcontents

\mainmatter

% \chapter{测度和积分}

% \section{可测空间}

% 令 $E$ 是集合,$\mathcal{E}$ 是 $E$ 的一个子集族。若对于任意 $A,B\in\mathcal{E}$ 有
% $A\cap B\in\mathcal{E}$,那么我们说 $\mathcal{E}$ \emph{对交封闭}。如果 $\mathcal{E}$
% 中任意可数个集合的交还在 $\mathcal{E}$ 中,那么我们说 $\mathcal{E}$ 对可数交封闭。
% 类似地,我们可以定义对补封闭、对并封闭和对可数并封闭的概念。

% \subsubsection{$\sigma$-代数}

% 如果 $E$ 的非空子集族 $\mathcal{E}$ 对补和有限并封闭,那么我们说 $\mathcal{E}$
% 是 $E$ 上的\emph{代数}。如果其对补和可数并封闭,那么我们说 $\mathcal{E}$
% 是 $E$ 上的\emph{$\sigma$-代数},即:
% \begin{alphenum}
%   \item $A\in\mathcal{E}\Rightarrow E\smallsetminus A\in\mathcal{E}$,
%   \item $A_1,A_2,\dotsc\in \mathcal{E}\Rightarrow\bigcup_n A_n\in\mathcal{E}$。
% \end{alphenum}
% 由于 $\left(\bigcup_n A_n\right)^c=\bigcap_n A_n^c\in\mathcal{E}$,所以对补和可数并封闭
% 可以自然导出对可数交封闭,即 $\sigma$-代数对可数交也封闭。

% 任取 $A\in \mathcal{E}$,那么 $E=A\cup(E \smallsetminus A)\in \mathcal{E}$,所以 $E$ 上
% 任意 $\sigma$-代数都至少包含 $E$ 和 $\emptyset$。事实上,$\mathcal{E}=\{E,\emptyset\}$
% 是 $E$ 上的最简单的 $\sigma$-代数,被称为\emph{平凡 $\sigma$-代数}。$E$ 上最大的
% $\sigma$-代数当然是 $\mathcal{E}=2^E$,即 $\mathcal{E}$ 就是 $E$ 的幂集,被
% 称为\emph{离散 $\sigma$-代数}。

% 不难看出,$E$ 上一族 $\sigma$-代数的任意交(不一定可数)还是 $E$ 上的 $\sigma$-代数。
% 给定 $E$ 的一个子集族 $\mathcal{C}$,我们可以考虑所有包含 $\mathcal{C}$ 
% 的 $\sigma$-代数(总是存在至少一个这样的 $\sigma$-代数,即 $2^E$),将这些
% $\sigma$-代数取交集,我们便得到了包含 $\mathcal{C}$ 的最小的 $\sigma$-代数,
% 被称为\emph{由 $\mathcal{C}$ 生成的} $\sigma$-代数,记为 $\sigma\mathcal{C}$。

% 如果 $E$ 是拓扑空间,由 $E$ 的所有开集族生成的 $\sigma$-代数被称为 $E$ 上的
% \emph{Borel $\sigma$-代数},记为 $\mathcal{B}(E)$ 或者 $\mathcal{B}_E$,
% 其元素被称为\emph{Borel 集}。

% \subsubsection{p-系和 d-系}

% 对于 $E$ 的子集族 $\mathcal{C}$,如果其对交封闭,那么我们说 $\mathcal{C}$
% 是一个 p-系,这里 p 代表 product,是“交”的另一种说法。
% $E$ 的子集族 $\mathcal{D}$ 被称为 d-系,如果其满足:
% \begin{alphenum}
%   \item $E\in\mathcal{D}$,
%   \item $\text{$A,B\in \mathcal{D}$ and $A\supseteq B\Rightarrow A \smallsetminus B\in \mathcal{D}$}$,
%   \item $\text{$(A_n)\subseteq \mathcal{D}$ and $A_n\nearrow A\Rightarrow A\in \mathcal{D}$}$。
% \end{alphenum}
% 其中 $(A_n)\subseteq D$ 表明 $(A_n)$ 是 $D$ 中的集合序列,$A_n\nearrow A$ 表明这个序列
% 递增于极限 $A$:
% \[
%   A_1\subseteq A_2\subseteq\cdots,\quad \bigcup_{n=1}^\infty A_n=A.
% \]

% 显然一个 $\sigma$-代数既是 p-系又是 d-系,其反面也是成立的。所以 p-系和 d-系是产生
% $\sigma$-代数的原始结构。

% \begin{proposition}\label{prop:equivalent condition of sigma algebra}
%   $E$ 的子集族是 $\sigma$-代数当且仅当其既是 p-系又是 d-系。
% \end{proposition}
% \begin{proof}
%   $(\Rightarrow)$ 若 $\mathcal{E}$ 是 $\sigma$-代数,其显然是 p-系并且满足 d-系的条件 (a) 和 (c)。
%   下面我们验证其满足 d-系的条件 (b)。任取 $A,B\in\mathcal{E}$ 且 $A\supseteq B$,
%   那么 $A \smallsetminus B=A\cap (E \smallsetminus B)\in \mathcal{E}$,
%   所以 $\mathcal{E}$ 是 d-系。

%   $(\Leftarrow)$ 若 $\mathcal{E}$ 既是 p-系又是 d-系。任取 $A\in \mathcal{E}$,
%   根据 d-系的 (a) 和 (b),我们有 $E \smallsetminus A\in \mathcal{E}$。
%   所以 $\mathcal{E}$ 对补封闭。然后我们说明对并封闭。任取 $A,B\in \mathcal{E}$, 
%   由于
%   \[
%     A\cup B=E \smallsetminus (A\cup B)^c=E \smallsetminus (A^c\cap B^c),
%   \]
%   结合 p-系对交封闭,所以 $A\cup B\in \mathcal{E}$。最后我们说明对可数并封闭。
%   如果 $(A_n)\subseteq \mathcal{E}$,令 $B_n=A_1\cup\cdots\cup A_n$,
%   那么 $(B_n)\subseteq \mathcal{E}$ 且 $B_n\nearrow A$,根据 d-系的 (c),所以
%   $A\in \mathcal{E}$,故 $\mathcal{E}$ 对可数并封闭。
% \end{proof}

% 下面的引理为本节的主要定理做准备。

% \begin{lemma}\label{lemma:hat D}
%   令 $\mathcal{D}$ 是 $E$ 上的 d-系,固定 $D\in \mathcal{D}$,令
%   \[
%     \hat{\mathcal{D}}=\{A\in \mathcal{D}: A\cap D\in \mathcal{D}\},
%   \]
%   那么 $\hat{\mathcal{D}}$ 仍然是 d-系。
% \end{lemma}

% \subsubsection{单调类定理}

% 这是一个非常有用的工具来证明某些集族是 $\sigma$-代数。

% \begin{theorem}\label{thm:monotone class theorem}
%   如果一个 d-系包含一个 p-系,那么其包含这个 p-系生成的 $\sigma$-代数。
% \end{theorem}
% \begin{proof}
%   设 $\mathcal{C}$ 是一个 p-系。令 $\mathcal{D}$ 是包含 $\mathcal{C}$
%   的最小的 d-系,即包含 $\mathcal{C}$ 的所有 d-系的交(不难看出 d-系的任意交是 d-系)。
%   我们证明 $\mathcal{D}$ 实际上是一个 $\sigma$-代数,这样包含 $\mathcal{C}$
%   的任意 d-系都包含 $\mathcal{D}$,而 $\mathcal{D}$ 作为包含 $\mathcal{C}$
%   的 $\sigma$-代数,其包含 $\sigma \mathcal{C}$。
%   根据 \autoref{prop:equivalent condition of sigma algebra},只需要说明 
%   $\mathcal{D}$ 既是 p-系又是 d-系,而 $\mathcal{D}$ 已经是 d-系,所以只需要
%   说明 $\mathcal{D}$ 是 p-系。

%   我们首先说明对于任意的 $D\in \mathcal{D}$ 和 $C\in \mathcal{C}$,
%   有 $D\cap C\in \mathcal{D}$。令
%   \[
%     \mathcal{D}_1=\{A\in \mathcal{D}: A\cap C\in \mathcal{D}\},  
%   \]
%   根据 \autoref{lemma:hat D},$\mathcal{D}_1$ 是 d-系。由于 $\mathcal{C}$
%   是 p-系,所以 $\mathcal{C}\subseteq \mathcal{D}_1$,即 $\mathcal{D}_1$
%   是包含 $\mathcal{C}$ 的 d-系,所以 $\mathcal{D}\subseteq \mathcal{D}_1$。
%   这就表明 $D\in \mathcal{D}_1$,即 $D\cap C\in \mathcal{D}$。

%   下面说明对于任意的 $D,B\in \mathcal{D}$,有 $D\cap B\in \mathcal{D}$。
%   令
%   \[
%     \mathcal{D}_2=\{A\in \mathcal{D}: A\cap D\in \mathcal{D}\} .
%   \]
%   同样根据 \autoref{lemma:hat D},$\mathcal{D}_2$ 是 d-系。根据上面的叙述,
%   有 $\mathcal{C}\subseteq \mathcal{D}_2$,即 $\mathcal{D}_2$ 是包含 
%   $\mathcal{C}$ 的 d-系,所以 $\mathcal{D}\subseteq \mathcal{D}_2$,
%   这就表明 $B\in \mathcal{D}_2$,即 $D\cap B\in \mathcal{D}$。
%   这就证明了 $\mathcal{D}$ 是 p-系。
% \end{proof}

% \subsubsection{可测空间}

% 一个\emph{可测空间}指的是二元组 $(E,\mathcal{E})$,其中 $E$ 是集合,
% $\mathcal{E}$ 是 $E$ 上的 $\sigma$-代数。此时,$\mathcal{E}$
% 的元素被称为\emph{可测集}。当 $E$ 是拓扑空间,$\mathcal{E}=\mathcal{B}_E$
% 的时候,可测集也被称为\emph{Borel 集}。

% \subsubsection{可测空间的积}

% 令 $(E,\mathcal{E})$ 和 $(F,\mathcal{F})$ 是可测空间。
% 如果 $A\in \mathcal{E}$ 和 $B\in \mathcal{F}$,那么 $A\times B$
% 被称为\emph{可测矩形}。我们用 $\mathcal{E}\otimes \mathcal{F}$
% 表示 $E\times F$ 上的由可测矩形集族生成的 $\sigma$-代数,
% 被称为\emph{乘积 $\sigma$-代数}。可测空间 $(E\times F,\mathcal{E}\otimes \mathcal{F})$
% 被称为 $(E,\mathcal{E})$ 和 $(F,\mathcal{F})$ 的积,我们通常使用
% $(E,\mathcal{E})\times (F,\mathcal{F})$ 来表示。

% \subsubsection{Exercises}

% \begin{exercise}{划分生成 $\sigma$-代数}{}
%   \begin{alphenum}[nosep]
%     \item 令 $\mathcal{C}=\{A,B,C\}$ 是 $E$ 的一个划分,列出
%     $\sigma \mathcal{C}$ 的元素。
%     \item 令 $\mathcal{C}$ 是 $E$ 的(可数)划分。证明 $\sigma \mathcal{C}$
%     的每个元素都是 $\mathcal{C}$ 中元素的可数并。
%     \item 令 $E=\mathbb{R}$,$\mathcal{C}$ 是 $\mathbb{R}$ 的所有单点子集
%     构成的子集族。证明 $\sigma \mathcal{C}$ 的元素要么是可数集要么
%     是可数集的补集。这表明从直观上来看,$\sigma \mathcal{C}$ 要比 $\mathcal{B}(\mathbb{R})$
%     小得多,例如开区间 $(0,1)$ 属于后者但是不属于前者。
%   \end{alphenum}
% \end{exercise}
% \begin{solution}
%   (a) 令
%   \[
%     \mathcal{E}=\{A,B,C,A\cup B,A\cup C,B\cup C,E\},
%   \]
%   显然 $\mathcal{E}$ 是一个 $\sigma$-代数。对于任意包含 $\mathcal{C}$
%   的 $\sigma$-代数,由于其对并封闭,所以其必须包含 $\mathcal{E}$,
%   所以 $\mathcal{E}=\sigma \mathcal{C}$。

%   (b) 令 $\mathcal{E}$ 为 $\mathcal{C}$ 中元素的所有可数并构成的
%   集族。根据 $\sigma$-代数对可数并的封闭性,所以 $\sigma \mathcal{C}\supseteq \mathcal{E}$。
%   设 $(A_n)$ 构成 $E$ 的可数划分,即 $(A_n)$ 两两不相交且 $E=\bigcup_n A_n$。
%   任取 $\bigcup_k A_{n_k}\in \mathcal{E}$,那么
%   $E \smallsetminus \left(\bigcup_k A_{n_k}\right)$ 依然是某些 $(A_n)$
%   的可数并,所以 $E \smallsetminus \left(\bigcup_k A_{n_k}\right)\in \mathcal{E}$,
%   即 $\mathcal{E}$ 对补封闭,所以 $\mathcal{E}$ 是 $\sigma$-代数,
%   所以 $\mathcal{E}=\sigma \mathcal{C}$。

%   (c) 令 $\mathcal{E}$ 为 $\mathbb{R}$ 的可数子集和补集可数的子集构成的子集族。
%   显然 $\mathcal{E}\subseteq \sigma \mathcal{C}$ 且不难验证 $\mathcal{E}$
%   是一个 $\sigma$-代数(可数个可数集的并是可数集),所以 $\mathcal{E}=\sigma \mathcal{C}$。
% \end{solution}

% \begin{exercise}{$\mathbb{R}$ 上的 Borel $\sigma$-代数}{}
%   $\mathbb{R}=(-\infty,+\infty)$ 的任意开子集都是开区间的可数并。使用这一事实证明
%   $\mathcal{B}(\mathbb{R})$ 由所有开区间构成的子集族生成。
% \end{exercise}
% \begin{proof}
%   设 $\mathcal{C}$ 为所有开区间构成的子集族,$\mathcal{T}$ 为所有开集构成的子集族(即 $\mathbb{R}$ 上的拓扑)。
%   显然 $\mathcal{C}\subseteq \mathcal{T}$,
%   所以 $\sigma \mathcal{C}\subseteq \sigma \mathcal{T}=\mathcal{B}(\mathbb{R})$。由于
%   $\mathcal{T}$ 中集合都是 $\mathcal{C}$ 中集合的可数并,所以 $\mathcal{T}\subseteq \sigma \mathcal{C}$,
%   这表明 $\mathcal{B}(\mathbb{R})=\sigma \mathcal{T}\subseteq \sigma \mathcal{C}$。
%   所以 $\mathcal{B}(\mathbb{R})=\sigma \mathcal{C}$ 由所有开区间构成的子集族生成。
% \end{proof}

% \begin{exercise}{$\mathbb{R}$ 上的 Borel $\sigma$-代数}{}
%   证明:$\mathbb{R}$ 中的任意区间都是 Borel 集。特别的,$(-\infty,x),(-\infty,x],(x,y],[x,y]$ 都是 Borel 集。
%   对于每个 $x$,单点集 $\{x\}$ 也是 Borel 集。 
% \end{exercise}
% \begin{proof}
%   只需注意到
%   \begin{gather*}
%     (-\infty,x)=\bigcup_{n=1}^\infty \left(-n+x,x\right), 
%     (-\infty, x]=\bigcap_{n=1}^\infty \left(-\infty,x+\frac{1}{n}\right),\\
%     (x,y]=\bigcap_{n=1}^\infty \left(x,y+\frac{1}{n}\right),
%     [x,y]=\bigcap_{n=1}^\infty\left(x-\frac{1}{n},y\right],
%     \{x\}=\bigcap_{n=1}^\infty\left(x-\frac{1}{n},x\right].
%   \end{gather*}
%   所以上述集合都是 Borel 集,对于其他的区间同理。
% \end{proof}

% \begin{exercise}{$\mathbb{R}$ 上的 Borel $\sigma$-代数}{}
%   证明 $\mathcal{B}(\mathbb{R})$ 可以由以下任意一种集族生成(实际上还有很多可能):
%   \begin{alphenum}[nosep]
%     \item 所有形如 $(-\infty, x]$ 的区间构成的子集族。
%     \item 所有形如 $(x,y]$ 的区间构成的子集族。
%     \item 所有形如 $[x,y]$ 的区间构成的子集族。
%     \item 所有形如 $(x,+\infty)$ 的区间构成的子集族。
%   \end{alphenum}
%   此外,在每种情况中 $x,y$ 可以被限制为有理数。
% \end{exercise}
% \begin{proof}
%   (a) 记该集族为 $\mathcal{C}$,由上题,这样的区间已经是 Borel 集,
%   所以 $\sigma\mathcal{C}\subseteq \mathcal{B}(\mathbb{R})$。任取 $\mathbb{R}$
%   的开区间 $(x,y)$,有
%   \[
%     (x,y)=(-\infty,x]^c\cap (-\infty, y)=
%     (-\infty,x]^c\cap\bigcup_{n=1}^\infty \left(-\infty,y-\frac{1}{n}\right]\in\sigma \mathcal{C},
%   \]
%   而 $\mathcal{B}(\mathbb{R})$ 由所有开区间生成,所以 $\mathcal{B}(\mathbb{R})\subseteq \sigma \mathcal{C}$。
%   所以 $\mathcal{B}(\mathbb{R})=\sigma \mathcal{C}$。

%   (b) (c) (d) 完全同理。
% \end{proof}

% \begin{exercise}{迹空间}{}
%   令 $(E,\mathcal{E})$ 是可测空间,固定 $D\subseteq E$,令
%   \[
%     \mathcal{D}=\mathcal{E}\cap D=\{A\cap D:A\in \mathcal{E}\}.  
%   \]
%   证明 $\mathcal{D}$ 是 $D$ 上的 $\sigma$-代数,被称为 $\mathcal{E}$
%   在 $D$ 上的\emph{迹}。$(D,\mathcal{D})$ 也被称为 $(E,\mathcal{E})$
%   在 $D$ 上的迹。
% \end{exercise}
% \begin{proof}
%   任取 $A\cap D\in \mathcal{D}$,其中 $A\in \mathcal{E}$,那么
%   \[
%     D \smallsetminus (A\cap D)= (E \smallsetminus A)\cap D\in \mathcal{D},
%   \]
%   所以 $\mathcal{D}$ 对补封闭。任取 $(A_n\cap D)\subseteq \mathcal{D}$,那么
%   \[
%     \bigcup_{n=1}^\infty  (A_n\cap D)=\left(\bigcup_{n=1}^\infty A_n\right)\cap D\in \mathcal{D},
%   \]
%   所以 $\mathcal{D}$ 对可数并封闭。
% \end{proof}

% \begin{exercise}{子集的 Borel $\sigma$-代数是迹}{}
%   设 $(E,\mathcal{T})$ 是拓扑空间,$(D,\mathcal{T}_D)$ 是
%   子空间。证明 $D$ 上的 Borel $\sigma$-代数 $\mathcal{B}_D$
%   与 $D$ 在 $E$ 上的迹 $\mathcal{B}_E\cap D$ 相同。
% \end{exercise}
% \begin{proof}
%   由于 $\mathcal{T}_D=\mathcal{T}\cap D\subseteq \mathcal{B}_E\cap D$,
%   由上题 $\mathcal{B}_E\cap D$ 是 $\sigma$-代数,所以 $\mathcal{B}_D\subseteq\mathcal{B}_E\cap D$。
%   记
%   \[
%     \mathcal{C}=\{A\subseteq E: A\cap D\in \mathcal{B}_D\},  
%   \]
%   那么 $\mathcal{T}\subseteq \mathcal{C}$。我们只需要证明 $\mathcal{C}$
%   是 $E$ 上的 $\sigma$-代数,那么就有 $\mathcal{C}\supseteq \sigma \mathcal{T}=\mathcal{B}_E$,即
%   $\mathcal{B}_D\supseteq \mathcal{C}\cap D\supseteq \mathcal{B}_E\cap D$。
%   任取 $A\in \mathcal{C}$,那么 $(E \smallsetminus A)\cap D=D \smallsetminus (A\cap D)\in \mathcal{B}_D$,
%   所以 $E \smallsetminus A\in \mathcal{C}$。任取 $(A_n)\subseteq \mathcal{C}$,那么
%   \[
%     \left(\bigcup_{n=1}^\infty A_n\right)\cap D=\bigcup_{n=1}^\infty (A_n\cap D)
%     \in \mathcal{B}_D,  
%   \]
%   所以 $\bigcup_n A_n\in \mathcal{C}$。这就表明 $\mathcal{C}$
%   是 $E$ 上的 $\sigma$-代数。
% \end{proof}


% \section{可测函数}

% \subsubsection{可测函数}

% 令 $(E,\mathcal{E})$ 和 $(F,\mathcal{F})$ 是可测空间,映射 $f:E\to F$ 如果使得任取 $B\in \mathcal{F}$,有
% $f^{-1}B\in \mathcal{E}$,那么我们说 $f$ 相对于 $\mathcal{E}$ 和 $\mathcal{F}$ \emph{可测}。

% \begin{proposition}\label{prop:equivalent condition of measurable}
%   映射 $f:E\to F$ 相对于 $\mathcal{E}$ 和 $\mathcal{F}$ 可测当且仅当对于任意生成 $\mathcal{F}$ 的子集族
%   $\mathcal{F}_0$,任取 $B\in \mathcal{F}_0$,有 $f^{-1}B\in \mathcal{E}$。
% \end{proposition}
% \begin{proof}
%   必要性显然。下证充分性。设 $\mathcal{F}_0$ 使得 $\mathcal{F}=\sigma \mathcal{F}_0$,且
%   对于任意的 $B\in \mathcal{F}_0$ 有 $f^{-1}B\in \mathcal{E}$。记
%   \[
%     \mathcal{F}_1=\{A\in \mathcal{F}: f^{-1}A\in \mathcal{E}\},
%   \]
%   显然 $\mathcal{F}_0\subseteq \mathcal{F}_1\subseteq \mathcal{F}$。由于
%   \[
%     f^{-1}\left(F \smallsetminus A\right)=E \smallsetminus(f^{-1}A),\quad
%     f^{-1}\left(\bigcup_{i\in I}A_i\right)=\bigcup_{i\in I} f^{-1}A_i,
%   \]
%   所以 $\mathcal{F}_1$ 是 $\sigma$-代数,所以 $\mathcal{F}=\mathcal{F}_1$,即 $f$ 
%   相对于 $\mathcal{E}$ 和 $\mathcal{F}$ 可测。
% \end{proof}

% \begin{proposition}
%   给定可测空间 $(E,\mathcal{E}),(F,\mathcal{F}),(G,\mathcal{G})$,如果 $f$ 相对于 $\mathcal{E}$ 和 
%   $\mathcal{F}$ 可测,$g$ 相对于 $\mathcal{F}$ 和 $\mathcal{G}$ 可测,那么复合
%   $g\circ f$ 相对于 $\mathcal{E}$ 和 $\mathcal{G}$ 可测。
% \end{proposition}
% \begin{proof}
%   任取 $C\in \mathcal{G}$,有
%   \[
%     (g\circ f)^{-1}(C)=f^{-1}\left(g^{-1}(C)\right),  
%   \]
%   $g$ 可测表明 $g^{-1}(C)\in \mathcal{F}$,$f$ 可测表明 $f^{-1}\left(g^{-1}(C)\right)\in \mathcal{E}$,
%   所以 $g\circ f$ 相对于 $\mathcal{E}$ 和 $\mathcal{G}$ 可测。
% \end{proof}

% \subsubsection{数值函数}

% 令 $(E,\mathcal{E})$ 是可测空间。回顾实数及扩充实数 $\mathbb{R}=(-\infty,+\infty)$,
% $\bar{\mathbb{R}}=[-\infty,+\infty]$,$\mathbb{R}_+=[0,+\infty)$,
% $\bar{\mathbb{R}}_+=[0,+\infty]$。$E$ 上的\emph{数值函数}指的是
% 从 $E$ 到 $\bar{\mathbb{R}}$ 或者 $\bar{\mathbb{R}}$ 的子集的映射。
% 如果这个映射的值在 $\mathbb{R}$ 中,那么我们一般称其为\emph{实值函数}。

% $E$ 上的数值函数如果相对于 $\mathcal{E}$ 和 $\mathcal{B}(\bar{\mathbb{R}})$
% 可测,那么我们说其是 $\mathcal{E}$-可测的。如果 $E$
% 是拓扑空间且 $\mathcal{E}=\mathcal{B}(E)$,那么 $\mathcal{E}$-可测函数
% 被称为\emph{Borel 函数}。
% 下面的命题是 \autoref{prop:equivalent condition of measurable} 的直接结果。

% \begin{proposition}
%   映射 $f:E\to\bar{\mathbb{R}}$ 是 $\mathcal{E}$-可测的当且仅当
%   对于每个 $r\in \mathbb{R}$,$f^{-1}[-\infty,r]\in \mathcal{E}$。
% \end{proposition}

% 上述命题中的 $[-\infty,r]$ 可以改为 $[-\infty,r)$,$[r,\infty]$,
% $(r,\infty]$ 中的任意一种。

% \subsubsection{函数的正部分和负部分}

% 对于 $a,b\in\bar{\mathbb{R}}$,我们记 $a\vee b$ 为 $a$ 和 $b$
% 中的最大者,$a\wedge b$ 为 $a$ 和 $b$ 中的最小者。
% 对于函数 $f,g$,用 $f\vee g$ 表示函数 $x\mapsto f(x)\vee g(x)$。
% 令 $(E,\mathcal{E})$ 是可测空间,$f$ 是 $E$ 上的数值函数。
% 那么
% \[
%   f^+=f\vee 0,\quad f^-=-(f\wedge 0)  
% \]
% 都是非负函数并且 $f=f^+-f^-$。函数 $f^+$ 被称为 $f$ 的\emph{正部分},
% $f^-$ 被称为 $f$ 的\emph{负部分}。

% \begin{proposition}
%   函数 $f$ 是 $\mathcal{E}$-可测的当且仅当 $f^+$ 和 $f^-$
%   都是 $\mathcal{E}$-可测的。
% \end{proposition}
% \begin{proof}
%   若 $f$ 是 $\mathcal{E}$-可测的。任取 $r\in \mathbb{R}$,
% 若 $r< 0$,则 $\left(f^+\right)^{-1}[-\infty,r]=\emptyset\in \mathcal{E}$。
%   若 $r\geq 0$,则
%   \[
%     \left(f^+\right)^{-1}[-\infty,r]=E \smallsetminus \left(f^+\right)^{-1}(r,\infty]
%     =E \smallsetminus f^{-1}(r,\infty],
%   \]
%   由于 $(r,\infty]$ 是 Borel 集,所以 $\left(f^+\right)^{-1}[-\infty,r]\in \mathcal{E}$。
%   综合起来,$f^+$ 是 $\mathcal{E}$-可测的。同理可证 $f^-$ 是 $\mathcal{E}$-可测的。

%   若 $f^+$ 和 $f^-$ 都是 $\mathcal{E}$-可测的。任取 $r\in \mathbb{R}$,若 $r<0$,那么
%   \[
%     f^{-1}[-\infty,r]=  \left(f^-\right)^{-1}[-r,\infty]\in \mathcal{E}.
%   \]
%   若 $r\geq 0$,那么
%   \[
%     f^{-1}[-\infty,r]= E \smallsetminus f^{-1}(r,\infty]
%     =E \smallsetminus \left(f^+\right)^{-1}(r,\infty]\in \mathcal{E}.
%   \]
%   所以 $f$ 是 $\mathcal{E}$-可测的。
% \end{proof}

% \subsubsection{指示函数和简单函数}

% 令 $A\subseteq E$,定义 $A$ 的指示函数为 $1_A$:
% \[
%   1_A(x)=\begin{cases}
%     1 & x\in A,\\
%     0 & x\notin A.
%   \end{cases}  
% \]
% 对于 $1_E$,我们简记为 $1$。显然,$1_A$ 是 $\mathcal{E}$-可测的当且仅当
% $A\in \mathcal{E}$。

% $E$ 上的函数 $f$ 如果形如
% \[
%   f=\sum_{i=1}^n a_i 1_{A_i},  
% \]
% 其中 $n\geq 1$,$a_1,\dots,a_n\in \mathbb{R}$,$A_1,\dots,A_n$ 是可测集,
% 那么我们说 $f$ 是\emph{简单函数}。
% 在这个定义中,若 $A_i\cap A_j\ne\emptyset$,那么我们可以将
% $a_i1_{A_i}+a_j1_{A_j}$ 拆为
% \[ 
%   a_i1_{A_i \smallsetminus (A_i\cap A_j)}+(a_i+a_j)1_{A_i\cap A_j}+a_j 1_{A_j \smallsetminus (A_i\cap A_j)},
% \]
% 所以我们可以假设 $A_i$ 两两不相交。此外,如果 $\bigcup_i A_i\neq E$,记
% $B=E \smallsetminus\bigcup_i A_i\in \mathcal{E}$,那么
% \[
%   f=  \sum_{i=1}^n a_i 1_{A_i}+0\cdot 1_{B},
% \]
% 所以我们还可以假设 $\bigcup_i A_i=E$。这意味着对于一个简单函数 $f$,
% 总存在整数 $m$,不同的实数 $b_1,\dots,b_m$ 和 $E$ 的可测划分
% $\{B_1,\dots,B_m\}$ 使得 $f=\sum_{i=1}^m b_i 1_{B_i}$,
% 这种表示被称为简单函数 $f$ 的\emph{标准型}。

% 利用简单函数的标准型,很容易验证简单函数都是 $\mathcal{E}$-可测的。
% 反之,若 $f$ 是 $\mathcal{E}$-可测的,只有有限个取值且值为实数,那么
% $f$ 为简单函数。特别地,任意常值函数是简单函数。最后,如果 $f,g$
% 是简单函数,那么
% \[
%   f+g,\quad f-g,\quad fg,\quad f/g,\quad f\vee g,\quad f\wedge g  
% \]
% 都是简单函数,其中 $f/g$ 要求 $g$ 的值始终非零。

% \subsubsection{函数列的极限}

% 令 $(f_n)$ 是 $E$ 上的一列数值函数,我们可以逐点定义
% \begin{equation}\label{eq:inf and sup}
%   \inf f_n,\quad \sup f_n,\quad \liminf f_n,\quad \limsup f_n,
% \end{equation}
% 例如,$\inf f_n$ 将 $x\in E$ 送到实数列 $(f_n(x))$ 的下确界。
% 如果
% \[
%   \liminf f_n=\limsup f_n=f,
% \]
% 那么我们说 $(f_n)$ 有逐点极限 $f$,记为 $f=\lim f_n$ 或者 $f_n\to f$。

% 如果 $(f_n)$ 单调递增,即 $f_1\leq f_2\leq \cdots$,那么根据单调有界定理,$\lim f_n$
% 存在且等于 $\sup f_n$。此时我们用 $f_n\nearrow f$ 来表示 $(f_n)$ 单调递增且有极限 $f$。
% 类似地,用 $f_n\searrow f$ 来表示 $(f_n)$ 单调递减且有极限 $f$。

% 下面的定理表明可测函数类对极限操作是封闭的。
% \begin{theorem}\label{thm:lim of measurable function is measurable}
%   令 $(f_n)$ 是一列 $\mathcal{E}$-可测函数,那么 \eqref{eq:inf and sup} 中的四个函数都是
%   $\mathcal{E}$-可测的。此外,如果 $\lim f_n$ 存在,那么 $\lim f_n$ 也是 $\mathcal{E}$-可测的。
% \end{theorem}
% \begin{proof}
%   记 $g=\sup f_n$。任取 $r\in \mathbb{R}$,注意到 $g(x)\leq r$ 当且仅当对于所有 $n$ 有 $f_n(x)\leq r$。
%   所以
%   \[
%     g^{-1}[-\infty,r]=\bigcap_{n=1}^\infty f_n^{-1}[-\infty,r],
%   \]
%   $f_n$ 可测表明 $f_n^{-1}[-\infty,r]\in \mathcal{E}$,所以 $g^{-1}[-\infty,r]\in \mathcal{E}$,即
%   $g$ 可测。

%   对于 $\inf f_n$,我们有 $\inf f_n=-\sup (-f_n)$,所以 $\inf f_n$ 也可测。
%   最后,注意到
%   \[
%     \liminf f_n=\sup\limits_m \mathop{\vphantom{\sup}\inf}_{n\geq m} f_n,\quad
%     \limsup f_n=\mathop{\vphantom{\sup}\inf}_{m}\sup_{n\geq m} f_n,
%   \] 
%   所以 $\liminf f_n$ 和 $\limsup f_n$ 可测。若二者相等,那么 $\lim f_n$ 也可测。
% \end{proof}

% \subsubsection{可测函数的逼近}

% \begin{lemma}\label{lemma:dn}
%   对于 $n\in \mathbb{N}^*$,令
%   \[
%     d_n(r)=\sum_{k=1}^{n2^n}\frac{k-1}{2^n}1_{\left[\frac{k-1}{2^n},\frac{k}{2^n}\right)}(r)
%     +n1_{[n,\infty]}(r),\quad r\in \bar{\mathbb{R}}_+.
%   \]
%   那么,$d_n(r)$ 是 $\bar{\mathbb{R}}_+$ 上单调递增的简单函数,并且对于每个 $r\in\bar{\mathbb{R}}_+$,
%   $d_n(r)$ 随着 $n$ 的增大是的单调递增的。
% \end{lemma}
% \begin{proof}
%   显然 $d_n(r)$ 是单调递增的简单函数,我们证明任取 $r\in \bar{\mathbb{R}}_+$,$d_n(r)$
%   是单调递增的。若 $r=\infty$,那么 $d_n(r)=n$ 是单调递增的。现在假设 $r\in \mathbb{R}_+$,
%   那么存在正整数 $m$ 使得 $m\leq r<m+1$,所以当 $n\leq m$ 的时候,$d_n(r)=n$ 单调递增。
%   当 $n>m$ 的时候,直观来看,$d_n$ 将区间 $[0,n]$ 等分为 $n2^{n}$ 份,$r\in [0,n]$
%   表明一定存在唯一的 $k_n$ 使得 $(k_n-1)/2^n\leq r <k_n/2^n$,可以发现 $k_{n}$ 满足 
%   递推关系 $k_{n+1}=2k_n-1$ 或者 $k_{n+1}=2k_n$,这表明
%   \[
%     d_{n+1}(r)= \frac{k_{n+1}-1}{2^{n+1}}\geq\frac{2k_n-2}{2^{n+1}}=\frac{k_n-1}{2^n}=d_n(r).
%   \]
%   综上,$d_n(r)$ 随着 $n$ 的增大是的单调递增的。
% \end{proof}

% \begin{theorem}\label{thm:approximation of measurable function}
%   $E$ 上的非负函数是 $\mathcal{E}$-可测的当且仅当其是一列单调递增的非负简单函数序列
%   的极限。
% \end{theorem}
% \begin{proof}
%   充分性来源于 \autoref{thm:lim of measurable function is measurable}。
%   对于必要性,设 $f:E\to\bar{\mathbb{R}}_+$ 是 $\mathcal{E}$-可测的非负函数。
%   记 $d_n$ 为上述引理中的函数,令 $f_n=d_n\circ f$。那么 $f_n$ 是非负的 $\mathcal{E}$-可测函数,
%   并且其取值只有有限个,所以是简单函数。由于 $(d_n)$ 单调递增,所以 $(f_n)$ 单调递增。
%   对于任意 $x\in E$,由于 $f_n(x)=d_n(f(x))$,所以 $n\to\infty$ 的时候
%   $f_n(x)\to f(x)$,故 $f=\lim f_n$。
% \end{proof}

% \subsubsection{函数的单调类}

% 令 $\mathcal{M}$ 为 $E$ 上数值函数的一个集合,记 $\mathcal{M}_+$ 为 $\mathcal{M}$
% 中非负函数组成的子集,$\mathcal{M}_b$ 为 $\mathcal{M}$ 中有界函数组成的子集。

% 如果 $\mathcal{M}$ 包含常值函数 $1$,$\mathcal{M}_b$ 构成 $\mathbb{R}$ 上的向量空间
% 以及 $\mathcal{M}_+$ 在递增极限下封闭,那么我们说 $\mathcal{M}$ 是一个\emph{单调类}。
% 更准确地说,$\mathcal{M}$ 是单调类当且仅当:
% \begin{alphenum}
%   \item $1\in \mathcal{M}$,
%   \item 若 $f,g\in \mathcal{M}_b$ 且 $a,b\in \mathbb{R}$,则 $af+bg\in \mathcal{M}$,
%   \item 若 $(f_n)\subseteq \mathcal{M}_+$ 且 $f_n\nearrow f$,那么
%   $f\in \mathcal{M}$。
% \end{alphenum}

% 下面的定理通常被用于证明所有 $\mathcal{E}$-可测函数拥有的某一性质。

% \begin{theorem}
%   令 $\mathcal{M}$ 是 $E$ 上函数的单调类。假设对于某个生成 $\mathcal{E}$
%   的 p-系 $\mathcal{C}$,任取 $A\in \mathcal{C}$,有 $1_A\in \mathcal{M}$。那么,
%   $\mathcal{M}$ 包含所有的非负 $\mathcal{E}$-可测函数以及所有的有界 $\mathcal{E}$-可测函数。
% \end{theorem} 
% \begin{proof}
%   首先证明对于任意的 $A\in \mathcal{E}$ 有 $1_A\in \mathcal{M}$。记
%   \[
%     \mathcal{D}=\{A\in \mathcal{E}:1_A\in \mathcal{M}\}.
%   \]
%   由于 $1=1_E\in \mathcal{M}$,所以 $E\in \mathcal{D}$。任取 $A,B\in \mathcal{D}$ 且 $A\supseteq B$,
%   那么 $1_{A \smallsetminus B}=1_A-1_B\in \mathcal{M}$,这表明 $A \smallsetminus B\in \mathcal{D}$。
%   设 $(A_n)\subseteq \mathcal{D}$ 且 $A_n\nearrow A$,那么 $(1_{A_n})\subseteq \mathcal{M}_+$
%   且 $1_{A_n}\nearrow 1_A$,所以 $1_A\in \mathcal{M}$,这表明 $A\in \mathcal{D}$。
%   这表明 $\mathcal{D}$ 是 d-系。由于 $\mathcal{C}$ 是 p-系且 $\mathcal{C}\subseteq \mathcal{D}$,
%   根据单调类定理 \ref{thm:monotone class theorem},$\mathcal{D}$ 包含 $\sigma \mathcal{C}=\mathcal{E}$,
%   所以对于任意的 $A\in \mathcal{E}$ 有 $A\in \mathcal{D}$,即 $1_A\in \mathcal{M}$。

%   再根据单调类的定义 (b),$\mathcal{M}$ 包含所有的简单函数。

%   令 $f$ 是非负 $\mathcal{E}$-可测函数,根据 \autoref{thm:approximation of measurable function},
%   $f$ 是函数序列 $(f_n)$ 的极限,其中 $f_n$ 是递增的非负简单函数,即 $(f_n)\subseteq \mathcal{M}_+$。
%   根据单调类的定义 (c),有 $f\in \mathcal{M}$。

%   令 $g$ 是有界 $\mathcal{E}$-可测函数,那么 $g^+$ 和 $g^-$ 都是非负 $\mathcal{E}$-可测函数,所以
%   $g^+,g^-\in \mathcal{M}$。显然 $g^+,g^-$ 也都是有界的,根据单调类的定义 (b),所以
%   $g=g^+-g^-\in \mathcal{M}$。
% \end{proof}

% \subsubsection{标准可测空间}

% 令 $(E,\mathcal{E})$ 和 $(F,\mathcal{F})$ 是可测空间。如果 $f:E\to F$ 是双射的
% 相对于 $\mathcal{E}$ 和 $\mathcal{F}$ 的可测函数,并且其逆映射 $f^{-1}:F\to E$
% 是相对于 $\mathcal{F}$ 和 $\mathcal{E}$ 的可测函数,那么我们说 $f$ 是\emph{同构}。

% 如果可测空间 $(E,\mathcal{E})$ 同构于 $(F,\mathcal{B}_F)$,其中 $F$ 是 $\mathbb{R}$
% 的某个 Borel 子集,那么我们说 $(E,\mathcal{E})$ 是\emph{标准可测空间}。
% 标准可测空间有非常多。如果 $E$ 是完备度量空间,那么 $(E,\mathcal{B}_E)$
% 是标准可测空间。如果 $E$ 是波兰空间,即可分的可完备度量化的拓扑空间,那么
% $(E,\mathcal{B}_E)$ 是标准可测空间。如果 $E$ 是可分的 Banach 空间,
% 那么 $(E,\mathcal{B}_E)$ 是标准可测空间。

% 显然,$[0,1]$ 和它的 Borel $\sigma$-代数构成标准可测空间。
% $\{1,2,\dots,n\}$ 和它的离散 $\sigma$-代数构成标准可测空间。
% $\mathbb{N}=\{0,1,2,\dots\}$ 和它的离散 $\sigma$-代数构成标准可测空间。
% 一个深刻的结果是,任意标准可测空间都同构于上述三者之一。


% \section{测度}

% 令 $(E,\mathcal{E})$ 是可测空间,$(E,\mathcal{E})$ 上的\emph{测度}
% 指的是一个映射 $\mu:\mathcal{E}\to\bar{\mathbb{R}}_+$,其满足:
% \begin{alphenum}
%   \item $\mu(\emptyset)=0$,
%   \item 对于不相交的子集列 $(A_n)\subseteq \mathcal{E}$,有 
%   $\mu(\bigcup_n A_n)=\sum_n \mu(A_n)$。
% \end{alphenum}
% 条件 (b) 被称为\emph{可列可加性}。需要注意 $\mu(A)$ 总是为正数且可以
% 为 $+\infty$。数 $\mu(A)$ 被称为 $A$ 的\emph{测度},也简记为 $\mu A$。

% 一个\emph{测度空间}指的是三元组 $(E,\mathcal{E},\mu)$,其中 $(E,\mathcal{E})$
% 是可测空间,$\mu$ 是 $(E,\mathcal{E})$ 上的测度。

% \subsubsection{例子}

% \begin{example}[Dirac 测度]
%   令 $(E,\mathcal{E})$ 是可测空间,固定 $x\in E$。对于每个 $A\in \mathcal{E}$,
%   令
%   \[
%     \delta_x(A)=\begin{cases}
%       1 & x\in A,\\
%       0 & x\notin A.
%     \end{cases}  
%   \]
%   那么 $\delta_x$ 是 $(E,\mathcal{E})$ 上的测度,被称为\emph{Dirac 测度}。
%   直观来看,其基于一个集合 $A$ 是否含有特定元素 $x$ 来给出这个集合的“大小”。
% \end{example}

% \begin{example}[计数测度]
%   令 $(E,\mathcal{E})$ 是可测空间,固定 $D\subseteq E$。对于每个
%   $A\in \mathcal{E}$,令 $\nu(A)$ 是 $A\cap D$ 中点的个数,
%   此时 $\nu$ 是 $(E,\mathcal{E})$ 上的测度,被称为\emph{计数测度}。
%   通常,集合 $D$ 被选取为可数集,在这种情况下
%   \[
%     \nu(A)=\sum_{x\in D}\delta_x(A),\quad A\in \mathcal{E}.  
%   \]
% \end{example}

% \begin{example}[离散测度]
%   令 $(E,\mathcal{E})$ 是可测空间,固定可数子集 $D\subseteq E$。
%   对于每个 $x\in D$,分配一个正数 $m(x)$。定义
%   \[
%     \mu(A)=\sum_{x\in D} m(x)\delta_x(A),\quad A\in \mathcal{E}.  
%   \]
%   那么 $\mu$ 是 $(E,\mathcal{E})$ 上的测度,被称为\emph{离散测度}。
%   我们可能会把 $m(x)$ 理解为点 $x$ 的质量,那么 $\mu(A)$ 就是集合 $A$
%   的质量。特别地,如果 $(E,\mathcal{E})$ 是离散可测空间,那么
%   每个测度 $\mu$ 都有这种形式。
% \end{example}

% \begin{example}[Lebesgue 测度]
%   $(\mathbb{R},\mathcal{B}_{\mathbb{R}})$ 上的测度 $\mu$ 如果对于每个
%   区间 $A$ 都满足 $\mu(A)$ 为 $A$ 的长度,那么我们说 $\mu$ 是\emph{Lebesgue 测度}。
%   类似地,$\mathbb{R}^2$ 上的 Lebesgue 测度是“面积”测度,
%   $\mathbb{R}^3$ 上的 Lebesgue 测度是“体积”测度等等。
%   我们将它们记作 $\Leb$。
% \end{example}

% \subsubsection{测度的性质}

% \begin{proposition}
%   令 $\mu$ 是可测空间 $(E,\mathcal{E})$ 上的测度,那么对于任意
%   可测集 $A,B$ 和 $A_1,A_2,\dots$,有:
%   \begin{description}[nosep,font=\sffamily\mdseries,itemindent=0pt]
%     \item[有限可加性] $A\cap B=\emptyset\Rightarrow \mu(A\cup B)=\mu(A)+\mu(B)$。
%     \item[单调性] $A\subseteq B\Rightarrow \mu(A)\leq \mu(B)$。
%     \item[连续性] $A_n\nearrow A\Rightarrow \mu(A_n)\nearrow \mu(A)$。
%     \item[Boole 不等式]  $\mu(\bigcup_n A_n)\leq \sum_n \mu(A_n)$。  
%   \end{description}
% \end{proposition}
% \begin{proof}
%   有限可加性是可列可加性的特殊情况,取 $A_1=A,A_2=B,A_3=A_4=\cdots=\emptyset$
%   即可。若 $A\subseteq B$,由于 $\mathcal{E}$ 是 d-系,所以 $B \smallsetminus A\in \mathcal{E}$,
%   所以
%   \[
%     \mu(B)=\mu(A\cup(B \smallsetminus A))= \mu(A)+\mu(B \smallsetminus A)
%     \geq \mu(A).
%   \]
%   若 $A_n\nearrow A$,令 $B_1=A_1$,$B_n=A_{n}\smallsetminus A_{n-1}$,那么
%   $B_n$ 互不相交且 $\bigcup_{k=1}^n B_k=A_n$,所以
%   \[
%     \lim \mu(A_n)=\lim\mu\left(\bigcup_{k=1}^n B_k\right)  
%     =\lim \sum_{k=1}^n \mu(B_k)=\sum_{k=1}^\infty \mu(B_k)
%     =\mu(A).
%   \]
%   对于 Boole 不等式,注意到
%   \[
%     \mu(A\cup B)=\mu(A\cup (B \smallsetminus A))=\mu(A)+\mu(B \smallsetminus A) 
%     \leq \mu(A)+\mu (B), 
%   \]
%   所以归纳可得
%   \[
%     \mu\left(\bigcup_{k=1}^n A_k\right)\leq \sum_{k=1}^n\mu(A_k),  
%   \]
%   令 $n\to\infty$,左边根据连续性即可得到 Boole 不等式。
% \end{proof}

% \subsubsection{有限测度}

% 令 $\mu$ 是可测空间 $(E,\mathcal{E})$ 上的测度,如果 $\mu(E)<\infty$,
% 那么 $\mu$ 被称为\emph{有限测度},根据单调性,此时对于任意 $A\in \mathcal{E}$,
% 都有 $\mu(A)<\infty$。如果 $\mu(E)=1$,那么 $\mu$ 被称为\emph{概率测度}。
% 如果存在 $E$ 的可测划分 $(E_n)$ 使得 $\mu(E_n)<\infty$,那么
% $\mu$ 被称为\emph{$\sigma$-有限测度}。如果存在一列有限测度 $\mu_n$
% 使得 $\mu=\sum_n \mu_n$,那么 $\mu$ 被称为\emph{$\Sigma$-有限测度}。
% 有限测度都是 $\sigma$-有限的,$\sigma$-有限测度都是 $\Sigma$-有限的。

% \begin{proposition}
%   令 $(E,\mathcal{E})$ 是可测空间,$\mu,\nu$ 是两个有限测度且 $\mu(E)=\nu(E)$,如果
%   $\mu,\nu$ 在生成 $\mathcal{E}$ 的某个 p-系上取值相同,那么
%   $\mu=\nu$。
% \end{proposition}
% \begin{proof}
%   设 $\mathcal{C}$ 是 p-系且 $\mathcal{E}=\sigma \mathcal{C}$,
%   任取 $A\in \mathcal{C}$ 有 $\mu(A)=\nu(A)$。令
%   \[
%     \mathcal{D}=\{A\in \mathcal{E}:\mu(A)=\nu(A)\},  
%   \]
%   那么 $\mathcal{C}\subseteq \mathcal{D}$。如果我们证明 $\mathcal{D}$
%   是 d-系,那么根据单调类定理,就有 $\mathcal{E}=\sigma \mathcal{C}\subseteq \mathcal{D}$,
%   即任取 $A\in \mathcal{E}$ 有 $\mu(A)=\nu(A)$。下面我们证明 $\mathcal{D}$
%   是 d-系。由于 $\mu(E)=\nu(E)$,所以 $E\in \mathcal{D}$。
%   若 $A,B\in \mathcal{D}$ 且 $A\supseteq B$,那么
%   \[
%     \mu(B)+\mu(A \smallsetminus B)= \mu(A)=\nu(A)=\nu(B)+\mu(A \smallsetminus B),
%   \]
%   由于 $\mu(B)=\nu(B)$,所以 $\mu(A \smallsetminus B)=\nu (A \smallsetminus B)$,
%   即 $A \smallsetminus B\in \mathcal{D}$。
%   任取 $(A_n)\subseteq \mathcal{D}$ 且 $A_n\nearrow A$,根据连续性,所以
%   $\mu(A_n)\nearrow \mu(A)$ 以及 $\nu(A_n)\nearrow \nu(A)$,所以
%   \[
%     \mu(A)=\lim \mu(A_n)=\lim \nu(A_n)=\nu(A),  
%   \]
%   所以 $A\in \mathcal{D}$。这就证明了 $\mathcal{D}$ 是 d-系。
% \end{proof}

% \begin{corollary}
%   令 $\mu,\nu$ 是 $\left(\bar{\mathbb{R}},\mathcal{B}\left(\bar{\mathbb{R}}\right)\right)$
%   上的概率测度,那么 $\mu=\nu$ 当且仅当对于任意的 $r\in \mathbb{R}$ 有
%   $\mu[-\infty,r]=\nu[-\infty, r]$。
% \end{corollary}

% \subsubsection{原子,纯原子测度和非原子测度}

% 令 $(E,\mathcal{E})$ 是可测空间,假设对于每个 $x\in E$,单点集 $\{x\}\in \mathcal{E}$,
% 这一点对于所有的标准可测空间都是成立的。令 $\mu$ 是 $(E,\mathcal{E})$ 上的测度,
% 如果点 $x$ 使得 $\mu\{x\}>0$,那么 $x$ 被称为 $\mu$ 的一个\emph{原子}。如果 $\mu$
% 没有任何原子,那么 $\mu$ 被称为\emph{非原子测度}。如果 $\mu$ 的原子的集合 $D$
% 是可数集并且 $\mu(E \smallsetminus D)=0$,那么 $\mu$ 被称为\emph{纯原子测度}。
% 例如,Lebesgue 测度是非原子测度,Dirac 测度是纯原子测度(其只有一个原子),离散测度
% 是纯原子测度。

% \begin{proposition}
%   令 $\mu$ 是 $(E,\mathcal{E})$ 上的 $\Sigma$-有限测度,那么
%   \[
%     \mu=\lambda+\nu,
%   \]
%   其中 $\lambda$ 是非原子测度,$\nu$ 是纯原子测度。
% \end{proposition}


% \subsubsection{完备性,零测集}

% 令 $(E,\mathcal{E},\mu)$ 是测度空间,如果可测集 $B$ 使得 $\mu(B)=0$,那么 $B$
% 被称为\emph{零测集}。$E$ 的任意子集如果被一个可测的零测集包含,那么也被称为\emph{零测集}。
% 如果 $E$ 的每个零测集都是可测集,那么我们说这个测度空间是\emph{完备的}。对于不完备的
% 测度空间,下面的结果表明可以通过包含所有的零测集来扩大 $\mathcal{E}$ 以及
% $\mu$ 来得到一个完备测度空间。测度空间 $(E,\bar{\mathcal{E}},\bar\mu)$ 被称为
% $(E,\mathcal{E},\mu)$ 的\emph{完备化}。当 $E=\mathbb{R}$,$\mathcal{E}=\mathcal{B}_{\mathbb{R}}$
% 和 $\mu=\Leb$ 的时候,$\bar{\mathcal{E}}$ 的元素被称为\emph{Lebesgue 可测集}。

% \begin{proposition}
%   令 $\mathcal{N}$ 是 $E$ 的所有零测子集的集合族,$\bar{\mathcal{E}}$ 为 $\mathcal{E}\cup \mathcal{N}$
%   生成的 $\sigma$-代数,那么
%   \begin{alphenum}[nosep]
%     \item 每个 $B\in\bar{\mathcal{E}}$ 都形如 $B=A\cup N$,其中 $A\in \mathcal{E}$ 以及 $N\in \mathcal{N}$,
%     \item 定义 $\bar\mu(A\cup N)=\mu(A)$,这给出了 $\bar{\mathcal{E}}$ 上的测度 $\bar\mu$,并且
%     是唯一的满足 $\bar\mu(A)=\mu(A)\ (A\in \mathcal{E})$ 的测度,此时测度空间 $(E,\mathcal{E},\mu)$
%     是完备的。
%   \end{alphenum}
% \end{proposition}

% \begin{exercise}{限制和迹}{}
%   令 $(E,\mathcal{E})$ 是可测空间,$\mu$ 是测度。令 $D\in \mathcal{E}$。
%   \begin{alphenum}[nosep]
%     \item 定义 $\nu(A)=\mu(A\cap D)$,证明 $\nu$ 是 $(E,\mathcal{E})$ 上的测度,
%     被称为 $\mu$ 在 $D$ 上的迹。
%     \item 令 $\mathcal{D}$ 为 $\mathcal{E}$ 在 $D$ 上的迹,对于 $A\in \mathcal{D}$,
%     定义 $\nu(A)=\mu(A)$,证明 $\nu$ 是 $(D,\mathcal{D})$ 上的测度,被称为
%     $\mu$ 在 $D$ 上的限制。
%   \end{alphenum}
% \end{exercise}
% \begin{proof}
%   (a) $\nu(\emptyset)=\mu(\emptyset)=0$。设 $(A_n)\subseteq \mathcal{E}$ 是不相交
%   的子集列,那么 $(A_n\cap D)\subseteq \mathcal{E}$ 仍然不相交,所以
%   \[
%     \nu\left(\bigcup_{n=1}^\infty A_n\right)=\mu\left(\bigcup_{n=1}^\infty (A_n\cap D)\right)
%     =\sum_{n=1}^\infty \mu(A_n\cap D)=\sum_{n=1}^\infty \nu(A_n),
%   \]
%   即 $\nu$ 是 $(E,\mathcal{E})$ 上的测度。

%   (b) $\nu(\emptyset)=\mu(\emptyset)=0$。设 $(A_n)\subseteq \mathcal{D}$
%   是不相交的子集列,于是
%   \[
%     \nu\left(\bigcup_{n=1}^\infty A_n\right)=\mu\left(\bigcup_{n=1}^\infty A_n\right)
%     =\sum_{n=1}^\infty\mu(A_n)=\sum_{n=1}^\infty\nu(A_n).\qedhere
%   \]
% \end{proof}

% \section{积分}

% 令 $(E,\mathcal{E},\mu)$ 是测度空间。我们同时用 $\mathcal{E}$ 表示
% $E$ 上的 $\mathcal{E}$-可测函数的集合,$\mathcal{E}_+$ 表示非负
% $\mathcal{E}$-可测函数构成的子集。下面对于所有合理的 $f\in \mathcal{E}$,
% 我们定义“$f$ 相对于 $\mu$ 的积分”,我们将记为:
% \[
%   \mu f=\mu(f)=\int_E f(x) \mu(\d x)  =\int_E f \d\mu.
% \]
% 并且我们将证明我们定义的积分蕴含下面的性质:对于任意 $a,b\in \mathbb{R}_+$
% 和 $f,g,f_n\in \mathcal{E}_+$,有:
% \begin{description}[nosep,font=\sffamily\mdseries,itemindent=0pt]
%   \item[非负性] $\mu f\geq 0$,若 $f=0$,则 $\mu f=0$。
%   \item[线性性] $\mu(af+bg)=a\mu f+b\mu g$。
%   \item[单调收敛定理] 若 $f_n\nearrow f$,则 $\mu f_n\nearrow \mu f$。  
% \end{description}

% \begin{definition}
%   我们从简单函数开始逐步定义积分的概念。
%   \begin{alphenum}
%     \item 令 $f$ 是简单非负函数,设其标准型为 $f=\sum_i a_i 1_{A_i}$,
%     定义
%     \[
%       \mu f=\sum_{i=1}^n a_i\mu(A_i).  
%     \]
%     \item 令 $f\in \mathcal{E}_+$,记 $f_n=d_n\circ f$,其中 $d_n$
%     为 \autoref{lemma:dn} 中的简单函数,那么每个 $f_n$ 是简单非负函数。
%     此时 $\mu f_n$ 是单调递增的,于是我们定义
%     \[
%       \mu f=\lim \mu f_n.  
%     \]
%     \item 令 $f\in \mathcal{E}$,那么 $f^+,f^-\in \mathcal{E}_+$。
%     由于 $f=f^+-f^-$,于是我们定义
%     \[
%       \mu f=\mu(f^+)-\mu (f^-).  
%     \]
%     上述定义中我们要求等式右端至少有一项是有限的。
%   \end{alphenum}
% \end{definition}
% \begin{remark}
%   令 $f,g$ 是简单非负函数。
%   \begin{alphenum}
%     \item 设 $f=\sum_i a_i 1_{A_i}$,这里我们不要求是标准型。若 $\bigcup_i A_i\neq E$,
%     那么我们可以添加零项,所以我们假设 $\bigcup_i A_i=E$。若 $A_i\cap A_j\neq \emptyset$,
%     那么
%     \[
%       a_i1_{A_i}+a_j1_{A_j}=
%       a_i1_{A_i \smallsetminus (A_i\cap A_j)}+(a_i+a_j)1_{A_i\cap A_j}+a_j 1_{A_j \smallsetminus (A_i\cap A_j)},
%     \]
%     于是这两项的积分为
%     \begin{align*}
%       &a_i\mu(A_i \smallsetminus (A_i\cap A_j))+(a_i+a_j)\mu(A_i\cap A_j)+
%       a_j\mu(A_j \smallsetminus (A_i\cap A_j))\\
%       ={}& a_i\mu(A_i)-a_i\mu(A_i\cap A_j)+(a_i+a_j)\mu(A_i\cap A_j)+a_j\mu(A_j)
%       -a_j\mu(A_i\cap A_j)\\
%       ={}& a_i\mu(A_i)+a_j\mu(A_j),
%     \end{align*}
%     所以对于简单函数的非标准形式而言,其积分依然形如 $\mu f=\sum_i a_i\mu(A_i)$。
%     \item 若 $a,b\in \mathbb{R}_+$,那么 $af+bg$ 依然是简单非负函数,再根据
%     (a),就有
%     \[
%       \mu(af+bg)=a\mu f+b\mu g.  
%     \]
%     \item 如果 $f\leq g$,那么
%     \[
%       \mu f\leq \mu f+\mu(g-f)=\mu(f+g-f)=\mu g.
%     \]
%     \item 若 $f_1\leq f_2\leq\cdots$,根据 (c),有 $\mu f_1\leq\mu f_2\leq \cdots$,
%     所以定义中的 $\lim \mu f_n$ 存在(可以为 $+\infty$)。
%   \end{alphenum}
% \end{remark}

% \subsubsection{示例}

% \begin{example}[离散测度]
%   固定 $x_0\in E$,考虑 Dirac 测度 $\delta_{x_0}$。任取 $f\in \mathcal{E}_+$,
%   所以
%   \[
%     \delta_{x_0}f=\lim \delta_{x_0}(d_n\circ f)=  f(x_0).
%   \]
%   那么对于任意 $f\in \mathcal{E}$,就有
%   \[
%     \delta_{x_0}f=\delta_{x_0}(f^+)-\delta_{x_0}(f^-)=f(x_0).  
%   \]
%   设 $\mu=\sum_{x\in D} m(x)\delta_x$ 是离散测度,其中 $D\in \mathcal{E}$ 是可数集,
%   $m(x)> 0$,那么
%   \[
%     \mu f=\sum_{x\in D}m(x) f(x).  
%   \]
% \end{example}

% \begin{example}[离散空间]
%   设 $(E,\mathcal{E})$ 是离散可测空间,即 $E$ 可数且 $\mathcal{E}=2^E$。此时
%   $E$ 上的任意数值函数都是 $\mathcal{E}$-可测的,且 $E$ 上的任意测度 $\mu$ 都满足
%   $\mu=\sum_{x\in E}\mu\{x\}\delta_x$,所以对于任意 $E$ 上的函数 $f$,有
%   \[
%     \mu f=\sum_{x\in E}\mu\{x\} f(x).
%   \]
% \end{example}

% \begin{example}[Lebesgue 积分]
%   设 $E$ 是 $\mathbb{R}^d$ 的 Borel 子集,$\mathcal{E}=\mathcal{B}(E)$。
%   设 $\mu$ 是 Lebesgue 测度在 $(E,\mathcal{E})$ 上的限制,对于 $f\in \mathcal{E}$,
%   我们使用下面的记号表示积分 $\mu f$:
%   \[
%     \mu f=\Leb_E f=\int_E f(x)\Leb(\d x)=\int_E f(x)\d x.  
%   \]
% \end{example}

% \subsubsection{可积性}

% 对于一个函数 $f\in \mathcal{E}$,如果 $\mu f$ 存在且为实数,
% 那么 $f$ 被称为\emph{可积的}。也就是说,$f$ 可积当且仅当 
% $\mu f^+<\infty$ 以及 $\mu f^-<\infty$。

% \subsubsection{在可测集上的积分}

% 令 $f\in \mathcal{E}$,$A$ 是可测集,那么 $f1_A\in \mathcal{E}$,
% 此时我们把 $f$ 在 $A$ 上的积分定义为 $f1_A$ 的积分,使用下面的记号:
% \[
%   \mu(f1_A)=\int_A f(x)\mu(\d x)=\int_A f \d\mu.  
% \]

% \begin{lemma}
%   令 $f\in \mathcal{E}_+$,$A,B\in \mathcal{E}$ 不相交且 $C=A\cup B$,
%   那么
%   \[
%     \mu(f1_A)+\mu(f1_B)=\mu(f1_C).  
%   \]
%   对于一般的 $f\in \mathcal{E}$ 也成立。
% \end{lemma}
% \begin{proof}
%   令 $f_n=d_n\circ f$,由于 $f_n$ 是简单函数,所以
%   \[
%     \mu(f_n1_A)+\mu(f_n1_B)=\mu(f_n1_C),  
%   \]
%   注意到 $f_n1_A=d_n\circ(f1_A)$,对 $B,C$ 同理。令 $n\to\infty$ 即得结论。
% \end{proof}

% \subsubsection{非负性和单调性}

% \begin{proposition}
%   若 $f\in \mathcal{E}_+$,那么 $\mu f\geq 0$。如果 $f,g\in \mathcal{E}_+$
%   且 $f\leq g$,那么 $\mu f\leq \mu g$。
% \end{proposition}
% \begin{proof}
%   $f\in \mathcal{E}_+$ 表明 $f_n\geq 0$,从而 $\mu f_n\geq 0$,
%   所以 $\mu f=\lim\mu f_n\geq 0$。如果 $f\leq g$,由于 $d_n$ 是单调递增函数,
%   所以 $f_n\leq g_n$,所以 $\mu f_n\leq \mu g_n$,所以 $\mu f\leq \mu g$。
% \end{proof}
% \begin{corollary}
%   若 $f,g\in \mathcal{E}$ 且 $f\leq g$,那么 $\mu f\leq \mu g$.
% \end{corollary}
% \begin{proof}
%   $f\leq g$ 表明 $f^+\leq g^+$ 以及 $f^-\geq g^-$,所以
%   $\mu f=\mu(f^+)-\mu(f^-)\leq \mu(g^+)-\mu (g^-)\leq \mu g$。
% \end{proof}

% \subsubsection{单调收敛定理}

% 该定理是交换积分和极限次序的关键工具。该定理表明映射
% \[
%   \mathcal{E}_+\to\bar{\mathbb{R}}_+,\quad f\mapsto \mu f  
% \]
% 在递增极限下是连续的。

% \begin{theorem}
%   令 $(f_n)$ 是 $\mathcal{E}_+$ 中的递增序列,那么
%   \[
%     \mu(\lim f_n)=\lim\mu f_n.  
%   \]
% \end{theorem}
% \begin{proof}
%   令 $f=\lim f_n\in \mathcal{E}_+$,由于 $f_n\leq f$,根据单调性,有
%   $\mu f_n\leq \mu f$,故
%   \[
%     \lim \mu f_n\leq \mu f.  
%   \]

%   任取满足 $0\leq s\leq f$ 的非负简单函数 $s$,给定 $0<\alpha <1$,
%   定义
%   \[
%     A_n=\{x\in E:f_n(x)\geq \alpha s(x)\},
%   \]
%   那么 $A_n=(f_n-\alpha s)^{-1}[0,\infty]\in \mathcal{E}$。不难验证
%   $A_n\subseteq A_{n+1}$。对于任意 $x\in E$,由于 $f_n\nearrow f$
%   且 $f(x)\geq s(x)$,所以总存在足够大的 $n$ 使得 $f_n(x)\geq s(x)>\alpha s(x)$,
%   即 $x\in A_n$,所以 $A_n\nearrow E$。定义 $(E,\mathcal{E})$ 上的测度
%   $\nu$ 为
%   \[
%     \nu(A)=\mu(s1_A)=\int_A s\d \mu,  
%   \]
%   不难验证这确实是一个测度。此时我们有
%   \[
%     \mu f_n\geq \mu(f_n1_{A_n})\geq \mu(\alpha s1_{A_n})=\alpha\mu(s1_{A_n})
%     =\alpha\nu(A_n),  
%   \]
%   令 $n\to\infty$,由于 $A_n\nearrow E$,所以 $\nu(A_n)\nearrow \nu(E)=\mu s$,
%   所以
%   \[
%     \lim \mu f_n\geq \alpha\mu s.  
%   \]
%   特别地,取 $s=d_k\circ f$,有 $\lim \mu f_n\geq \alpha\mu(d_k\circ f)$,
%   令 $k\to\infty$,所以 $\lim \mu f_n\geq \alpha\mu f$。再取
%   $\alpha=1-1/k$,令 $k\to\infty$,即得
%   \[
%     \lim\mu f_n\geq \mu f.  \qedhere
%   \]
% \end{proof}

% \subsubsection{积分的线性性}

% \begin{proposition}
%   对于 $f,g\in \mathcal{E}_+$ 和 $a,b\in \mathbb{R}_+$,有
%   \[
%     \mu(af+bg)=a\mu f+b\mu g.  
%   \]
%   对于 $f,g\in \mathcal{E}$ 和 $a,b\in \mathbb{R}$ 也是正确的。
% \end{proposition}
% \begin{proof}
%   已知对于简单函数 $f_n=d_n\circ f$ 和 $g_n=d_n\circ g$,有
%   \[
%     \mu(af_n+bg_n)=a\mu f_n+b\mu g_n,  
%   \]
%   而 $f_n\nearrow f$,$g_n\nearrow g$,根据单调收敛定理,就有
%   \[
%     \mu(af+bg)=a\mu f+b\mu g.  
%   \]
%   对于一般的 $f,g$,只需将其拆为正部分和负部分即可验证。
% \end{proof}

% \subsubsection{积分的不敏感性}

% \begin{proposition}
%   如果 $A\in \mathcal{E}$ 是零测集,那么对于任意 $f\in \mathcal{E}$,有 $\mu(f1_A)=0$。
%   如果 $f,g\in \mathcal{E}$ 且几乎处处有 $f=g$,那么 $\mu f=\mu g$。
%   如果 $f\in \mathcal{E}_+$ 并且 $\mu f=0$,那么几乎处处 $f=0$。
% \end{proposition}
% \begin{proof}
%   先假设 $f$ 是简单非负函数,其标准型为 $f=\sum_i a_i 1_{A_i}$,那么
%   \[
%     f1_A=\sum_{i=1}^n a_i 1_{A_i\cap A},
%   \]
%   所以 $\mu(f1_A)=\sum_i a_i\mu(A_i\cap A)$。根据单调性有 $\mu(A_i\cap A)\leq \mu(A)=0$,
%   所以 $\mu(f1_A)=0$。然后假设 $f\in \mathcal{E}_+$,那么 $\mu((d_n\circ f)1_A)=0$,所以
%   $\mu(f1_A)=0$。最后,设 $f\in \mathcal{E}_+$,那么 $\mu(f1_A)=\mu(f^+1_A)-\mu (f^-1_A)=0$。

%   记 $A=\{x\in E:f(x)\neq g(x)\}$,那么 $A=E \smallsetminus(f-g)^{-1}(0)$ 可测,
%   几乎处处 $f=g$ 表明 $A$ 是零测集。于是 $\mu(f1_A)=\mu(g1_A)=0$,所以
%   \begin{align*}
%     \mu f&=\mu(f1_A)+\mu(f1_{E \smallsetminus A})=\mu(f1_{E \smallsetminus A})
%     =
%     \mu(g1_{E \smallsetminus A})\\&=\mu(g1_{E \smallsetminus A})+\mu(g1_A)=\mu g.
%   \end{align*}

%   记 $N=\{x\in E: f(x)>0\}$,$N_k=\{x\in E:f(x)>1/k\}$,显然 $N_k\nearrow N$,故
%   $\mu(N_k)\nearrow \mu(N)$。此时 $f> 1/k1_{N_k}$,所以 $0=\mu f\geq 1/k\mu(N_k)$,
%   这表明 $\mu(N_k)=0$,所以 $\mu(N)=\lim\mu(N_k)=0$。
% \end{proof}


% \subsubsection{Fatou 引理}

% \begin{lemma}
%   令 $(f_n)\subseteq \mathcal{E}_+$,那么 $\mu(\liminf f_n)\leq \liminf\mu f_n$。
% \end{lemma}
% \begin{proof}
%   记 $g_m=\inf_{n\geq m}f_n$,那么 $g_m\in \mathcal{E}_+$ 且递增,根据单调收敛定理,有
%   \[
%     \mu(\liminf f_n)=\mu(\lim g_m)=\lim\mu g_m.
%   \]
%   又因为 $n\geq m$ 的时候 $g_m\leq f_n$,所以 $\mu g_m\leq \mu f_n$,所以
%   $\mu g_m\leq \inf_{n\geq m} \mu f_n$,令 $m\to\infty$,即得
%   \[
%     \mu(\liminf f_n)=\lim\mu g_m\leq \liminf \mu f_n.\qedhere
%   \]
% \end{proof}

% \begin{corollary}
%   令 $(f_n)\subseteq \mathcal{E}$,如果存在可积函数 $g$ 使得 $f_n\geq g$,那么
%   \[
%     \mu(\liminf f_n)\leq\liminf \mu f_n.
%   \]
%   如果存在可积函数 $g$ 使得 $f_n\leq g$,那么
%   \[
%     \mu(\limsup f_n)\geq \limsup \mu f_n.
%   \]
% \end{corollary}
% \begin{proof}
%   若可积函数 $g$ 使得 $f_n\geq g$,那么 $A=\{x\in E:g(x)\in \mathbb{R}\}$ 的补集是零测集(练习)。
%   那么几乎处处 $f_n1_A=f_n$ 以及 $g1_A=g$。由于 $g1_A$ 是实值函数,所以
%   $f_n1_A-g1_A$ 是有意义的,故 $f_n1_A-g1_A\in \mathcal{E}_+$ 且可积,根据 Fatou 引理,有
%   \begin{align*}
%     \mu\bigl(\liminf (f_n1_A)\bigr)-\mu(g1_A)&=\mu\bigl(\liminf (f_n1_A)- g1_A\bigr)
%     =\mu\bigl(\liminf (f_n1_A-g1_A)\bigr)\\
%     &\leq \liminf\mu(f_n1_A-g1_A)=\liminf \mu(f_n1_A)-\mu(g1_A),
%   \end{align*}
%   由于几乎处处 $f_n1_A=f_n$,故几乎处处 $\liminf (f_n1_A)=\liminf f_n$,
%   故 $\mu(f_n1_A)=\mu f_n$ 以及 $\mu\bigl(\liminf (f_n1_A)\bigr)=\mu(\liminf f_n)$,这就表明
%   \[
%     \mu(\liminf f_n)\leq\liminf \mu f_n.
%   \]
%   对于第二点,考虑 $g1_A-f_n1_A\in \mathcal{E}_+$ 并且 $\limsup r_n=-\liminf(-r_n)$ 即可。
% \end{proof}


\part{测度论}

\chapter{可测空间}

\section{可测集}

\begin{definition}
  集合 $E$ 上的 $\sigma$-域 $\mathcal{A}$ 指的是 $E$ 的一个子集族,
  其满足下面的性质:
  \begin{enumerate}
    \item $E\in \mathcal{A}$;
    \item $A\in \mathcal{A}\Rightarrow A^c\in \mathcal{A}$;
    \item 如果一列子集 $A_n\in \mathcal{A}$,那么
    $\bigcup_{n\in \mathbb{N}} A_n\in \mathcal{A}$。
  \end{enumerate}
\end{definition}

$\mathcal A$ 的元素被称为\emph{可测集},$(E,\mathcal{A})$ 被称为\emph{可测空间}。
根据定义,我们很容易得出下面的结果:
\begin{itemize}[nosep]
  \item $\emptyset=E^c\in \mathcal{A}$。
  \item 如果一列子集 $A_n\in \mathcal{A}$,那么
  \[
    \bigcap_{n\in \mathbb{N}}A_n=\biggl(\bigcup_{n\in \mathbb{N}}A_n\biggr) ^c\in \mathcal{A}.
  \]
  \item $\mathcal{A}$ 对有限并和有限交也是封闭的,只需要从某一项 $A_n$ 开始
  全部取空集即可。
\end{itemize}

\begin{example}
  根据可测集的定义,很容易构造出一些最简单的例子:
  \begin{enumerate}
    \item $\mathcal{A}=\mathcal{P}(E)$,当 $E$ 是有限集或者可数集的时候
    我们通常会使用这样的 $\sigma$-域,其他情况则很少使用。
    \item $\mathcal{A}=\{\emptyset,E\}$,平凡 $\sigma$-域。
    \item $E$ 的所有至多可数的子集以及所有补集至多可数的子集构成
    $E$ 上的一个 $\sigma$-域。
  \end{enumerate}
\end{example}

为了产生更多的例子,我们注意到 $E$ 上任意 $\sigma$-域的交集仍然是
$\sigma$-域,这导出了下面的定义。

\begin{definition}
  令 $\mathcal{C}$ 是 $\mathcal{P}(E)$ 的子集,$E$ 上包含 $\mathcal{C}$
  的最小的 $\sigma$-域被记为 $\sigma(\mathcal{C})$,不难看出其是所有包含 $\mathcal{C}$
  的 $\sigma$-域的交集。我们称 $\sigma(\mathcal{C})$ 是由 $\mathcal{C}$ 生成的
  $\sigma$-域。
\end{definition}

\begin{definition}
  设 $(E,\mathcal{O})$ 是拓扑空间,所有开集 $\mathcal{O}$ 生成的 $\sigma$-域
  $\sigma(\mathcal{O})$ 被称为 $E$ 上的 Borel $\sigma$-域,记为 $\mathcal{B}(E)$。
\end{definition}

$E$ 上的 Borel $\sigma$-域是包含所有开集的最小的 $\sigma$-域。
$\mathcal{B}(E)$ 的元素被称为 $E$ 的\emph{Borel 子集}。显然,
$E$ 中的闭集也都是 Borel 子集。

\begin{example}[$\mathbb{R}$ 上的 Borel $\sigma$-域]
  记 $\mathcal{C}_1$ 为 $\mathbb{R}$ 中开区间的集合:
  \[
    \mathcal{C}_1=\{(a,b)\,|\, a,b\in \mathbb{R},a<b\}  ,
  \]
  显然有 $\mathcal{C}_1\subseteq \mathcal{B}(\mathbb{R})$,于是
  $\sigma(\mathcal{C}_1)\subseteq \mathcal{B}(\mathbb{R})$。
  下面我们说明 $\mathcal{B}(\mathbb{R})\subseteq \sigma(\mathcal{C}_1)$。
  我们不加证明地使用一个结论(Lindel\"of 定理):$\mathbb{R}$ 的任意开子集 $U$ 都是开区间的可数并。
  那么根据 $\sigma$-域的定义,任意开区间都在 $\sigma(\mathcal{C}_1)$ 中,
  故 $\mathcal{B}(\mathbb{R})\subseteq \sigma(\mathcal{C}_1)$。
  这表明 $\mathcal{B}(\mathbb{R})$ 可以由所有开区间生成。

  此外,如果注意到
  \[
    (a,b)=(-\infty,b)\cap (-\infty,a)^c,  
  \]
  还可以证明 $\mathcal{B}(\mathbb{R})$ 由 $\mathcal{C}_2$ 生成,其中
  \[
    \mathcal{C}_2=\{(-\infty,a)\,|\, a\in \mathbb{R}\}.
  \]
  最后,不难证明这里的开区间都可以换成闭区间。
\end{example}

在后文中,每当我们考虑拓扑空间(例如 $\mathbb{R}$ 或者 $\mathbb{R}^d$)时,
除非有特别说明,否则我们总是假设它们配备 Borel $\sigma$-域。

下一个非常重要的 $\sigma$-域是乘积 $\sigma$-域。

\begin{definition}
  令 $(E_1,\mathcal{A}_1)$ 和 $(E_2,\mathcal{A}_2)$ 是可测空间,定义
  $E_1\times E_2$ 上的 $\sigma$-域 $\mathcal{A}_1\otimes \mathcal{A}_2$ 为
  \[
    \mathcal{A}_1\otimes \mathcal{A}_2=\sigma\bigl(\{A_1\times A_2\,|\, A_1\in \mathcal{A}_1,A_2\in \mathcal{A}_2\}\bigr).
  \]
\end{definition}

\begin{lemma}
  设 $E$ 和 $F$ 是可分(有可数的稠密子集)的拓扑空间,$E\times F$ 配备积拓扑,那么
  $\mathcal{B}(E\times F)=\mathcal{B}(E)\otimes \mathcal{B}(F)$。
\end{lemma}


\section{正测度}

令 $(E,\mathcal{A})$ 是可测空间。

\begin{definition}
  $(E,\mathcal{A})$ 上的正测度指的是一个映射 $\mu:\mathcal{A}\to [0,\infty]$,
  其满足下面的性质:
  \begin{enumerate}
    \item $\mu(\emptyset)=0$;
    \item ($\sigma$-可加性) 对于任意可数个不相交的可测集序列 $(A_n)_{n\in \mathbb{N}}$,有
    \[
      \mu\biggl(\bigcup_{n\in \mathbb{N}}A_n\biggr) =\sum_{n\in \mathbb{N}}\mu(A_n).
    \]
  \end{enumerate}
  此时,三元组 $(E,\mathcal{A},\mu)$ 被称为\emph{测度空间}。
  值 $\mu(E)$ 被称为测度 $\mu$ 的总质量。
\end{definition}

需要注意的是,我们允许 $\mu$ 的值为 $+\infty$,此时级数
$\sum_{n\in \mathbb{N}}\mu(A_n)$ 作为正向级数在 $[0,\infty]$
中总是有意义的。
根据 $\sigma$-可加性,如果我们令 $n>n_0$ 开始 $A_n=\emptyset$,
便可以得到有限可加性。

\begin{proposition}[测度的性质]
  根据定义,测度 $\mu$ 满足下面的性质:
  \begin{enumerate}
    \item 如果 $A\subseteq B$,那么 $\mu(A)\leq\mu(B)$。此外,如果
    还满足 $\mu(A)<\infty$,那么
    \[
        \mu(B \smallsetminus A)=\mu(B)-\mu(A).
    \]
    \item 如果 $A,B\in \mathcal{A}$,那么
    \[
      \mu(A)+\mu(B)=\mu(A\cup B)+\mu(A\cap B).  
    \]
    \item 如果 $A_n\in \mathcal{A}$ 且 $A_n\subseteq A_{n+1}$,那么
    \[
      \mu\biggl(\bigcup_{n\in \mathbb{N}}A_n\biggr)  
      =\lim_{n\to\infty}\mu(A_n).
    \]
    \item 如果 $B_n\in \mathcal{A}$ 且 $B_{n+1}\subseteq B_n$,
    $\mu(B_1)<\infty$,那么
    \[
      \mu\biggl(\bigcap_{n\in \mathbb{N}}B_n\biggr)  
      =\lim_{n\to\infty}\mu(B_n).
    \]
    \item 如果 $A_n\in \mathcal{A}$,那么
    \[
      \mu\biggl(\bigcup_{n\in \mathbb{N}}A_n\biggr)\leq \sum_{n\in \mathbb{N}}\mu(A_n).
    \]
  \end{enumerate}
\end{proposition}
\begin{proof}
  (1) 若 $A\subseteq B$,那么 $B=A\bigcup (B \smallsetminus A)$ 是无交并,所以
  \[
    \mu(B)=\mu(A)+\mu(B \smallsetminus A)\geq \mu(A).  
  \]

  (2) 若 $\mu(A),\mu(B)$ 中有至少一个为无穷,那么根据 (1),
  $\mu(A\cup B)$ 为无穷,所以结论成立。下面假设
  $\mu(A),\mu(B)$ 均有限,记 $C=A\cap B$,那么
  $A\cup B=(A \smallsetminus C)\cup C\cup(B \smallsetminus C)$ 是无交并,
  所以
  \[
    \mu(A\cup B)=\mu(A \smallsetminus C)+\mu(C)+\mu(B \smallsetminus C)
    =\mu(A)+\mu(B)-\mu(C),
  \]
  结论 (2) 成立。

  (3) 令 $C_1=A_1$,对于 $n\geq 2$ 的时候,令
  \[
    C_n=A_{n}  \smallsetminus A_{n-1},
  \]
  那么 $A_n=\bigcup_{k\leq n}C_k$ 是无交并,所以
  \[
    \mu\biggl(\bigcup_{n\in \mathbb{N}}A_n\biggr)  
    =\mu\biggl(\bigcup_{n\in \mathbb{N}}C_n \biggr)
    =\sum_{n\in \mathbb{N}}\mu(C_n)=\lim_{n\to\infty}
    \sum_{k=1}^n \mu(C_k)=\lim_{n\to\infty} \mu(A_n).
  \]

  (4) 令 $A_n=B_1 \smallsetminus B_n$,那么 $A_n\subseteq A_{n+1}$,此时
  \[
    \mu\biggl(\bigcap_{n\in \mathbb{N}}B_n\biggr)  =
    \mu(B_1)-\mu\biggl(B_1 \smallsetminus\bigcap_{n\in \mathbb{N}}B_n\biggr)
    = \mu(B_1)-\mu\biggl(\bigcup_{n\in \mathbb{N}}A_n\biggr),
  \]
  再根据 (3),就有
  \[
    \mu\biggl(\bigcap_{n\in \mathbb{N}}B_n\biggr)  =\mu(B_1)-\lim_{n\to\infty}\mu(A_n)
    =\lim_{n\to\infty}\mu(B_1 \smallsetminus A_n)=\lim_{n\to\infty}\mu(B_n).
  \]

  (5) 令 $C_1=A_1$,对于 $n\geq 2$ 的时候,令
  \[
    C_n=A_n \smallsetminus\bigcup_{k=1}^{n-1}A_k,
  \]
  那么 $C_n$ 之间互不相交,所以
  \[
    \mu\biggl(\bigcup_{n\in \mathbb{N}}A_n\biggr)=\mu\biggl(\bigcup_{n\in \mathbb{N}}C_n\biggr)
    =\sum_{n\in \mathbb{N}}\mu(C_n)\leq \sum_{n\in \mathbb{N}}\mu(A_n).\qedhere
  \]
\end{proof}

\begin{example}[常见的测度]
  \mbox{}
  \begin{enumerate}
    \item 令 $E=\mathbb{N}$,$A=\mathcal{P}(\mathbb{N})$,定义%
    \emph{计数测度}为
    \[
      \mu(A)=\card(A).  
    \]
    \item 如果 $A$ 是 $E$ 的子集,定义 $A$ 的示性函数 $\mathbf{1}_A:E\to\{0,1\}$
    为
    \[
      \mathbf{1}_A(x)=\begin{cases}
        1 & x\in A,\\
        0 & x\notin A.
      \end{cases}  
    \]
    令 $(E,\mathcal{A})$ 是可测空间,固定 $x\in E$。对于每个 $A\in \mathcal{A}$,令
    $\delta_x(A)=\mathbf{1}_A(x)$,这给出了 $(E,\mathcal{A})$ 上的一个测度,被称为
    \emph{$\mathbold x$ 处的 Dirac 测度}。更一般的,如果 $(x_n)_{n\in \mathbb{N}}$
    是 $E$ 中的点列,$(\alpha_n)_{n\in \mathbb{N}}$ 是 $[0,\infty]$ 中的点列,
    我们可以考虑测度 $\sum_{n\in \mathbb{N}}\alpha_n\delta_{x_n}$ 为
    \[
      \biggl(\sum_{n\in \mathbb{N}}\alpha_n\delta_{x_n}\biggr)
      (A)=\sum_{n\in \mathbb{N}}\alpha_n\delta_{x_n}(A)=
      \sum_{n\in \mathbb{N}}\alpha_n\mathbold 1_{A}(x_n),
    \]
    这个测度被称为 $E$ 上的\emph{点测度}。
    \item 可以证明,在 $(\mathbb{R},\mathcal{B}(\mathbb{R}))$ 上存在唯一的正测度 $\lambda$
    使得:对于每个闭区间 $[a,b]$,有 $\lambda\bigl([a,b]\bigr)=b-a$。这个
    测度 $\lambda$ 被称为\emph{Lebesgue 测度}。
    Lebesgue 测度的唯一性可以由 \autoref{coro:uniqueness of measure} 保证,
    存在性由 ?保证。
  \end{enumerate}
\end{example}

如果 $\mu$ 是 $(E,\mathcal{A})$ 上的正测度,$C\in \mathcal{A}$,
那么可以定义 $\mu$ 在 $C$ 上的\emph{限制} $\nu$ 为:
\[
  \nu(A)=\mu(A\cap C),\quad \forall A\in \mathcal{A}.
\]
不难验证 $\nu$ 还是 $(E,\mathcal{A})$ 上的正测度。

\begin{definition}
  \mbox{}
  \begin{itemize}[nosep]
    \item 如果 $\mu(E)<\infty$,那么我们说测度 $\mu$ 是\emph{有限的}。
    \item 如果 $\mu(E)=1$,那么我们说测度 $\mu$ 是\emph{概率测度},$(E,\mathcal{A},\mu)$ 是\emph{概率空间}。
    \item 如果存在一列可测集 $(E_n)_{n\in \mathbb{N}}$ 使得 $E=\bigcup_n E_n$ 以及每个
    $\mu(E_n)<\infty$,那么我们说测度 $\mu$ 是\emph{$\mathbold\sigma$-有限的}。
    \item 如果 $x\in E$ 使得单点集 $\{x\}\in \mathcal{A}$ 并且 $\mu(\{x\})>0$,那么我们说
    $x$ 是测度 $\mu$ 的一个\emph{原子}。
    \item 如果测度 $\mu$ 没有原子,那么我们说 $\mu$ 是\emph{扩散测度}。
  \end{itemize}
\end{definition}

如果 $(A_n)_{n\in \mathbb{N}}$ 是一列可测集,类比数列的上下极限,我们可以定义
集合列的上下极限分别为:
\[
  \limsup A_n=\bigcap_{n=1}^\infty\biggl(\bigcup_{k=n}^\infty A_k\biggr),\quad
  \liminf A_n=\bigcup_{n=1}^\infty\biggl(\bigcap_{k=n}^\infty A_k\biggr).
\]
注意到对于任意 $m$,都有
\[
  \bigcup_{n=1}^m\biggl(\bigcap_{k=n}^\infty A_k\biggr)
  =\bigcap_{k=m}^\infty A_k,\quad 
  \bigcap_{n=1}^m\biggl(\bigcup_{k=n}^\infty A_k\biggr)=\bigcup_{k=m}^\infty A_k,
\]
所以显然有 $\liminf A_n\subseteq \limsup A_n$。

\begin{lemma}\label{lemma:liminf and limsup ineq}
  令 $\mu$ 是 $(E,\mathcal{A})$ 上的测度,那么
  \[
    \mu(\liminf A_n)\leq \liminf \mu(A_n).
  \]
  如果 $\mu$ 是有限测度,或者更一般地,$\mu\left(\bigcup_{n=1}^\infty A_n\right)<\infty$,
  那么
  \[
    \mu(\limsup A_n)\geq\limsup\mu(A_n).
  \]
\end{lemma}
\begin{proof}
  对于任意的 $n$,有 
  \[
    \mu\biggl(\bigcap_{k=n}^\infty A_k\biggr)\leq \inf_{k\geq n}\mu(A_k),
  \]
  所以
  \[
    \mu(\liminf A_n)=\lim_{n\to\infty}\mu\biggl(\bigcap_{k=n}^\infty A_k\biggr)
    \leq \lim_{n\to\infty}\inf_{k\geq n}\mu(A_k)=\liminf \mu(A_n).
  \]
  第二个结论同理。
\end{proof}

\section{可测函数}

\begin{definition}
  令 $(E,\mathcal{A})$ 和 $(F,\mathcal{B})$ 是两个可测空间,如果映射 $f:E\to F$ 满足:
  \[
    \forall B\in \mathcal{B},\  f^{-1}(B)\in \mathcal{A},
  \]
  那么我们说 $f$ 是\emph{可测映射}。当 $E,F$ 是两个配备了 Borel $\sigma$-域的拓扑空间时,
  我们说 $f$ 是\emph{Borel 可测的}。
\end{definition}

显然,可测映射的复合是可测映射。

\begin{proposition}
  令 $(E,\mathcal{A})$ 和 $(F,\mathcal{B})$ 是两个可测空间,映射 $f:E\to F$。$f$ 可测
  当且仅当对于某个生成 $\mathcal{B}$ 的子集族 $\mathcal{C}$ (即 $\mathcal{B}=\sigma(\mathcal{C})$),
  有 $f^{-1}(B)\in \mathcal{A}\ (\forall B\in \mathcal{C})$。
\end{proposition}
\begin{proof}
  只需证明充分性。记
  \[
    \mathcal{G}=\{B\in \mathcal{B}\,|\, f^{-1}(B)\in \mathcal{A}\},
  \]
  直接验证可知 $\mathcal{G}$ 是一个 $\sigma$-域,又因为 $\mathcal{C}\subseteq \mathcal{G}$,
  所以 $\mathcal{B}=\sigma(\mathcal{C})\subseteq \mathcal{G}\subseteq \mathcal{B}$,
  所以 $\mathcal{G}=\mathcal{B}$,这就表明 $f$ 是可测的。
\end{proof}

\begin{example}
  若 $(F,\mathcal{B})=(\mathbb{R},\mathcal{B}(\mathbb{R}))$,要证明 $f$ 是可测的,只需说明
  集合 $f^{-1}((a,b))$ 是可测的,或者 $f^{-1}((-\infty,a))$ 是可测的。
\end{example}

\begin{corollary}
  设 $E,F$ 是两个配备 Borel $\sigma$-域的拓扑空间,那么连续映射 $f:E\to F$ 都是可测的。
\end{corollary}

\begin{lemma}
  令 $(E,\mathcal{A}),(F_1,\mathcal{B}_1)$ 和 $(F_2,\mathcal{B}_2)$ 是可测空间,乘积
  $F_1\times F_2$ 配备乘积 $\sigma$-域 $\mathcal{B}_1\otimes \mathcal{B}_2$,令映射
  $f_1:E\to F_1$ 和 $F_2:E\to F_2$,定义 $f:E\to F_1\times F_2$ 为 $f(x)=(f_1(x),f_2(x))$,
  那么 $f$ 可测当且仅当 $f_1,f_2$ 都可测。
\end{lemma}

\begin{corollary}
  令 $(E,\mathcal{A})$ 是可测空间,$f,g$ 是从 $E$ 到 $\mathbb{R}$ 的可测函数,那么函数
  \[ 
    f+g,fg,\min(f,g),\max(f,g)
  \] 
  都是可测的。
\end{corollary}

记扩充实数 $\bar{\mathbb{R}}=\mathbb{R}\cup\{-\infty,+\infty\}$,其拓扑为序拓扑。
与 $\mathbb{R}$ 类似,$\bar{\mathbb{R}}$ 的 Borel $\sigma$-域由区间 $[-\infty,a)$
生成。

\begin{proposition}
  令 $(f_n)_{n\in \mathbb{N}}$ 是 $E\to\bar{\mathbb{R}}$ 的可测函数列,那么
  \[
    \sup f_n,\quad \inf f_n,\quad \limsup_{n\to\infty}f_n,\quad \liminf_{n\to\infty} f_n
  \]
  都是可测函数。特别地,如果 $(f_n)$ 逐点收敛,那么极限 $\lim f_n$ 是可测函数。
\end{proposition}

\begin{definition}
  令 $(E,\mathcal{A})$ 和 $(F,\mathcal{B})$ 是可测空间,$\varphi:E\to F$ 是可测映射, 
  $\mu$ 是 $(E,\mathcal{A})$ 上的测度,定义 $(F,\mathcal{B})$ 上的测度 $\nu$ 为
  \[
    \nu(B)=\mu(\varphi^{-1}(B)),\quad \forall B\in \mathcal{B}.
  \]
  $\nu$ 被称为\emph{$\mathbold\mu$ 在 $\mathbold\varphi$ 下的推前},记为
  $\varphi(\mu)$,有时也记为 $\varphi_*\mu$。
\end{definition}

\section{单调类}

本节我们陈述单调类定理,这是测度论甚至概率论中的一个基本工具。

\begin{definition}
  $\mathcal{P}(E)$ 的一个子集 $\mathcal{M}$ 如果满足:
  \begin{enumerate}
    \item $E\in \mathcal{M}$;
    \item 对于任意 $A,B\in \mathcal{M}$ 且 $A\subseteq B$,有
    $B \smallsetminus A\in \mathcal{M}$;
    \item 如果一列子集 $A_n\subseteq \mathcal{M}$ 且 $A_n\subseteq A_{n+1}$,
    那么 $\bigcup_{n\in \mathbb{N}}A_n\in \mathcal{M}$,
  \end{enumerate}
  那么我们说 $\mathcal{M}$ 是一个\emph{单调类}。
\end{definition}

显然,$\sigma$-域都是单调类。反之,一个单调类是 $\sigma$-域当且仅当
其对有限交封闭。这很容易证明,若单调类 $\mathcal{M}$ 对有限交封闭,那么
任取一列子集 $A_n\subseteq \mathcal{M}$,对于任意的 $n$,有 
\[
  \bigcup_{k=1}^n  A_k=E \smallsetminus\bigcap_{k=1}^n A_k^c\in \mathcal{M},
\]
所以
\[
  \bigcup_{n\in \mathbb{N}}A_n=\bigcup_{n\in \mathbb{N}}
  \biggl(\bigcup_{k=1}^n A_k\biggr)  \in \mathcal{M},
\]
这就表明 $\mathcal{M}$ 是一个 $\sigma$-域。

与 $\sigma$-域类似,显然单调类的任意交仍然是单调类。如果
$\mathcal{C}\subseteq \mathcal{P}(E)$,那么我们可以定义
由 $\mathcal{C}$ 生成的单调类 $\mathcal{M}(\mathcal{C})$,
即包含 $\mathcal{C}$ 的最小的单调类,其可以通过对所有
包含 $\mathcal{C}$ 的单调类取交集得到。

\begin{theorem}[单调类定理]
  令 $\mathcal{C}\subseteq \mathcal{P}(E)$ 对有限交封闭,
  那么 $\mathcal{M}(\mathcal{C})=\sigma(\mathcal{C})$。
  因此,如果 $\mathcal{M}$ 是包含 $\mathcal{C}$ 的任意单调类,
  那么 $\sigma(\mathcal{C})\subseteq \mathcal{M}$。
\end{theorem}
\begin{proof}
  显然有 $\mathcal{M}(\mathcal{C})\subseteq \sigma(\mathcal{C})$。
  要证明 $\sigma(\mathcal{C})\subseteq \mathcal{M}(\mathcal{C})$,
  只需要说明 $\mathcal{M}(\mathcal{C})$ 是 $\sigma$-域。
  根据上面的叙述,这只需要说明 $\mathcal{M}(\mathcal{C})$ 对
  有限交封闭。

  对于 $A\in \mathcal{P}{(E)}$,记
  \[
    \mathcal{M}_A=\{B\in \mathcal{M}(\mathcal{C})\,|\,
    A\cap B\in \mathcal{M}(\mathcal{C})\}.  
  \]
  直接验证可知 $\mathcal{M}_A$ 是一个单调类。
  下面任取 $A\in \mathcal{C}$,由于 $\mathcal{C}$
  对有限交封闭,所以 $\mathcal{C}\subseteq \mathcal{M}_A$,这就表明
  $\mathcal{M}(\mathcal{C})\subseteq \mathcal{M}_A$。

  接下来任取 $D\in \mathcal{M}(\mathcal{C})$,上面的叙述告诉我们
  $\mathcal{C}\subseteq \mathcal{M}_D$,所以 $\mathcal{M}(\mathcal{C})\subseteq \mathcal{M}_D$。
  这就表明 $\mathcal{M}(\mathcal{C})$ 对有限交封闭,所以
  $\mathcal{M}(\mathcal{C})$ 是 $\sigma$-域。
\end{proof}

单调类定理最重要的应用是证明某些测度的唯一性。

\begin{corollary}\label{coro:uniqueness of measure}
  令 $\mu,\nu$ 是 $(E,\mathcal{A})$ 上的两个测度。假设存在一个
  子集族 $\mathcal{C}\subseteq \mathcal{A}$ 满足 $\mathcal{C}$
  对有限交封闭且 $\mathcal{A}=\sigma(\mathcal{C})$,并且对于
  任意 $A\in \mathcal{C}$ 都有 $\mu(A)=\nu(A)$。
  \begin{enumerate}
    \item 如果 $\mu(E)=\nu(E)<\infty$,那么 $\mu=\nu$。
    \item 如果存在一列 $\mathcal{C}$ 中的递增序列 $(E_n)_{n\in \mathbb{N}}$ 
    使得 $E=\bigcup_{n\in \mathbb{N}}E_n$,并且
    $\mu(E_n)=\nu(E_n)<\infty$,那么 $\mu=\nu$。
  \end{enumerate}
\end{corollary}
\begin{proof}
  (1) 令
  \[
    \mathcal{G}=\{A\in \mathcal{A}\,|\,\mu(A)=\nu(A)\},
  \]
  那么 $\mathcal{C}\subseteq \mathcal{G}$ 且不难验证 $\mathcal{G}$
  是单调类,根据单调类定理,有 $\mathcal{A}=\sigma(\mathcal{C})\subseteq \mathcal{G}$,
  即 $\mu=\nu$。

  (2) 记 $\mu_n$ 为 $\mu$ 在 $E_n$ 上的限制,$\nu_n$ 同理。
  那么
  \[
    \mu_n(E)=\mu(E\cap E_n)=\mu(E_n)=\nu(E_n)=\nu(E\cap E_n)  
    =\nu_n(E),
  \]
  根据 (1),有 $\mu_n=\nu_n$。于是任取 $A\in \mathcal{A}$,有
  \begin{align*}
    \mu (A)&=\mu(A\cap E)=\mu\biggl(\bigcup_{n\in \mathbb{N}}(A\cap E_n)\biggr)
    =\ulim[n\to\infty]\mu(A\cap E_n)\\
    &=\ulim[n\to\infty]\mu_n(A)=\ulim[n\to\infty]\nu_n(A)
    =\ulim[n\to\infty]\nu(A\cap E_n)\\
    &=\nu\biggl(\bigcup_{n\in \mathbb{N}}(A\cap E_n)\biggr)
    =\nu(A\cap E)=\nu(A),
  \end{align*}
  这就表明 $\mu=\nu$。
\end{proof}

\autoref{coro:uniqueness of measure} 表明了 Lebesgue 测度的唯一性。
即若 $\lambda$ 是 $(\mathbb{R},\mathcal{B}(\mathbb{R}))$ 上的 
正测度,且使得 $\lambda\bigl([a,b]\bigr)=b-a$,那么这样的测度 $\lambda$
是唯一的。这是因为我们可以取
\[
  \mathcal{C}=\big\{[a,b]\,|\, a,b\in \mathbb{R},a<b\big\}  ,
\]
此时 $\mathcal{C}$ 对有限交封闭并且 $\mathcal{B}(\mathbb{R})=\sigma(\mathcal{C})$。
取 $E_n=[-n,n]\in \mathcal{C}$,那么 $\mathbb{R}=\bigcup_{n\in \mathbb{N}} E_n$
且 $\lambda(E_n)<\infty$,应用 \autoref{coro:uniqueness of measure}
的 (2) 即可表明唯一性。








\chapter{可测函数的积分}

\section{非负函数的积分}

在本章中,我们考虑配备正测度 $\mu$ 的可测空间 $(E,\mathcal{A})$。

\paragraph{简单函数}
如果可测函数 $f:E\to \mathbb{R}$ 的值域是有限集,那么我们说 $f$ 的\emph{简单函数}。
假设 $f$ 的所有可能的取值为 $\alpha_1,\dots,\alpha_n$,不妨假设 $\alpha_1<\alpha_2<\cdots<\alpha_n$。
那么 $f$ 可以表示为
\[
  f(x)=\sum_{i=1}^n \alpha_i\mathbold 1_{A_i}(x),
\]
其中 $A_i=f^{-1}(\{\alpha_i\})\in \mathcal{A}$。注意到 $E$ 是 $A_1,\dots,A_n$ 的无交并。
上述公式 $f=\sum_{i=1}^n \alpha_i\mathbold 1_{A_i}$ 被称为 $f$ 的标准表示。

\begin{definition}
  令 $f$ 是取值在 $\mathbb{R}_+$ 中的简单函数,标准表示为 $f=\sum_{i=1}^n \alpha_i\mathbold 1_{A_i}$。
  定义\emph{$\mathbold f$ 相对于 $\mathbold\mu$ 的积分}为
  \[
    \int f\d\mu=\sum_{i=1}^n\alpha_i\mu(A_i).
  \]
  在 $\alpha_i=0$ 和 $\mu(A_i)=\infty$ 的情况下,约定 $0\times\infty=0$。
\end{definition}

注意上述定义中 $\sum_{i=1}^n\alpha_i\mu(A_i)$ 的取值为 $[0,\infty]$。所以在上述定义中
我们只考虑非负的简单函数,这是为了避免出现 $\infty-\infty$ 之类的表达式。

值得注意的是,如果简单函数 $f$ 有表达
\[
  f=\sum_{j=1}^m\beta_j\mathbold 1_{B_j},
\]
其中 $B_j$ 仍然构成 $E$ 的一个划分,但是 $\beta_j$ 不再是两两不同的。此时
$f$ 的积分仍然为
\[
  \int f\d\mu=\sum_{j=1}^m\beta_j\mu(B_j).
\]
这是因为对于每个 $A_i$,某些 $B_j$ 构成了 $A_i$ 的划分,即
\[
  A_i=\bigcup_{\{j\,|\,\beta_j=\alpha_i\}}B_j,
\]
那么
\[
  \alpha_i\mu(A_i)=\alpha_i\sum_{\{j\,|\,\beta_j=\alpha_i\}}\mu( B_j)=
  \sum_{\{j\,|\,\beta_j=\alpha_i\}}\beta_j\mu( B_j).
\]

非负简单函数的积分满足下面的一些基本的性质。

\begin{proposition}
  令 $f,g$ 是 $E$ 上的非负简单函数。
  \begin{enumerate}
    \item 对于每个 $a,b\in \mathbb{R}_+$,有
    \[
      \int(af+bg)\d\mu=a\int f\d\mu+b\int g\d\mu.
    \]
    \item 如果 $f\leq g$,那么
    \[
      \int f\d\mu\leq \int g\d\mu.
    \]
  \end{enumerate}
\end{proposition}
\begin{proof}
  (1) 设 $f,g$ 的标准表示分别为
  \[
    f=\sum_{i=1}^n\alpha_i\mathbold 1_{A_i},\quad  
    g=\sum_{j=1}^m\beta_j\mathbold 1_{B_j}.  
  \]
  那么每个 $A_i$ 都是某些 $A_i\cap B_j$ 的无交并,同理,
  每个 $B_j$ 都是某些 $A_i\cap B_j$ 的无交并,于是我们可以
  使用一个新的划分 $\{C_1,\dots,C_p\}$ 使得
  \[
    f=\sum_{k=1}^p \gamma_k\mathbold 1_{C_k} ,\quad
    g=\sum_{k=1}^p \theta_k\mathbold 1_{C_k} ,
  \]
  此时 $\gamma_k$ 不一定互不相同,$\theta_k$ 也不一定互不相同,
  根据命题前面的叙述,我们有
  \begin{align*}
    \int
    (af+bg)\d\mu&=\sum_{k=1}^p(a\gamma_k+b\theta_k)  \mu(C_k)\\&=
    a\sum_{k=1}^p\gamma_k\mu(C_k)+b\sum_{k=1}^p\theta_k\mu(C_k)\\
    &=a\int f\d\mu+b\int g\d\mu.
  \end{align*}

  (2) 由 (1),有
  \[
    \int g\d\mu=\int(g-f)\d\mu+\int f\d\mu\geq \int f\d\mu.\qedhere  
  \]
\end{proof}

我们用 $\mathcal{E}_+$ 来表示 $E$ 上的非负简单函数的集合。

\begin{definition}
  令 $f:E\to[0,\infty]$ 是可测函数,定义\emph{$\mathbold f$ 相对于 $\mathbold\mu$ 的积分}
  为
  \[
    \int f\d\mu=\sup_{h\in \mathcal{E}_+,h\leq f}\int h\d\mu.
  \]
\end{definition}

$f$ 相对于 $\mu$ 的积分通常有很多写法,下面的表达
\[
  \int f\d\mu,\ \int f(x)\d\mu(x),\ \int f(x)\mu(\d x),\ \int\mu(\d x)f(x)
\]
表示的含义是完全相同的。此外,如果 $A$ 是 $E$ 的可测子集,我们定义
\[
  \int_A f\d\mu=\int f\mathbold 1_A\d\mu.
\]

从现在开始,我们用非负可测函数表示 $E\to [0,\infty]$ 的可测函数(值可以为无穷)。
需要注意的是,我们前面定义的非负简单函数值必须有限。

\begin{proposition}\label{prop:elementary property of positive measurable function}
  令 $f,g$ 是 $E$ 上的非负可测函数。
  \begin{enumerate}
    \item 如果 $f\leq g$,那么 $\int f\d\mu\leq \int g\d\mu$。
    \item 如果 $\mu(\{x\in E\,|\, f(x)> 0\})=0$,那么 $\int f\d\mu=0$。
  \end{enumerate}
\end{proposition}
\begin{proof}
  (1) 显然
  \[
      \bigl\{h\in \mathcal{E}_+\,|\, h\leq f\bigr\}\subseteq 
      \bigl\{h\in \mathcal{E}_+\,|\, h\leq g\bigr\},
  \]
  根据定义即得 $\int f\d\mu\leq \int g\d\mu$。

  (2) 设 $h\in \mathcal{E}_+$ 且 $h\leq f$,设 $h$ 的标准表示
  为 $h=\sum_{i=1}^n \alpha_i\mathbold 1_{A_i}$,若 $\alpha_i> 0$,那么
  \[
    \mu(A_i)\leq   \mu (\{x\in E\,|\, h(x)> 0\})\leq \mu(\{x\in E\,|\, f(x)> 0\})=0,
  \]
  所以
  \[
    \int h\d\mu=\sum_{\{i\,|\,\alpha_i=0\}}\alpha_i\mu(A_i)+
    \sum_{\{i\,|\,\alpha_i>0\}}\alpha_i\mu(A_i)=0+0=0,
  \]
  故 $\int f\d\mu=0$。
\end{proof}

下面的单调收敛定理是测度论中的一个极为重要的基本定理,其表明
对于一列递增的非负可测函数,极限和积分可以交换次序。

\begin{theorem}[单调收敛定理]\label{thm:monotone convergence thm}
  令 $(f_n)_{n\in \mathbb{N}}$ 是 $E$ 上的一列递增的非负可测函数,即 $f_n\leq f_{n+1}$,
  记 $f=\ulim f_n$,那么
  \[
    \int f\d\mu=\ulim[n\to\infty]\int f_n\d\mu.
  \]
\end{theorem}
\begin{proof}
  由于 $f_n\leq f$,所以 $\int f_n\d\mu\leq\int f\d\mu$,
  所以 $\ulim \int f_n\d\mu\leq\int f\d\mu$,于是我们只需要证明
  反向的不等式。

  假设非负可测函数 $h=\sum_{i=1}^k \alpha_i\mathbold 1_{A_i}$
  满足 $h\leq f$,任取 $a\in[0,1)$,定义一列可测集
  \[
    E_n=\{x\in E\,|\, ah(x)\leq f_n(x)\}  ,
  \]
  此时对于任意的 $x\in E$,都有 $ah(x)<h(x)\leq f(x)$,而 $f=\ulim f_n$,
  所以总存在足够大的 $n$,使得 $ah(x)\leq f_n(x)$,这表明 $E=\bigcup_{n\in \mathbb{N}}E_n$。
  此外,$f_n\leq f_{n+1}$ 表明 $E_n\subseteq E_{n+1}$。

  显然 $f_n\geq ah\mathbold 1_{E_n}$,所以
  \[
    \int f_n\d\mu\geq a\int_{E_n} h\d\mu=
    a\sum_{i=1}^k\alpha_i\mu(A_i\cap E_n),  
  \]
  由于 $A_i=A_i\cap E=\bigcup_{n\in \mathbb{N}}(A_i\cap E_n)$,
  所以
  \[
    \mu(A_i)=\mu\biggl(\bigcup_{n\in \mathbb{N}}(A_i\cap E_n)\biggr)
    =\ulim[n\to\infty]\mu(A_i\cap E_n),
  \]
  于是
  \[
    \ulim[n\to\infty]\int f_n\d\mu\geq a\sum_{i=1}^k
    \alpha_i\ulim[n\to\infty]\mu(A_i\cap E_n)
    =  a\sum_{i=1}^k\alpha_i\mu(A_i)
    =a\int h\d\mu,
  \]
  由于 $a$ 可以任意接近 $1$,所以
  \[
    \ulim[n\to\infty]\int f_n\d\mu\geq \int h\d\mu,
  \]
  所以
  \[
    \ulim[n\to\infty]\int f_n\d\mu\geq \int f\d\mu=
    \sup_{h\in \mathcal{E}_+,h\leq f}\int h\d\mu.\qedhere
  \]
\end{proof}


\begin{proposition}\label{prop:property of integral of positive function}
  \mbox{}
  \begin{enumerate}
    \item 设 $f$ 是 $E$ 上的非负可测函数,那么存在一列递增的非负简单函数 $(f_n)_{n\in \mathbb{N}}$
    使得 $f=\ulim f_n$。如果 $f$ 有界,那么 $f_n\to f$ 一致收敛。
    \item 令 $f,g$ 是两个 $E$ 上的非负可测函数,$a,b\in \mathbb{R}_+$,那么
    \[
      \int (af+bg)\d\mu=a\int f\d\mu+b\int g\d\mu.
    \]
    \item 令 $(f_n)_{n\in \mathbb{N}}$ 是一列 $E$ 上的非负可测函数,那么
    \[
      \int\biggl(\sum_{n\in \mathbb{N}}f_n\biggr)\d\mu=\sum_{n\in \mathbb{N}}\int f_n\d\mu.
    \]
  \end{enumerate}
\end{proposition}
\begin{proof}
  (1) 令 $d_n:[0,\infty]\to \mathbb{R}_+$ 为
  \[
    d_n=\sum_{k=1}^{n2^n}\frac{k-1}{2^n}  
    \mathbold 1_{\left[\frac{k-1}{2^n},\frac{k}{2^n}\right)}+
    n\mathbold 1_{[n,\infty]},
  \]
  显然 $d_n$ 是非负简单函数。直观上来看,$d_n$ 将区间 $[0,n]$
  等分为了 $n2^n$ 份,即将 $[0,1]$ 等分为了 $2^n$ 份。
  那么对于 $x\in [0,n)$,总存在唯一的 $k_n$ 使得 
  $(k_n-1)/2^n\leq x< k_n/2^n$,此时 $k_{n+1}=2k_n$
  或者 $k_{n+1}=2k_n-1$,所以
  \[
    d_{n+1}(x)  =\frac{k_{n+1}-1}{2^{n+1}}\geq \frac{k_n-1 }{2^n}
    =d_n(x),
  \]
  这表明 $d_n\leq d_{n+1}$。此外,不难看出 $\lim d_n(x)=x$。
  
  令 $f_n=d_n\circ f$,由于 $f_n$ 只有有限多个取值,所以
  $f_n$ 是非负简单函数。$d_n\leq d_{n+1}$ 表明
  $f_n\leq f_{n+1}$。且 $\lim f_n=\lim d_n\circ f=f$,
  所以 $f_n$ 就是一列递增的非负简单函数且 $f=\ulim f_n$。
  $f$ 有界表明在 $n$ 足够大的时候有 $0\leq f-f_n\leq 2^{-n}$,
  即 $f_n\to f$ 一致收敛。

  (2) 由 (1),设 $f=\ulim f_n$,$g=\ulim g_n$,其中 $(f_n),(g_n)$
  均为一列递增的简单函数,那么
  \[
    \int (af_n+bg_n)\d\mu=a\int f_n\d\mu+b\int g_n\d\mu,  
  \]
  令 $n\to\infty$,再根据单调收敛定理,就有
  \[
    \int (af+bg)\d\mu=a\int f\d\mu+b\int g\d\mu.
  \]

  (3) 根据 (2),有
  \[
    \int\biggl(\sum_{n=1}^m f_n\biggr)  \d\mu=
    \sum_{n=1}^m \int f_n\d\mu,
  \]
  令 $m\to\infty$,再根据单调收敛定理,就有
  \[
    \int\biggl(\sum_{n\in \mathbb{N}}f_n\biggr)\d\mu=\sum_{n\in \mathbb{N}}\int f_n\d\mu.
    \qedhere
  \]
\end{proof}

\begin{remark}
  \autoref{prop:property of integral of positive function} 和
  单调收敛定理 \ref{thm:monotone convergence thm} 给出了证明
  关于非负可测函数积分的命题的一种基本范式,即根据
  \autoref{prop:property of integral of positive function} 的 (1),
  假设一列非负简单函数逼近原函数,先证明命题对非负简单函数成立,
  这通常是非常容易的,再使用单调收敛定理证明命题对所有的
  非负可测函数成立。
\end{remark}

下面的推论在概率论中十分有用,其对应于随机变量的概率密度函数。其
证明是上述注释中技巧的一个典型运用。

\begin{corollary}\label{coro:density of measure}
  令 $g$ 是非负可测函数,对于 $A\in \mathcal{A}$,令
  \[
    \nu(A)=\int_A g\d\mu=\int g\mathbold 1_A \d\mu,
  \]
  那么 $\nu$ 是 $E$ 上的正测度,被称为密度 $g$ 相对于 $\mu$
  的测度,记为 $\nu=g\cdot \mu$。此外,对于非负可测函数
  $f$,有
  \[
    \int f\d\nu=\int fg\d\mu.
  \]
\end{corollary}
\begin{proof}
  显然 $\nu(\emptyset)=0$。任取一列不相交的 $A_n\in \mathcal{A}$,
  那么
  \[
    \nu\biggl(\bigcup_{n\in \mathbb{N}}A_n\biggr)  
    =\int \biggl(\sum_{n\in \mathbb{N}}g\mathbold 1_{A_n}\biggr)
    \d\mu=\sum_{n\in \mathbb{N}}\int g\mathbold 1_{A_n}\d\mu
    =\sum_{n\in \mathbb{N}}\mu(A_n),
  \]
  这就表明 $\mu$ 是 $E$ 上的正测度。

  对于任意示性函数 $\mathbold 1_A$,有
  \[
    \int \indicator{A}\d\nu=\nu(A)=\int \indicator{A}g\d\mu,
  \]
  进一步的,令 $f=\ulim f_n$,其中 $f_n$ 是非负简单函数,
  对于每个 $f_n$,根据积分的线性性,都有
  \[
    \int f_n \d\nu=\int f_ng\d\mu, 
  \]
  令 $n\to\infty$,根据单调收敛定理,就有 
  \[
    \int f\d\nu=\int fg\d\mu.\qedhere
  \]
\end{proof}
\begin{remark}
  在实际中,我们通常也会写作 $\nu(\d x)=g(x)\mu(\d x)$,或者
  $g=\d\nu/\d\mu$。
\end{remark}

在测度论中,命题通常在\emph{几乎处处}(almost everywhere)的意义下成立,也就是说,
对于不满足该命题的所有 $x\in E$ 的集合,这个集合的 $\mu$-测度为 $0$,
我们使用简写 $\alev{\mu}$ 来表示这个意思。也就是说,当我们
写到 $f=g,\ \alev{\mu}$ 的时候,我们表示的意思实际上是
\[
  \mu\bigl(\{x\in E\,|\, f(x)\neq g(x)\}\bigr)=0.
\]

\begin{proposition}
  令 $f$ 是非负可测函数。
  \begin{enumerate}
    \item 对于每个 $a\in(0,\infty)$,有
    \[
      \mu(\{x\in E\,|\, f(x)\geq a\})\leq \frac{1}{a}\int f\d\mu.
    \]
    \item 我们有
    \[
      \int f\d\mu<\infty\Rightarrow f<\infty,\ \text{$\mu$ a.e.}
    \]
    \item 我们有
    \[
      \int f\d\mu=0\Leftrightarrow f=0,\ \text{$\mu$ a.e.}
    \]
    \item 如果 $g$ 是非负可测函数,
    \[
      f=g,\ \text{$\mu$ a.e.}\Rightarrow \int f\d\mu=\int g\d\mu.
    \]
  \end{enumerate}
\end{proposition}
\begin{proof}
  (1) 令可测集
  $
    A=\{x\in E\,|\, f(x)\geq a\}
  $,
  那么 $f\geq a\indicator{A}$,所以
  \[
    \int f\d\mu\geq a\int \indicator{A}\d\mu  
    =a\mu(A).
  \]

  (2) 令可测集
  $
    A_n=\{x\in E\,|\, f(x)\geq n\}  
  $ 以及 $A_\infty=\{x\in E\,|\, f(x)=\infty\}$,
  那么 $A_{n+1}\subseteq A_n$ 且 $A_\infty=\bigcap_{n\in \mathbb{N}}A_n$。
  根据 (1),有
  \[
    \mu(A_1)\leq \int f\d\mu<\infty,  
  \]
  所以
  \[
    \mu(A_\infty)=\dlim[n\to\infty] \mu(A_n)\leq
    \dlim[n\to\infty]\frac{1}{n}\int f\d\mu=0, 
  \]
  所以 $\mu(A_\infty)=0$,即 $f<\infty,\ \alev{\mu}$。

  (3) 充分性由 \autoref{prop:property of integral of positive function} 的 (2) 保证。
  下证必要性。令可测集
  $
    A_n=\{x\in E\,|\, f(x)\geq 1/n\}  
  $ 以及
  $A_\infty=\{x\in E\,|\, f(x)\neq 0\}$,那么
  $A_n\subseteq A_{n+1}$ 且 $A_\infty=\bigcup_{n\in \mathbb{N}}A_n$。
  根据 (1),有
  \[
    \mu(A_\infty)=\ulim[n\to\infty]\mu(A_n)\leq
    \ulim[n\to\infty]n\int f\d\mu=0,
  \]
  所以 $\mu(A_\infty)=0$。

  (4) 记 $f\wedge g=\min(f,g)$ 及 $f\vee g=\max(f,g)$,那么
  $f=g,\ \alev{\mu}$ 表明  $f\vee g=f\wedge g,\ \alev{\mu}$。
  根据 (3),有
  \[
    \int f\vee g\d\mu=\int f\wedge g\d\mu+\int (f\vee g-f\wedge g)\d\mu
    =\int f\wedge g\d\mu,  
  \]
  又因为 $\int f\wedge g\d\mu\leq \int f\d\mu\leq \int f\vee g\d\mu$,
  对于 $g$ 类似,所以
  \[
    \int f\d\mu=\int g\d\mu.\qedhere  
  \]
\end{proof}

\begin{theorem}[Fatou 引理]
  令 $(f_n)_{n\in \mathbb{N}}$ 是一列非负可测函数,那么
  \[
    \int \liminf f_n\d\mu\leq \liminf \int f_n\d\mu.
  \]
\end{theorem}
\begin{proof}
  只需证明
  \[
    \int\lim_{n\to\infty}\inf_{k\geq n}f_k \d\mu\leq 
    \lim_{n\to\infty}\inf_{k\geq n}\int f_k\d\mu,  
  \]
  令 $g_n=\inf_{k\geq n}f_k$,那么 $g_n\leq g_{n+1}$,根据
  单调收敛定理,有
  \[
    \int \lim_{n\to\infty}g_n\d\mu=\ulim[n\to\infty]\int g_n\d\mu.
  \]
  对于任意 $n$ 和 $k\geq n$,有 $\int g_n\d\mu\leq \int f_k\d\mu$,所以
  \[
    \int \lim_{n\to\infty}g_n\d\mu=\ulim[n\to\infty]\int g_n\d\mu
    \leq\inf_{k\geq n} \int f_k\d\mu,
  \]
  令 $n\to\infty$,即可得到
  \[
    \int \liminf f_n\d\mu\leq \liminf \int f_n\d\mu.\qedhere
  \]
\end{proof}

\begin{proposition}\label{prop:change variable}
  令 $(F,\mathcal{B})$ 是可测空间,$\varphi:E\to F$ 是可测映射。令
  $\nu$ 是 $\mu$ 在 $\varphi$ 下的推前。那么,对于任意 $F$ 上的非负可测函数 $h$,
  我们有
  \[
    \int_Eh(\varphi(x))\mu(\d x)=\int_Fh(y)\nu(\d y).
  \]
\end{proposition}
\begin{proof}
  若 $h=\indicator{B}$ 是示性函数,那么
  \[
    \int_E h(\varphi(x))\mu(\d x)=
    \mu(\varphi^{-1}(B)) =\nu(B)=\int_F h(y)\nu(\d y).
  \]
  若 $h=\sum_{i=1}^n\alpha_i\indicator{B_i}$ 是非负简单函数,
  那么根据积分的线性性,结论也成立。
  若 $h$ 是一般的非负可测函数,设 $(h_n)_{n\in \mathbb{N}}$ 是一列递增的非负简单函数
  且 $h=\ulim h_n$,根据单调收敛定理,即可证明结论。
\end{proof}



\section{可积函数}

本节我们讨论可变号的可测函数。
如果 $f:E\to \mathbb{R}$ 是可测函数,记 $f$ 正部分 $f^+=\max(f,0)$,
负部分 $f^-=\max(-f,0)$,需要注意 $f^+$ 和 $f^-$ 此时都是非负可测函数
并且 $f=f^+-f^-$,$|f|=f^++f^-$。

\begin{definition}
  令 $f:E\to \mathbb{R}$ 是可测函数,如果
  \[
    \int|f|\d\mu<\infty,
  \]
  那么我们说 $f$ 相对于 $\mu$ \emph{可积}。在这种情况下,
  我们定义
  \[
    \int f\d\mu=\int f^+\d\mu-\int f^-\d\mu.
  \]
  如果 $A\in \mathcal{A}$,记
  \[
    \int_A f\d\mu=\int f\mathbold 1_A\d\mu.
  \]
\end{definition}

我们使用 $\mathcal{L}^1(E,\mathcal{A},\mu)$ 来表示所有可积函数
$f:E\to \mathbb{R}$ 构成的空间。$\mathcal{L}_+^1(E,\mathcal{A},\mu)$ 来表示
所有非负可积函数构成的空间。

\begin{proposition}[可积函数的性质]
  \mbox{}
  \begin{enumerate}
    \item 对于任意 $f\in \mathcal{L}^1(E,\mathcal{A},\mu)$,有
    $\bigl|\int f\d\mu\bigr|\leq \int |f|\d\mu$。
    \item $\mathcal{L}^1(E,\mathcal{A},\mu)$ 是 $\mathbb{R}$-向量空间。
    \item 如果 $f,g\in \mathcal{L}^1(E,\mathcal{A},\mu)$ 且
    $f\leq g$,那么 $\int f\d\mu\leq \int g\d\mu$。
    \item 如果 $f\in \mathcal{L}^1(E,\mathcal{A},\mu)$,$g:E\to [0,\infty]$
    是非负可测函数使得 $f=g,\ \alev{\mu}$,那么
    $g\in \mathcal{L}^1(E,\mathcal{A},\mu)$ 且 $\int f\d\mu=\int g\d\mu$。
    \item 令 $(F,\mathcal{B})$ 是可测空间,$\varphi:E\to F$ 是可测映射。令
    $\nu$ 是 $\mu$ 在 $\varphi$ 下的推前。那么,对于任意可测函数 $h:F\to \mathbb{R}$,
    $h$ 是 $\nu$-可积的当且仅当 $h\circ\varphi$ 是 $\mu$-可积的,并且我们有
    \[
      \int_Eh(\varphi(x))\mu(\d x)=\int_Fh(y)\nu(\d y).
    \]
  \end{enumerate}
\end{proposition}

\begin{theorem}[控制收敛定理]
  令 $(f_n)_{n\in \mathbb{N}}$ 是 $\mathcal{L}^1(E,\mathcal{A},\mu)$
  中的一列函数,如果:
  \begin{enumerate}
    \item 存在可测函数 $f:E\to \mathbb{R}$ 使得
    \[
      f_n(x)\to f(x),\quad\alev{\mu}  
    \]
    \item 存在非负可测函数 $g$ 使得 $\int g\d\mu<\infty$,并且
    对于每个 $n\in \mathbb{N}$,都有
    \[
      |f_n(x)|\leq g(x),\quad \alev{\mu}   
    \]
  \end{enumerate}
  那么 $f\in \mathcal{L}^1(E,\mathcal{A},\mu)$ 且我们有
  \[
    \lim_{n\to\infty}\int f_n\d\mu=\int f\d\mu,\quad
    \lim_{n\to\infty}\int |f_n-f|\d\mu=0.  
  \]
\end{theorem}
\begin{proof}
  我们首先将两个条件中的几乎处处去掉,证明结论成立。
  由于 $|f_n|\leq g$,所以 $|f|=\lim|f_n|\leq g$,所以
  $\int |f|\d\mu\leq \int g\d\mu<\infty$,故
  $f\in L^1(E,\mathcal{A},\mu)$。由于 $|f-f_n|\leq 2g$
  以及 $\lim |f-f_n|=0$,根据 Fatou 引理,有
  \[
    \liminf \int\bigl(2g-|f-f_n|\bigr)\d\mu\geq
    \int \bigl(2g-\limsup|f-f_n|\bigr)\d\mu 
    =\int 2g\d\mu,
  \]
  再根据积分的线性性,有
  \[
    \int 2g\d\mu-\limsup\int|f-f_n|\d\mu\geq \int 2g\d\mu,  
  \]
  这表明
  \[
    \limsup\int |f-f_n|\d\mu = 0,  
  \]
  所以 $\lim \int |f-f_n|\d\mu$ 存在且为 $0$。最后,我们有
  \[
    \left|\int f\d\mu-\int f_n\d\mu\right|\leq \int |f-f_n|\d\mu\to 0,
  \]
  所以 $\int f\d\mu=\lim\int f_n\d\mu$。

  现在我们证明几乎处处的情况。记
  \[
    A=\big\{x\in E\,|\, f_n(x)\to f(x),|f_n(x)|\leq g(x)\big\}  ,
  \]
  那么 $A$ 可测且条件表明 $\mu(A^c)=0$。定义
  \[
    \tilde f_n(x)=\indicator{A}(x)f_n(x),\quad \tilde{f}(x)=\indicator{A}(x)f(x),  
  \]
  于是在几乎处处的意义下有 $f_n=\tilde{f}_n$ 以及 $f=\tilde{f}$,
  所以 $\int f_n\d\mu=\int\tilde{f}_n\d\mu$,
  $\int f\d\mu=\int\tilde f\d\mu$ 以及 $\int|f-f_n|\d\mu=\int|\tilde f-\tilde f_n|\d\mu$。
  对 $\tilde{f}_n$ 和 $\tilde{f}$ 应用上面的结论即可。
\end{proof}

\section{含参积分}

我们考虑带有一个参数的函数的积分。
设 $(U,d)$ 是一个度量空间,参数位于这个空间中。

\begin{theorem}[含参积分的连续性]
  令 $f:U\times E\to \mathbb{R}$ (or $\mathbb{C}$),$u_0\in U$。假设:
  \begin{enumerate}
    \item 对于每个 $u\in U$,函数 $x\mapsto f(u,x)$ 可测;
    \item $\alev{\mu(\d x)}$,函数 $u\mapsto f(u,x)$ 在 $u_0$ 处连续;
    \item 存在函数 $g\in \mathcal{L}_+^1(E,\mathcal{A},\mu)$
    使得任取 $u\in U$ 有
    \[
      |f(u,x)|\leq g(x)\quad \alev{\mu(\d x)}  
    \]
  \end{enumerate}
  那么函数 $F(u)=\int f(u,x)\mu(\d x)$ 是良好定义的且在 $u_0$
  处连续。
\end{theorem}
\begin{proof}
  条件 (1) 保证了 $x\mapsto f(u,x)$ 是可测的,所以 $F(u)$ 是良好定义的。
  设 $(u_n)_{n\in \mathbb{N}}$ 是任意趋于 $u_0$ 的点列,那么 
  对于几乎处处的 $x$,$u\mapsto f(u,x)$ 连续表明 $f(u_n,x)\to f(u,x)$,
  再根据条件 (3) 和控制收敛定理,就有
  \[
    F(u_0)=\int \lim_{n\to\infty} f(u_n,x)\mu(\d x)
    =  \lim_{n\to\infty}\int f(u_n,x)\mu(\d x)=\lim_{n\to\infty}F(u_n),
  \]
  这就表明 $F(u)$ 在 $u_0$ 处连续。
\end{proof}

\begin{example}
  \mbox{}
  \begin{enumerate}
    \item \emph{Fourier 变换}。令 $\lambda$ 表示 $\mathbb{R}$ 上的 Lebesgue 测度。
    如果 $\varphi\in \mathcal{L}^1(\mathbb{R},\mathcal{B}(\mathbb{R}),\lambda)$,
    定义函数 $\hat\varphi:\mathbb{R}\to \mathbb{C}$ 为:
    \[
      \hat\varphi(u)=\int e^{iux}\varphi(x)\lambda(\d x),  
    \]
    根据上面的定理,$\hat\varphi$ 是连续函数。函数 $\hat\varphi$ 被称为 $\varphi$
    的 \emph{Fourier 变换}。在概率论中,我们经常会考虑有限测度的 Fourier 变换。
    如果 $\mu$ 是 $\mathbb{R}$ 上的有限测度,定义 $\mu$ 的 Fourier 变换为
    \[
      \hat{\mu}(u)=\int e^{iux}\mu(\d x)\quad u\in \mathbb{R}.  
    \]
    此时 $|e^{iux}|\leq \indicator{\mathbb{R}}$ 是可积函数,所以
    $\hat\mu$ 是连续函数。
    \item \emph{卷积}。令 $\varphi\in \mathcal{L}^1(\mathbb{R},\mathcal{B}(\mathbb{R}),\lambda)$,
    $h: \mathbb{R}\to \mathbb{R}$ 是有界连续函数,那么定义函数 $h*\varphi$ 为
    \[
      h*\varphi(u)=\int h(u-x)\varphi(x)\lambda(\d x),  
    \]
    这是一个连续函数。
  \end{enumerate}
\end{example}

下面我们叙述含参积分的可微性。令 $I\subseteq \mathbb{R}$ 是开区间。

\begin{theorem}
  考虑函数 $f:I\times E\to \mathbb{R}$,$u_0\in I$,假设
  \begin{enumerate}
    \item 对于每个 $u\in I$,函数 $x\mapsto f(u,x)$ 是可积函数;
    \item $\alev{\mu(\d x)}$,函数 $u\mapsto f(u,x)$ 在 $u_0$ 处可导,
    导数记为
    \[
      \frac{\partial f}{\partial u}(u_0,x);  
    \]
    \item 存在函数 $g\in \mathcal{L}_+^1(E,\mathcal{A},\mu)$ 使得
    对于任意 $u\in I$ 有
    \[
      |f(u,x)-f(u_0,x)|\leq g(x)|u-u_0|,\quad \alev{\mu(\d x)}  
    \]
  \end{enumerate}
  那么函数 $F(u)=\int f(u,x)\mu(\d x)$ 在 $u_0$ 处可导,并且
  \[
    F'(u_0)=\int \frac{\partial f}{\partial u}(u_0,x)\mu(\d x).
  \]
\end{theorem}
\begin{proof}
  
\end{proof}


\begin{exercise}
  计算
  \[
    \lim_{n\to\infty}\int_0^n\left(1+\frac{x}{n}\right)^ne^{-2x}\d x.  
  \]
  令 $\alpha\in \mathbb{R}$,证明极限
  \[
    \lim_{n\to\infty}\int_0^n\left(1-\frac{x}{n}\right)^n x^{\alpha-1}\d x  
  \]
  在 $[0,\infty]$ 上存在,且极限值有限当且仅当 $\alpha>0$。
\end{exercise}
\begin{proof}
  由于
  \[
    \int_0^n\left(1+\frac{x}{n}\right)^ne^{-2x}\d x=\int_0^1
    n(1+x)^ne^{-2nx}\d x,
  \]
  对于任意 $x\in (0,1)$,有
  \[
    \lim_{n\to\infty}n(1+x)^n e^{-2nx}=\lim_{n\to\infty}n\bigl((1+x)e^{-2x}\bigr) ^n
    =0,
  \]
\end{proof}



\chapter{测度的构造}

\section{外测度}

\begin{definition}
  令 $E$ 是集合,映射 $\mu^*:\mathcal{P}(E)\to [0,\infty]$ 如果满足:
  \begin{enumerate}
    \item  $\mu^*(\emptyset)=0$;
    \item $A\subseteq B\Rightarrow \mu^*(A)\leq\mu^*(B)$;
    \item ($\sigma$-次可加性) 对于 $\mathcal{P}(E)$ 中的一列子集 $(A_k)_{k\in \mathbb{N}}$,
    有
    \[
      \mu^*\biggl(\bigcup_{k\in \mathbb{N}}A_k \biggr)  \leq
      \sum_{k\in \mathbb{N}}\mu^*(A_k).
    \]
  \end{enumerate}
  那么我们说 $\mu^*$ 是一个\emph{外测度}。
\end{definition}

外测度的要求不如测度严格,首先 $\sigma$-可加性被替换为 $\sigma$-次可加性,
其次外测度是在幂集 $\mathcal{P}(E)$ 上定义的,而测度只能在
$\sigma$-域上定义。

我们本节的目标是从外测度 $\mu^*$ 开始,在某个 $\sigma$-域 $\mathcal{M}(\mu^*)$
上构造一个测度。从现在开始,我们固定一个外测度 $\mu^*$。

\begin{definition}
  对于 $E$ 的子集 $B$,如果任取 $A\subseteq E$,都有
  \[
    \mu^*(A)=\mu^*(A\cap B)+\mu^*(A\cap B^c),  
  \]
  那么我们说 $B$ 是\emph{$\mathbold{\mu^*}$-可测的}。用
  $\mathcal{M}(\mu^*)$ 表示所有 $\mu^*$-可测的子集构成的子集族。
\end{definition}

\begin{remark}
  根据 $\sigma$-次可加性,总是有
  \[
    \mu^*(A)\leq \mu^*(A\cap B)+\mu^*(A\cap B^c),
  \]
  所以要验证子集 $B$ 是 $\mu^*$-可测的,只需要说明反向的不等式即可。
\end{remark}

\begin{theorem}\label{thm:outer measure}
  \mbox{}
  \begin{enumerate}
    \item $\mathcal{M}(\mu^*)$ 是 $\sigma$-域,并且其包含
    所有的满足 $\mu^*(B)=0$ 的子集 $B\subseteq E$。
    \item $\mu^*$ 在 $\mathcal{M}(\mu^*)$ 上的限制是一个测度。
  \end{enumerate}
\end{theorem}


\section{Lebesgue 测度}

对于任意子集 $A\subseteq \mathbb{R}$,定义
\[
  \lambda^*(A)=\inf\bigg\{\sum_{i\in \mathbb{N}}(b_i-a_i)\,|\, A\subseteq \bigcup_{i\in \mathbb{N}}(a_i,b_i)\bigg\}  .
\]
注意这个下确界的取值范围为 $[0,\infty]$:如果 $A$ 无界,那么
将会得到 $\infty$。

\begin{theorem}
  \mbox{}
  \begin{enumerate}
    \item $\lambda^*$ 是 $\mathbb{R}$ 上的一个外测度。
    \item $\sigma$-域 $\mathcal{M}(\lambda^*)$ 包含 $\mathcal{B}(\mathbb{R})$。
    \item 对于任意实数 $a\leq b$,$\lambda^*([a,b])=\lambda^*((a,b))=b-a$。
  \end{enumerate}
\end{theorem}
\begin{proof}
  (1) 显然 $\lambda^*(\emptyset)=0$ 并且 $A\subseteq B$ 表明 $\lambda^*(A)\leq\lambda^*(B)$。
  下面证明 $\sigma$-次可加性。任取 $\mathbb{R}$ 的一列子集 $(A_n)_{n\in \mathbb{N}}$,
  不妨假设每个 $\lambda^*(A_n)<\infty$。任取 $\varepsilon>0$,对于每个 $A_n$,都存在
  一列开区间 $\bigl(a_i^{(n)},b_i^{(n)}\bigr)$ 使得
  \[
    \lambda^*(A_n)\leq \sum_{i\in \mathbb{N}}\bigl(b_i^{(n)}-a_i^{(n)}\bigr)
    <\lambda^*(A_n)+\frac{\varepsilon}{2^n},
  \]
  注意到所有的开区间 $\bigl(a_i^{(n)},b_{i}^{(n)}\bigr)\ (i,n\in \mathbb{N})$
  构成了 $\bigcup_{n\in \mathbb{N}}A_n$ 的一个可数开覆盖,所以
  \[
    \lambda^*\biggl(\bigcup_{n\in \mathbb{N}}A_n\biggr)\leq
    \sum_{n\in \mathbb{N}}\sum_{i\in \mathbb{N}}\bigl(b_i^{(n)}-a_i^{(n)}\bigr)
    \leq \sum_{n\in \mathbb{N}}\lambda^*(A_n)+\varepsilon,
  \]
  由于 $\varepsilon$ 的任意性,所以 $\lambda^*$ 满足 $\sigma$-次可加性。

  (2) 
\end{proof}


\chapter{$L^p$ 空间}

\section{定义与 H\"older 不等式}

在本章中,我们考虑测度空间 $(E,\mathcal{A},\mu)$。对于实数 $p\geq 1$,
我们令 $\mathcal{L}^p(E,\mathcal{A},\mu)$ 表示所有满足
\[
  \int |f|^p\d\mu<\infty  
\]
的可测函数 $f:E\to \mathbb{R}$ 构成的空间。
此外,我们引入 $\mathcal{L}^\infty(E,\mathcal{A},\mu)$ 表示所有
几乎处处有界的可测函数 $f:E\to \mathbb{R}$ 构成的空间,即存在
常数 $C\in \mathbb{R}_+$ 使得
\[
  |f|\leq C,\ \alev{\mu}  
\]

对于每个 $p\in [1,\infty]$,我们可以定义 $\mathcal{L}^p$ 上
的一个等价关系:
\[
  f\sim g\Leftrightarrow f=g,\ \alev{\mu}  
\]
于是我们可以考虑商空间
\[
  L^p(E,\mathcal{A},\mu)=\mathcal{L}^p(E,\mathcal{A},\mu)/\sim.  
\]
也就是说,我们只考虑几乎处处相等意义上的函数,如果两个函数几乎处处
相等,那么我们认为这是同一个函数。

在没有歧义的情况下,我们使用 $L^p(\mu)$ 或者 $L^p$ 表示
$L^p(E,\mathcal{A},\mu)$。注意到 $L^1$ 就是所有可积函数
构成的空间。

对于可测函数 $f:E\to \mathbb{R}$ 和 $p\in [1,\infty)$,我们定义
\[
  \norm{f}_p=\left(\int |f|^p\d\mu\right)^{1/p}  .
\]
约定 $\infty^{1/p}=\infty$。定义
\[
  \norm{f}_\infty=\inf\{C\in[0,\infty]\,|\, |f|\leq C,\ \alev{\mu}\}  .
\]
如果 $f,g$ 是两个几乎处处相等的可测函数,那么有 $\norm{f}_p=\norm{g}_p$,
所以我们可以针对 $f\in L^p(E,\mathcal{A},\mu)$ 良好的定义 $\norm{f}_p$。

对于 $p,q\in [1,\infty]$,我们说 $p$ 和 $q$ 是\emph{共轭指数},如果
\[
  \frac{1}{p}+\frac{1}{q}=1.  
\]
特别地,$1$ 和 $\infty$ 是共轭的。

\begin{theorem}[H\"older 不等式]
  令 $p,q$ 是共轭指数,$f,g$ 是两个 $E\to \mathbb{R}$ 的可测函数,那么
  \[
    \int|fg|\d\mu\leq \norm{f}_p\norm{g}_q.  
  \]
  特别地,如果 $f\in L^p$ 以及 $g\in L^q$,那么
  $fg\in L^1$。
\end{theorem}
\begin{proof}
  若 $\norm{f}_p=0$,那么 $|f|=0\,\ \alev{\mu}$,这表明 
  $\int |fg|\d\mu=0$,结论显然成立,所以我们不妨假设
  $\norm{f}_p>0$ 以及 $\norm{g}_p>0$。进一步的,我们还可以假设
  $f\in L^p$ 以及 $g\in L^q$,否则右边为 $\infty$ 显然成立。

  先假设 $p=1$ 和 $q=\infty$,那么
  \[
    \int|fg|\d\mu\leq \norm{g}_{\infty}\int |f|\d\mu
    =  \norm{f}_1\norm{g}_\infty.
  \]
  下面假设 $1<p,q<\infty$。

  设 $\alpha\in (0,1)$,那么对于 $x\in [0,\infty)$ 有不等式
  \[
    x^\alpha-\alpha x\leq 1-\alpha,  
  \]
  取 $x=u/v\ (u\geq 0,v>0)$,我们有
  \[
    u^\alpha v^{1-\alpha}\leq \alpha u+(1-\alpha) v,  
  \]
  该不等式在 $v=0$ 时也成立。取 $\alpha=1/p$,$1-\alpha=1/q$,
  以及
  \[
    u=\frac{|f|^p}{\norm{f}_p^p} ,\quad v=\frac{|g|^q}{\norm{g}_q^q},
  \]
  那么
  \[
  \frac{|fg|}{\norm{f}_p^p\norm{g}_q^q}\leq
  \frac{1}{p}\frac{|f|^p}{\norm{f}_p^p}+
  \frac{1}{q}\frac{|g|^q}{\norm{g}_q^q},
  \]
  两边积分,即得
  \[
    \int|fg|\d\mu\leq   \norm{f}_p^p\norm{g}_q^q.\qedhere
  \]
\end{proof}

\begin{corollary}[Cauchy-Schwarz 不等式]
  取 $p=q=2$,即得
  \[
    \int |fg|\d\mu\leq \left(\int |f|^2 \d\mu\right)^{1/2}  
    \left(\int |g|^2 \d\mu\right)^{1/2}  .
  \]
\end{corollary}
 
\begin{corollary}
  假设 $\mu$ 是有限测度,$p,q$ 是共轭指数且 $p>1$,那么对于
  任意可测函数 $f:E\to \mathbb{R}$,有
  \[
    \norm{f}_1\leq \mu(E)^{1/q}\norm{f}_p,  
  \]
  因此,对于任意 $p\in (1,\infty]$,有 $L^p\subseteq L^1$。
  更一般地,对于任意 $1\leq r< r'<\infty$,有
  \[
    \norm{f}_r\leq \mu(E)^{\frac{1}{r}-\frac{1}{r'}}  
    \norm{f}_{r'},
  \]
  因此,对于任意 $1\leq p<q\leq \infty$,有 $L^q\subseteq L^p$,
  特别地,当 $\mu$ 是概率测度的时候,还有 $\norm{f}_p\leq\norm{f}_q$。
\end{corollary}
\begin{proof}
  取 $g=\indicator{E}$,即得
  \[
    \int|f|\d\mu=\int|f \indicator{E}|\d\mu\leq
    \norm{f}_p \norm{\indicator{E}}_q=\mu(E)^{1/q}\norm{f}_p. 
  \]
  用 $f^r$ 替代 $f$,取 $p=r'/r$,$1/q=1-r/r'$,那么
  \[
    \norm{f}_r\leq \mu(E)^{1/r-1/r'}\norm{f^r}_{r'/r}^{1/r}  
    =\mu(E)^{1/r-1/r'}\norm{f}_{r'}.\qedhere
  \]
\end{proof}

\section{Banach 空间 $L^p(E,\mathcal{A},\mu)$}



\chapter{积测度}

\section{积 $\sigma$-域}

令 $(E,\mathcal{A})$ 和 $(F,\mathcal{B})$ 是两个可测空间。
回顾第一章,我们定义 $E\times F$ 上的乘积 $\sigma$-域
\[
  \mathcal{A}\otimes \mathcal{B}=\sigma(A\times B\,|\, A\in \mathcal{A},B\in \mathcal{B}).  
\]
不难验证 $\mathcal{A}\otimes \mathcal{B}$ 是使得两个投影映射
$\pi_1:E\times F\to E$ 和 $\pi_2:E\times F\to F$ 都可测的最小的
$\sigma$-域。

如果 $C\subseteq E\times F$,$x\in E$,记
\[
  C_x=\{y\in F\,|\, (x,y)\in C\}\subseteq F,  
\]
如果 $y\in F$,记
\[
  C^y=\{x\in E\,|\, (x,y)\in C\}  \subseteq E.
\]
如果 $f$ 是 $E\times F$ 上的函数,$x\in E$,我们记 $f_x$ 表示
$F$ 上的函数 $f_x(y)=f(x,y)$。类似地,如果 $y\in F$,我们记
$f^y(x)=f(x,y)$ 表示 $E$ 上的函数。

\begin{proposition}
  \mbox{}
  \begin{enumerate}
    \item 令 $C\in \mathcal{A}\otimes \mathcal{B}$,那么
    对于任意 $x\in E$,$C_x\in \mathcal{B}$,
    对于任意 $y\in F$,$C^y\in \mathcal{A}$。
    \item 令 $(G,\mathcal{G})$ 是可测空间,$f:E\times F\to G$
    是可测函数,那么对于任意 $x\in E$,$f_x:F\to G$ 是可测的,
    对于任意 $y\in F$,$f^y:E\to G$ 是可测的。
  \end{enumerate}
\end{proposition}
\begin{proof}
  (1) 对于 $x\in E$,令
  \[
    \mathcal{C}  =\{C\in \mathcal{A}\otimes \mathcal{B}\,|\, C_x\in \mathcal{B}\},
  \]
  那么不难验证 $\mathcal{C}$ 是一个 $\sigma$-域且包含所有的可测矩形,
  于是 $\mathcal{C}=\mathcal{A}\otimes \mathcal{B}$,
  即表明对于任意 $C\in \mathcal{A}\otimes \mathcal{B}$ 都有 $C_x\in \mathcal{B}$。
  $C^y\in \mathcal{A}$ 同理。

  (2) 对于 $x\in E$,任取 $D\in \mathcal{G}$,有
  \[
    f_x^{-1}(D)=\bigl(f^{-1}(D)\bigr)_x \in \mathcal{B}.\qedhere
  \]
\end{proof}


\section{积测度}

\begin{theorem}\label{thm:product measure}
  令 $\mu$ 和 $\nu$ 分别是 $(E,\mathcal{A})$ 和 $(F,\mathcal{B})$
  上的 $\sigma$-有限测度,那么
  \begin{enumerate}
    \item 存在唯一的 $(E\times F,\mathcal{A}\otimes \mathcal{B})$
    上的测度 $m$,使得对于每个 $A\in \mathcal{A},B\in \mathcal{B}$,都有
    \[
      m(A\times B)=\mu(A)\nu(B),  
    \]
    约定 $0\times \infty=0$。测度 $m$ 也是 $\sigma$-有限的,
    记为 $m=\mu\otimes\nu$。
    \item 对于每个 $C\in \mathcal{A}\otimes \mathcal{B}$,函数
    $x\mapsto \nu(C_x)$ 是 $\mathcal{A}$-可测的,
    $y\mapsto \mu(C^y)$ 是 $\mathcal{B}$-可测的,并且我们有
    \[
      \mu\otimes \nu(C)=\int_E \nu(C_x)\mu(\d x)=\int_F\mu(C^y)\nu(\d y).  
    \]
  \end{enumerate}
\end{theorem}
\begin{proof}
  我们首先说明这样的测度 $m$ 一定是唯一的。我们使用 \autoref{coro:uniqueness of measure}
  说明唯一性。若测度 $m'$ 也满足性质 (1)。首先所有可测矩形对有限交封闭且生成 $\mathcal{A}\otimes \mathcal{B}$,
  并且 $m$ 和 $m'$ 在可测矩形上取值相同。$\mu$ 是 $\sigma$-有限的表明存在
  递增的可测子集 $(A_n)_{n\in \mathbb{N}}$ 使得 $\bigcup_{n\in \mathbb{N}}A_n=E$
  并且 $\mu(A_n)$ 有限。同理存在递增的可测子集 $(B_n)_{n\in \mathbb{N}}$ 
  使得 $\bigcup_{n\in \mathbb{N}}B_n=F$ 并且 $\nu(B_n)$ 有限。
  令 $G_n=A_n\times B_n$,那么 $G_n\subseteq G_{n+1}$ 并且
  $E\times F=\bigcup_{n\in \mathbb{N}}G_n$,此时
  \[
    m'(G_n)=\mu(A_n)\nu(B_n)=m(G_n)<\infty,  
  \] 
  所以 \autoref{coro:uniqueness of measure} 表明 $m=m'$。

  然后我们说明存在性。对于 $C\in \mathcal{A}\otimes \mathcal{B}$,定义
  \begin{equation}\label{eq:def of product measure}
    m(C) =\int_E\nu(C_x)\mu(\d x).
  \end{equation}
  对于任意 $x\in E$,有 $C_x\in \mathcal{B}$,所以 $\nu(C_x)$
  是良好定义的。下面我们证明 $x\mapsto \nu(C_x)$ 是可测函数。
  \begin{enumerate}[label=(\arabic*)]
    \item 首先假设 $\nu$ 是有限测度。
    令 $\mathcal{G}=\{C\in \mathcal{A}\otimes \mathcal{B}\,|\, \text{$x\mapsto \nu(C_x)$ 可测}\}$。
    那么对于可测矩形 $A\times B\in \mathcal{A}\otimes \mathcal{B}$,有
    $\nu((A\times B)_x)=\indicator{A}(x)\nu(B)$,此时 $x\mapsto \indicator{A}(x)\nu(B)$
    当然是可测函数。故 $\mathcal{G}$ 包含所有的可测矩形。
    其次,我们证明 $\mathcal{G}$ 是一个单调类。如果 $C,C'\in \mathcal{G}$
    且 $C\subseteq  C'$,那么利用 $\nu$ 的有限性,就有
    \[
      \nu\bigl((C'\smallsetminus C)_x\bigr)  
      =\nu(C'_x \smallsetminus C_x)=\nu(C_x')-\nu(C_x),
    \]
    所以 $x\mapsto \nu\bigl((C'\smallsetminus C)_x\bigr)$ 是可测函数,
    即 $C' \smallsetminus C\in \mathcal{G}$。如果 $(C_n)_{n\in \mathbb{N}}$
    是 $\mathcal{G}$ 中的一个递增序列,那么
    \[
      \nu\biggl(\biggl(\bigcup_{n\in \mathbb{N}}C_n\biggr)_x\biggr)  
      =\nu\biggl(\bigcup_{n\in \mathbb{N}}(C_n)_x\biggr)
      =\ulim[n\to\infty]\nu((C_n)_x),
    \] 
    而 $x\mapsto \ulim \nu((C_n)_x)$ 是可测函数,所以 $\bigcup_{n\in \mathbb{N}}C_n\in \mathcal{G}$。
    这就表明 $\mathcal{G}$ 是单调类。由于 $\mathcal{G}$ 包含可测矩形,可测矩形对有限交封闭,
    根据单调类定理,所以 $\mathcal{G}$ 是 $\sigma$-域,所以 $\mathcal{G}=\mathcal{A}\otimes \mathcal{B}$,
    这表明对于任意 $C\in \mathcal{A}\otimes \mathcal{B}$,$x\mapsto \nu(C_x)$ 都是可测函数。
    \item 然后假设 $\nu$ 是 $\sigma$-有限测度。此时存在 $\mathcal{B}$ 的一列递增
    子集 $(B_n)_{n\in \mathbb{N}}$ 使得 $F=\bigcup_{n\in \mathbb{N}}B_n$
    以及 $\nu(B_n)<\infty$。令 $\nu_n$ 表示测度 $\nu$ 在 $B_n$ 上的限制,那么
    根据上面的叙述,任取 $C\in \mathcal{A}\otimes \mathcal{B}$,函数 $x\mapsto \nu_n(C_x)$ 是可测函数。
    注意到
    \[
      \nu(C_x)=\nu\biggl(\bigcup_{n\in \mathbb{N}}(C_x\cap B_n)\biggr)  
      =\ulim[n\to\infty] \nu_n(C_x),
    \]
    所以 $x\mapsto \ulim \nu(C_x)$ 是可测函数。
  \end{enumerate}

  于是我们证明了 $x\mapsto \nu(C_x)$ 是可测函数,这表明定义 \eqref{eq:def of product measure}
  式是有意义的。下面我们验证 $m$ 满足测度的条件。显然 $m(\emptyset)=0$。
  任取 $(C_n)_{n\in \mathbb{N}}$ 是 $\mathcal{A}\otimes \mathcal{B}$
  中的一列不相交的子集,那么
  \begin{align*}
    m\biggl(\bigcup_{n\in \mathbb{N}}C_n\biggr)&=\int_E
    \nu\biggl(\bigcup_{n\in \mathbb{N}}(C_n)_x\biggr)\mu(\d x)
    =\int_E\sum_{n\in \mathbb{N}}\nu((C_n)_x)\mu(\d x)\\
    &=\sum_{n\in \mathbb{N}}\int_E\nu((C_n)_x)\mu(\d x)
    =\sum_{n\in \mathbb{N}}m(C_n),
  \end{align*}
  其中第三个等号利用了单调收敛定理。这就表明 $m$ 确实是一个测度。

  对于可测矩形 $A\times B\in \mathcal{A}\otimes \mathcal{B}$,有
  \[
    m(A\times B)=\int_E\nu((A\times B)_x)\mu(\d x)
    =\int_E\nu(B)\indicator{A}(x)\mu(\d x)  
    =\mu(A)\nu(B),
  \]
  所以这样的 $m$ 是唯一的。对于 $C\in \mathcal{A}\otimes \mathcal{B}$,定义
  \begin{equation*}
    m'(C) =\int_F\mu(C^y)\nu(\d y),
  \end{equation*}
  重复上面的过程,可以证明 $m'$ 满足和 $m$ 相同的性质,所以 $m=m'$。
\end{proof}

\section{Fubini 定理}

考虑可测空间 $(E,\mathcal{A})$ 和 $(F,\mathcal{B})$。

\begin{theorem}[Fubini-Tonelli]\label{thm:Fubini-Tonelli}
  令 $\mu$ 和 $\nu$ 分别是 $(E,\mathcal{A})$ 和 $(F,\mathcal{B})$ 上的两个
  $\sigma$-有限的测度。令 $f:E\times F\to [0,\infty]$ 是可测函数。
  \begin{enumerate}
    \item 函数
    \[
      E\ni x\mapsto \int_F f(x,y)\nu(\d y),\quad 
      F\ni y\mapsto \int_E f(x,y)\mu(\d x)  
    \]
    是值在 $[0,\infty]$ 中的可测函数。
    \item 我们有
    \[
      \int_{E\times F}f\d\mu\otimes\nu 
      =\int_E\left(\int_F f(x,y)\nu(\d y)\right)\mu(\d x)=\int_F
      \left(\int_E f(x,y)\mu(\d x)\right)\nu(\d y).
    \]
  \end{enumerate}
\end{theorem}
\begin{proof}
  (1) 设 $f=\ulim f_n$,$(f_n)_{n\in \mathbb{N}}$ 是一列递增的非负简单函数,
  那么根据单调收敛定理,有
  \[
    \int_F f(x,y)\nu(\d y)=\ulim[n\to\infty]\int_Ff_n(x,y)\nu(\d y)  ,
  \]
  所以我们只需要说明对于任意非负简单函数 $h$,$x\mapsto \int_F h(x,y)\nu(\d y)$
  可测即可。对于示性函数 $\indicator{C}$,有
  $\int_F \indicator{C}(x,y) \nu(\d y)=\nu(C_x)$,\autoref{thm:product measure}
  表明 $x\mapsto \nu(C_x)$ 是可测的,再根据线性性,这就说明了
  $x\mapsto \int_F h(x,y)\nu(\d y)$ 可测。
  对于 $x\mapsto\int_E f(x,y)\mu(\d x)$ 同理。

  (2) 设 $f=\ulim f_n$,$(f_n)_{n\in \mathbb{N}}$ 是一列递增的非负简单函数,
  那么根据单调收敛定理,有
  \[
    \int_{E\times F} f\d\mu\otimes \nu
    =\ulim[n\to\infty]\int_{E\times F}f_n(x,y)\nu(\d y),  
  \]
  所以只需要证明结论对于非负简单函数成立即可。根据线性性,只需要证明结论对
  示性函数成立即可。任取示性函数 $\indicator{C}$,根据 \autoref{thm:product measure},有
  \[
    \int_{E\times F} \indicator{C}\d\mu\otimes\nu=
    \mu\otimes \nu(C)=\int_E\nu(C_x)\mu(\d x)  
    =\int_E\left(\int_F \indicator{C}(x,y)\nu(\d y)\right)\mu(\d x),
  \]
  另一个等式同理。
\end{proof}

\autoref{thm:Fubini-Tonelli} 也可以推广到任意符号的版本。

\begin{theorem}[Fubini-Lebesgue]
  令 $f\in \mathcal{L}^1(E\times F,\mathcal{A}\otimes \mathcal{B},\mu\otimes\nu)$,那么
  \begin{enumerate}
    \item $\alev{\mu(\d x)}$,函数 $y\mapsto f(x,y)$ 属于 $\mathcal{L}^1(F,\mathcal{B},\nu)$。
    $\alev{\nu(\d y)}$,函数 $x\mapsto f(x,y)$ 属于 $\mathcal{L}^1(E,\mathcal{A},\mu)$。
    \item 函数 $x\mapsto \int_F f(x,y)\nu(\d y)$ 属于 $\mathcal{L}^1(E,\mathcal{A},\mu)$。
    函数 $y\mapsto \int_E f(x,y)\mu(\d x)$ 属于 $\mathcal{L}^1(F,\mathcal{B},\nu)$。
    \item 我们有
    \[
      \int_{E\times F}f\d\mu\otimes\nu 
      =\int_E\left(\int_F f(x,y)\nu(\d y)\right)\mu(\d x)=\int_F
      \left(\int_E f(x,y)\mu(\d x)\right)\nu(\d y).
    \]
  \end{enumerate}
\end{theorem}





\part{概率论}

\chapter{概率论基础}

\section{一般定义}

\subsection{概率空间}

令 $(\Omega,\mathcal{A})$ 是可测空间,$\mathbb{P}$ 是 $(\Omega,\mathcal{A})$
上的概率测度,我们说 $(\Omega,\mathcal{A},\mathbb{P})$ 是\emph{概率空间}。

因此,概率空间是测度空间的一个特例。然而,概率论的观点与测度论有很大不同。在概率论中,
我们的目标是一个“随机实验”的数学模型:
\begin{itemize}[nosep]
  \item $\Omega$ 表示实验的所有可能的结果的集合。
  \item $\mathcal{A}$ 是所有“事件”的集合。这里的事件指的是 $\Omega$ 的一个子集,其概率
  可以被计算(也就是可测集)。我们应当把事件 $A$ 视为满足某一属性的所有 $\omega\in\Omega$ 
  构成的子集。
  \item 对于每个 $A\in \mathcal{A}$,$\mathbb{P}(A)$ 表示事件 $A$ 发生的概率。
\end{itemize}

当然,一个自然的疑问是,为什么需要考虑事件域 $\mathcal{A}$?换句话说,为什么不能对
$\Omega$ 的任意子集都计算一个概率?原因在于,一般不可能在 $\Omega$ 的幂集 $\mathcal{P}(\Omega)$
上定义我们感兴趣的概率测度(除开 $\Omega$ 是可数集这一简单情况)。例如,取 $\Omega=[0,1]$,
配备 Borel $\sigma$-域和 Lebesgue 测度,但是,可以证明不可能将 Lebesgue 测度扩展到
$[0,1]$ 的任意子集上使得其仍然满足测度的定义。

\begin{example}\label{exa:dice model}
  一些常见的概率模型。
  \begin{enumerate}
    \item 考虑扔两次骰子这一实验,那么
    \[
      \Omega=\{1,2,\dots,6\}^2,\quad \mathcal{A}=\mathcal{P}(\Omega),\quad
      \mathbb{P}(A)=\frac{\card(A)}{36}.
    \]
    这里概率 $\mathbb{P}$ 的选取意味着让所有结果都有相同的概率。更一般地,如果 $\Omega$
    是有限集,$\mathcal{A}=\mathcal{P}(\Omega)$,概率测度 $\mathbb{P}(\{\omega\})=1/\card(\Omega)$
    被称为 $\Omega$ 上的\emph{均匀概率测度}。
    \item 现在我们考虑实验:扔骰子,直到出现 $6$ 为止。由于得到 $6$ 所需的投掷次数是无界的
    (即使你扔了 $1000$ 次骰子,仍有可能没有得到 $6$),所以 $\Omega$ 的正确选择是想象
    我们扔了无限次骰子:
    \[
      \Omega=\{1,2,\dots,6\}^{\mathbb{N}}.
    \]
    $\Omega$ 上的 $\sigma$-域 $\mathcal{A}$ 被定义为包含形如
    \[
      \{\omega\in\Omega\,|\, \omega_1=i_1,\dots,\omega_n=i_n\}
    \]
  \end{enumerate}
\end{example}

与测度论类似,零测集也会出现在概率论的很多叙述中,如果某个
命题对于某个概率为 $1$ 的事件中的每个 $\omega\in\Omega$ 都成立,那么
我们说这个命题\emph{几乎肯定}成立,用缩写 a.s. 表示。


\subsection{随机变量}

在本章的剩余部分,我们都考虑一个概率空间 $(\Omega,\mathcal{A},\mathbb{P})$,并且所有
随机变量都将在这个概率空间上定义。

\begin{definition}
  令 $(E,\mathcal{E})$ 是可测空间,值在 $E$ 中的\emph{随机变量}指的是一个可测映射
  $X:\Omega\to E$。
\end{definition}

\begin{example}
  回顾 \eqref{exa:dice model} 中的模型。
  \begin{enumerate}
    \item $X((i,j))=i+j$ 定义了值在 $\{2,3,\dots,12\}$ 中的随机变量。
    \item $X(\omega)=\inf\{j\,|\, \omega_j=6\}$,约定 $\inf \emptyset=\infty$,
    定义了值在 $\bar{\mathbb{N}}=\mathbb{N}\cup\{\infty\}$ 中的随机变量。为了验证
    $X$ 的可测性,只需要注意到
    \[
      X^{-1}(\{k\})=\{\omega\in\Omega\,|\, \omega_1\neq 6,\dots,\omega_{k-1}\neq 6,\omega_k=6\}.
    \]
  \end{enumerate}
\end{example}

\begin{definition}
  令 $X$ 是值在 $(E,\mathcal{E})$ 中的随机变量,定义随机变量 $X$ 的
  \emph{分布律} $\mathbb{P}_X$ 是概率测度 $\mathbb{P}$ 在 $X$ 下的推前。
  也就是说,$\mathbb{P}_X$ 是 $(E,\mathcal{E})$ 上的概率测度,满足
  \[
    \mathbb{P}_X(B)=\mathbb{P}(X^{-1}(B)),\quad \forall B\in \mathcal{E}.
  \]
  两个值在 $(E,\mathcal{E})$ 中的随机变量 $Y,Y'$ 如果有相同的分布
  $\mathbb{P}_Y=\mathbb{P}_{Y'}$,那么我们说 $Y$ 和 $Y'$ 是\emph{同分布}的。
\end{definition}

在概率论中,我们通常将 $\mathbb{P}_X(B)$ 写为 $\mathbb{P}(X\in B)$
而不是 $\mathbb{P}(X^{-1}(B))$。这里 $X\in B$ 是集合
$\{\omega\in\Omega\,|\, X(\omega)\in B\}$ 的简写,这是一个一般性
的简写规则,在概率论中参数 $\omega$ 通常被隐藏。

\paragraph{离散型随机变量}
当 $E$ 是有限或者可数($\mathcal{E}=\mathcal{P}(E)$)的时候,
$X$ 的分布是点测度,这是因为
\[
  \mathbb{P}_X(B)=\mathbb{P}(X\in B)=\mathbb{P}\biggl(\bigcup_{x\in B}\{X=x\}\biggr)
  =\sum_{x\in B}\mathbb{P}(X=x)=\sum_{x\in E}p_x\delta_x(B),
\]
其中 $p_x=\mathbb{P}(X=x)$。这就表明
\[
  \mathbb{P}_X=\sum_{x\in E}p_x\delta_x 
\]
是 $E$ 上的点测度。

\begin{example}
  我们考虑 \eqref{exa:dice model} 中的第二个例子,随机变量
  为 $X(\omega)=\inf\{j\,|\,\omega_j=6\}$。那么
  \begin{align*}
    \mathbb{P}(X=k)&=\mathbb{P}
    \biggl(\bigcup_{1\leq i_1,\dots,i_k\leq 5}\{\omega\,|\, 
    \omega_1=i_1,\dots,\omega_{k-1}=i_{k-1},\omega_k=6\}\biggr)\\
    &=5^{k-1}\left(\frac{1}{6}\right)^k=\frac{1}{6}\left(\frac{5}{6}\right)^{k-1}.
  \end{align*}
  注意到
  \[
    \sum_{k=1}^\infty \mathbb{P}(X=k)=\frac{1}{6}\frac{1}{1-\frac{5}{6}}=1
  \]
  并且 $\{X=\infty\}\cup\bigcup_{k=1}^\infty \{X=k\}=\Omega$,所以
  \[
    \mathbb{P}(X=\infty)=1-\sum_{k=1}^\infty \mathbb{P}(X=k)=0,
  \]  
  但是 $\{X=\infty\}\neq\emptyset$。
\end{example}

\paragraph{具有密度的随机变量} $\mathbb{R}^d$ 上的密度函数是一个非负的 Borel 函数
$p:\mathbb{R}^d\to \mathbb{R}_+$,其满足 
\[
  \int_{\mathbb{R}^d} p(x)\d x=1.
\]
对于一个值在 $\mathbb{R}^d$ 中的随机变量 $X$,如果存在密度 $p$ 使得
\[
  \mathbb{P}_X(B)=\int_B p(x)\d x
\]
对于任意 Borel 子集 $B$ 都成立,那么我们说 $X$ 有密度函数 $p$。
换句话说,$p$ 是 $\mathbb{P}_X$ 相对于 Lebesgue 测度 $\lambda$ 的密度
(\autoref{coro:density of measure}),
也记为 $\mathbb{P}_X(\d x)=p(x)\lambda(\d x)=p(x)\d x$。

注意到密度 $p$ 实际上是在相差一个 Lebesgue 零测集的意义下由 $\mathbb{P}_X$ 确定的。
在我们遇到的大多数例子中,$p$ 在 $\mathbb{R}^d$ 上连续,在这种情况下,$p$ 由
$\mathbb{P}_X$ 唯一确定。

在 $d=1$ 的时候,我们有
\[
  \mathbb{P}(\alpha\leq X\leq \beta)=\int_{\alpha}^\beta p(x)\d x.
\]

\subsection{数学期望}

\begin{definition}
  令 $X$ 是定义在 $(\Omega,\mathcal{A},\mathbb{P})$ 上的实随机变量,我们定义
  \[
    \mathbb{E}[X]=\int_\Omega X(\omega)\mathbb{P}(\d\omega)=\int X\d \mathbb{P},
  \]
  只要上述积分有意义,我们就说 $\mathbb{E}[X]$ 是 $X$ 的\emph{期望}。
\end{definition}

根据前面的内容,上述积分有意义的条件为下列二者之一:
\begin{itemize}[nosep]
  \item $X\geq 0$,此时 $\mathbb{E}[X]\in [0,\infty]$。
  \item $X$ 符号任意,但是 $\mathbb{E}[|X|]=\int|X| \d \mathbb{P}<\infty$。
\end{itemize}


上面的定义可以拓展到多元随机变量 $X=(X_1,\dots,X_d)\in \mathbb{R}^d$,
此时我们定义 $\mathbb{E}[X]=\bigl(\mathbb{E}[X_1],\dots,\mathbb{E}[X_d]\bigr)$。
类似的,如果 $M$ 是随机矩阵(值在实矩阵空间中的随机变量),我们可以定义
矩阵 $\mathbb{E}[M]$ 为对 $M$ 的每个分量求期望构成的矩阵。

注意到若 $X=\mathbold 1_B$,那么
\[
  \mathbb{E}[X]=\int\mathbold 1_B \d \mathbb{P}=\mathbb{P}(B).
\]

对于一些特殊的随机变量,下面的命题被频繁地使用。

\begin{proposition}
  令 $X$ 是值在 $[0,\infty]$ 中的随机变量,那么
  \[
    \mathbb{E}[X]=\int_0^\infty \mathbb{P}(X\geq x) \d x.
  \]
  令 $Y$ 是值在 $\mathbb{Z}_+$ 中的随机变量,那么
  \[
    \mathbb{E}[Y]=\sum_{k=0}^\infty k \mathbb{P}(X=k)=\sum_{k=1}^\infty \mathbb{P}(Y\geq k).
  \]
\end{proposition}
\begin{proof}
  根据 Fubini 定理,我们有
  \[
    \mathbb{E}[X]=\mathbb{E}\left[
      \int_0^\infty \mathbold 1_{\{x\leq X\}}\d x
    \right]=\int_0^\infty \mathbb{E}[\mathbold 1_{\{x\leq X\}}] \d x
    =\int_0^\infty \mathbb{P}(X\geq x) \d x.
  \]
  对于随机变量 $Y$,我们有
  \[
    \mathbb{E}[Y]=\mathbb{E}\left[
      \sum_{k=0}^\infty k\mathbold 1_{\{Y=k\}}
    \right]=\int \biggl(\sum_{k=0}^\infty k\mathbold 1_{\{Y=k\}}\biggr)\d \mathbb{P}
    =\sum_{k=0}^\infty k \mathbb{P}(Y=k).
  \]
  对于第二个等式,只需注意到
  \[
    Y=\sum_{k=1}^\infty \mathbold 1_{\{Y\geq k\}}.\qedhere
  \]
\end{proof}

下面的命题是 \autoref{prop:change variable} 的特例,由于其结果十分重要,所以我们再次叙述一遍。

\begin{proposition}\label{prop:use law to calculate exception}
  令 $X$ 是值在 $(E,\mathcal{E})$ 中的随机变量,对于任意可测函数 $f:E\to [0,\infty]$,
  我们有
  \[
    \mathbb{E}[f(X)]=\int_\Omega f(X(\omega)) \mathbb{P}(\d\omega)=\int_E f(x) \mathbb{P}_X(\d x).
  \]
\end{proposition}

如果可测函数 $f:E\to \mathbb{R}$,上面的命题在两端有意义的情况下也是成立的,
即 $\mathbb{E}[|f(X)|]<\infty$ 的时候。特别地,如果 $X$ 是实值随机变量
且使得 $\mathbb{E}[|X|]<\infty$,那么有
\[
  \mathbb{E}[X]  =\int_{\Omega} X(\omega)\mathbb{P}(\d\omega)
  =\int_{\mathbb{R}} x \mathbb{P}_X(\d x).
\]

\autoref{prop:use law to calculate exception} 告诉我们可以使用
分布 $\mathbb{P}_X$ 来计算 $f(X)$ 的期望。实际上这个过程可以倒过来,
如果我们能找到 $E$ 上的测度 $\nu$ 使得
\[
  \mathbb{E}[f(X)]=\int f\d\nu,  
\]
其中 $f:E\to \mathbb{R}$ 是任意示性函数,此时对于任意
$E$ 的可测子集 $A$,有
\[
  \mathbb{P}_X(A)=\int \indicator{A} \d \mathbb{P}_X=\mathbb{E}[\indicator{A}(X)]=  \int \indicator{A}\d\nu=\nu(A),
\]
所以分布 $\mathbb{P}_X=\nu$。下面的命题应用了这样的思想。

\begin{proposition}\label{prop:margin pdf}
  令 $X=(X_1,\dots,X_d)$ 是值在 $\mathbb{R}^d$ 中的随机变量,假设
  $X$ 有密度 $p(x_1,\dots,x_d)$。那么,对于任意 $1\leq j\leq d$,
  $X_j$ 的密度为
  \[
    p_j(x)=\int_{\mathbb{R}^{d-1}}p(x_1,\dots,x_{j-1},x,x_{j+1},\dots,x_d)
    \d x_1\cdots \d x_{j-1}\d x_{j+1}\cdots \d x_d.  
  \]
\end{proposition}
\begin{proof}
  记 $\pi_j$ 是投影函数 $\pi_j(x_1,\dots,x_d)=x_j$。对于任意的
  Borel 函数 $f:\mathbb{R}\to \mathbb{R}_+$,根据 Fubini 定理,有
  \begin{align*}
    \mathbb{E}[f(X_j)]&=\mathbb{E}[f\circ\pi_j(X)]\\
    &=\int_{\mathbb{R}^d}f(\pi_j(x)) \mathbb{P}_X(\d x)\\
    &=\int_{\mathbb{R}^d}f(x_j)p(x_1,\dots,x_d) \d x_1\cdots\d x_d \\
    &=\int_{\mathbb{R}}f(x_j)\left(
      \int_{\mathbb{R}^{d-1}}p(x_1,\dots,x_d)\d x_1\cdots
      \d x_{j-1}\d x_{j+1}\cdots \d x_d
    \right)\d x_j\\
    &=\int_{\mathbb{R}}f(x_j)p_j(x_j)\d x_j
    =\int_{\mathbb{R}}f(x_j) \mathbb{P}_{X_j}(\d x_j),
  \end{align*}
  这就表明对于任意 Borel 子集 $A$ 有
  \[
    \mathbb{P}_{X_j}(A)=\int_A p_j(x_j)\d x_j,  
  \]
  即 $X_j$ 有密度函数 $p_j$。 
\end{proof}

如果 $X=(X_1,\dots,X_d)$ 是值在 $\mathbb{R}^d$ 中的随机变量,那么
概率测度 $\mathbb{P}_{X_j}$ 被称为 $X$ 的\emph{边缘分布},分布律
$\mathbb{P}_{X_j}$ 由 $\mathbb{P}_X$ 完全决定:$\mathbb{P}_{X_j}$
就是 $\mathbb{P}_X$ 在投影 $\pi_j$ 下的推前。需要注意反之不是正确的,
也就是说即使确定了所有的边缘分布 $\mathbb{P}_{X_1},\dots,\mathbb{P}_{X_j}$,
也不能确定 $\mathbb{P}_X$。
 

\subsection{经典分布}

本小节我们列举一些重要的概率分布。

\paragraph{离散分布}
\begin{enumerate}
  \item \emph{均匀分布}。如果 $E$ 是有限集,值在 $E$ 中的随机变量
  $X$ 如果满足
  \[
    \mathbb{P}(X=x)=\frac{1}{\card(E)},\quad \forall x\in E,  
  \]
  那么我们说 $X$ 是 $E$ 上的均匀分布。
  \item \emph{参数 $\mathbold{p\in[0,1]}$ 的 Bernoulli 分布}。如果值在
  $\{0,1\}$ 中的随机变量 $X$ 满足
  \[
    \mathbb{P}(X=1)=p,\quad \mathbb{P}(X=0)=1-p,  
  \]
  那么我们说 $X$ 是 $E$ 上参数 $p$ 的 Bernoulli 分布。
  \item \emph{二项分布 $\mathbold{\mathcal{B}(n,p)\ (n\in \mathbb{N},p\in[0,1])}$}。
  如果值在 $\{0,1,\dots,n\}$ 中的随机变量 $X$ 满足
  \[
    \mathbb{P}(X=k)=\binom{n}{k}p^k(1-p)^{n-k} ,\quad \forall k\in\{0,1,\dots,n\} ,
  \]
  那么我们说 $X$ 是 $E$ 上的二项分布。
  \item \emph{参数 $\mathbold{p\in(0,1)}$ 的几何分布}。如果
  值在 $\mathbb{Z}_+$ 中的随机变量 $X$ 使得
  \[
    \mathbb{P}(X=k)=(1-p)p^k  ,\quad k\in \mathbb{Z}_+,
  \]
  那么我们说 $X$ 是 $E$ 上参数 $p$ 的几何分布。
  \item \emph{参数 $\mathbold{\lambda>0}$ 的 Poisson 分布}。
  如果值在 $\mathbb{Z}_+$ 中的随机变量 $X$ 使得
  \[
    \mathbb{P}(X=k)=\frac{\lambda^k}{k!}e^{-\lambda},\quad\forall k\in \mathbb{Z}_+,  
  \]
  那么我们说 $X$ 是 $E$ 上参数 $\lambda$ 的 Poisson 分布。容易计算
  \[
    \mathbb{E}[X]=\sum_{k=0}^\infty k \mathbb{P}(X=k)=
    \sum_{k=1}  ^\infty \frac{\lambda^k}{(k-1)!}e^{-\lambda}=\lambda,
  \]
  Poisson 分布在实际应用中非常重要,通常被用于建模某个“罕见事件”
  在长时间段内发生的次数。准确的数学叙述是 Poisson 分布是
  二项分布的近似。对于每个 $n\geq 1$,记 $X_n$ 为服从二项分布
  $\mathcal{B}(n,p_n)$ 的随机变量,如果在 $n\to\infty$ 的时候
  有 $np_n\to\lambda$,那么对于每个 $k\in \mathbb{N}$,有
  \[
    \lim_{n\to\infty} \mathbb{P}(X_n=k)=
    \frac{\lambda^k}{k!}e^{-\lambda}.
  \]
  这可以解释为,如果每天有很小的概率 $p_n\approx\lambda/n$
  发生地震,那么地震在 $n$ 天内发生的次数将近似服从泊松分布。
\end{enumerate}

\paragraph{连续分布}
在下面的五个例子中,$X$ 都指的是一个有密度 $p$ 的实值随机变量。
\begin{enumerate}
  \item \emph{$\mathbold{[a,b]}$ 上的均匀分布}:
  \[
    p(x)=\frac{1}{b-a}\indicator{[a,b]}(x).  
  \]
  \item \emph{参数 $\mathbold{\lambda>0}$ 的指数分布}:
  \[
    p(x)=\lambda e^{-\lambda x}\indicator{\mathbb{R}_+}(x),  
  \]
  此时对于 $a\geq 0$,有
  \[
    \mathbb{P}(X\geq a)=\int_a^\infty p(x)\d x=
    e^{-\lambda a}.  
  \]
  这表明指数分布有下面的重要性质:对于 $a,b\geq 0$,有
  \begin{equation}
    \mathbb{P}(X\geq a+b)=\mathbb{P}(X\geq a)\mathbb{P}(X\geq b).
  \end{equation}
  \item \emph{Gamma 分布 $\mathbold{\Gamma(a,\lambda)\ (a>0,\lambda>0)}$}:
  \[
    p(x)=\frac{\lambda^a}{\Gamma(a)}x^{a-1}e^{-\lambda x}\indicator{\mathbb{R}_+}(x),  
  \]
  这是指数分布的推广,$a=1$ 时即指数分布。
  \item \emph{参数 $\mathbold{a>0}$ 的 Cauchy 分布}:
  \[
    p(x)=\frac{1}{\pi}\frac{a}{a^2+x^2} , 
  \]
  注意到服从 Cauchy 分布的随机变量的数学期望是不存在的,因为
  \[
    \mathbb{E}[|X|]=\int_{-\infty}^{\infty}\frac{1}{\pi}\frac{a|x|}{a^2+x^2}\d x=\infty.  
  \]
  \item \emph{正态分布 $\mathbold{\mathcal{N}(m,\sigma^2)\ (m\in \mathbb{R},\sigma>0)}$}:
  \[
    p(x)=\frac{1}{\sigma\sqrt{2\pi}}\exp\left(-\frac{(x-m)^2}{2\sigma^2}\right)  .
  \]
  正态分布与 Poisson 分布一起成为概率论中最重要的两个分布。正态分布
  的密度曲线呈著名的钟形曲线。按定义很容易验证
  \[
    m=\mathbb{E}[X],\quad \sigma^2 =\mathbb{E}[(X-m)^2]. 
  \]

  对于 $a,b\in \mathbb{R}$,考虑随机变量 $Y=aX+b$,那么
  对于任意的 Borel 函数 $f:\mathbb{R}\to \mathbb{R}_+$,有
  \begin{align*}
    \mathbb{E}[f(Y)]&=\mathbb{E}[f(aX+b)]=\int_{\mathbb{R}}f(ax+b)\mathbb{P}_X(\d x)\\
    &=\int_{\mathbb{R}}f(ax+b)p(x)\d x
    =\frac{1}{a}\int_{\mathbb{R}}f(y)p\left(\frac{y-b}{a}\right)\d y\\
    &=\int_{\mathbb{R}}f(y)\frac{1}{a}p\left(\frac{y-b}{a}\right)\d y, 
  \end{align*}
  这表明
  \[
    p(y)=\frac{1}{a\sigma\sqrt{2\pi}}\exp\left(-\frac{(y-(am+b))^2}{2(a\sigma)^2}\right),
  \]
  即 $aX+b$ 服从分布 $\mathcal{N}(am+b,a^2\sigma^2)$。
\end{enumerate}

\subsection{实值随机变量的分布函数}

令 $X$ 是实值随机变量,定义 $X$ 的\emph{分布函数}为 $F_X:\mathbb{R}\to [0,1]$,
其满足
\[
  F_X(t)=\mathbb{P}(X\leq t)=\mathbb{P}_X((-\infty,t]),\quad \forall t\in \mathbb{R}.  
\]
根据 \autoref{coro:uniqueness of measure},$F_X$ 实际上完全刻画了分布 $\mathbb{P}_X$。
确切的说,如果知道了 $F_X$,即相当于知道了所有 $\mathbb{P}_X((-\infty,t])$ 的值,
而所有区间 $(-\infty,t]$ 构成的子集族对有限交封闭,又因为 $\mathbb{P}_X$
为有限测度,所以 $\mathbb{P}_X$ 在所有区间 $(-\infty,t]$ 上的值可以完全
确定 $\mathbb{P}_X$ 在 $\mathcal{B}(\mathbb{R})$ 上的值。

显然函数 $F_X$ 是递增的、右连续的并且在 $-\infty$ 处极限为 $0$、
在 $+\infty$ 处极限为 $1$。反之,如果 $F:\mathbb{R}\to [0,1]$ 满足
上面的性质,定理 ?表明存在(唯一的) $\mathbb{R}$ 上的概率测度
$\mu$ 使得 $\mu((-\infty,t])=F(t)$。即这样的函数 $F$ 总能
解释为某个实值随机变量的分布函数。

令 $F_X(a-)$ 表示 $F_X$ 在 $a\in \mathbb{R}$ 处的左极限。那么
容易验证
\begin{align*}
  \mathbb{P}(a\leq X\leq b)&=F_X(b)-F_X(a-),\\
  \mathbb{P}(a<X<b)&=F_X(b-)-F_X(a).
\end{align*}
特别的,$\mathbb{P}(X=a)=F_X(a)-F_X(a-)$。这表明 $F_X$ 的间断点的个数
恰为 $\mathbb{P}_X$ 的原子个数。


\section{随机变量的矩}

\subsection{矩和方差}

令 $X$ 是实值随机变量,$p\in \mathbb{N}$。定义 $X$ 的\emph{$p$-阶矩}
为 $\mathbb{E}[X^p]$,其仅在 $X\geq 0$ 或者 $\mathbb{E}[|X|^p]<\infty$
的时候有定义。

因为期望是相对于测度 $\mathbb{P}_X$ 的一种积分,所以我们有下面的结果。
如果 $X$ 是值在 $[0,\infty]$ 中的随机变量,那么我们有
\begin{itemize}[nosep]
  \item $\mathbb{E}[X]<\infty\Rightarrow X<\infty,\ \alsu{\mathbb{P}_X}$
  \item $\mathbb{E}[X]=0\Rightarrow X=0\,\ \alsu{\mathbb{P}_X}$
\end{itemize}
此外,各种极限与积分交换次序的定理也可以直接改写为期望的形式:
\begin{itemize}[nosep]
  \item \emph{单调收敛定理}。如果 $(X_n)_{n\in \mathbb{N}}$ 是一列
  值在 $[0,\infty]$ 中递增的随机变量,那么
  \[
    \ulim[n\to\infty]  \mathbb{E}[X_n]=\mathbb{E}\left[\ulim[n\to\infty]X_n\right].
  \]
  \item \emph{Fatou 引理}。如果 $(X_n)_{n\in \mathbb{N}}$ 是一列
  值在 $[0,\infty]$ 中的随机变量,那么
  \[
    \mathbb{E}[\liminf X_n]\leq \liminf \mathbb{E}[X_n].  
  \]
  \item \emph{控制收敛定理}。如果 $(X_n)_{n\in \mathbb{N}}$ 是一列
  实值随机变量,并且存在值在 $[0,\infty]$ 中的随机变量 $Z$ 使得
  \[
    |X_n|\leq Z,\quad \mathbb{E}[Z]<\infty, \quad X_n\to X,\ 
    \alsu{\mathbb{P}_X}  
  \]
  那么
  \[
    \lim_{n\to\infty}\mathbb{E}[X_n]=\mathbb{E}\left[\lim_{n\to\infty}X_n\right]=\mathbb{E}[X],
    \quad \lim_{n\to\infty}\mathbb{E}[|X_n-X|]=0.
  \]
\end{itemize}

对于每个 $p\in[1,\infty]$,考虑空间 $L^p(\Omega,\mathcal{A},\mathbb{P})$。
H\"older 不等式表明对于任意实值随机变量 $X,Y$,如果 $p,q\in (1,\infty)$
使得 $1/p+1/q=1$,那么
\[
  \mathbb{E}[|XY|]\leq \mathbb{E}[|X|^p]^{1/p}\mathbb{E}[|Y|^q]^{1/q}. 
\]
取 $Y=1$,我们得到 $\norm{X}_1\leq\norm{X}_p$。此外,
如果 $1\leq p<q\leq \infty$,有 $\norm{X}_p\leq \norm{X}_q$,
这也表明 $L^q(\Omega, \mathcal{A},\mathbb{P})\subseteq L^p(\Omega,\mathcal{A},\mathbb{P})$。

Hilbert 空间 $L^2(\Omega,\mathcal{A},\mathbb{P})$ 上的内积定义为
$\langle X,Y\rangle_{L^2}=\mathbb{E}[XY]$,Cauchy-Schwarz 不等式表明
\[
  \mathbb{E}[|XY|]\leq \mathbb{E}[X^2]  ^{1/2}\mathbb{E}[Y^2]^{1/2}.
\]
特别地,我们有
\[
  \mathbb{E}[|X|]^2\leq \mathbb{E}[X^2].  
\]

如果 $X\in L^1(\Omega,\mathcal{A},\mathbb{P})$,$f:\mathbb{R}\to \mathbb{R}_+$
是凸函数,那么 Jensen 不等式表明 
\[
  \mathbb{E}[f(X)]\geq f(\mathbb{E}[X]).
\]

\begin{definition}
  令 $X\in L^2(\Omega,\mathcal{A},\mathbb{P})$,定义 $X$
  的\emph{方差}为
  \[
    \var(X)=\mathbb{E}\bigl[(X-\mathbb{E}[X])^2\bigr]\geq 0,
  \]
  $X$ 的\emph{标准差}为
  \[
    \sigma_X=\sqrt{\var(X)}.  
  \]
\end{definition}

\begin{proposition}
  令 $X\in L^2(\Omega, \mathcal{A},\mathbb{P})$,方差
  $\var(X)=\mathbb{E}[X^2]-\bigl(\mathbb{E}[X]\bigr)^2$。
  对于任意的 $a\in \mathbb{R}$,有
  \[
    \mathbb{E}[(X-a)^2]=\var(X)+\bigl(\mathbb{E}[X]-a\bigr)  ^2.
  \]
\end{proposition}



\chapter{独立性}

\section{独立事件}

在本章中,我们考虑概率空间 $(\Omega,\mathcal{A},\mathbb{P})$。如果 
$A,B\in \mathcal{A}$ 且
\[
  \mathbb{P}(A\cap B)=\mathbb{P}(A)\mathbb{P}(B),  
\]
那么我们说 $A$ 和 $B$ 是\emph{独立的}。

\section{$\sigma$-域和随机变量的独立性}

如果 $\mathcal{B}\subseteq \mathcal{A}$ 是一个 $\sigma$-域,那么我们说
$\mathcal{B}$ 是 $\mathcal{A}$ 的\emph{子 $\mathbold{\sigma}$-域}。我们可以认为
子 $\sigma$-域 $\mathcal{B}$ 反映了概率空间的部分信息,即 $\mathcal{B}$ 中
发生的事件。例如,如果 $\mathcal{B}=\sigma(X)$,$X$ 是随机变量,那么
$\mathcal{B}$ 反映了 $X$ 的值的信息。这暗示了子 $\sigma$-域的独立性的概念:
我们希望两个子 $\sigma$-域 $\mathcal{B}$ 和 $\mathcal{B}'$ 是独立的当且仅当
它们中的任意两个事件都是独立的。

\begin{definition}
  令 $\mathcal{B}_1,\dots,\mathcal{B}_n$ 是 $\mathcal{A}$ 的 $n$ 个
  $\sigma$-子域,我们说 $\mathcal{B}_1,\dots,\mathcal{B}_n$ 是独立的,如果
  对于任意 $A_1\in \mathcal{B}_1,\dots,A_n\in \mathcal{B}_n$,都有
  \[
    \mathbb{P}(A_1\cap\cdots\cap A_n)=\mathbb{P}(A_1)\cdots \mathbb{P}(A_n).  
  \]
\end{definition}

令 $X_1,\dots,X_n$ 分别是值在 $(E_1,\mathcal{E}_1),\dots,(E_n,\mathcal{E}_n)$
中的随机变量,我们说 $X_1,\dots,X_n$ 是独立的当且仅当
$\sigma(X_1),\dots,\sigma(X_n)$ 是独立的。这等价于
任取 $F_1\in \mathcal{E}_1,\dots,F_n\in \mathcal{E}_n$ 有
\[
  \mathbb{P}(\{X_1\in F_1\}\cap\cdots\cap\{X_n\in F_n\})  
  =\mathbb{P}(X_1\in F_1)\cdots \mathbb{P}(X_n\in F_n).
\]

\begin{theorem}
  令 $X_1,\dots,X_n$ 分别是值在 $(E_1,\mathcal{E}_1),\dots,(E_n,\mathcal{E}_n)$
  中的随机变量。那么 $X_1,\dots,X_n$ 是独立的当且仅当 $(X_1,\dots,X_n)$
  的分布是 $X_1,\dots,X_n$ 的分布的乘积测度,即
  \[
    \mathbb{P}_{(X_1,\dots,X_n)}=\mathbb{P}_{X_1}\otimes\cdots
    \otimes \mathbb{P}_{X_n}.  
  \]
  此外,我们有
  \[
    \mathbb{E}\biggl[\prod_{i=1}^n f_i(X_i)\biggr] =\prod_{i=1}^n \mathbb{E}[f_i(X_i)],
  \]
  其中 $f_i$ 是 $(E_i,\mathcal{E}_i)$ 上的非负可测函数。
\end{theorem}
\begin{proof}
  令 $F_i\in \mathcal{E}_i$,$X_1,\dots,X_n$ 独立当且仅当
  \[
    \mathbb{P}(\{X_1\in F_1\}\cap\cdots\cap\{X_n\in F_n\})  
    =\mathbb{P}(X_1\in F_1)\cdots \mathbb{P}(X_n\in F_n),
  \]
  这表明
  \[ 
    \mathbb{P}_{(X_1,\dots,X_n)}(F_1\times\cdots\times F_n)
    =\mathbb{P}_{X_1}(F_1)\cdots \mathbb{P}_{X_n}(F_n).
  \]
  所以 $\mathbb{P}_{(X_1,\dots,X_n)}$ 就是乘积测度 
  $\mathbb{P}_{X_1}\otimes\cdots\otimes \mathbb{P}_{X_n}$。

  记 $\pi_i:E_1\times \cdots\times E_n\to E_i$ 为投影,根据 Fubini 定理,有
  \begin{align*}
    \mathbb{E}\biggl[\prod_{i=1}^n f_i(X_i)\biggr]&=
    \int_{E_1\times\cdots \times E_n}\prod_{i=1}^n f_i\circ\pi_i
    \d\mathbb{P}_{(X_1,\dots,X_n)}\\
    &=\prod_{i=1}^n \int_{E_i} f_i(x_i) \mathbb{P}_{X_i}(\d x_i)\\
    &=\prod_{i=1}^n \mathbb{E}[f_i(X_i)].\qedhere
  \end{align*}
\end{proof}
\begin{remark}
  该定理中 $f_i$ 也可以是任意符号,此时如果 $\mathbb{E}[|f_i(X_i)|]<\infty$,
  也即 $f_i\in \mathcal{L}^1(E_i,\mathcal{E}_i,\mathbb{P}_{X_i})$,那么
  定理的结论依然成立。
\end{remark}

\begin{corollary}\label{coro:independence by pdf}
  令 $X_1,\dots,X_n$ 是实值随机变量。
  \begin{enumerate}
    \item 假设 $X_i$ 有密度 $p_i$ 并且 $X_1,\dots,X_n$ 是独立的,那么
    $(X_1,\dots,X_n)$ 有密度
    \[
      p(x_1,\dots,x_n)=\prod_{i=1}^n p_i(x_i).  
    \]
    \item 反之,假设 $(X_1,\dots,X_n)$ 有密度 $p$,并且
    $p$ 可以表达为
    \[
    p(x_1,\dots,x_n)=\prod_{i=1}^n q_i(x_i),  
    \]
    其中 $q_i$ 是 $\mathbb{R}$ 上的非负可测函数。那么 $X_1,\dots,X_n$
    是独立的并且 $X_i$ 有密度 $p_i=C_iq_i$,其中 $C_i>0$ 为常数。
  \end{enumerate}
\end{corollary}
\begin{proof}
  (1) $X_i$ 有密度 $p_i$ 表明 $\mathbb{P}_{X_i}(\d x)=p_i(x)\d x$,
  $X_1,\dots,X_n$ 独立表明
  \begin{align*}
    \mathbb{P}_{(X_1,\dots,X_n)}(A)&=
    \mathbb{P}_{X_1}\otimes\cdots\otimes \mathbb{P}_{X_n}(A)
    =\int_{\mathbb{R}^n} \indicator{A} \d \mathbb{P}_{X_1}\otimes\cdots\otimes \mathbb{P}_{X_n}
    \\
    &=\int_{\mathbb{R}}\cdots \int_{\mathbb{R}}\indicator{A}(x_1,\dots,x_n)
    \mathbb{P}_{X_1}(\d x_1)\cdots \mathbb{P}_{X_n}(\d x_n) \\
    &=\int_{\mathbb{R}}\cdots \int_{\mathbb{R}}\indicator{A}(x_1,\dots,x_n)
    \prod_{i=1}^n p_i(x_i)
    \d x_1\cdots \d x_n\\
    &=\int_{\mathbb{R}^n} \indicator{A}\prod_{i=1}^n p_i \d \lambda
    =\int_A \prod_{i=1}^np_i(x_i) \d x_1\cdots\d x_n,
  \end{align*}
  这就表明 
  \[
    p(x_1,\dots,x_n)=\prod_{i=1}^n p_i(x_i).  
  \]

  (2) 根据 \autoref{prop:margin pdf},有
  \[
    p_i(x_i)=\int_{\mathbb{R}^{n-1}}p(x_1,\dots,x_n)\d x_1\cdots \d x_{i-1}
    \d x_{i+1}\cdots\d x_n
    =q_i(x_i)\prod_{j\neq i}\int_{\mathbb{R}}q_j(x_j)\d x_j,
  \]
  故 $p_i=C_iq_i$。此时
  \begin{align*}
    p(x_1,\dots,x_n)=\prod_{i=1}^n q_i(x_i)
    =\prod_{i=1}^n \frac{1}{C_i}p_i(x_i),
  \end{align*}
  两边积分可知 $\prod_{i=1}^n C_i=1$,所以
  \[
    p(x_1,\dots,x_n)=\prod_{i=1}^n p_i(x_i),  
  \]
  这就表明 $\mathbb{P}_{(X_1,\dots,X_n)}=\mathbb{P}_{X_1}\otimes\cdots \mathbb{P}_{X_n}$,
  即 $X_1,\dots,X_n$ 独立。
\end{proof} 

\begin{example}
  令 $U$ 是服从参数 $1$ 的指数分布的随机变量,$V$ 是服从 $[0,1]$ 上
  的均匀分布的随机变量,假设 $U,V$ 是独立的,记
  \[
    X=\sqrt{U}\cos(2\pi V),\quad Y=\sqrt{U}\sin(2\pi V),  
  \]
  证明 $X,Y$ 是独立的随机变量。
\end{example}
\begin{proof}
  任取非负可测函数 $\varphi:\mathbb{R}^2\to \mathbb{R}$,有
  \begin{align*}
    \mathbb{E}[\varphi(X,Y)]
    &=\mathbb{E}
    \bigl[\varphi\bigl(\sqrt{U}\cos(2\pi V),\sqrt{U}\sin(2\pi V)\bigr)\bigr]\\
    &=\int_{\mathbb{R}^2}\varphi\bigl(\sqrt{u}\cos(2\pi v),\sqrt{u}\sin(2\pi v)\bigr)
    \d \mathbb{P}_{(U,V)}\\
    &=\int_{\mathbb{R}}\int_{\mathbb{R}}\varphi\bigl(\sqrt{u}\cos(2\pi v),\sqrt{u}\sin(2\pi v)\bigr)
      e^{-u}\indicator{[0,\infty)}(u)\indicator{[0,1]}(v)\d u\d v\\
    &=\int_{0}^\infty\int_0^1\varphi\bigl(\sqrt{u}\cos(2\pi v),\sqrt{u}\sin(2\pi v)\bigr)
    e^{-u}\d u\d v \\
    &=\frac{1}{\pi}\int_0^\infty\int_0^{2\pi}
    \varphi(r\cos\theta,r\sin\theta)re^{-r^2}\d r\d\theta\\
    &=\frac{1}{\pi}\int_{\mathbb{R}^2}\varphi(x,y)e^{-x^2-y^2}\d x\d y.
  \end{align*}
  这表明 $(X,Y)$ 有概率密度 $p(x,y)=\pi^{-1}\exp(-x^2-y^2)=\pi^{-1}\exp(-x^2)\exp(-y^2)$,
  根据 \autoref{coro:independence by pdf},这表明 $X,Y$
  是独立的。
\end{proof}

\section{Borel-Cantelli 引理}

回顾集合极限的定义:如果 $(A_n)_{n\in \mathbb{N}}$ 是一列集合,
我们记
\[
  \limsup A_n=\bigcap_{n=1}^\infty \bigcup_{k=n}^\infty A_k,  
\]
不难发现点 $\omega\in\limsup A_n$ 当且仅当存在无限多个
$n$ 使得 $\omega\in A_n$。注意到
\ref{lemma:liminf and limsup ineq} 告诉我们
$\mathbb{P}(\limsup A_n)\geq \limsup \mathbb{P}(A_n)$。

\begin{lemma}
  令 $(A_n)_{n\in \mathbb{N}}$ 是一列事件。
  \begin{enumerate}
    \item 如果 $\sum_{n\in \mathbb{N}}\mathbb{P}(A_n)<\infty$,那么
    \[
      \mathbb{P}(\limsup A_n)=0,  
    \]
    \item 如果 $\sum_{n\in \mathbb{N}}\mathbb{P}(A_n)=\infty$,
    事件 $A_n$ 是独立的,那么
    \[
      \mathbb{P}(\limsup A_n)=1.  
    \]
  \end{enumerate}
\end{lemma}

\paragraph{两个应用}
(2) 我们令
\[
  (\Omega,\mathcal{A},\mathbb{P})=\bigl([0,1),\mathcal{B}([0,1)),\lambda\bigr) ,
\]
其中 $\lambda$ 表示 Lebesgue 测度。对于每个 $n\in \mathbb{N}$,令
\[
  X_n(\omega)=\lfloor 2^n\omega\rfloor-2\lfloor 2^{n-1}\omega\rfloor,
\]
其中 $\lfloor x\rfloor$ 表示向下取整。那么 $X_n(\omega)\in\{0,1\}$
并且容易验证对于任意 $\omega\in [0,1)$ 有
\[
  0\leq\omega-\sum_{k=1}^nX_k(\omega)2^{-k}<2^{-n}.  
\]
这表明
\[
  \omega=\sum_{k=1}^\infty X_k(\omega)2^{-k}.  
\]




\end{document}
