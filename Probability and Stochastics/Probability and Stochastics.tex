\documentclass[fontset=none]{Notes}

\makeatletter
\DeclareRobustCommand{\em}{%
  \@nomath\em \if b\expandafter\@car\f@series\@nil
  \normalfont \else \bfseries \fi}
\makeatother

\usepackage{tikz-cd,wrapstuff}
\usepackage{fixdif,siunitx,tikz,nicematrix}
\usetikzlibrary{matrix,calc}

\ProvidesFile{font.def}

\setCJKmainfont{Source Han Serif SC}[
  UprightFont=*-Regular,
  BoldFont=*-Bold,
  ItalicFont=HYKaiTi S,
  ItalicFeatures={Scale=1.1}
]
\newCJKfontfamily[zhsong]\songti{Source Han Serif SC}[
  UprightFont=*-Regular,
  BoldFont=*-Bold,
  ItalicFont=HYKaiTi S,
  ItalicFeatures={Scale=1.1}
]
\setCJKsansfont{Source Han Sans SC}[
  UprightFont=*-Regular,
  BoldFont=*-Bold
]
\newCJKfontfamily[zhhei]\heiti{Source Han Sans SC}[
  UprightFont=*-Regular,
  BoldFont=*-Bold
]
\setCJKmonofont{HYFangSong S}[
  BoldFont=*,
  ItalicFont=*,
  BoldItalicFont=*
]
\newCJKfontfamily[zhfs]\fangsong{HYFangSong S}[
  BoldFont=*,
  ItalicFont=*,
  BoldItalicFont=*
]
\newCJKfontfamily[zhkai]\kaishu{HYKaiTi S}[
  BoldFont=*,
  ItalicFont=*,
  BoldItalicFont=*
]

\setmainfont{texgyretermes}[
  Extension=.otf,
  UprightFont=*-regular,
  BoldFont=*-bold,
  ItalicFont=*-italic,
  BoldItalicFont=*-bolditalic,
  SlantedFont=*-italic
]
%\setmathrm{texgyretermes}[
%  Extension=.otf,
%  UprightFont=*-regular,
%  BoldFont=*-bold,
%  ItalicFont=*-italic,
%  BoldItalicFont=*-bolditalic,
%  SlantedFont=*-italic
%]
\setsansfont{Cantarell}[
  UprightFont=* Regular,
  ItalicFont=* Italic,
  BoldFont=* Bold,
  BoldItalicFont=* Bold Italic,
  SmallCapsFont=Alegreya Sans SC
]
\setmonofont{Ubuntu Mono}[
  UprightFont=*,
  ItalicFont=* Italic,
  BoldFont=* Bold,
  BoldItalicFont=* Bold Italic
]
%\setmathfont{texgyretermes-math.otf}
%\setmathfont[range={\mathcal,\mathbfcal,\mathfrak},StylisticSet=1]{XITSMath-Regular.otf}
%\setmathfont[range={\mathbb}]{KpMath-Sans.otf}



\usepackage[subscriptcorrection,nofontinfo,mtpbb,mtpfrak]{mtpro2}

\tikzcdset{
  arrow style=tikz,
  diagrams={>={Straight Barb[scale=0.8]}}
}

\allowdisplaybreaks[1]

\newlength{\mymathln}
\newcommand{\aligninside}[2]{
  \settowidth{\mymathln}{#2}
  \mathmakebox[\mymathln]{#1}
}

\DeclareMathOperator\Spec{Spec}
\DeclareMathOperator\im{im}
\DeclareMathOperator\nil{nil}
\DeclareMathOperator\rad{rad}
\DeclareMathOperator\Ann{Ann}
\DeclareMathOperator\Max{Max}
\DeclareMathOperator\GL{GL}
\DeclareMathOperator\End{End}
\DeclareMathOperator\Int{Int}
\DeclareMathOperator\Tor{Tor}
\DeclareMathOperator\Frac{Frac}
\DeclareMathOperator\Tr{Tr}
\DeclareMathOperator\Hom{Hom}
\DeclareMathOperator\supp{supp}
\DeclareMathOperator\Id{Id}
\DeclareMathOperator\rk{rank}
\DeclareMathOperator\coker{coker}

\newcommand{\ideal}[1]{\mathfrak{#1}}
\newcommand{\mat}[1]{\mathbold{#1}}
\newcommand{\uline}{\underline{\hphantom{X}}}
\newcommand{\abs}[1]{\left|#1\right|}

\usepackage{enumitem}

\setlist[enumerate]{nosep}

%\DeclareMathAlphabet\mathcal{OMS}{cmsy}{m}{n}

\newlength\stextwidth
\newcommand\makesamewidth[3][c]{%
  \settowidth{\stextwidth}{#2}%
  \makebox[\stextwidth][#1]{#3}%
}



\begin{document}

\frontmatter

\tableofcontents

\mainmatter

\chapter{测度和积分}

\section{测度空间}

令 $E$ 是集合,$\mathcal{E}$ 是 $E$ 的一个子集族。若对于任意 $A,B\in\mathcal{E}$ 有
$A\cap B\in\mathcal{E}$,那么我们说 $\mathcal{E}$ \emph{对交封闭}。如果 $\mathcal{E}$
中任意可数个集合的交还在 $\mathcal{E}$ 中,那么我们说 $\mathcal{E}$ 对可数交封闭。
类似地,我们可以定义对补封闭、对并封闭和对可数并封闭的概念。

\subsubsection{$\sigma$-代数}

如果 $E$ 的非空子集族 $\mathcal{E}$ 对补和有限并封闭,那么我们说 $\mathcal{E}$
是 $E$ 上的\emph{代数}。如果其对补和可数并封闭,那么我们说 $\mathcal{E}$
是 $E$ 上的\emph{$\sigma$-代数},即:
\begin{alphenum}
  \item $A\in\mathcal{E}\Rightarrow E\smallsetminus A\in\mathcal{E}$,
  \item $A_1,A_2,\dotsc\in \mathcal{E}\Rightarrow\bigcup_n A_n\in\mathcal{E}$。
\end{alphenum}
由于 $\left(\bigcup_n A_n\right)^c=\bigcap_n A_n^c\in\mathcal{E}$,所以对补和可数并封闭
可以自然导出对可数交封闭,即 $\sigma$-代数对可数交也封闭。

任取 $A\in \mathcal{E}$,那么 $E=A\cup(E \smallsetminus A)\in \mathcal{E}$,所以 $E$ 上
任意 $\sigma$-代数都至少包含 $E$ 和 $\emptyset$。事实上,$\mathcal{E}=\{E,\emptyset\}$
是 $E$ 上的最简单的 $\sigma$-代数,被称为\emph{平凡 $\sigma$-代数}。$E$ 上最大的
$\sigma$-代数当然是 $\mathcal{E}=2^E$,即 $\mathcal{E}$ 就是 $E$ 的幂集,被
称为\emph{离散 $\sigma$-代数}。

不难看出,$E$ 上一族 $\sigma$-代数的任意交(不一定可数)还是 $E$ 上的 $\sigma$-代数。
给定 $E$ 的一个子集族 $\mathcal{C}$,我们可以考虑所有包含 $\mathcal{C}$ 
的 $\sigma$-代数(总是存在至少一个这样的 $\sigma$-代数,即 $2^E$),将这些
$\sigma$-代数取交集,我们便得到了包含 $\mathcal{C}$ 的最小的 $\sigma$-代数,
被称为\emph{由 $\mathcal{C}$ 生成的} $\sigma$-代数,记为 $\sigma\mathcal{C}$。

如果 $E$ 是拓扑空间,由 $E$ 的所有开集族生成的 $\sigma$-代数被称为 $E$ 上的
\emph{Borel $\sigma$-代数},记为 $\mathcal{B}(E)$ 或者 $\mathcal{B}_E$,
其元素被称为\emph{Borel 集}。

\subsubsection{p-系和 d-系}

对于 $E$ 的子集族 $\mathcal{C}$,如果其对交封闭,那么我们说 $\mathcal{C}$
是一个 p-系,这里 p 代表 product,是“交”的另一种说法。
$E$ 的子集族 $\mathcal{D}$ 被称为 d-系,如果其满足:
\begin{alphenum}
  \item $E\in\mathcal{D}$,
  \item $\text{$A,B\in \mathcal{D}$ and $A\supseteq B\Rightarrow A \smallsetminus B\in \mathcal{D}$}$,
  \item $\text{$(A_n)\subseteq \mathcal{D}$ and $A_n\nearrow A\Rightarrow A\in \mathcal{D}$}$。
\end{alphenum}
其中 $(A_n)\subseteq D$ 表明 $(A_n)$ 是 $D$ 中的集合序列,$A_n\nearrow A$ 表明这个序列
递增于极限 $A$:
\[
  A_1\subseteq A_2\subseteq\cdots,\quad \bigcup_{n=1}^\infty A_n=A.
\]

显然一个 $\sigma$-代数既是 p-系又是 d-系,其反面也是成立的。所以 p-系和 d-系是产生
$\sigma$-代数的原始结构。

\begin{proposition}
  $E$ 的子集族是 $\sigma$-代数当且仅当其既是 p-系又是 d-系。
\end{proposition}
\begin{proof}
  $(\Rightarrow)$ 若 $\mathcal{E}$ 是 $\sigma$-代数,其显然是 p-系并且满足 d-系的条件 (a) 和 (c)。
  下面我们验证其满足 d-系的条件 (b)。任取 $A,B\in\mathcal{E}$ 且 $A\supseteq B$,
  那么 $A \smallsetminus B=A\cap (E \smallsetminus B)\in \mathcal{E}$,
  所以 $\mathcal{E}$ 是 d-系。

  $(\Leftarrow)$ 若 $\mathcal{E}$ 既是 p-系又是 d-系。任取 $A\in \mathcal{E}$,
  根据 d-系的 (a) 和 (b),我们有 $E \smallsetminus A\in \mathcal{E}$。
  所以 $\mathcal{E}$ 对补封闭。然后我们说明对并封闭。任取 $A,B\in \mathcal{E}$, 
  由于
  \[
    A\cup B=E \smallsetminus (A\cup B)^c=E \smallsetminus (A^c\cap B^c),
  \]
  结合 p-系对交封闭,所以 $A\cup B\in \mathcal{E}$。最后我们说明对可数并封闭。
  如果 $(A_n)\subseteq \mathcal{E}$,令 $B_n=A_1\cup\cdots\cup A_n$,
  那么 $(B_n)\subseteq \mathcal{E}$ 且 $B_n\nearrow A$,根据 d-系的 (c),所以
  $A\in \mathcal{E}$,故 $\mathcal{E}$ 对可数并封闭。
\end{proof}

下面的引理为本节的主要定理做准备。

\begin{lemma}
  令 $\mathcal{D}$ 是 $E$ 上的 d-系,固定 $D\in \mathcal{D}$,令
  \[
    \hat{\mathcal{D}}=\{A\in \mathcal{D}: A\cap D\in \mathcal{D}\},
  \]
  那么 $\hat{\mathcal{D}}$ 仍然是 d-系。
\end{lemma}

\subsubsection{单调类定理}

这是一个非常有用的工具来证明某些集族是 $\sigma$-代数。

\begin{theorem}
  如果一个 d-系包含一个 p-系,那么其包含这个 p-系生成的 $\sigma$-代数。
\end{theorem}
\begin{proof}
  
\end{proof}












\end{document}