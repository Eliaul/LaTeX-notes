\documentclass[fontset=none]{Notes}

\makeatletter
\DeclareRobustCommand{\em}{%
  \@nomath\em \if b\expandafter\@car\f@series\@nil
  \normalfont \else \bfseries \fi}
\makeatother

\usepackage{tikz-cd,wrapstuff}
\usepackage{siunitx,tikz,nicematrix}
\usetikzlibrary{matrix,calc}
\usetikzlibrary{intersections}
\usetikzlibrary{arrows.meta}
\usetikzlibrary{decorations.markings}

\ProvidesFile{font.def}

\setCJKmainfont{Source Han Serif SC}[
  UprightFont=*-Regular,
  BoldFont=*-Bold,
  ItalicFont=HYKaiTi S,
  ItalicFeatures={Scale=1.1}
]
\newCJKfontfamily[zhsong]\songti{Source Han Serif SC}[
  UprightFont=*-Regular,
  BoldFont=*-Bold,
  ItalicFont=HYKaiTi S,
  ItalicFeatures={Scale=1.1}
]
\setCJKsansfont{Source Han Sans SC}[
  UprightFont=*-Regular,
  BoldFont=*-Bold
]
\newCJKfontfamily[zhhei]\heiti{Source Han Sans SC}[
  UprightFont=*-Regular,
  BoldFont=*-Bold
]
\setCJKmonofont{HYFangSong S}[
  BoldFont=*,
  ItalicFont=*,
  BoldItalicFont=*
]
\newCJKfontfamily[zhfs]\fangsong{HYFangSong S}[
  BoldFont=*,
  ItalicFont=*,
  BoldItalicFont=*
]
\newCJKfontfamily[zhkai]\kaishu{HYKaiTi S}[
  BoldFont=*,
  ItalicFont=*,
  BoldItalicFont=*
]

\setmainfont{texgyretermes}[
  Extension=.otf,
  UprightFont=*-regular,
  BoldFont=*-bold,
  ItalicFont=*-italic,
  BoldItalicFont=*-bolditalic,
  SlantedFont=*-italic
]
%\setmathrm{texgyretermes}[
%  Extension=.otf,
%  UprightFont=*-regular,
%  BoldFont=*-bold,
%  ItalicFont=*-italic,
%  BoldItalicFont=*-bolditalic,
%  SlantedFont=*-italic
%]
\setsansfont{Cantarell}[
  UprightFont=* Regular,
  ItalicFont=* Italic,
  BoldFont=* Bold,
  BoldItalicFont=* Bold Italic,
  SmallCapsFont=Alegreya Sans SC
]
\setmonofont{Ubuntu Mono}[
  UprightFont=*,
  ItalicFont=* Italic,
  BoldFont=* Bold,
  BoldItalicFont=* Bold Italic
]
%\setmathfont{texgyretermes-math.otf}
%\setmathfont[range={\mathcal,\mathbfcal,\mathfrak},StylisticSet=1]{XITSMath-Regular.otf}
%\setmathfont[range={\mathbb}]{KpMath-Sans.otf}



\usepackage[subscriptcorrection,nofontinfo,mtpbb,mtpfrak]{mtpro2}
\usepackage[normal]{fixdif}

\tikzcdset{
  arrow style=tikz,
  diagrams={>={Straight Barb[scale=0.8]}}
}

\tikzset{
  every picture/.style={
    thick,
    >={Latex[width=6pt, length=8pt]}
  },
  point/.style={
    circle, fill, minimum width=5pt,
    inner sep=0pt
  },
  straight arrow/.style={
    Straight Barb[scale=0.8]
  }
}

\allowdisplaybreaks[1]

\newlength{\mymathln}
\newcommand{\aligninside}[2]{
  \settowidth{\mymathln}{#2}
  \mathmakebox[\mymathln]{#1}
}

\DeclareMathOperator\Spec{Spec}
\DeclareMathOperator\im{im}
\DeclareMathOperator\sgn{sgn}
\DeclareMathOperator\rad{rad}
\DeclareMathOperator\Alt{Alt}
\DeclareMathOperator\Max{Max}
\DeclareMathOperator\card{card}
\DeclareMathOperator\GL{GL}
\DeclareMathOperator\Orth{O}
\DeclareMathOperator\SO{SO}
\DeclareMathOperator\SU{SU}
\DeclareMathOperator\cls{cls}
\DeclareMathOperator\Lie{Lie}
\DeclareMathOperator\End{End}
\DeclareMathOperator\Int{Int}
\DeclareMathOperator\Sym{Sym}
\DeclareMathOperator\tr{tr}
\DeclareMathOperator\Hom{Hom}
\DeclareMathOperator\supp{supp}
\DeclareMathOperator\Id{Id}
\DeclareMathOperator\rk{rank}
\DeclareMathOperator\grad{grad}
\DeclareMathOperator\rank{rank}
\DeclareMathOperator\Euc{E}
\DeclareMathOperator\ob{ob}
\DeclareMathOperator\diam{diam}
\DeclareMathOperator\rel{rel}
\newcommand{\LL}{{\mathrm{L}}}

\newcommand{\norm}[1]{\left\lVert#1\right\rVert}
\newcommand{\mat}[1]{\mathbold{#1}}
\newcommand{\cat}[1]{\mathsf{#1}}
\newcommand{\uline}{\underline{\hphantom{X}}}
\newcommand{\abs}[1]{\left|#1\right|}
\newcommand{\lie}[1]{\mathfrak{#1}}
\newcommand{\inn}[1]{\left\langle #1\right\rangle}
\newcommand{\partI}{\partial I}
\newcommand{\relhomo}{\rel\partI}

\usepackage{enumitem}

\setlist[enumerate]{nosep}

%\DeclareMathAlphabet\mathcal{OMS}{cmsy}{m}{n}

\newlength\stextwidth
\newcommand\makesamewidth[3][c]{%
  \settowidth{\stextwidth}{#2}%
  \makebox[\stextwidth][#1]{#3}%
}



\begin{document}

\frontmatter

\tableofcontents

\mainmatter

\chapter{黎曼积分}

\section{黎曼积分回顾}

\begin{definition}
  设 $f:[a,b]\to \mathbb{R}$ 是有界函数,$P$ 是 $[a,b]$ 的一个划分
  $x_0,\dots,x_n$,定义 \emph{黎曼下和} $L(f,P,[a,b])$ 和 \emph{黎曼上和}
  $U(f,P,[a,b])$ 为
  \begin{align*}
    L(f,P,[a,b])&=\sum_{i=1}^{n} (x_i-x_{i-1})\inf_{[x_{i-1},x_i]}f,\\
    U(f,P,[a,b])&=\sum_{i=1}^{n} (x_i-x_{i-1})\sup_{[x_{i-1},x_i]}f.
  \end{align*}
\end{definition}

\begin{lemma}[黎曼和不等式]
  设 $f:[a,b]\to \mathbb{R}$ 是有界函数,$P,P'$ 是 $[a,b]$ 的两个划分,
  且 $P$ 确定的点列是 $P'$ 确定的点列的一个子列(即 $P'$ 划分的更细),那么
  \[
    L(f,P,[a,b])\leq L(f,P',[a,b])\leq U(f,P',[a,b])\leq U(f,P,[a,b]).
  \]
\end{lemma}
\begin{proof}
  假设 $P$ 给出划分 $x_0,\dots,x_n$,$P'$ 给出划分 $x_0',\dots,x_N'$,
  那么对于每个 $1\leq i\leq n$,都存在整数 $k,m$ 使得
  $x_{i-1}=x_k'< x_{k+1}'<\cdots<x_{k+m}'=x_i$,
  所以
  \begin{align*}
    (x_{i-1}-x_i)\inf_{[x_{i-1},x_i]}f&=\sum_{j=1}^m (x_{k+j}'-x_{k+j-1}')
    \inf_{[x_{i-1},x_i]}f\\
    &\leq \sum_{j=1}^m (x_{k+j}'-x_{k+j-1}')\inf_{[x_{k+j-1}',x_{k+j}']}f,
  \end{align*}
  这就表明 $L(f,P,[a,b])\leq L(f,P',[a,b])$。对于上和有类似的估计。
\end{proof}

\begin{lemma}[黎曼下和小于黎曼上和]\label{lemma:lower sum leq upper sum}
  设 $f:[a,b]\to \mathbb{R}$ 是有界函数,$P,P'$ 是 $[a,b]$ 的两个划分,那么
  \[
    L(f,P,[a,b])\leq U(f,P',[a,b]).
  \]
\end{lemma}
\begin{proof}
  令 $P''$ 是 $P,P'$ 合并得到的划分,那么
  \[
    L(f,P,[a,b])\leq L(f,P'',[a,b])\leq U(f,P'',[a,b])\leq U
    (f,P',[a,b]).\qedhere
  \]
\end{proof}

\begin{definition}
  设 $f:[a,b]\to \mathbb{R}$ 是有界函数,定义 $f$ 的\emph{黎曼下积分}
  $L(f,[a,b])$ 和\emph{黎曼上积分} $U(f,[a,b])$ 分别为
  \begin{equation*}
    L(f,[a,b])=\sup_P L(f,P,[a,b]),\quad  
    U(f,[a,b])=\inf_P U(f,P,[a,b]).
  \end{equation*}
  其中上下确界取遍 $[a,b]$ 的所有划分 $P$。
\end{definition}

\begin{proposition}[黎曼下积分小于黎曼上积分]\label{prop:lower integral leq upper integral}
  设 $f:[a,b]\to \mathbb{R}$ 是有界函数,那么
  \[
    L(f,[a,b])\leq U(f,[a,b]).
  \]
\end{proposition}
\begin{proof}
  根据 \autoref{lemma:lower sum leq upper sum} 即得。
\end{proof}

\begin{definition}
  对于闭区间上的有界函数,如果其黎曼下积分和黎曼上积分相等,那么我们说
  它是\emph{黎曼可积的}。
  如果 $f:[a,b]\to \mathbb{R}$ 是黎曼可积的,那么定义\emph{黎曼积分}为
  \[
    \int_a^b f =L(f,[a,b])=U(f,[a,b]).
  \]
\end{definition}

\begin{example}\label{exa:integral of x2}
  计算 $f:[0,1]\to \mathbb{R}$,$f(x)=x^2$ 的黎曼积分。
\end{example}
\begin{solution}
  记 $P_n$ 是 $[0,1]$ 的划分 $0,1/n,2/n,\dots,n/n=1$,那么
  黎曼下和
  \[
    L(f,P_n,[0,1])=\sum_{i=1}^n\frac{1}{n}\inf_{[i/n,(i-1)/n]} f=
    \sum_{i=1}^n \frac{(i-1)^2}{n^3}=\frac{2n^2-3n+1}{6n^2},
  \]
  黎曼上和
  \[
    U(f,P_n,[0,1])=\sum_{i=1}^n\frac{1}{n}\sup_{[i/n,(i-1)/n]} f=
    \sum_{i=1}^n \frac{i^2}{n^3}=\frac{2n^2+3n+1}{6n^2},
  \]
  于是
  \[
    L(f,[0,1])\geq \sup_{n\geq 1} L(f,P_n,[0,1])=
    \sup_{n\geq 1}\frac{2n^2-3n+1}{6n^2}=\frac{1}{3},
  \]
  以及
  \[
    U(f,[0,1])\leq \inf_{n\geq 1} U(f,P_n,[0,1])=
    \inf_{n\geq 1} \frac{2n^2+3n+1}{6n^2}=\frac{1}{3},
  \]
  所以
  \[
    U(f,[0,1])\leq \frac{1}{3}\leq L(f,[0,1]),
  \]
  再结合 \autoref{prop:lower integral leq upper integral},所以
  $f$ 黎曼可积,并且
  \[
    \int_0^1 f=\frac{1}{3}.\qedhere
  \]
\end{solution}

\begin{theorem}[连续函数黎曼可积]
  闭区间上的实值连续函数是黎曼可积的。
\end{theorem}
\begin{proof}
  设 $f:[a,b]\to \mathbb{R}$ 连续,那么 $f$ 有界且一致连续。
  与 \autoref{exa:integral of x2} 的计算类似,我们寻求一系列等距划分 $P_n$
  进行计算即可。

  由于 $f$ 一致连续,所以任取 $\varepsilon>0$,都存在 $\delta>0$
  使得 $|x-y|<\delta$ 的时候有 $|f(x)-f(y)|<\varepsilon$。
  此时存在 $n$ 使得 $(b-a)/n<\delta$,记 $P_n$ 是 $[a,b]$ 的 $n$ 等分划分,
  那么
  \begin{align*}
    \abs{L(f,P_n,[a,b])-U(f,P_n,[a,b])}&\leq 
    \sum_{i=1}^n\frac{b-a}{n}\abs{\inf_{[x_{i-1},x_i]}f-\sup_{[x_{i-1},x_i]} f}\\
    &\leq  \sum_{i=1}^n\frac{b-a}{n}\varepsilon=(b-a)\varepsilon,
  \end{align*}
  这表明
  \[
    \abs{L(f,[a,b])-U(f,[a,b])}\leq \abs{L(f,P_n,[a,b])-U(f,P_n,[a,b])}\leq
    \varepsilon,
  \]
  所以 $L(f,[a,b])=U(f,[a,b])$,即 $f$ 黎曼可积。
\end{proof}

\begin{proposition}
  假设 $f:[a,b]\to \mathbb{R}$ 黎曼可积,那么
  \[
    (b-a)\inf_{[a,b]}f\leq \int_a^b f\leq (b-a)\sup_{[a,b]}f.
  \]
\end{proposition}
\begin{proof}
  取 $P$ 是平凡划分 $x_0=a,x_1=b$,那么 
  \[
    (b-a)\inf_{[a,b]}f=L(f,P,[a,b])\leq L(f,[a,b])=\int_a^b f
    .\qedhere
  \]
\end{proof}

\begin{problem}{}{}
  设 $f:[a,b]\to \mathbb{R}$ 是有界函数且对于某个 $[a,b]$ 的划分 $P$
  有
  \[
    L(f,P,[a,b])=U(f,P,[a,b]),
  \]
  证明 $f$ 在 $[a,b]$ 上是常值函数。
\end{problem}
\begin{proof}
  这表明我们有
  \[
    \sum_{i=1}^n (x_i-x_{i-1})\left(\sup_{[x_{i-1},x_i]}f-\inf_{[x_{i-1},x_i]}f\right)=0,
  \]
  这个求和项中每一项都大于等于 $0$,所以必须有
  \[
    \sup_{[x_{i-1},x_i]}f=\inf_{[x_{i-1},x_i]}f,
  \]
  所以 $f$ 在每个 $[x_{i-1},x_i]$ 上都是常值函数,即 $f$ 在 $[a,b]$ 上是常值函数。
\end{proof}

\begin{problem}{}{}
  设 $a\leq s<t\leq b$,定义 $f:[a,b]\to \mathbb{R}$ 为
  \[
    f(x)=\begin{cases}
      1 & \text{if $s<x<t$},\\
      0 & \text{otherwise}.
    \end{cases}
  \]
  证明 $f$ 在 $[a,b]$ 上黎曼可积并且 $\int_a^b f=t-s$。
\end{problem}
\begin{proof}
  设 $P$ 是 $[s,t]$ 的任意划分,那么 $P$ 附带 $a,b$ 构成 $[a,b]$
  的划分 $P'$,此时有
  \[
    L(f,P',[a,b])=t-s=U(f,P',[a,b]),
  \]
  所以 $L(f,[a,b])\geq t-s\geq U(f,[a,b])$,故 $f$ 黎曼可积且
  $\int_a^b f=t-s$。
\end{proof}

\begin{problem}{}{}
  设 $f:[a,b]\to \mathbb{R}$ 是有界函数,证明 $f$ 是黎曼可积的当且仅当
  对于任意 $\varepsilon>0$,存在 $[a,b]$ 的划分 $P$ 使得
  \begin{equation}\label{eq:prob1-3}
    U(f,P,[a,b])-L(f,P,[a,b])<\varepsilon.
  \end{equation}
\end{problem}
\begin{proof}
  若 $f$ 黎曼可积,那么 $L(f,[a,b])=U(f,[a,b])$。根据上下积分的定义,
  对于任意 $\varepsilon>0$,都存在划分 $P$ 使得
  \[
    L(f,P,[a,b])>L(f,[a,b])-\varepsilon/2,\quad U(f,P,[a,b])<U(f,[a,b])+\varepsilon/2,
  \]
  故 
  \[
    U(f,P,[a,b])-L(f,P,[a,b])<\varepsilon.
  \]

  反之,若对于任意 $\varepsilon>0$,都存在划分 $P$ 使得
  \eqref{eq:prob1-3} 式成立。由于 
  $L(f,[a,b])\geq L(f,P,[a,b])$ 以及 $U(f,[a,b])\leq U(f,P,[a,b])$,
  所以
  \[
    U(f,[a,b])-L(f,[a,b])\leq U(f,P,[a,b])-L(f,P,[a,b])<\varepsilon,
  \]
  即 $L(f,[a,b])=U(f,[a,b])$,$f$ 黎曼可积。
\end{proof}

\begin{problem}{}{}
  设 $f,g:[a,b]\to \mathbb{R}$ 黎曼可积,证明 $f+g$ 黎曼可积,并且
  \[
    \int_a^b(f+g)=\int_a^b f+\int_a^b g.
  \]
\end{problem}
\begin{proof}
  记 $\int_a^b f=I,\int_a^b g=J$。任取 $\varepsilon>0$,那么存在划分 $P$ 使得 
  \[
    I-\varepsilon<L(f,P,[a,b])\leq U(f,P,[a,b])<I+\varepsilon,
  \]
  同理存在划分 $P'$ 使得
  \[
    J-\varepsilon<L(g,P',[a,b])\leq U(g,P',[a,b])<J+\varepsilon,
  \]
  将划分 $P,P'$ 合并,得到划分 $P''$ 使得
  \begin{gather*}
    I-\varepsilon<L(f,P'',[a,b])\leq U(f,P'',[a,b])<I+\varepsilon,\\
    J-\varepsilon<L(g,P'',[a,b])\leq U(g,P'',[a,b])<J+\varepsilon.
  \end{gather*}
  于是 
  \begin{align*}
    L(f+g,P'',[a,b])&\geq L(f,P'',[a,b])+L(g,P'',[a,b])>I+J-2\varepsilon,\\
    U(f+g,P'',[a,b])&\leq U(f,P'',[a,b])+U(g,P'',[a,b])<I+J+2\varepsilon,
  \end{align*}
  这就表明 $f+g$ 可积且 $\int_a^b (f+g)=I+J$。
\end{proof}





\section{黎曼积分还不够好}

黎曼积分有下列三个主要缺陷:
\begin{itemize}
  \item 黎曼积分不能处理函数有太多不连续点的情况;
  \item 黎曼积分不能处理无界函数;
  \item 黎曼积分与极限的相容性不够好。
\end{itemize}

\begin{example}[一个不黎曼可积的函数]
  定义 $f:[0,1]\to \mathbb{R}$ 为
  \[
    f(x)=\begin{cases}
      1 & \text{$x$ is rational},\\
      0 & \text{$x$ is irrational}.
    \end{cases}
  \]
  对于任意子区间 $[a,b]\subseteq [0,1]$,都有
  \[
    \inf_{[a,b]}f=0,\quad \sup_{[a,b]}f=1.
  \]
  所以对于 $[0,1]$ 的任意划分 $P$ 都有 $L(f,P,[0,1])=0$ 以及
  $U(f,P,[0,1])=1$。所以 $L(f,[0,1])=0$ 以及 $U(f,[0,1])=1$,
  故 $f$ 不是黎曼可积的。

  这个例子令人困扰,因为直觉上有理数远少于无理数,所以 $f$ 的积分
  在某种意义上应该是 0,但是实际上黎曼积分确是没有定义的。
\end{example}

\begin{example}[黎曼积分不能处理无界函数]
  定义 $f:[0,1]\to \mathbb{R}$ 为
  \[
    f(x)=\begin{cases}
      \frac{1}{\sqrt{x}} & 0<x\leq 1,\\
      0 & x=0.
    \end{cases}
  \]
  如果 $x_0,x_1,\dots,x_n$ 是 $[0,1]$ 的划分,那么 $\sup_{[x_0,x_1]}f=\infty$,
  所以对于任意划分 $P$ 都有 $U(f,P,[0,1])=\infty$。

  但是,我们可以发现 $f$ 的图像面积应该是 $2$ 而不是 $\infty$。因为
  \[
    \lim_{a\downarrow  0}\int_a^1 f=\lim_{a\downarrow 0}(2-2\sqrt{a})=2.
  \]
  
\end{example}


\chapter{测度}

\section{$\mathbb{R}$ 上的外测度}

\subsection{外测度的定义}

黎曼积分来源于使用长方形面积的和来估计函数图像的面积,这些长方形
的高是函数在定义域的某个子区间上的值,宽是对应子区间的长度,即
$x_i-x_{i-1}$。

为了让更多的函数可以做积分,我们将把函数的定义域写为更加复杂的一些
子集的并,而不仅仅是使用子区间。我们将为这样的每个子集分配一个大小,
其大小是区间长度的某种扩展定义。例如,我们期望
$(1,3)\cup(7,10)$ 的大小是 $2+3=5$。

为 $\mathbb{R}$ 的子集分配大小是一件不平凡的任务。本章我们处理这个任务,并且
将其延伸到其他内容。在下一章,我们将看到本章发展的工具创造了丰富的积分理论。

我们先叙述我们期望给出的开区间的长度定义。

\begin{definition}
  一个开区间 $I$ 的\emph{长度} $\ell(I)$ 定义为
  \[
    \ell(I)=\begin{cases}
      b-a & \text{if $I=(a,b)$ for some $a,b\in \mathbb{R}$ with $a<b$},\\
      0 & \text{if $I=\emptyset$},\\
      \infty & \text{if $I=(-\infty,a)$ or $I=(a,\infty)$ for some $a\in \mathbb{R}$},\\
      \infty & \text{if $I=(-\infty,\infty)$}.
    \end{cases}
  \]
\end{definition}

假设 $A\subseteq \mathbb{R}$,$A$ 的大小至多也只能是一列并起来包含 $A$
的开区间的长度之和,所以将所有这样可能的和取下确界,有理由将其作为 $A$
的大小的定义,我们记为 $|A|$。

\begin{definition}
  集合 $A\subseteq \mathbb{R}$ 的\emph{外测度} $|A|$ 定义为
  \[
    |A|=\inf\left\{\sum_{k=1}^\infty\ell(I_k) \middle| \text{$I_1,I_2,\dots$ are open intervals such that $A\subseteq\bigcup_{k=1}^\infty I_k$}\right\}.
  \]
\end{definition}

外测度的定义涉及到无限和,对于无限和 $\sum_{k=1}^\infty t_k$,
其中 $t_i\in[0,\infty]$。如果某个 $t_k=\infty$,定义求和结果为 $\infty$。
否则定义为普通的级数求和。

\begin{example}[有限集的外测度为零]
  假设 $A=\{a_1,\dots,a_n\}$ 是有限集。任取 $\varepsilon>0$,$k\leq n$ 时定义
  $I_k=(a_k-\varepsilon,a_k+\varepsilon)$,$k>n$ 时定义 $I_k=\emptyset$,那么
  $\sum \ell(I_k)=2n\varepsilon$,所以 $|A|\leq 2n\varepsilon$,
  即 $|A|=0$。
\end{example}

\subsection{外测度的良好性质}

外测度有一些很好的性质,我们首先改进上一个例子的结果。

\begin{proposition}[可数集的外测度为零]
  $\mathbb{R}$ 的任意可数子集的外测度是零。
\end{proposition}
\begin{proof}
  设 $A=\{a_1,a_2,\dots\}$ 是可数集。任取 $\varepsilon>0$,
  定义 $I_k=(a_k-\varepsilon/2^k,a_k+\varepsilon/2^k)$,那么
  \[
    \sum_{k=1}^\infty \ell(I_k)=\sum_{k=1}^\infty\frac{\varepsilon}{2^{k-1}}
    =2\varepsilon,
  \]
  所以 $|A|\leq 2\varepsilon$,故 $|A|=0$。
\end{proof}

\begin{proposition}[外测度保序]
  假设 $A\subseteq B$,那么 $|A|\leq|B|$。
\end{proposition}
\begin{proof}
  设 $I_1,I_2,\dots$ 是覆盖 $B$ 的开区间,那么也覆盖 $A$,所以
  \[
    |A|\leq\sum_{k=1}^\infty\ell(I_k),
  \]
  对所有覆盖 $B$ 的一列开区间取下确界,所以 $|A|\leq|B|$。
\end{proof}

我们还希望 $\mathbb{R}$ 的子集的大小应该具有平移不变性,外测度
恰好满足这个性质。

\begin{proposition}[外测度具有平移不变性]
  设 $t\in \mathbb{R}$ 且 $A\subseteq \mathbb{R}$,那么 
  $|t+A|=|A|$。
\end{proposition}
\begin{proof}
  假设 $I_1,I_2,\dots$ 是一列覆盖 $A$ 的开区间,那么 $t+I_1,t+I_2,\dots$
  是覆盖 $t+A$ 的一列开区间,所以
  \[
    |t+A|\leq \sum_{k=1}^\infty\ell(t+I_k)=\sum_{k=1}^\infty \ell(I_k),
  \]
  这表明 $|t+A|\leq |A|$。

  另一边,我们有 $A=-t+(t+A)$,由此可得 $|A|\leq |t+A|$,所以
  $|t+A|=|A|$。
\end{proof}

\begin{proposition}[外测度的次可加性]
  设 $A_1,A_2,\dots$ 是 $\mathbb{R}$ 的一列子集,那么
  \[
    \left|\bigcup_{k=1}^\infty A_k\right|\leq\sum_{k=1}^\infty|A_k|.
  \]
\end{proposition}
\begin{proof}
  若其中某个 $|A_k|=\infty$,那么不等式显然成立。下面假设所有 $|A_k|<\infty$。

  任取 $\varepsilon>0$,对于每个 $k$,令 $I_{1,k},I_{2,k},\dots$ 是一列
  并集包含 $A_k$ 的开区间,并且满足
  \[
    \sum_{j=1}^\infty \ell(I_{j,k})\leq\frac{\varepsilon}{2^k}+|A_k|,
  \]
  所以
  \begin{equation*}
    \sum_{k=1}^\infty\sum_{j=1}^\infty\ell(I_{j,k})\leq\varepsilon+
    \sum_{k=1}^\infty|A_k|.
  \end{equation*}
  那么 
  \[
    \bigcup_{k=1}^\infty A_k\subseteq \bigcup_{k,j}I_{j,k},
  \]
  所以
  \[
    \abs{\bigcup_{k=1}^\infty A_k}\leq \sum_{k,j}\ell(I_{j,k})
    \leq \varepsilon+\sum_{k=1}^\infty |A_k|,
  \]
  $\varepsilon$ 的任意性即表明 $\left|\bigcup_{k=1}^\infty A_k\right|\leq\sum_{k=1}^\infty|A_k|$。
\end{proof}

\subsection{有界闭区间的外测度}

我们将证明如果 $a<b$,那么 $[a,b]$ 有外测度 $b-a$。实际上,
如果 $\varepsilon>0$,那么 $(a-\varepsilon,b+\varepsilon),\emptyset,\emptyset,\dots$
是并集包含 $[a,b]$ 的开区间,所以 $\abs{[a,b]}\leq b-a+2\varepsilon$,于是我们得到
\[
  \abs{[a,b]}\leq b-a.
\]

\begin{proposition}
  设 $a<b$,那么 $\abs{[a,b]}=b-a$。
\end{proposition}
\begin{proof}
  我们只需要说明 $\abs{[a,b]}\geq b-a$。设 $I_1,I_2,\dots$ 是一列开区间使得
  $[a,b]\subseteq\bigcup_{k=1}^\infty I_k$,由于 $[a,b]$ 是紧集,所以存在
  $n$ 使得
  \[
    [a,b]\subseteq I_1\cup \cdots \cup I_n,
  \]
  下面我们证明 $\sum_{k=1}^n\ell(I_k)\geq b-a$,从而说明 $\sum_{k=1}^\infty\ell(I_k)$,
  即 $\abs{[a,b]}\geq b-a$。

  对 $n$ 进行归纳,当 $n=1$ 的时候,那么 $[a,b]\subseteq I_1$ 显然表明
  $\ell(I_1)\geq b-a$。假设 $n>1$ 的时候成立,那么在 $n+1$ 的时候有
  \[
    [a,b]\subseteq I_1\cup\cdots\cup I_n\cup I_{n+1},
  \]
  那么 $b$ 至少在其中一个 $I_1,\dots,I_{n+1}$ 中,通过重新编排顺序,不妨假设
  $b\in I_{n+1}$。设 $I_{n+1}=(c,d)$,如果 $c\leq a$,那么 $\ell(I_{n+1})\geq b-a$,
  即结论成立。若 $a<c<b<d$,那么
  \[
    [a,c]\subseteq I_1\cup\cdots \cup I_n,
  \]
  根据归纳假设,有 $\sum_{k=1}^n\ell(I_k)\geq c-a$,所以
  \[
    \sum_{k=1}^{n+1}\ell(I_k)\geq (c-a)+\ell(I_{n+1})
    =d-a\geq b-a.\qedhere
  \]
\end{proof}

\begin{corollary}
  设 $a<b$,那么 $\abs{(a,b)}=\abs{(a,b]}=\abs{[a,b)}=b-a$。
\end{corollary}
\begin{proof}
  由于 $(a,b)\cup\{a\}\cup\{b\}=[a,b]$,所以 
  \[
    \abs{[a,b]}\leq \abs{(a,b)}+\abs{\{a\}}+\abs{\{b\}}=\abs{(a,b)},
  \]
  $(a,b)\subseteq [a,b]$ 又表明 $\abs{(a,b)}\leq \abs{[a,b]}$。
  对于其他的情况同理。
\end{proof}

\subsection{外测度没有可加性}

我们已经看到外测度有很多良好的性质,现在我们发现外测度的一个不尽人意的性质。

如果外测度可以完美地为集合分配大小的概念,那么两个不相交集合的并集的外测度
理应等于两个集合的外测度之和,但是,下面的结果表明外测度并没有这种性质。

\begin{theorem}\label{thm:Vitali set}
  存在 $\mathbb{R}$ 的两个不相交子集 $A,B$ 使得
  \[
    \abs{A\cup B}\neq\abs{A}+\abs{B}.
  \]
\end{theorem}
\begin{remark}
  这样的两个子集是十分“病态的”,因为可以证明,只要 $A,B$ 之间的距离大于零,
  那么一定有 $\abs{A\cup B}=\abs{A}+\abs{B}$ (见 Stein 的《Real Analysis》),
  实际上,该定理的证明需要使用选择公理。
\end{remark}
\begin{proof}
  对于 $a\in [-1,1]$,记 $\bar a$ 为等价类 $a+\mathbb{Q}$,即
  \[
    \bar a=\{c\in [-1,1]\,|\, a-c\in \mathbb{Q}\}.
  \]

  如果 $a,b\in[-1,1]$ 且 $\bar a\cap\bar b\neq\emptyset$,那么存在 $c\in [-1,1]$
  使得 $c-a\in \mathbb{Q}$ 以及 $c-b\in \mathbb{Q}$,所以 $a-b=(a-c)-(b-c)\in \mathbb{Q}$,
  所以 $\bar a=\bar b$。

  显然 $[-1,1]=\bigcup_{a\in[-1,1]}\bar a$,把这些子集搜集起来:
  \[
    \{\bar a\,|\, a\in[-1,1]\},
  \]
  在每个 $\bar a$ 中取一个代表元,放到一个新的集合中,记为 $V$。
  换句话说,对于每个 $a\in [-1,1]$,$V\cap\bar a$ 恰有一个元素。
  注意,这里 $V$ 的构造\emph{使用了选择公理}。

  令 $r_1,r_2,\dots$ 是一列不同的有理数使得
  \[
    [-2,2]\cap \mathbb{Q}=\{r_1,r_2,\dots\},
  \]
  那么
  \[
    [-1,1]\subseteq\bigcup_{k=1}^\infty (r_k+V),
  \]
  这是因为如果 $a\in [-1,1]$,设 $v\in V\cap \bar a$,那么 $a-v\in \mathbb{Q}$,
  这就表明存在某个 $k$ 使得 $a=r_k+v\in r_k+V$。

  根据外测度的次可加性和平移不变性,有
  \[
    2=\abs{[-1,1]}\leq\sum_{k=1}^\infty \abs{r_k+V}=\sum_{k=1}^\infty\abs{V},
  \]
  这表明 $\abs{V}>0$。

  注意到 $r_1+V,r_2+V,\dots$ 是互不相交的。若 $t\in (r_k+V)\cap (r_j+V)$,
  那么 $t-r_k\in V$ 以及 $t-r_j\in V$,于是 $(t-r_k)-(t-r_j)=r_j-r_k\in \mathbb{Q}$,
  根据 $V$ 的构造,有 $t-r_k=t-r_j$,即 $r_k=r_j$。

  令 $n\geq 1$,因为 $V\subseteq [-1,1]$ 以及 $r_k\in [-2,2]$,所以
  \[
    \bigcup_{k=1}^n(r_k+V)\subseteq [-3,3],
  \]
  所以
  \[
    \abs{\bigcup_{k=1}^n(r_k+V)}\leq 6.
  \]  
  但是
  \[
    \sum_{k=1}^n\abs{r_k+V}=\sum_{k=1}^n\abs{V}=n\abs{V},
  \]
  这暗示我们选取足够大的 $n$ 使得 $n\abs{V}>6$,所以 
  \[
    \abs{\bigcup_{k=1}^n(r_k+V)}<\sum_{k=1}^n\abs{r_k+V}.
  \]

  也就是说,如果对于所有不相交的子集 $A,B$ 有 $\abs{A\cup B}=\abs{A}+\abs{B}$,
  那么归纳可得对于不相交子集 $A_1,\dots,A_n$ 有 $\abs{\bigcup_{k=1}^n A_k}=\sum_{k=1}^n\abs{A_k}$,
  这与上面的严格不等号矛盾。所以存在不相交子集 $A,B$ 使得 $\abs{A\cup B}\neq\abs{A}+\abs{B}$。
\end{proof}

\section{可测空间和函数}

上一节的结果表明外测度不是可加的。那么这个漏洞能否使用其他的某种“测量方式”
弥补?下面的结果表明不存在满足所有期望属性的测量方式。

\begin{theorem}[不可能对 $\mathbb{R}$ 的所有子集都赋予一个大小的概念]\label{thm:noexistence of size}
  不存在 $\mathbb{R}$ 的幂集上的函数 $\mu$ 同时满足下面的性质:
  \begin{enumerate}
    \item $\mu$ 是到 $[0,\infty]$ 的函数。
    \item 对于每个开区间 $I$ 有 $\mu(I)=\ell(I)$。
    \item 对于不相交子集 $A_1,A_2,\dots$ 有 $\mu\left(\bigcup_{k=1}^\infty A_k\right)=\sum_{k=1}^\infty\mu(A_k)$。
    \item 对于每个 $A\subseteq \mathbb{R}$ 和 $t\in \mathbb{R}$ 有 $\mu(t+A)=\mu(A)$。
  \end{enumerate}
\end{theorem}
\begin{proof}
  假设 $\mu$ 同时满足上面的性质。如果 $A\subseteq B$,那么 $\mu(A)\leq\mu(B)$,
  这是因为对于 $A,B\setminus A,\emptyset,\emptyset,\dots$,所以 
  \[
    \mu(B)=\mu(A)+\mu(B\setminus A)+0+0+\cdots=\mu(A)+\mu(B\setminus A)\geq \mu(A).
  \]

  如果 $a<b$,那么对于任意 $\varepsilon>0$ 有 $(a,b)\subseteq [a,b]\subseteq (a-\varepsilon,b+\varepsilon)$,
  所以 $b-a\leq \mu([a,b])\leq b-a+2\varepsilon$,所以 $\mu([a,b])=b-a$。

  如果 $A_1,A_2,\dots$ 是一列子集,那么 $A_1,A_2\setminus A_1,A_3\setminus (A_1\cup A_2),\dots$
  是一列不相交的子集,并且它们的并集是 $\bigcup_{k=1}^\infty A_k$,所以
  \begin{align*}
    \mu\left(\bigcup_{k=1}^\infty A_k\right)&=\mu\bigl(A_1\cup(A_2\setminus A_1)\cup(A_3\setminus(A_1\cup A_2))\cup\cdots\bigr)\\
    &=\mu(A_1)+\mu(A_2\setminus A_1)+\mu(A_3\setminus (A_1\cup A_2))+\cdots \\
    &\leq\sum_{k=1}^\infty\mu(A_k).
  \end{align*}

  这意味着这样的函数 $\mu$ 实际上满足外测度的所有性质,那么采用
  \autoref{thm:Vitali set} 相同的构造,即可发现存在不相交子集 $A,B$
  使得 $\mu(A\cup B)\neq\mu(A)+\mu(B)$,这与性质 (3) 矛盾。
\end{proof}

\subsection{$\sigma$-代数}

上面的结果表明我们必须放弃把区间长度的概念延申到 $\mathbb{R}$ 的每个子集上。
但是,我们不能放弃 \autoref{thm:noexistence of size} 的性质 (2),因为我们希望
区间的大小是它们的长度。我们也不能放弃 \autoref{thm:noexistence of size} 的
性质 (3),因为可数可加性需要用于证明与极限相关的定理。我们还不能放弃
\autoref{thm:noexistence of size} 的性质 (4),因为不满足平移不变性的大小概念
与我们的直觉严重不符。

这表明我们除开放宽 \autoref{thm:noexistence of size} 的性质 (1) 之外别无选择,
即必须存在一些子集使得它们没有大小的概念。大量的经验表明为了允许取极限的操作,
可定义长度的子集族必须满足对补和可数并操作封闭,于是我们给出下面的定义。

\begin{definition}
  设 $X$ 是一个集合,$\mathcal{S}$ 是 $X$ 的一个子集族,如果 $\mathcal{S}$
  满足:
  \begin{enumerate}
    \item $\emptyset\in \mathcal{S}$;
    \item 如果 $E\in \mathcal{S}$,那么 $X\setminus E\in \mathcal{S}$;
    \item 如果 $E_1,E_2,\dots$ 是 $\mathcal{S}$ 中的一列子集,
    那么 $\bigcup_{k=1}^\infty E_k\in \mathcal{S}$,
  \end{enumerate}
  那么我们说 $\mathcal{S}$ 是 $X$ 上的一个\emph{$\sigma$-代数}。
\end{definition}



\end{document}
