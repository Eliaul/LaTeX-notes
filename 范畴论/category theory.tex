\documentclass[fontset=none]{Notes}

\makeatletter
\DeclareRobustCommand{\em}{%
  \@nomath\em \if b\expandafter\@car\f@series\@nil
  \normalfont \else \bfseries \fi}
\makeatother

\usepackage{tikz-cd,wrapstuff}
\usepackage{fixdif,siunitx,tikz,nicematrix,tabularray}
\usetikzlibrary{matrix,calc}

\ProvidesFile{font.def}

\setCJKmainfont{Source Han Serif SC}[
  UprightFont=*-Regular,
  BoldFont=*-Bold,
  ItalicFont=HYKaiTi S,
  ItalicFeatures={Scale=1.1}
]
\newCJKfontfamily[zhsong]\songti{Source Han Serif SC}[
  UprightFont=*-Regular,
  BoldFont=*-Bold,
  ItalicFont=HYKaiTi S,
  ItalicFeatures={Scale=1.1}
]
\setCJKsansfont{Source Han Sans SC}[
  UprightFont=*-Regular,
  BoldFont=*-Bold
]
\newCJKfontfamily[zhhei]\heiti{Source Han Sans SC}[
  UprightFont=*-Regular,
  BoldFont=*-Bold
]
\setCJKmonofont{HYFangSong S}[
  BoldFont=*,
  ItalicFont=*,
  BoldItalicFont=*
]
\newCJKfontfamily[zhfs]\fangsong{HYFangSong S}[
  BoldFont=*,
  ItalicFont=*,
  BoldItalicFont=*
]
\newCJKfontfamily[zhkai]\kaishu{HYKaiTi S}[
  BoldFont=*,
  ItalicFont=*,
  BoldItalicFont=*
]

\setmainfont{texgyretermes}[
  Extension=.otf,
  UprightFont=*-regular,
  BoldFont=*-bold,
  ItalicFont=*-italic,
  BoldItalicFont=*-bolditalic,
  SlantedFont=*-italic
]
%\setmathrm{texgyretermes}[
%  Extension=.otf,
%  UprightFont=*-regular,
%  BoldFont=*-bold,
%  ItalicFont=*-italic,
%  BoldItalicFont=*-bolditalic,
%  SlantedFont=*-italic
%]
\setsansfont{Cantarell}[
  UprightFont=* Regular,
  ItalicFont=* Italic,
  BoldFont=* Bold,
  BoldItalicFont=* Bold Italic,
  SmallCapsFont=Alegreya Sans SC
]
\setmonofont{Ubuntu Mono}[
  UprightFont=*,
  ItalicFont=* Italic,
  BoldFont=* Bold,
  BoldItalicFont=* Bold Italic
]
%\setmathfont{texgyretermes-math.otf}
%\setmathfont[range={\mathcal,\mathbfcal,\mathfrak},StylisticSet=1]{XITSMath-Regular.otf}
%\setmathfont[range={\mathbb}]{KpMath-Sans.otf}



\usepackage[subscriptcorrection,nofontinfo,mtpbb,mtpfrak]{mtpro2}

\tikzcdset{
  arrow style=tikz,
  diagrams={>={Straight Barb[scale=0.8]}}
}

\allowdisplaybreaks[1]

\newlength{\mymathln}
\newcommand{\aligninside}[2]{
  \settowidth{\mymathln}{#2}
  \mathmakebox[\mymathln]{#1}
}
\newenvironment{arr}[1][]{%
  $\begin{tikzcd}[cramped,#1]
}{\end{tikzcd}$}

\DeclareMathOperator\Spec{Spec}
\DeclareMathOperator\im{im}
\DeclareMathOperator\nil{nil}
\DeclareMathOperator\rad{rad}
\DeclareMathOperator\Ann{Ann}
\DeclareMathOperator\Max{Max}
\DeclareMathOperator\GL{GL}
\DeclareMathOperator\End{End}
\DeclareMathOperator\Int{Int}
\DeclareMathOperator\Tor{Tor}
\DeclareMathOperator\Frac{Frac}
\DeclareMathOperator\Tr{Tr}
\DeclareMathOperator\Hom{Hom}
\DeclareMathOperator\Leb{Leb}
\DeclareMathOperator\supp{supp}
\DeclareMathOperator\Id{Id}
\DeclareMathOperator\rk{rank}
\DeclareMathOperator\var{var}
\DeclareMathOperator\card{card}
\DeclareMathOperator\coker{coker}
\DeclareMathOperator\ob{ob}

\newcommand{\cat}[1]{\mathsf{#1}}
\newcommand{\opcat}[1]{\mathsf{#1}^{\mathrm{op}}}

\usepackage{enumitem}

\setlist[enumerate]{
  nosep,
  label=(\alph*),
  itemindent=\labelwidth,
  leftmargin=*,
  listparindent=\parindent
}

%\DeclareMathAlphabet\mathcal{OMS}{cmsy}{m}{n}

\newlength\stextwidth
\newcommand\makesamewidth[3][c]{%
  \settowidth{\stextwidth}{#2}%
  \makebox[\stextwidth][#1]{#3}%
}



\begin{document}

\frontmatter

\tableofcontents

\mainmatter

\chapter{范畴、函子和自然变换}

\section{范畴}

\begin{definition}
  一个\emph{范畴} $\cat{A}$ 由以下内容组成:
  \begin{itemize}[nosep]
    \item 一族\emph{对象} $\ob(\cat A)$;
    \item 对于每个 $A,B\in\ob(\cat A)$,存在一族从 $A$ 到 $B$ 的\emph{态射}
    $\Hom(A,B)$;
    \item 对于每个 $A,B,C\in\ob(\cat A)$,有一个\emph{复合映射}:
    \[
      \Hom(A,B)\times \Hom(B,C)\to \Hom(A,C),
      \quad (g,f)\mapsto g\circ f;
    \]
    \item 对于每个 $A\in\ob(\cat A)$,存在 $A$ 上的\emph{单位}
    $1_A\in\Hom(A,A)$,
  \end{itemize}
  此外态射需要满足:
  \begin{itemize}[nosep]
    \item 对于每个 $f\in\Hom(A,B)$,$g\in\Hom(B,C)$ 与 $h\in\Hom(C,D)$,
    有 $(h\circ g)\circ f=h\circ(g\circ f)$;
    \item 对于每个 $f\in \Hom(A,B)$,有 $f\circ 1_A=f=1_B\circ f$。
  \end{itemize}
\end{definition}

\begin{remark}
  我们通常使用 $A\in \cat A$ 表示 $A\in\ob(\cat A)$,
  $f:A\to B$ 或者
  \begin{arr}[sep=small]
    A\arrow[r,"f"] & B
  \end{arr}
  表示 $f\in\Hom(A,B)$,$gf$ 表示 $g\circ f$。
\end{remark}

\begin{example}[数学结构的范畴]%
  \begin{enumerate}
    \item 集合范畴 $\cat{Set}$。对象为集合,给定集合 $A,B$,
    $A$ 到 $B$ 的态射就是集合意义下 $A$ 到 $B$ 的映射,态射的复合
    即映射的复合,此时单位 $1_A$ 就是恒等映射 $A\to A$。
    \item 群范畴 $\cat{Grp}$。对象为群,态射为群同态。
    \item 环范畴 $\cat{Ring}$。对象为环,态射为环同态。
    \item 给定域 $k$,有 $k$ 上的向量空间范畴 $\cat{Vect}_k$,
    对象是向量空间,态射是线性映射。
    \item 拓扑空间范畴 $\cat{Top}$。对象是拓扑空间,态射
    是连续映射。
  \end{enumerate}
\end{example}

\begin{definition}
  对于态射 $f:A\to B$,如果存在态射 $g:B\to A$ 使得
  $gf=1_A$ 以及 $fg=1_B$,那么我们说 $f$ 是\emph{同构}。
  此时我们说 $g$ 是 $f$ 的\emph{逆},记为 $g=f^{-1}$。
\end{definition}

如果 $A$ 到 $B$ 之间存在一个同构,那么我们说 $A$ 和
$B$ \emph{同构},记作 $A\cong B$。

\begin{example}
  $\cat{Set}$ 中的同构等同于双射。当然,这句话在逻辑上其实是在
  表明:一个映射具有双边逆映射当且仅当其是双射。
\end{example}

\begin{example}
  $\cat{Grp}$ 中的同构等同于群同构。在一些抽象代数教材中,群同构
  被定义为双射的群同态,如果是这样,那么实际上需要证明:
  双射的群同态的逆映射也是群同态。类似地,
  $\cat{Ring}$ 中的同构等同于环同构。
\end{example}

\begin{example}
  $\cat{Top}$ 中的同构是同胚。与 $\cat{Grp}$ 或者 $\cat{Ring}$
  不同的是,$\cat{Top}$ 中双射的连续映射不一定是同构,即
  连续映射的逆映射可以是连续的。下面是一个经典的例子:
  考虑映射 $f:[0,1)\to \mathbb{S}^1$ 为 $f(t)=e^{2\pi it}$,
  $f$ 是双射的连续映射,但是 $f$ 不是同胚,因为 $[0,1)$
  不是紧的,但是 $\mathbb{S}^1$ 是紧的。
\end{example}

目前为止范畴的例子中对象都是具有某些结构的集合(例如群结构、拓扑结构或者
只有集合结构),态射都是保持这些结构的映射(群同态、连续映射或者普通的映射)。
但是,并非所有的范畴都长成这样,实际上范畴的含义相当广泛,
其对象也不一定是“配备了额外结构的集合”,因此在一般的范畴中,
谈论对象的“元素”是没有意义的。类似地,在一般意义上的范畴中,
态射也不必是集合之间的映射。总的来说:
\emph{范畴的对象不必类似于集合,态射也不必类似于映射}。
下面的例子解释了这些观点。

\begin{example}[范畴作为数学结构]
  \begin{enumerate}
    \item 一个范畴可以通过直接说出对象、态射、复合和单位来指定。
    例如空范畴 $\emptyset$,其没有任何对象或者态射。
    范畴 $\cat 1$ 由一个对象和唯一的单位态射构成。
    也可以构造一个只有两个对象的范畴:
    \[
      \begin{tikzcd}
        \bullet\arrow[r] & \bullet,
      \end{tikzcd}
    \]
    这个范畴只有两个对象,每个对象有一个单位态射,两个对象之间有唯一的一个
    非单位态射。在这些例子中,我们并没有将对象视为一个类似集合的东西,
    也没有将态射视为一个映射,此时态射更多的类似于一个抽象的“箭头”。
    \item 有些范畴中的态射只有单位态射,即任意两个不同的对象之间
    都不存在任意态射,这样的范畴被称为\emph{离散范畴}。离散范畴
    是最极端的情况,即不同的对象之间完全隔离。
    \item 一个群本质上和只有一个对象且所有态射都是同构的范畴是一样的。
    我们来说明这一点。考虑只有一个对象的范畴 $\cat A$,记这个对象为
    $A$,那么范畴 $\cat A$ 的态射只有 $\Hom(A,A)$。我们要求
    $\cat A$ 中的每个态射都是同构,也就是说每个 $f\in\Hom(A,A)$
    都有一个逆 $g\in\Hom(A,A)$ 使得 $fg=1_A=gf$。
    实际上这样的范畴 $\cat A$ 和群没有本质区别,对应关系如下所示。
    {

      \vspace*{5pt}
      \centering
      \begin{tblr}{
        colspec={X[c]X[c]}
      }
      范畴 $\cat A$ & 群 $G$\\
      态射 $f\in \Hom(A,A)$ & 元素 $g\in G$ \\
      态射的复合 $\circ$ & 元素的乘法 $\cdot$ \\
      单位态射 $1_A$ & 单位元 $1\in G$
      \end{tblr} 
      \vspace*{5pt}

    }
    \item 在上一个例子中,由于态射的逆不一定是必须的,所以考虑
    “没有逆元的群”也是必要的,这被称为\emph{幺半群}。具有
    一个对象的范畴本质上和幺半群是相同的,其论证完全仿照上例。
    \item 一个\emph{预序}指的是满足自反性和传递性的二元关系。
    一个\emph{预序集} $(S,\leq)$ 指的是一个集合 $S$ 配备预序 $\leq$。
    例如 $S=\mathbb{Z}$,$\leq$ 是整除关系。

    一个预序集可以被视为范畴 $\cat A$,其中对于每个 $A,B\in\cat A$,
    至多只有一个从 $A$ 到 $B$ 的态射。此时我们用 $A\leq B$ 来表示存在
    态射 $A\to B$。因为 $\cat A$ 是范畴,所以 $A\leq B\leq C$
    表明 $A\leq C$。由于始终存在 $A\to A$ 的态射(即 $1_A$),所以
    $A\leq A$。所以 $\cat A$ 实际上就表示了一族对象,配备了
    一个具备自反性和传递性的二元关系。

    一个\emph{偏序}指的是满足 $A\leq B,B\leq A\Rightarrow A=B$
    的预序 $\leq$。等价地说,即上述范畴 $\cat A$ 中若 $A\cong B$
    能推出 $A=B$。
  \end{enumerate}
\end{example}

\begin{example}[反范畴]
  每个范畴 $\cat A$ 都有一个\emph{反范畴} $\opcat A$,
  其定义为将 $\cat A$ 的所有箭头反向。准确的说,
  有 $\ob(\opcat A)=\ob(\cat A)$,对于任意对象 $A,B$,有
  $\Hom_{\opcat A}(A,B)=\Hom_{\cat A}(B,A)$。此时
  $\opcat A$ 中若有
  \begin{arr}[sep=small]
    A\arrow[r,"f"] & B\arrow[r,"g"] & C
  \end{arr},那么意味着在 $\cat A$ 中有
  \begin{arr}[sep=small]
    A & B\arrow[l,"f"'] & C\arrow[l,"g"']
  \end{arr}。
\end{example}

\begin{example}[积范畴]
  给定两个范畴 $\cat A$ 和 $\cat B$,构造\emph{积范畴}
  $\cat A\times\cat B$,满足
  \begin{align*}
    \ob(\cat A\times\cat B)&=\ob(A)\times\ob(\cat B),\\
    \Hom\bigl((A,B),(A',B')\bigr)&=\Hom(A,A')\times\Hom(B,B').
  \end{align*}
  即态射 $(A,B)\to(A',B')$ 是一对 $(f,g)$,其中 $f:A\to A'$
  是 $\cat A$ 中的态射,$g:B\to B'$ 是 $\cat B$ 中的态射。
\end{example}

\section{函子}

\begin{definition}
  令 $\cat A,\cat B$ 是范畴,\emph{函子} $F:\cat A\to \cat B$
  由以下内容组成:
  \begin{itemize}[nosep]
    \item 映射 $\ob(\cat A)\to \ob(B)$,记作 $A\mapsto F(A)$;
    \item 对于每个 $A,A'\in\cat A$,有映射 
    $\Hom(A,A')\to\Hom(F(A),F(A'))$,记作 $f\mapsto F(f)$。
  \end{itemize}
  还要满足:
  \begin{itemize}[nosep]
    \item 当 $\cat A$ 中有
    \begin{arr}[sep=small]
      A\arrow[r,"f"] & A'\arrow[r,"f'"] & A''
    \end{arr}
    的时候,有 $F(f'\circ f)=F(f')\circ F(f)$;
    \item 对于任意 $A\in\cat A$,有 $F(1_A)=1_{F(A)}$。
  \end{itemize}
\end{definition}

\begin{remark}
  我们已经熟悉将一种结构和保持结构的映射视为一个范畴(例如 $\cat{Grp}$,
  $\cat{Ring}$ 等等),实际上,这种思想也可以应用于范畴和函子:
  如果对象为范畴,态射是函子,这也构成一个范畴,我们记为 $\cat{CAT}$。
  这一论断来源于函子的复合性,即给定函子
  \begin{arr}[sep=small]
    \cat{A}\arrow[r,"F"] & \cat{B}\arrow[r,"G"] & \cat{C}
  \end{arr},那么存在函子
  \begin{arr}
    \cat{A}\arrow[r,"G\circ F"] & \cat{C}
  \end{arr},$G\circ F$ 由自然的方式定义。此外,对于任意范畴
  $\cat A$,存在单位态射 $1_{\cat A}:\cat A\to \cat A$。
\end{remark}

\begin{example}
  最简单的函子的例子是\emph{遗忘函子}(这是一个非正式的用语,没有准确的定义)。
  下面是一些例子:
  \begin{enumerate}
    \item 存在一个函子 $U:\cat{Grp}\to\cat{Set}$ 定义如下:
    如果 $G$ 是一个群,那么 $U(G)$ 是 $G$ 的底集合。如果 $f:G\to H$
    是群同态,那么 $U(f):U(G)\to U(H)$ 是将 $f$ 视为集合间的映射。
    所以 $U$ 忘掉了群的群结构以及群同态作为同态的结构。
    \item 类似的,存在函子 $\cat{Ring}\to\cat{Set}$ 忘掉环结构
    以及函子 $\cat{Vect}_k\to\cat{Set}$ 忘掉向量空间结构。
    \item 遗忘函子不必忘掉\emph{所有的}结构。例如,令 $\cat{Ab}$
    是交换群范畴。那么存在函子 $\cat{Ring}\to\cat{Ab}$ 忘掉环
    的所有结构,仅仅记住了环的加法群结构。或者,令 $\cat{Mon}$
    是幺半群范畴,那么存在函子 $\cat{Ring}\to\cat{Mon}$
    忘掉加法结构,仅仅记住环的乘法幺半群结构。
    \item 存在一个包含函子 $U:\cat{Ab}\to\cat{Grp}$,将交换群
    $A$ 送到 $U(A)=A$ 以及交换群同态 $f$ 送到 $U(f)=f$。
    这忘掉了交换群是交换的。
  \end{enumerate}
\end{example}

\begin{example}
  \emph{自由函子}在某种意义上是遗忘函子的对偶概念。
  \begin{enumerate}
    \item 给定一个集合 $S$,我们可以构造一个 $S$ 上的\emph{自由群}
    $F(S)$。$F(S)$ 的元素是一些\emph{字}的形式表达,例如
    $x^{-4}yx^2zy^{-3}$,其中 $x,y,z\in S$。如果两个字可以通过
    约化得到同一个字,那么说这两个字相等,例如
    $x^3xy,x^4y,x^2y^{-1}yx^2y$ 表示 $F(S)$ 中同一个元素。
    两个字的乘法定义为拼接运算,例如 $x^{-4}yx\cdot xzy^{-3}=x^{-4}yx^2zy^{-3}$。

    这个构造给每个集合 $S$ 都分配了一个群 $F(S)$。实际上,$F$ 是一个函子:
    任意集合的映射 $f:S\to S'$ 都可以提升为一个群同态 $F(f):F(S)\to F(S')$。
    例如,定义映射 $f:\{w,x,y,z\}\to\{u,v\}$ 为 $f(w)=f(x)=f(y)=u$
    以及 $f(z)=v$,那么这给出一个同态
    $F(f)$,将 $x^{-4}yx^2zy^{-3}\in F(\{w,x,y,z\})$ 
    送到 $u^{-4}uu^2vu^{-3}=u^{-1}vu^{-3}\in F(\{u,v\})$。
    \item 类似的,我们可以构造集合 $S$ 上的一个自由交换环 $F(S)$,
    给出一个从 $\cat{Set}$ 到交换环范畴 $\cat{CRing}$ 
    的函子 $F$。实际上,$F(S)$ 就是 $\mathbb{Z}$ 上的以 $x_s\ (s\in S)$
    为未定元的多项式环。例如,如果 $S$ 有两个元素,那么
    $F(S)\simeq \mathbb{Z}[x,y]$。
    \item 我们也可以构造一个集合上的自由向量空间。固定一个域 $k$。
    自由函子 $F:\cat{Set}\to\cat{Vect}_k$ 将 $F(S)$ 作为有基 $S$
    的向量空间。粗略地说,$F(S)$ 是所有的 $S$ 中元素的形式 $k$-线性组合,
    即表达式
    \[
      \sum_{s\in S}\lambda_ss\quad \lambda_s\in k,  
    \]
    其中只有有限多个 $s$ 使得 $\lambda_s\neq 0$。
    $F(S)$ 的元素可以相加:
    \[
      \sum_{s\in S}\lambda_ss+\sum_{s\in S}\mu_ss=\sum_{s\in S}
      (\lambda_s+\mu_s)s.  
    \]
    也可以做标量乘法:
    \[
      c\sum_{s\in S}\lambda_ss=\sum_{s\in S}(c\lambda_s)s\quad c\in k.  
    \]
    通过这种方式,$F(S)$ 成为一个向量空间。

    为了避免这种“形式定义”,我们也可以将 $F(S)$ 定义为所有使得集合
    $\{s\in S\,|\, \lambda(s)\ne 0\}$ 为有限集的映射
    $\lambda:S\to k$ 的集合。那么加法定义为
    \[
      (\lambda+\mu)(s)=\lambda(s)+\mu(s).  
    \]
    乘法定义为 $(c\lambda)(s)=c\lambda(s)$。
  \end{enumerate}
\end{example}

\begin{example}
  任取多项式方程组,例如
  \begin{align}
    2x^2+y^2-3z^2&=1,\\
    x^3+x&=y^2,
  \end{align}
  给出了一个函子 $F:\cat{CRing}\to\cat{Set}$。对于每个交换环
  $A$,令 $F(A)$ 是方程组的零点集。此时若 $f:A\to B$ 是环同态
  且 $(x,y,z)\in F(A)$,那么有 $(f(x),f(y),f(z))\in F(B)$,所以
  诱导了映射 $F(f):F(A)\to F(B)$。这定义了函子 $F$。
\end{example}

\begin{definition}
  令 $\cat A,\cat B$ 是范畴,函子 $\opcat A\to\opcat B$ 被称为
  $\cat A$ 到 $\cat B$ 的\emph{逆变函子}。
\end{definition}

为了避免混淆,我们使用叙述“$\cat A$ 到 $\cat B$ 的逆变函子”
而不是“$\cat A\to\cat B$ 的逆变函子”。出于强调意义,通常意义下 $\cat A\to\cat B$
的函子被称为\emph{协变函子}。







\end{document}