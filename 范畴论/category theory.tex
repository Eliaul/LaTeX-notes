\documentclass[fontset=none]{Notes}

\makeatletter
\DeclareRobustCommand{\em}{%
  \@nomath\em \if b\expandafter\@car\f@series\@nil
  \normalfont \else \bfseries \fi}
\makeatother

\usepackage{tikz-cd,wrapstuff}
\usepackage{fixdif,siunitx,tikz,nicematrix,tabularray}
\usetikzlibrary{matrix,calc}

\ProvidesFile{font.def}

\setCJKmainfont{Source Han Serif SC}[
  UprightFont=*-Regular,
  BoldFont=*-Bold,
  ItalicFont=HYKaiTi S,
  ItalicFeatures={Scale=1.1}
]
\newCJKfontfamily[zhsong]\songti{Source Han Serif SC}[
  UprightFont=*-Regular,
  BoldFont=*-Bold,
  ItalicFont=HYKaiTi S,
  ItalicFeatures={Scale=1.1}
]
\setCJKsansfont{Source Han Sans SC}[
  UprightFont=*-Regular,
  BoldFont=*-Bold
]
\newCJKfontfamily[zhhei]\heiti{Source Han Sans SC}[
  UprightFont=*-Regular,
  BoldFont=*-Bold
]
\setCJKmonofont{HYFangSong S}[
  BoldFont=*,
  ItalicFont=*,
  BoldItalicFont=*
]
\newCJKfontfamily[zhfs]\fangsong{HYFangSong S}[
  BoldFont=*,
  ItalicFont=*,
  BoldItalicFont=*
]
\newCJKfontfamily[zhkai]\kaishu{HYKaiTi S}[
  BoldFont=*,
  ItalicFont=*,
  BoldItalicFont=*
]

\setmainfont{texgyretermes}[
  Extension=.otf,
  UprightFont=*-regular,
  BoldFont=*-bold,
  ItalicFont=*-italic,
  BoldItalicFont=*-bolditalic,
  SlantedFont=*-italic
]
%\setmathrm{texgyretermes}[
%  Extension=.otf,
%  UprightFont=*-regular,
%  BoldFont=*-bold,
%  ItalicFont=*-italic,
%  BoldItalicFont=*-bolditalic,
%  SlantedFont=*-italic
%]
\setsansfont{Cantarell}[
  UprightFont=* Regular,
  ItalicFont=* Italic,
  BoldFont=* Bold,
  BoldItalicFont=* Bold Italic,
  SmallCapsFont=Alegreya Sans SC
]
\setmonofont{Ubuntu Mono}[
  UprightFont=*,
  ItalicFont=* Italic,
  BoldFont=* Bold,
  BoldItalicFont=* Bold Italic
]
%\setmathfont{texgyretermes-math.otf}
%\setmathfont[range={\mathcal,\mathbfcal,\mathfrak},StylisticSet=1]{XITSMath-Regular.otf}
%\setmathfont[range={\mathbb}]{KpMath-Sans.otf}



\usepackage[subscriptcorrection,nofontinfo,mtpbb,mtpfrak]{mtpro2}

\tikzcdset{
  arrow style=tikz,
  diagrams={>={Straight Barb[scale=0.8]}}
}

\allowdisplaybreaks[1]

\newlength{\mymathln}
\newcommand{\aligninside}[2]{
  \settowidth{\mymathln}{#2}
  \mathmakebox[\mymathln]{#1}
}
\newenvironment{arr}[1][]{%
  $\begin{tikzcd}[cramped,#1]
}{\end{tikzcd}$}

\DeclareMathOperator\Spec{Spec}
\DeclareMathOperator\im{im}
\DeclareMathOperator\nil{nil}
\DeclareMathOperator\rad{rad}
\DeclareMathOperator\Ann{Ann}
\DeclareMathOperator\Max{Max}
\DeclareMathOperator\GL{GL}
\DeclareMathOperator\End{End}
\DeclareMathOperator\Int{Int}
\DeclareMathOperator\Tor{Tor}
\DeclareMathOperator\Frac{Frac}
\DeclareMathOperator\Tr{Tr}
\DeclareMathOperator\Hom{Hom}
\DeclareMathOperator\Leb{Leb}
\DeclareMathOperator\supp{supp}
\DeclareMathOperator\Id{Id}
\DeclareMathOperator\rk{rank}
\DeclareMathOperator\var{var}
\DeclareMathOperator\card{card}
\DeclareMathOperator\coker{coker}
\DeclareMathOperator\ob{ob}

\newcommand{\cat}[1]{\mathsf{#1}}

\usepackage{enumitem}

\setlist[enumerate]{
  nosep,
  label=(\alph*),
  itemindent=\labelwidth,
  leftmargin=*,
  listparindent=\parindent
}

%\DeclareMathAlphabet\mathcal{OMS}{cmsy}{m}{n}

\newlength\stextwidth
\newcommand\makesamewidth[3][c]{%
  \settowidth{\stextwidth}{#2}%
  \makebox[\stextwidth][#1]{#3}%
}



\begin{document}

\frontmatter

\tableofcontents

\mainmatter

\chapter{范畴、函子和自然变换}

\section{范畴}

\begin{definition}
  一个\emph{范畴} $\cat{A}$ 由以下内容组成:
  \begin{itemize}[nosep]
    \item 一族\emph{对象} $\ob(\cat A)$;
    \item 对于每个 $A,B\in\ob(\cat A)$,存在一族从 $A$ 到 $B$ 的\emph{态射}
    $\Hom(A,B)$;
    \item 对于每个 $A,B,C\in\ob(\cat A)$,有一个\emph{复合映射}:
    \[
      \Hom(A,B)\times \Hom(B,C)\to \Hom(A,C),
      \quad (g,f)\mapsto g\circ f;
    \]
    \item 对于每个 $A\in\ob(\cat A)$,存在 $A$ 上的\emph{单位}
    $1_A\in\Hom(A,A)$,
  \end{itemize}
  此外态射需要满足:
  \begin{itemize}[nosep]
    \item 对于每个 $f\in\Hom(A,B)$,$g\in\Hom(B,C)$ 与 $h\in\Hom(C,D)$,
    有 $(h\circ g)\circ f=h\circ(g\circ f)$;
    \item 对于每个 $f\in \Hom(A,B)$,有 $f\circ 1_A=f=1_B\circ f$。
  \end{itemize}
\end{definition}

\begin{remark}
  我们通常使用 $A\in \cat A$ 表示 $A\in\ob(\cat A)$,
  $f:A\to B$ 或者
  \begin{arr}[sep=small]
    A\arrow[r,"f"] & B
  \end{arr}
  表示 $f\in\Hom(A,B)$,$gf$ 表示 $g\circ f$。
\end{remark}

\begin{example}[数学结构的范畴]%
  \begin{enumerate}
    \item 集合范畴 $\cat{Set}$。对象为集合,给定集合 $A,B$,
    $A$ 到 $B$ 的态射就是集合意义下 $A$ 到 $B$ 的映射,态射的复合
    即映射的复合,此时单位 $1_A$ 就是恒等映射 $A\to A$。
    \item 群范畴 $\cat{Grp}$。对象为群,态射为群同态。
    \item 环范畴 $\cat{Ring}$。对象为环,态射为环同态。
    \item 给定域 $k$,有 $k$ 上的向量空间范畴 $\cat{Vect}_k$,
    对象是向量空间,态射是线性映射。
    \item 拓扑空间范畴 $\cat{Top}$。对象是拓扑空间,态射
    是连续映射。
  \end{enumerate}
\end{example}

\begin{definition}
  对于态射 $f:A\to B$,如果存在态射 $g:B\to A$ 使得
  $gf=1_A$ 以及 $fg=1_B$,那么我们说 $f$ 是\emph{同构}。
  此时我们说 $g$ 是 $f$ 的\emph{逆},记为 $g=f^{-1}$。
\end{definition}

如果 $A$ 到 $B$ 之间存在一个同构,那么我们说 $A$ 和
$B$ \emph{同构},记作 $A\cong B$。

\begin{example}
  $\cat{Set}$ 中的同构等同于双射。当然,这句话在逻辑上其实是在
  表明:一个映射具有双边逆映射当且仅当其是双射。
\end{example}

\begin{example}
  $\cat{Grp}$ 中的同构等同于群同构。在一些抽象代数教材中,群同构
  被定义为双射的群同态,如果是这样,那么实际上需要证明:
  双射的群同态的逆映射也是群同态。类似地,
  $\cat{Ring}$ 中的同构等同于环同构。
\end{example}

\begin{example}
  $\cat{Top}$ 中的同构是同胚。与 $\cat{Grp}$ 或者 $\cat{Ring}$
  不同的是,$\cat{Top}$ 中双射的连续映射不一定是同构,即
  连续映射的逆映射可以是连续的。下面是一个经典的例子:
  考虑映射 $f:[0,1)\to \mathbb{S}^1$ 为 $f(t)=e^{2\pi it}$,
  $f$ 是双射的连续映射,但是 $f$ 不是同胚,因为 $[0,1)$
  不是紧的,但是 $\mathbb{S}^1$ 是紧的。
\end{example}

目前为止范畴的例子中对象都是具有某些结构的集合(例如群结构、拓扑结构或者
只有集合结构),态射都是保持这些结构的映射(群同态、连续映射或者普通的映射)。
但是,并非所有的范畴都长成这样,实际上范畴的含义相当广泛,
其对象也不一定是“配备了额外结构的集合”,因此在一般的范畴中,
谈论对象的“元素”是没有意义的。类似地,在一般意义上的范畴中,
态射也不必是集合之间的映射。总的来说:
\emph{范畴的对象不必类似于集合,态射也不必类似于映射}。
下面的例子解释了这些观点。

\begin{example}[范畴作为数学结构]
  \begin{enumerate}
    \item 一个范畴可以通过直接说出对象、态射、复合和单位来指定。
    例如空范畴 $\emptyset$,其没有任何对象或者态射。
    范畴 $\cat 1$ 由一个对象和唯一的单位态射构成。
    也可以构造一个只有两个对象的范畴:
    \[
      \begin{tikzcd}
        \bullet\arrow[r] & \bullet,
      \end{tikzcd}
    \]
    这个范畴只有两个对象,每个对象有一个单位态射,两个对象之间有唯一的一个
    非单位态射。在这些例子中,我们并没有将对象视为一个类似集合的东西,
    也没有将态射视为一个映射,此时态射更多的类似于一个抽象的“箭头”。
    \item 有些范畴中的态射只有单位态射,即任意两个不同的对象之间
    都不存在任意态射,这样的范畴被称为\emph{离散范畴}。离散范畴
    是最极端的情况,即不同的对象之间完全隔离。
    \item 一个群本质上和只有一个对象且所有态射都是同构的范畴是一样的。
    我们来说明这一点。考虑只有一个对象的范畴 $\cat A$,记这个对象为
    $A$,那么范畴 $\cat A$ 的态射只有 $\Hom(A,A)$。我们要求
    $\cat A$ 中的每个态射都是同构,也就是说每个 $f\in\Hom(A,A)$
    都有一个逆 $g\in\Hom(A,A)$ 使得 $fg=1_A=gf$。
    实际上这样的范畴 $\cat A$ 和群没有本质区别,对应关系如下所示。
    {

      \vspace*{5pt}
      \centering
      \begin{tblr}{
        colspec={X[c]X[c]}
      }
      范畴 $\cat A$ & 群 $G$\\
      态射 $f\in \Hom(A,A)$ & 元素 $g\in G$ \\
      态射的复合 $\circ$ & 元素的乘法 $\cdot$ \\
      单位态射 $1_A$ & 单位元 $1\in G$
      \end{tblr} 
      \vspace*{5pt}

    }
    \item 在上一个例子中,由于态射的逆不一定是必须的,所以考虑
    “没有逆元的群”也是必要的,这被称为\emph{幺半群}。具有
    一个对象的范畴本质上和幺半群是相同的,其论证完全仿照上例。
    \item 一个\emph{预序}指的是满足自反性和传递性的二元关系。
    一个\emph{预序集} $(S,\leq)$ 指的是一个集合 $S$ 配备预序 $\leq$。
    例如 $S=\mathbb{Z}$,$\leq$ 是整除关系。

    一个预序集可以被视为范畴 $\cat A$,其中对于每个 $A,B\in\cat A$,
    至多只有一个从 $A$ 到 $B$ 的态射。此时我们用 $A\leq B$ 来表示存在
    态射 $A\to B$。因为 $\cat A$ 是范畴,所以 $A\leq B\leq C$
    表明 $A\leq C$。由于始终存在 $A\to A$ 的态射(即 $1_A$),所以
    $A\leq A$。所以 $\cat A$ 实际上就表示了一族对象,配备了
    一个具备自反性和传递性的二元关系。
  \end{enumerate}
\end{example}





\end{document}