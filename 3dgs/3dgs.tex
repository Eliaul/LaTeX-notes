\documentclass[fontset=none]{Notes}

\makeatletter
\DeclareRobustCommand{\em}{%
  \@nomath\em \if b\expandafter\@car\f@series\@nil
  \normalfont \else \bfseries \fi}
\makeatother

\usepackage{tikz-cd,wrapstuff}
\usepackage{siunitx,tikz,nicematrix,subcaption}
\usepackage{minted}
\definecolor{bg}{rgb}{.95,.95,.95}
\setminted{bgcolor=bg,numbersep=5pt}
\newminted[pycode]{python3}{fontsize=\small}
\newmintinline[py]{python3}{}

\usetikzlibrary{matrix,calc}
\usetikzlibrary{intersections}
\usetikzlibrary{arrows.meta}
\usetikzlibrary{decorations.markings}

\ProvidesFile{font.def}

\setCJKmainfont{Source Han Serif SC}[
  UprightFont=*-Regular,
  BoldFont=*-Bold,
  ItalicFont=HYKaiTi S,
  ItalicFeatures={Scale=1.1}
]
\newCJKfontfamily[zhsong]\songti{Source Han Serif SC}[
  UprightFont=*-Regular,
  BoldFont=*-Bold,
  ItalicFont=HYKaiTi S,
  ItalicFeatures={Scale=1.1}
]
\setCJKsansfont{Source Han Sans SC}[
  UprightFont=*-Regular,
  BoldFont=*-Bold
]
\newCJKfontfamily[zhhei]\heiti{Source Han Sans SC}[
  UprightFont=*-Regular,
  BoldFont=*-Bold
]
\setCJKmonofont{HYFangSong S}[
  BoldFont=*,
  ItalicFont=*,
  BoldItalicFont=*
]
\newCJKfontfamily[zhfs]\fangsong{HYFangSong S}[
  BoldFont=*,
  ItalicFont=*,
  BoldItalicFont=*
]
\newCJKfontfamily[zhkai]\kaishu{HYKaiTi S}[
  BoldFont=*,
  ItalicFont=*,
  BoldItalicFont=*
]

\setmainfont{texgyretermes}[
  Extension=.otf,
  UprightFont=*-regular,
  BoldFont=*-bold,
  ItalicFont=*-italic,
  BoldItalicFont=*-bolditalic,
  SlantedFont=*-italic
]
%\setmathrm{texgyretermes}[
%  Extension=.otf,
%  UprightFont=*-regular,
%  BoldFont=*-bold,
%  ItalicFont=*-italic,
%  BoldItalicFont=*-bolditalic,
%  SlantedFont=*-italic
%]
\setsansfont{Cantarell}[
  UprightFont=* Regular,
  ItalicFont=* Italic,
  BoldFont=* Bold,
  BoldItalicFont=* Bold Italic,
  SmallCapsFont=Alegreya Sans SC
]
\setmonofont{Ubuntu Mono}[
  UprightFont=*,
  ItalicFont=* Italic,
  BoldFont=* Bold,
  BoldItalicFont=* Bold Italic
]
%\setmathfont{texgyretermes-math.otf}
%\setmathfont[range={\mathcal,\mathbfcal,\mathfrak},StylisticSet=1]{XITSMath-Regular.otf}
%\setmathfont[range={\mathbb}]{KpMath-Sans.otf}



\usepackage[subscriptcorrection,nofontinfo,mtpbb,mtpfrak]{mtpro2}
\usepackage[normal]{fixdif}

\tikzcdset{
  arrow style=tikz,
  diagrams={>={Straight Barb[scale=0.8]}}
}

\tikzset{
  every picture/.style={
    thick,
    >={Latex[width=6pt, length=8pt]}
  },
  point/.style={
    circle, fill, minimum width=5pt,
    inner sep=0pt
  },
  straight arrow/.style={
    Straight Barb[scale=0.8]
  }
}

\allowdisplaybreaks[1]

\newlength{\mymathln}
\newcommand{\aligninside}[2]{
  \settowidth{\mymathln}{#2}
  \mathmakebox[\mymathln]{#1}
}

\DeclareMathOperator\Spec{Spec}
\DeclareMathOperator\im{im}
\DeclareMathOperator\sgn{sgn}
\DeclareMathOperator\rad{rad}
\DeclareMathOperator\Alt{Alt}
\DeclareMathOperator\Max{Max}
\DeclareMathOperator\card{card}
\DeclareMathOperator\GL{GL}
\DeclareMathOperator\Orth{O}
\DeclareMathOperator\SO{SO}
\DeclareMathOperator\SU{SU}
\DeclareMathOperator\cls{cls}
\DeclareMathOperator\Lie{Lie}
\DeclareMathOperator\End{End}
\DeclareMathOperator\Int{Int}
\DeclareMathOperator\Sym{Sym}
\DeclareMathOperator\tr{tr}
\DeclareMathOperator\Hom{Hom}
\DeclareMathOperator\supp{supp}
\DeclareMathOperator\Id{Id}
\DeclareMathOperator\rk{rank}
\DeclareMathOperator\grad{grad}
\DeclareMathOperator\rank{rank}
\DeclareMathOperator\Euc{E}
\DeclareMathOperator\ob{ob}
\DeclareMathOperator\diam{diam}
\DeclareMathOperator\rel{rel}
\DeclareMathOperator\inte{int}
\DeclareMathOperator\bd{bd}
\DeclareMathOperator\St{St}
\DeclareMathOperator\Sd{Sd}
\DeclareMathOperator\Lk{Lk}
\DeclareMathOperator\Ver{Vert}
\DeclareMathOperator\Nrv{Nrv}
\DeclareMathOperator\Alpha{Alpha}
\DeclareMathOperator\Rips{Rips}
\DeclareMathOperator\Cech{\text{\v Cech}}
\newcommand{\LL}{{\mathrm{L}}}

\newcommand{\norm}[1]{\left\lVert#1\right\rVert}
\newcommand{\mat}[1]{\mathbold{#1}}
\newcommand{\cat}[1]{\mathsf{#1}}
\newcommand{\uline}{\underline{\hphantom{X}}}
\newcommand{\abs}[1]{\left|#1\right|}
\newcommand{\lie}[1]{\mathfrak{#1}}
\newcommand{\inn}[1]{\left\langle #1\right\rangle}
\newcommand{\partI}{\partial I}
\newcommand{\relhomo}{\rel\partI}

\usepackage{enumitem}

\setlist[enumerate]{nosep}

%\DeclareMathAlphabet\mathcal{OMS}{cmsy}{m}{n}

\newlength\stextwidth
\newcommand\makesamewidth[3][c]{%
  \settowidth{\stextwidth}{#2}%
  \makebox[\stextwidth][#1]{#3}%
}



\begin{document}

\frontmatter

\tableofcontents

\mainmatter

\chapter{3DGS 原理与实现}

代码仓库与版本:\href{https://github.com/graphdeco-inria/gaussian-splatting}{\ttfamily 54c035f7834b564019656c3e3fcc3646292f727d}。

\section{高斯椭球表示、优化与代码实现}

本节代码位于 \py|scene/gaussian_model.py| 文件中。

\subsection{椭球定义}

一个高斯球由以下几个核心属性定义:
\begin{enumerate}
  \item 位置:\py|_xyz|。表示每个高斯球在三维空间中的中心点位置。
  \item 协方差:描述高斯球的形状和方向。为了保证矩阵在优化过程中始终为正定,
  假设协方差矩阵 $\Sigma$ 可以由一个缩放矩阵 $S$ 和一个旋转矩阵 $R$ 分解得到,
  即 $\Sigma = R S S^TR^T$。在代码中,为了存储方便,其存储的是一个三维向量
  与一个四元数。
  \begin{itemize}[nosep]
    \item 缩放:\py|_scaling|。一个三维向量,即缩放矩阵 $S$ 的对角线元素。
    \item 旋转:\py|_rotation|。一个四元数,即旋转矩阵 $R$ 的四元数表示。
  \end{itemize}
  \item 颜色:表示高斯球的颜色属性。使用 $Y_{lm}$ ($l$ 阶 $m$ 次球谐函数)来表示颜色,
  这样高斯球的颜色可以随着观察角度的变化而变化,从而模拟更复杂的光照和材质效果。
  简单来说,假设视角是 $(\theta,\varphi)$,那么颜色可以表示为球面 $\mathbb{S}^2$
  上的函数 $C(\theta,\varphi)$,其可以展开为球谐函数的线性组合:
  \[
    C(\theta,\varphi) = \sum_{l=0}^{\infty} \sum_{m=-l}^{l} c_{lm} Y_{lm}(\theta,\varphi).
  \]
  在实际应用中,通常只考虑低阶的球谐函数(例如 $l=0,1,2$),以减少计算复杂度。
  在 3DGS 代码中取的是最高 $3$ 阶。对于 $0,1,\dots,l$ 阶的球谐函数,共有
  $1+3+\cdots+(2l+1)=(l+1)^2$ 个球谐系数 $c_{lm}$ 需要优化。
  在代码中,这些系数被分开存储在两个属性中。
  \begin{itemize}[nosep]
    \item 漫反射系数:\py|_feature_dc|。也即球谐函数的零阶系数 $c_{00}$,
    这代表了高斯球的基础颜色,不随视角变化。这是因为 $Y_{00}=1/\sqrt{4\pi}$
    是常数。
    \item 高阶系数:\py|_feature_rest|。也即剩下的从 $1$ 阶到 $l$ 阶的球谐系数。
  \end{itemize}
  所以,对于一个颜色通道,若最高阶数为 $l$,则总共有 $(l+1)^2$ 个球谐系数。
  对于 RGB 三个颜色通道,总共有 $3(l+1)^2$ 个系数需要存储。
  \item 不透明度:\py|_opacity|。表示高斯球的透明度属性。
\end{enumerate}
总的来说,对于一个高斯球,需要存储 $3+3+4+3(l+1)^2+1=11+3(l+1)^2$ 个参数。
若 $l=3$,也即最高计算 $3$ 阶球谐系数,则总共有 $11+3\times16=59$ 个参数。

由于上述参数有些需要一定的数学约束,例如四元数的四个参数必须满足单位长度约束,
缩放参数必须为正等等,因此在代码实现中,3DGS 使用无约束的参数进行优化,
然后通过一些变换将其映射到有约束的参数空间中。例如,四元数参数通过归一化来确保其单位长度,
缩放参数通过指数映射来确保其为正值等等。具体来说,代码中定义了以下参数变换函数:
\begin{pycode}
def setup_functions(self):
    def build_covariance_from_scaling_rotation(scaling, scaling_modifier, rotation):
        L = build_scaling_rotation(scaling_modifier * scaling, rotation)
        actual_covariance = L @ L.transpose(1, 2)
        symm = strip_symmetric(actual_covariance)
        return symm
    
    self.scaling_activation = torch.exp
    self.scaling_inverse_activation = torch.log
    self.covariance_activation = build_covariance_from_scaling_rotation
    self.opacity_activation = torch.sigmoid
    self.inverse_opacity_activation = inverse_sigmoid
    self.rotation_activation = torch.nn.functional.normalize
\end{pycode}
可以看到,\py|scaling_activation| 定义为指数函数,用于确保缩放参数属性后为正值;
\py|covariance_activation| 定义为通过缩放和旋转构建协方差矩阵的函数(即 $\Sigma=RSS^TR^T$);
\py|opacity_activation| 定义为 Sigmoid 函数,确保不透明度在 $[0,1]$ 范围内;
\py|rotation_activation| 定义为归一化函数,确保四元数属性变换后为单位四元数。
在返回实际属性时,3DGS 会调用这些激活函数将无约束参数映射到有约束的属性空间中,
例如:
\begin{pycode}
@property
def get_scaling(self):
    return self.scaling_activation(self._scaling)
\end{pycode}

\subsection{场景创建与 I/O}

首先从一个初始点云创建高斯模型,这是训练开始的第一步,具体代码实现如下:  
\begin{pycode}[breaklines]
def create_from_pcd(self, pcd : BasicPointCloud, cam_infos : int, spatial_lr_scale : float):
    self.spatial_lr_scale = spatial_lr_scale
    fused_point_cloud = torch.tensor(np.asarray(pcd.points)).float().cuda()
    fused_color = RGB2SH(torch.tensor(np.asarray(pcd.colors)).float().cuda())
    features = torch.zeros((fused_color.shape[0], 3, (self.max_sh_degree + 1) ** 2)).float().cuda()
    features[:, :3, 0 ] = fused_color
    features[:, 3:, 1:] = 0.0

    print("Number of points at initialisation : ", fused_point_cloud.shape[0])

    dist2 = torch.clamp_min(distCUDA2(torch.from_numpy(np.asarray(pcd.points)).float().cuda()), 0.0000001)
    scales = torch.log(torch.sqrt(dist2))[...,None].repeat(1, 3)
    rots = torch.zeros((fused_point_cloud.shape[0], 4), device="cuda")
    rots[:, 0] = 1

    opacities = self.inverse_opacity_activation(0.1 * torch.ones((fused_point_cloud.shape[0], 1), dtype=torch.float, device="cuda"))

    self._xyz = nn.Parameter(fused_point_cloud.requires_grad_(True))
    self._features_dc = nn.Parameter(features[:,:,0:1].transpose(1, 2).contiguous().requires_grad_(True))
    self._features_rest = nn.Parameter(features[:,:,1:].transpose(1, 2).contiguous().requires_grad_(True))
    self._scaling = nn.Parameter(scales.requires_grad_(True))
    self._rotation = nn.Parameter(rots.requires_grad_(True))
    self._opacity = nn.Parameter(opacities.requires_grad_(True))
    self.max_radii2D = torch.zeros((self.get_xyz.shape[0]), device="cuda")
    self.exposure_mapping = {cam_info.image_name: idx for idx, cam_info in enumerate(cam_infos)}
    self.pretrained_exposures = None
    exposure = torch.eye(3, 4, device="cuda")[None].repeat(len(cam_infos), 1, 1)
    self._exposure = nn.Parameter(exposure.requires_grad_(True))
\end{pycode}
首先将点云的坐标和颜色转换成 \py|Tensor| 类型,初始的零阶球谐系数就是对应的 RGB 颜色,
高阶球谐系数全为零。然后通过计算每个点到其最近邻点的
距离 (\py|distCUDA2()| 函数) 来估算一个初始大小,避免高斯球过大或过小。
初始化旋转为单位四元数。初始化不透明度为一个较小的值 (0.1),以确保初始模型较为透明。
最后将这些初始参数封装为 \py|nn.Parameter| 对象,以便在训练过程中进行优化。

高斯模型可以输出为 \py|.ply| 格式的点云文件,里面包含高斯球的所有属性,代码如下:
\begin{pycode}[breaklines]
def save_ply(self, path):
    mkdir_p(os.path.dirname(path))

    xyz = self._xyz.detach().cpu().numpy()
    normals = np.zeros_like(xyz)
    f_dc = self._features_dc.detach().transpose(1, 2).flatten(start_dim=1).contiguous().cpu().numpy()
    f_rest = self._features_rest.detach().transpose(1, 2).flatten(start_dim=1).contiguous().cpu().numpy()
    opacities = self._opacity.detach().cpu().numpy()
    scale = self._scaling.detach().cpu().numpy()
    rotation = self._rotation.detach().cpu().numpy()

    dtype_full = [(attribute, 'f4') for attribute in self.construct_list_of_attributes()]

    elements = np.empty(xyz.shape[0], dtype=dtype_full)
    attributes = np.concatenate((xyz, normals, f_dc, f_rest, opacities, scale, rotation), axis=1)
    elements[:] = list(map(tuple, attributes))
    el = PlyElement.describe(elements, 'vertex')
    PlyData([el]).write(path)
\end{pycode}
这个函数将高斯球的各个属性提取出来,转换为 NumPy 数组,然后按照 PLY 文件格式的要求进行组织和存储。
每个点的属性标签由 \py|construct_list_of_attributes()| 函数生成,
首先是六个位置属性 \py|'x','y','z','nx','ny','nz'|,
然后是颜色属性 \py|'f_dc_i','f_rest_i'|,
最后是透明度和形状属性 \py|'opacity','scale_i','rot_i'|。

当然,也可以从上述格式的 PLY 文件中加载高斯模型,代码位于 \py|load_ply()| 函数中,
这里不再赘述。

\subsection{训练与优化}

接下来我们介绍高斯模型的训练与优化过程,代码位于 \py|train.py| 文件中。代码的主要流程
都在 \py|training()| 函数中,我们逐步进行解读。首先初始化各项设置:
\begin{pycode}[breaklines,linenos]
first_iter = 0
tb_writer = prepare_output_and_logger(dataset)
gaussians = GaussianModel(dataset.sh_degree, opt.optimizer_type)
scene = Scene(dataset, gaussians)
gaussians.training_setup(opt)
\end{pycode}
其中 \py|scene| 对象维护着相机和高斯模型,并提供渲染接口。

\section{Splatting 渲染原理与实现}

\subsection{前向传播渲染}





\end{document}