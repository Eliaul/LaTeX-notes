\documentclass[fontset=none]{Notes}

\makeatletter
\DeclareRobustCommand{\em}{%
  \@nomath\em \if b\expandafter\@car\f@series\@nil
  \normalfont \else \bfseries \fi}
\makeatother

\usepackage{tikz-cd,wrapstuff}
\usepackage{siunitx,tikz,nicematrix,subcaption}
\usepackage{minted}
\definecolor{bg}{rgb}{.95,.95,.95}
\setminted{bgcolor=bg,numbersep=5pt}
\newminted[pycode]{python3}{fontsize=\small}
\newmintinline[py]{python3}{}

\usetikzlibrary{matrix,calc}
\usetikzlibrary{intersections}
\usetikzlibrary{arrows.meta}
\usetikzlibrary{decorations.markings}

\ProvidesFile{font.def}

\setCJKmainfont{Source Han Serif SC}[
  UprightFont=*-Regular,
  BoldFont=*-Bold,
  ItalicFont=HYKaiTi S,
  ItalicFeatures={Scale=1.1}
]
\newCJKfontfamily[zhsong]\songti{Source Han Serif SC}[
  UprightFont=*-Regular,
  BoldFont=*-Bold,
  ItalicFont=HYKaiTi S,
  ItalicFeatures={Scale=1.1}
]
\setCJKsansfont{Source Han Sans SC}[
  UprightFont=*-Regular,
  BoldFont=*-Bold
]
\newCJKfontfamily[zhhei]\heiti{Source Han Sans SC}[
  UprightFont=*-Regular,
  BoldFont=*-Bold
]
\setCJKmonofont{HYFangSong S}[
  BoldFont=*,
  ItalicFont=*,
  BoldItalicFont=*
]
\newCJKfontfamily[zhfs]\fangsong{HYFangSong S}[
  BoldFont=*,
  ItalicFont=*,
  BoldItalicFont=*
]
\newCJKfontfamily[zhkai]\kaishu{HYKaiTi S}[
  BoldFont=*,
  ItalicFont=*,
  BoldItalicFont=*
]

\setmainfont{texgyretermes}[
  Extension=.otf,
  UprightFont=*-regular,
  BoldFont=*-bold,
  ItalicFont=*-italic,
  BoldItalicFont=*-bolditalic,
  SlantedFont=*-italic
]
%\setmathrm{texgyretermes}[
%  Extension=.otf,
%  UprightFont=*-regular,
%  BoldFont=*-bold,
%  ItalicFont=*-italic,
%  BoldItalicFont=*-bolditalic,
%  SlantedFont=*-italic
%]
\setsansfont{Cantarell}[
  UprightFont=* Regular,
  ItalicFont=* Italic,
  BoldFont=* Bold,
  BoldItalicFont=* Bold Italic,
  SmallCapsFont=Alegreya Sans SC
]
\setmonofont{Ubuntu Mono}[
  UprightFont=*,
  ItalicFont=* Italic,
  BoldFont=* Bold,
  BoldItalicFont=* Bold Italic
]
%\setmathfont{texgyretermes-math.otf}
%\setmathfont[range={\mathcal,\mathbfcal,\mathfrak},StylisticSet=1]{XITSMath-Regular.otf}
%\setmathfont[range={\mathbb}]{KpMath-Sans.otf}



\usepackage{biblatex}
\addbibresource{ref.bib}

\usepackage[subscriptcorrection,nofontinfo,mtpbb,mtpfrak]{mtpro2}
\usepackage[normal]{fixdif}

\tikzcdset{
  arrow style=tikz,
  diagrams={>={Straight Barb[scale=0.8]}}
}

\tikzset{
  every picture/.style={
    thick,
    >={Latex[width=6pt, length=8pt]}
  },
  point/.style={
    circle, fill, minimum width=5pt,
    inner sep=0pt
  },
  straight arrow/.style={
    Straight Barb[scale=0.8]
  }
}

\allowdisplaybreaks[1]

\newlength{\mymathln}
\newcommand{\aligninside}[2]{
  \settowidth{\mymathln}{#2}
  \mathmakebox[\mymathln]{#1}
}

\DeclareMathOperator\Spec{Spec}
\DeclareMathOperator\im{im}
\DeclareMathOperator\sgn{sgn}
\DeclareMathOperator\rad{rad}
\DeclareMathOperator\Alt{Alt}
\DeclareMathOperator\Max{Max}
\DeclareMathOperator\card{card}
\DeclareMathOperator\GL{GL}
\DeclareMathOperator\Orth{O}
\DeclareMathOperator\SO{SO}
\DeclareMathOperator\SU{SU}
\DeclareMathOperator\cls{cls}
\DeclareMathOperator\Lie{Lie}
\DeclareMathOperator\End{End}
\DeclareMathOperator\Int{Int}
\DeclareMathOperator\Sym{Sym}
\DeclareMathOperator\tr{tr}
\DeclareMathOperator\Hom{Hom}
\DeclareMathOperator\supp{supp}
\DeclareMathOperator\Id{Id}
\DeclareMathOperator\rk{rank}
\DeclareMathOperator\grad{grad}
\DeclareMathOperator\rank{rank}
\DeclareMathOperator\Euc{E}
\DeclareMathOperator\ob{ob}
\DeclareMathOperator\diam{diam}
\DeclareMathOperator\rel{rel}
\DeclareMathOperator\inte{int}
\DeclareMathOperator\bd{bd}
\DeclareMathOperator\St{St}
\DeclareMathOperator\Sd{Sd}
\DeclareMathOperator\Lk{Lk}
\DeclareMathOperator\Ver{Vert}
\DeclareMathOperator\Nrv{Nrv}
\DeclareMathOperator\Alpha{Alpha}
\DeclareMathOperator\Rips{Rips}
\DeclareMathOperator\Cech{\text{\v Cech}}
\newcommand{\LL}{{\mathrm{L}}}

\newcommand{\norm}[1]{\left\lVert#1\right\rVert}
\newcommand{\mat}[1]{\mathbold{#1}}
\newcommand{\cat}[1]{\mathsf{#1}}
\newcommand{\uline}{\underline{\hphantom{X}}}
\newcommand{\abs}[1]{\left|#1\right|}
\newcommand{\lie}[1]{\mathfrak{#1}}
\newcommand{\inn}[1]{\left\langle #1\right\rangle}
\newcommand{\partI}{\partial I}
\newcommand{\relhomo}{\rel\partI}

\usepackage{enumitem}

\setlist[enumerate]{nosep}

%\DeclareMathAlphabet\mathcal{OMS}{cmsy}{m}{n}

\newlength\stextwidth
\newcommand\makesamewidth[3][c]{%
  \settowidth{\stextwidth}{#2}%
  \makebox[\stextwidth][#1]{#3}%
}



\begin{document}

\frontmatter

\tableofcontents

\mainmatter

\chapter{3DGS 原理与实现}

代码仓库与版本:\href{https://github.com/graphdeco-inria/gaussian-splatting}{\ttfamily 54c035f7834b564019656c3e3fcc3646292f727d}。

\section{高斯椭球表示、优化与代码实现}

本节代码位于 \py|scene/gaussian_model.py| 文件中。

\subsection{椭球定义}

一个高斯球由以下几个核心属性定义:
\begin{enumerate}
  \item 位置:\py|_xyz|。表示每个高斯球在三维空间中的中心点位置。
  \item 协方差:描述高斯球的形状和方向。为了保证矩阵在优化过程中始终为正定,
  假设协方差矩阵 $\Sigma$ 可以由一个缩放矩阵 $S$ 和一个旋转矩阵 $R$ 分解得到,
  即 $\Sigma = R S S^TR^T$。在代码中,为了存储方便,其存储的是一个三维向量
  与一个四元数。
  \begin{itemize}[nosep]
    \item 缩放:\py|_scaling|。一个三维向量,即缩放矩阵 $S$ 的对角线元素。
    \item 旋转:\py|_rotation|。一个四元数,即旋转矩阵 $R$ 的四元数表示。
  \end{itemize}
  \item 颜色:表示高斯球的颜色属性。使用 $Y_{lm}$ ($l$ 阶 $m$ 次球谐函数)来表示颜色,
  这样高斯球的颜色可以随着观察角度的变化而变化,从而模拟更复杂的光照和材质效果。
  简单来说,假设视角是 $(\theta,\varphi)$,那么颜色可以表示为球面 $\mathbb{S}^2$
  上的函数 $C(\theta,\varphi)$,其可以展开为球谐函数的线性组合:
  \[
    C(\theta,\varphi) = \sum_{l=0}^{\infty} \sum_{m=-l}^{l} c_{lm} Y_{lm}(\theta,\varphi).
  \]
  在实际应用中,通常只考虑低阶的球谐函数(例如 $l=0,1,2$),以减少计算复杂度。
  在 3DGS 代码中取的是最高 $3$ 阶。对于 $0,1,\dots,l$ 阶的球谐函数,共有
  $1+3+\cdots+(2l+1)=(l+1)^2$ 个球谐系数 $c_{lm}$ 需要优化。
  在代码中,这些系数被分开存储在两个属性中。
  \begin{itemize}[nosep]
    \item 漫反射系数:\py|_feature_dc|。也即球谐函数的零阶系数 $c_{00}$,
    这代表了高斯球的基础颜色,不随视角变化。这是因为 $Y_{00}=1/\sqrt{4\pi}$
    是常数。
    \item 高阶系数:\py|_feature_rest|。也即剩下的从 $1$ 阶到 $l$ 阶的球谐系数。
  \end{itemize}
  所以,对于一个颜色通道,若最高阶数为 $l$,则总共有 $(l+1)^2$ 个球谐系数。
  对于 RGB 三个颜色通道,总共有 $3(l+1)^2$ 个系数需要存储。
  \item 不透明度:\py|_opacity|。表示高斯球的透明度属性。
\end{enumerate}
总的来说,对于一个高斯球,需要存储 $3+3+4+3(l+1)^2+1=11+3(l+1)^2$ 个参数。
若 $l=3$,也即最高计算 $3$ 阶球谐系数,则总共有 $11+3\times16=59$ 个参数。

由于上述参数有些需要一定的数学约束,例如四元数的四个参数必须满足单位长度约束,
缩放参数必须为正等等,因此在代码实现中,3DGS 使用无约束的参数进行优化,
然后通过一些变换将其映射到有约束的参数空间中。例如,四元数参数通过归一化来确保其单位长度,
缩放参数通过指数映射来确保其为正值等等。具体来说,代码中定义了以下参数变换函数:
\begin{pycode}
def setup_functions(self):
    def build_covariance_from_scaling_rotation(scaling, scaling_modifier, rotation):
        L = build_scaling_rotation(scaling_modifier * scaling, rotation)
        actual_covariance = L @ L.transpose(1, 2)
        symm = strip_symmetric(actual_covariance)
        return symm
    
    self.scaling_activation = torch.exp
    self.scaling_inverse_activation = torch.log
    self.covariance_activation = build_covariance_from_scaling_rotation
    self.opacity_activation = torch.sigmoid
    self.inverse_opacity_activation = inverse_sigmoid
    self.rotation_activation = torch.nn.functional.normalize
\end{pycode}
可以看到,\py|scaling_activation| 定义为指数函数,用于确保缩放参数属性后为正值;
\py|covariance_activation| 定义为通过缩放和旋转构建协方差矩阵的函数(即 $\Sigma=RSS^TR^T$);
\py|opacity_activation| 定义为 Sigmoid 函数,确保不透明度在 $[0,1]$ 范围内;
\py|rotation_activation| 定义为归一化函数,确保四元数属性变换后为单位四元数。
在返回实际属性时,3DGS 会调用这些激活函数将无约束参数映射到有约束的属性空间中,
例如:
\begin{pycode}
@property
def get_scaling(self):
    return self.scaling_activation(self._scaling)
\end{pycode}

\subsection{场景创建与 I/O}

首先从一个初始点云创建高斯模型,这是训练开始的第一步,具体代码实现如下:  
\begin{pycode}[breaklines]
def create_from_pcd(self, pcd : BasicPointCloud, cam_infos : int, spatial_lr_scale : float):
    self.spatial_lr_scale = spatial_lr_scale
    fused_point_cloud = torch.tensor(np.asarray(pcd.points)).float().cuda()
    fused_color = RGB2SH(torch.tensor(np.asarray(pcd.colors)).float().cuda())
    features = torch.zeros((fused_color.shape[0], 3, (self.max_sh_degree + 1) ** 2)).float().cuda()
    features[:, :3, 0 ] = fused_color
    features[:, 3:, 1:] = 0.0

    print("Number of points at initialisation : ", fused_point_cloud.shape[0])

    dist2 = torch.clamp_min(distCUDA2(torch.from_numpy(np.asarray(pcd.points)).float().cuda()), 0.0000001)
    scales = torch.log(torch.sqrt(dist2))[...,None].repeat(1, 3)
    rots = torch.zeros((fused_point_cloud.shape[0], 4), device="cuda")
    rots[:, 0] = 1

    opacities = self.inverse_opacity_activation(0.1 * torch.ones((fused_point_cloud.shape[0], 1), dtype=torch.float, device="cuda"))

    self._xyz = nn.Parameter(fused_point_cloud.requires_grad_(True))
    self._features_dc = nn.Parameter(features[:,:,0:1].transpose(1, 2).contiguous().requires_grad_(True))
    self._features_rest = nn.Parameter(features[:,:,1:].transpose(1, 2).contiguous().requires_grad_(True))
    self._scaling = nn.Parameter(scales.requires_grad_(True))
    self._rotation = nn.Parameter(rots.requires_grad_(True))
    self._opacity = nn.Parameter(opacities.requires_grad_(True))
    self.max_radii2D = torch.zeros((self.get_xyz.shape[0]), device="cuda")
    self.exposure_mapping = {cam_info.image_name: idx for idx, cam_info in enumerate(cam_infos)}
    self.pretrained_exposures = None
    exposure = torch.eye(3, 4, device="cuda")[None].repeat(len(cam_infos), 1, 1)
    self._exposure = nn.Parameter(exposure.requires_grad_(True))
\end{pycode}
首先将点云的坐标和颜色转换成 \py|Tensor| 类型,初始的零阶球谐系数就是对应的 RGB 颜色,
高阶球谐系数全为零。然后通过计算每个点到其最近邻点的
距离 (\py|distCUDA2()| 函数) 来估算一个初始大小,避免高斯球过大或过小。
初始化旋转为单位四元数。初始化不透明度为一个较小的值 (0.1),以确保初始模型较为透明。
最后将这些初始参数封装为 \py|nn.Parameter| 对象,以便在训练过程中进行优化。

高斯模型可以输出为 \py|.ply| 格式的点云文件,里面包含高斯球的所有属性,代码如下:
\begin{pycode}[breaklines]
def save_ply(self, path):
    mkdir_p(os.path.dirname(path))

    xyz = self._xyz.detach().cpu().numpy()
    normals = np.zeros_like(xyz)
    f_dc = self._features_dc.detach().transpose(1, 2).flatten(start_dim=1).contiguous().cpu().numpy()
    f_rest = self._features_rest.detach().transpose(1, 2).flatten(start_dim=1).contiguous().cpu().numpy()
    opacities = self._opacity.detach().cpu().numpy()
    scale = self._scaling.detach().cpu().numpy()
    rotation = self._rotation.detach().cpu().numpy()

    dtype_full = [(attribute, 'f4') for attribute in self.construct_list_of_attributes()]

    elements = np.empty(xyz.shape[0], dtype=dtype_full)
    attributes = np.concatenate((xyz, normals, f_dc, f_rest, opacities, scale, rotation), axis=1)
    elements[:] = list(map(tuple, attributes))
    el = PlyElement.describe(elements, 'vertex')
    PlyData([el]).write(path)
\end{pycode}
这个函数将高斯球的各个属性提取出来,转换为 NumPy 数组,然后按照 PLY 文件格式的要求进行组织和存储。
每个点的属性标签由 \py|construct_list_of_attributes()| 函数生成,
首先是六个位置属性 \py|'x','y','z','nx','ny','nz'|,
然后是颜色属性 \py|'f_dc_i','f_rest_i'|,
最后是透明度和形状属性 \py|'opacity','scale_i','rot_i'|。

当然,也可以从上述格式的 PLY 文件中加载高斯模型,代码位于 \py|load_ply()| 函数中,
这里不再赘述。

\subsection{训练与优化}

接下来我们介绍高斯模型的训练与优化过程,代码位于 \py|train.py| 文件中。代码的主要流程
都在 \py|training()| 函数中,我们逐步进行解读。首先初始化各项设置:
\begin{pycode}[breaklines,linenos]
first_iter = 0
tb_writer = prepare_output_and_logger(dataset)
gaussians = GaussianModel(dataset.sh_degree, opt.optimizer_type)
scene = Scene(dataset, gaussians)
gaussians.training_setup(opt)
\end{pycode}
其中 \py|scene| 对象维护着相机和高斯模型,并提供渲染接口。

\section{Splatting 渲染原理与实现}

\subsection{前向传播渲染}

3DGS 的渲染使用了 EWA Splatting 技术 \cite{1021576},
大致流程是:首先将每个高斯球投影到图像平面上,形成一个椭圆形的 splat,
然后根据每个 splat 的颜色和不透明度属性,对图像进行逐像素的颜色累积和混合。

对于一个高斯球,设其参数为:均值 $\mu\in \mathbb{R}^3$,协方差 $\Sigma\in \mathbb{R}^{3\times 3}$,
颜色 $c\in \mathbb{R}^3$ 和不透明度 $o\in \mathbb{R}$。
假设相机外参是
\[
  T_{\text{cw}}=\begin{bmatrix}
    R_{\text{cw}} & t_{\text{cw}} \\
    0 & 1
  \end{bmatrix}.
\]
首先将世界坐标系中的 $\mu$ 转换到相机坐标系:
\[
  t=T_{\text{cw}}\begin{bmatrix}
    \mu \\
    1 
  \end{bmatrix}.
\]
得到相机坐标系的中心点 $t=(t_x,t_y,t_z,1)^T$。
下一步是计算像素坐标,但是并不是用内参矩阵直接投影到像素坐标系,因为我们需要遵循图形学(OpenGL/DirectX)的标准管线,
以便利用 GPU 的硬件加速和排序算法。因此,我们需要把 $t$
转换到 NDC 坐标系(归一化设备坐标系),这是三维图形渲染管线中的中间坐标系统,
用于将不同分辨率的屏幕坐标统一到一个空间中,$xyz$ 的范围都是 $[-1,1]$。
假设内参矩阵
\[
  K=\begin{bmatrix}
    f_x & 0 & c_x \\
    0 & f_y & c_y \\
    0 & 0 & 1
  \end{bmatrix}.
\]
图像宽高分别为 $w,h$。根据传统的方法,利用内参得到的像素坐标
应该为
\begin{align*}
  \begin{bmatrix}
    u \\ 
    v \\
    1
  \end{bmatrix}=K \begin{bmatrix}
    t_x/t_z \\
    t_y/t_z \\
    1
  \end{bmatrix}=\begin{bmatrix}
    f_x t_x/t_z + c_x \\
    f_y t_y/t_z + c_y\\ 
    1
  \end{bmatrix}.
\end{align*}
然后我们需要构建一个线性映射把 $u,v$ 的值变到 $[-1,1]$ 内,
最简单的方法为:
\begin{equation}\label{eq:ndc}
  t_x' = 2\frac{u}{w} - 1, \quad t_y' = 2\frac{v}{h}-1.
\end{equation}
显然 $u=0$ 时 $t_x'=-1$,$u=w$ 时 $t_y'=1$,在这个过程中,$t_x',t_y'$ 在 $[-1,1]$ 外的点会被扔掉。于是
\[
  t_x' = 2\frac{f_x t_x/t_z + c_x}{w} - 1, \quad t_y' = 2\frac{f_y t_y/t_z + c_y}{h}-1.
\]
% 在图形学中,通常假设光心 $c_x$ 和 $c_y$ 位于图像中心,即 $c_x=w/2$,$c_y=h/2$。
% 带入后得到
% \[
%   t_x' = \frac{2 f_x t_x}{w t_z}, \quad t_y' = \frac{2 f_y t_y}{h t_z}.
% \]
于是投影矩阵 $P$ 应该满足
\begin{equation}\label{eq:proj_matrix}
    \begin{bmatrix}
    t_x' \\ t_y' \\ t_z' \\ t_w'
  \end{bmatrix}=\begin{bmatrix}
    2f_x/w & 0 & 2c_x/w-1 & 0 \\
    0 & 2f_y/h & 2c_y/h-1 & 0 \\
    0 & 0 & A & B \\
    0 & 0 & 1 & 0
  \end{bmatrix}
  \begin{bmatrix}
    t_x/t_z \\ t_y/t_z \\ 1 \\ 1/t_z
  \end{bmatrix}
  ,
\end{equation}
接下来处理深度 $t_z$,将其映射到 $[-1,1]$ 范围内,假设近平面和远平面分别为 $n,f$,
那么 $t_z=n$ 的时候应该映射到 $t_z'=-1$,$t_z=f$ 的时候映射到 $t_z'=1$。
而 $t_z'=A+B/t_z$,所以
\[
  \begin{cases}
    A + B/n = -1 \\
    A + B/f = 1
  \end{cases} \implies
  \begin{cases}
    A = (f+n)/(f-n) \\
    B = -2fn/(f-n)
  \end{cases}.
\]
综上所述,投影矩阵为
\[
  P=\begin{bmatrix}
    2f_x/w & 0 & 2c_x/w-1 & 0 \\
    0 & 2f_y/h & 2c_y/h-1 & 0 \\
    0 & 0 & (f+n)/(f-n) & -2fn/(f-n) \\
    0 & 0 & 1 & 0
  \end{bmatrix}.
\]
NDC 坐标为 $t'=Pt\in \mathbb{R}^4$。最后,再把 $t'$ 变回像素坐标系。
也即把 \eqref{eq:ndc} 反过来,得到
\begin{equation}\label{eq:pixel_coord}
  \mu_u' = \frac{w}{2}(t_x' + 1)+\frac{1}{2}, \quad \mu_v' = \frac{h}{2}(t_y' + 1)+\frac{1}{2}.
\end{equation}
注意这里加上了 $0.5$,是因为像素坐标系的原点在像素的中心而不是左上角。
通常情况下光心 $c_x$ 和 $c_y$ 位于图像中心,即 $c_x=w/2$,$c_y=h/2$,
此时 $P$ 可以简化为
\[
  P=\begin{bmatrix}
    2f_x/w & 0 & 0 & 0 \\
    0 & 2f_y/h & 0 & 0 \\
    0 & 0 & (f+n)/(f-n) & -2fn/(f-n) \\
    0 & 0 & 1 & 0
  \end{bmatrix}.
\]

下面我们需要计算投影后的协方差矩阵。首先,经过外参变换点 $x\mapsto R_{\text{cw}}x+t_{\text{cw}}$,
这个变换后依然是高斯椭球,协方差变为 $\Sigma\mapsto R_{\text{cw}}\Sigma R_{\text{cw}}^\top $。
然后是像素坐标转换
\[
  u=f_x \frac{t_x}{t_z}+c_x,\quad v=f_y \frac{t_y}{t_z}+c_y,
\]
这不是一个仿射变换,所以变换后实际上不再是高斯分布。但是我们可以在 
$( t_x, t_y, t_z)$ 处做微分,得到近似后的高斯分布。我们有
\[
  J=\begin{bmatrix}
    \dfrac{\partial u}{\partial t_x} & \dfrac{\partial u}{\partial t_y} & \dfrac{\partial u}{\partial t_z} \\[3mm]
    \dfrac{\partial v}{\partial t_x} & \dfrac{\partial v}{\partial t_y} & \dfrac{\partial v}{\partial t_z}
  \end{bmatrix}=
  \begin{bmatrix}
    f_x/t_z & 0 & -f_x t_x/t_z^2 \\[2mm]
    0 & f_y/t_z & -f_y t_y/t_z^2
  \end{bmatrix}.  
\]
于是 splatting 后的 2D 协方差矩阵为
\[
  \Sigma' = J (R_{\text{cw}}\Sigma R_{\text{cw}}^\top) J^\top\in \mathbb{R}^{2\times 2}.
\]
为了方便后面的反向传播,设四元数 $q=(q_x,q_y,q_z,q_w)$,尺度 $s\in \mathbb{R}^3$,
那么四元数对应的旋转矩阵为
\begin{equation}\label{eq:quat_rot}
    R=\begin{bmatrix}
    1-2q_y^2-2q_z^2 & 2q_x q_y - 2 q_z q_w & 2 q_x q_z + 2 q_y q_w \\
    2 q_x q_y + 2 q_z q_w & 1 - 2 q_x^2 - 2 q_z^2 & 2 q_y q_z - 2 q_x q_w \\
    2 q_x q_z - 2 q_y q_w & 2 q_y q_z + 2 q_x q_w & 1 - 2 q_x^2 - 2 q_y^2
  \end{bmatrix}.
\end{equation}
此时 $\Sigma=RSS^\top R^\top$。

在高斯球的混合过程中,\cite{kerbl3Dgaussians} 使用以下排序策略。将屏幕划分
为 $16\times 16$ 的小块,然后对每个块内的 2D 高斯按照深度排序。具体来说,
对于每个高斯球,先计算一个包围盒,这个包围盒的大小覆盖了 99\% 的置信区间(即 3 倍标准差),
如果一个高斯球的包围盒与某个小块相交,它就会被放入该小块的处理队列中。
然后在每个小块内,对这些高斯球按照其深度进行排序,深度较小的高斯球优先渲染。

然后对每个小块内排序好的高斯球进行光栅化。对于像素 $i$ 的颜色,设该像素
被 $N$ 个高斯球影响,那么这个像素的颜色为
\begin{equation}\label{eq:color_mix}
  C_i=\sum_{n\leq N}c_n\alpha_n T_n,\quad T_n=\prod_{m< n}(1-\alpha_m),
\end{equation}
其中 $c_n$ 是第 $n$ 个高斯球的颜色,由球谐函数算出,$\alpha_n$
是第 $n$ 个高斯球在该像素处的不透明度,由以下公式计算:
\[
  \alpha_n=o_n\exp(-\sigma_n),\quad \sigma_n=\frac{1}{2}\Delta_n^\top \Sigma_n'^{-1}\Delta_n,
\]
这里 $\Delta_n\in \mathbb{R}^2$ 表示 2D 高斯球中心 $\mu'\in \mathbb{R}^2$ 和像素中心
的偏移,$o_n$ 是高斯球的全局不透明度属性。$T_n$ 是透射率,表示
光线到达当前第 $n$ 个高斯球时,还剩下多少光没被前面的遮挡。如果
前面的高斯球 $m<n$ 非常不透明 ($\alpha_m$ 接近 1),那么 $1−\alpha_m$ 接近 0,
累乘后 $T_n$ 就会变得非常小。这意味着后面的高斯球会被前面的挡住,
对最终颜色的贡献几乎为 0。

\subsection{反向传播优化}

我们现在计算一个标量 loss 相对于高斯参数的梯度。给定输出图像相对于
每个像素的一个标量 loss $\mathcal{L}$,我们利用链式法则
反向计算到每个输入参数的梯度。本节主要参考了 \cite{ye2023mathematicalsupplementtextttgsplatlibrary}。

我们先介绍一些矩阵求导的符号和基本法则。对于两个矩阵 $X,Y$,定义内积
\[
  \langle X,Y \rangle = \mathrm{Tr}(X^\top Y).
\]
这实际上就是把 $X$ 和 $Y$ 展开成向量后做点积。容易得出这个内积的一些基本性质:
\begin{align*}
  \langle X,Y\rangle&=\langle Y,X\rangle, \\
  \langle X^\top, Y^\top \rangle&=\langle X,Y\rangle, \\
  \langle X,YZ\rangle&=\langle Y^\top X, Z\rangle=\langle XZ^\top, Y\rangle,\\
  \langle X,Y+Z\rangle&=\langle X,Y\rangle+\langle X,Z\rangle.
\end{align*}
现在假设有一个矩阵函数 $f:\mathbb R^{m\times n}\to \mathbb{R}$,以及一个
矩阵变量 $X:\mathbb{R}\to \mathbb{R}^{m\times n}$,那么复合函数
$f\circ X:\mathbb{R}\to \mathbb{R}$ 的导数为
\[
  (f\circ X)'(x)=\sum_{i,j} \frac{\partial f}{\partial X_{ij}}(X_{ij}(x)) X_{ij}'(x) 
  = \left\langle \frac{\partial f}{\partial X}(X(x)), X'(x)\right\rangle,
\]
我们简记为
\[
  \frac{\partial f}{\partial x}=\biggl\langle \frac{\partial f}{\partial X}, \frac{\partial X}{\partial x}\biggr\rangle
  \in \mathbb{R}.
\]
这里 $\frac{\partial f}{\partial X},\frac{\partial X}{\partial x}\in \mathbb{R}^m\times n$
就是各个元素的偏导数组成的矩阵。在这个记号下,链式法则仍然是有效的。
例如,如果 $X=AY$,其中 $A:\mathbb{R}\to \mathbb{R}^{m\times p}$,
$Y:\mathbb{R}\to \mathbb{R}^{p\times n}$,那么 
\begin{align*}
  \frac{\partial f}{\partial x} &=\biggl\langle \frac{\partial f}{\partial X}, \frac{\partial (AY)}{\partial x}\biggr\rangle
  =\biggl\langle \frac{\partial f}{\partial X}, \frac{\partial A}{\partial x}Y + A\frac{\partial Y}{\partial x}\biggr\rangle \\
  &=\biggl\langle \frac{\partial f}{\partial X}Y^\top, \frac{\partial A}{\partial x}\biggr\rangle
  +\biggl\langle A^\top \frac{\partial f}{\partial X}, \frac{\partial Y}{\partial x}\biggr\rangle.
\end{align*}
如果把 $x$ 替换为 $A_{ij}$,可以得到
\[
  \frac{\partial f}{\partial A_{ij}}=\biggl\langle \frac{\partial f}{\partial X}Y^\top, \frac{\partial A}{\partial A_{ij}}\biggr\rangle
  =\left(\frac{\partial f}{\partial X}Y^\top\right)_{ij},
\]
所以
\[
  \frac{\partial f}{\partial A}=\frac{\partial f}{\partial X}Y^\top\in \mathbb{R}^{m\times p}.
\]
同理可得
\[
  \frac{\partial f}{\partial Y}=A^\top \frac{\partial f}{\partial X}\in \mathbb{R}^{p\times n}.
\]
于是,我们有
\[
  \frac{\partial f}{\partial x}=\biggl\langle \frac{\partial f}{\partial A}, \frac{\partial A}{\partial x}\biggr\rangle
  +\biggl\langle \frac{\partial f}{\partial Y}, \frac{\partial Y}{\partial x}\biggr\rangle.
\]

下面我们分两个部分计算梯度。

\paragraph{深度混合梯度}

我们首先把像素 $i$ 的梯度传播到该像素影响的每个高斯球上。
具体来说,对于像素 $i$ 影响的第 $n$ 个高斯球,我们需要计算相对于颜色
的梯度 $\partial \mathcal{L}/\partial c_n\in \mathbb{R}^3$,
不透明度的梯度 $\partial \mathcal{L}/\partial o_n\in \mathbb{R}$,
2D 均值的梯度 $\partial \mathcal{L}/\partial \mu_n'\in \mathbb{R}^2$,
以及 2D 协方差的梯度 $\partial \mathcal{L}/\partial \Sigma_n'\in \mathbb{R}^{2\times 2}$。

对于颜色,根据 \eqref{eq:color_mix} 式,每个颜色通道 $k$ 的梯度为
\begin{equation}
  \frac{\partial C_{i}(k)}{\partial c_n(k)}=\alpha_n T_n.
\end{equation}
为了加速 $T_n$ 的计算,我们在前向传播步骤会保存 $T_N$ 的值,
然后依次向前计算 $T_{N-1},T_{N-2},\dots,T_1$,只需要
利用递推公式 $T_{n-1}=T_n/(1-\alpha_{n-1})$。

对于 $\alpha_n$ 的梯度,对于每个颜色通道 $k$,有
\begin{equation}
  \frac{\partial C_i(k)}{\partial\alpha_n}=c_n(k)T_n-\frac{S_n(k)}{1-\alpha_n},
  \quad S_n=\sum_{m>n}c_m\alpha_mT_m.
\end{equation}
同样,我们也可以从后往前计算 $S_n$。只需要注意到 
\[
  S_N(k)=0,\quad S_{n-1}(k)=c_n(k)\alpha_nT_n+S_n(k).
\]

对于不透明度 $o_n$ 和 $\sigma_n$,有 
\begin{equation}
    \frac{\partial \alpha_n}{\partial o_n}=\exp(-\sigma_n),\quad
  \frac{\partial \alpha_n}{\partial \sigma_n}=-o_n \exp(-\sigma_n)=-\alpha_n.
\end{equation}

对于 2D 均值 $\mu_n'$,有
\begin{equation}
  \frac{\partial \sigma_n}{\partial \mu_n'}
  = \frac{\partial \sigma_n}{\partial \Delta_n}= \Sigma_n'^{-1}\Delta_n.
\end{equation}

对于 2D 协方差 $\Sigma_n'$,回顾 $\sigma_n=\frac{1}{2}\Delta_n^\top\Sigma_n'^{-1}\Delta_n$,
我们先计算
\begin{align*}
  \frac{\partial \sigma_n}{\partial x}&=
  \biggl\langle \frac{\partial \sigma_n}{\partial(\Sigma_n'^{-1})}, \frac{\partial(\Sigma_n'^{-1})}{\partial x}\biggr\rangle
  =\biggl\langle
  \frac{1}{2}\Delta_n\Delta_n^\top,
  -\Sigma_n'^{-1}\frac{\partial \Sigma_n'}{\partial x}\Sigma_n'^{-1}
  \biggr\rangle\\
  &=\biggl\langle
  -\frac{1}{2}\Sigma_n'^{-1}\Delta_n\Delta_n^\top \Sigma_n'^{-1},
  \frac{\partial \Sigma_n'}{\partial x}
  \biggr\rangle,
\end{align*}
再取 $x=(\Sigma_n')_{ij}$,便得到 
\begin{equation}
  \frac{\partial \sigma_n}{\partial\Sigma_n'}=
  -\frac{1}{2}\Sigma_n'^{-1}\Delta_n\Delta_n^\top \Sigma_n'^{-1}.
\end{equation}

\paragraph{投影梯度}
接下来我们处理投影变换导致的梯度传播。给定 $\mathcal{L}$
相对于 2D 均值 $\mu_n'$ 和 2D 协方差 $\Sigma_n'$ 的梯度,我们需要
传播到相对于 3D 均值 $\mu$ 和 3D 协方差 $\Sigma$ 的梯度。
这个步骤只会涉及到单个高斯球,所以我们可以扔掉下标 $n$,目标
是计算 $\partial \mathcal{L}/\partial \mu\in \mathbb{R}^3$
以及 $\partial \mathcal{L}/\partial \Sigma\in \mathbb{R}^{2\times 2}$。

首先我们注意到 loss $\mathcal{L}$ 可以视为 $\mu'$ 和 $\Sigma'$ 的函数,
而 $\mu'$ 和 $\Sigma'$ 又是相机系坐标 $t\in \mathbb{R}^3$ (这里我们去掉最后一个常数 $1$)的函数,所以 $\mathcal{L}$
相对于相机系坐标 $t\in \mathbb{R}^3$ 的梯度为
\begin{equation}\label{eq:grad_t}
  \frac{\partial \mathcal{L}}{\partial t_i}=
  \biggl\langle \frac{\partial \mathcal{L}}{\partial \mu'},\frac{\partial \mu'}{\partial t_i}\biggr\rangle
  +\biggl\langle \frac{\partial \mathcal{L}}{\partial \Sigma'}, \frac{\partial \Sigma'}{\partial t_i}\biggr\rangle.
\end{equation}
其中 $\partial \mathcal{L}/\partial \mu'\in \mathbb{R}^2$ 和
$\partial \mathcal{L}/\partial \Sigma'\in \mathbb{R}^{2\times 2}$
是前一步计算得到的梯度。所以接下来我们需要计算 $\partial \mu'/\partial t_i$ 和 $\partial \Sigma'/\partial t_i$。

对于 2D 均值 $\mu'$ 相对于 $t$ 的梯度,这就是矩阵 $J$,即
\begin{equation}\label{eq:grad_mu_t}
  \frac{\partial \mu'}{\partial t}=J=\begin{bmatrix}
    f_x/t_z & 0 & -f_x t_x/t_z^2 \\
    0 & f_y/t_z & -f_y t_y/t_z^2  
  \end{bmatrix}=
  \begin{bmatrix}
    \dfrac{\partial \mu'}{\partial t_x} & \dfrac{\partial \mu'}{\partial t_y} & \dfrac{\partial \mu'}{\partial t_z}
  \end{bmatrix}.
\end{equation}

对于 2D 协方差 $\Sigma'$,回顾 $\Sigma' = J (R_{\text{cw}}\Sigma R_{\text{cw}}^\top) J^\top$,
其中只有 $J$ 是 $t$ 的函数。我们有
\begin{equation}\label{eq:grad_sigma_t}
  \frac{\partial \Sigma'}{\partial t_i}=
  \frac{\partial(J R_{\text{cw}})}{\partial t_i} \Sigma (J R_{\text{cw}})^\top
  + (J R_{\text{cw}}) \Sigma \frac{\partial (J R_{\text{cw}})^\top}{\partial t_i},
\end{equation}
对于 $J$,有
\begin{gather*}
    \frac{\partial J}{\partial t_x}=
  \begin{bmatrix}
    0 & 0 -f_x/t_z^2 \\
    0 & 0 & 0
  \end{bmatrix},\ 
  \frac{\partial J}{\partial t_y}=
  \begin{bmatrix}
    0 & 0 & 0 \\
    0 & 0 & -f_y/t_z^2
  \end{bmatrix},\\
  \frac{\partial J}{\partial t_z}=
  \begin{bmatrix}
    -f_x/t_z^2 & 0 & 2 f_x t_x/t_z^3 \\
    0 & -f_y/t_z^2 & 2 f_y t_y/t_z^3
  \end{bmatrix}.
\end{gather*}

把 \eqref{eq:grad_mu_t} 和 \eqref{eq:grad_sigma_t} 带入 \eqref{eq:grad_t} 中,
就可以得到
\begin{align}
  \frac{\partial \mathcal{L}}{\partial t_i}&=
  \biggl\langle \frac{\partial \mathcal{L}}{\partial \mu'}, \frac{\partial \mu'}{\partial t_i}\biggr\rangle 
  +\biggl\langle \frac{\partial \mathcal{L}}{\partial \Sigma'}, 
  \frac{\partial J}{\partial t_i}R_{\text{cw}}\Sigma  R_{\text{cw}}^\top J^\top
  +JR_{\text{cw}}\Sigma R_{\text{cw}}^\top\frac{\partial J^\top}{\partial t_i}
  \biggr\rangle\notag{}\\
  &=\biggl\langle \frac{\partial \mathcal{L}}{\partial \mu'}, \frac{\partial \mu'}{\partial t_i}\biggr\rangle
  +\biggl\langle
  \frac{\partial \mathcal{L}}{\partial \Sigma'} J R_{\text{cw}}\Sigma^\top R_{\text{cw}}^\top
  +
  \bigl(\frac{\partial \mathcal{L}}{\partial \Sigma'}\bigr)^\top
  J R_{\text{cw}}\Sigma R_{\text{cw}}^\top
  , \frac{\partial J}{\partial t_i} 
  \biggr\rangle,\notag{}\\
  &=\biggl\langle \frac{\partial \mathcal{L}}{\partial \mu'}, \frac{\partial \mu'}{\partial t_i}\biggr\rangle
  +\biggl\langle
  \left(\frac{\partial \mathcal{L}}{\partial \Sigma'} 
  +\bigl(\frac{\partial \mathcal{L}}{\partial \Sigma'}\bigr)^\top
  \right)J R_{\text{cw}}\Sigma R_{\text{cw}}^\top,
  \frac{\partial J}{\partial t_i} 
  \biggr\rangle,\label{eq:grad_t_final}
\end{align}
然后只需要计算 $t_i$ 对 $\mu$ 的梯度即可完成对 $\mu$ 的反向传播。
由于 $t=R_{\text{cw}}\mu + t_{\text{cw}}$,所以
\begin{equation}
  \frac{\partial \mathcal{L}}{\partial \mu} = 
  R_{\text{cw}}^\top \frac{\partial \mathcal{L}}{\partial t}.
\end{equation}

下面我们转向对 $\Sigma$ 的梯度传播。由于 $\mu'$ 不依赖于 $\Sigma$,
所以只有 $\Sigma'$ 会传递这部分的梯度,于是
\begin{align*}
  \frac{\partial \mathcal{L}}{\partial \Sigma_{ij}}&=
  \biggl\langle \frac{\partial \mathcal{L}}{\partial \Sigma'}, \frac{\partial \Sigma'}{\partial \Sigma}\biggr\rangle=
  \biggl\langle \frac{\partial \mathcal{L}}{\partial \Sigma'}, J R_{\text{cw}} \frac{\partial \Sigma}{\partial \Sigma_{ij}} R_{\text{cw}}^\top J^\top\biggr\rangle\\
  &=
  \biggl\langle
  R_{\text{cw}}^\top J^\top \frac{\partial \mathcal{L}}{\partial \Sigma'} J R_{\text{cw}},
  \frac{\partial \Sigma}{\partial \Sigma_{ij}}
  \biggr\rangle.
\end{align*}
于是
\begin{equation}
  \frac{\partial \mathcal{L}}{\partial \Sigma}=
  R_{\text{cw}}^\top J^\top \frac{\partial \mathcal{L}}{\partial \Sigma'} J R_{\text{cw}}.
\end{equation}

\paragraph{尺度和旋转梯度}
最后,我们只需要把 $\Sigma$ 的梯度传播到尺度 $s$ 和旋转四元数 $q$ 上。
回顾 $\Sigma=RSS^\top R^\top$,于是
\begin{align*}
  \frac{\partial \mathcal{L}}{\partial S_{ij}}&=
  \biggl\langle \frac{\partial \mathcal{L}}{\partial \Sigma}, \frac{\partial (RSS^\top R^\top)}{\partial S_{ij}}\biggr\rangle
  =\biggl\langle \frac{\partial \mathcal{L}}{\partial \Sigma}, R \frac{\partial S}{\partial S_{ij}} S^\top R^\top
  + R S \frac{\partial S^\top}{\partial S_{ij}} R^\top \biggr\rangle\\
  &=\biggl\langle R^\top \frac{\partial \mathcal{L}}{\partial \Sigma} R,
  \frac{\partial S}{\partial S_{ij}} S^\top + S \frac{\partial S^\top}{\partial S_{ij}}
  \biggr\rangle=
  \biggl\langle 
    R^T\biggl(\frac{\partial \mathcal{L}}{\partial \Sigma}
    +\bigl(\frac{\partial \mathcal{L}}{\partial \Sigma}\bigr)^\top
    \biggr) RS
    ,
    \frac{\partial S}{\partial S_{ij}}
  \biggr\rangle,
\end{align*}
所以
\begin{equation}
  \frac{\partial \mathcal{L}}{\partial S}=
    R^T\biggl(\frac{\partial \mathcal{L}}{\partial \Sigma}
    +\bigl(\frac{\partial \mathcal{L}}{\partial \Sigma}\bigr)^\top
    \biggr) RS.
\end{equation}
最后,注意到 $S$ 是对角矩阵,且 $S_{ii}=s_i$,所以
\[
  \frac{\partial S}{\partial s_i}=E_{ii}.
\]
这里 $E_{ii}$ 是只有第 $(i,i)$ 元素为 1,其余元素为 0 的矩阵。

对于四元数,首先有
\begin{align*}
  \frac{\partial \mathcal{L}}{\partial R_{ij}}&=
  \biggl\langle \frac{\partial \mathcal{L}}{\partial \Sigma}, \frac{\partial (RSS^\top R^\top)}{\partial R_{ij}}\biggr\rangle
  =\biggl\langle \frac{\partial \mathcal{L}}{\partial \Sigma},
  \frac{\partial R}{\partial R_{ij}} S S^\top R^\top
  + R S S^\top \frac{\partial R^\top}{\partial R_{ij}}
  \biggr\rangle\\
  &=\biggl\langle
    \biggl(\frac{\partial \mathcal{L}}{\partial \Sigma}
    +\bigl(\frac{\partial \mathcal{L}}{\partial \Sigma}\bigr)^\top
    \biggr)RSS^\top
  ,
  \frac{\partial R}{\partial R_{ij}}
  \biggr\rangle.
\end{align*}
也即
\begin{equation}
  \frac{\partial \mathcal{L}}{\partial R}=
    \biggl(\frac{\partial \mathcal{L}}{\partial \Sigma}
    +\bigl(\frac{\partial \mathcal{L}}{\partial \Sigma}\bigr)^\top
    \biggr)RSS^\top.
\end{equation}
最后,根据旋转矩阵和四元数的关系 \eqref{eq:quat_rot},可以计算
\begin{gather*}
  \frac{\partial R}{\partial q_x}=
  2\begin{bmatrix}
    0 & q_y & q_z \\
    q_y & -2 q_x & q_w \\
    q_z & -q_w & -2 q_x
  \end{bmatrix},\quad
  \frac{\partial R}{\partial q_y}=
  2\begin{bmatrix}
    -2 q_y & q_x & -q_w \\
    q_x & 0 & q_z \\
    q_w & q_z & -2 q_y
  \end{bmatrix},\\
  \frac{\partial R}{\partial q_z}=
  2\begin{bmatrix}
    -2 q_z & q_w & q_x \\
    -q_w & -2 q_z & q_y \\
    q_x & q_y & 0
  \end{bmatrix},\quad
  \frac{\partial R}{\partial q_w}=
  2\begin{bmatrix}
    0 & -q_z & q_y \\
    q_z & 0 & -q_x \\
    -q_y & q_x & 0
  \end{bmatrix}.  
\end{gather*}







\printbibliography



\end{document}