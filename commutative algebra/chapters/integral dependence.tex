\chapter{整相关和赋值}

\section{整相关}

令 $B$ 是环,$A$ 是 $B$ 的子环,$x\in B$ 在 $A$ 上是\emph{整的},指的是
$x$ 是某个 $A$ 中系数的首一多项式的根。显然 $A$ 的每个元素在 $A$ 上都是整的。

\begin{proposition}\label{prop:integral dependence}
  下面的说法是等价的:
  \begin{enumerate}
    \item $b\in B$ 在 $A$ 上是整的;
    \item $A[b]$ 是有限生成 $A$-模;
    \item $A[b]$ 被包含在 $B$ 的一个子环 $C$ 中,其中 $C$ 是有限生成 $A$-模;
    \item 存在一个忠实的 $A[b]$-模 $M$(即 $\Ann(M)=0$),其作为 $A$-模是有限生成的。
  \end{enumerate}
\end{proposition}
\begin{proof}
  $(1)\Rightarrow (2)$ 设 $b$ 满足方程
  \[
    x^n+a_{1}x^{n-1}+\cdots+  a_{n-1}x+a_n=0,
  \]
  那么 $b^{n+k}\ (k\geq 0)$ 都可以写为 $1,b,\dots,b^{n-1}$ 的 $A$-线性组合,
  所以 $A[b]$ 是有限生成 $A$-模。

  $(2)\Rightarrow (3)$ 取 $C=A[b]$ 即可。

  $(3)\Rightarrow (4)$ 取 $M=C$,若 $f(b)\in A[b]$ 使得 $f(b)M=0$,特别地,
  有 $f(b)=f(b)\cdot 1=0$,故 $\Ann(M)=0$,所以 $M$ 是忠实的 $A[b]$-模。

  $(4)\Rightarrow (1)$ 根据 \autoref{prop:cayley},取 $\phi:M\to M$
  为 $\phi(x)=bx$,那么存在 $a_i\in A$ 使得
  \[
    \phi^n+a_1\phi^{n-1}+\cdots+a_n=0,  
  \]
  任取非零的 $x\in M$,那么 $\phi^k(x)=b^k x$,所以
  \[
    (b^n+a_1b^{n-1}+\cdots+a_n)x=0,  
  \]
  $M$ 是忠实的 $A[b]$-模表明 $b^n+a_1b^{n-1}+\cdots+a_n=0$,即 $b\in B$ 在 $A$ 上是整的。
\end{proof}

\begin{corollary}\label{coro:f.g. algebra is f.g. module}
  令 $b_i\ (1\leq i\leq n)$ 是 $B$ 的元素,并且在 $A$ 上是整的。那么环
  $A[b_1,\dots,b_n]$ 是有限生成 $A$-模。
\end{corollary}
\begin{proof}
  对 $n$ 归纳,$n=1$ 的时候成立。设 $n=k$ 的时候成立,那么
  $A[b_1,\dots,b_{k+1}]=A[b_1,\dots,b_k][b_{k+1}]$,并且 $b_{k+1}$
  显然在 $A[b_1,\dots,b_k]$ 上是整的,所以 $A[b_1,\dots,b_{k+1}]$
  是有限生成 $A$-模。
\end{proof}

\begin{corollary}
  $B$ 中所有在 $A$ 上整的元素的集合 $C$ 是 $B$ 的包含 $A$ 的子环。
\end{corollary}
\begin{proof}
  如果 $c_1,c_2\in C$,那么 $A[c_1,c_2]$ 是有限生成 $A$-模,此时
  $A[c_1+c_2]\subseteq A[c_1,c_2]$,根据 \autoref{prop:integral dependence}
  的 $(3)$,所以 $c_1+c_2$ 在 $A$ 上是整的,故 $c_1+c_2\in C$。
  同理 $c_1-c_2,c_1c_2\in C$,故 $C$ 是子环。显然 $C$ 包含 $A$。
\end{proof}

上述推论中的 $C$ 被称为 $A$ 在 $B$ 中的\emph{整闭包}。如果 $C=A$,那么说
$A$ 在 $B$ 中是\emph{整闭的}。如果 $C=B$,那么说 $B$ 在 $A$ 上是整的。

\begin{corollary}
  若 $A\subseteq B\subseteq C$ 是环,$B$ 在 $A$ 上是整的,$C$ 在 $B$ 上是
  整的,那么 $C$ 在 $A$ 上是整的。
\end{corollary}
\begin{proof}
  任取 $c\in C$,那么 $c$ 满足方程
  \[
    x^n+b_1x^{n-1}+\cdots+b_n=0\quad (b_i\in B),  
  \]
  由于 $b_i$ 在 $A$ 上是整的,所以 $A[b_1,\dots,b_n]$ 是有限生成 $A$-模,
  此时 $c$ 在 $A[b_1,\dots,b_n]$ 上是整的,故 $A[b_1,\dots,b_n,c]$ 是
  有限生成 $A[b_1,\dots,b_n]$-模,所以 $A[b_1,\dots,b_n,c]$ 是
  有限生成 $A$-模,故 $c$ 在 $A$ 上是整的。
\end{proof}

\begin{corollary}
  令 $A\subseteq B$ 是环,$C$ 为 $A$ 在 $B$ 中的整闭包,那么 $C$ 在 
  $B$ 中是整闭的。
\end{corollary}
\begin{proof}
  假设 $b\in B$ 在 $C$ 上是整的,那么 $b$ 在 $A$ 上是整的,所以 $b\in C$。
\end{proof}

下面的命题表明商和分式化的操作保持整相关性质。

\begin{proposition}\label{prop:quotient preserve integral dependence}
  令 $A\subseteq B$ 是环,$B$ 在 $A$ 上是整的。
  \begin{enumerate}
    \item 如果 $\ideal b$ 是 $B$ 的理想,$\ideal a=\ideal b^c=A\cap\ideal b$,
    那么 $B/\ideal b$ 在 $A/\ideal a$ 上是整的。
    \item 如果 $R$ 是平坦 $A$-代数,那么 $B\otimes_AR$ 在 $R$ 上是整的。
    \item 如果 $S$ 是 $A$ 的乘性子集,那么 $S^{-1}B$ 在 $S^{-1}A$ 上是整的。
  \end{enumerate}
\end{proposition}
\begin{proof}
  (1) 对于环同态 $f:A\to B/\ideal b$,显然 $\ideal a=\ker f$,故
  $f$ 诱导出单的环同态 $\bar f:A/\ideal a\to B/\ideal b$,
  故 $A/\ideal a$ 可以视为 $B/\ideal b$ 的子环。
  任取 $b+\ideal b\in B/\ideal b$,由于 $b$ 在 $A$ 上是整的,所以
  \[
    b^n+a_{1}b^{n-1}+\cdots+a_n=0\quad (a_i\in A),  
  \]
  那么
  \begin{align*}
    (b+\ideal b)^n+\bar f(a_1+\ideal a)(b+\ideal b)^{n-1}+\cdots+\bar f(a_n+\ideal a)=
    (b^n+a_{1}b^{n-1}+\cdots+a_n)+\ideal b=0,
  \end{align*}
  所以 $b+\ideal b$ 在 $A/\ideal a$ 上是整的。

  (2) $R$ 平坦表明 $R\simeq A\otimes_AR\to B\otimes_AR$ 是单射,
  故通过 $r\mapsto 1\otimes r$,$R$ 可以视为 $B\otimes_AR$ 的子环。任取 $b\otimes r\in B\otimes_AR$,
  由于 $b$ 在 $A$ 上是整的,所以
  \[
    b^n+a_{1}b^{n-1}+\cdots+a_n=0\quad (a_i\in A),  
  \]
  故
  \begin{gather*}
    (b\otimes r)^n+(a_1r)(b\otimes r)^{n-1}+\cdots+(a_{n-1}r^{n-1})(b\otimes r)+a_nr^n\\
    =b^n\otimes r^n+(a_1b^{n-1})\otimes r^n+\cdots+(a_{n-1}b)\otimes r^n+a_n\otimes r^n=0,
  \end{gather*}
  所以 $b\otimes r$ 在 $R$ 上是整的。所以任意有限和 $\sum b_i\otimes r_i$ 在 $R$ 上是整的,
  故 $B\otimes_AR$ 在 $R$ 上是整的。
  
  (3) 在 (2) 中取 $R=S^{-1}A$ 即可,注意到 $S^{-1}A$ 是平坦 $A$-模,以及
  $B\otimes_A S^{-1}A\simeq S^{-1}B$。
\end{proof}
      
\section{上行定理}

\begin{proposition}
  令 $A\subseteq B$ 是整环,$B$ 在 $A$ 上是整的。那么 $B$ 是域当且仅当 $A$ 是域。
\end{proposition}
\begin{proof}
  假设 $A$ 是域,任取非零的 $b\in B$,取以 $b$ 为零点的 $A[x]$ 中的最低次多项式,即
  \[
    b^n+a_1b^{n-1}+\cdots+a_{n-1}b+a_n=0,  
  \]
  那么 $a_n\neq 0$,否则 $B$ 是整环表明 $b^{n-1}+a_1b^{n-2}+\cdots+a_{n-1}=0$,违背了
  最低次的假设。于是
  \[
    -b(b^{n-1}+a_1b^{n-2}+\cdots+a_{n-1})a_n^{-1}=1,
  \]
  所以 $b$ 可逆,逆元为 $-a_n^{-1}(b^{n-1}+a_1b^{n-2}+\cdots+a_{n-1})$,故 $B$ 是域。

  假设 $B$ 是域,任取非零的 $a\in A\subseteq B$,那么 $a^{-1}\in B$ 在 $A$ 上是整的,所以存在
  \[
    a^{-m}+a_1'a^{-m+1}+\cdots+a_m'=0,  
  \]
  两边乘以 $a^{m-1}$,这表明
  \[
     a^{-1}=-(a_1'+\cdots+a_m'a^{m-1})\in A,
  \]
  故 $a$ 可逆,$A$ 为域。
\end{proof}

\begin{corollary}\label{coro:maximal ideal of integral}
  令 $A\subseteq B$ 是环,$B$ 在 $A$ 上是整的,令 $\ideal q$ 是 $B$ 是素理想,
  $\ideal p=\ideal q^c=A\cap\ideal q$,那么 $\ideal q$ 是极大理想当且仅当
  $\ideal p$ 是极大理想。
\end{corollary}
\begin{proof}
  根据 \autoref{prop:quotient preserve integral dependence},$B/\ideal q$ 
  在 $A/\ideal p$ 上是整的,并且二者都是整环,故 $\ideal q$ 是极大理想当且仅当
  $B/\ideal q$ 是域当且仅当 $A/\ideal p$ 是域当且仅当 $\ideal p$ 是极大理想。
\end{proof}

\begin{corollary}\label{coro:contraction of prime ideal}
  令 $A\subseteq B$ 是环,$B$ 在 $A$ 上是整的,令 $\ideal q,\ideal q'$ 是 $B$
  的素理想并且满足 $\ideal q\subseteq\ideal q'$ 以及 $\ideal q^c=\ideal q'^c=\ideal p$,
  那么 $\ideal q=\ideal q'$。
\end{corollary}
\begin{proof}
  根据 \autoref{prop:quotient preserve integral dependence},$B_{\ideal p}$
  在 $A_{\ideal p}$ 上是整的,令 $\ideal m$ 为 $\ideal p$ 在 $A_{\ideal p}$ 中的扩张,
  $\ideal n,\ideal n'$ 分别为 $\ideal q,\ideal q'$ 在 $B_{\ideal p}$ 中的扩张。
  那么 $\ideal m$ 是 $A_{\ideal p}$ 的极大理想,$\ideal n\subseteq\ideal n'$,并且
  $\ideal n^c=\ideal n'^c=\ideal m$,根据上面的推论,所以
  $\ideal n,\ideal n'$ 是极大理想,故 $\ideal n=\ideal n'$,所以 $\ideal q=\ideal q'$。
\end{proof}

\begin{theorem}\label{thm:lying over}
  令 $A\subseteq B$ 是环,$B$ 在 $A$ 上是整的,$\ideal p$ 是 $A$ 的素理想,
  那么存在 $B$ 的素理想 $\ideal q$ 使得 $\ideal q\cap A=\ideal p$。
\end{theorem}
\begin{proof}
  根据 \autoref{prop:quotient preserve integral dependence},$B_{\ideal p}$
  在 $A_{\ideal p}$ 上是整的,我们有交换图
  \[
    \begin{tikzcd}
      A\arrow[r,"j"]\arrow[d,"\alpha"'] & B\arrow[d,"\beta"] \\
      A_{\ideal p}\arrow[r,"j_{\ideal p}"] & B_{\ideal p}
    \end{tikzcd}  
  \]
  $j$ 和 $j_{\ideal p}$ 都是单射。取 $B_{\ideal p}$ 的一个极大理想 $\ideal n$,
  根据 \autoref{coro:maximal ideal of integral},$\ideal n^c=\ideal n\cap A_{\ideal p}$
  是极大理想,$A_{\ideal p}$ 是局部环,所以 $\ideal m=\ideal n^c$ 是 $A_{\ideal p}$
  的唯一的极大理想,所以 $\alpha^{-1}(\ideal m)=\ideal p$。令 $\ideal q=\beta^{-1}(\ideal n)$,那么
  \[
    \ideal q\cap A=j^{-1}(\ideal q)=(\beta\circ j)^{-1}(\ideal n)=
    (j_{\ideal p}\circ \alpha)^{-1}(\ideal n)=\alpha^{-1}(\ideal m)=\ideal p.\qedhere
  \]
\end{proof}

\begin{theorem}[上行定理]
  令 $A\subseteq B$ 是环,$B$ 在 $A$ 上是整的,$\ideal p_1\subseteq\cdots\subseteq\ideal p_n$
  是 $A$ 的素理想链,$\ideal q_1\subseteq\cdots\subseteq\ideal q_m\ (m< n)$ 是 
  $B$ 的素理想链,并且 $\ideal q_i\cap A=\ideal p_i\ (1\leq i\leq m)$,
  那么 $\ideal q_1\subseteq\cdots\subseteq\ideal q_m$ 可以被扩张到链
  $\ideal q_1\subseteq\cdots\subseteq\ideal q_n$ 使得 $\ideal q_i\cap A=\ideal p_i\ (1\leq i\leq n)$。
  如 \autoref{fig:going-up} 所示。
\end{theorem}

\begin{figure}[htb]
    \centering
    \begin{tikzpicture}[
      compare/.code n args={4}{%
        \ifnum#1>#2
          \pgfkeysalso{#3}
        \else
          \pgfkeysalso{#4}
        \fi
      }
    ]
      \matrix (M) [matrix of math nodes,row sep=2.4em,column sep=1em]
      {
        B\vphantom{\ideal q_1} &[5mm] \ideal q_1 & \ideal q_2 & \cdots & \ideal q_m & |[red]|\ideal q_{m+1} &
        |[red]|\cdots & |[red]|\ideal q_n \\
        A & \ideal p_1 & \ideal p_2 & \cdots & \ideal p_m & \ideal p_{m+1} &
        \cdots & \ideal p_n \\
      };
      \foreach \i [evaluate=\i as \j using int(\i+1)] in {2,...,7}
      {
        \node [compare={\i}{4}{red}{black}] at ($(M-1-\i)!0.5!(M-1-\j)$) {$\subseteq$};
        \node at ($(M-2-\i)!0.5!(M-2-\j)$) {$\subseteq$};
        \ifnum\i=4
        \else
          \ifnum\i=7
          \else
            \draw [compare={\i}{5}{red}{black}] (M-1-\i) -- (M-2-\i);
          \fi
        \fi
      }
      \draw [red] (M-1-8) -- (M-2-8);
      \draw (M-1-1) -- (M-2-1);
    \end{tikzpicture}
    \caption{上行定理}\label{fig:going-up}
\end{figure}

\begin{proof}
  我们只需要证明存在 $\ideal q_{m+1}$
  使得 $\ideal q_{m+1}\cap A=\ideal p_{m+1}$ 即可。
  根据 \autoref{prop:quotient preserve integral dependence},$\bar B=B/\ideal q_{m}$
  在 $\bar A=A/\ideal p_m$ 上是整的,我们有交换图
  \[
    \begin{tikzcd}
      A\arrow[r,"j"]\arrow[d,"\pi_1"'] & B\arrow[d,"\pi_2"] \\
      \bar A\arrow[r,"\bar j"] & \bar B
    \end{tikzcd}  
  \]
  $\bar{\ideal p}_{m+1}$ 是 $\bar A$ 的素理想,
  根据 \autoref{thm:lying over},存在 $\bar B$ 的素理想 $\bar{\ideal q}_{m+1}$
  使得 $\bar j^{-1}(\bar{\ideal q}_{m+1})=\bar{\ideal p}_{m+1}$,
  $\bar{\ideal q}_{m+1}$ 对应到 $B$ 的素理想 $\ideal q_{m+1}$,此时
  \[
    \ideal q_{m+1}\cap A=j^{-1}(\ideal q_{m+1})
    =(\pi_2\circ j)^{-1}(\bar{\ideal q}_{m+1})=(\bar j\circ\pi_1)^{-1}(\bar{\ideal q}_{m+1})
    =\ideal p_{m+1}.\qedhere
  \]
\end{proof}

\section{整闭整环和下行定理}

\begin{proposition}\label{prop:integrally closed in fraction}
  令 $A\subseteq B$ 是环,$C$ 是 $A$ 在 $B$ 中的整闭包,令 $S$ 是 $A$ 的乘性子集,
  那么 $S^{-1}C$ 是 $S^{-1}A$ 在 $S^{-1}B$ 中的整闭包。
\end{proposition}
\begin{proof}
  根据 \autoref{prop:quotient preserve integral dependence},$S^{-1}C$ 在 $S^{-1}A$ 
  上是整的。任取 $b/s\in S^{-1}B$,若 $b/s$ 在 $S^{-1}A$ 上是整的,
  则 $b/s$ 满足某个 $S^{-1}A$ 为系数的方程
  \[
    (b/s)^n+(a_1/s_1)(b/s)^{n-1}+\cdots+(a_n/s_n)=0,  
  \]
  令 $t=s_1\cdots s_n$,两边同时乘以 $(st)^n/1$,那么
  \[
    (bt)^n/1+(a_1st/s_1)(bt)^{n-1}/1+\cdots+(a_ns^nt^n/s_n)=0,
  \]
  所以存在 $u\in S$ 使得
  \[
    u(bt)^n+ua_1' (bt)^{n-1}+\cdots+ua_n'=0,
  \] 
  两边同时乘以 $u^{n-1}$,那么
  \[
    (btu)^n+ua_1'(btu)^{n-1}  +\cdots+u^na_n'=0,
  \]
  这表明 $btu\in C$,所以 $b/s=btu/stu\in S^{-1}C$,所以 $S^{-1}C$ 是 $S^{-1}A$ 在 $S^{-1}B$ 中的整闭包。
\end{proof}

一个整环被称为\emph{整闭的},如果它在它的分式域中是整闭的。
例如,$\mathbb{Z}$ 是整闭的。

\begin{example}
  UFD 都是整闭的。设 $R$ 是 UFD,任取 $x/y\in \Frac(R)$,$y\neq 0$,
  利用唯一因子分解,可以假设 $x,y$ 互素。若 $x/y$ 在 $R$ 上是整的,
  那么 $x/y$ 满足
  \[
    (x/y)^n+r_1(x/y)^{n-1}+\cdots+r_n=0\quad (r_i\in R),
  \]
  两边乘以 $y^n$,那么
  \[
    x^n+r_1yx^{n-1}  +\cdots+r_ny^n=0,
  \]
  这表明 $y\mid x^n$,$x,y$ 互素表明 $y\in R^\times$,所以 $x/y=xy^{-1}/1\in R$,
  故 $R$ 在 $\Frac(R)$ 中整闭。
\end{example}

整闭是局部性质。

\begin{proposition}
  令 $A\subseteq B$ 是环,那么下面的说法是等价的:
  \begin{enumerate}
    \item $A$ 在 $B$ 中整闭;
    \item 对于每个素理想 $\ideal p\subseteq A$,$A_{\ideal p}$ 在 $B_{\ideal p}$
    中是整闭的;
    \item 对于每个极大理想 $\ideal m\subseteq A$,$A_{\ideal m}$ 在 $B_{\ideal m}$
    中是整闭的。
  \end{enumerate}
\end{proposition}
\begin{proof}
  $(1)\Rightarrow (2)$ \autoref{prop:integrally closed in fraction}。
  $(2)\Rightarrow (3)$ 极大理想都是素理想。

  $(3)\Rightarrow (1)$ 令 $C$ 为 $A$ 在 $B$ 中的整闭包,证明单位映射 $f:A\to C$ 是满射即可。
  根据 \autoref{prop:injective is local},等价于证明 $f_{\ideal m}:A_{\ideal m}\to C_{\ideal m}$
  是满射。根据 \autoref{prop:integrally closed in fraction},$C_{\ideal m}$ 是 $A_{\ideal m}$
  在 $B_{\ideal m}$ 中的整闭包,所以 $f_{\ideal m}$ 满射。
\end{proof}

\begin{corollary}
  令 $A$ 是整环,那么下面的说法是等价的:
  \begin{enumerate}
    \item $A$ 是整闭的;
    \item 对于每个素理想 $\ideal p\subseteq A$,$A_{\ideal p}$ 是整闭的;
    \item 对于每个极大理想 $\ideal m\subseteq A$,$A_{\ideal m}$ 是整闭的。
  \end{enumerate}
\end{corollary}

令 $A\subseteq B$ 是环,$\ideal a$ 是 $A$ 的理想。元素 $b\in B$ 如果满足
首一的以 $\ideal a$ 中元素为系数的多项式,那么我们说 $b$ \emph{在 $\ideal a$ 上是整的}。
$\ideal a$ \emph{在 $B$ 中的整闭包}指的是 $B$ 中所有在 $\ideal a$ 上整的元素的集合。

\begin{lemma}\label{lemma:closure of ideal}
  令 $C$ 是 $A$ 在 $B$ 中的整闭包,$\ideal aC$ 为 $\ideal a$ 在 $C$ 中的扩张。
  那么 $\ideal a$ 在 $B$ 中的整闭包是根式理想 $\sqrt{\ideal aC}$(从而对加减乘封闭)。
\end{lemma}
\begin{proof}
  若 $x\in B$ 在 $\ideal a$ 上是整的(自然在 $A$ 上也是整的),那么 $x$ 满足
  \[
    x^n+a_1x^{n-1}+\cdots+a_n=0\quad (a_i\in\ideal a),  
  \]
  所以 $x^n\in\ideal aC$,所以 $x\in\sqrt{\ideal aC}$。

  反过来,若 $x\in\sqrt{\ideal aC}$,那么存在 $n$ 使得 $x^n=\sum_{i=1}^m a_ic_i$,
  其中 $a_i\in\ideal a,c_i\in C$。因为 $c_1,\dots,c_m$ 在 $A$ 上是整的,
  所以 $M=A[c_1,\dots,c_m]$ 是有限生成 $A$-模。考虑 $\phi:M\to M$ 为
  $\phi(f(c_1,\dots,c_m))=x^nf(c_1,\dots,c_m)$,那么 $\phi$ 是 $A$-模同态,
  并且 $\phi(M)=x^nM\subseteq\ideal aM$,
  根据 Cayley-Hamilton 定理,存在 $a_i'\in \ideal a$ 使得
  \[
    \phi^k+a_1'\phi^{k-1}+\cdots+a_k'=0,  
  \]
  由于 $\phi(1)=x^n$,所以 
  \[
    x^{kn}+a_1'x^{(k-1)n}+\cdots+a_k'=0,
  \]
  即 $x$ 在 $\ideal a$ 上是整的。
\end{proof}

若 $A\subseteq B$ 是整环,$x\in B$ 在 $A$ 上是整的,$x$ 在 $\Frac(A)$ 上当然是代数的,
但是一般情况下,$x$ 在 $\Frac(A)$ 上的极小多项式不一定处于 $A[x]$ 中。例如,
记 $\omega=e^{2\pi i/3}$,$A=\mathbb{Z}[\sqrt{-3}]$,
$B=\mathbb{Z}[\omega]=\mathbb{Z}[\frac{1+\sqrt{-3}}{2}]$,$F=\Frac(A)=\mathbb{Q}(\sqrt{-3})$。
$\omega$ 在 $F$ 上的极小多项式显然为 $x-\omega$,但是 $\omega\notin A$。如果
$A$ 是整闭的,那么这个结论是正确的,即下面的命题。这个例子中,$B$
在 $A$ 上是整的,并且 $B$ 是 ED,从而是 UFD,从而是整闭的,所以 $A$ 在 $F$
中的整闭包是 $B$。

\begin{proposition}\label{prop:minimal polynominal of x}
  令 $A\subseteq B$ 是整环,$A$ 是整闭的,$x\in B$ 在 $A$ 的某个理想 $\ideal a$ 
  上是整的。那么 $x$ 在分式域 $K=\Frac(A)$ 上是代数的。此外,设 $x$ 在 $K$ 上的极小多项式
  为 $t^n+a_1t^{n-1}+\cdots+a_n$,那么 $a_1,\dots,a_n\in\sqrt{\ideal a}$。
\end{proposition}
\begin{proof}
  显然 $x$ 在 $K$ 上是代数的。记 $L$ 为 $f(t)=t^n+a_1t^{n-1}+\cdots+a_n$ 的分裂域,
  $f(t)$ 在 $L$ 上分裂为 $f(t)=(t-\alpha_1)\cdots(t-\alpha_n)$,不妨设 $\alpha_1=x$。
  那么我们有 $K(\alpha_i)\simeq K[t]/(f(t))\simeq K(x)$。
  设 $x$ 满足
  \[
    x^m+a_1'x^{m-1}+\cdots +a_m'=0\quad (a_i'\in \ideal a),  
  \]
  将上式放到 $L$ 中,由于 $K(\alpha_i)\simeq K(x)$,那么对于 $\alpha_i$,同样有 
  \[
    \alpha_i^m+a_1'\alpha_i^{m-1}+\cdots +a_m'=0,  
  \]
  所以 $\alpha_i$ 在 $\ideal a$ 上是整的。由 Vieta 定理,$a_i$ 可以表示为 $\alpha_1,\dots,\alpha_n$
  的对称多项式,考虑 $A\subseteq L$,根据 \autoref{lemma:closure of ideal},
  所以 $a_i$ 在 $\ideal a$ 上是整的(封闭性)。
  由于 $a_i\in K$,考虑 $A\subseteq K$,
  再根据 \autoref{lemma:closure of ideal},$a_i\in\sqrt{\ideal a^e}=\sqrt{\ideal a}$。
\end{proof}

\begin{theorem}[下行定理]
  令 $A\subseteq B$ 是整环,$A$ 是整闭的,$B$ 在 $A$ 上是整的。令 
  $\ideal p_1\supseteq \cdots\supseteq\ideal p_n$ 是 $A$ 的素理想链,
  $\ideal q_1\supseteq\cdots\supseteq \ideal q_m\ (m< n)$ 是 $B$ 的素理想链,
  并且 $\ideal q_i\cap A=\ideal p_i\ (1\leq i\leq m)$。那么 
  $\ideal q_1\supseteq\cdots\supseteq \ideal q_m$ 可以被扩张到链
  $\ideal q_1\supseteq\cdots\supseteq \ideal q_n$ 使得 $\ideal q_i\cap A=\ideal p_i\ (1\leq i\leq n)$。
  如 \autoref{fig:going-down} 所示。
\end{theorem}

\begin{figure}[htb]
  \centering
  \begin{tikzpicture}[
    compare/.code n args={4}{%
      \ifnum#1>#2
        \pgfkeysalso{#3}
      \else
        \pgfkeysalso{#4}
      \fi
    }
  ]
    \matrix (M) [matrix of math nodes,row sep=2.4em,column sep=1em]
    {
      B\vphantom{\ideal q_1} &[5mm] \ideal q_1 & \ideal q_2 & \cdots & \ideal q_m & |[red]|\ideal q_{m+1} &
      |[red]|\cdots & |[red]|\ideal q_n \\
      A & \ideal p_1 & \ideal p_2 & \cdots & \ideal p_m & \ideal p_{m+1} &
      \cdots & \ideal p_n \\
    };
    \foreach \i [evaluate=\i as \j using int(\i+1)] in {2,...,7}
    {
      \node [compare={\i}{4}{red}{black}] at ($(M-1-\i)!0.5!(M-1-\j)$) {$\supseteq$};
      \node at ($(M-2-\i)!0.5!(M-2-\j)$) {$\supseteq$};
      \ifnum\i=4
      \else
        \ifnum\i=7
        \else
          \draw [compare={\i}{5}{red}{black}] (M-1-\i) -- (M-2-\i);
        \fi
      \fi
    }
    \draw [red] (M-1-8) -- (M-2-8);
    \draw (M-1-1) -- (M-2-1);
  \end{tikzpicture}
  \caption{下行定理}\label{fig:going-down}
\end{figure}

\begin{proof}
  我们只需要证明存在 $\ideal q_{m+1}\subseteq\ideal q_m$ 使得 $\ideal q_{m+1}\cap A=\ideal p_{m+1}$ 即可,
  我们有交换图
  \[
    \begin{tikzcd}
      A\arrow[r,"j"]\arrow[d,"\alpha"'] & B\arrow[d,"\beta"]\\
      A_{\ideal p_{m}}\arrow[r,"j'"] & B_{\ideal q_{m}}
    \end{tikzcd}  
  \]
  注意由于 $A-\ideal p_{m}\subseteq B-\ideal q_m$,所以 $A_{\ideal p_m}\to B_{\ideal q_m}$
  的映射是良定义的。考虑 $\beta\circ j:A\to B_{\ideal q_m}$,注意 $A,B$
  是整环表明 $\alpha,\beta$ 是单射。由于 $B$ 的被 $\ideal q_m$ 包含的素理想一一
  对应到 $B_{\ideal q_m}$ 的素理想,
  所以等价于证明 $\ideal p_{m+1}$ 是 $B_{\ideal q_m}$ 的某个素理想的收缩。根据
  \autoref{prop:condition of contraction},等价于证明
  $\bigl(\ideal p_{m+1}B_{\ideal q_m}\bigr)\cap A=\ideal p_{m+1}$(由于 $\beta\circ j$
  是单射,所以可以将 $A$ 视为 $B_{\ideal q_m}$ 的子环)。
  由于我们总是有 $\bigl(\ideal p_{m+1}B_{\ideal q_m}\bigr)\cap A\supseteq \ideal p_{m+1}$,
  所以只需要证明 $\bigl(\ideal p_{m+1}B_{\ideal q_m}\bigr)\cap A\subseteq\ideal p_{m+1}$。

  任取 $y/s\in\bigl(\ideal p_{m+1}B_{\ideal q_m}\bigr)\cap A$,其中 $y\in \ideal p_{m+1}B$,
  $s\in B-\ideal q_m$,根据 \autoref{lemma:closure of ideal},所以 $y$ 在 $\ideal p_{m+1}$
  上是整的。根据 \autoref{prop:minimal polynominal of x},$y$ 在 $\Frac(A)$ 上的极小多项式
  为
  \[
    y^n+u_1y^{n-1}+\cdots+u_n=0\quad (u_i\in\sqrt{\ideal p_{m+1}}=\ideal p_{m+1}).  
  \]

  $y/s\in A$ 说明存在 $x\in A$ 使得 $y/s=x/1$,即存在 $t\in B-\ideal q_m$ 使得
  $t(y-sx)=0$,$t\neq 0$ 表明 $y=sx$,那么在 $\Frac(B)\supseteq \Frac(A)$ 中有 $s=yx^{-1}$。
  那么 $s$ 在 $\Frac(A)$ 上的极小多项式为
  \[
    s^n+v_1s^{n-1}+\cdots+s_n=0,  
  \]
  其中 $v_i=u_i/x^i$。所以,我们有 $x^iv_i=u_i\in\ideal p_{m+1}$。

  因为 $s$ 在 $A$ 上是整的,根据 \autoref{prop:minimal polynominal of x},
  所以 $s$ 在 $\Frac(A)$ 上的极小多项式的系数在 $A$ 中,即 $v_i\in A$。
  假设 $x\notin \ideal p_{m+1}$,那么 $x^iv_i\in\ideal p_{m+1}$ 表明 $v_i\in\ideal p_{m+1}$,
  所以 $s^n\in\ideal p_{m+1}B\subseteq \ideal p_mB\subseteq \ideal q_m$,所以 $s\in\ideal q_m$,
  矛盾。故 $x\in\ideal p_{m+1}$,即 $\bigl(\ideal p_{m+1}B_{\ideal q_m}\bigr)\cap A\subseteq\ideal p_{m+1}$。
\end{proof}

下面我们简要回顾双线性型的一些内容。令 $V$ 是有限维 $F$-向量空间,$B:V\times V\to F$
是一个双线性型($F$-双线性映射)。用张量积的性质来说,$B$ 对应一个 $V\otimes_F V\to F$
的线性映射,即 $B\in (V\otimes_F V)^*$。$B$ 诱导出两个 $V\to V^*$ 的线性映射,即
\[
  B_l:x\mapsto (B(x,\uline{}):y\mapsto B(x,y)) ,\quad
  B_r:x\mapsto (B(\uline{},x):y\mapsto B(y,x)) .
\]
如果上述两个线性映射是单射,那么我们说 $B$ 是\emph{非退化的}。
由于 $\dim V=\dim V^*$,所以 $B$ 是非退化的当且仅当 $B_l$ 和 $B_r$
是同构。设 $V$ 的一组基为 $e_1,\dots,e_n$。记矩阵 $\mathbold B=\bigl(B(e_i,e_j)\bigr)_{n\times n}$,
对于 $u,v\in V$,设 $u=u_1e_1+\cdots+u_ne_n$ 以及 $v=v_1e_1+\cdots+v_ne_n$,
那么
\[
  B(u,v)=B\left(\sum_i u_ie_i,\sum_j v_je_j\right)  =
  \sum_i\sum_j u_iv_jB(e_i,e_j)=\mathbold u^\top\mathbold B\mathbold v,
\]
其中 $\mathbold u=(u_1,\dots,u_n)$ 以及 $\mathbold v=(v_1,\dots,v_n)$。
不难证明 $B$ 非退化当且仅当 $\mathbold B$ 可逆。

现在假设 $B$ 是对称双线性型。如果 $B$ 非退化,那么 $B_l:V\to V^*$
是同构。设 $V$ 的一组基为 $e_1,\dots,e_n$,那么 $V^*$ 的基为 $e_1^*,\dots,e_n^*$,满足
$e_i^*(e_j)=\delta_{ij}$。于是存在基 $v_1,\dots,v_n\in V$ 使得 $B(v_i,\uline)=e_i^*$,
即 $B(v_i,e_j)=\delta_{ij}$。

然后我们简要介绍一下关于域扩张中迹的概念。令 $K/F$ 是有限扩张,$[K:F]=n$。对于 $a\in K$,
定义映射 $L_a:K\to K$ 为 $L_a(x)=ax$。容易验证 $L_a$ 为 $F$-线性映射,那么我们可以定义
元素 $a$ 的迹为线性映射 $L_a$ 的迹,即
\[
  \Tr_{K/F}(a)=\Tr(L_a).  
\]
这诱导了 $F$-线性映射 $\Tr_{K/F}:K\to F$。当 $c\in F$ 的时候,设 $K$ 的一组基
为 $v_1,\dots,v_n$,那么 $L_c$ 在这组基下的表示矩阵为 $cI_n$,所以
$\Tr_{K/F}(c)=cn$。

$\Tr_{K/F}$ 也诱导了一个 $F$-双线性映射 $\Tr_{K/F}:K\times K\to F$,即
$\Tr_{K/F}(a,b)=\Tr_{K/F}(ab)$。容易验证 $\Tr_{K/F}$ 是对称双线性型。
我们不加证明地给出结论:$K/F$ 是可分扩张当且仅当 $\Tr_{K/F}$ 是非退化的。

\begin{proposition}\label{prop:basis of integral closure}
  $A$ 是任意整闭整环,$K$ 是 $A$ 的分式域,$L$ 是 $K$ 的有限可分扩张,$B$ 是 $A$ 
  在 $L$ 中的整闭包。那么存在 $K$-向量空间 $L$ 的一组基 $v_1,\dots,v_n$
  使得 $B\subseteq \sum_{j=1}^n Av_j$。
\end{proposition}
\begin{proof}
  设 $[L:K]=n$。首先我们断言:存在 $L$ 的一组基 $u_1,\dots,u_n$ 使得 $u_i\in B$。
  任取 $L$ 的一组基 $u_1',\dots,u_n'$,那么 $u_i'$ 在 $K$ 上是代数的,所以满足方程
  (总是可以乘以公分母使得系数处于 $A$ 中)
  \[
    a_m(u_i')^m+a_{m-1}(u_i')^{m-1}+\cdots+  a_0=0\quad (a_m\neq 0, a_j\in A).
  \]
  两边乘以 $a_m^{m-1}$,所以
  \[
    (a_mu_i')^m+a_{m-1}a_m(a_mu_i')^{m-1}+\cdots+a_0 a_m^{m-1}=0,  
  \]
  令 $u_i=a_mu_i'$,那么 $u_i$ 在 $A$ 上是整的,并且仍然是 $L$ 的一组基。

  根据上面的叙述,$L/K$ 可分表明 $\Tr_{L/K}$ 是非退化的,所以存在 $L$ 的一组基 $v_1,\dots,v_n$
  使得 $\Tr_{L/K}(u_iv_j)=\delta_{ij}$。任取 $x\in B$,设 $x=\sum_j x_jv_j$,其中 $x_j\in K$,
  那么
  \[
    x_j=\sum_i\delta_{ji}x_i=\sum_i\Tr_{L/K}(u_jv_i)x_i
    =\Tr_{L/K}(xu_j),
  \]
  所以 $x=\sum_j \Tr_{L/K}(xu_j)v_j$。下面我们说明,对于任意的 $y\in B$,都有
  $\Tr_{L/K}(y)\in A$ 即可。由于
  \begin{align*}
    \Tr_{L/K}(y)&=\Tr_{K(y)/K}\circ \Tr_{L/K(y)}(y)=\Tr_{K(y)/K}\bigl([L:K(y)]\cdot y\bigr)\\
    &=[L:K(y)]\cdot \Tr_{K(y)/K}(y).
  \end{align*}
  设 $[K(y):K]=m$,那么 $K(y)$ 的一组基为 $1,y,\dots,y^{m-1}$。设 $y$ 在 $K$
  上的极小多项式为
  \[
    y^m+a_{m-1}  y^{m-1}+\cdots+a_0=0,
  \]
  根据 \autoref{prop:minimal polynominal of x},$a_i\in A$。线性映射 $L_y:K(y)\to K(y)$
  在上述基下的表示矩阵为
  \[
    \begin{pmatrix*}
      0 & & & & -a_0\\
      1 & 0 & && -a_1\\
      & 1 & \ddots & & \vdots\\
      & & \ddots& 0 &-a_{m-1}\\
      & & & 1 & -a_m\\
    \end{pmatrix*}  ,
  \]
  所以 $\Tr_{L/K}(y)=[L:K(y)]\cdot (-a_{m})\in A$。
\end{proof}

\section{赋值环}

$K$ 是一个域,$K$ 的一个\emph{离散赋值}指的是一个映射 $v:K^\times\to\mathbb{Z}$,
其满足
\begin{enumerate}
  \item $v(xy)=v(x)+v(y)$,即 $v$ 是群同态;
  \item $v(x+y)\geq\min(v(x),v(y))$。
\end{enumerate}
为了方便起见,定义 $v(0)=+\infty$。容易观察到这蕴含着 $v(1)=v(-1)=0$ 以及
$v(x)=v(-x)$。

\begin{example}
  \mbox{}
  \begin{enumerate}
    \item 固定素数 $p\in\mathbb{Z}$,任意的 $x\in\mathbb{Q}^\times$ 可以唯一地写为
    $x=p^r\frac{a}{b}$,其中 $a,b$ 互素并且 $p\nmid a$ 以及 $p\nmid b$。定义
    $v_p:\mathbb{Q}^\times \to\mathbb{Z}$ 为 $v_p(x)=r$。
    对于 $x=p^r\frac{a}{b}$ 以及 $y=p^s\frac{c}{d}$,有 $xy=p^{r+s}\frac{ac}{bd}$,
    此时 $p\nmid ac$ 以及 $p\nmid bd$,所以 $v_p(xy)=r+s=v_p(x)+v_p(y)$。
    不妨设 $r\leq s$,我们有
    \[
      x+y=p^r\left(\frac{a}{b}+p^{s-r}\frac{c}{d}\right)=p^r\frac{ad+p^{s-r}bc}{bd}  ,
    \]
    此时 $p\nmid bd$,当 $r<s$ 时有 $p\nmid (ad+p^{s-r}bc)$,此时 $v_p(x+y)=r$。
    当 $r=s$ 时 $ad+bc$ 可能含有 $p$ 作为因子,此时 $v_p(x+y)\geq r$。所以
    $v_p$ 是 $\mathbb{Q}$ 的离散赋值。
  \end{enumerate}
\end{example}

\begin{lemma}
  $K$ 是一个域,有离散赋值 $v:K^\times\to\mathbb{Z}$,若 $v(x)\neq v(y)$,那么
  $v(x+y)=\min(v(x),v(y))$。
\end{lemma}
\begin{proof}
  不妨设 $v(x)>v(y)$,那么 $v(x+y)\geq \min(v(x),v(y))=v(y)$。
  另一方面,我们有
  \[
    v(y)=v(x+y-x)\geq \min(v(x+y),v(-x))=  \min(v(x+y),v(x)),
  \]
  若 $v(x)\leq v(x+y)$,那么 $v(y)\geq v(x)$,与 $v(x)>v(y)$ 矛盾。所以只可能
  $v(x)>v(x+y)$,故 $v(y)\geq v(x+y)$。所以 $v(x+y)=v(y)=\min(v(x),v(y))$。
\end{proof}

\begin{proposition}
  $K$ 是一个域,有离散赋值 $v:K^\times\to\mathbb{Z}$,定义
  \[
    A=\{x\in K\,|\, v(x)\geq 0\},\quad \ideal m=\{x\in K\,|\, v(x)>0\},
  \]
  那么 
  \begin{enumerate}
    \item $A$ 是 $K$ 的一个子环,并且对于任意的 $x\in K^\times$,有 $x\in A$ 或者
    $x^{-1}\in A$(或者均有)。
    \item $\ideal m$ 是 $A$ 的唯一的极大理想。
  \end{enumerate}
\end{proposition}
\begin{proof}
  (1) $v$ 是群同态表明 $v(1)=0$,故 $1\in A$。任取 $x,y\in A$,
  那么 $v(x-y)\geq \min (v(x),v(-y))=\min(v(x),v(y))\geq 0$,
  所以 $x-y\in A$。此外,$v(xy)=v(x)+v(y)\geq 0$,所以 $xy\in A$。这就表明 $A$ 是
  $K$ 的子环。任取 $x\in K^\times$,如果 $x\notin A$,那么 $v(x)<0$,那么 $0=v(1)=v(x)+v(x^{-1})$,
  故 $v(x^{-1})>0$,所以 $x^{-1}\in A$,所以必有 $x\in A$ 或者 $x^{-1}\in A$。

  (2) 任取 $x\in\ideal m$,$a\in A$,那么 $v(ax)=v(a)+v(x)>0$,所以 $ax\in\ideal m$。
  任取 $x,y\in\ideal m$,那么 $v(x-y)\geq \min(v(x),v(y))>0$,所以 $x-y\in\ideal m$。
  这表明 $\ideal m$ 是 $A$ 的一个理想。任取 $x\in A-\ideal m$,那么
  $v(x)=0$,所以 $0=v(1)=v(x)+v(x^{-1})$ 表明 $v(x^{-1})=0$,故 $x^{-1}\in A$,
  所以 $x\in A^\times$。这就表明 $A-\ideal m=A^\times$,所以 $\ideal m$ 是 $A$
  唯一的极大理想,$A$ 是局部环。
\end{proof}

令 $B$ 是整环,$K$ 是 $B$ 的分式域,如果对于 $x\neq 0$,总是有 $x\in B$ 或者 $x^{-1}\in B$(或者都有),
那么称 $B$ 是 $K$ 的\emph{赋值环}。反之,如果 $K$ 的子环 $B$ 满足
对于 $x\neq 0$,总是有 $x\in B$ 或者 $x^{-1}\in B$,那么 $x=1/x^{-1}\in\Frac(B)$,所以
此时 $K$ 就是 $B$ 的分式域,即 $B$ 是 $K$ 的赋值环。

\begin{proposition}
  \mbox{}
  \begin{enumerate}
    \item $B$ 是局部环。
    \item 如果 $B'$ 是环使得 $B\subseteq B'\subseteq K$,那么 $B'$ 也是 $K$ 的赋值环。
    \item $B$ 是整闭的(在 $K$ 中)。
  \end{enumerate}
\end{proposition}
\begin{proof}
  令 $\ideal m$ 为 $B$ 中所有不是单位的元素的集合。若 $x\in\ideal m$,$b\in B$,
  那么 $x^{-1}\notin B$,所以 $(bx)^{-1}=b^{-1}x^{-1}\notin B$,这就表明一定有
  $bx\in B$ 且 $bx$ 在 $B$ 中不可逆,故 $bx\in\ideal m$。
  若 $x,y\in\ideal m$,那么 $xy^{-1}\in B$ 或者 $x^{-1}y\in B$,所以
  $x-y=x(1-x^{-1}y)=y(xy^{-1}-1)\in \ideal m$。这表明 $\ideal m$ 是理想,而
  $B-\ideal m=B^\times$,所以 $\ideal m$ 是 $B$ 的唯一的极大理想。

  (2) 显然。

  (3) 令 $x\in K$ 在 $B$ 上是整的。那么 $x$ 满足方程
  \[
    x^n+b_1x^{n-1}+\cdots+b_n=0\quad (b_i\in B).  
  \]
  如果 $x^{-1}\in B$,两边乘以 $x^{1-n}$,那么
  \[
    x=-(b_1+b_2x^{-1}+\cdots+b_nx^{1-n}) \in B. 
  \]
  如果 $x^{-1}\notin B$,根据定义必有 $x\in B$。所以总是有 $x\in B$,
  即 $B$ 是整闭的。
\end{proof}

\begin{proposition}
  $B$ 是域 $K$ 的赋值环当且仅当存在 $K$ 的离散赋值 $v:K^\times\to \mathbb{Z}$,
  使得
  \[
    B=\{x\in K\,|\, v(x)\geq 0\}  .
  \]
\end{proposition}

对于两个局部环 $(A,\ideal m)$ 和 $(B,\ideal n)$,我们说 $B$ 控制 $A$ 当且仅当
$A\subseteq B$ 并且 $\ideal m\subseteq\ideal n$。

\begin{theorem}\label{thm:existence of valuation ring}
  令 $K$ 是一个域,$A\subseteq K$ 是子环,$\ideal p$ 是 $A$ 的素理想,$\Omega$
  是包含剩余域 $k(\ideal p)$ 的代数闭域,那么存在 $K$ 的一个赋值环 $(B,\ideal m)$
  并且其控制 $(A_{\ideal p},\ideal pA_{\ideal p})$,此外,存在环同态 $g:B\to\Omega$
  使得 $\ker g=\ideal m$。
\end{theorem}
\begin{remark}
  若取 $\Omega$ 为 $k(\ideal p)$ 的代数闭包,该定理的含义为如下交换图:
  \[
    \begin{tikzcd}[row sep=2.4em]
      A\arrow[r] & A_{\ideal p}\arrow[r]\arrow[d,dashed,hookrightarrow,red] 
      & k(\ideal p)\arrow[r,hookrightarrow] \arrow[d,dashed,hookrightarrow,red] 
      & \overline{k(\ideal p)} \\
      & |[red]|B\arrow[r,red] & |[red]|B/\ideal m\arrow[ur,dashed,hookrightarrow,red]  & 
    \end{tikzcd}  
  \]
  也就是说:令 $K$ 是一个域,$A\subseteq K$ 是子环,$\ideal p$ 是 $A$ 的素理想,
  那么存在 $K$ 的一个赋值环 $(B,\ideal m)$ 并且其控制 $(A_{\ideal p},\ideal pA_{\ideal p})$,
  此外,$B/\ideal m$ 是 $k(\ideal p)$ 的代数扩张。
\end{remark}
\begin{proof}
  对于环 $A$,诱导出映射 $f:A\to A_{\ideal p}\to k(\ideal p)\hookrightarrow\Omega$。
  记集合 $\Sigma$ 是二元组 $(A',f')$ 的集合,其中 $A'$ 是满足 $A\subseteq A'\subseteq K$
  的子环,映射 $f':A'\to\Omega$ 满足 $f'|_A=f$。首先 $(A,f)\in\Omega$,所以 $\Omega\neq\emptyset$。
  定义 $\Omega$ 上的偏序:$(A',f')\leq (A'',f'')$ 当且仅当 $A'\subseteq A''$ 以及 $f''|_{A'}=f'$。

  上述偏序的含义如下:记 $\ideal p'=\ker f'$,由于 $f':A'\to\Omega$ 表明 $A'/\ideal p'$
  同构于 $\Omega$ 的子环是整环,所以 $\ideal p'$ 是素理想。类似地,记素理想 $\ideal p''=\ker f''$。
  那么 $f''|_{A'}=f'$ 表明 $\ideal p''\cap A'=\ideal p'$,即对于 $A'\hookrightarrow A''$ 有
  $\ideal p''^{c}=\ideal p'$。也就是说我们有如下交换图:
  \[
    \begin{tikzcd}[row sep=2.4em]
      A'\arrow[r]\arrow[d,hookrightarrow]
      & A'/\ideal p'\arrow[r]\arrow[d,hookrightarrow] 
      & k(\ideal p')\arrow[r,hookrightarrow]\arrow[d,hookrightarrow] 
      & \Omega \\
      A''\arrow[r] & A''/\ideal p''\arrow[r] & k(\ideal p'')\arrow[ur,hookrightarrow] & 
    \end{tikzcd}  
  \]

  令 $\{(A_i,f_i)\}_{i\in I}$ 是 $\Sigma$ 的一条链。令 $A_0=\bigcup_{i\in I}A_i$,对于 $x\in A_0$,
  那么存在 $i\in I$ 使得 $x\in A_i$,定义 $f_0:A_0\to\Omega$ 为 $f_0(x)=f_i(x)$。$f_0$
  是良定义的,因为若同时有 $x\in A_i$ 以及 $x\in A_j$,设 $(A_i,f_i)\leq (A_j,f_j)$,那么
  $f_i(x)=f_j(x)$。此时 $(A_0,f_0)$ 为这条链的上界,根据 Zorn 引理,$\Sigma$ 有极大元,
  记为 $(B,g)$。下面我们分三步证明 $B$ 就是我们要找的赋值环。

  Step 1. 证明 $B$ 是局部环,有极大理想 $\ideal m=\ker g$。首先 $\ideal m$ 是素理想,
  并且 $g(B-\ideal m)\neq 0$,所以 $g(B-\ideal m)\in\Omega^\times$,根据分式环的泛性质
  \ref{prop:universal property of fraction ring},存在交换图
  \[
    \begin{tikzcd}
      B\arrow[r,"g"]\arrow[d] & \Omega \\
      B_{\ideal m}\arrow[ur,dashed,"g'"'] &
    \end{tikzcd}  
  \]
  $B$ 是整环,所以 $B\to B_{\ideal m}$ 是单射,所以 $(B,g)\leq (B_{\ideal m},g')$,
  $(B,g)$ 是极大元表明 $B=B_{\ideal m}$。

  Step 2. 证明若 $x\notin B$,则 $\ideal mB[x]=B[x]$。这里 $B[x]$ 是 $K$ 中同时包含 $B$
  和 $x$ 的最小的子环。反证法,假设 $\ideal mB[x]\neq B[x]$,那么存在 $B[x]$ 的极大理想 $\ideal m'$
  包含 $\ideal mB[x]$,那么对于 $B\hookrightarrow B[x]$ 有 $\ideal m\subseteq\ideal m'\cap B$,
  同时注意到 $\ideal m'\cap B\neq B$,所以 $\ideal m$ 是 $B$ 的极大理想表明 $\ideal m'\cap B=\ideal m$。
  于是我们有如下交换图:
  \[
    \begin{tikzcd}[row sep=2.4em]
      B\arrow[r]\arrow[d,hookrightarrow] 
      & B/\ideal m\arrow[r,hookrightarrow]\arrow[d,hookrightarrow] 
      & \Omega \\
      B[x]\arrow[r]\arrow[urr,dashed,red,bend right=60,"g'"'] 
      & B[x]/\ideal m'\arrow[ur,dashed,hookrightarrow] &
    \end{tikzcd}  
  \]
  \vskip-10pt \noindent
  也就是说,通过 $b+\ideal m\mapsto b+\ideal m'$,$B/\ideal m$ 可以视为
  $B[x]/\ideal m'$ 的子域。注意到 $B[x]/\ideal m'$ 的元素都可以写为
  $h(x)+\ideal m'=\bar h(\bar x)=\bar h(x+\ideal m')$ 的形式(其中 $h\in B[t],\bar h\in B/\ideal m[t]$),
  所以 $B[x]/\ideal m'$ 可以视为 $B/\ideal m[\bar x]$,而 $B/\ideal m[\bar x]$ 是域,所以
  $\bar x$ 在 $B/\ideal m$ 上是代数的,那么就存在嵌入 $B[x]/\ideal m'\hookrightarrow\Omega$。
  那么可以令 $g':B[x]\to\Omega$ 为上述交换图中的复合映射 $B[x]\to B[x]/\ideal m'\to\Omega$,
  此时有 $(B,g)\leq (B[x],g')$,所以 $B=B[x]$,这与 $x\notin B$ 矛盾。这就证明了 $\ideal mB[x]=B[x]$。

  Step 3. 证明 $B$ 是 $K$ 的赋值环。若 $x\notin B$,那么 $\ideal mB[x]=B[x]$,所以存在
  $u_i\in\ideal m$ 使得
  \[
    u_0+u_1x+\cdots+u_nx^n=1,  
  \]
  即 $1-u_0-u_1x-\cdots-u_nx^n=0$,因为 $1-u_0\in B-\ideal m=B^\times$,所以
  \[
    x^{-n}-(1-u_0)^{-1}u_1x^{1-n}-\cdots-(1-u_0)^{-1}u_n=0, 
  \]
  所以 $x^{-1}$ 在 $B$ 上是整的,$B[x^{-1}]$ 在 $B$ 上是整的。根据 \autoref{thm:lying over},
  存在 $B[x^{-1}]$ 的素理想 $\ideal m'$ 使得 $\ideal m'\cap B=\ideal m$。实际上 $\ideal m'$
  必为极大理想,否则存在 $B[x^{-1}]$ 的极大理想 $\ideal n'$ 使得 $\ideal m'\subseteq\ideal n'$,
  于是 $\ideal m=\ideal m'\cap B\subseteq\ideal n'\cap B$,那么 $\ideal n'\cap B=\ideal m$,
  根据 \autoref{coro:contraction of prime ideal},有 $\ideal m'=\ideal n'$。再重复上面的叙述,
  我们可以得到 $g':B[x^{-1}]\to\Omega$ 使得 $(B,g)\leq (B[x^{-1}],g')$,所以 $B[x^{-1}]=B$,
  故 $x^{-1}\in B$,所以 $B$ 是 $K$ 的赋值环。此外,$(B,g)\in\Sigma$ 就表明
  $A\subseteq B$ 以及 $\ideal pA_{\ideal p}\subseteq\ideal m=\ker g$。
\end{proof}

\begin{corollary}\label{coro:integral closure is intersection of valuation ring}
  令 $A$ 是域 $K$ 的子环,那么 $A$ 在 $K$ 中的整闭包 $\bar A$ 是 $K$ 的所有包含 $A$
  的赋值环的交。
\end{corollary}
\begin{proof}
  令 $B$ 是 $K$ 的赋值环且 $A\subseteq B$。因为 $B$ 是整闭的,所以 $\bar A\subseteq B$。
  故 $\bar A$ 被 $K$ 的所有包含 $A$ 的赋值环的交包含。

  反之,令 $x\notin\bar A$,那么 $x\notin A[x^{-1}]=A'$,否则存在 $a_0,\dots,a_n\in A$
  使得
  \[
    x=a_0+ a_1x^{-1}+\cdots+a_nx^{-n},
  \]
  即 $x^{n+1}-a_0x^n-a_1x^{n-1}-\cdots-a_n=0$,这与 $x\notin\bar A$ 矛盾。所以 $x^{-1}$
  在 $A'$ 中不可逆,因此存在 $A'$ 的极大理想 $\ideal m'$ 使得 $x^{-1}\in\ideal m'$。
  那么存在 $K$ 的赋值环 $(B,\ideal m)$ 控制 $(A',\ideal m')$,注意 $\ideal m$ 为 $B$
  中不可逆元的集合,所以 $x^{-1}\in A'\subseteq B$ 表明 $x\notin B$。
  
  综上,这就表明 $\bar A$ 是 $K$ 的所有包含 $A$ 的赋值环的交。
\end{proof}

\begin{proposition}
  令 $A\subseteq B$ 是整环,$B$ 是有限生成 $A$-代数。令 $v$ 是 $B$ 中的非零元,那么存在
  $A$ 中的非零元 $u$ 使得:任取代数闭域 $\Omega$,任意同态 $f:A\to\Omega$ 如果满足
  $f(u)\neq 0$,那么 $f$ 可以被延拓为同态 $g:B\to\Omega$ 使得 $g(v)\neq 0$。
\end{proposition}
\begin{remark}
  记 $\ideal p=\ker f$ 和 $\ideal q=\ker g$ 是素理想,那么
  $g|_A=f$ 表明 $\ideal q\cap A=\ideal p$,并且 $f$
  可以通过商环分解为 $f:A\to A/\ideal p\to\Omega$,$g$ 同理,所以此时有交换图
  \[
    \begin{tikzcd}
      A\arrow[r]\arrow[d,hookrightarrow] 
      & A/\ideal p\arrow[r]\arrow[d,hookrightarrow]  & \Omega \\
      B\arrow[r] & B/\ideal q\arrow[ur,dashed,red] &
    \end{tikzcd}  
  \]
\end{remark}
\begin{proof}
  对 $B$ 的生成元的个数归纳,只需要证明 $B=A[x]$ 的情况即可。

  若 $x$ 在 $A$ 上是超越的(即 $x$ 不是任何 $A$ 中系数多项式的根)。对于非零元 $v\in B$,
  存在 $a_0,\dots,a_n\in A$ 使得 $v=a_0x^n+a_1x^{n-1}+\cdots+a_n$,其中 $a_0\neq 0$。
  取 $u=a_0$。如果同态 $f:A\to\Omega$ 满足 $f(u)\neq 0$,考虑 $\Omega[t]$ 中的多项式
  $f(a_0)t^n+f(a_1)t^{n-1}+\cdots+f(a_n)$,$\Omega$ 为代数闭域表明 $\Omega$ 为无限域,
  所以一定存在 $\xi\in\Omega$ 使得 $f(a_0)\xi^n+f(a_1)\xi^{n-1}+\cdots+f(a_n)\neq 0$。
  定义 $g:B\to\Omega$ 满足 $g(x)=\xi$ 以及 $g|_A=f$,那么 $g(v)\neq 0$ 满足条件。

  若 $x$ 在 $A$ 上是代数的(即 $x$ 是某个 $A$ 中系数多项式的根)。
  那么 $v^{-1}\in \Frac(B)$ 也是 $A[x]$ 中多项式的根,所以我们有
  \begin{gather*}
    a_0x^m+a_1x^{m-1}+\cdots+a_m=0\quad (a_i\in A),\\
    a_0'v^{-n}+a_1'v^{1-n}+\cdots+a_n'=0\quad (a_j'\in A).
  \end{gather*}
  令 $u=a_0a_0'$,若 $f:A\to\Omega$ 满足 $f(u)\neq 0$,那么 $f$ 可以延拓为
  $f_1:A[u^{-1}]\to\Omega$,满足 $f_1(u^{-1})=f(u)^{-1}$。根据 \autoref{thm:existence of valuation ring},
  存在 $\Frac(B)$ 的赋值环 $C$ 使得 $C\supseteq A[u^{-1}]$ 以及同态 $h:C\to\Omega$ 满足 $h|_A=f_1$。
  由于 $a_0x^m+a_1x^{m-1}+\cdots+a_m=0$,所以 $ux^m+a_1a_0'x^{m-1}+\cdots+a_ma_0'=0$,所以
  在 $A[u^{-1}]$ 中有
  \[
    x^m+a_1a_0'u^{-1}x^{m-1}+\cdots+a_ma_0'u^{-1}=0,
  \]
  即 $x$ 在 $A[u^{-1}]$ 上是整的。根据 \autoref{coro:integral closure is intersection of valuation ring},
  $x\in C$,所以 $B\subseteq C$。特别地,有 $v\in C$。另一方面,可以证明 $v^{-1}\in\Frac(B)$
  在 $A[u^{-1}]$ 上也是整的,所以 $v^{-1}\in C$。故 $v\in C^\times$,所以 $h(v)\neq 0$。
  令 $g=h|_B$ 即满足要求。
\end{proof}

\begin{corollary}[Zariski 引理]\label{coro:Zariski lemma}
  令 $k$ 是域,$B$ 是有限生成 $k$-代数,如果 $B$ 是域,那么 $B$ 是 $k$
  的有限(代数)扩张。
\end{corollary}
\begin{proof}
  取 $A=k$,$v=1$,$\Omega=\bar k$,那么嵌入 $j:k\to\bar k$ 可以延拓为同态
  $g:B\to\bar k$,$g(1)\neq 0$ 表明 $g$ 是非零同态,$B$ 是域进而表明 $g$
  是单同态,所以 $B$ 是 $k$ 的代数扩张。
\end{proof}

\autoref{coro:Zariski lemma} 是 Hilbert 零点定理(Hilbert's Nullstellensatz)的一种形式。
后面我们会介绍另一种证明方法(\autoref{prop:Zariski lemma 2})。

\section{EXERCISES}

\begin{problem}
  令 $f:A\to B$ 是环的整同态(即 $B$ 在 $ f(A)$ 上是整的),证明 $f^*:\Spec(B)\to\Spec(A)$
  是闭映射。
\end{problem}
\begin{proof}
  
\end{proof}

\begin{problem}
  令 $A$ 是 $B$ 的子环,$C$ 是 $A$ 在 $B$ 中的整闭包,证明 $C[x]$ 是 $A[x]$
  在 $B[x]$ 中的整闭包。
\end{problem}
\begin{proof}
  
\end{proof}

\begin{problem}
  $A,B$ 是两个局部环,如果 $A$ 是 $B$ 的子环并且 $A$ 的极大理想 $\ideal m$
  被 $B$ 的极大理想 $\ideal n$ 包含(等价地说,$\ideal m=\ideal n\cap A$),
  那么称 $B$ \emph{控制} $A$。令 $K$ 是域,$\Sigma$ 是 $K$ 的局部子环的集合,
  控制构成了 $\Sigma$ 上的一个偏序,证明 $\Sigma$ 有极大元,并且 $A\in\Sigma$
  是极大元当且仅当 $A$ 是 $K$ 的赋值环。
\end{problem}

\begin{problem}
  $A$ 是整环,$K$ 是 $A$ 的分式域,证明下面的说法是等价的:
  \begin{enumerate}
    \item $A$ 是 $K$ 的赋值环;
    \item 如果 $\ideal a,\ideal b$ 是 $A$ 的两个理想,那么 $\ideal a\subseteq \ideal b$
    或者 $\ideal b\subseteq\ideal a$。
  \end{enumerate}
  由此推断出,如果 $A$ 是一个赋值环,$\ideal p$ 是 $A$ 的一个素理想,那么 $A_{\ideal p}$
  和 $A/\ideal p$ 都是它们分式域的赋值环。
\end{problem}
\begin{proof}
  若 $A$ 是 $K$ 的赋值环,令 $\ideal a,\ideal b$ 是 $A$ 的两个不相同的理想,
  假设 $\ideal a\not\subseteq \ideal b$,我们证明 $\ideal b\subseteq\ideal a$。
  任取 $b\in\ideal b$,$\ideal a\not\subseteq\ideal b$ 表明存在 $a\in\ideal a-\ideal b$。
  $A$ 是赋值环表明 $a^{-1}b$ 或者 $ab^{-1}$ 在 $A$ 中。
  若 $ab^{-1}\in A$,那么 $a=b(ab^{-1})\in\ideal b$,这与
  $a\notin\ideal b$ 矛盾,所以 $a^{-1}b\in A$,所以
  $b=a(a^{-1}b)\in\ideal a$,故 $\ideal b\subseteq \ideal a$。

  反过来,令 $x\in K^\times$,假设 $x\notin A$。设 $x=a/b$,其中 $a,b\neq 0$,
  那么 $(a)\subseteq (b)$ 或者 $(b)\subseteq (a)$,即 $b\mid a$ 或者 $a\mid b$,
  即存在非零的 $c\in A$ 使得 $a=bc$ 或者 $b=ac$,如果 $a=bc$,那么 $x=c/1\in A$,
  所以有 $b=ac$,所以 $x^{-1}=b/a=c/1\in A$,所以 $A$ 是赋值环。

  $A_{\ideal p}$ 的理想都形如 $\ideal a_{\ideal p}$,其中 $\ideal a$ 是 $A$ 的被 $\ideal p$ 包含的理想。
  所以若 $\ideal a_{\ideal p},\ideal b_{\ideal p}$ 是 $A_{\ideal p}$ 的两个理想,
  那么 $\ideal a$ 和 $\ideal b$ 是 $A$ 的两个理想,所以 $\ideal a\subseteq \ideal b$
  或者 $\ideal b\subseteq \ideal a$,所以 $\ideal a_{\ideal p}\subseteq\ideal b_{\ideal p}$
  或者 $\ideal b_{\ideal p}\subseteq \ideal a_{\ideal p}$,即 $A_{\ideal p}$
  是赋值环。$A/\ideal p$ 的理想都形如 $\ideal a/\ideal p$,其中 $\ideal a$ 是包含 $\ideal p$ 的理想。
  剩余的叙述同理。
\end{proof}

\begin{problem}
  $A$ 是域 $K$ 的赋值环,$A$ 的单位群 $A^\times$ 是 $K$ 的乘法群 $K^\times$ 的子群。
  
  令 $\Gamma=K^\times/A^\times$,如果 $\bar x,\bar y\in\Gamma$,定义 $\bar x\geq \bar y$
  为 $xy^{-1}\in A$。证明这定义了 $\Gamma$ 上的一个全序,并且与群结构相容
  (即 $\bar x\geq \bar y\Rightarrow \bar x\bar w\geq \bar y\bar w\ (\forall\bar w\in\Gamma)$)。
  换句话说,$\Gamma$ 是一个全序的交换群,被称为 $A$ 的\emph{值群}。

  令 $v:K^\times \to\Gamma$ 是自然同态,证明 $v(x+y)\geq\min(v(x),v(y))$。
\end{problem}
\begin{proof}
  若 $\bar x_1=\bar x_2$ 以及 $\bar y_1=\bar y_2$,那么 $x_1^{-1}x_2\in A^\times$ 以及
  $y_1^{-1}y_2\in A^\times$。如果 $\bar x_1\geq \bar y_1$,那么
  $x_1y_1^{-1}\in A$,所以
  \[
    x_2y_2^{-1}=(x_2x_1^{-1})(x_1y_1^{-1})(y_1y_2^{-1})\in A,
  \]
  所以 $\bar x_2\geq \bar y_2$,所以 $\geq$ 是良定义的。
  
  对于任意的 $\bar x\in\Gamma$,由于 $xx^{-1}=1\in A$,所以 $\bar x\geq \bar x$。
  若 $\bar x\geq\bar y$ 以及 $\bar y\geq \bar z$,那么 $xy^{-1}\in A$ 以及
  $yz^{-1}\in A$,所以 $xz^{-1}=(xy^{-1})(yz^{-1})\in A$,所以 $\bar x\geq\bar z$。
  若 $\bar x\geq\bar y$ 以及 $\bar y\geq \bar x$,那么 $xy^{-1}\in A$ 以及
  $yx^{-1}\in A$,所以 $y^{-1}x=xy^{-1}\in A^\times$,所以 $x A^\times=yA^\times$,
  即 $\bar x=\bar y$。这表明 $\geq$ 是 $\Gamma$ 上的一个偏序。
  任取 $\bar x,\bar y\in\Gamma$,$A$ 是赋值环表明 $xy^{-1}\in A$ 或者 $yx^{-1}\in A$,
  即 $x\geq y$ 或者 $y\geq x$,所以 $\geq$ 是 $\Gamma$ 上的一个全序。

  若 $\bar x\geq \bar y$,任取 $\bar w\in\Gamma$,那么 $xw(yw)^{-1}=xy^{-1}\in A$,即
  $\bar x\bar w\geq \bar y\bar w$,所以 $\ge$ 与群结构相容。

  不妨设 $v(x)\geq v(y)$,即 $xy^{-1}\in A$,那么 $(x+y)y^{-1}=xy^{-1}+1\in A$,所以
  $v(x+y)\geq v(y)=\min(v(x),v(y))$。
\end{proof}


