\chapter{模}

\section{自由模与有限生成模}

给定一个环 $A$,以及任意集合 $I$,那么 $I$ 上的\emph{自由 $A$-模}指的是
直和 $\bigoplus_{i\in I}A$,通常记为 $A^{(I)}$。当 $|I|=n$ 是有限集的时候,
显然 $A^{(I)}\simeq A^n$ 是有限生成模。注意到 $I$ 可以被嵌入到
$A^{(I)}$ 中,即映射 $j:I\to A^{(I)}$,满足
\[
  j(x)=(a_i)_{i\in I}\quad
  a_i=
  \begin{cases}
    1 & i=x,\\
    0 & i\neq x.
  \end{cases}
\]


\begin{theorem}[自由模的泛性质]
  $A^{(I)}$ 满足以下泛性质:$B$ 是任意 $A$-模,$\varphi:I\to B$ 是集合间的映射,那么
  存在唯一的 $A$-模同态 $\bar{\varphi}:A^{(I)}\to B$ 满足 $\varphi=\bar\varphi\circ j$,
  即使得下面的图表交换。
  \[
    \begin{tikzcd}
      I \arrow[d,"j"']\arrow[r,"\varphi"] & B \\
      A^{(I)}\arrow[ur,dashed,"\bar{\varphi}"'] & 
    \end{tikzcd}
  \]
\end{theorem}
\begin{proof}
  定义
  \[
    \bar{\varphi}\left(\sum_{k=1}^n x_{i_k}\right)=\sum_{k=1}^nx_{i_k}\varphi(i_k),
  \] 
  其中 $i_k\in I,n\geq 1$。显然 $\bar\varphi$ 是 $A$-模同态,并且有 $\bar\varphi(j(i))=\varphi(i)$。
  由于 $\bar\varphi$ 必须满足 $\bar\varphi(j(i))=\varphi(i)$ 且所有的 $j(i)\ (i\in I)$ 组成了
  $A^{(I)}$ 的生成元,所以这样的 $\bar\varphi$ 是唯一的。
\end{proof}

\begin{example}
    环 $R=\mathbb{Z}[x_1,x_2,\dots]$ 是有限生成 $R$-模,生成元就是 $1$。
    但是理想 $(x_1,x_2,\dots)$ 作为 $R$-模不是有限生成的。
\end{example}

\begin{theorem}
  $M$ 是有限生成 $A$-模当且仅当对于某个整数 $n$,存在满的 $A$-模同态 
  $\varphi:A^n\to M$。
\end{theorem}
\begin{proof}
  若 $M$ 是有限生成的,设 $x_1,\dots,x_n$ 是一组生成元,考虑映射 $\varphi:A^n\to M$ 为
  \[
    (a_1,\dots,a_n)\mapsto a_1x_1+\cdots+a_nx_n,
  \]
  容易验证这是一个模同态,并且由于 $x_1,\dots,x_n$ 是生成元,所以 $\varphi$ 是满同态。

  反过来,记 $e_i=(0,\dots,0,1,0,\dots,0)\in A^n$,其中第 $i$ 个分量为 $1$。那么
  任取 $m\in M$,存在 $(a_1,\dots,a_n)\in A^n$ 使得 $\varphi(a_1,\dots,a_n)=m$,
  即
  \[
    m=\varphi(a_1,\dots,a_n)=\varphi(a_1e_1+\cdots+a_ne_n)
    =a_1\varphi(e_1)+\cdots+a_n\varphi(e_n),  
  \]
  故 $M$ 由 $\varphi(e_1),\dots,\varphi(e_n)$ 生成。
\end{proof}

\begin{definition}
  一个 $A$-模 $M$ 被称为\emph{Noether}模,如果它的每个子模都是有限生成 $A$-模。
  一个环 $A$ 被称为 Noether 环,如果它作为 $A$-模是 Noether 模。
\end{definition}

\begin{proposition}
  $M$ 是一个 $A$-模,$N$ 是 $M$ 的子模,那么 $M$ 是 Noether 模当且仅当 $N$
  和 $M/N$ 都是 Noether 模。
\end{proposition}
\begin{proof}
  若 $M$ 是 Noether 模,那么 $N$ 当然是 Noether 模。根据对应定理,$M/N$ 的子模
  形如 $P/N$,$P$ 是 $M$ 的包含 $N$ 的子模,设 $P$ 由 $x_1,\dots,x_n$ 生成,
  那么 $P/N$ 由 $x_1+N,\dots,x_n+N$ 生成,所以 $M/N$ 是 Noether 模。

  若 $N$ 和 $M/N$ 都是 Noether 模,设 $P$ 是 $M$ 的子模,根据同构定理,我们有
  $(P+N)/N\simeq P/(P\cap N)$,同构关系为 $x+P\cap N\mapsto x+N$,其中 $x\in P$。
  此时 $P\cap N$ 作为 $N$ 的子模是有限生成的,设生成元为 $x_1,\dots,x_n$,
  $(P+N)/N$ 作为 $M/N$ 的子模是有限生成的,设生成元为 $y_1+N,\dots,y_m+N$。
  那么任取 $z\in P$,存在 $a_1,\dots,a_m\in A$,使得
  \[
    z+N=a_1(y_1+N)  +\cdots+a_m(y_m+N),
  \]
  即 $z-a_1y_1-\cdots-a_my_m\in P\cap N$,那么存在 $b_1,\dots,b_n\in A$ 使得
  \[
    z-a_1y_1-\cdots-a_my_m=b_1x_1+\cdots+b_nx_n,  
  \]
  这表明 $P$ 可以由 $x_1,\dots,x_n,y_1,\dots,y_m$ 生成。所以 $M$ 是 Noether 模。
\end{proof}

\begin{corollary}\label{coro:finite generated module over Noether ring}
  $A$ 是 Noether 环,$M$ 是有限生成 $A$-模,则 $M$ 是 Noether 模。
\end{corollary}
\begin{proof}
  根据 \autoref{lemma:finite generated module},$M$ 同构于 $A^n$ 的一个商模,所以
  我们证明 $A^n$ 是 Noether 模即可。对 $n$ 归纳,$n=1$ 时 $A$ 是 Noether 环。
  假设 $A^{n-1}$ 是 Noether 模,考虑最后一个分量的投影 $\pi_n:A^n\to A$,
  那么有 $A^n/A^{n-1}\simeq A$,所以 $A^{n-1}$ 和 $A^n/A^{n-1}$ 都是 Noether 模,
  故 $A^n$ 是 Noether 模。
\end{proof}

下面我们扩展向量空间中“基”的概念。对于 $A$-模 $M$ 的子集 $S$,根据自由模
$A^{(S)}$ 的泛性质,存在唯一的 $A$-模同态 $\varphi:A^{(S)}\to M$,使得下面的图表交换:
\[
  \begin{tikzcd}
    S \arrow[d,"j"']\arrow[r,"\mathbb{1}"] & M \\
    A^{(S)}\arrow[ur,dashed,"\varphi"'] & 
  \end{tikzcd}
\]

\begin{definition}
  如果上述映射 $\varphi$ 是单射,那么我们说 $S$ \emph{线性无关},如果
  $\varphi$ 是满射,那么我们说 $S$ \emph{生成 $M$}。  
\end{definition}

若 $S$ 线性无关,则表明任取有限个元素 $e_1,\dots,e_n\in S$,如果 $a_1,\dots,a_n\in A$
使得
\[
  a_1e_1+\cdots+a_ne_n=0,  
\]
那么
\[
  \varphi(a_1j(e_1)+\cdots+a_nj(e_n))
  =a_1e_1+\cdots+a_ne_n=0,
\]
$\varphi$ 是单射表明 $a_1j(e_1)+\cdots+a_nj(e_n)=0\in \bigoplus_{s\in S}A$,
故 $a_1=\cdots=a_n=0$,反之也是成立的。这和我们熟悉的线性无关的定义是一致的。

\begin{lemma}
  $M$ 是 $A$-模,$S\subseteq M$ 是线性无关子集,那么存在包含 $S$ 的极大线性无关子集。
\end{lemma}


\section{ED 上的有限生成模与相似标准型}

本节的目标是证明 ED(Euclid 整环)上的有限生成模的结构定理:

\begin{theorem}[基本定理:不变因子型]\label{thm:invariant form}
  $A$ 是 ED,$M$ 是有限生成 $A$-模,那么 $M$ 同构于有限多个循环模的直和,即
  \[
    M\simeq A^r\oplus A/(a_1)\oplus A/(a_1)\oplus\cdots  \oplus A/(a_m),
  \]
  其中 $r\geq 0$,$a_1,a_2,\dots,a_m$ 是 $A$ 中非零非单位的元素,并且满足关系
  \[
    a_1\mid a_2\mid\cdots\mid a_m.  
  \]
  此时 $r$ 称为 $M$ 的\emph{自由秩},$a_1,a_2,\dots,a_m$ 称为 $M$ 的\emph{不变因子}。
\end{theorem}

实际上上述定理对于 PID 上的有限生成模也成立,但是 PID 没有一个好的算法来寻找
$a_1,a_2,\dots,a_m$,我们将看到 ED 上的 Euclid 算法将发挥重要作用,并且这也是
线性代数中有理标准型和 Jordan 标准型理论的来源。

设 $M$ 是有限生成 $A$-模,根据 \autoref{lemma:finite generated module},我们知道
存在满同态 $\varphi:A^n\to M$,即 $M\simeq A^n/\ker\varphi$。再根据
\autoref{coro:finite generated module over Noether ring},我们知道 $A^n$
和 $\ker\varphi$ 都是有限生成 $A$-模,设 $x_1,\dots,x_n$ 是 $A^n$ 的基,
$y_1,\dots,y_m$ 是 $\ker\varphi$ 的生成元,注意此时我们考虑生成元的顺序。那么对于每个 $1\leq i\leq m$,
都存在 $a_{i1},\dots,a_{in}\in A$,使得
\[
  y_i=a_{i1}x_1+\cdots+a_{in}x_n  ,
\]
记矩阵(与线性代数中基的过渡矩阵的写法是一致的)
\[
  \mat A=(a_{ji})=
  \begin{pmatrix}
    a_{11} & a_{21} & \cdots & a_{m1} \\
    a_{12} & a_{22} & \cdots & a_{m2} \\
    \vdots & \vdots & \ddots & \vdots \\
    a_{1n} & a_{2n} & \cdots & a_{mn}
  \end{pmatrix}  \in M_{n,m}(A).
\]
那么矩阵 $\mat A$ 表述了 $A^n$ 的基和 $\ker\varphi$ 的生成元的关系,我们称其为
$A^n$ 到 $\ker\varphi$ 的\emph{关系矩阵}。

下面我们观察对关系矩阵 $\mat A$ 做初等变换会有什么样的后果。下面是初等列变换。
\begin{enumerate}
  \item 如果用 $u\in A^\times$ 乘以 $\mat A$ 的第 $i$ 列,那么相当于将 $\ker\varphi$ 的生成元
  $y_i$ 变为 $uy_i$,即给出了 $\ker\varphi$ 的一组新的生成元
  $y_1,\dots,uy_i,\dots,y_m$。
  \item 如果交换 $\mat A$ 的第 $i$ 列和第 $j$ 列,那么相当于交换 $\ker\varphi$ 的生成元
  $y_i$ 和 $y_j$,即给出了 $\ker\varphi$ 的一组新的生成元
  $y_1,\dots,y_j,\dots ,y_i,\dots,y_m$。
  \item 如果将 $\mat A$ 的第 $i$ 列乘以 $a\in A$ 加到第 $j$ 列,那么相当于将 $\ker\varphi$
  的生成元 $y_j$ 替换为 $y_j+ay_i$,即给出了 $\ker\varphi$ 的一组新的生成元
  $y_1,\dots,y_i,\dots ,y_j+ay_i,\dots,y_m$。
\end{enumerate}
下面是初等行变换。
\begin{enumerate}
  \item 如果用 $u\in A^\times$ 乘以 $\mat A$ 的第 $i$ 行,那么相当于将 $A^n$ 的生成元
  $x_i$ 变为 $u^{-1}x_i$,即给出了 $A^n$ 的一组新的生成元
  $x_1,\dots,u^{-1}x_i,\dots,x_n$。
  \item 如果交换 $\mat A$ 的第 $i$ 行和第 $j$ 行,那么相当于交换 $A^n$ 的生成元
  $x_i$ 和 $x_j$,即给出了 $A^n$ 的一组新的生成元
  $x_1,\dots,x_j,\dots ,x_i,\dots,x_n$。
  \item 如果将 $\mat A$ 的第 $i$ 行乘以 $a\in A$ 加到第 $j$ 行,那么相当于将 $A^n$
  的生成元 $x_i$ 替换为 $x_i-ax_j$,即给出了 $A^n$ 的一组新的生成元
  $x_1,\dots,x_i-ax_j,\dots ,x_j,\dots,x_n$。
\end{enumerate}

\begin{theorem}[Smith 标准型]
  如果关系矩阵为零矩阵,那么 $\ker\varphi=0$,从而
  $M\simeq A^n$。所以我们假设 $\ker\varphi\neq 0$,
  令 $a_1$ 是初始关系矩阵 $\mat A$ 中所有元素的最大公因子。
  那么通过上述六种初等变换,可以将 $A$ 变换为
  \[
    \begin{pmatrix}
      D_k & \\
      & O_{n-k,m-k}
    \end{pmatrix}  ,
  \]
  其中 $D_k$ 是对角阵,对角线为 $a_1,a_2,\dots,a_k\ (k\leq m)$,并且满足
  $a_1\mid a_2\mid\cdots\mid a_k$。
\end{theorem}
\begin{proof}
  (1) 首先注意到初等行或者列变换并不会改变所有元素的最大公因子,因为新矩阵的元素显然被
  原矩阵元素的最大公因子整除,而初等变换的操作是可逆的,所以原矩阵的元素又被新矩阵元素的
  最大公因子整除,所以新矩阵和原矩阵元素的最大公因子相同。

  我们先考虑对第一行的操作:假设第一行的最大公因子是 $d=\gcd(a_{11},\dots,a_{m1})$,
  那么我们可以按照下面的方法将 $d$ 放到第一行第一列:
  \begin{itemize}[nosep]
    \item 首先仅涉及前两列,使用辗转相除法,将 $\gcd(a_{11},a_{21})$ 放到第一行第一列;
    \item 然后仅涉及第一列和第三列,使用辗转相除法,将 $\gcd(\gcd(a_{11},a_{21}),a_{31})
    =\gcd(a_{11},a_{21},a_{31})$ 放到第一行第一列;
    \item 重复上述步骤,直到把 $d$ 放到第一行第一列。
  \end{itemize}
  上述操作可以导出下面的算法:
  \begin{enumerate}[nosep]
    \item 使用上述操作将 $d=\gcd(a_{11},\dots,a_{m1})$ 放到第一行第一列;
    \item 此时 $d$ 整除 $a_{21},\dots,a_{m1}$,那么我们可以将第一行除开 $d$ 之外全变为零;
    \item 如果此时矩阵的所有元素都被 $d$ 整除,那么操作停止。否则,存在元素 $a_{ji}\,(i>1)$,
    使得 $d\nmid a_{ji}$;
    \item 如果 $d\nmid a_{1i}$,那么我们跳过这一步。否则,我们将第 $j$ 列
    加到第 $1$ 列使得 $a\nmid a_{1i}$,注意此时并没有改变第一行;
    \item 计算新的最大公因子 $d'=\gcd(d,a_{1i})$,通过仅涉及第 $1$ 行和第 $i$ 行的操作,
    可以将 $d'$ 放到第一行第一列,注意到通过第四步,有 $d\nmid a_{1i}$,所以
    $d'$ 的次数严格小于 $d$ 和 $a_{1i}$ 的次数;
    \item 回到操作 1。
  \end{enumerate}
  这个算法必然在有限步结束,因为每进行一次第 5 步,第一行第一列的元素的次数就会严格变小,
  最多也只能变到零次,从而在第 3 步退出操作。最后我们就得到了 $a_{11}$ 整除所有其他元素的新
  矩阵,而根据前面的观察,这个新矩阵元素的最大公因子等于初始矩阵元素的最大公因子,而新矩阵
  元素的最大公因子显然就是 $a_{11}$,这就完成了第一步。

  (2) 在 (1) 的基础上,使用初等变换将第一行第一列的非 $(1,1)$-元全变成零即可。

  (3) 在 (2) 的基础上,右下角的分块矩阵又回到了 (1) 的形式,注意到对该分块
  矩阵的操作不会影响大矩阵的第一行第一列,所以重复 (1) 的操作,
  又可以将这个分块矩阵化为 (2) 的形式,即使得 $a_1\mid a_2$ 并且 $a_2$ 整除其他所有元素。

  (4) 重复前三步的操作,便可以得到形如
  $
  \begin{psmallmatrix}
    D_k & 0 \\ 0 & O_{n-k,m-k}
  \end{psmallmatrix}
  $
  的关系矩阵。
\end{proof}

\begin{proof}[Proof of \ref{thm:invariant form}]
  根据前面的叙述,$M\simeq A^n/\ker\varphi$,并且通过初等变换将关系矩阵变为
  Smith 标准型后,此时表明存在 $A^n$ 的一组基 $x_1,\dots,x_n$ 和 $\ker\varphi$ 的一组生成元
  $y_1,\dots,y_k$,此时 $k\leq n$ 并且 $y_i=a_ix_i$,故
  \begin{align*}
    M&\simeq A^n/\ker\varphi
    =(A\oplus\cdots\oplus A)/(Ay_1\oplus \cdots\oplus Ay_k\oplus 0\oplus\cdots \oplus0)\\
    &\simeq A/(a_1)\oplus A/(a_2)\oplus\cdots\oplus A/(a_k)\oplus A^{n-k}  .
    \qedhere
  \end{align*}
\end{proof}

\begin{corollary}
  $A$ 是 ED,$M$ 是自由模 $A^n$ 的一个子模,那么 $M$ 是秩 $m\leq n$ 的自由模。
\end{corollary}
\begin{proof}
  存在 $A^n$ 的一组基 $x_1,\dots,x_n$ 和 $M$ 的一组生成元
  $y_1,\dots,y_m$,使得 $m\leq n$ 并且 $y_i=a_ix_i$,此时 $y_i$ 线性无关,故
  $M=Ay_1\oplus\cdots\oplus Ay_m$ 是秩 $m$ 的自由模。
\end{proof}

\begin{corollary}
  $A$ 是 ED,$M$ 是有限生成 $A$-模,那么
  \[
    \Tor(M)=A/(a_1)\oplus A/(a_2)\oplus\cdots\oplus A/(a_m),
  \]
  其中 $a_1\mid a_2\mid\cdots\mid a_m$。特别地,$M$ 是无扭模当且仅当 $M$ 是自由模,
  $M$ 是扭模当且仅当自由秩 $r=0$,此时 $\Ann(M)=(a_m)$。
\end{corollary}

\autoref{thm:invariant form} 的结果还可以进一步改进,设 $a$ 是 $A$ 中的非零非单位的元素,
那么 $a$ 可以唯一分解为 $a=p_1^{\alpha_1}\cdots p_s^{\alpha_s}$,其中 $p_i$ 为互不相同的素元,
那么根据中国剩余定理,有环同构(同时也是 $A$-模同构)
\[
  A/(a)\simeq A/(p_1^{\alpha_1})  \oplus A/(p_2^{\alpha_2})\oplus\cdots\oplus A/(p_s^{\alpha_s}).
\]
\autoref{thm:invariant form} 的右端每一项都可以做类似的分解,于是我们得到了下面定理。

\begin{theorem}[基本定理:初等因子型]
  $A$ 是 ED,$M$ 是有限生成 $A$-模,那么 $M$ 同构于有限多个循环模的直和,即
  \[
    M\simeq A^r\oplus A/(p_1^{\alpha_1})\oplus A/(p_1^{\alpha_2})\oplus\cdots  \oplus A/(p_t^{\alpha_t}),
  \]
  其中 $r\geq 0$,$p_1^{\alpha_1},p_2^{\alpha_2},\dots,p_t^{\alpha_t}$ 是 $A$ 中素元的幂次
  (不需要互不相同)。 $p_1^{\alpha_1},p_2^{\alpha_2},\dots,p_t^{\alpha_t}$
  被称为 $M$ 的\emph{初等因子}。
\end{theorem}

给定 $n$ 维 $F$-向量空间 $V$,设有一组基 $e_1,\dots,e_n$,$T$ 是 $V$ 上的线性变换,那么 $V$ 可以成为一个
$F[x]$-模,通过定义 $x\cdot v=T(v)$,这当然是一个有限生成 $F[x]$-模,$F[x]$
为 ED,这允许我们使用前面的结果。取 $F[x]^n$ 的一组基为 $\xi_1,\dots,\xi_n$,
那么存在满同态 $\varphi:F[x]^n\to V$ 定义为 $\varphi(\xi_i)=e_i$,下面我们说明
$T$ 的特征矩阵和 $\ker\varphi$ 之间的关系。

\begin{proposition}
  如果 $T$ 在上述基 $e_1,\dots,e_n$ 下的表示矩阵为 $\mat A$,特征矩阵
  \[
    xI-\mat A=
    \begin{pmatrix}
      x-a_{11} & -a_{12} & \cdots & -a_{1n}\\
      -a_{21} & x-a_{22} & \cdots & -a_{2n}\\
      \vdots & \vdots & \ddots & \vdots \\
      -a_{n1} & -a_{n2} & \cdots & x-a_{nn}
    \end{pmatrix}  ,
  \]
  那么 $xI-\mat A$ 可以作为 $F[x]^n$ 到 $\ker\varphi$ 的关系矩阵。
\end{proposition}
\begin{proof}
  也就是说我们要证明 $1\leq i\leq n$ 时,
  \[
    \mu_i=-a_{1i}\xi_1-\cdots-a_{i-1,i}\xi_{i-1}+(x-a_{ii})\xi_i-a_{i+1,i}\xi_{i+1}
    -\cdots-a_{ni}\xi_n  
  \]
  组成了 $\ker\varphi$ 的一组生成元。

  直接验证可知
  \[
    \varphi(\mu_i)=-(a_{1i}e_1+\cdots+a_{ni}e_n)+T(e_i)=0,  
  \]
  所以 $\mu_i\in\ker\varphi$。

  由于 
  \[
    x\xi_i=\mu_i+\sum_{j=1}^na_{ji}\xi_j\in 
    (F[x]\mu_1+\cdots F[x]\mu_n)+(F\xi_1+\cdots+F\xi_n),
  \]
  所以
  \[
    x^2\xi_i=x(x\xi_i)=  x\mu_i+\sum_{j=1}^na_{ji}(x\xi_j)
    \in (F[x]\mu_1+\cdots F[x]\mu_n)+(F\xi_1+\cdots+F\xi_n),
  \]
  以此类推,可知任意的 $k\geq 0$ 有
  \[
    x^k\xi_i\in   (F[x]\mu_1+\cdots F[x]\mu_n)+(F\xi_1+\cdots+F\xi_n),
  \]
  所以
  \[
    F[x]^n=F[x]\xi_1+\cdots+F[x]\xi_n=   (F[x]\mu_1+\cdots F[x]\mu_n)+(F\xi_1+\cdots+F\xi_n).
  \]
  那么任取 $\sum_{i=1}^nf_i(x)\mu_i+\sum_{i=1}^n a_i\xi_i\in \ker\varphi$,就有
  \[
    \sum_{i=1}^n a_ie_i=\varphi\left(\sum_{i=1}^nf_i(x)\mu_i+\sum_{i=1}^n a_i\xi_i\right) =0,
  \]
  故 $a_i=0$,这就证明了 $\mu_1,\dots,\mu_n$ 是 $\ker\varphi$ 的生成元。
\end{proof}


\section{有限生成模}

\begin{proposition}[Cayley-Hamilton]\label{prop:cayley}
  $M$ 是有限生成 $A$-模,$\ideal a$ 是 $A$ 的理想,令 $\phi$ 是 $M$ 的自同态并且使得
  $\phi(M)\subseteq \ideal aM$,那么存在 $a_i\in\ideal a$ 使得 $\phi$ 满足下面的等式
  \[
    \phi^n+a_1\phi^{n-1}+\cdots+a_n=0.  
  \]
\end{proposition}
\begin{proof}
  设 $x_1,\dots,x_n$ 为 $M$ 的生成元,那么对于每个 $x_i$,存在 $a_{ij}\in\ideal a$
  使得 $\phi(x_i)=\sum_{j=1}^n a_{ij}x_j$,即
  \[
    \begin{pmatrix}
      \phi-a_{11} & -a_{12} & \cdots & -a_{1n} \\
      -a_{21} & \phi-a_{22} & \cdots & -a_{2n} \\
      \vdots & \vdots & \ddots & \vdots \\
      -a_{n1} & -a_{n2} & \cdots & \phi-a_{nn}
    \end{pmatrix}  
    \begin{pmatrix}
      x_1 \\ x_2\\\vdots \\ x_n 
    \end{pmatrix}=0,
  \]
  左乘左端矩阵的伴随矩阵,那么 $\det(\delta_{ij}\phi-a_{ij})=0$,这就给出了上述等式。
\end{proof}

\begin{corollary}
  令 $M$ 是有限生成 $A$-模,$\ideal a$ 是 $A$ 的理想,使得 $\ideal aM=M$,那么
  存在 $x\equiv 1\pmod{\ideal a}$ 使得 $xM=0$。或者说,存在 $a\in\ideal a$,
  使得 $ay=y\ (\forall y\in M)$。
\end{corollary}
\begin{proof}
  取 $\phi$ 为单位映射 $\mathbb{1}$,那么存在 $a_1,\dots,a_n\in\ideal a$
  使得
  \[
    \mathbb{1}^n+a_1\mathbb{1}^{n-1}+\cdots+a_n\mathbb{1}=0,  
  \]
  那么 $x=1+a_1+\cdots+a_n$ 满足 $xM=0$,$a=-(a_1+\cdots+a_n)$ 满足 $ay=y\ (\forall y\in M)$。
\end{proof}

\begin{proposition}[Nakayama 引理]
  令 $M$ 是有限生成 $A$-模,理想 $\ideal a\subseteq\rad(A)$,如果
  $\ideal aM=M$,那么 $M=0$。
\end{proposition}
\begin{proof}
  存在 $x\equiv 1\pmod{\ideal a}$ 使得 $xM=0$,此时 $x-1\in\rad (A)$,所以
  $x\in A^\times $,故 $M=x^{-1}xM=0$。
\end{proof}

\begin{corollary}\label{coro:Nakayama}
  令 $M$ 是有限生成 $A$-模,$N$ 是 $M$ 的子模,理想 $\ideal a\subseteq\rad(A)$,
  如果 $M=\ideal aM+N$,那么 $M=N$。
\end{corollary}
\begin{proof}
  $M/N$ 是有限生成 $A$-模,由于 $\ideal a(M/N)=(\ideal aM+N)/N=M/N$,所以
  $M/N=0$,即 $M=N$。
\end{proof}

$A$ 是一个局部环,$\ideal m$ 是极大理想,$A/\ideal m$ 是剩余域。$M$ 是有限生成
$A$-模,那么 $M/\ideal mM$ 被 $\ideal m$ 零化,所以 $M/\ideal mM$ 可以视为
$A/\ideal m$-模,即一个有限维 $A/\ideal m$-向量空间。

\begin{proposition}
  令 $x_i\ (1\leq i\leq n)$ 是 $M$ 的元素,并且其在 $M/\ideal mM$ 中的像组成了
  $M/\ideal mM$ 的一组基,那么 $x_i$ 生成了 $M$。
\end{proposition}
\begin{proof}
  设 $N$ 是 $x_i$ 生成的子模,由于 $x_i+M/\ideal m M$ 组成了一组基,
  所以任取 $m\in M$,存在 $a_i\in A$ 使得 $m+\ideal mM=\sum a_ix_i+\ideal mM$,
  即 $m\in N+\ideal mM$,故 $M=\ideal mM+N$,所以 $M=N$。
\end{proof}

\section{正合列}

一列 $A$-模和 $A$-模同态
\[
  \begin{tikzcd}
    \cdots\arrow[r] & M_{i-1}\arrow[r,"f_i"] & M_i\arrow[r,"f_{i+1}"]
    & M_{i+1}\arrow[r] & \cdots 
  \end{tikzcd}  
\]
被称为\emph{上链复形},如果 $f_{i+1}\circ f_i=0$。记 $Z_i=\ker f_{i+1}$,
$B_i=\im f_i$,$H^i=Z_{i}/B_i$ 被称为 $i$-次上同调。若 $H^i=0$,则称
$(M_*,f_*)$ 是正合列。

\begin{proposition}\label{prop:hom functor and exact sequence}
  \mbox{}
  \begin{enumerate}
    \item 序列
    \[
      \begin{tikzcd}[column sep=scriptsize]
        M'\arrow[r,"u"] & M\arrow[r,"v"] & M''\arrow[r] & 0
      \end{tikzcd}  
    \]
    是正合列当且仅当对于任意 $A$-模 $N$,序列
    \[
      \begin{tikzcd}[column sep=scriptsize]
        0\arrow[r] & \Hom(M'',N)\arrow[r,"\bar v"] &
        \Hom(M,N) \arrow[r,"\bar u"] & \Hom(M',N)
      \end{tikzcd}  
    \]
    是正合列,这表明 $\Hom(\uline,N)$ 是左正合(逆变)函子。
    \item 序列
    \[
      \begin{tikzcd}[column sep=scriptsize]
        0\arrow[r] & N'\arrow[r,"u"] & N\arrow[r,"v"] &
        N''
      \end{tikzcd}  
    \]
    是正合列当且仅当对于任意 $A$-模 $M$,序列
    \[
      \begin{tikzcd}[column sep=scriptsize]
        0\arrow[r] & \Hom(M,N')\arrow[r,"\bar u"] &
        \Hom(M,N)\arrow[r,"\bar v"] & \Hom(M,N'')
      \end{tikzcd}  
    \]
    是正合列,这表明 $\Hom(M,\uline)$ 是左正合(协变)函子。
  \end{enumerate}
\end{proposition}
\begin{proof}
  我们只证明第一个。若 
  $
  \begin{tikzcd}[column sep=small,cramped]
    M'\arrow[r,"u"] & M\arrow[r,"v"] & M''\arrow[r] & 0
  \end{tikzcd}  
  $
  是正合列,令 $\bar v(f)=f v=0$,$v$ 是满射,所以存在
  右逆 $v'$ 使得 $vv'=\mathbb{1}_{M''}$,故 $f=0$,所以
  $\bar v$ 是单射。任取 $f\in\Hom(M'',N)$,那么
  $\bar{u}(\bar{v}(f))=\bar u(fv)=fvu=0$,所以 $\im \bar v\subseteq\ker\bar u$。
  任取 $g\in\ker\bar u$,即 $gu=0$。对于 $m''\in M$,有 $m\in M$ 使得 $m''=v(m)$,
  令 $h:M''\to N$ 为 $h(m'')=g(m)$,我们说明 $h$ 是良定义的,
  即若 $m''=v(m_1)=v(m_2)$,则有 $g(m_1)=g(m_2)$。$v(m_1)=v(m_2)$ 表明 $m_1-m_2\in \ker v=\im u$,
  故 $g(m_1-m_2)=0$,所以 $g(m_1)=g(m_2)$。显然 $h$ 是同态,并且 $g=hv=\bar{v}(h)$,这就表明
  $\ker\bar u\subseteq\im\bar v$。所以 $\im\bar v=\ker\bar u$。

  若对于任意 $A$-模 $N$,
  $
  \begin{tikzcd}[column sep=small,cramped]
    0\arrow[r] & \Hom(M'',N)\arrow[r,"\bar v"] &
    \Hom(M,N) \arrow[r,"\bar u"] & \Hom(M',N)
  \end{tikzcd}  
  $
  是正合列。取 $N=\coker v=M''/\im v$,那么自然映射 $\pi:M''\to N$ 满足
  $\bar{v}=\pi v=0$,$\bar v$ 是单射表明 $\pi=0$,即 $M''=\im v$,$v$ 是满射。取
  $N=M''$,那么 $\mathbb{1}_{M''}\in\Hom(M'',M'')$ 满足 $\bar{u}\bar v(\mathbb{1}_{M''})=0$,
  即 $vu=\mathbb{1}_{M''}vu=0$,故 $\im u\subseteq\ker v$。取 $N=\coker u=M/\im u$,
  $\pi:M\to N$ 为自然映射,那么 $\bar u(\pi)=\pi u=0$,故 $\pi\in\ker\bar u=\im \bar v$,
  故存在 $h\in\Hom(M'',N)$ 使得 $\pi=\bar v(h)=hv$,任取 $m\in\ker v$,那么
  $m+\im u=\pi(v)=hv(m)=0$,故 $m\in\im u$,即 $\ker v\subseteq\im u$。所以 $\im u=\ker v$。
\end{proof}

\begin{proposition}[蛇引理]
  设有两个 $A$-模的正合列,并且满足如下交换图(中间两行),那么存在
  下述的红色正合列。
  \[
    \begin{tikzcd}[sep=2.4em]
      & 0\arrow[d] & 0\arrow[d]& 0\arrow[d] &\\
      0\arrow[r,red] & \ker\alpha\arrow[r,red,"\bar u"]\arrow[d] & \ker\beta\arrow[r,red,"\bar v"]\arrow[d] 
      & \ker\gamma \arrow[d]\\
      0\arrow[r] & M'\arrow[r,"u"]\arrow[d,"\alpha"{near start}] & M\arrow[r,"v"]\arrow[d,phantom,""{coordinate,name=O}] 
      \arrow[d,"\beta"{near start}] 
      & M''\arrow[r]\arrow[d,"\gamma"{near start}] & 0\\
      0\arrow[r] & N'\arrow[r,"u'"]\arrow[d] & N\arrow[r,"v'"]\arrow[d] & N''\arrow[r]\arrow[d] & 0\\
      & \coker\alpha\arrow[r,red,"\bar u'"]
      \arrow[uuurr,
        red,
        crossing over clearance=1ex,
        leftarrow,
        rounded corners,
        crossing over,
        to path={ 
          -- ([xshift=-.8em]\tikztostart.west)
          |- (O)[near end]\tikztonodes
          -| node[right,pos=0.9]{$d$} ([xshift=1.2em]\tikztotarget.east)
          -- (\tikztotarget)
        }
      ]\arrow[d] & \coker\beta\arrow[r,red,"\bar v'"]\arrow[d] & \coker\gamma\arrow[r,red]\arrow[d] & 0\\
      & 0 & 0 & 0 &
    \end{tikzcd}  
  \]
\end{proposition}
\begin{proof}
  由于 $\beta\bar u(\ker\alpha)=\beta u(\ker\alpha)=u'\alpha(\ker\alpha)=0$,所以
  $\bar u(\ker\alpha)\subseteq\ker\beta$,而 $\bar u$ 作为 $u$ 的限制自然是单射。
  故红色序列在 $\ker\alpha$ 处是正合的。显然 $\bar v\bar u=0$。任取 $m\in\ker\bar v$,那么
  $m\in\ker v=\im u$,故存在 $m'\in M'$ 使得 $m=u(m')$,此时 $u'\alpha(m')=\beta u(m')=0$,
  $u'$ 是单射表明 $\alpha(m')=0$,即 $m'\in\ker\alpha$,故 $m\in\im\bar u$。这就表明
  $\ker\bar v=\im\bar u$,红色序列在 $\ker\beta$ 处是正合的。任取 $m''\in\ker\gamma$,
  那么存在 $m\in M$ 使得 $m''=v(m)$,此时 $v'\beta(m)=\gamma v(m)=0$,故 $\beta(m)\in\ker v'=\im u'$,
  $u'$ 是单射表明存在唯一的 $n'\in N$ 使得 $\beta(m)=u'(n')$,定义 $d:\ker\gamma\to\coker\alpha$ 为
  $d(m)=n'+\im\alpha$。首先我们说明 $d$ 是良定义的,如果 $m_1,m_2\in M$ 都满足 $m''=v(m_1)=v(m_2)$,
  那么 $m_1-m_2\in\ker v=\im u$,设 $n_1',n_2'\in N'$ 分别满足 $u'(n_1')=\beta(m_1)$ 以及
  $u'(n_2')=\beta(m_2)$,设 $m'\in M'$ 使得 $m_1-m_2=u(m')$,所以 
  $u'\alpha(m')=\beta u(m')=\beta(m_1-m_2)=u'(n_1')-u'(n_2')$,故
  $n_1'-n_2'=\alpha(m')$,即 $n_1'+\im\alpha=n_2'+\im\alpha$,所以 $d$ 是良定义的。
  不难验证 $d$ 是模同态。显然有 $d\bar v=0$。任取 $m''\in\ker d$,存在 $m\in M$
  使得 $m''=v(m)$,设 $n'\in N'$ 使得 $\beta(m)=u'(n')$,那么 $d(m'')=n'+\im\alpha$,
  $m''\in\ker d$ 表明 $n'\in\im\alpha$,故存在 $m'\in M'$ 使得 $n'=\alpha(m')$,
  所以 $\beta u(m')=u'\alpha(m')=\beta(m)$,故 $m-u(m')\in\ker\beta$,并且
  $\bar v(m-u(m'))=v(m-u(m'))=v(m)=m''$,所以 $m''\in\im\bar v$。故 $\ker d=\im\bar v$,
  红色序列在 $\ker\gamma$ 处是正合的。剩下的证明与上面的类似。 
\end{proof}

令 $\mathcal{C}$ 是 $A$-模的一个类,$\lambda$ 是 $\mathcal{C}$ 上的函数,取值为
$\mathbb{Z}$(一般的,也可以是交换群 $G$)。如果对于任意处于 $\mathcal{C}$ 中的短正合列
$
\begin{tikzcd}[cramped,column sep=small]
  0\arrow[r] & M'\arrow[r] & M\arrow[r] & M''\arrow[r] & 0
\end{tikzcd}
$,有 $\lambda(M')-\lambda(M)+\lambda(M'')=0$,那么我们说 $\lambda$
是\emph{加性的}。

\begin{example}
  $A$ 是域 $k$,$\mathcal{C}$ 是有限维 $k$-向量空间 $V$ 的类,那么 $\lambda(V)=\dim V$
  是 $\mathcal{C}$ 上的加性函数。
\end{example}

\begin{proposition}
  令
  $
  \begin{tikzcd}[cramped,column sep=small]
    0\arrow[r] & M_0\arrow[r] & M_1\arrow[r] & \cdots\arrow[r] & M_n\arrow[r] & 0
  \end{tikzcd}
  $
  是 $A$-模序列的上链复形,并且每个 $M_i$、同态核以及上同调 $H^i$ 都在 $\mathcal{C}$ 中,
  那么对于任意 $\mathcal{C}$ 上的加性函数 $\lambda$ 有
  \[
    \sum_{i=0}^n(-1)^i\lambda(M_i)=\sum_{i=0}^n(-1)^i\lambda(H^i).
  \]
\end{proposition}
\begin{proof}
  对于复形
  \[
    \begin{tikzcd}
      \cdots\arrow[r] & M_{i-1}\arrow[r,"f_i"] & M_i\arrow[r,"f_{i+1}"] & M_{i+1}
      \arrow[r] & \cdots,
    \end{tikzcd}  
  \]
  其蕴含了两个短正合列:
  \newlength{\tmpB}
  \newlength{\tmpM}
  \settowidth{\tmpB}{$B_{i+1}$}
  \settowidth{\tmpM}{$M_{i}$}
  \begin{gather*}
    \begin{tikzcd}[ampersand replacement=\&]
      0\arrow[r] \& Z_i\arrow[r] \& M_i\arrow[r,"f_{i+1}"] \& B_{i+1}
      \arrow[r] \& 0,
    \end{tikzcd}  \\
    \begin{tikzcd}[ampersand replacement=\&]
      0\arrow[r] \& B_i\arrow[r] \& \makebox[\tmpM]{$Z_i$}\arrow[r,"\pi"] \& \makebox[\tmpB]{$H^i$}
      \arrow[r] \& 0.
    \end{tikzcd}
  \end{gather*}
  分别表明
  \begin{equation*}
    \lambda(M_i)=\lambda(Z_i)+\lambda(B_{i+1}),\quad
    \lambda(Z_i)=\lambda(B_i)+\lambda(H^i),
  \end{equation*}
  所以
  \[
    \lambda(M_i)=\lambda(B_i)+\lambda(B_{i+1})+\lambda(H^i) , 
  \]
  注意到 $B_0=0$ 以及 $B_{n+1}=0$,故
  \[
    \sum_{i=0}^n(-1)^i\lambda(M_i)=\sum_{i=0}^n(-1)^i\lambda(H^i).\qedhere
  \]
\end{proof}

\section{模的张量积}

给定 $A$-模 $M,N$,$M,N$ 的张量积指的是 $A$-模 $T$,
并且存在 $A$-双线性映射 $g:M\times N\to T$,$T$ 还满足下面的泛性质:
任给 $A$-模 $P$ 和 $A$-双线性映射 $f:M\times N\to P$,存在唯一的
$A$-模同态(线性映射)$\bar f:T\to P$ 使得 $f=\bar fg$,即有交换图:
\[
  \begin{tikzcd}
    M\times N\arrow[d,"g"']\arrow[r,"f"] & P \\
    T\arrow[ur,dashed,"\bar f"'] & 
  \end{tikzcd}  
\]

\begin{proposition}
  上述 $A$-模 $T$ 存在且在同构的意义下唯一。这样的 $T$ 记为
  $M\otimes_A N$。
\end{proposition}
\begin{proof}
  (唯一性)如果 $T,T'$ 均满足上述性质,那么同时有下面两个交换图成立
  \[
    \begin{tikzcd}
      M\times N\arrow[d,"g"']\arrow[r,"g'"] & T' \\
      T\arrow[ur,dashed,"\bar g'"'] & 
    \end{tikzcd}  \quad
    \begin{tikzcd}
      M\times N\arrow[d,"g'"']\arrow[r,"g"] & T \\
      T'\arrow[ur,dashed,"\bar g"'] & 
    \end{tikzcd}
  \]
  即 $g'=\bar g'g=\bar g'\bar gg'$ 以及 $g=\bar gg'=\bar g\bar g'g$。
  此时 $\bar g'\bar g:T'\to T'$ 也是 $A$-模同态,并且
  $g'=\mathbb{1}_{T'}g'=\bar g'\bar gg'$,根据唯一性,所以 $\bar g'\bar g=\mathbb{1}_{T'}$。
  类似地有 $\bar g\bar g'=\mathbb{1}_T$,所以 $\bar g$ 是同构映射。

  (存在性)做 $A$-上的自由模 $A^{(M\times N)}$,这里 $M\times N$ 仅仅代表集合的
  Cartesian 积,我们知道 $A^{(M\times N)}$ 中的元素为有限的形式和
  $\sum_{i=1}^n a_i(x_i,y_i)$,其中 $a_i\in A,x_i\in M,y_i\in N$。
  
  令 $D$ 是 $A^{(M\times N)}$ 的子模,其由以下四类元素生成
  \[
    \begin{cases}
      (x_1+x_2,y)-(x_1,y)-(x_2,y)\\
      (x,y_1+y_2)-(x,y_1)-(x,y_2)\\
      (ax,y)-a(x,y)\\
      (x,ay)-a(x,y)
    \end{cases}  
  \]
  令 $T=A^{(M\times N)}/D$,我们记 $(m,n)+D\in T$ 为 $m\otimes n$。根据构造,我们知道
  \begin{gather*}
    (x_1+x_2)\otimes y=x_1\otimes y+x_2\otimes y,\\
    x\otimes(y_1+y_2)=x\otimes y_1+x\otimes y_2,\\
    (ax)\otimes y=a(x\otimes y)=x\otimes (ay).
  \end{gather*}
  
  下面我们证明 $T$ 就是我们需要的张量积。$g:M\times N\to T$ 定义为
  $g(m,n)=m\otimes n$ 是一个自然的双线性映射。根据自由模的泛性质,
  任意双线性映射 $f:M\times N\to P$ 可以诱导出唯一的同态 $f':A^{(M\times N)}\to P$,
  其满足 $f'|_{M\times N}=f$,此时 $D\subseteq \ker f'$,所以 $f'$ 诱导出
  良定义的同态 $\bar f:T\to P$,满足 $\bar f(m\otimes n)=f'(m,n)=f(m,n)$,即
  $f=\bar fg$。
\end{proof}

对于多个模的张量积有完全类似地定义和证明。

\begin{corollary}\label{coro:tensor product finite generated}
  令 $x_i\in M,y_i\in N$ 使得在 $M\otimes N$ 中有 $\sum x_i\otimes y_i=0$,那么存在
  $M$ 的有限生成子模 $M_0$ 和 $N$ 的有限生成子模 $N_0$ 使得在 $M_0\otimes N_0$
  中有 $\sum x_i\otimes y_i=0$。
\end{corollary}
\begin{proof}
  若 $M\otimes N$ 中有 $\sum x_i\otimes y_i=0$,这表明 $\sum (x_i,y_i)\in D$,
  所以 $\sum (x_i,y_i)$ 可以表示为上述四类元素的有限和,令 $M_0$ 为所有 $x_i$
  和上述四类元素的第一个分量生成的子模,$N_0$ 为所有 $y_i$ 和上述四类元素的第二个
  分量生成的子模即可。
\end{proof}

\begin{proposition}
  $M,N,P$ 是 $A$-模,那么存在以下唯一的同构
  \begin{enumerate}
    \item $M\otimes N\to N\otimes M$
    \item $(M\otimes N)\otimes P\to M\otimes(N\otimes P)\to M\otimes N\otimes P$
    \item $(M\oplus N)\otimes P\to (M\otimes P)\oplus (N\otimes P)$
    \item $A\otimes M\to M$
  \end{enumerate}
  分别使得
  \begin{enumerate}
    \item $x\otimes y\mapsto y\otimes x$
    \item $(x\otimes y)\otimes z\mapsto x\otimes(y\otimes z)\mapsto x\otimes y\otimes z$
    \item $(x,y)\otimes z\mapsto (x\otimes z,y\otimes z)$
    \item $a\otimes x\mapsto ax$.
  \end{enumerate}
\end{proposition}
\begin{proof}
  (1) 映射 $(x,y)\mapsto y\otimes x$ 是双线性映射,所以诱导出唯一的同态 $x\otimes y\mapsto y\otimes x$。
  另一方面,$(y,x)\mapsto x\otimes y$ 也是双线性映射,诱导出唯一的同态 $y\otimes x\mapsto x\otimes y$,
  上述两个同态互为逆映射,所以是同构。

  (2) 给定 $z_0\in P$,映射 $(x,y)\mapsto x\otimes y\otimes z_0$ 是双线性映射,所以诱导出唯一的同态
  $x\otimes y\mapsto x\otimes y\otimes z_0$,这表明 $(x\otimes y,z)\mapsto x\otimes y\otimes z$ 是良定义
  的双线性映射,所以诱导出唯一的同态 $(x\otimes y)\otimes z\mapsto x\otimes y\otimes z$。
  另一方面,映射 $(x,y,z)\mapsto (x\otimes y)\otimes z$ 是三线性映射,所以诱导出唯一的同态
  $x\otimes y\otimes z\mapsto (x\otimes y)\otimes z$,这与上述同态互为逆映射,所以是同构,
  这就证明了 $(M\otimes N)\otimes P\simeq M\otimes N\otimes P$。另一边同理。

  (3) 映射 $((x,y),z)\mapsto (x\otimes z,y\otimes z)$ 是双线性映射,所以诱导出唯一的同态
  $(x,y)\otimes z\mapsto (x\otimes z,y\otimes z)$。另一方面,映射 $(x,z)\mapsto (x,0)\otimes z$
  和 $(y,z)\mapsto (0,y)\otimes z$ 都是双线性映射,分别诱导出同态
  $x\otimes z\mapsto (x,0)\otimes z$ 和 $y\otimes z\mapsto (0,y)\otimes z$,那么我们有
  同态 $(x\otimes z,y\otimes z)\mapsto (x,0)\otimes z+(0,y)\otimes z=(x,y)\otimes z$,
  这与上述同态互为逆映射。

  (4) 映射 $(a,x)\mapsto ax$ 是双线性映射,所以诱导出唯一的同态 $a\otimes x\mapsto ax$。
  另一方面,同态 $x\mapsto 1\otimes x$ 是上述映射的逆映射。
\end{proof}

\begin{corollary}\label{coro:tensor product of vector space}
  若 $V,W$ 分别为 $m,n$ 维 $k$-向量空间,且分别有一组基
  $v_1,\dots,v_m$ 与 $w_1,\dots,w_n$,那么 $V\otimes W$ 是 $mn$ 维 $k$-向量空间,
  基为 $v_i\otimes w_j\ (1\leq i\leq m,1\leq j\leq n)$。
\end{corollary}
\begin{proof}
  我们有
  \[
    V\otimes W\simeq\left(\bigoplus_{i=1}^m k\right)  \otimes W\simeq
    \bigoplus_{i=1}^m (k\otimes W)\simeq \bigoplus_{i=1}^m W,
  \]
  同构映射为 $v_i\otimes w_j\mapsto (0,\dots,1,\dots,0)\otimes w_j\mapsto (0,\dots,1\otimes w_j,\dots,0)\mapsto 
  (0,\dots,w_j,\dots,0)$。
\end{proof}

令 $A,B$ 是环,$M$ 是 $A$-模,$P$ 是 $B$-模,$N$ 是 $(A,B)$-双模(即 $N$ 既是 $A$-模又是 $B$-模,
并且有相容的模结构:$a(xb)=(ax)b$ 对于 $a\in A,b\in B,x\in N$)。此时,通过定义 
$b(m\otimes n)\coloneqq m\otimes (nb)$,使得 $M\otimes_AN$ 自然地成为一个 $B$-模,
此时 $M\otimes_AN$ 是一个 $(A,B)$-双模,因为
\begin{align*}
  (a(m\otimes n))b&= (m\otimes (an))b=m\otimes((an)b)=
  m\otimes(a(nb))\\&=a(m\otimes(nb))=a((m\otimes n)b).
\end{align*}
类似地,$N\otimes_BP$ 也自然地成为一个 $(A,B)$-双模,类似上面的证明,我们同时有
$A$-模或者 $B$-模同构
\[
  (M\otimes_AN)\otimes_BP \simeq M\otimes_A(N\otimes_B P). 
\]

$f:M\to M'$ 和 $g:N\to N'$ 是 $A$-模同态的张量积。定义 $h:M\times N\to M'\otimes N'$ 为
$h(x,y)=f(x)\otimes g(y)$,容易验证 $h$ 是双线性映射,所以诱导出同态
\[
  f\otimes g:M\otimes N\to M'\otimes N'  
\]
为
\[
  (f\otimes g)(x\otimes y)=f(x)\otimes g(y).  
\]

令 $f':M'\to M''$ 和 $g':N'\to N''$ 为 $A$-模同态,注意到
\begin{align*}
  \bigl((f'\circ f)\otimes (g'\circ g)\bigr)(x\otimes y)
  &=(f'\circ f)(x\otimes y)\otimes  (g'\circ g)(x\otimes y)\\
  &=(f'\otimes g')\bigl(f(x\otimes y)\otimes g(x\otimes y)\bigr) \\
  &=(f'\otimes g')\circ (f\otimes g)(x\otimes y),
\end{align*}
所以
\[
  (f'\circ f)\otimes (g'\circ g)=(f'\otimes g')\circ (f\otimes g).  
\]

\section{标量的扩张与限制}

$f:A\to B$ 是环同态,$N$ 是 $B$-模,那么 $N$ 自然有一个 $A$-模结构,为
$ax\coloneqq f(a)x$。这个 $A$-模称为 $N$ 通过标量的限制得到的。特别地,
$f$ 定义了 $B$ 上的 $A$-模结构。

\begin{proposition}
  设 $N$ 作为 $B$-模是有限生成的,$B$ 作为 $A$-模是有限生成的,那么 $N$
  作为 $A$-模是有限生成的。
\end{proposition}
\begin{proof}
  设 $x_1,\dots,x_n$ 是 $N$ 作为 $B$-模的生成元,$b_1,\dots,b_m$ 是 $B$
  作为 $A$-模的生成元,那么 $b_ix_j$ 就是 $N$ 作为 $A$-模的生成元。
\end{proof}

现在令 $M$ 是 $A$-模,如上所述,$B$ 可以视为 $A$-模,那么我们可以构造
$A$-模 $M_B=B\otimes_A M$,实际上 $M_B$ 拥有 $B$-模结构,通过
$b(b'\otimes x)\coloneqq bb'\otimes x$,这个 $B$-模 $M_B$
称为 $M$ 通过标量的扩张得到的。

\begin{proposition}
  若 $M$ 作为 $A$-模是有限生成的,那么 $M_B$ 是有限生成 $B$-模。
\end{proposition}
\begin{proof}
  如果 $x_1,\dots,x_n$ 是 $M$ 的生成元,那么
  $1\otimes x_1,\dots,1\otimes x_n$ 是 $M_B$ 的 $B$-模生成元。
\end{proof}

在线性代数中,经常将实向量空间 $\mathbb{R}^n$ 视为复向量空间 $\mathbb{C}^n$ 的子空间。
这实际上是在说将 $\mathbb{R}^n$ 的标量扩张为 $\mathbb{C}$,可以预见应该有
$\mathbb{C}$-模同构 $\mathbb{C}\otimes_{\mathbb{R}}\mathbb{R}^n\simeq\mathbb{C}^n$。
现在我们来证明这一点,对于更一般的情况,我们证明 $B$-模同构 $B\otimes_A A^n\simeq B^n$。
考虑双线性映射 $\varphi:B\times A^n\to B^n$ 为
\[
  \varphi(x,(y_1,\dots,y_n))  =(xf(y_1),\dots,xf(y_n)),
\]
这诱导出 $A$-模同态 $\bar\varphi:B\otimes_A A^n\to B^n$。定义同态 $\psi:B^n\to B\otimes_A A^n$
为
\[
  \psi(x_1,\dots,x_n)=x_1\otimes e_1+\cdots+x_n\otimes e_n,  
\]
其中 $e_i=(0,\dots,1,\dots,0)\in A^n$ 是生成元,容易验证 $\bar\varphi$ 和 $\psi$ 互为逆映射。
所以 $\bar\varphi$ 是 $A$-模同构。容易验证 $\bar\varphi$ 也是 $B$-模同构。

还是按照上面的记号,标量的限制实际上可以视为 $\text{$B$-\textsf{Mod}}\to\text{$A$-\textsf{Mod}}$ 的协变函子
$\mathcal R$,将 $B$-模 $M$ 送到限制标量后的 $A$-模 $\mathcal R(M)=M$,将 $B$-模同态 $f:M\to M'$ 送到 $A$-模同态
$\mathcal R(f):\mathcal R(M)\to \mathcal R(M')$,作用为 $\mathcal R(f)(x)=f(x)$,即 $\mathcal R(f)=f$。
同样的,标量的扩张可以视为 $\text{$A$-\textsf{Mod}}\to\text{$B$-\textsf{Mod}}$ 的协变函子
$\mathcal E$,将 $A$-模 $M$ 送到 $\mathcal E(M)=B\otimes_AM$,
将 $A$-模同态 $g:M\to M'$ 送到 $B$-模同态 $\mathcal E(g):\mathcal E(M)\to \mathcal E(M')$,作用为
$\mathcal E(g)(b\otimes x)=b\otimes g(x)$,即 $\mathcal E(g)=\mathbb{1}\otimes g$。
我们证明对于任意 $A$-模 $M$ 和 $B$-模 $N$,都有
\[
  \Hom_{\text{$A$-\textsf{Mod}}}  (M,\mathcal R(N))\simeq \Hom_{\text{$B$-\textsf{Mod}}}
  (\mathcal E(M),N).
\]
任取 $A$-模同态 $g:M\to \mathcal R(N)$,诱导出一个 $B$-模同态 $\bar g:\mathcal E(M)\to N$,通过
\[
  \bar g(b\otimes x)=  bg(x).
\]
反之,任取 $B$-模同态 $\bar f:\mathcal E(M)\to N$,诱导出一个 $A$-模同态 $ f:M\to \mathcal R(N)$,通过
\[
   f(x)=\bar f(1\otimes x).  
\]
容易验证 $g\mapsto \bar g$ 和 $\bar f\mapsto f$ 互为逆映射,所以这给出了上述一一对应。
这就表明对于任意 $A$-模 $M$ 和 $B$-模 $N$,有同构
\[
  \nu_{M,N}:   \Hom_{\text{$A$-\textsf{Mod}}}  (M,\mathcal R(N))\to \Hom_{\text{$B$-\textsf{Mod}}}
  (\mathcal E(M),N).
\]
任取 $A$-模 $M_1,M_2$ 和 $B$-模 $N_1,N_2$,$\varphi:M_2\to M_1$ 和 $\psi:N_1\to N_2$ 分别是
$A$-模同态与 $B$-模同态,我们有交换图
\[
  \begin{tikzcd}[row sep=2.4em,column sep=7.2em]
    \Hom_{\text{$A$-\textsf{Mod}}}(M_1,\mathcal R(N_1))
    \arrow[r,"{\Hom_{\text{$A$-\textsf{Mod}}}(\varphi,\mathcal R(\psi))}"]
    \arrow[d,"{\nu_{M_1,N_1}}"']
    &
    \Hom_{\text{$A$-\textsf{Mod}}}(M_2,\mathcal R(N_2))
    \arrow[d,"{\nu_{M_2,N_2}}"] \\
    \Hom_{\text{$B$-\textsf{Mod}}}(\mathcal E(M_1),N_1)
    \arrow[r,"{\Hom_{\text{$B$-\textsf{Mod}}}(\mathcal E(\varphi),\psi)}"]
    &
    \Hom_{\text{$B$-\textsf{Mod}}}(\mathcal E(M_2),N_2)
  \end{tikzcd}  
\]
上面的叙述说明了存在自然同构 
$\nu:\Hom_{\text{$A$-\textsf{Mod}}}(\uline,\mathcal R(\uline))\Rightarrow 
\Hom_{\text{$B$-\textsf{Mod}}}(\mathcal E(\uline),\uline)$,所以 $\mathcal R$ 和 $\mathcal E$
是一对伴随函子,$\mathcal E$ 是 $\mathcal R$ 的左伴随。

\section{张量积的正合性质}

给定环 $A$ 和 $A$-模 $N$,
考虑 $\text{$A$-\textsf{Mod}}\to \text{$A$-\textsf{Mod}}$ 的张量积函子 $\uline\otimes N$
和 Hom 函子 $\Hom_{\text{$A$-\textsf{Mod}}}(N,\uline)$,为了简洁,后面我们略去
下标 $\text{$A$-\textsf{Mod}}$。张量积函子 $\uline\otimes N$ 将一个 $A$-模 $M$
送到 $A$-模 $M\otimes_A N$,将 $A$-模同态 $f:M\to M'$ 送到 $f\otimes \mathbb{1}:M\otimes N\to M\otimes N$。
Hom 函子 $\Hom(N,\uline)$ 将 一个 $A$-模 $P$ 送到 $A$-模 $\Hom(N,P)$,将 $A$-模同态 $f:P\to P'$ 送到
$\bar f:\Hom(N,P)\to\Hom(N,P')$,作用为 $\bar f(\varphi)=f\circ \varphi$。我们证明
这两个函子是一对伴随函子。

任取 $A$-模 $M$ 和 $P$,我们说明存在 $A$-模同构
\begin{equation}\label{eq:hom isomorphism}
  \nu_{M,P}: \Hom(M,\Hom(N,P))\to \Hom(M\otimes N,P).
\end{equation}
任取同态 $g:M\to \Hom(N,P)$,那么对于每个 $x\in M$,诱导了一个双线性映射 $(x,y)\mapsto g(x)(y)$,
进而诱导出同态 $\bar g:M\otimes N\to P$,故定义 $\nu_{M,P}(g)=\bar g$,满足
$\bar g(x\otimes y)=g(x)(y)$。反之,任取同态 $\bar f:M\otimes N\to P$,
其对应一个双线性映射 $f:M\times N\to P$,对于每个 $x\in M$,这个双线性映射可以
诱导出同态 $f_x:N\to P$,不难验证 $\bar f\mapsto (x\mapsto f_x)$ 是上述映射的逆映射。
此外,容易验证 $\nu_{M,P}$ 是 $A$-模同态。

任取 $A$-模 $M_1,M_2$ 和 $P_1,P_2$,$\varphi:M_2\to M_1$ 和 $\psi:P_1\to P_2$,
我们有交换图
\[
  \begin{tikzcd}[row sep=2.4em,column sep=7.2em]
    \Hom(M_1,\Hom(N,P_1))
    \arrow[r,"{\Hom(\varphi,\Hom(N,\psi))}"]
    \arrow[d,"\nu_{M_1,P_1}"']
    & \Hom(M_2,\Hom(N,P_2))
    \arrow[d,"\nu_{M_2,P_2}"]
     \\
    \Hom(M_1\otimes N,P_1)
    \arrow[r,"{\Hom(\varphi\otimes N,\psi)}"] & \Hom(M_2\otimes N,P_2)
  \end{tikzcd}  
\]
这就表明 $\Hom(N,\uline)$ 和  $\uline\otimes N$ 是一对伴随函子,
$\uline\otimes N$ 是 $\Hom(N,\uline)$ 的左伴随。


\begin{proposition}\label{prop:tensor functor and exact sequence}
  令
  \[
    \begin{tikzcd}
      M'\arrow[r,"f"] & M\arrow[r,"g"] & M''\arrow[r] & 0
    \end{tikzcd}  
  \]
  是正合列,$N$ 是任意 $A$-模,那么序列
  \[
    \begin{tikzcd}
      M'\otimes N\arrow[r,"f\otimes 1"] & M\otimes N\arrow[r,"g\otimes 1"] & M''\otimes N\arrow[r] & 0
    \end{tikzcd}  
  \]
  是正合列。
\end{proposition}
\begin{proof}
  任取 $A$-模 $P$,根据 \autoref{prop:hom functor and exact sequence} 的 (1),
  记 $D=\Hom(N,P)$,我们知道
  \[
    \begin{tikzcd}
      0\arrow[r] &
      \Hom(M'',D)\arrow[r,"\bar g"] & 
      \Hom(M,D)\arrow[r,"\bar f"] &
      \Hom(M',D)
    \end{tikzcd}  
  \]
  是正合列,其中 $\bar g:\phi\mapsto \phi\circ g$,$\bar f:\psi\mapsto \psi\circ f$。
  根据 \eqref{eq:hom isomorphism} 式,有
  \[
    \begin{tikzcd}
      0\arrow[r] &
      \Hom(M''\otimes N,P)\arrow[r,"\tilde g"] & 
      \Hom(M\otimes N,P)\arrow[r,"\tilde f"] &
      \Hom(M'\otimes N,P)
    \end{tikzcd}  
  \]
  是正合列,其中 $\tilde{g}=\nu_{M,P}\circ \bar g\circ \nu_{M'',P}^{-1}$,
  $\tilde{f}=\nu_{M',P}\circ \bar f\circ\nu_{M,P}^{-1}$。
  再根据 \autoref{prop:hom functor and exact sequence} 的 (1),
  并且注意到任取 $\bar\phi\in\Hom(M''\otimes N,P)$,对于任意 $x\otimes y\in M\otimes N$,有
  \begin{align*}
    \tilde{g}(\bar\phi)(x\otimes y)&=
    \bigl(\nu_{M,P}\circ \bar g\circ \nu_{M'',P}^{-1}(\bar\phi)\bigr)(x\otimes y)
    =\Bigl(\nu_{M,P}\bigl(\nu_{M'',P}^{-1}(\bar\phi)\circ g\bigr)\Bigr)(x\otimes y)\\
    &=\bigl(\nu_{M'',P}^{-1}(\bar\phi)\circ g\bigr)(x)(y)
    =\phi_{g(x)}(y)\\
    &=\bar\phi(g(x)\otimes y)=\bar\phi\circ (g\otimes \mathbb{1})(x\otimes y),
  \end{align*}
  所以
  \[
    \begin{tikzcd}
      M'\otimes N\arrow[r,"f\otimes 1"] & M\otimes N\arrow[r,"g\otimes 1"] & M''\otimes N\arrow[r] & 0
    \end{tikzcd}  
  \]
  是正合列。
\end{proof}

上述证明实际上表明任意左伴随的函子都是右正合的。

一般来说,如果 
$
  \begin{tikzcd}[cramped,column sep=small]
    M'\arrow[r] & M\arrow[r] & M''
  \end{tikzcd}
$
是正合列,那么对于任意 $A$-模 $N$,通过张量积函子 $\uline\otimes N$ 作用后的序列
$
  \begin{tikzcd}[cramped,column sep=small]
    M'\otimes N\arrow[r] & M\otimes N\arrow[r] & M''\otimes N
  \end{tikzcd}
$
不一定是正合列。对于一个 $A$-模 $N$,如果张量积函子 $\uline\otimes N$ 
是正合函子,即能把任意正合列变为一个正合列,那么我们称 $N$ 是\emph{平坦模}。

\begin{proposition}
  对于一个 $A$-模 $N$,下面的说法是等价的:
  \begin{enumerate}
    \item $N$ 是平坦模。
    \item 如果 
    $
      \begin{tikzcd}[cramped,column sep=small]
        0\arrow[r] & M'\arrow[r] & M\arrow[r] & M''\arrow[r] & 0
      \end{tikzcd}
    $
    是任意 $A$-模的短正合列,那么张量积序列
    \[
      \begin{tikzcd}
        0\arrow[r] & M'\otimes N\arrow[r] & M\otimes N\arrow[r] & M''\otimes N\arrow[r] & 0
      \end{tikzcd}
    \]
    是短正合列。
    \item 如果 $f:M'\to M$ 是单同态,那么 $f\otimes 1:M'\otimes N\to M\otimes N$
    是单同态。
    \item 如果 $f:M'\to M$ 是单同态并且 $M,M'$ 是有限生成模,那么
    $f\otimes 1:M'\otimes N\to M\otimes N$ 是单同态。
  \end{enumerate}
\end{proposition}
\begin{proof}
  $(1)\Rightarrow (2)$ 显然。

  $(2)\Rightarrow (1)$ 若
  $
    \begin{tikzcd}[cramped,column sep=small]
      M'\arrow[r,"f"] & M\arrow[r,"g"] & M''
    \end{tikzcd}
  $
  是正合列,那么其蕴含两个短正合列
  \begin{gather*}
    \begin{tikzcd}[ampersand replacement=\&]
      0\arrow[r] \& \ker f\arrow[r] \& M'\arrow[r,"\bar f"] \& \im f\arrow[r] \& 0,
    \end{tikzcd}\\
    \begin{tikzcd}[ampersand replacement=\&]
      0\arrow[r] \& \ker g\arrow[r] \& \makesamewidth[c]{$M'$}{$M$}\arrow[r,"\bar g"] \& \im g\arrow[r] \& 0,
    \end{tikzcd}
  \end{gather*}
  其中 $\bar f$ 和 $\bar g$ 表示值域的限制。根据条件,有短正合列
  \begin{gather*}
    \begin{tikzcd}[ampersand replacement=\&]
      0\arrow[r] \& \ker f\otimes N\arrow[r] \& M'\otimes N\arrow[r,"\bar f\otimes\mathbb{1}"] \& \im f\otimes N\arrow[r] \& 0,
    \end{tikzcd}\\
    \begin{tikzcd}[ampersand replacement=\&]
      0\arrow[r] \& \ker g\otimes N\arrow[r] \& \makesamewidth[c]{$M'\otimes N$}{$M\otimes N$}\arrow[r,"\bar g\otimes\mathbb{1}"] \& \im g\otimes N\arrow[r] \& 0,
    \end{tikzcd}
  \end{gather*}
  所以 $\im (f\otimes\mathbb{1})=\im(\bar f\otimes \mathbb{1})=\im f\otimes N=\ker g\otimes N$,
  接下来证明 $\ker g\otimes N=\ker (g\otimes\mathbb{1})$ 即可。
  注意到我们还有短正合列
  \[
    \begin{tikzcd}
      0\arrow[r] & \im g\arrow[r,"i"] & M''\arrow[r] & \coker g\arrow[r] & 0,
    \end{tikzcd}  
  \]
  所以有短正合列
  \[
    \begin{tikzcd}
      0\arrow[r] & \im g\otimes N\arrow[r,"i\otimes\mathbb{1}"] & M''\otimes N\arrow[r] & \coker g\otimes N\arrow[r] & 0,
    \end{tikzcd}  
  \]
  注意到 $\bar g\otimes \mathbb{1}$ 是满射,$i\otimes\mathbb{1}$ 是单射,以及
  \[
    (i\otimes\mathbb{1})\circ (\bar g\otimes\mathbb{1})=
    (i\circ \bar g)\otimes (\mathbb{1}\circ\mathbb{1})=  g\otimes\mathbb{1},
  \]
  所以
  \[
    \ker(g\otimes\mathbb{1})=\ker \bigl((i\otimes\mathbb{1})\circ (\bar g\otimes\mathbb{1})\bigr)
    =\ker g\otimes N.
  \]

  $(2)\Leftrightarrow (3)$ 由 \autoref{prop:tensor functor and exact sequence} 立即得到。

  $(3)\Rightarrow (4)$ 显然。

  $(4)\Rightarrow (3)$ 令 $f:M'\to M$ 是单射,$\sum x_i'\otimes y_i\in \ker(f\otimes \mathbb{1})$,
  也就是说在 $M\otimes N$ 中有 $\sum f(x_i')\otimes y_i=0$。令 $M_0'$ 为 $x_i'$ 生成的子模,根据
  \autoref{coro:tensor product finite generated},存在 $M$ 的有限生成子模 $M_0$ 使得
  在 $M_0\otimes N$ 中有 $\sum f(x_i')\otimes y_i=0$,那么 $f_0:M_0'\to M_0$ 满足
  $\sum x_i'\otimes y_i\in\ker(f_0\otimes\mathbb{1})$,$f_0$ 作为 $f$ 的限制是单射,所以
  $\sum x_i'\otimes y_i=0$,故 $f\otimes\mathbb{1}$ 是单射。
\end{proof}

如果 $f:A\to B$ 是环同态,$M$ 是平坦 $A$-模,我们证明 $M_B=B\otimes_AM$ 是平坦 $B$-模。
若 $\varphi:N\to N'$ 是 $B$-模同态,我们说明
\[
  \varphi\otimes_B\mathbb{1}_{M_B}:N\otimes_B(B\otimes_AM)\to N'\otimes_B(B\otimes_AM)  
\]
是单同态。我们知道有 $B$-模同构
\begin{gather*}
  \psi_1:N\otimes_B(B\otimes_AM)\simeq (N\otimes_BB)\otimes_AM,  \\
  \psi_2:N'\otimes_B(B\otimes_AM)\simeq (N'\otimes_BB)\otimes_AM,  
\end{gather*}
所以我们有 $(N\otimes_BB)\otimes_AM\to (N'\otimes_BB)\otimes_AM$ 的 $B$-模同态
$\psi_2\circ(\varphi\otimes_B\mathbb{1}_{M_B})\circ\psi_1$,自然也是 $A$-模同态,
$M$ 是平坦 $A$-模表明上述映射是单射,所以 $\varphi\otimes_B\mathbb{1}_{M_B}$ 是单射,
从而 $M_B$ 是平坦 $B$-模。

设 $A$ 是环,其自身作为一个 $A$-模,给定 $a\in A$ 且 $a$ 不是零因子,考虑 $A$-模同态 $a:A\to A$
为 $a(x)=ax$,那么这是一个单同态。此时,如果 $N$ 是平坦 $A$-模,那么
$a\otimes\mathbb{1}_N:A\otimes N\to A\otimes N$ 是单同态,又因为
有 $A$-模同构 $A\otimes N\simeq N$,所以 $x\mapsto ax$ 是 $N\to N$ 的单同态。这就表明,
如果 $N$ 是平坦 $A$-模,那么对于任意非零因子的 $a\in A$,集合
\[
  \{x\in N\,|\, ax=0\}=\{0\}
\]
是单点集。特别地,如果 $A$ 是整环,那么 $N$ 是平坦 $A$-模表明 $N$ 是无扭模。
如果 $A$ 是 PID,实际上我们有更强的结论:$N$ 是平坦 $A$-模当且仅当 $N$ 是无扭模。

\section{代数}

令 $f:A\to B$ 是环同态,如果 $a\in A$ 和 $b\in B$,定义模的乘积
\[
  a\cdot b\coloneqq f(a)b,  
\]
这使得 $B$ 成为一个 $A$-模,即之前提到的标量的限制。所以 $B$ 同时拥有环结构和 $A$-模结构,
并且这两个结构是相容的,即
\[
  a\cdot(b_1b_2)=(a\cdot b_1)b_2=b_1(a\cdot b_2).
\]
这样的配备了 $A$-模结构的环 $B$,我们称之为\emph{$A$-代数}。更明确地说,$A$-代数指的是一个环
$B$ 以及一个环同态 $f:A\to B$。当然,上面的两种说法是等价的,即如果环 $B$ 有一个与环结构相容的 $A$-模
结构,那么 $B$ 是一个 $A$-代数。因为此时我们可以构造环同态 $f:A\to B$ 为 $f(a)=a\cdot 1_B$,
\[
  f(aa')=(aa')\cdot 1_B=a\cdot(a'\cdot 1_B)=a\cdot (1_B(a'\cdot 1_B))
  =(a\cdot 1_B)(a'\cdot 1_B)=f(a)f(a')
\]
表明这确实是环同态。

关于代数的定义有两个特殊的例子。如果 $A$ 是一个域 $K$,并且 $B$ 非零,那么同态 $f:K\to B$
一定是单同态,因此 $K$ 可以视为 $B$ 的子域,所以一个 $K$-代数就是包含 $K$ 作为子域的环。
如果 $A$ 是任意环,此时总是有 $\mathbb{Z}\to A$ 的同态,为 $n\mapsto n1_A$,所以每个环自动地
成为一个 $\mathbb{Z}$-代数。

对于两个 $A$-代数 $B,C$,一个 \emph{$A$-代数同态} $h:B\to C$ 指的是一个环同态并且同时是一个 $A$-模同态。
令 $f:A\to B$,$g:B\to C$ 分别是 $B,C$ 对应的环同态,我们证明环同态 $h$ 是 $A$-代数同态当且仅当 $h\circ f=g$。
若 $h$ 是 $A$-代数同态,那么
\[
  h(f(a))= h(a\cdot 1_B)=a\cdot h(1_B)=a\cdot 1_C=g(a).
\]
反之,如果环同态 $h$ 满足 $h\circ f=g$,那么
\[
  h(a\cdot b)=h(f(a)b)=h(f(a))h(b)=g(a)h(b)=a\cdot h(b), 
\]
从而 $h$ 是 $A$-代数同态。

\section{代数的张量积}

令 $B,C$ 是两个 $A$-代数,$f:A\to B$ 和 $g:A\to C$ 是对应的环同态。因为 $B$ 和 $C$ 是 $A$-模,
所以我们可以作张量积得到 $A$-模 $D=B\otimes_A C$,实际上我们可以赋予 $D$ 环结构。

考虑映射 $B\times C\times B\times C\to D$ 为
\[
  (b,c,b',c')\mapsto bb'\otimes cc'.  
\]
容易验证这是 $4$-多重线性映射,所以诱导了 $A$-模同态 
\[
  B\otimes C\otimes B\otimes C\to D,  
\]
所以我们有对应的 $A$-模同态 $D\otimes D\to D$,这个同态又对应到双线性映射
$
  \mu:D\times D\to D   
$
为
\[
  \mu(b\otimes c,b'\otimes c')=bb'\otimes cc'.  
\]
这给出了 $D$ 上的乘法,对于单张量,我们定义
\[
  (b\otimes c)(b'\otimes c')=bb'\otimes cc',  
\]
对于一般的张量,我们定义
\[
  \left(\sum_i (b_i\otimes c_i)\right) 
  \left(\sum_j(b_j\otimes c_j)\right)
  =\sum_{i,j}(b_ib_j\otimes c_ic_j).
\]
可以验证这个乘法使得 $D$ 成为一个交换环,单位元为 $1_B\otimes 1_C$。此外,
$D$ 成为了一个 $A$-代数,对应的环同态为 
\[
  a\mapsto a(1_B\otimes 1_C)=f(a)\otimes 1_C=1_B\otimes g(a).
\]


\section{EXERCISES}

\begin{problem}
  $k$ 是域,$A=k[[x_1,\dots,x_n]]$ 是 $n$ 个未定元的形式幂级数环,记
  $\ideal m$ 是极大理想 $\ideal m=(x_1,\dots,x_n)$。令
  $f_1,\dots,f_n\in \ideal m$,证明 $\ideal m=(f_1,\dots,f_n)$
  当且仅当 Jacobi 行列式
  \[
    \frac{\partial(f_1,\dots,f_n)}{\partial(x_1,\dots,x_n)}(0)\neq 0.  
  \]
\end{problem}
\begin{proof}
  若
\end{proof}

\begin{problem}
  $f:A\to B$ 是环同态,$N_1,N_2$ 是 $B$-模,通过标量的限制,我们也可以将它们
  视为 $A$-模,证明存在 $B$-模同构:
  \[
    N_1\otimes_BN_2\simeq (N_1\otimes_AN_2)/D,  
  \]
  $D$ 为所有形如 $bx\otimes y-x\otimes by\ (b\in B,x\in N_1,y\in N_2)$ 的元素生成的
  $A$-子模。
\end{problem}
\begin{proof}
  考虑映射 $\varphi:N_1\times N_2\to N_1\otimes_BN_2$ 为
  \[
    \varphi(x,y)=x\otimes y,  
  \]
  那么 $\varphi$ 显然是 $A$-双线性映射,故诱导出 $A$-模
  同态 $\bar\varphi:N_1\otimes_AN_2\to N_1\otimes_BN_2$。由于
  \[
    \bar\varphi(b(x\otimes_A y))=  \bar\varphi(bx\otimes_Ay)=bx\otimes_By=
    b(x\otimes_By),
  \]
  所以 $\bar\varphi$ 也是 $B$-模同态。由于
  \[
    \bar\varphi(bx\otimes_Ay-x\otimes_Aby)
    =bx\otimes_By-x\otimes_Bby=0,  
  \]
  所以 $D\subseteq\ker\bar\varphi$,同时由于
  \[
    b'(bx\otimes_Ay-x\otimes_Aby)=b'bx\otimes_Ay-b'x\otimes_Aby
    =b(b'x)\otimes_Ay-(b'x)\otimes_aby\in D,  
  \]
  所以 $D$ 也是 $B$-子模。这表明 $\bar\varphi$ 诱导出 $B$-模同态
  $\psi:(N_1\otimes_AN_2)/D\to N_1\otimes_BN_2$。
  令 $\lambda:N_1\times N_2\to (N_1\otimes_AN_2)/D$ 为
  \[
      \lambda(x,y)=x\otimes_Ay+D,
  \]
  由于
  \begin{gather*}
    \lambda(bx,y)=bx\otimes_Ay+D=b(x\otimes_Ay+D)=b\lambda(x,y),\\
    \lambda(x,by)=x\otimes_Aby+D=bx\otimes_Ay+D=b(x\otimes_Ay+D)=b\lambda(x,y),
  \end{gather*}
  所以 $\lambda$ 是 $B$-双线性映射,故诱导出 $B$-模同态
  $\bar\lambda:N_1\otimes_BN_2\to (N_1\otimes_AN_2)/D$。
  容易验证 $\bar\lambda$ 和 $\psi$ 互为逆映射,故 $\psi$ 为同构。
\end{proof}

\begin{problem}
  令 $e_1,\dots,e_n$ 是 $k$-向量空间 $V$ 的一组基,那么对偶空间 $V^*\coloneqq \Hom_k(V,k)$ 有一组基
  $e_1^*,\dots,e_n^*$,它们满足
  \[
    e_i^*(e_j)  =\delta_{ij}\quad 1\leq i,j\leq n.
  \]
  令 $\varphi\in\End_k(V)$ 为自同态,那么诱导出 $\varphi^*\in\End_k(V^*)$ 为
  \[
    \varphi^*(f)=f\circ \varphi.  
  \]
  证明如果 $\varphi$ 在 $\{e_1,\dots,e_n\}$ 下的表示矩阵为 $A=(a_{ij})$,那么
  $\varphi^*$ 在 $\{e_1^*,\dots,e_n^*\}$ 下的表示矩阵为转置 $A^\top=(a_{ji})$。
\end{problem}
\begin{proof}
  任意 $f\in V^*$ 可以表示为
  \[
    f=\sum_{i=1}^n f(e_i)e_i^*,  
  \]
  所以
  \begin{align*}
    \varphi^*(e_j^*)&=e_j^*\circ\varphi=\sum_{k=1}^n (e_j^*\circ\varphi)(e_k)e_k^*\\
    &=\sum_{k=1}^ne_j^*\left(\sum_{i=1}^n a_{ik}e_i\right)e_k^*\\
    &=\sum_{k=1}^n a_{jk}e_k^*,
  \end{align*}
  即 $\varphi^*$ 在 $\{e_1^*,\dots,e_n^*\}$ 下的表示矩阵为转置 $A^\top=(a_{ji})$。
\end{proof}

\begin{problem}
  令 $V,W$ 分别为 $m,n$ 维 $k$-向量空间,$\{e_1,\dots,e_m\}$ 和 $\{u_1,\dots,u_n\}$ 分别是
  $V,W$ 的一组基,\autoref{coro:tensor product of vector space} 告诉我们
  $\{e_i\otimes u_j\}$ 构成了 $V\otimes W$ 的一组基。

  设线性变换 $\varphi:V\to V$ 相对于 $\{e_1,\dots,e_m\}$ 的表示矩阵为 $A=(a_{ij})$,
  线性变换 $\psi:W\to W$ 相对于 $\{u_1,\dots,u_n\}$ 的表示矩阵为 $B=(b_{ij})$,证明
  $\varphi\otimes \psi:V\otimes W\to V\otimes W$ 相对于 $\{e_i\otimes u_j\}=\{e_1\otimes u_1,\dots,e_1\otimes u_n,e_2\otimes u_1,\dots,e_m\otimes u_n\}$ 
  的表示矩阵为
  \[
    A\otimes B=
    \begin{pmatrix}
      a_{11} B & a_{12} B & \cdots  & a_{1m} B \\
      a_{21} B & a_{22} B & \cdots & a_{2m} B \\
      \vdots & \vdots & \ddots & \vdots \\
      a_{m1} B & a_{m2} B & \cdots & a_{mm} B 
    \end{pmatrix}  \in M_{mn}(k).
  \]
  上述乘法被称为矩阵的 Kronecker 积。
\end{problem}
\begin{proof}
  我们有
  \begin{align*}
    (\varphi\otimes \psi)(e_i\otimes u_j) &=
    \varphi(e_i)\otimes \psi(u_j)=\left(\sum_{k=1}^m a_{ki}e_k\right)\otimes \psi(u_j)\\
    &=\sum_{k=1}^m a_{ki}e_k\otimes \psi(u_j)
    =\sum_{k=1}^m a_{ki}\sum_{l=1}^n b_{lj} e_k\otimes u_j \\
    &=\sum_{k=1}^m \left(a_{ki}\sum_{l=1}^n b_{lj}\right)e_k\otimes u_j.\qedhere
  \end{align*}
\end{proof}


\begin{problem}
  证明 $m,n$ 互素的时候有 $(\mathbb{Z}/m\mathbb{Z})\otimes_{\mathbb{Z}}(\mathbb{Z}/n\mathbb{Z})=0$。
\end{problem}
\begin{proof}
  我们证明更一般的情况,即 
  $(\mathbb{Z}/m\mathbb{Z})\otimes_{\mathbb{Z}}(\mathbb{Z}/n\mathbb{Z})\simeq\mathbb{Z}/d\mathbb{Z}$,
  其中 $d=\gcd(m,n)$。由于 $\bar a\otimes \bar b=ab(\bar 1\otimes\bar 1)$,所以
  $(\mathbb{Z}/m\mathbb{Z})\otimes_{\mathbb{Z}}(\mathbb{Z}/n\mathbb{Z})$ 是由 $\bar 1\otimes\bar 1$
  生成的 $\mathbb{Z}$-模(循环群)。注意到
  \[
    m(\bar 1\otimes \bar 1)=\bar m\otimes \bar 1  =0,
    \quad 
    n(\bar 1\otimes \bar 1)=\bar 1\otimes \bar n  =0,
  \]
  所以 $\bar 1\otimes \bar 1$ 作为加法群元素的阶整除 $d$。另一方面,考虑映射
  $\varphi:(\mathbb{Z}/m\mathbb{Z})\times (\mathbb{Z}/n\mathbb{Z})\to\mathbb{Z}/d\mathbb{Z}$
  为 $\varphi(\bar a,\bar b)=\overline{ab}$,容易验证 $\varphi$ 是良定义的双线性映射,所以
  $\varphi$ 诱导出 $\mathbb{Z}$-模同态 
  $\bar\varphi:(\mathbb{Z}/m\mathbb{Z})\otimes_{\mathbb{Z}}
  (\mathbb{Z}/n\mathbb{Z})\to\mathbb{Z}/d\mathbb{Z}$,
  由于 $\bar\varphi(\bar 1\otimes \bar 1)=\bar 1$ 是 $d$ 阶元,所以 $\bar1\otimes\bar 1$ 的阶
  不可能小于 $d$,故 $\bar 1\otimes\bar 1$ 的阶等于 $d$,所以 $\bar\varphi$ 是同构。
\end{proof}

\begin{problem}\label{prob:isomorphism of tensor}
  令 $A$ 是环,$\ideal a$ 是理想,$M$ 是 $A$-模,证明 $(A/\ideal a)\otimes_A M$ 同构于
  $M/\ideal aM$。
\end{problem}
\begin{proof}
  我们有正合列
  $
    \begin{tikzcd}[cramped,column sep=small]
      0\arrow[r] &\ideal a\arrow[r,"i"] & A\arrow[r] & A/\ideal a\arrow[r] & 0
    \end{tikzcd}
  $,根据 $\uline\otimes_A M$ 的右正合性质,所以
  有正合列
  $
    \begin{tikzcd}[cramped,column sep=small]
      \ideal a\otimes M\arrow[r,"i\otimes\mathbb{1}"] & A\otimes M\arrow[r] & (A/\ideal a)\otimes M\arrow[r] & 0
    \end{tikzcd}
  $,于是我们有
  \[
    (A/\ideal a)\otimes M\simeq (A\otimes M)/\im(i\otimes\mathbb{1}).  
  \]
  我们知道 $A\otimes M\to M$ 有同构映射 $a\otimes x\mapsto ax$,与自然同态符合得到同态
  $A\otimes M\to M/\ideal aM$,容易验证这个同态的核就是 $\im(i\otimes\mathbb{1})$,
  所以 
  \[
    (A/\ideal a)\otimes M\simeq (A\otimes M)/\im(i\otimes\mathbb{1})
    \simeq M/\ideal aM.\qedhere
  \]
\end{proof}

\begin{problem}
  令 $A$ 是局部环,$M$ 和 $N$ 是有限生成 $A$-模,证明:如果 $M\otimes N=0$,那么
  $M=0$ 或者 $N=0$。
\end{problem}
\begin{proof}
  令 $k=A/\ideal m$ 是剩余域,$M\otimes_A N=0$ 表明 $k\otimes_A(M\otimes_A N)=0$,即
  $(k\otimes_A M)\otimes_AN=0$,根据上一题,有 $(M/\ideal mM)\otimes_A N=0$,此时
  $M/\ideal mM$ 可以视为 $k$-向量空间,故 $M/\ideal mM\simeq (M/\ideal mM)\otimes_k k$,故
  $(M/\ideal mM)\otimes_k (k\otimes_AN)=0$,这就表明 $(M/\ideal mM)\otimes_k (N/\ideal mN)=0$,
  而 $M/\ideal mM$ 和 $N/\ideal mN$ 都是有限维 $k$-向量空间,所以 $M/\ideal mM=0$ 或者 $N/\ideal mN=0$,
  即 $\ideal mM=M$ 或者 $\ideal mN=N$,根据 Nakayama 引理,所以 $M=0$ 或者 $N=0$。 
\end{proof}

\begin{problem}
  令 $M_i(i\in I)$ 是一族 $A$-模,令 $M$ 是它们的直和,证明 $M$ 是平坦模当且仅当每个
  $M_i$ 是平坦模。
\end{problem}
\begin{proof}
  设 $\varphi:N\to N'$ 是单同态,那么同态 $\varphi\otimes \mathbb{1}_{M_i}:N\otimes M_i\to N'\otimes M_i$
  可以分解为复合映射
  \[
    N\otimes M_i\xlongrightarrow{j_i}\bigoplus_{i\in I}(N\otimes M_i)
    \xlongrightarrow{\sim} N\otimes M\xlongrightarrow{\varphi\otimes\mathbb{1}_{M}}
    N'\otimes M\xlongrightarrow{\sim}
    \bigoplus_{i\in I}(N'\otimes M_i)\xlongrightarrow{\pi_i}N'\otimes M_i,
  \]
  不难验证所有的 $\varphi\otimes\mathbb{1}_{M_i}$ 为单射当且仅当 $\varphi\otimes \mathbb{1}_M$
  为单射,这就表明 $M$ 是平坦模当且仅当 $M_i$ 都是平坦模。
\end{proof}

\begin{problem}
  令 $A$ 是环,证明 $A[x]$ 是平坦 $A$-代数。
\end{problem}
\begin{proof}
  $A[x]=\bigoplus_{i=0}^\infty A$ 是自由 $A$-模,而 $A$ 是平坦 $A$-模,
  所以 $A[x]$ 是平坦 $A$-模。
\end{proof}

\begin{problem}
  令 $\ideal p$ 是 $A$ 的素理想,证明 $\ideal p[x]$ 是 $A[x]$ 的素理想。如果
  $\ideal m$ 是极大理想,$\ideal m[x]$ 是 $A[x]$ 的极大理想吗?
\end{problem}
\begin{proof}
  考虑满同态 $\varphi:A[x]\to A/\ideal p[x]$ 为
  \[
    \varphi\left(\sum a_ix^i\right)  =\sum (a_i+\ideal p)x^i,
  \]
  显然 $\ker\varphi=\ideal p[x]$,所以 $A[x]/\ideal p[x]\simeq A/\ideal p[x]$ 是整环,
  故 $\ideal p[x]$ 是 $A[x]$ 的素理想。类似地,
  $A[x]/\ideal m[x]\simeq A/\ideal m[x]$ 是域上的多项式环,为 PID,所以 $\ideal m[x]$ 不是极大理想。
\end{proof}

\begin{problem}
  (1) 如果 $M,N$ 是平坦 $A$-模,证明 $M\otimes_AN$ 也是平坦 $A$-模。\\
  (2) 如果 $B$ 是平坦 $A$-代数,$N$ 是平坦 $B$-模,那么 $N$ 也是平坦 $A$-模。
\end{problem}
\begin{proof}
  (1) 令 $\varphi:P\to P'$ 是 $A$-模的单同态,那么 $\varphi\otimes\mathbb{1}_M:P\otimes_AM\to P'\otimes_AM$
  是 $A$-模的单同态,进而 $\varphi\otimes\mathbb{1}_M\otimes\mathbb{1}_N:P\otimes_AM\otimes_AN\to P'\otimes_AM\otimes_AN$ 是 $A$-模的单同态,
  再根据张量积的结合性,所以 $M\otimes_AN$ 是平坦 $A$-模。

  (2) 令 $\varphi:P\to P'$ 是 $A$-模的单同态,那么 $\varphi\otimes \mathbb{1}_B:P\otimes_AB\to P'\otimes_AB$
  是 $A$-模的单同态,此时
  \[
      (\varphi\otimes\mathbb{1}_B)(b(x\otimes y))=x\otimes by=b(\varphi\otimes\mathbb{1}_B)(x\otimes y),
  \]
  所以 $\varphi$ 也是 $B$-模的单同态,故 $\varphi\otimes \mathbb{1}_B\otimes \mathbb{1}_N:
  (P\otimes_AB)\otimes_BN\to (P'\otimes_AB)\otimes_BN$ 是 $B$-模的单同态,从而也是
  $A$-模的单同态。又因为有 $B$-模
  同构 $(P\otimes_AB)\otimes_BN\simeq P\otimes_A(B\otimes_BN)\simeq P\otimes_AN$,所以
  $N$ 是平坦 $A$-模。
\end{proof}

\begin{problem}
  令 
  $
    \begin{tikzcd}[cramped,column sep=small]
      0\arrow[r] & M'\arrow[r,"u"] & M\arrow[r,"v"] & M''\arrow[r] & 0
    \end{tikzcd}
  $
  是 $A$-模的短正合列。如果 $M'$ 和 $M''$ 是有限生成的,那么 $M$ 也是有限生成的。
\end{problem}
\begin{proof}
  设 $x_1,\dots,x_n$ 是 $M'$ 的生成元,$y_1,\dots,y_k$ 是 $M''$ 的生成元,
  任取 $m\in M$,那么 $v(m)=a_1y_1+\cdots+a_ky_k$,$v$ 是满射表明存在
  $z_1,\dots,z_k\in M$ 使得 $y_i=v(z_i)$,所以
  $v(m)=v(a_1z_1+\cdots+a_kz_k)$,即 $m-a_1z_1-\cdots-a_kz_k\in\ker v=\im u$,
  故 $m-a_1z_1-\cdots-a_kz_k=u(b_1x_1+\cdots+b_nx_n)$,即
  $m=b_1u(x_1)+\cdots+b_nu(x_n)+a_1z_1+\cdots+a_kz_k$,所以
  $M$ 由 $u(x_1),\dots,u(x_n),z_1,\dots,z_k$ 生成。
\end{proof}

\begin{problem}
  $A$ 是环,理想 $\ideal a\subseteq\rad(A)$,令 $M$ 是 $A$-模,$N$ 是有限生成
  $A$-模,$u:M\to N$ 是同态。如果诱导的同态 $M/\ideal aM\to N/\ideal aN$
  是满射,那么 $u$ 是满射。
\end{problem}
\begin{proof}
  $M/\ideal aM\to N/\ideal aN$ 是满射表明任取 $n\in N$,有
  $n+\ideal aN=u(m)+\ideal aN$,即 $n\in u(M)+\ideal aN$,所以
  $N=u(M)+\ideal aN$,根据 \ref{coro:Nakayama},有 $N=u(M)$。
\end{proof}

\begin{problem}
  令 $A$ 是非零环,证明:
  \begin{enumerate}
    \item $A^m\simeq A^n$ 表明 $m=n$。
    \item 如果 $\phi:A^m\to A^n$ 是满射,那么 $m\geq n$。
    \item 如果 $\phi:A^m\to A^n$ 是单射,这种情况下一定有 $m\leq n$ 吗?
  \end{enumerate}
\end{problem}
\begin{proof}
  (1) 取 $A$ 的一个极大理想 $\ideal m$。设 $\varphi:A^m\to A^n$ 是同构映射,
  那么我们有正合列 
  $
    \begin{tikzcd}[cramped,column sep=small]
      0\arrow[r] & A^m\arrow[r,"\varphi"] & A^n\arrow[r] & 0
    \end{tikzcd},
  $
  记 $k=A/\ideal m$,根据张量积的右正合性质,所以
  $\varphi\otimes\mathbb{1}:A^m\otimes_Ak\to A^n\otimes_Ak$ 是 $A$-模同构,
  由 \autoref{prob:isomorphism of tensor} 可知 $A^m\otimes_Ak\simeq A^m/\ideal mA^m\simeq k^m$,
  所以有 $A$-模同构 $k^m\simeq k^n$,容易验证这同时是 $k$-向量空间同构,所以 $m=n$。

  (2) 仿照上面,取 $A$ 的一个极大理想 $\ideal m$。设 $\varphi:A^m\to A^n$ 是满射,
  那么我们有正合列 
  $
    \begin{tikzcd}[cramped,column sep=small]
      \ker\varphi \arrow[r] & A^m\arrow[r,"\varphi"] & A^n\arrow[r] & 0
    \end{tikzcd},
  $
  那么 $\varphi\otimes \mathbb{1}:A^m\otimes_Ak\to A^n\otimes_Ak$ 是满射,
  所以诱导出 $k^m\to k^n$ 的满线性映射,那么有 $m\geq n$。
\end{proof}

\begin{problem}
  令 $M$ 是有限生成 $A$-模,$\phi:M\to A^n$ 是满同态,证明 $\ker\phi$ 是有限生成的。
\end{problem}
\begin{proof}
  
\end{proof}


\begin{problem}
  一个偏序集 $I$ 被称为\emph{正向集},如果对于 $I$ 中的每对 $i,j$,都存在
  $k\in I$ 使得 $i\leq k$ 以及 $j\leq k$。

  令 $A$ 是环,$I$ 是一个正向集,$(M_i)_{i\in I}$ 是一族 $A$-模。对于 $I$ 中
  的每对满足 $i\leq j$ 的 $i,j$,令 $\mu_{ij}:M_i\to M_j$ 是 $A$-模同态,并且
  其满足下面的公理:
  \begin{enumerate}
    \item 对于每个 $i\in I$,$\mu_{ii}$ 是 $M_i$ 的单位映射;
    \item 当 $i\leq j\leq k$ 的时候有 $\mu_{ik}=\mu_{jk}\circ\mu_{ij}$。
  \end{enumerate}
  那么称模 $M_i$ 和同态 $\mu_{ij}$ 组成了正向集 $I$ 上的\emph{正向系统}
  $\mathbf{M}=(M_i,\mu_{ij})$。

  我们将构建正向系统 $\mathbf{M}$ 的\emph{正向极限} $M$。令 $C$ 是 $M_i$
  的直和,并且将 $M_i$ 自然地视为 $C$ 的子模。令 $D$ 是 $C$ 的子模,由
  所有的形如 $x_i-\mu_{ij}(x_i)\ (i\leq j,x_i\in M_i)$ 的元素生成。令
  $M=C/D$,$\mu:C\to M$ 为自然投影,$\mu_i$ 是 $\mu$ 在 $M_i$ 上的限制。

  模 $M$(更准确地说,应该是模 $M$ 与一族同态 $\mu_i:M_i\to M$)被称为
  正向系统 $\mathbf{M}$ 的\emph{正向极限},记为 $\varinjlim M_i$。从这个构造
  容易看出 $i\leq j$ 的时候有 $\mu_i=\mu_j\circ\mu_{ij}$。
\end{problem}
\begin{proof}
  我们说明 $\mu_i=\mu_j\circ\mu_{ij}$。任取 $x_i\in M_i$,有
  \[
    \mu_j\circ\mu_{ij}(x_i)=\mu_{ij}(x_i)+D=x_i+D=\mu_i(x_i).\qedhere
  \]
\end{proof}

\begin{problem}
  在上一题的前提下,证明
  \begin{enumerate}
    \item $M$ 的每个元素都可以写成 $\mu_i(x_i)$ 的形式,其中 $i\in I$,$x_i\in M_i$。
    \item 如果 $\mu_i(x_i)=0$,那么存在 $j\geq i$ 使得 $\mu_{ij}(x_i)=0$。
  \end{enumerate}
\end{problem}
\begin{proof}
  (1) 任取 $\sum_{j\in J}x_j+D\in M$,其中 $J\subseteq I$ 是有限子集。首先我们说明
  存在 $k\in I$ 使得任意的 $j\in J$ 都满足 $j\leq k$。对 $|J|=n$ 归纳,$n=2$ 的时候
  是正向集的定义,假设结论在 $n-1$ 时成立。那么存在 $k_1\in I$ 使得 $j_m\leq k_1\ (1\leq m\le n-1)$,
  同时存在 $k\in I$ 使得 $k_1\leq k$ 以及 $j_n\leq k$,那么 $k$ 就
  使得任意的 $j\in J$ 都满足 $j\leq k$。

  对于上述的 $k$,$x_j-\mu_{jk}(x_j)\in D\ (\forall j\in J)$,所以
  \[
    \sum_{j\in J}x_j+D=\sum_{j\in J}\mu_{jk}(x_j)+D=\mu_k\left(
      \sum_{j\in J}\mu_{jk}(x_j)
    \right)  ,
  \]
  其中 $\sum_{j\in J}\mu_{jk}(x_j)\in M_k$,这就证明了 (1)。

  (2) $\mu_i(x_i)=0$ 表明 $x_i\in D\cap M_i$,那么 $x_i$ 是形如 $y_j-\mu_{j\ell}(y_j)$ 的
  有限 $A$-线性组合,注意到 $a(y_j-\mu_{j\ell}(y_j))=(ay_j)-\mu_{j\ell}(ay_j)$,所以
  $x_i$ 可以表示为 $y_j-\mu_{j\ell}(y_j)$ 的有限和,即
  \[
    x_i=\sum_{k=1}^n \bigl(y_{j_k}-\mu_{j_k\ell_k}(y_{j_k})\bigr)  ,
  \]
  其中 $j_k,\ell_k\in I$,$y_{j_k}\in M_{j_k}$,并且 $j_k\leq\ell_k$。根据 (1) 的叙述,
  可以选取一个 $c\in I$,使得所有的 $j_k,\ell_k\leq c$ 以及 $i\leq c$。

  由于 $C$ 是 $M_i$ 的直和,所以上式右端的第 $i$ 个分量为 $x_i$,其余全为零。记
  $\pi_a:C\to M_a$ 是投影,那么
  \[
    \pi_a(x_i)=\sum_{j_k=a}y_{j_k}-\sum_{\ell_k=a}\mu_{j_k\ell_k}(y_{j_k}),  
  \]
  那么
  \begin{align*}
    \mu_{ac}(\pi_a(x_i))&=\sum_{j_k=a}\mu_{j_kc}(y_{j_k})
    -\sum_{\ell_k=a}\mu_{\ell_kc}\circ\mu_{j_k\ell_k}(y_{j_k})\\
    &=\sum_{j_k=a}\mu_{j_kc}(y_{j_k})
    -\sum_{\ell_k=a}\mu_{j_kc}(y_{j_k}),
  \end{align*}
  对 $a$ 求和,我们有
  \begin{gather*}
    \mu_{ic}(x_i)=\sum_{a\in I}\mu_{ac}(\pi_a(x_i))\\=
    (\mu_{j_1c}(y_{j_1})+\cdots+\mu_{j_kc}(y_{j_k}))-(\mu_{j_1c}(y_{j_1})+\cdots+\mu_{j_kc}(y_{j_k}))=0.
    \qedhere
  \end{gather*}
\end{proof}

\begin{problem}
  证明正向极限可以通过下面的性质刻画。令 $N$ 是 $A$-模,对于每个 $i\in I$,令 $\alpha_i:M_i\to N$
  是 $A$-模同态,并且使得 $i\leq j$ 的时候有 $\alpha_i=\alpha_j\circ\mu_{ij}$。那么存在
  唯一的同态 $\alpha:M\to N$ 使得 $\alpha_i=\alpha\circ\mu_i$ 对于所有的 $i\in I$ 成立。
\end{problem}
\begin{proof}
  首先说明正向极限 $M=\varinjlim M_i$ 满足这个性质,即有交换图
  \[
    \begin{tikzcd}[sep=2.4em]
      M_i\arrow[r,"\alpha_i"]\arrow[d,"\mu_i"'] & N \\
      \varinjlim M_i \arrow[ur,dashed,"\exists!\alpha"'] & 
    \end{tikzcd}  
  \]
  令 $\bar\alpha:C\to N$ 满足 $\bar\alpha|_{M_i}=\alpha_i$。由于
  \[
    \bar{\alpha}(x_i-\mu_{ij}(x_i))=\alpha_i(x_i)-\alpha_j\circ\mu_{ij}(x_i)=0,  
  \]
  所以 $D\subseteq\ker\bar\alpha$,故 $\bar\alpha$ 诱导出同态 $\alpha:M\to N$,那么此时有
  $\alpha_i(x_i)=\bar\alpha(x_i)=\alpha(x_i+D)=\alpha\circ\mu_i(x_i)$。

  现在证明 $M$ 在同构的意义下唯一。设 $M$ 和 $M'$ 都满足上述泛性质。
  对于同态 $\mu_i':M_i\to M'$,存在唯一的同态 $\alpha:M\to M'$
  使得 $\mu_i'=\alpha\circ\mu_i$。同样的,对于 $\mu_i:M_i\to M$,存在
  唯一的同态 $\beta:M'\to M$,使得 $\mu_i=\beta\circ\mu_i'$。
  也就是说,我们有下面的交换图:
  \[
    \begin{tikzcd}[sep=2.4em]
      M_i
      \arrow[dd,"\mu_{ij}"']
      \arrow[dr,"\mu_i"]
      \arrow[drr,bend left=15,"\mu_i'"]
      & &[.8em] \\
      &
      M
      \arrow[r,dashed,shift left,"\alpha"]
      &
      M' 
      \arrow[l,dashed,shift left,"\beta"]
      \\
      M_j
      \arrow[ur,"\mu_j"']
      \arrow[urr,bend right=15,"\mu_j'"']
      & & 
    \end{tikzcd}  
  \]
  所以 $\mu_i=\beta\circ\mu_i'=\beta\circ\alpha\circ\mu_i:M_i\to M$,又因为
  $\mathbb{1}_{M}$ 也满足 $\mu_i=\mathbb{1}_M\circ\mu_i$,根据唯一性,所以
  $\beta\circ\alpha=\mathbb{1}_M$。同理可得 $\alpha\circ\beta=\mathbb{1}_{M'}$,故
  $\alpha$ 是同构。
\end{proof}

\begin{problem}
  令 $(M_i)_{i\in I}$ 是一个 $A$-模的一族子模,并且对于每对 $i,j\in I$,都存在
  $k\in I$ 使得 $M_i+M_j\subseteq M_k$。定义 $i\leq j$ 意味着 $M_i\subseteq M_j$,
  令 $\mu_{ij}:M_i\to M_j$ 表示嵌入映射,证明
  \[
    \varinjlim M_i=\sum M_i=\bigcup M_i.  
  \]
  特别地,任意 $A$-模都是它的有限生成子模的正向极限。
\end{problem}
\begin{proof}
  先说明 $\sum M_i=\bigcup M_i$。显然有 $\bigcup M_i\subseteq \sum M_i$。任取
  $y=\sum x_i\in \sum M_i$ 是有限和,记 $S=\{i\in I\,|\, x_i\neq 0\}$,那么 $S$
  是有限集,$I$ 是正向集表明存在 $j\in I$ 使得 $i\leq j\ (\forall i\in S)$,所以
  $y\in M_j\subseteq\bigcup M_i$。这就表明 $\sum M_i=\bigcup M_i$。

  下面我们说明 $\bigcup M_i$ 满足上述泛性质。令 $\alpha_i:M_i\to N$ 是一族模同态,
  并且对于 $i\leq j$ 有 $\alpha_i=\alpha_j\circ\mu_{ij}$。对于每个 $i$
  总存在嵌入映射 $\mu_i:M_i\to \bigcup M_i$。令 $\alpha=\bigcup\alpha_i:\bigcup M_i\to N$,
  也就是说 $\alpha$ 限制在 $M_i$ 上就是 $\alpha_i$。这个 $\alpha$ 是良定义的,也就是说
  对于 $i\leq j$,任取 $x_i\in M_i\subseteq M_j$,
  有 $\alpha_i(x_i)=\alpha_j\circ\mu_{ij}(x_i)=\alpha_j(x_i)$。显然 $\alpha$ 是同态,且满足
  $\alpha_i=\alpha\circ\mu_i$。所以 $\varinjlim M_i\simeq\bigcup M_i$。

  对于一个 $A$-模 $M$,令 $(N_i)_{i\in I}$ 是 $M$ 的所有有限生成子模,注意到
  \[
    M=\bigcup_{x\in M} Ax\subseteq\bigcup N_i,  
  \]
  所以 $M$ 是所有有限生成子模的并集。对于每对 $i,j\in I$,此时 $N_i+N_j$ 也是有限生成子模,
  故所有的有限生成子模和包含关系构成了一个正向系统。
\end{proof}

\begin{problem}
  
\end{problem}

