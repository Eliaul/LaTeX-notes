\documentclass[fontset=none,zihao=-4]{Notes}

\makeatletter
\DeclareRobustCommand{\em}{%
  \@nomath\em \if b\expandafter\@car\f@series\@nil
  \normalfont \else \bfseries \fi}
\makeatother

\usepackage{tikz-cd,wrapstuff}
\usepackage{fixdif,siunitx,tikz,nicematrix}

\usetikzlibrary{hobby,calc,arrows}
\usetikzlibrary{positioning}
\usetikzlibrary{decorations.markings}
\usetikzlibrary{decorations.pathreplacing}

\ProvidesFile{font.def}

\setCJKmainfont{Source Han Serif SC}[
  UprightFont=*-Regular,
  BoldFont=*-Bold,
  ItalicFont=HYKaiTi S,
  ItalicFeatures={Scale=1.1}
]
\newCJKfontfamily[zhsong]\songti{Source Han Serif SC}[
  UprightFont=*-Regular,
  BoldFont=*-Bold,
  ItalicFont=HYKaiTi S,
  ItalicFeatures={Scale=1.1}
]
\setCJKsansfont{Source Han Sans SC}[
  UprightFont=*-Regular,
  BoldFont=*-Bold
]
\newCJKfontfamily[zhhei]\heiti{Source Han Sans SC}[
  UprightFont=*-Regular,
  BoldFont=*-Bold
]
\setCJKmonofont{HYFangSong S}[
  BoldFont=*,
  ItalicFont=*,
  BoldItalicFont=*
]
\newCJKfontfamily[zhfs]\fangsong{HYFangSong S}[
  BoldFont=*,
  ItalicFont=*,
  BoldItalicFont=*
]
\newCJKfontfamily[zhkai]\kaishu{HYKaiTi S}[
  BoldFont=*,
  ItalicFont=*,
  BoldItalicFont=*
]

\setmainfont{texgyretermes}[
  Extension=.otf,
  UprightFont=*-regular,
  BoldFont=*-bold,
  ItalicFont=*-italic,
  BoldItalicFont=*-bolditalic,
  SlantedFont=*-italic
]
%\setmathrm{texgyretermes}[
%  Extension=.otf,
%  UprightFont=*-regular,
%  BoldFont=*-bold,
%  ItalicFont=*-italic,
%  BoldItalicFont=*-bolditalic,
%  SlantedFont=*-italic
%]
\setsansfont{Cantarell}[
  UprightFont=* Regular,
  ItalicFont=* Italic,
  BoldFont=* Bold,
  BoldItalicFont=* Bold Italic,
  SmallCapsFont=Alegreya Sans SC
]
\setmonofont{Ubuntu Mono}[
  UprightFont=*,
  ItalicFont=* Italic,
  BoldFont=* Bold,
  BoldItalicFont=* Bold Italic
]
%\setmathfont{texgyretermes-math.otf}
%\setmathfont[range={\mathcal,\mathbfcal,\mathfrak},StylisticSet=1]{XITSMath-Regular.otf}
%\setmathfont[range={\mathbb}]{KpMath-Sans.otf}



\DeclareMathOperator\Int{Int}
\DeclareMathOperator\supp{supp}
\DeclareMathOperator\im{im}
\DeclareMathOperator\End{End}
\DeclareMathOperator\Ann{Ann}
\DeclareMathOperator\Hom{Hom}
\DeclareMathOperator\diag{diag}
\DeclareMathOperator\Or{O}
\DeclareMathOperator\rank{rank}
\DeclareMathOperator\Ob{Ob}
\DeclareMathOperator\id{id}

\newcommand{\cat}[1]{\mathsf{#1}}

\newcommand{\inn}[1]{\left\langle#1\right\rangle}
\newcommand{\norm}[1]{\left\lVert#1\right\rVert}
\newcommand{\spa}[1]{\operatorname{span}\left(#1\right)}


\makeatletter
\DeclareRobustCommand\bigop[2][1]{%
  \mathop{\vphantom{\sum}\mathpalette\bigop@{{#1}{#2}}}\slimits@
}
\newcommand{\bigop@}[2]{\bigop@@#1#2}
\newcommand{\bigop@@}[3]{%
  \vcenter{%
    \sbox\z@{$#1\sum$}%
    \hbox{\resizebox{\ifx#1\displaystyle#2\fi\dimexpr\ht\z@+\dp\z@}{!}{$\m@th#3$}}%
  }%
}
\makeatother

\newcommand{\fprod}{\DOTSB\bigop[1.05]{\ast}}

\tikzcdset{
  arrow style=tikz,
  diagrams={>={Straight Barb[scale=0.8]}}
}

\tikzset{
  point/.style={
    circle,fill,inner sep=0pt,minimum width=5pt
  }
}

\usepackage[subscriptcorrection,nofontinfo,mtpbb]{mtpro2}



\setlist[enumerate]{nosep,label=(\arabic*)}
\setlist[itemize]{nosep}

\title{\sffamily 范畴中的积与余积}
\author{Eliauk}


\begin{document}

\maketitle

% \tableofcontents

\section{积}

\begin{definition}
  令 $\cat C$ 是范畴,$\{X_i\}_{i\in I}$ 是 $\cat C$
  中的一族对象。如果对象 $P\in \Ob(\cat C)$ 和一族态射 $\{p_i:P\to X_i\}_{i\in I}$ 
  满足下面的范性质:给定任意 $Z\in\Ob(\cat C)$ 和一族态射
  $\{f_i:Z\to X_i\}_{i\in I}$,都存在唯一的态射 $\Phi:Z\to P$
  使得对于每个 $i\in I$ 都有 $f_i=p_i\circ\Phi$,也即下面的交换图成立:
  \[
    \begin{tikzcd}[column sep=huge]
    Z\arrow[r,"\exists !\Phi",dashed]\arrow[dr,"f_i"'] & P\arrow[d,"p_i"] \\
    & X_i.
    \end{tikzcd}
  \]
  那么我们说 $\bigl(P,\{p_i:P\to X_i\}_{i\in I}\bigr)$ 是 $\{X_i\}_{i\in I}$ 的积。通常我们记作
  $\prod_{i\in I} X_i$,态射 $p_i:\prod_{i\in I}X_i\to X_i$
  被称为自然投影。
\end{definition}

\begin{lemma}[积的唯一性]
  令 $\cat C$ 是范畴,$\{X_i\}_{i\in I}$ 是一族对象。如果
  有两个积 $\bigl(P,\{p_i:P\to X_i\}_{i\in I}\bigr)$ 和
  $\bigl(\wtilde P,\{\tilde p_i:\wtilde P\to X_i\}_{i\in I}\bigr)$,
  那么存在唯一的同构 $\Phi:\wtilde P\to P$ 和 $\wtilde \Phi:P\to\wtilde P$,
  使得它们互逆并且对于每个 $i\in I$ 都满足下面的交换图:
  \[
    \begin{tikzcd}[column sep=huge]
      \wtilde P\arrow[rr,"\Phi",yshift=2pt]
      \arrow[dr,"\tilde p_i"'] & 
      & P\arrow[ll,"\wtilde\Phi",yshift=-2pt]\arrow[dl,"p_i"] \\
      & X_i.
    \end{tikzcd}
  \]
\end{lemma}
\begin{proof}
  根据范性质,对于每个 $i\in I$,存在唯一的 $\Phi:\wtilde P\to P$ 使得
  $\tilde p_i=p_i\circ\Phi$,同时也存在唯一的 $\wtilde\Phi:P\to\wtilde P$
  使得 $p_i=\tilde p_i\circ\wtilde \Phi$。所以 $p_i=p_i\circ\bigl(\Phi\circ\wtilde\Phi\bigr)$,
  同时 $p_i=p_i\circ \id_P$,根据唯一性,所以 $\id_P=\Phi\circ\wtilde \Phi$。
  同理,有 $\id_{\wtilde P}=\wtilde{\Phi}\circ\Phi$。于是 $\Phi,\wtilde\Phi$
  是同构。
\end{proof}

\begin{proposition}[集合范畴的积]
  令 $\{X_i\}_{i\in I}$ 是 $\cat{Set}$ 中的一族对象。
  那么 $\cat{Set}$ 中的积就是集合的直积:
  \[
    \prod_{i\in I}X_i=\left\{
      f:I\to\bigcup_{i\in I} X_i\,\middle|\,
      \forall i\in I,f(i)\in X_i.
    \right\},
  \]
  附带显然的投影映射 $p_i:f\mapsto f(i)$。
\end{proposition}
\begin{proof}
  令 $Z\in\Ob(\cat{Set})$ 和 $\{f_i:Z\to X_i\}_{i\in I}$ 是一族映射,
  定义 $\Phi:Z\to \prod_{i\in I}X_i$ 为
  \[
    \Phi(z)=\bigl(
      i\mapsto f_i(z)
    \bigr).
  \]
  根据定义,显然有 $f_i=p_i\circ \Phi$。下面只需要说明 $\Phi$
  是唯一的。假设 $\Psi:Z\to\prod_{i\in I}X_i$ 也满足 $f_i=p_i\circ\Psi$。
  那么对于每个 $z\in Z$ 和 $i\in I$,有 $f_i(z)=p_i(\Psi(z))=\bigl(\Psi(z)\bigr)(i)$。
  另一方面,还有 $f_i(z)=p_i(\Phi(z))=\bigl(\Phi(z)\bigr)(i)$,
  所以 $\Phi(z)(i)=\Psi(z)(i)$,所以 $\Psi=\Phi$。 
\end{proof}

\begin{proposition}[群范畴和向量空间范畴的积]
  群范畴和向量空间范畴的积的构造与集合范畴一致,
  运算按照逐分量定义。
\end{proposition}

\begin{example}
  令 $(P,\leq)$ 是偏序集,其可以视为一个范畴。给定一族对象
  $\{X_i\}_{i\in I}$,也即 $P$ 中的一些元素。可以证明
  \[
    \prod_{i\in I} X_i=
    \begin{cases}
      \inf \bigl(\{X_i\}_{i\in I}\bigr) & 
      \text{$\{X_i\}_{i\in I}$ 的最大的下界},\\
      \emptyset & \text{若不存在下确界}.
    \end{cases}
  \]
  对于任意对象 $Z\in P$,若 $f_i:Z\to X_i$ 是态射,
  也即 $Z\leq X_i$,那么 $Z\leq \prod_{i\in I}X_i$,
  这表明存在态射 $\Phi:Z\to \prod_{i\in I}X_i$,于是
  $Z\leq \prod_{i\in I}X_i\leq X_i$,
  即 $f_i=p_i\circ\Phi$。
\end{example}

\section{余积}

\begin{definition}
  令 $\cat C$ 是范畴,$\{X_i\}_{i\in I}$ 是 $\cat C$
  中的一族对象。如果对象 $C\in \Ob(\cat C)$ 和一族态射 $\{\iota_i:X_i\to C\}_{i\in I}$ 
  满足下面的范性质:给定任意 $Z\in\Ob(\cat C)$ 和一族态射
  $\{f_i:X_i\to Z\}_{i\in I}$,都存在唯一的态射 $\Phi:C\to Z$
  使得对于每个 $i\in I$ 都有 $f_i=\Phi\circ\iota_i$,也即下面的交换图成立:
  \[
    \begin{tikzcd}[column sep=huge]
    X_i\arrow[d,"\iota_i"']\arrow[dr,"f_i"] & \\
    C\arrow[r,"\exists !\Phi"',dashed] & Z.
    \end{tikzcd}
  \]
  那么我们说 $\bigl(C,\{\iota_i:X_i\to C\}_{i\in I}\bigr)$ 是 $\{X_i\}_{i\in I}$ 的余积。
\end{definition}

可以看出,$\cat C$ 中的余积实际上就是反范畴 $\cat C^{\mathrm{op}}$
中的积。所以不难证明余积也是唯一的。

\begin{proposition}[集合范畴的余积]
  令 $\{X_i\}_{i\in I}$ 是 $\cat{Set}$ 中的一族对象。
  那么 $\cat{Set}$ 中的余积是集合的无交并:
  \[
    \coprod_{i\in I}X_i=\bigcup_{i\in I} \bigl(
      X_i\times \{i\}
    \bigr),
  \]
  附带自然的嵌入映射。
\end{proposition}
\begin{proof}
  任取 $Z\in\Ob(\cat{Set})$ 以及态射 $\{f_i:X_i\to Z\}_{i\in I}$,
  定义 $\Phi:\coprod_{i\in I}X_i\to Z$ 为
  \[
    \Phi(x_i,i)=f_i(x_i).
  \]
  根据定义立马得出 $f_i=\Phi\circ\iota_i$。假设 $\Psi:\coprod_{i\in I}X_i\to Z$
  也满足 $f_i=\Psi\circ\iota_i$。那么对于每个 $i\in I$
  和 $x_i\in X_i$,有 $\Phi(x_i,i)=f_i(x_i)=\Psi(x_i,i)$,
  这就表明 $\Psi=\Phi$。
\end{proof}

\begin{proposition}[Abel 群范畴的余积]
  令 $\{X_i\}_{i\in I}$ 是 $\cat{Ab}$ 中的一族对象,那么
  $\cat{Ab}$ 中的余积是直和
  \[
    \bigoplus_{i\in I} X_i=
    \left\{
      (g_i)_{i\in I}\in\prod_{i\in I} X_i\,\middle|\,
      \text{只有有限多个 $g_i$ 是非平凡的}
    \right\}.
  \]
  附带同态映射 $\iota_i:X_i\to \bigoplus_{i\in I}X_i$ 为 
  \[
    \iota_i(x_i)(j)=\begin{cases}
      x_i & j=i,\\
      0_{X_i} & j\neq i.
    \end{cases} 
  \] 
\end{proposition}
\begin{proof}
  任取 $Z\in\Ob(\cat{Ab})$ 以及态射 $\{f_i:X_i\to Z\}_{i\in I}$,
  定义 $\Phi:\bigoplus_{i\in I}X_i\to Z$ 为 
  \[
    \Phi\bigl((g_i)_{i\in I}\bigr)=
    \sum_{i\in I} f_i\bigl(g_i(i)\bigr).
  \]
  注意右边只有有限多项不为 $0_Z$,所以求和是有意义的。
  根据定义,对于每个 $j\in I$,有
  \[
    \Phi\circ\iota_j(x_j)=\sum_{i\in I}f_i\bigl(\iota_j(x_j)(i)\bigr)
    =f_j\bigl(\iota_j(x_j)(j)\bigr)=f_j(x_j),
  \]
  所以 $f_j=\Phi\circ \iota_j$。假设 $\Psi:\bigoplus_{i\in I}X_i\to Z$
  也满足 $f_j=\Psi\circ \iota_j$。对于任意 $(g_i)_{i\in I}\in \bigoplus_{i\in I}X_i$,
  假设 $g_{i_1},\dots,g_{i_k}$ 是非平凡的,那么有 
  \[
    \Psi\bigl((g_i)_{i\in I}\bigr)=\Psi\left(
      \sum_{r=1}^k g_{i_r}
    \right)=\sum_{r=1}^k\Psi\bigl(g_{i_r}\bigr)
    =\sum_{r=1}^k\Psi\circ\iota_{i_r}\bigl(g_{i_r}(i_r)\bigr)
    =\sum_{r=1}^k f_{i_r}\bigl(g_{i_r}(i_r)\bigr)=\Phi\bigl((g_i)_{i\in I}\bigr),
  \]
  这就表明 $\Psi=\Phi$。
\end{proof}

\begin{proposition}
  向量空间范畴的余积和 Abel 群范畴的余积构造相同。
\end{proposition}

注意,群范畴中的余积并不是 Abel 群范畴中直和的构造。在 $\cat{Ab}$ 中,对于直和 $X_1\oplus X_2$,
同态 $f_1:X_1\to Z$ 和 $f_2:X_2\to Z$ 导出同态 $f:X_1\oplus X_2\to Z$,满足
$f(x_1,x_2)=f_1(x_1)+f_2(x_2)$。而 $f$ 是群同态要求 
\[
  f(x_1+y_1,x_2+y_2)=f_1(x_1+y_1)+f_2(x_2+y_2)=f_1(x_1)+f_2(x_2)+f_1(y_1)+f_2(y_2),
\]
这里是使用了 $Z$ 的交换性的。所以说这种构造不适用于一般的群范畴。
为了构造群范畴的余积,我们需要定义一种名为\emph{自由积}的新概念。

\begin{definition}
  令 $\{X_s\}_{s\in S}$ 的一族群。
  \begin{enumerate}
    \item $\{X_s\}_{s\in S}$ 中的\emph{既约字}指的是一个有限长序列 $(x_1,\dots,x_m)$ 满足:
    \begin{itemize}
      \item 每个 $x_i$ 都是某个群 $X_s$ 中的元素,
      \item 每个 $x_i$ 都不是某个群 $X_s$ 的单位元,
      \item 任意两个连续的 $x_j$ 是两个不同群的元素。
    \end{itemize}
    此外,我们允许空序列 $()$。
    \item 记 $\fprod_{s\in S}X_s$ 为所有既约字的集合。
    \item 令 $(x_1,\dots,x_m)$ 和 $(y_1,\dots,y_n)$ 是 $\fprod_{s\in S} X_s$ 中的两个元素。
    我们定义乘法如下:
    \begin{enumerate}[label=(\alph*)]
      \item 若 $x_m$ 和 $y_1$ 属于不同的群,那么定义
      \[
        (x_1,\dots,x_m)\cdot (y_1,\dots,y_n)=(x_1,\dots,x_m,y_1,\dots,y_n).
      \]
      \item 若 $x_m$ 和 $y_1$ 是同一个群的元素。令 $j\in \{1,\dots,\min\{m,n\}\}$
      是使得 $x_{m+1-i}$ 和 $y_i$ 处于同一个群并且 $x_{m+1-i}=y_i^{-1}$ 的最大的数。
      如果 $j<\min\{m,n\}$,那么定义
      \[
        (x_1,\dots,x_m)\cdot (y_1,\dots,y_n)=(x_1,\dots,x_{m-j}\cdot y_{j+1},\dots,y_n).
      \]
      如果 $j=m$,那么定义
      \[
        (x_1,\dots,x_m)\cdot (y_1,\dots,y_n)=(y_{m+1},\dots,y_n).
      \]
      如果 $j=n$,那么定义
      \[
        (x_1,\dots,x_m)\cdots (y_1,\dots,y_n)=(x_1,\dots,x_{m-n}).
      \]
    \end{enumerate}
    \item 我们说 $\fprod_{s\in S}X_s$ 附带上述乘法是 $X_s$ 的自由积。
    对于有限多个 $X_1,\dots,X_k$,记作 $X_1\ast \cdots\ast X_k$。
  \end{enumerate}
\end{definition}

\begin{example}
  考虑自由积 $\langle {\color{blue}s}\rangle \ast \langle {\color{red} t}\rangle$。其中的元素形如
  $({\color{blue} s^3},{\color{red} t^{-5}},{\color{blue} s^2})$ 以及 $({\color{blue} s^{-2}},{\color{red} t^4},{\color{blue} s^2},{\color{red} t^{-1}})$。
  它们的自由积为
  \[
    ({\color{blue} s^3},{\color{red} t^{-5}},{\color{blue} s^2})\cdot ({\color{blue} s^{-2}},{\color{red} t^4},{\color{blue} s^2},{\color{red} t^{-1}})
    =({\color{blue} s^3},{\color{red} t^{-1}},{\color{blue} s^2},{\color{red} t^{-1}}).
  \]
  注意 $({\color{blue} s})\cdot ({\color{red} t})=({\color{blue} s},{\color{red} t})\neq ({\color{red} t},{\color{blue} s})=({\color{red} t})\cdot ({\color{blue} s})$,
  所以自由积一般来说是高度非交换的。
\end{example}

\begin{lemma}
  令 $\{X_s\}_{s\in S}$ 是一族群。那么自由积 $\fprod_{s\in S}X_s$ 确实是一个群。
  $\fprod_{s\in S}X_s$ 的单位元是空序列 $()$。$(x_1,\dots,x_m)\in \fprod_{s\in S}X_s$
  的逆元是 $(x_m^{-1},\dots,x_1^{-1})\in \fprod_{s\in S}X_s$。
\end{lemma}
\begin{proof}
  验证结合律是非常繁琐的,这里略去。
\end{proof}

\begin{proposition}[群范畴的余积]
  令 $\{X_s\}_{s\in S}$ 是一族群。对于任意 $t\in S$,定义映射
  $\iota_t:X_t\to \fprod_{s\in S}X_s$ 为
  \[
    \iota_t(x_t)=\begin{cases}
      (x_t) & x_t\neq e,\\
      () & x_t=e.
    \end{cases}
  \]
  那么 $\iota_t$ 是单同态。并且 $\fprod_{s\in S}X_s$ 附带同态 
  $\iota_t:X_t\to\fprod_{s\in S}X_s$ 是 $\cat{Grp}$ 中的余积。
\end{proposition}
\begin{proof}
  验证 $\iota_t$ 是单同态只需要按照定义即可。任取 $Z\in\Ob(\cat{Grp})$
  和一族群同态 $\{f_s:X_s\to Z\}_{s\in S}$,定义 $\Phi:\fprod_{s\in S}X_s\to Z$
  为 
  \[
    \Phi(x_1,\dots,x_m)=f_{s_1}(x_1)\cdots f_{s_m}(x_m),
  \]
  其中 $x_i\in X_{s_i}$。不难验证 $\Phi$ 是群同态。任取 $s\in S$,有
  \[
    \Phi(\iota_s(x_s))=\Phi((x_s))=f_s(x_s),
  \]
  所以 $f_s=\Phi\circ\iota_s$。下面只需要说明 $\Phi$ 的唯一性。
  假设群同态 $\Psi:\fprod_{s\in S}\to Z$ 也满足 $f_s=\Psi\circ\iota_s$。
  那么对于任意 $s\in S$,有
  \[
    \Psi(x_1,\dots,x_m)=\Psi\left(\iota_{s_1}(x_1)\cdots\iota_{s_m}(x_m)\right)
    =f_{s_1}(x_1)\cdots f_{s_m}(x_m)=\Phi(x_1,\dots,x_m),
  \]
  所以 $\Psi=\Phi$。
\end{proof}




\end{document}