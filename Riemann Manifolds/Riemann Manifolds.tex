\documentclass[fontset=none]{Notes}

\makeatletter
\DeclareRobustCommand{\em}{%
  \@nomath\em \if b\expandafter\@car\f@series\@nil
  \normalfont \else \bfseries \fi}
\makeatother

\usepackage{tikz-cd,wrapstuff}
\usepackage{siunitx,tikz,nicematrix}
\usetikzlibrary{matrix,calc}

\ProvidesFile{font.def}

\setCJKmainfont{Source Han Serif SC}[
  UprightFont=*-Regular,
  BoldFont=*-Bold,
  ItalicFont=HYKaiTi S,
  ItalicFeatures={Scale=1.1}
]
\newCJKfontfamily[zhsong]\songti{Source Han Serif SC}[
  UprightFont=*-Regular,
  BoldFont=*-Bold,
  ItalicFont=HYKaiTi S,
  ItalicFeatures={Scale=1.1}
]
\setCJKsansfont{Source Han Sans SC}[
  UprightFont=*-Regular,
  BoldFont=*-Bold
]
\newCJKfontfamily[zhhei]\heiti{Source Han Sans SC}[
  UprightFont=*-Regular,
  BoldFont=*-Bold
]
\setCJKmonofont{HYFangSong S}[
  BoldFont=*,
  ItalicFont=*,
  BoldItalicFont=*
]
\newCJKfontfamily[zhfs]\fangsong{HYFangSong S}[
  BoldFont=*,
  ItalicFont=*,
  BoldItalicFont=*
]
\newCJKfontfamily[zhkai]\kaishu{HYKaiTi S}[
  BoldFont=*,
  ItalicFont=*,
  BoldItalicFont=*
]

\setmainfont{texgyretermes}[
  Extension=.otf,
  UprightFont=*-regular,
  BoldFont=*-bold,
  ItalicFont=*-italic,
  BoldItalicFont=*-bolditalic,
  SlantedFont=*-italic
]
%\setmathrm{texgyretermes}[
%  Extension=.otf,
%  UprightFont=*-regular,
%  BoldFont=*-bold,
%  ItalicFont=*-italic,
%  BoldItalicFont=*-bolditalic,
%  SlantedFont=*-italic
%]
\setsansfont{Cantarell}[
  UprightFont=* Regular,
  ItalicFont=* Italic,
  BoldFont=* Bold,
  BoldItalicFont=* Bold Italic,
  SmallCapsFont=Alegreya Sans SC
]
\setmonofont{Ubuntu Mono}[
  UprightFont=*,
  ItalicFont=* Italic,
  BoldFont=* Bold,
  BoldItalicFont=* Bold Italic
]
%\setmathfont{texgyretermes-math.otf}
%\setmathfont[range={\mathcal,\mathbfcal,\mathfrak},StylisticSet=1]{XITSMath-Regular.otf}
%\setmathfont[range={\mathbb}]{KpMath-Sans.otf}



\usepackage[subscriptcorrection,nofontinfo,mtpbb,mtpfrak]{mtpro2}
\usepackage[normal]{fixdif}

\tikzcdset{
  arrow style=tikz,
  diagrams={>={Straight Barb[scale=0.8]}}
}

\allowdisplaybreaks[1]

\newlength{\mymathln}
\newcommand{\aligninside}[2]{
  \settowidth{\mymathln}{#2}
  \mathmakebox[\mymathln]{#1}
}

\DeclareMathOperator\Spec{Spec}
\DeclareMathOperator\im{im}
\DeclareMathOperator\sgn{sgn}
\DeclareMathOperator\rad{rad}
\DeclareMathOperator\Alt{Alt}
\DeclareMathOperator\Max{Max}
\DeclareMathOperator\GL{GL}
\DeclareMathOperator\Orth{O}
\DeclareMathOperator\SO{SO}
\DeclareMathOperator\SU{SU}
\DeclareMathOperator\Lie{Lie}
\DeclareMathOperator\End{End}
\DeclareMathOperator\Int{Int}
\DeclareMathOperator\Sym{Sym}
\DeclareMathOperator\Tr{Tr}
\DeclareMathOperator\Hom{Hom}
\DeclareMathOperator\supp{supp}
\DeclareMathOperator\Id{Id}
\DeclareMathOperator\rk{rank}
\DeclareMathOperator\grad{grad}
\DeclareMathOperator\Euc{E}
\newcommand{\LL}{{\mathrm{L}}}

\newcommand{\ideal}[1]{\mathfrak{#1}}
\newcommand{\mat}[1]{\mathbold{#1}}
\newcommand{\uline}{\underline{\hphantom{X}}}
\newcommand{\abs}[1]{\left|#1\right|}
\newcommand{\lie}[1]{\mathfrak{#1}}
\newcommand{\inn}[1]{\left\langle #1\right\rangle}

\usepackage{enumitem}

\setlist[enumerate]{nosep}

%\DeclareMathAlphabet\mathcal{OMS}{cmsy}{m}{n}

\newlength\stextwidth
\newcommand\makesamewidth[3][c]{%
  \settowidth{\stextwidth}{#2}%
  \makebox[\stextwidth][#1]{#3}%
}



\begin{document}

\frontmatter

\tableofcontents

\mainmatter

\chapter{黎曼度量}


\chapter{联络}

\section{向量场作微分的问题}

令 $I\subseteq \mathbb{R}$ 是区间, $\gamma:I\to \mathbb{R}^n$ 是光滑曲线,
在标准坐标中可以表示为 $\gamma(t)=\bigl(\gamma^1(t),\dots,\gamma^n(t)\bigr)$。
这样的曲线有良定义的\emph{速度} $\gamma'(t)$ 和\emph{加速度} $\gamma''(t)$,
计算为
\begin{align}\label{eq:velocity of curve in Euclid space}
  \gamma'(t)&=\dot{\gamma}^1(t)\frac{\partial}{\partial x^1}\bigg|_{\gamma(t)}
  +\cdots+\dot\gamma^n(t)\frac{\partial}{\partial x^n}\bigg|_{\gamma(t)},\\
  \gamma''(t)&=\ddot{\gamma}^1(t)\frac{\partial}{\partial x^1}\bigg|_{\gamma(t)}
  +\cdots+\ddot\gamma^n(t)\frac{\partial}{\partial x^n}\bigg|_{\gamma(t)}.
\end{align}
注意到 $\gamma$ 是直线当且仅当 $\gamma''(t)\equiv 0$。

我们也可以定义 $\mathbb{R}^n$ 上的向量场的方向导数,只需要计算标准坐标中
分量函数的方向导数即可:给定向量场 $Y\in \mathfrak{X}(\mathbb{R}^n)$
和向量 $v\in T_p \mathbb{R}^n$,定义\emph{$Y$ 在 $v$ 方向上的 Euclid 方向导数}
为
\[
  \wbar\nabla_v Y=v(Y^1)\frac{\partial}{\partial x^1}\bigg|_p
  + \cdots + v(Y^n)\frac{\partial}{\partial x^n}\bigg|_p,
\]
其中 $v(Y^i)$ 为
\[
  v(Y^i)=v^1\frac{\partial Y^i}{\partial x^1}(p)+\cdots +
  v^n\frac{\partial Y^i}{\partial x^n}(p).
\]
如果 $X$ 是另一个向量场,我们可以通过在每个点处计算 $\wbar\nabla_{X_p}Y$
得到一个新的向量场 $\wbar\nabla_XY$:
\begin{equation}
  \wbar\nabla_XY=X(Y^1)  \frac{\partial}{\partial x^1}+\cdots+
  X(Y^n)  \frac{\partial}{\partial x^n}.
\end{equation}

更一般地,我们可以对 $\mathbb{R}^n$ 的子流形上的曲线和向量场做同样的推广。
假设 $M\subseteq \mathbb{R}^n$ 是嵌入子流形,考虑光滑曲线 $\gamma:I\to M$。
我们想将 $M$ 中的测地线想象为一条“尽可能直”的曲线。当然,如果 $M$ 本身是弯曲的,
那么 $\gamma'(t)$ (视为 $\mathbb{R}^n$ 中的向量)必须要有所不同的定义,
否则曲线将会离开 $M$。一种方式是计算上面的 Euclid 加速度 $\gamma''(t)$,
然后使用切向投影 $\pi^\top:T_{\gamma(t)}\mathbb{R}^n\to T_{\gamma(t)}M$。
这会导出切向于 $M$ 的向量 $\gamma''(t)^\top=\pi^\top(\gamma''(t))$,
我们称为\emph{$\gamma$ 的切向加速度}。此时可以合理地认为,当 $M$ 中曲线
的切向加速度为零的时候,它是尽可能直的。

类似地,假设 $Y$ 是 $M$ 上的一个光滑向量场,我们想知道 $Y$ 沿着 
$v\in T_pM$ 方向在 $M$ 中变化了多少。一种合理的方式是将 $Y$ 延拓为
$\mathbb{R}^n$ 的某个开子集上的光滑向量场 $\wtilde Y$,然后计算
$\wtilde Y$ 在 $v$ 方向的 Euclid 方向导数,然后正交投影到 $T_pM$ 上。
即定义\emph{$Y$ 在 $v$ 方向上的切向方向导数}为
\begin{equation}
  \nabla_v^\top Y=\pi^\top\bigl(\wbar\nabla_v\wtilde Y\bigr).
\end{equation}
这个切向方向导数可以证明是良定义的并且保持刚体运动。但是,此时没有理由
相信切向方向导数是 $M$ 是内在不变量(即仅仅依赖于 $M$ 上的诱导度量)。

在抽象的黎曼流形上,由于没有“环境 Euclid 空间”的存在,因此该技巧不可用。
因此我们必须找到某种方法理解抽象黎曼流形中光滑曲线的加速度。令 $\gamma:I\to M$
是光滑曲线,我们知道在 $t\in I$ 时刻 $\gamma$ 的速度被定义为切向量
$\gamma'(t)\in T_{\gamma(t)}M$,其在坐标中的表示为 \eqref{eq:velocity of curve in Euclid space} 式。

但是,与速度不同,加速度并没有这样一个坐标无关的解释。例如,考虑参数化的圆周
$\gamma(t)=(\cos t,\sin t)$。作为 $\mathbb{R}^2$ 中的曲线,其有加速度
\[
  \gamma''(t)=-\cos t\frac{\partial}{\partial x}\bigg|_{\gamma(t)}
  -\sin t   \frac{\partial}{\partial y}\bigg|_{\gamma(t)}.
\]
但是在极坐标中,同样的曲线被描述为 $(r(t),\theta(t))= (1,t)$,在这个坐标下,
加速度变为
\[
  \gamma''(t)=r''(t)\frac{\partial}{\partial r}\bigg|_{\gamma(t)}   
  +\theta''(t)\frac{\partial}{\partial \theta}\bigg|_{\gamma(t)}=0.
\]

总的来说,问题是这样的:为了通过对 $\gamma'(t)$ 微分来定义 $\gamma''(t)$,
我们必须对向量 $\gamma'(t+h)$ 和 $\gamma'(t)$ 的差商取极限,但是
$\gamma'(t+h)$ 和 $\gamma'(t)$ 生活在不同的切空间中,所以它们的减法是没有意义的。
而在 $\mathbb{R}^n$ 中光滑曲线的加速度能够计算是因为它的每个切空间都可以自然地
视为 $\mathbb{R}^n$ 本身。在一般的流形上,是不存在这样的自然的等同的。

速度向量 $\gamma'(t)$ 是沿着曲线的速度场的一个例子。为了解释在流形中曲线的加速度,
我们需要的是某种独立于坐标的方式去将向量场沿曲线做微分。为此,我们需要
一种方法来比较向量场在不同点的值,或者直观的说,即“联接”相邻的切空间。
这引出了联络的概念:这是流形上的一个额外的数据,一种计算向量场的方向导数的规则。

\section{联络}

事实证明首先定义联络作为区分向量丛截面的方式是最简单的。这个定义
旨在捕获 Euclid 方向导数算符和切向方向导数算符($\wbar\nabla$ 和 $\wbar\nabla^\top$)
的基本性质。在定义一般情况的联络之后,我们将把定义应用到向量丛沿曲线的情况。

令 $\pi:E\to M$ 是光滑流形 $M$ 上的光滑向量丛,$\Gamma(E)$ 是 $E$ 的光滑截面空间。
\emph{$E$ 中的联络}指的是一个映射
\[
  \nabla:\mathfrak{X}(M)\times \Gamma(E)\to\Gamma(E),  
\]
通常记为 $(X,Y)\mapsto \nabla_XY$,其满足下面的属性:
\begin{enumerate}
  \item $\nabla_XY$ 在 $X$ 上是 $C^\infty(M)$-线性的:对于
  $f_1,f_2\in C^\infty(M)$ 和 $X_1,X_2\in \mathfrak{X}(M)$,有
  \[
    \nabla_{f_1X_1+f_2X_2}Y=f_1\nabla_{X_1}Y+f_2\nabla_{X_2}Y.  
  \]
  \item $\nabla_XY$ 在 $Y$ 上是 $\mathbb{R}$-线性的:对于
  $a_1,a_2\in \mathbb{R}$ 和 $Y_1,Y_2\in\Gamma(E)$,有
  \[
    \nabla_X(a_1Y_1+a_2Y_2)=a_1\nabla_XY_1+a_2\nabla_XY_2.  
  \]
  \item $\nabla$ 满足乘积法则:对于 $f\in C^\infty(M)$,有
  \[
    \nabla_X(fY)=f\nabla_XY+(Xf)Y.  
  \]
\end{enumerate}
$\nabla_XY$ 被称为\emph{$Y$ 在 $X$ 方向的协变导数}。

针对不同的情况有很多种类型的联络。我们这里定义的联络被称为\emph{Koszul 联络}。
由于我们在本书中不需要考虑其他类型的联络,因此我们将 Koszul 联络简称为联络。

尽管联络是在全局截面上定义的,但是实际上这是一个\emph{局部算符}。

\begin{lemma}[局部性]\label{lemma:locality of connection}
  假设 $\nabla$ 是光滑向量丛 $\pi:E\to M$ 中的联络。对于任意 $X\in \mathfrak{X}(M)$,
  $Y\in \Gamma(E)$ 和 $p\in M$,协变导数 $\nabla_XY|_p$ 只与 $X,Y$ 在 $p$ 处的一个任意小
  的邻域上有关。准确的说,如果在 $p$ 的某个邻域上有 $X=\wtilde X$ 和 $Y=\wtilde Y$,
  那么 $\nabla_XY|_p=\nabla_{\wtilde X}\wtilde Y|_p$。
\end{lemma}
\begin{proof}
  首先考虑 $Y$。通过将 $Y$ 替换为 $Y-\wtilde Y$,我们只需要说明如果 $Y$ 在 $p$
  的某个邻域上为零,那么 $\nabla_XY|_p=0$。假设 $Y$ 在 $p$ 的某个邻域 $U$ 上为零,
  选取支在 $U$ 中的满足 $\varphi(p)=1$ 的鼓包函数 $\varphi\in C^\infty(M)$。
  于是在 $M$ 中有 $\varphi Y\equiv 0$。那么对于 $X\in \mathfrak{X}(M)$,我们有
  $\nabla_X(\varphi Y)=\nabla_X(0\cdot \varphi Y)=0\nabla_X(\varphi Y)=0$,
  所以
  \[
    0=\nabla_X(\varphi Y)=\varphi\nabla_XY+(X\varphi)Y,  
  \]
  注意到在 $\varphi$ 的支集上有 $Y\equiv 0$,所以 $(X\varphi)Y\equiv 0$,
  故 $\varphi\nabla_XY=0$,所以 $\nabla_XY|_p=0$。

  然后考虑 $X$,同样的,假设 $X$ 在 $p$ 的某个邻域 $U$ 上为零,选取支在 $U$ 中的满足 $\varphi(p)=1$ 的鼓包函数 $\varphi\in C^\infty(M)$。
  于是在 $M$ 中有 $\varphi X\equiv 0$。那么任取 $Y\in\Gamma(E)$,有
  $\nabla_{\varphi X}Y=\nabla_{0\cdot\varphi X}Y=0\nabla_{\varphi X}Y=0$,所以
  \[
    0=\nabla_{\varphi X}Y=\varphi\nabla_XY,  
  \]
  于是 $\nabla_XY|_p=0$。

  利用上面两点,如果在 $p$ 的某个邻域上有 $X=\wtilde X$ 和 $Y=\wtilde Y$,
  那么就有
  \[
    \nabla_{\wtilde X}\wtilde Y|_p
    =  \nabla_{X}\wtilde Y|_p=
    \nabla_XY|_p.\qedhere
  \]
\end{proof}
 
\begin{proposition}[联络的限制]\label{prop:restriction of connection}
  假设 $\nabla$ 是光滑向量丛 $E\to M$ 中的联络。对于每个开子集 $U\subseteq M$,
  存在唯一的在限制丛 $E|_U$ 上的联络 $\nabla^U$ 满足:对于每个
  $X\in \mathfrak{X}(M)$ 和 $Y\in\Gamma(E)$,有
  \begin{equation}
    \nabla_{X|_U}^U(Y|_U)=(\nabla_XY)|_U.
  \end{equation}
\end{proposition}
\begin{proof}
  首先我们证明唯一性。假设 $\nabla^U$ 是这样一个联络,$X\in \mathfrak{X}(U)$
  和 $Y\in\Gamma(E|_U)$ 是任意的。给定 $p\in U$,我们可以使用鼓包函数去构造一个
  光滑向量场 $\wtilde X\in \mathfrak{X}(M)$ 和光滑截面 $\wtilde Y\in \Gamma(E)$
  使得 $\wtilde X|_U$ 和 $ X$ 在 $p$ 的某个邻域上重合,
  $\wtilde Y|_U$ 和 $ Y$ 在 $p$ 的某个邻域上重合,那么
  \autoref{lemma:locality of connection} 表明
  \[
    \nabla_X^UY\big|_p= \nabla_{\wtilde X|_U}^U (\wtilde Y|_U)\big|_p
    =(\nabla_{\wtilde X}\wtilde Y)|_p.
  \]
  右边完全由 $\nabla$ 确定,所以这表明 $\nabla^U$ 如果存在则唯一。

  为了证明存在性,给定 $X\in \mathfrak{X}(U)$ 和 $Y\in \Gamma(E|_U)$,
  对于每个 $p\in U$,我们与上面一样的方式构造 $\wtilde X$ 和 $\wtilde Y$,
  然后定义 $\nabla_X^UY|_p=(\nabla_{\wtilde X}\wtilde Y)|_p$,根据 \autoref{lemma:locality of connection},
  这与 $\wtilde X$ 和 $\wtilde Y$ 的选取无关,不难验证这满足联络的定义。
\end{proof}

在上述命题的情况下,我们通常将 $\nabla$ 的限制仍记为 $\nabla$,这个命题保证
这样简写没有歧义。

\autoref{lemma:locality of connection} 告诉我们我们可以在仅仅知道
$X,Y$ 在 $p$ 的某个邻域的值的情况下计算 $\nabla_XY$。实际上,下面的命题表明,
对于 $X$ 而言,我们甚至只需要知道 $X$ 在 $p$ 处一个点的值即可。

\begin{proposition}\label{prop:connection depends on X at one point}
  在 \autoref{lemma:locality of connection} 的假设下,$\nabla_XY|_p$
  仅仅依赖于 $Y$ 在 $p$ 的某个邻域上的值以及 $X$ 在 $p$ 处的值。
\end{proposition}
\begin{proof}
  关于 $Y$ 的断言在 \autoref{lemma:locality of connection} 已经说明。
  对于 $X$,根据线性性,只需要说明在 $X_p=0$ 的情况下有
  $\nabla_XY|_p=0$ 即可。选取 $p$ 的一个坐标邻域 $U$,那么 $X$ 可以局部表示为
  $X=X^i\partial_i$,满足 $X^i(p)=0$。对于每个 $Y\in\Gamma(E|_U)$,我们有
  \[
    \nabla_XY|_p=X^i(p)\nabla_{\partial_i}Y|_p=0.\qedhere  
  \]
\end{proof}

多亏了 \autoref{prop:restriction of connection} 和 \autoref{prop:connection depends on X at one point},
我们可以定义记号 $\nabla_vY$,其中 $v$ 是 $T_pM$ 的某个元素,$Y$ 是 $E$ 的定义在
$p$ 的某个邻域上的光滑局部截面。如果我们要计算它,只需要令 $X$ 是 $p$
的邻域上的向量场并且 $X|_p=v$,然后令 $\nabla_vY=\nabla_XY|_p$ 即可。
\autoref{prop:connection depends on X at one point} 表明这个结果不依赖于
延拓的选取。此后,我们将以这种方式解释丛的局部截面的协变导数,而不再进一步注释。

\subsection{切丛中的联络}

切丛中的联络通常被称为\emph{$M$ 上的联络}(有时也被称为\emph{仿射联络}或者\emph{线性联络})。

设 $M$ 是光滑流形,根据定义,$TM$ 上的联络是一个映射
\[
  \nabla:\mathfrak{X}(M)\times \mathfrak{X}(M)\to \mathfrak{X}(M)  
\]
且满足联络的三个条件。为了计算,我们需要检查联络如何作用在局部标架上。
令 $(E_i)$ 是 $TM$ 的在开子集 $U\subseteq M$ 上的光滑局部标架。
对于任意指标 $i$ 和 $j$,我们可以将向量场 $\nabla_{E_i}E_j$ 在同一组标架
下表示为
\begin{equation}
  \nabla_{E_i}E_j=\Gamma_{ij}^k E_k.
\end{equation}
这定义了 $n^3$ 个光滑函数 $\Gamma_{ij}^k:U\to \mathbb{R}$,称为
$\nabla$ 相对于这组标架的\emph{联络系数}。下面的命题表明联络由联络系数
完全确定。

\begin{proposition}
  令 $M$ 是光滑流形,$\nabla$ 是 $TM$ 中的联络。假设 $(E_i)$ 是开子集
  $U\subseteq M$ 上的光滑局部标架,$\{\Gamma_{ij}^k\}$ 是联络系数。
  对于光滑向量场 $X,Y\in \mathfrak{X}(U)$,设 $X=X^iE_i$,
  $Y=Y^jE_j$,那么
  \begin{equation}
    \nabla_XY=\bigl(X(Y^k)+X^iY^j\Gamma_{ij}^k\bigr)E_k.
  \end{equation}
\end{proposition}
\begin{proof}
  只需要利用联络的定义进行计算:
  \begin{align*}
    \nabla_XY&=\nabla_X\bigl(Y^jE_j\bigr)\\
    &=X\bigl(Y^j\bigr)E_j+Y^j\nabla_{X^iE_i}E_j\\
    &=X\bigl(Y^j\bigr)E_j+X^iY^j\nabla_{E_i}E_j\\
    &=X\bigl(Y^j\bigr)E_j+X^iY^j\Gamma_{ij}^kE_k.\qedhere
  \end{align*}
\end{proof}

一旦联络系数在某个局部标架中确定,下面的命题表明在同一开集上任意其他的
局部标架中的联络系数也可以确定。

\begin{proposition}[联络系数的变换法则]
  令 $M$ 是光滑流形,$\nabla$ 是 $TM$ 中的联络。假设在开集 $U\subseteq M$
  上有两个光滑局部标架 $(E_i)$ 和 $\bigl(\wtilde E_j\bigr)$,其中
  $\wtilde E_i=A_i^jE_j$。令 $\Gamma_{ij}^k$ 和 $\wtilde\Gamma_{ij}^k$
  分别为 $\nabla$ 在这两组标架中的联络系数,那么 
  \begin{equation}
    \wtilde \Gamma_{ij}^k=\bigl(A^{-1}\bigr)_p^kA_i^qA_j^r\Gamma_{qr}^p
    +\bigl(A^{-1}\bigr)_p^kA_i^qE_q\bigl(A_j^p\bigr).
  \end{equation}
\end{proposition}


\end{document}
