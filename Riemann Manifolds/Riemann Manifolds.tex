\documentclass[fontset=none]{Notes}

\makeatletter
\DeclareRobustCommand{\em}{%
  \@nomath\em \if b\expandafter\@car\f@series\@nil
  \normalfont \else \bfseries \fi}
\makeatother

\usepackage{tikz-cd,wrapstuff}
\usepackage{siunitx,tikz,nicematrix}
\usetikzlibrary{matrix,calc}

\ProvidesFile{font.def}

\setCJKmainfont{Source Han Serif SC}[
  UprightFont=*-Regular,
  BoldFont=*-Bold,
  ItalicFont=HYKaiTi S,
  ItalicFeatures={Scale=1.1}
]
\newCJKfontfamily[zhsong]\songti{Source Han Serif SC}[
  UprightFont=*-Regular,
  BoldFont=*-Bold,
  ItalicFont=HYKaiTi S,
  ItalicFeatures={Scale=1.1}
]
\setCJKsansfont{Source Han Sans SC}[
  UprightFont=*-Regular,
  BoldFont=*-Bold
]
\newCJKfontfamily[zhhei]\heiti{Source Han Sans SC}[
  UprightFont=*-Regular,
  BoldFont=*-Bold
]
\setCJKmonofont{HYFangSong S}[
  BoldFont=*,
  ItalicFont=*,
  BoldItalicFont=*
]
\newCJKfontfamily[zhfs]\fangsong{HYFangSong S}[
  BoldFont=*,
  ItalicFont=*,
  BoldItalicFont=*
]
\newCJKfontfamily[zhkai]\kaishu{HYKaiTi S}[
  BoldFont=*,
  ItalicFont=*,
  BoldItalicFont=*
]

\setmainfont{texgyretermes}[
  Extension=.otf,
  UprightFont=*-regular,
  BoldFont=*-bold,
  ItalicFont=*-italic,
  BoldItalicFont=*-bolditalic,
  SlantedFont=*-italic
]
%\setmathrm{texgyretermes}[
%  Extension=.otf,
%  UprightFont=*-regular,
%  BoldFont=*-bold,
%  ItalicFont=*-italic,
%  BoldItalicFont=*-bolditalic,
%  SlantedFont=*-italic
%]
\setsansfont{Cantarell}[
  UprightFont=* Regular,
  ItalicFont=* Italic,
  BoldFont=* Bold,
  BoldItalicFont=* Bold Italic,
  SmallCapsFont=Alegreya Sans SC
]
\setmonofont{Ubuntu Mono}[
  UprightFont=*,
  ItalicFont=* Italic,
  BoldFont=* Bold,
  BoldItalicFont=* Bold Italic
]
%\setmathfont{texgyretermes-math.otf}
%\setmathfont[range={\mathcal,\mathbfcal,\mathfrak},StylisticSet=1]{XITSMath-Regular.otf}
%\setmathfont[range={\mathbb}]{KpMath-Sans.otf}



\usepackage[subscriptcorrection,nofontinfo,mtpbb,mtpfrak]{mtpro2}
\usepackage[normal]{fixdif}

\tikzcdset{
  arrow style=tikz,
  diagrams={>={Straight Barb[scale=0.8]}}
}

\allowdisplaybreaks[1]

\newlength{\mymathln}
\newcommand{\aligninside}[2]{
  \settowidth{\mymathln}{#2}
  \mathmakebox[\mymathln]{#1}
}

\DeclareMathOperator\Spec{Spec}
\DeclareMathOperator\im{im}
\DeclareMathOperator\sgn{sgn}
\DeclareMathOperator\rad{rad}
\DeclareMathOperator\Alt{Alt}
\DeclareMathOperator\Max{Max}
\DeclareMathOperator\GL{GL}
\DeclareMathOperator\Orth{O}
\DeclareMathOperator\SO{SO}
\DeclareMathOperator\SU{SU}
\DeclareMathOperator\Lie{Lie}
\DeclareMathOperator\End{End}
\DeclareMathOperator\Int{Int}
\DeclareMathOperator\Sym{Sym}
\DeclareMathOperator\tr{tr}
\DeclareMathOperator\Hom{Hom}
\DeclareMathOperator\supp{supp}
\DeclareMathOperator\Id{Id}
\DeclareMathOperator\rk{rank}
\DeclareMathOperator\grad{grad}
\DeclareMathOperator\Euc{E}
\newcommand{\LL}{{\mathrm{L}}}

\newcommand{\ideal}[1]{\mathfrak{#1}}
\newcommand{\mat}[1]{\mathbold{#1}}
\newcommand{\uline}{\underline{\hphantom{X}}}
\newcommand{\abs}[1]{\left|#1\right|}
\newcommand{\lie}[1]{\mathfrak{#1}}
\newcommand{\inn}[1]{\left\langle #1\right\rangle}

\usepackage{enumitem}

\setlist[enumerate]{nosep}

%\DeclareMathAlphabet\mathcal{OMS}{cmsy}{m}{n}

\newlength\stextwidth
\newcommand\makesamewidth[3][c]{%
  \settowidth{\stextwidth}{#2}%
  \makebox[\stextwidth][#1]{#3}%
}



\begin{document}

\frontmatter

\tableofcontents

\mainmatter

\chapter{黎曼度量}


\chapter{联络}

\section{向量场作微分的问题}

令 $I\subseteq \mathbb{R}$ 是区间, $\gamma:I\to \mathbb{R}^n$ 是光滑曲线,
在标准坐标中可以表示为 $\gamma(t)=\bigl(\gamma^1(t),\dots,\gamma^n(t)\bigr)$。
这样的曲线有良定义的\emph{速度} $\gamma'(t)$ 和\emph{加速度} $\gamma''(t)$,
计算为
\begin{align}\label{eq:velocity of curve in Euclid space}
  \gamma'(t)&=\dot{\gamma}^1(t)\frac{\partial}{\partial x^1}\bigg|_{\gamma(t)}
  +\cdots+\dot\gamma^n(t)\frac{\partial}{\partial x^n}\bigg|_{\gamma(t)},\\
  \gamma''(t)&=\ddot{\gamma}^1(t)\frac{\partial}{\partial x^1}\bigg|_{\gamma(t)}
  +\cdots+\ddot\gamma^n(t)\frac{\partial}{\partial x^n}\bigg|_{\gamma(t)}.
\end{align}
注意到 $\gamma$ 是直线当且仅当 $\gamma''(t)\equiv 0$。

我们也可以定义 $\mathbb{R}^n$ 上的向量场的方向导数,只需要计算标准坐标中
分量函数的方向导数即可:给定向量场 $Y\in \mathfrak{X}(\mathbb{R}^n)$
和向量 $v\in T_p \mathbb{R}^n$,定义\emph{$Y$ 在 $v$ 方向上的 Euclid 方向导数}
为
\[
  \wbar\nabla_v Y=v(Y^1)\frac{\partial}{\partial x^1}\bigg|_p
  + \cdots + v(Y^n)\frac{\partial}{\partial x^n}\bigg|_p,
\]
其中 $v(Y^i)$ 为
\[
  v(Y^i)=v^1\frac{\partial Y^i}{\partial x^1}(p)+\cdots +
  v^n\frac{\partial Y^i}{\partial x^n}(p).
\]
如果 $X$ 是另一个向量场,我们可以通过在每个点处计算 $\wbar\nabla_{X_p}Y$
得到一个新的向量场 $\wbar\nabla_XY$:
\begin{equation}
  \wbar\nabla_XY=X(Y^1)  \frac{\partial}{\partial x^1}+\cdots+
  X(Y^n)  \frac{\partial}{\partial x^n}.
\end{equation}

更一般地,我们可以对 $\mathbb{R}^n$ 的子流形上的曲线和向量场做同样的推广。
假设 $M\subseteq \mathbb{R}^n$ 是嵌入子流形,考虑光滑曲线 $\gamma:I\to M$。
我们想将 $M$ 中的测地线想象为一条“尽可能直”的曲线。当然,如果 $M$ 本身是弯曲的,
那么 $\gamma'(t)$ (视为 $\mathbb{R}^n$ 中的向量)必须要有所不同的定义,
否则曲线将会离开 $M$。一种方式是计算上面的 Euclid 加速度 $\gamma''(t)$,
然后使用切向投影 $\pi^\top:T_{\gamma(t)}\mathbb{R}^n\to T_{\gamma(t)}M$。
这会导出切向于 $M$ 的向量 $\gamma''(t)^\top=\pi^\top(\gamma''(t))$,
我们称为\emph{$\gamma$ 的切向加速度}。此时可以合理地认为,当 $M$ 中曲线
的切向加速度为零的时候,它是尽可能直的。

类似地,假设 $Y$ 是 $M$ 上的一个光滑向量场,我们想知道 $Y$ 沿着 
$v\in T_pM$ 方向在 $M$ 中变化了多少。一种合理的方式是将 $Y$ 延拓为
$\mathbb{R}^n$ 的某个开子集上的光滑向量场 $\wtilde Y$,然后计算
$\wtilde Y$ 在 $v$ 方向的 Euclid 方向导数,然后正交投影到 $T_pM$ 上。
即定义\emph{$Y$ 在 $v$ 方向上的切向方向导数}为
\begin{equation}
  \nabla_v^\top Y=\pi^\top\bigl(\wbar\nabla_v\wtilde Y\bigr).
\end{equation}
这个切向方向导数可以证明是良定义的并且保持刚体运动。但是,此时没有理由
相信切向方向导数是 $M$ 是内在不变量(即仅仅依赖于 $M$ 上的诱导度量)。

在抽象的黎曼流形上,由于没有“环境 Euclid 空间”的存在,因此该技巧不可用。
因此我们必须找到某种方法理解抽象黎曼流形中光滑曲线的加速度。令 $\gamma:I\to M$
是光滑曲线,我们知道在 $t\in I$ 时刻 $\gamma$ 的速度被定义为切向量
$\gamma'(t)\in T_{\gamma(t)}M$,其在坐标中的表示为 \eqref{eq:velocity of curve in Euclid space} 式。

但是,与速度不同,加速度并没有这样一个坐标无关的解释。例如,考虑参数化的圆周
$\gamma(t)=(\cos t,\sin t)$。作为 $\mathbb{R}^2$ 中的曲线,其有加速度
\[
  \gamma''(t)=-\cos t\frac{\partial}{\partial x}\bigg|_{\gamma(t)}
  -\sin t   \frac{\partial}{\partial y}\bigg|_{\gamma(t)}.
\]
但是在极坐标中,同样的曲线被描述为 $(r(t),\theta(t))= (1,t)$,在这个坐标下,
加速度变为
\[
  \gamma''(t)=r''(t)\frac{\partial}{\partial r}\bigg|_{\gamma(t)}   
  +\theta''(t)\frac{\partial}{\partial \theta}\bigg|_{\gamma(t)}=0.
\]

总的来说,问题是这样的:为了通过对 $\gamma'(t)$ 微分来定义 $\gamma''(t)$,
我们必须对向量 $\gamma'(t+h)$ 和 $\gamma'(t)$ 的差商取极限,但是
$\gamma'(t+h)$ 和 $\gamma'(t)$ 生活在不同的切空间中,所以它们的减法是没有意义的。
而在 $\mathbb{R}^n$ 中光滑曲线的加速度能够计算是因为它的每个切空间都可以自然地
视为 $\mathbb{R}^n$ 本身。在一般的流形上,是不存在这样的自然的等同的。

速度向量 $\gamma'(t)$ 是沿着曲线的速度场的一个例子。为了解释在流形中曲线的加速度,
我们需要的是某种独立于坐标的方式去将向量场沿曲线做微分。为此,我们需要
一种方法来比较向量场在不同点的值,或者直观的说,即“联接”相邻的切空间。
这引出了联络的概念:这是流形上的一个额外的数据,一种计算向量场的方向导数的规则。

\section{联络}

事实证明首先定义联络作为区分向量丛截面的方式是最简单的。这个定义
旨在捕获 Euclid 方向导数算符和切向方向导数算符($\wbar\nabla$ 和 $\wbar\nabla^\top$)
的基本性质。在定义一般情况的联络之后,我们将把定义应用到向量丛沿曲线的情况。

令 $\pi:E\to M$ 是光滑流形 $M$ 上的光滑向量丛,$\Gamma(E)$ 是 $E$ 的光滑截面空间。
\emph{$E$ 中的联络}指的是一个映射
\[
  \nabla:\mathfrak{X}(M)\times \Gamma(E)\to\Gamma(E),  
\]
通常记为 $(X,Y)\mapsto \nabla_XY$,其满足下面的属性:
\begin{enumerate}
  \item $\nabla_XY$ 在 $X$ 上是 $C^\infty(M)$-线性的:对于
  $f_1,f_2\in C^\infty(M)$ 和 $X_1,X_2\in \mathfrak{X}(M)$,有
  \[
    \nabla_{f_1X_1+f_2X_2}Y=f_1\nabla_{X_1}Y+f_2\nabla_{X_2}Y.  
  \]
  \item $\nabla_XY$ 在 $Y$ 上是 $\mathbb{R}$-线性的:对于
  $a_1,a_2\in \mathbb{R}$ 和 $Y_1,Y_2\in\Gamma(E)$,有
  \[
    \nabla_X(a_1Y_1+a_2Y_2)=a_1\nabla_XY_1+a_2\nabla_XY_2.  
  \]
  \item $\nabla$ 满足乘积法则:对于 $f\in C^\infty(M)$,有
  \[
    \nabla_X(fY)=f\nabla_XY+(Xf)Y.  
  \]
\end{enumerate}
$\nabla_XY$ 被称为\emph{$Y$ 在 $X$ 方向的协变导数}。

针对不同的情况有很多种类型的联络。我们这里定义的联络被称为\emph{Koszul 联络}。
由于我们在本书中不需要考虑其他类型的联络,因此我们将 Koszul 联络简称为联络。

尽管联络是在全局截面上定义的,但是实际上这是一个\emph{局部算符}。

\begin{lemma}[局部性]\label{lemma:locality of connection}
  假设 $\nabla$ 是光滑向量丛 $\pi:E\to M$ 中的联络。对于任意 $X\in \mathfrak{X}(M)$,
  $Y\in \Gamma(E)$ 和 $p\in M$,协变导数 $\nabla_XY|_p$ 只与 $X,Y$ 在 $p$ 处的一个任意小
  的邻域上有关。准确的说,如果在 $p$ 的某个邻域上有 $X=\wtilde X$ 和 $Y=\wtilde Y$,
  那么 $\nabla_XY|_p=\nabla_{\wtilde X}\wtilde Y|_p$。
\end{lemma}
\begin{proof}
  首先考虑 $Y$。通过将 $Y$ 替换为 $Y-\wtilde Y$,我们只需要说明如果 $Y$ 在 $p$
  的某个邻域上为零,那么 $\nabla_XY|_p=0$。假设 $Y$ 在 $p$ 的某个邻域 $U$ 上为零,
  选取支在 $U$ 中的满足 $\varphi(p)=1$ 的鼓包函数 $\varphi\in C^\infty(M)$。
  于是在 $M$ 中有 $\varphi Y\equiv 0$。那么对于 $X\in \mathfrak{X}(M)$,我们有
  $\nabla_X(\varphi Y)=\nabla_X(0\cdot \varphi Y)=0\nabla_X(\varphi Y)=0$,
  所以
  \[
    0=\nabla_X(\varphi Y)=\varphi\nabla_XY+(X\varphi)Y,  
  \]
  注意到在 $\varphi$ 的支集上有 $Y\equiv 0$,所以 $(X\varphi)Y\equiv 0$,
  故 $\varphi\nabla_XY=0$,所以 $\nabla_XY|_p=0$。

  然后考虑 $X$,同样的,假设 $X$ 在 $p$ 的某个邻域 $U$ 上为零,选取支在 $U$ 中的满足 $\varphi(p)=1$ 的鼓包函数 $\varphi\in C^\infty(M)$。
  于是在 $M$ 中有 $\varphi X\equiv 0$。那么任取 $Y\in\Gamma(E)$,有
  $\nabla_{\varphi X}Y=\nabla_{0\cdot\varphi X}Y=0\nabla_{\varphi X}Y=0$,所以
  \[
    0=\nabla_{\varphi X}Y=\varphi\nabla_XY,  
  \]
  于是 $\nabla_XY|_p=0$。

  利用上面两点,如果在 $p$ 的某个邻域上有 $X=\wtilde X$ 和 $Y=\wtilde Y$,
  那么就有
  \[
    \nabla_{\wtilde X}\wtilde Y|_p
    =  \nabla_{X}\wtilde Y|_p=
    \nabla_XY|_p.\qedhere
  \]
\end{proof}
 
\begin{proposition}[联络的限制]\label{prop:restriction of connection}
  假设 $\nabla$ 是光滑向量丛 $E\to M$ 中的联络。对于每个开子集 $U\subseteq M$,
  存在唯一的在限制丛 $E|_U$ 上的联络 $\nabla^U$ 满足:对于每个
  $X\in \mathfrak{X}(M)$ 和 $Y\in\Gamma(E)$,有
  \begin{equation}
    \nabla_{X|_U}^U(Y|_U)=(\nabla_XY)|_U.
  \end{equation}
\end{proposition}
\begin{proof}
  首先我们证明唯一性。假设 $\nabla^U$ 是这样一个联络,$X\in \mathfrak{X}(U)$
  和 $Y\in\Gamma(E|_U)$ 是任意的。给定 $p\in U$,我们可以使用鼓包函数去构造一个
  光滑向量场 $\wtilde X\in \mathfrak{X}(M)$ 和光滑截面 $\wtilde Y\in \Gamma(E)$
  使得 $\wtilde X|_U$ 和 $ X$ 在 $p$ 的某个邻域上重合,
  $\wtilde Y|_U$ 和 $ Y$ 在 $p$ 的某个邻域上重合,那么
  \autoref{lemma:locality of connection} 表明
  \[
    \nabla_X^UY\big|_p= \nabla_{\wtilde X|_U}^U (\wtilde Y|_U)\big|_p
    =(\nabla_{\wtilde X}\wtilde Y)|_p.
  \]
  右边完全由 $\nabla$ 确定,所以这表明 $\nabla^U$ 如果存在则唯一。

  为了证明存在性,给定 $X\in \mathfrak{X}(U)$ 和 $Y\in \Gamma(E|_U)$,
  对于每个 $p\in U$,我们与上面一样的方式构造 $\wtilde X$ 和 $\wtilde Y$,
  然后定义 $\nabla_X^UY|_p=(\nabla_{\wtilde X}\wtilde Y)|_p$,根据 \autoref{lemma:locality of connection},
  这与 $\wtilde X$ 和 $\wtilde Y$ 的选取无关,不难验证这满足联络的定义。
\end{proof}

在上述命题的情况下,我们通常将 $\nabla$ 的限制仍记为 $\nabla$,这个命题保证
这样简写没有歧义。

\autoref{lemma:locality of connection} 告诉我们我们可以在仅仅知道
$X,Y$ 在 $p$ 的某个邻域的值的情况下计算 $\nabla_XY$。实际上,下面的命题表明,
对于 $X$ 而言,我们甚至只需要知道 $X$ 在 $p$ 处一个点的值即可。

\begin{proposition}\label{prop:connection depends on X at one point}
  在 \autoref{lemma:locality of connection} 的假设下,$\nabla_XY|_p$
  仅仅依赖于 $Y$ 在 $p$ 的某个邻域上的值以及 $X$ 在 $p$ 处的值。
\end{proposition}
\begin{proof}
  关于 $Y$ 的断言在 \autoref{lemma:locality of connection} 已经说明。
  对于 $X$,根据线性性,只需要说明在 $X_p=0$ 的情况下有
  $\nabla_XY|_p=0$ 即可。选取 $p$ 的一个坐标邻域 $U$,那么 $X$ 可以局部表示为
  $X=X^i\partial_i$,满足 $X^i(p)=0$。对于每个 $Y\in\Gamma(E|_U)$,我们有
  \[
    \nabla_XY|_p=X^i(p)\nabla_{\partial_i}Y|_p=0.\qedhere  
  \]
\end{proof}

多亏了 \autoref{prop:restriction of connection} 和 \autoref{prop:connection depends on X at one point},
我们可以定义记号 $\nabla_vY$,其中 $v$ 是 $T_pM$ 的某个元素,$Y$ 是 $E$ 的定义在
$p$ 的某个邻域上的光滑局部截面。如果我们要计算它,只需要令 $X$ 是 $p$
的邻域上的向量场并且 $X|_p=v$,然后令 $\nabla_vY=\nabla_XY|_p$ 即可。
\autoref{prop:connection depends on X at one point} 表明这个结果不依赖于
延拓的选取。此后,我们将以这种方式解释丛的局部截面的协变导数,而不再进一步注释。

\subsection{切丛中的联络}

切丛中的联络通常被称为\emph{$M$ 上的联络}(有时也被称为\emph{仿射联络}或者\emph{线性联络})。

设 $M$ 是光滑流形,根据定义,$TM$ 上的联络是一个映射
\[
  \nabla:\mathfrak{X}(M)\times \mathfrak{X}(M)\to \mathfrak{X}(M)  
\]
且满足联络的三个条件。为了计算,我们需要检查联络如何作用在局部标架上。
令 $(E_i)$ 是 $TM$ 的在开子集 $U\subseteq M$ 上的光滑局部标架。
对于任意指标 $i$ 和 $j$,我们可以将向量场 $\nabla_{E_i}E_j$ 在同一组标架
下表示为
\begin{equation}
  \nabla_{E_i}E_j=\Gamma_{ij}^k E_k.
\end{equation}
这定义了 $n^3$ 个光滑函数 $\Gamma_{ij}^k:U\to \mathbb{R}$,称为
$\nabla$ 相对于这组标架的\emph{联络系数}。下面的命题表明联络由联络系数
完全确定。

\begin{proposition}
  令 $M$ 是光滑流形,$\nabla$ 是 $TM$ 中的联络。假设 $(E_i)$ 是开子集
  $U\subseteq M$ 上的光滑局部标架,$\{\Gamma_{ij}^k\}$ 是联络系数。
  对于光滑向量场 $X,Y\in \mathfrak{X}(U)$,设 $X=X^iE_i$,
  $Y=Y^jE_j$,那么
  \begin{equation}
    \nabla_XY=\bigl(X(Y^k)+X^iY^j\Gamma_{ij}^k\bigr)E_k.
  \end{equation}
\end{proposition}
\begin{proof}
  只需要利用联络的定义进行计算:
  \begin{align*}
    \nabla_XY&=\nabla_X\bigl(Y^jE_j\bigr)\\
    &=X\bigl(Y^j\bigr)E_j+Y^j\nabla_{X^iE_i}E_j\\
    &=X\bigl(Y^j\bigr)E_j+X^iY^j\nabla_{E_i}E_j\\
    &=X\bigl(Y^j\bigr)E_j+X^iY^j\Gamma_{ij}^kE_k.\qedhere
  \end{align*}
\end{proof}

一旦联络系数在某个局部标架中确定,下面的命题表明在同一开集上任意其他的
局部标架中的联络系数也可以确定。

\begin{proposition}[联络系数的变换法则]
  令 $M$ 是光滑流形,$\nabla$ 是 $TM$ 中的联络。假设在开集 $U\subseteq M$
  上有两个光滑局部标架 $(E_i)$ 和 $\bigl(\wtilde E_j\bigr)$,其中
  $\wtilde E_i=A_i^jE_j$。令 $\Gamma_{ij}^k$ 和 $\wtilde\Gamma_{ij}^k$
  分别为 $\nabla$ 在这两组标架中的联络系数,那么 
  \begin{equation}
    \wtilde \Gamma_{ij}^k=\bigl(A^{-1}\bigr)_p^kA_i^qA_j^r\Gamma_{qr}^p
    +\bigl(A^{-1}\bigr)_p^kA_i^qE_q\bigl(A_j^p\bigr).
  \end{equation}
\end{proposition}

\subsection{联络的存在性}


\section{张量场的协变导数}

我们在本节证明 $TM$ 中的每个联络都自动诱导了 $M$ 上所有张量丛中的联络,
因此给出了一种计算任意类型张量场的协变导数的方法。

\begin{proposition}
  令 $M$ 是光滑流形,$\nabla$ 是 $TM$ 中的联络。那么 $\nabla$
  唯一确定了每个张量丛 $T^{(k,l)}TM$ 中的联络,同样记为 $\nabla$,
  其满足下面的四个条件。
  \begin{enumerate}
    \item 在 $T^{(1,0)}TM=TM$ 中,$\nabla$ 与给定的联络相同。
    \item 在 $T^{(0,0)}TM=M\times \mathbb{R}$ 中,$\nabla$ 给出了通常意义下函数的微分:
    \[
      \nabla_Xf=Xf.  
    \]
    \item $\nabla$ 服从下面的乘积法则:
    \[
      \nabla_X(F\otimes G)=(\nabla_XF)\otimes G+F\otimes(\nabla_XG).  
    \]
    \item $\nabla$ 与缩并操作交换:如果 $\tr$ 是任意一对协变和逆变指标上的迹,
    那么
    \[
      \nabla_X(\tr F)=\tr(\nabla_XF).  
    \]
  \end{enumerate}
  这个联络同时遵循下面的两条额外属性:
  \begin{enumerate}[label=(\alph{*})]
    \item $\nabla$ 相对于余向量场 $\omega$ 和向量场 $Y$ 的自然配对,服从下面的乘积法则:
    \[
      \nabla_X\inn{\omega,Y}=\inn{\nabla_X\omega,Y}+\inn{\omega,\nabla_XY}.  
    \]
    \item 对于所有的 $F\in\Gamma\bigl(T^{(k,l)}TM\bigr)$,光滑 $1$-形式
    $\omega^1,\dots,\omega^k$ 和光滑向量场 $Y_1,\dots,Y_l$,有
    \begin{equation}\label{eq:convariant derivative of tensor field}
      \begin{gathered}
        (\nabla_XF)\bigl(\omega^1,\dots,\omega^k,Y_1,\dots,Y_l\bigr)=
        X\bigl(F\bigl(\omega^1,\dots,\omega^k,Y_1,\dots,Y_l\bigr)\bigr)\\
        -\sum_{i=1}^kF\bigl(\omega^1,\dots,\nabla_X\omega^i\dots,\omega^k,Y_1,\dots,Y_l\bigr)\\
        -\sum_{j=1}^lF\bigl(\omega^1,\dots,\omega^k,Y_1,\dots,\nabla_XY_j,\dots,Y_l\bigr).
      \end{gathered}
    \end{equation}
  \end{enumerate}
\end{proposition}
\begin{proof}
  首先我们证明满足 (1)--(4) 的联络一定满足 (a) 和 (b)。
  对于 (a),注意到 $\inn{\omega,Y}=\omega(Y)=\omega_iY^i$。
  另一方面,对于 $\omega\otimes Y\in \Gamma\bigl(T^{(1,1)}TM\bigr)$,
  其诱导 $C^\infty(M)$-线性映射 $\mathcal{F}:\mathfrak{X}(M)\to \mathfrak{X}(M)$,
  满足任取 $\alpha\in \mathfrak{X}^*(M)$ 有
  $\alpha(\mathcal{F}(X))=(\alpha Y)(\omega X)$,即
  \[
    \mathcal{F}(X)=\omega_jX^j Y^i\partial_i,
  \]
  这表明 $\mathcal{F}$ 在每个点处作为 $T_pM\to T_pM$ 的线性映射,
  其在基 $\partial/\partial x^i|_p$ 下的表示矩阵为
  $(\omega_j Y^i)(p)$,故 $\tr(\omega\otimes Y)=\omega_iY^i$。这表明
  $\inn{\omega,Y}=\tr(\omega\otimes Y)$。那么条件 (1)--(4) 表明
  \begin{align*}
    \nabla_X\inn{\omega,Y}&=\nabla_X\bigl(\tr(\omega\otimes Y)\bigr)\\
    &=\tr\bigl(\nabla_X(\omega\otimes Y)\bigr)\\
    &=\tr\bigl(\nabla_X\omega\otimes Y+\omega\otimes\nabla_XY\bigr)\\
    &=\inn{\nabla_X\omega,Y}+\inn{\omega,\nabla_XY}.
  \end{align*}
  对于 (b),我们可以归纳证明等式:
  \[
    F\bigl(\omega^1,\dots,\omega^k,Y_1,\dots,Y_l\bigr)=
    \underbrace{\tr\circ\cdots\circ\tr}_{k+l}
    \bigl(F\otimes\omega^1\otimes\cdots\otimes\omega^k\otimes Y_1\otimes\cdots\otimes Y_l\bigr),
  \]
  其中每个 $\tr$ 算符作用在 $F$ 的一个上升指标和对应余向量场的下降指标上,
  或者 $F$ 的一个下降指标和对应向量场的上升指标上。以 $F\in \Gamma\bigl(T^{(1,1)}TM\bigr)$
  为例,由于 $F\otimes\omega\otimes Y=F_{j_1}^{i_1}\omega_{j_2} Y^{i_2}\partial_{i_1}\otimes \d x^{j_1}\otimes \d x^{j_2}\otimes \partial_{i_2}$,
  所以
  \[
    \tr(F\otimes\omega\otimes Y)_j^i=
    F^m_{j}\omega_m Y^{i},
  \]
  所以 
  \[
    (\tr\circ\tr)(F\otimes\omega\otimes Y)=F_n^m\omega_m Y^n
    =F(\omega,Y).  
  \]
  然后使用类似的计算即可,例如
  \begin{align*}
    (\nabla_XF)(\omega,Y)&=\tr\circ\tr \bigl(\nabla_XF\otimes\omega\otimes Y\bigr)
    \\
    &=\tr\circ\tr \bigl(\nabla_X(F\otimes\omega\otimes Y)
    -F\otimes\nabla_X(\omega\otimes Y)\bigr)\\
    &=\tr\circ\tr \bigl(\nabla_X(F\otimes\omega\otimes Y)
    -F\otimes\nabla_X\omega\otimes Y-F\otimes\omega\otimes\nabla_XY\bigr)\\
    &=\nabla_X\bigl(F(\omega,Y)\bigr)-F\bigl(\nabla_X\omega,Y\bigr)
    -F\bigl(\omega,\nabla_XY\bigr)\\
    &=X\bigl(F(\omega,Y)\bigr)-F\bigl(\nabla_X\omega,Y\bigr)
    -F\bigl(\omega,\nabla_XY\bigr).
  \end{align*}

  下面我们证明唯一性。假设 $\nabla$ 满足 (1)--(4),那么也满足 (a) 和 (b)。
  对于余向量场 $\omega$,有
  \begin{equation}\label{eq:convariant derivative of covector field}
    (\nabla_X\omega)(Y)=\nabla_X(\omega Y)-\omega(\nabla_XY),
  \end{equation}
  这表明 $T^*M$ 中的联络由 $TM$ 中的联络唯一确定。类似地,(b) 表明
  每个张量场 $F$ 的协变导数都由向量场和余向量场的协变导数唯一确定。

  现在证明存在性。我们首先根据 \eqref{eq:convariant derivative of covector field}
  式定义余向量场的协变导数,然后根据 \eqref{eq:convariant derivative of tensor field} 
  式定义一般张量场的协变导数。首先我们需要验证这样的定义在每个 $\omega^i$
  和 $Y_j$ 上都是 $C^\infty(M)$-线性的,从而确实定义了一个光滑张量场。
  然后我们验证其满足联络的条件。这些都是直接按定义验证即可。
\end{proof}

虽然 \eqref{eq:convariant derivative of covector field} 和 \eqref{eq:convariant derivative of tensor field}
式给出了一般张量场的协变导数的定义,但是在计算中并不实用,因为
计算 $\nabla_XF$ 在一个点处的值需要将所有的参数延拓到某个开集上的
向量场或者余向量场上,然后计算大量的导数。为了在局部坐标中
计算协变导数,下面的公式更加有用。

\begin{proposition}
  令 $M$ 是光滑流形,$\nabla$ 是 $TM$ 中的联络。设 $(E_i)$
  是 $M$ 的一个局部标架,$\bigl(\varepsilon^i\bigr)$ 是其对偶标架,
  $\{\Gamma_{ij}^k\}$ 是 $\nabla$ 相对于这组标架的联络系数。令 $X$
  是光滑向量场,$X^iE_i$ 是 $X$ 的局部表示。
  \begin{enumerate}
    \item 余向量场 $\omega=\omega_i\varepsilon^i$ 的协变导数为
    \[
      \nabla_X\omega=\bigl(X(\omega_k)-X^j\omega_i\Gamma_{jk}^i\bigr)  \varepsilon^k.
    \]
    \item 如果 $F\in\Gamma\bigl(T^{(k,l)}TM\bigr)$ 局部表示为
    \[
      F=F^{i_1\dots i_k}_{j_1\dots j_l}E_{i_1}\otimes
      \cdots\otimes E_{i_k}\otimes\varepsilon^{j_1}\otimes\cdots\otimes\varepsilon^{j_l},
    \]
    那么 $F$ 的协变导数为
    \begin{multline*}
      \nabla_XF=\biggl(X\left(F^{i_1\dots i_k}_{j_1\dots j_l}\right)+
      \sum_{s=1}^kX^m F^{i_1\dots p\dots i_k}_{j_1\dots j_l}\Gamma_{mp}^{i_s}-
      \sum_{s=1}^lX^mF^{i_1\dots i_k}_{j_1\dots p\dots j_l}\Gamma_{mj_s}^p\biggr)\times\\
      E_{i_1}\otimes
      \cdots\otimes E_{i_k}\otimes\varepsilon^{j_1}\otimes\cdots\otimes\varepsilon^{j_l}.
    \end{multline*}
  \end{enumerate}
\end{proposition}

因为 $\nabla_XF$ 在 $X$ 上是 $C^\infty(M)$-线性的,所以 $F$ 在所有
方向上的协变导数可以被编码为一个秩比 $F$ 大 $1$ 的张量场,如下所示。

\begin{proposition}[全协变导数]
  令 $M$ 是光滑流形,$\nabla$ 是 $TM$ 中的联络,对于每个
  $F\in\Gamma\bigl(T^{(k,l)}TM\bigr)$,定义映射
  \[
    \nabla F:\underbrace{\Omega^1(M)\times\cdots\times\Omega^1(M)}_{\text{$k$ copies}}
    \times \underbrace{\mathfrak{X}(M)\times\cdots\times \mathfrak{X}(M)}_{\text{$l+1$ copies}}
    \to C^\infty(M)
  \]
  为
  \begin{equation}
    (\nabla F)\bigl(\omega^1,\dots,\omega^k,Y_1,\dots,Y_l,X\bigr)
    =(\nabla_XF)\bigl(\omega^1,\dots,\omega^k,Y_1,\dots,Y_l\bigr).
  \end{equation}
  这定义了一个光滑 $(k,l+1)$-张量场,被称为\emph{$F$ 的全协变导数}。
\end{proposition}
\begin{proof}
  根据张量表征引理即得。
\end{proof}

当我们在局部标架中书写全协变导数的分量的时候,一个标准记号是
使用分号分隔前面的指标和做微分的指标。例如,如果 $Y=Y^iE_i$
是向量场,那么 $(1,1)$-张量场 $\nabla Y$ 的分量写为 ${Y^i}_{;j}$,
即
\[
  \nabla Y=  {Y^i}_{;j}E_i\otimes \varepsilon^j,
\]
此时
\[
  (\nabla Y)(\omega,X)={Y^i}_{;j} \omega(E_i)\varepsilon^j(X)
  =\omega_i X^j{Y^i}_{;j},  
\]
又因为
\[
  \nabla_XY=\bigl(X(Y^k)+X^iY^j\Gamma_{ij}^k\bigr)  E_k,
\]
所以另一方面有
\[  
  (\nabla Y)(\omega,X)=(\nabla_XY)(\omega)=
  \bigl(X(Y^k)+X^iY^j\Gamma_{ij}^k\bigr)\omega_k,
\]
对比系数即得
\[
  {Y^i}_{;j}=E_jY^i+Y^k\Gamma_{jk}^i.  
\]

\begin{proposition}\label{prop:components of total convariant derivative}
  令 $M$ 是光滑流形,$\nabla$ 是 $TM$ 中的联络,$(E_i)$ 是 $TM$ 的光滑局部标架,
  $\{\Gamma_{ij}^k\}$ 是联络系数。那么 $(k,l)$-张量场 $F$ 的全协变导数的分量
  为
  \[
    F_{j_1\dots j_l;m}^{i_1\dots i_k}  =
    E_m\Bigl(F_{j_1\dots j_l}^{i_1\dots i_k}\Bigr)
    +\sum_{s=1}^k F_{j_1\dots j_l}^{i_1\dots p\dots i_k}\Gamma_{mp}^{i_s}
    -\sum_{s=1}^l F_{j_1\dots p\dots j_l}^{i_1\dots i_k}\Gamma_{m j_s}^p.
  \]
\end{proposition}



\section{沿曲线的向量场和张量场}

现在,我们可以解决一个最初激发联络定义的问题:我们如何
理解向量场沿一条曲线的导数?

令 $M$ 是光滑流形,给定光滑曲线 $\gamma:I\to M$,\emph{沿 $\gamma$ 的向量场}
指的是一个连续映射 $V:I\to TM$ 使得 $V(t)\in T_{\gamma(t)}M$。
令 $\mathfrak{X}(\gamma)$ 是所有沿 $\gamma$ 的光滑向量场的集合。
在逐点向量加法和数乘下,这是一个实向量空间。并且是一个
$C^\infty(I)$-模,使用逐点的乘法定义:
\[
  (fX)(t)=f(t)X(t).  
\]
最简单的沿 $\gamma$ 的光滑向量场的例子是曲线的速度
$\gamma'(t)\in T_{\gamma(t)}M$。

设 $\gamma:I\to M$ 是光滑曲线,$\wtilde V$ 是 $M$ 的包含 $\gamma(I)$
的开集上的光滑向量场,定义 $V:I\to M$ 满足 $V(t)= \wtilde V_{\gamma(t)}$,
也即 $V=\wtilde V\circ \gamma$,此时 $V$ 是沿 $\gamma$ 的光滑向量场。
对于一个沿 $\gamma$ 的光滑向量场 $V$,如果存在 $\gamma(I)$ 的一个邻域上的
光滑向量场 $\wtilde V$ 使得 $V=\wtilde V\circ\gamma$,那么我们说
$V$ 是\emph{可延拓的}。注意到并不是所有向量场都是可延拓的,例如,
如果 $\gamma(t_1)=\gamma(t_2)$ 但是 $\gamma'(t_1)\neq \gamma'(t_2)$,那么
速度 $\gamma'(t)$ 就不是可延拓的。事实上,即使 $\gamma$ 是单射,
其速度也不一定是可延拓的。

更一般地,\emph{沿 $\gamma$ 的张量场} 指的是一个连续映射 $\sigma:I\to T^{(k,l)}TM$
使得 $\sigma(t)\in T^{(k,l)}\bigl(T_{\gamma(t)}M\bigr)$。同样,如果存在
$\gamma(I)$ 的某个邻域上的光滑张量场 $\tilde\sigma$ 使得 $\sigma=\tilde\sigma\circ\gamma$,
那么我们说 $\sigma$ 是\emph{可延拓的}。

\subsection{沿曲线的协变导数}

沿曲线的协变导数给出了一种对联络的解释:即给出了向量场沿曲线求导的方法。

\begin{theorem}[沿曲线的协变导数]\label{thm:convariant derivative along a curve}
  $M$ 是光滑流形,$\nabla$ 是 $TM$ 上的联络。对于光滑曲线 $\gamma:I\to M$,
  联络确定了唯一的算符:
  \[
    D_t:\mathfrak{X}(\gamma)\to \mathfrak{X}(\gamma),  
  \]
  我们称为\emph{沿 $\gamma$ 的协变导数},其满足下面的性质:
  \begin{enumerate}
    \item $\mathbb{R}$-线性:
    \[
      D_t(aV+bW)=aD_tV+bD_tW\quad a,b\in \mathbb{R}.  
    \]
    \item 乘积法则:
    \[
      D_t(fV)=f'V+fD_tV\quad f\in C^\infty(I).  
    \]
    \item 如果 $V\in \mathfrak{X}(\gamma)$ 是可延拓的,那么对于 $V$ 的任意延拓 $\wtilde V$,有
    \[
      D_tV(t)=\nabla_{\gamma'(t)}\wtilde V.  
    \]
  \end{enumerate}
\end{theorem}
\begin{proof}
  首先我们证明唯一性。假设 $D_t$ 是这样的算符,任取 $t_0\in I$。
  与 \autoref{lemma:locality of connection} 的证明类似,可以说明 $D_tV$
  在 $t_0$ 处的值只与 $V$ 在包含 $t_0$ 的任意小区间 $(t_0-\varepsilon,t_0+\varepsilon)$
  上的值有关。

  选取 $M$ 的在 $\gamma(t_0)$ 的某个邻域中给定光滑坐标 $\bigl(x^i\bigr)$,此时我们有
  \[
    V(t)=V^j(t)\partial_j|_{\gamma(t)}.  
  \]
  其中 $t$ 接近 $t_0$,$V^1,\dots,V^n$ 是定义在 $t_0$ 在 $I$ 中的某个邻域上的 
  光滑实值函数。根据 $D_t$ 的性质,又因为 $\partial_j$ 都是可延拓的,那么
  \begin{equation}\label{eq:convariant derivative along a curve}
    \begin{aligned}
      D_tV(t)&=\dot V^j(t)\partial_j|_{\gamma(t)}+V^j(t)\nabla_{\gamma'(t)}\partial_j\\
      &=\Bigl(\dot V^k(t)+\dot\gamma^i(t)V^j(t)\Gamma_{ij}^k(\gamma(t))\Bigr)
      \partial_k|_{\gamma(t)}.
    \end{aligned}
  \end{equation}
  这表明 $D_t$ 一定是唯一的。

  对于存在性,如果 $\gamma(I)$ 被单个坐标卡包含,那么我们可以使用 \eqref{eq:convariant derivative along a curve} 
  定义 $D_tV$,不难验证这满足协变导数的条件。在一般情况下,我们可以用坐标卡覆盖 $\gamma(I)$
  然后在每个坐标卡上定义 $D_tV$,唯一性表明在两个坐标卡的重叠处的定义相同。
\end{proof}

\eqref{eq:convariant derivative along a curve} 提供了一个实用的公式去计算
坐标中这样的协变导数。

现在我们可以证明 $\nabla_vY$ 实际上只与 $Y$ 在沿光滑曲线 $\gamma$ 上的取值有关,
其中某个 $t_0$ 使得 $\gamma(t_0)=p$ 以及 $\gamma'(t_0)=v$。

\begin{proposition}
  $M$ 是光滑流形,$\nabla$ 是 $TM$ 上的联络,令 $p\in M$ 以及 
  $v\in T_pM$。假设 $Y$ 和 $\wtilde Y$ 是两个光滑向量场,它们在
  光滑曲线 $\gamma:I\to M$ 的像集上的取值相同,其中
  $\gamma(t_0)=p$ 以及 $\gamma'(t_0)=v$,那么 $\nabla_vY=\nabla_v\wtilde Y$。
\end{proposition}
\begin{proof}
  我们可以定义沿 $\gamma$ 的光滑向量场 $Z$,令 $Z(t)=Y_{\gamma(t)}=\wtilde Y_{\gamma(t)}$。
  因为 $Y$ 和 $\wtilde Y$ 都是 $Z$ 的延拓,那么根据
  \autoref{thm:convariant derivative along a curve},
  $\nabla_vY$ 和 $\nabla_v\wtilde Y$ 都等于 $D_tZ(t_0)$。
\end{proof}

\section{测地线}

现在我们可以定义曲线的加速度和测地线的概念了。

令 $M$ 是光滑流形,$\nabla$ 是 $TM$ 中的联络。对于每个光滑曲线
$\gamma:I\to M$,我们可以定义\emph{$\gamma$ 的加速度}是
沿 $\gamma$ 的向量场 $D_t\gamma'$。如果光滑曲线 $\gamma$ 的加速度
是零:$D_t\gamma'\equiv 0$,那么我们说 $\gamma$ 是\emph{测地线}。
凭借光滑坐标 $\bigl(x^i\bigr)$,如果我们将 $\gamma$ 的分量写为
$\gamma(t)=\bigl(x^1(t),\dots,x^n(t)\bigr)$,那么根据 \eqref{eq:convariant derivative along a curve}
式,$\gamma$ 是测地线当且仅当其分量满足下面的\emph{测地线方程}:
\begin{equation}\label{eq:geodesic curve equation}
  \ddot{x}^k(t)+\dot x^i(t)\dot x^j(t)\Gamma_{ij}^k(x(t))=0,
\end{equation}
其中我们使用 $x(t)$ 作为 $\bigl(x^1(t),\dots,x^n(t)\bigr)$ 的简写。
这是一个关于实值函数 $x^1,\dots,x^n$ 的二阶常微分方程组。

\begin{theorem}[测地线的存在唯一性]
  令 $M$ 是光滑流形,$\nabla$ 是 $TM$ 中的联络。对于每个 $p\in M$,
  $w\in T_pM$ 和 $t_0\in \mathbb{R}$,都存在一个包含 $t_0$
  的开区间 $I\subseteq \mathbb{R}$ 和测地线 $\gamma:I\to M$ 满足
  $\gamma(t_0)=p$ 和 $\gamma'(t_0)=w$。此外,任意两个这样的测地线在它们的公共
  定义域一定是相等的。
\end{theorem}

一个测地线 $\gamma:I\to M$ 被称为\emph{极大的},如果其不能延拓为一个更大区间
上的测地线。一个\emph{测地线段}指的是定义域是紧区间的测地线。

\begin{corollary}
  令 $M$ 是光滑流形,$\nabla$ 是 $TM$ 中的联络。对于每个 $p\in M$,
  $v\in T_pM$,存在唯一的极大测地线 $\gamma:I\to M$ 使得
  $\gamma(0)=p$ 以及 $\gamma'(0)=v$,且 $I$ 是包含 $0$
  的开区间。
\end{corollary}

这个唯一的满足 $\gamma(0)=p$ 和 $\gamma'(0)=v$ 的极大测地线 $\gamma$
被称为\emph{有起点 $p$ 和初速度 $v$ 的测地线},记为 $\gamma_v$。

\section{平行移动}

令 $M$ 是光滑流形,$\nabla$ 是 $TM$ 中的联络。一个沿光滑曲线 $\gamma$
的光滑向量场 $V$ 如果使得 $D_tV\equiv 0$,那么我们说 $V$ 是
\emph{沿 $\gamma$ 平行的}。测地线可以解释为速度向量场沿自身平行的曲线。

关于沿曲线平行的向量场的一个基本事实是:曲线上任何一点的任意切向量
都可以延拓为一个沿整条曲线平行的向量场。我们首先观察坐标中平行性代表的方程。
给定一个光滑曲线 $\gamma$,设其有局部坐标表示 
$\gamma(t)=\bigl(\gamma^1(t),\dots,\gamma^n(t)\bigr)$,
\eqref{eq:convariant derivative along a curve} 式表明向量场 $V$ 
沿 $\gamma$ 平行当且仅当
\begin{equation}
  \dot V^k(t)=-\dot\gamma^i(t)V^j(t)\Gamma_{ij}^k(\gamma(t)),
  \quad k=1,\dots,n.
\end{equation}

\begin{theorem}[线性 ODE 解的存在性、唯一性和光滑性]
  令 $I\subseteq \mathbb{R}$ 是开子集,对于 $1\leq j,k\leq n$,
  令 $A_j^k:I\to \mathbb{R}$ 是光滑函数。对于每个 $t_0\in I$ 和
  初值向量 $\bigl(c^1,\dots,c^n\bigr)\in \mathbb{R}^n$,线性初值问题
  \begin{equation}
    \begin{aligned}
      \dot V^k(t)&=A_j^k(t)V^j(t),\\
      V^k(t_0)&=c^k,
    \end{aligned}
  \end{equation}
  在 $I$ 上有唯一的光滑解,并且这个解光滑依赖于 $(t,c)\in I\times \mathbb{R}^n$。
\end{theorem}

\begin{theorem}[平行移动的存在性]\label{thm:existence of parallel transport}
  设 $M$ 是光滑流形,$\nabla$ 是 $TM$ 中的联络。给定光滑曲线
  $\gamma:I\to M$,$t_0\in I$ 和向量 $v\in T_{\gamma(t_0)}M$,
  那么存在唯一的沿 $\gamma$ 平行的向量场 $V$ 使得 $V(t_0)=v$。
\end{theorem}

\autoref{thm:existence of parallel transport} 中给出的向量场
被称为\emph{$v$ 沿 $\gamma$ 的平行移动}。对于每个 $t_0,t_1\in I$,
我们定义映射
\begin{equation}
  P_{t_0t_1}^\gamma:T_{\gamma(t_0)}M\to T_{\gamma(t_1)}M,
\end{equation}
称为\emph{平行移动映射},定义为 $P_{t_0t_1}^\gamma(v)=V(t_1)$。
这个映射是线性映射,因为平行性方程是线性的。事实上这是一个同构,
因为 $P_{t_0t_1}^\gamma$ 有逆映射 $P_{t_1t_0}^\gamma$。


\chapter{Levi-Civita 联络}

\section{黎曼流形上的联络}

\subsection{度量联络}

令 $g$ 是光滑流形 $M$ 上的黎曼度量或者伪黎曼度量。对于 $TM$ 中的联络
$\nabla$,任取 $X,Y,Z\in \mathfrak{X}(M)$,如果有
\begin{equation}
  \nabla_X\inn{Y,Z}=\inn{\nabla_XY,Z}+\inn{Y,\nabla_XZ},
\end{equation}
那么我们说 $\nabla$ 是\emph{与 $g$ 相容的},或者说 $\nabla$ 是\emph{度量联络}。

\begin{proposition}[度量联络的特征]\label{prop:metric connection}
  令 $(M,g)$ 是黎曼流形或者伪黎曼流形,$\nabla$ 是 $TM$ 中的联络。那么下面的条件等价:
  \begin{enumerate}
    \item $\nabla$ 与 $g$ 相容:$\nabla_X\inn{Y,Z}=\inn{\nabla_XY,Z}+\inn{Y,\nabla_XZ}$。
    \item $g$ 相对于 $\nabla$ 平行:$\nabla g\equiv 0$。
    \item 在光滑局部标架 $(E_i)$ 中,$\nabla $ 的联络系数满足
    \begin{equation}\label{eq:Delta g equiv 0}
      \Gamma_{ki}^lg_{lj}+\Gamma_{kj}^lg_{il}=E_k(g_{ij}).
    \end{equation}
    \item 如果 $V,W$ 是沿光滑曲线 $\gamma$ 的向量场,那么
    \begin{equation}
      \frac{\d}{\d t}\inn{V,W}=\inn{D_tV,W}+\inn{V,D_tW}.
    \end{equation}
    \item 如果 $V,W$ 是沿光滑曲线 $\gamma$ 平行的向量场,那么 $\inn{V,W}$
    沿 $\gamma$ 是常数。
    \item 给定任意光滑曲线 $\gamma$,每个沿 $\gamma$ 的平行移动映射
    都是线性等距。
    \item 给定任意光滑曲线 $\gamma$,在 $\gamma$ 的某个点处的每组
    正交基都可以延拓为一组沿 $\gamma$ 平行的正交标架。
  \end{enumerate}
\end{proposition}
\begin{proof}
  首先证明 $(1)\Leftrightarrow (2)$。$(0,2)$-张量场 $g$ 的全协变导数
  为
  \[
    (\nabla g)(Y,Z,X)=(\nabla_Xg)(Y,Z)  
    =X\bigl(g(Y,Z)\bigr)-g\bigl(\nabla_XY,Z\bigr)
    -g\bigl(Y,\nabla_XZ\bigr),
  \]
  所以 $\nabla$ 与 $g$ 相容当且仅当 $\nabla g\equiv 0$。

  然后证明 $(2)\Leftrightarrow (3)$。\autoref{prop:components of total convariant derivative}
  表明 $\nabla g$ 在局部标架 $(E_i)$ 中的分量为
  \[
    g_{ij;k}=E_k(g_{ij})-\Gamma_{ki}^lg_{lj}-\Gamma_{kj}^lg_{il}.  
  \]
  所以 $\nabla g\equiv 0$ 当且仅当 \eqref{eq:Delta g equiv 0} 式成立。

  现在我们证明 $(1)\Leftrightarrow (4)$。假设 $(1)$ 成立。令 $V,W$
  是沿光滑曲线 $\gamma:I\to M$ 的光滑向量场。给定 $t_0\in I$,在 $\gamma(t_0)$
  的邻域中选取局部坐标 $(x^i)$,那么有 $V=V^i\partial_i$ 和 $W=W^j\partial_j$,其中 $V^i,W^j:I\to \mathbb{R}$
  是光滑函数。注意这里我们将 $\partial_i$ 和 $\partial_i\circ\gamma$ 等同。那么
  \begin{align*}
    \frac{\d}{\d t}\inn{V,W}&=\frac{\d}{\d t}\bigl(V^iW^j\inn{\partial_i,\partial_j}\bigr)\\
    &=\bigl(\dot V^iW^j+V^i\dot W^j\bigr)\inn{\partial_i,\partial_j}+
    V^iW^j \nabla_{\gamma'(t)}\inn{\partial_i,\partial_j}\\
    &=\bigl(\dot V^iW^j+V^i\dot W^j\bigr)\inn{\partial_i,\partial_j}+V^iW^j
    \bigl(\inn{\nabla_{\gamma'(t)}\partial_i,\partial_j}+\inn{\partial_i,\nabla_{\gamma'(t)}\partial_j}\bigr)
    \\
    &=\inn{\dot V^i\partial_i,W}+\inn{V,\dot W^j\partial_j}+
    \inn{V^i\nabla_{\gamma'(t)}\partial_i,W}+\inn{V,W^j\nabla_{\gamma'(t)}\partial_j}\\
    &=\inn{D_tV,W}+\inn{V,D_tW}.
  \end{align*}
  反之,如果 $(4)$ 成立,根据 $D_tV=\nabla_{\gamma'(t)}\wtilde V$ 即得 (1)。

  接下来我们证明 $(4)\Rightarrow (5)\Rightarrow(6)\Rightarrow(7)\Rightarrow(4)$。
  假设 $(4)$ 成立。如果 $V,W$ 是沿 $\gamma$ 平行的,那么
  \[
    \frac{\d}{\d t}\inn{V,W}=\inn{D_tV,W}+\inn{V,D_tW}=0,
  \]
  所以 $\inn{V,W}$ 是沿 $\gamma$ 的常值函数。

  假设 $(5)$ 成立。令 $v_0,w_0\in T_{\gamma(t_0)}M$,$V,W$ 是它们沿
  $\gamma$ 的平行移动,即 $V(t_0)=v_0$,$W(t_0)=w_0$,$P_{t_0t_1}^\gamma(v_0)=V(t_1)$
  以及 $P_{t_0t_1}^\gamma(w_0)=W(t_1)$。因为 $\inn{V,W}$ 沿 $\gamma$ 是常数,所以
  \[
    \inn{P_{t_0t_1}^\gamma(v_0),P_{t_0t_1}^\gamma(w_0)}  
    =\inn{V(t_1),W(t_1)}=\inn{V(t_0),W(t_0)}=\inn{v_0,w_0},
  \]
  所以 $P_{t_0t_1}^\gamma$ 是线性等距。

  假设 $(6)$ 成立。设 $(b_i)$ 是 $T_{\gamma(t_0)}M$ 的一组正交基。
  对于每个 $b_i$,记 $E_i$ 是 $b_i$ 沿 $\gamma$ 的平行移动,由于
  平行移动映射是线性等距,所以 $(E_i)$ 在 $\gamma$ 的每个点处
  都是正交基。

  最后假设 $(7)$ 成立。令 $(E_i)$ 是沿 $\gamma$ 平行的正交标架。
  给定沿 $\gamma$ 的两个向量场 $V$ 和 $W$,我们可以将其表示为
  $V=V^iE_i$ 以及 $W=W^jE_j$。$(E_i)$ 是正交标架表明沿 $\gamma$
  时度量系数 $g_{ij}=\inn{E_i,E_j}$ 是常数 $\pm 1$ 或者 $0$。
  $E_i$ 的平行性表明 $D_tV=\dot V^iE_i$ 以及 $D_tW=\dot W^iE_i$,
  所以 
  \begin{align*}
    \frac{\d}{\d t}\inn{V,W}&=\frac{\d}{\d t}V^iW^j\inn{E_i,E_j}\\
    &=\bigl(\dot V^iW^j+V^i\dot W^j\bigr)\inn{E_i,E_j}\\
    &=\inn{D_tV,W}+\inn{V,D_tW}.\qedhere
  \end{align*}
\end{proof}
 
\begin{corollary}
  设 $(M,g)$ 是黎曼流形,$\nabla$ 是 $M$ 上的度量联络,$\gamma:I\to M$
  是光滑曲线。
  \begin{enumerate}
    \item $|\gamma'(t)|$ 是常数当且仅当对于所有的 $t\in I$ 有 $D_t\gamma'(t)$ 正交于 $\gamma'(t)$。
    \item 如果 $\gamma$ 是测地线,那么 $|\gamma'(t)|$ 是常数。
  \end{enumerate}
\end{corollary}
\begin{proof}
  (1) $|\gamma'(t)|$ 是常数当且仅当 $\d/\d t\inn{\gamma'(t),\gamma'(t)}\equiv 0$,
  当且仅当
  \[
    \inn{D_t\gamma'(t),\gamma'(t)} \equiv 0,
  \]
  即 $D_t\gamma'(t)$ 正交于 $\gamma'(t)$。

  (2) 若 $\gamma$ 是测地线,那么 $D_t\gamma'\equiv 0$,此时 $D_t\gamma'(t)$
  始终正交于 $\gamma'(t)$,即 $|\gamma'(t)|$ 是常数。
\end{proof}

\subsection{对称联络}

我们说 $TM$ 中的联络 $\nabla$ 是\emph{对称的},如果
\[
  \nabla_XY-\nabla_YX\equiv [X,Y]\quad \forall X,Y\in \mathfrak{X}(M).  
\]

对称条件也可以使用\emph{挠率张量}表示。挠率张量是一个 $(1,2)$-张量场
$\tau:\mathfrak{X}(M)\times \mathfrak{X}(M)\to \mathfrak{X}(M)$,定义为
\[
  \tau(X,Y)=\nabla_XY-\nabla_YX-[X,Y].  
\]
所以 $\nabla$ 是对称的当且仅当挠率张量为零。

\begin{theorem}[黎曼几何基本定理]
  令 $(M,g)$ 是黎曼流形或者伪黎曼流形,那么存在唯一的 $TM$ 中的联络
  $\nabla$ 使得其与 $g$ 相容并且对称,这个联络被称为\emph{$g$ 的 Levi-Civita 联络}
  (当 $g$ 是正定的时候,被称为\emph{黎曼联络})。
\end{theorem}
\begin{proof}
  我们首先通过推导 $\nabla$ 的公式证明唯一性。假设 $\nabla$ 是这样的联络
  以及 $X,Y,Z\in \mathfrak{X}(M)$。根据相容性,我们有
  \begin{align*}
    X\inn{Y,Z}&=\inn{\nabla_XY,Z}+\inn{Y,\nabla_XZ},\\
    Y\inn{Z,X}&=\inn{\nabla_YZ,X}+\inn{Z,\nabla_YX},\\
    Z\inn{X,Y}&=\inn{\nabla_ZX,Y}+\inn{X,\nabla_ZY}.
  \end{align*}
  再根据对称性条件,有
  \begin{align*}
    X\inn{Y,Z}&=\inn{\nabla_XY,Z}+\inn{Y,\nabla_ZX}+\inn{Y,[X,Z]},\\
    Y\inn{Z,X}&=\inn{\nabla_YZ,X}+\inn{Z,\nabla_XY}+\inn{Z,[Y,X]},\\
    Z\inn{X,Y}&=\inn{\nabla_ZX,Y}+\inn{X,\nabla_YZ}+\inn{X,[Z,Y]}.
  \end{align*}
  把前两个式子相加减去第三个式子,得到
  \begin{multline*}
    X\inn{Y,Z}+Y\inn{Z,X}-Z\inn{X,Y}=\\
    2\inn{\nabla_XY,Z}+\inn{Y,[X,Z]}+\inn{Z,[Y,X]}-
    \inn{X,[Z,Y]},
  \end{multline*}
  这表明
  \begin{multline}\label{eq:Levi-Civita connection}
    \inn{\nabla_XY,Z}=\frac{1}{2}\Bigl(X\inn{Y,Z}+Y\inn{Z,X}-Z\inn{X,Y}\\
    -\inn{Y,[X,Z]}-\inn{Z,[Y,X]}+
    \inn{X,[Z,Y]}\Bigr).
  \end{multline}

  现在假设 $\nabla^1$ 和 $\nabla^2$ 都是与 $g$ 相容的对称的联络。由于
  \eqref{eq:Levi-Civita connection} 式右端不依赖于联络,所以
  $\inn{\nabla_X^1Y-\nabla_X^2Y,Z}=0$,这表明 $\nabla_X^1Y=\nabla_X^2Y$,
  所以 $\nabla^1=\nabla^2$。

  下面说明存在性。我们使用 \eqref{eq:Levi-Civita connection} 式或者其坐标版本。
  只需要说明这样的联络在每个坐标卡中存在,然后唯一性保证它们在重合的坐标卡中
  是一致的。

  令 $(x^i)$ 是任意光滑坐标,将 \eqref{eq:Levi-Civita connection} 式写为坐标
  版本,并注意到 $\bigl[\partial_i,\partial_j\bigr]=0$,所以
  \begin{equation}\label{eq:coordinate version of Levi-Civita connection}
    \inn{\nabla_{\partial_i}\partial_j,\partial_l}=
    \frac{1}{2}\Bigl(\partial_i\inn{\partial_j,\partial_l}
    +\partial_j\inn{\partial_l,\partial_i}-\partial_l\inn{\partial_i,\partial_j}\Bigr).  
  \end{equation}
  由于
  \[
    g_{ij}=\inn{\partial_i,\partial_j}, \qquad \nabla_{\partial_i}\partial_j=\Gamma_{ij}^m  \partial_m,
  \]
  代入 \eqref{eq:coordinate version of Levi-Civita connection} 式,得到
  \begin{equation}
    \Gamma_{ij}^m g_{ml}=\frac{1}{2}\bigl(\partial_i g_{jl}+\partial_j g_{li}-\partial_l g_{ij}\bigr).
  \end{equation}
  最后将等式两边乘以逆矩阵 $g^{kl}$,注意到 $g_{ml}g^{kl}=\delta^k_m$,所以
  \begin{equation}
    \Gamma_{ij}^k=\frac{1}{2}g^{kl}\bigl(\partial_i g_{jl}+\partial_j g_{li}-\partial_l g_{ij}\bigr).
  \end{equation}

  注意到 $\Gamma_{ij}^k=\Gamma_{ji}^k$,所以这是一个对称联络。接下来说明
  $\nabla$ 与 $g$ 相容即可。我们有
  \begin{align*}
    \Gamma_{ki}^lg_{lj}+\Gamma_{kj}^lg_{il}&=\frac{1}{2}
    \bigl(\partial_kg_{ij}+\partial_ig_{kj}-\partial_jg_{ki}\bigr)
    +\frac{1}{2}
    \bigl(\partial_kg_{ij}+\partial_jg_{ki}-\partial_ig_{kj}\bigr)\\
    &=\partial_k g_{ij}.
  \end{align*}
  根据 \autoref{prop:metric connection},所以 $\nabla$ 是度量联络。
\end{proof}

上述证明的过程给出了 Levi-Civita 联络的明确公式。

\begin{corollary}[Levi-Civita 联络公式]
  令 $(M,g)$ 是黎曼流形或者伪黎曼流形,$\nabla$ 是 Levi-Civita 联络。
  \begin{enumerate}
    \item \emph{(Koszul 公式)} 如果 $X,Y,Z$ 是光滑张量场,那么
    \begin{multline}
      \inn{\nabla_XY,Z}=\frac{1}{2}\Bigl(X\inn{Y,Z}+Y\inn{Z,X}-Z\inn{X,Y}\\
      -\inn{Y,[X,Z]}-\inn{Z,[Y,X]}+
      \inn{X,[Z,Y]}\Bigr).
    \end{multline}
    \item \emph{(坐标表示)} 在任意光滑坐标卡中,Levi-Civita 联络的系数
    可以表示为
    \begin{equation}
      \Gamma_{ij}^k=\frac{1}{2}g^{kl}\bigl(\partial_i g_{jl}+\partial_j g_{li}-\partial_l g_{ij}\bigr).
    \end{equation}
    \item \emph{(局部标架表示)} 令 $(E_i)$ 是开集 $U\subseteq M$ 上的光滑局部标架,
    $c_{ij}^k:U\to \mathbb{R}$ 是光滑函数,定义为
    \begin{equation}
      [E_i,E_j]=c_{ij}^kE_k.
    \end{equation}
    那么 Levi-Civita 联络在这个标架中的系数为
    \begin{equation}
      \Gamma_{ij}^k=\frac{1}{2}g^{kl}\bigl(E_ig_{jl}+E_jg_{il}-E_lg_{ij}-g_{jm}c_{il}^m-g_{lm}c_{ji}^m+g_{im}c_{lj}^m\bigr).
    \end{equation}
    \item \emph{(局部正交标架表示)} 如果 $g$ 是黎曼度量,$(E_i)$ 是光滑正交局部标架,
    $c_{ij}^k$ 的定义不变,那么
    \begin{equation}
      \Gamma_{ij}^k=\frac{1}{2}\bigl(c_{ij}^k-c_{ik}^j-c_{jk}^i\bigr).
    \end{equation}
  \end{enumerate}
\end{corollary}

\section{指数映射}

在本节中,我们令 $(M,g)$ 是一个黎曼流形或者伪黎曼流形,配备 Levi-Civita
联络。我们已经知道每个点 $p\in M$ 和初速度 $v\in T_pM$ 都确定了唯一的
极大测地线 $\gamma_v$。为了更深入的研究测地线,我们需要研究它们的集体行为,
尤其是当我们改变初始点和初速度的时候测地线会如何改变。

\begin{lemma}[缩放引理]
  对于每个 $p\in M$,$v\in T_pM$ 和 $c,t\in \mathbb{R}$,当下面的式子两端都有
  定义的时候,有
  \begin{equation}
    \gamma_{cv}(t)=\gamma_v(ct).
  \end{equation}
\end{lemma}
\begin{proof}
  
\end{proof}

注意到 $v\mapsto \gamma_v$ 定义了从 $TM$ 到测地线集合的一个映射。
更重要的是,根据缩放引理,这允许我们定义从 $TM$ 的一个子集到 $M$ 的映射,
其将 $T_pM$ 的每条过原点的直线送到一个测地线。

定义子集 $\mathcal{E}\subseteq TM$ 为
\[
  \mathcal{E}=\bigl\{v\in TM\bigm| \text{$\gamma_v$ 定义在包含 $[0,1]$ 的某个开区间上}\bigr\} ,
\]
被称为\emph{指数映射的定义域}。定义\emph{指数映射} $\exp:\mathcal{E}\to M$ 为
\[
  \exp(v)=\gamma_v(1).  
\]
对于每个 $p\in M$,定义 \emph{$p$ 处的限制指数映射} $\exp_p$ 为 $\exp$
在集合 $\mathcal{E}_p=\mathcal{E}\cap T_pM$ 上的限制。






\end{document}
