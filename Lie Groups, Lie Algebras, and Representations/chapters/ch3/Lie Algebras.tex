\chapter{李代数}

\section{定义和初步例子}

\begin{definition}
  一个\emph{有限维实或者复李代数}指的是一个有限维的实或者复向量空间 $\lie g$,
  配备一个映射 $[\cdot,\cdot]:\lie g\times \lie g\to\lie g$,
  满足:
  \begin{enumerate}
    \item $[\cdot,\cdot]$ 是双线性的。
    \item $[\cdot,\cdot]$ 是反对称的:对于任意 $X,Y\in\lie g$ 有 $[X,Y]=-[Y,X]$。
    \item Jacobi 恒等式:对于任意 $X,Y,Z\in \lie g$ 有 
    \[
      [X,[Y,Z]]+[Y,[Z,X]]+[Z,[X,Y]]=0.
    \]
  \end{enumerate}
  若 $[X,Y]=0$,那么我们说 $X,Y$ 是\emph{可交换的}。如果对于所有 $X,Y\in\lie g$
  都有 $[X,Y]=0$,那么我们说 $\lie g$ 是\emph{可交换的}。
\end{definition}

$[\cdot,\cdot]$ 通常被称为 $\lie g$ 上的李括号。注意到反对称性表明
$[X,X]=0$。李括号运算通常不满足结合律,然而 Jacobi 恒等式可以被视为结合律
的替代方案。

\begin{example}
  令 $\lie g= \mathbb{R}^3$,$[\cdot,\cdot]:\mathbb{R}^3\times \mathbb{R}^3\to \mathbb{R}^3$
  定义为
  \[
    [x,y]=x\times y,
  \]
  其中 $x\times y$ 是向量叉乘。那么 $\lie g$ 是一个李代数。
\end{example}
\begin{proof}
  双线性性和反对称性是显然的。根据双线性性,只需要对基向量 $e_1,e_2,e_3$
  验证 Jacobi 恒等式即可。如果 $j,k,l$ 互不相同,那么 $e_j,e_k,e_l$ 
  中任意两个的叉乘等于第三个或者第三个的相反方向,所以 Jacobi 恒等式中每一项
  都是 $0$。于是只需要验证 $j,k,l$ 中有两个相同的情况即可,通过重新排序,
  只需要验证
  \[
    [e_j,[e_j,e_k]]+[e_j,[e_k,e_j]]+[e_k,[e_j,e_j]]=0,
  \]
  上式的前两项相反,第三项为零,故叉乘满足 Jacobi 恒等式。
\end{proof}

\begin{example}\label{exa:lie algebra of associative algebra}
  令 $\mathcal A$ 是结合代数,$\lie g$ 是 $\mathcal A$ 的一个子空间,使得
  任意的 $X,Y\in\lie g$ 有 $XY-YX\in\lie g$。那么 $\lie g$ 是一个李代数,有李括号
  \[
    [X,Y]=XY-YX.
  \]
\end{example}
\begin{proof}
  双线性性和反对称性是显然的。对于 Jacobi 恒等式,每个双层李括号会产生 4 项,
  所以总共有 12 项,即
  \[
    [X,[Y,Z]]=[X,YZ-ZY]=XYZ-XZY-YZX+ZYX,
  \]
  对 $X,Y,Z$ 进行轮换,那么正项负项刚好抵消,故这是一个李代数。
\end{proof}

如果我们仔细观察 Jacobi 恒等式的证明,我们会发现 $XYZ$ 实际上以两种方式出现,
一种是 $X(YZ)$,一种是 $(XY)Z$。所以代数 $\mathcal A$ 的结合性是重要的。
对于任意李代数,Jacobi 恒等式意味着李括号的行为\emph{就像}在某个结合代数
中的 $XY-YX$ 一样,即使这个李括号本身不是这样定义的(比如叉乘)。
实际上,可以证明每个李代数 $\lie g$ 都可以嵌入到一个结合代数 $\mathcal A$
中,使得其李括号变成 $XY-YX$。

\begin{example}
  令 $\lie{sl}(n,\mathbb{C})$ 是所有满足 $\tr X=0$ 的 $X\in M_n(\mathbb{C})$
  构成的空间。那么 $\lie{sl}(n,\mathbb{C})$ 是李代数,有李括号
  $[X,Y]=XY-YX$。
\end{example}
\begin{proof}
  我们有
  \[
    \tr(XY-YX)=\tr(XY)-\tr(YX)=0,
  \]
  所以可以应用 \autoref{exa:lie algebra of associative algebra}。
\end{proof}

\begin{definition}
  实或者复李代数 $\lie g$ 的一个\emph{子代数}指的是一个子空间 $\lie h$
  使得任取 $H_1,H_2\in\lie h$ 有 $[H_1,H_2]\in\lie h$。如果 $\lie g$
  是复李代数,$\lie h$ 是 $\lie g$ 的实子空间并且对李括号封闭,那么
  $\lie h$ 被称为 $\lie g$ 的\emph{实子代数}。

  李代数 $\lie g$ 的一个子代数 $\lie h$ 被称为 $\lie g$ 中的\emph{理想},
  如果对于任意 $H\in \lie h,X\in\lie g$ 有 $[X,H]\in\lie h$。

  李代数 $\lie g$ 的\emph{中心}指的是一些 $X\in\lie g$ 的集合,对于每个
  $X$,其使得任取 $Y\in\lie g$,有 $[X,Y]=0$。
\end{definition}

\begin{definition}
  如果 $\lie g,\lie h$ 是李代数,线性映射 $\phi:\lie g\to\lie h$ 
  满足 $\phi([X,Y])=[\phi(X),\phi(Y)]$,那么 $\phi$ 被称为\emph{李代数同态}。
  此外,如果 $\phi$ 是双射,那么 $\phi$ 被称为\emph{李代数同构}。
\end{definition}

\begin{definition}
  如果 $\lie g$ 是李代数,$X\in\lie g$,定义线性映射 $\ad_X:\lie g\to\lie g$
  为 
  \[
    \ad_X(Y)=[X,Y].
  \]
  映射 $X\mapsto \ad_X$ 被称为\emph{伴随映射}或者\emph{伴随表示}。
\end{definition}

虽然 $\ad_X(Y)$ 就是 $[X,Y]$,但是 $\ad$ 的记号是有方便的。例如,我们可以把
\[ 
[X,[X,[X,[X,Y]]]]
\]  
写为 $(\ad_X)^4(Y)$。此外,映射 $X\mapsto \ad_X$ 可以视为 $\lie g\to\End(\lie g)$
的映射。Jacobi 恒等式等价于 $\ad_X$ 是李括号的\emph{导子}:
\begin{equation}\label{eq:Jacobi identity of ad}
  \ad_X([Y,Z])=[\ad_X(Y),Z]+[Y,\ad_X(Z)].
\end{equation}

\begin{proposition}
  如果 $\lie g$ 是李代数,那么
  \[
    \ad_{[X,Y]}=\ad_X\ad_Y-\ad_Y\ad_X=[\ad_X,\ad_Y],
  \]
  也就是说 $\ad:\lie g\to\End(\lie g)$ 是李代数同态。
\end{proposition}
\begin{proof}
  注意到
  \[
    \ad_{[X,Y]}(Z)=[[X,Y],Z],
  \]
  并且
  \[
    [\ad_X,\ad_Y](Z)=[X,[Y,Z]]-[Y,[X,Z]],
  \]
  所以上式等价于 Jacobi 恒等式。
\end{proof}

\begin{definition}
  如果 $\lie g_1,\lie g_2$ 是李代数,那么直和 $\lie g_1\oplus\lie g_2$
  也是李代数,配备李括号
  \[
    [(X_1,X_2),(Y_1,Y_2)]=([X_1,Y_1],[X_2,Y_2]).
  \]
  如果 $\lie g$ 是李代数,$\lie g_1,\lie g_2$ 是两个子代数,
  作为向量空间有 $\lie g=\lie g_1\oplus\lie g_2$ 并且对于 $X_1\in\lie g_1,X_2\in\lie g_2$
  有 $[X_1,X_2]=0$,那么我们说 $\lie g$ 分解为 $\lie g_1$ 和 $\lie g_2$
  的直和。
\end{definition}

\begin{definition}
  令 $\lie g$ 是有限维实或者复李代数,$X_1,\dots,X_N$ 是 $\lie g$ 的一组基,
  那么有唯一的常数 $c_{jkl}$ 使得 
  \[
    [X_j,X_k]=\sum_{l=1}^N c_{jkl}X_l,
  \]
  $c_{jkl}$ 被称为 $\lie g$ 的\emph{结构常数}。
\end{definition}

虽然我们不会经常遇到结构常数,但是在物理课程中会经常使用。结构常数满足下面两个恒等式:
对于 $j,k,l,m$ 有 
\begin{align*}
  c_{jkl}+c_{kjl}&=0,\\
  \sum_n (c_{jkn}c_{nlm}+c_{kln}c_{njm}+c_{ljn}c_{nkm})&=0,
\end{align*}
第一个式子来源于反对称性,第二个式子来源于 Jacobi 恒等式。

\section{单、可解和幂零的李代数}

\begin{definition}
  一个李代数 $\lie g$ 被称为\emph{不可约的},如果 $\lie g$ 中的理想只有
  $\lie g$ 和 $\{0\}$。$\lie g$ 被称为\emph{单的},如果 $\lie g$
  是不可约的且 $\dim \lie g\geq 2$。
\end{definition}

一维的李代数一定是不可约的,因为它没有非平凡的子空间,所以没有非平凡的子代数,
进而没有非平凡的理想。但是,根据定义,一维的李代数不被认为是单的!

此外,还可以注意到一维李代数 $\lie g$ 一定是可交换的,因为对于任意
$X\in\lie g$ 和标量 $a,b$ 都有 $[aX,bX]=ab[X,X]=0$。另一方面,如果 $\lie g$
是可交换的,那么 $\lie g$ 的任意子空间都是理想,所以对于可交换的李代数而言,
只有一维的情况才是不可约的。因此,单李代数的等价定义是其\emph{不可约且不交换}。

显然,这些概念在群论中有对应的类比。其中子群类比于子代数,正规子群类比于理想。
(例如,李代数同态的核总是是一个理想,群同态的核总是为正规子群)。
群论中没有非平凡正规子群的群被称为单群,李代数中没有非平凡理想的李代数被称为
单李代数。

\begin{proposition}
  李代数 $\lie{sl}(2,\mathbb{C})$ 是单的。
\end{proposition}
\begin{proof}
  我们使用下列 $\lie{sl}(2,\mathbb{C})$ 的基:
  \[
    X=\begin{pmatrix}
      0 & 1 \\ 0 & 0
    \end{pmatrix},\quad
    Y=\begin{pmatrix}
      0 & 0 \\ 1 & 0
    \end{pmatrix},\quad
    H=\begin{pmatrix}
      1 & 0 \\ 0 & -1
    \end{pmatrix}.
  \]
  计算可知它们满足 $[X,Y]=H,[H,X]=2X,[H,Y]=-2Y$。设 $\lie h$
  是 $\lie{sl}(2,\mathbb{C})$ 中的理想并且 $\lie h$ 包含元素 $Z=aX+bH+cY$,
  其中 $a,b,c\in \mathbb{C}$ 是不全为零的复数。
  首先假设 $c\neq 0$,那么
  \[
    [X,[X,Z]]=[X,-2bX+cH]=-2cX
  \]
  是 $X$ 的非零倍数。$\lie h$ 是理想表明 $X\in\lie h$。另一方面,
  有 $[Y,X]=-H$ 以及 $[Y,[Y,X]]=2Y$,所以 $Y,H\in \lie h$。
  这表明此时 $\lie h=\lie{sl}(2,\mathbb{C})$。

  现在假设 $c=0,b\neq 0$。那么 $[X,Z]=-2bX$ 表明 $X\in\lie h$,
  然后同样可得 $\lie h=\lie{sl}(2,\mathbb{C})$。
  最后,如果 $c=b=0$ 但是 $a\neq 0$,那么 $X=Z/a\in\lie h$,仍然得到 
  $\lie h=\lie{sl}(2,\mathbb{C})$。这就表明 $\lie{sl}(2,\mathbb{C})$
  是单李代数。
\end{proof}

\begin{definition}
  如果 $\lie g$ 是李代数,那么 $\lie g$ 中的\emph{换位子理想} $[\lie g,\lie g]$
  定义为所有换位子的线性组合张成的空间,即 $Z\in[\lie g,\lie g]$ 当且仅当
  \[
    Z=c_1[X_1,Y_1]+\cdots+c_m[X_m,Y_m].
  \]
\end{definition}

对于任意 $X,Y\in\lie g$,换位子 $[X,Y]\in[\lie g,\lie g]$,这表明
$[\lie g,\lie g]$ 确实是一个理想。

\begin{definition}
  对于李代数 $\lie g$,我们定义一个子代数序列 $\lie g_0,\lie g_1,\lie g_2,\dots$
  为:$\lie g_0=\lie g$,$\lie g_1=[\lie g_0,\lie g_0]$,
  $\lie g_2=[\lie g_1,\lie g_1]$,等等。这些子代数被称为 $\lie g$ 的\emph{导出列}。
  如果对于某个 $j$ 使得 $\lie g_j=\{0\}$,那么我们说 $\lie g$ 是\emph{可解的}。
\end{definition}

利用 Jacobi 恒等式不难证明每个 $\lie g_j$ 都是 $\lie g$ 的理想,例如对于
$[X,Y]\in\lie g_2$,其中 $X,Y\in\lie g_1$,那么对于任意 $Z\in \lie g$,有
\[
  [Z,[X,Y]]=-[X,[Y,Z]]-[Y,[Z,X]]\in\lie g_2.
\]

\begin{definition}
  对于任意李代数 $\lie g$,定义理想序列 $\lie g^j$ 为:
  $\lie g^0=\lie g$,$\lie g^{j+1}$ 为所有的形如 $[X,Y]$ 的换位子的线性组合
  构成的空间,其中 $X\in\lie g$ 以及 $Y\in\lie g^j$。这些子代数被称为
  $\lie g$ 的\emph{上中心列}。如果对于某个 $j$ 有 $\lie g^j=\{0\}$,那么
  我们说 $\lie g$ 是\emph{幂零的}。
\end{definition}

等价地说,$\lie g^j$ 由所有的 $j$-重换位子张成:
\[
  [X_1,[X_2,[X_3,\dots,[X_j,X_{j+1}]\dots]]].
\]
注意到 $j$-重换位子也是 $(j-1)$-重换位子,所以 $\lie g^{j-1}\supseteq \lie g^j$。
对于任意 $X\in\lie g$ 和 $Y\in\lie g^j$,我们有 $[X,Y]\in \lie g^{j+1}\subseteq\lie g^j$,
所以 $\lie g^j$ 是 $\lie g$ 的理想。此外,显然有 $\lie g_j\subseteq \lie g^j$,
因此幂零李代数都是可解的。

\begin{proposition}
  如果 $\lie g\subseteq M_3(\mathbb{R})$ 是 $3\times 3$ 上三角矩阵并且对角线为零。
  那么 $\lie g$ 满足 \autoref{exa:lie algebra of associative algebra},
  并且是一个幂零李代数。
\end{proposition}
\begin{proof}
  显然 $\lie g$ 是李代数。我们选取基
  \[
    X=\begin{pmatrix}
      0 & 1 & 0 \\
      0 & 0 & 0 \\
      0 & 0 & 0 
    \end{pmatrix},\quad 
    Y=\begin{pmatrix}
      0 & 0 & 1 \\
      0 & 0 & 0 \\
      0 & 0 & 0 
    \end{pmatrix},\quad
    Z=\begin{pmatrix}
      0 & 0 & 0 \\
      0 & 0 & 1 \\
      0 & 0 & 0 
    \end{pmatrix}.
  \]
  直接计算得 $[X,Y]=Z$ 以及 $[X,Z]=[Y,Z]=0$。
  故 $[\lie g,\lie g]$ 由 $Z$ 张成,进而 $[\lie g,[\lie g,\lie g]]=0$,
  所以 $\lie g$ 是幂零的。
\end{proof}

\begin{proposition}
  如果 $\lie g\subseteq  M_2(\mathbb{C})$ 是形如
  \[
    \begin{pmatrix}
      a & b \\ 0 & c
    \end{pmatrix}\quad a,b,c\in \mathbb{C}
  \]
  的 $2\times 2$ 矩阵,那么 $\lie g$ 满足 \autoref{exa:lie algebra of associative algebra},
  并且是可解但不幂零的李代数。
\end{proposition}
\begin{proof}
  直接计算得 
  \[
    \left[\begin{pmatrix}
      a & b \\ 0 & c
    \end{pmatrix},\begin{pmatrix}
      d & e \\ 0 & f
    \end{pmatrix}\right]=\begin{pmatrix}
      0 & h \\ 0 & 0 
    \end{pmatrix},
  \]
  其中 $h=ae+bf-bd-ce$,这表明 $\lie g$ 是一个子代数。此外,
  还表明换位子理想 $[\lie g,\lie g]$ 是一维的,所以是可交换的。
  因此 $\lie g_2=0$,故 $\lie g$ 是可解的。另一方面,考虑
  \[
    H=\begin{pmatrix}
      1 & 0 \\ 0 & -1
    \end{pmatrix},\quad
    X=\begin{pmatrix}
      0 & 1 \\ 0 & 0
    \end{pmatrix}.
  \]
  我们有 $[H,X]=2X$,所以
  \[
    [H,[H,[H,\dots,[H,X]\dots]]]
  \]
  永远是 $X$ 的非零倍数,所以对于任意 $j$ 都有 $\lie g^j\neq 0$,故 $\lie g$ 不是幂零的。
\end{proof}

\section{矩阵李群的李代数}

\begin{definition}
  令 $G$ 是一个矩阵李群。$G$ 的李代数 $\lie g$ 定义为所有矩阵 $X$ 的集合,
  其中 $X$ 使得对于任意实数 $t$,指数 $e^{tX}\in G$。
\end{definition}

熟悉流形理论的读者可以发现,这实际上就是再说 $G$ 在单位元处的切空间,因为
$\gamma(t)=e^{tX}$ 是以单位元为起点的切向量为 $X$ 的光滑曲线。

\begin{proposition}
  令 $G$ 是矩阵李群,$X\in\lie g$,那么 $e^X$ 是 $G$ 的单位分支 $G_0$
  中的元素。
\end{proposition}
\begin{proof}
  根据定义,$e^{tX}$ 就是连接单位元和 $e^X$ 的道路。
\end{proof}

\begin{theorem}\label{thm:lie algebra of matrix group}
  令 $G$ 是矩阵李群,有李代数 $\lie g$。如果 $X,Y\in\lie g$,那么
  \begin{enumerate}
    \item 对于任意 $A\in G$ 有 $AXA^{-1}\in\lie g$。
    \item 对于实数 $s$ 有 $sX\in\lie g$。
    \item $X+Y\in\lie g$。
    \item $XY-YX\in\lie g$。
  \end{enumerate}
\end{theorem}
\begin{proof}
  (1) 对于任意的实数 $t$,我们有
  \[
    e^{t(AXA^{-1})}=Ae^{tX}A^{-1}\in G,
  \]
  所以 $AXA^{-1}\in\lie g$。

  (2) 任取实数 $t$,有 $
    e^{t(sX)}=e^{(ts)X}\in G
  $,所以 $sX\in\lie g$。

  (3) 任取实数 $t$,利用李乘积公式,有
  \[
    e^{t(X+Y)}=\lim_{m\to\infty}\left(e^{t\frac{X}{m}}e^{t\frac{Y}{m}}\right)^m,
  \]
  由于 $e^{tX/m}e^{tY/m}\in G$,所以右端是 $G$ 中点列的极限,由于 $G$
  是闭集,所以 $e^{t(X+Y)}\in G$,所以 $X+Y\in\lie g $。

  (4) 我们有
  \[
    \frac{d}{dt}\bigg|_{t=0}\left(e^{tX}Ye^{-tX}\right)=
    (XY)e^0+(e^0Y)(-X)=XY-YX,
  \]
  由 (1),$e^{tX}Ye^{-tX}\in\lie g$。由 (2) 和 (3),$\lie g$ 是
  $M_n(\mathbb C)$ 的实向量子空间,所以是闭集,所以
  \[
    XY-YX=\lim_{h\to 0}\frac{e^{hX}Ye^{-hX}- Y}{h}\in \lie g.\qedhere
  \]
\end{proof}

注意到 \autoref{thm:lie algebra of matrix group} 的第二点表明即使
$G$ 的元素是复矩阵,$\lie g$ 也不需要是复向量空间。不过,在一些情况下
$\lie g$ 确实是一个复向量空间。

\begin{definition}
  矩阵李群 $G$ 被称为\emph{复的},如果其李代数 $\lie g$ 是复向量空间,
  也就是说,对于所有的 $X\in\lie g$ 有 $iX\in\lie g$。
\end{definition}

\begin{proposition}
  如果 $G$ 是可交换的,那么 $\lie g$ 是可交换的。
\end{proposition}
\begin{proof}
  对于任意两个 $X,Y\in M_n(\mathbb{C})$,那么换位子为
  \[
    [X,Y]=\frac{d}{dt}\bigg|_{t=0}\left(\frac{d}{ds}\bigg|_{s=0}e^{tX}e^{sY}e^{-tX}\right),
  \]
  如果 $G$ 可交换,那么 $e^{tX}e^{sY}e^{-tX}=e^{sY}$ 与 $t$ 无关,所以 
  $[X,Y]=0$。
\end{proof}

\section{示例}

\begin{example}\label{exa:base of su2 and so3}
  由于 $\lie{su}(2)=\{A\in M_2(\mathbb C)\,|\, A^*+A=0,\tr A=0\}$
  以及 $\lie{so}(3)=\{A\in M_3(\mathbb R)\,|\, A^T+A=0\}$。
  所以 $\lie{su}(2)$ 有基
  \[
    E_1=\frac{1}{2}\begin{pmatrix}
      i & 0 \\
      0 & -i
    \end{pmatrix},\quad
    E_2=\frac{1}{2}\begin{pmatrix}
      0 & i \\
      i & 0
    \end{pmatrix},\quad
    E_3=\frac{1}{2}\begin{pmatrix}
      0 & -1 \\
      1 & 0
    \end{pmatrix},
  \]
  它们满足 $[E_1,E_2]=E_3,[E_2,E_3]=E_1,[E_3,E_1]=E_2$。
  $\lie{so}(3)$ 有基
  \[
    F_1=\begin{pmatrix}
      0 & 0 & 0 \\
      0 & 0 & -1 \\
      0& 1 & 0
    \end{pmatrix},\quad
    F_2=\begin{pmatrix}
      0 & 0 & 1 \\
      0 & 0 & 0 \\
      -1 & 0 & 0 
    \end{pmatrix},\quad
    F_3=\begin{pmatrix}
      0 & -1 & 0\\
      1 & 0 & 0 \\
      0 & 0 & 0
    \end{pmatrix},
  \]
  它们满足 $[F_1,F_2]=F_3,[F_2,F_3]=F_1,[F_3,F_1]=F_2$。

  注意到 $E_1,E_2,E_3$ 和 $F_1,F_2,F_3$ 有相同的交换关系,所以这两个李代数是同构的。
\end{example}

\section{李群和李代数同态}

\begin{theorem}
  令 $G,H$ 是矩阵李群,分别有李代数 $\lie g,\lie h$。
  设 $\Phi:G\to H$ 是李群同态。那么存在唯一的实线性映射 $\phi:\lie g\to\lie h$
  使得对所有的 $X\in\lie g$ 有
  \[
    \Phi(e^X)=e^{\phi(X)}.
  \]
  并且 $\phi$ 有性质:
  \begin{enumerate}
    \item 对于所有 $X\in\lie g,A\in G$ 有
    $\phi(AXA^{-1})=\Phi(A)\phi(X)\Phi(A)^{-1}$。
    \item 对于所有 $X,Y\in\lie g$ 有 $\phi([X,Y])=[\phi(X),\phi(Y)]$。
    \item 对于所有 $X\in\lie g$ 有 $\phi(X)=\frac{d}{dt}\big|_{t=0}\Phi(e^{tX})$。
  \end{enumerate}
\end{theorem}
\begin{proof}
  
\end{proof}

\begin{example}\label{exa:homo of SU2 and SO3}
  我们构造一个 $\SU(2)\to \SO(3)$ 的满同态。这是一个非常重要的例子。
  我们知道 $\Orth(3)$ 可以视为 $V\to V$ 的正交变换群,其中 $V$
  是 $3$ 维实内积空间。我们定义 $V$ 是 $2\times 2$ 的迹为零的 Hermite 矩阵
  全体,即 $X\in V$ 当且仅当 $X^*=X$ 以及 $\tr X=0$,故可设
  \begin{equation}\label{eq:expr of X}
    X=\begin{pmatrix}
      x_1 & x_2+ix_3\\
      x_2-ix_3 & -x_1
    \end{pmatrix}.    
  \end{equation}
  此时 $\mathbb{R}^3$ 上的标准内积对应于
  \[
    \langle X_1,X_2\rangle =\frac{1}{2}\tr(X_1X_2).
  \]
  这是因为
  \[
    \frac{1}{2}\tr\left(
      \begin{pmatrix}
        x_1 & x_2+ix_3\\
        x_2-ix_3 & -x_1
      \end{pmatrix}
      \begin{pmatrix}
        x_1' & x_2'+ix_3'\\
        x_2'-ix_3' & -x_1'
      \end{pmatrix}
    \right)=x_1x_1'+x_2x_2'+x_3x_3'.
  \]

  对于每个 $U\in \SU(2)$,我们定义线性映射 $\Phi_U:V\to V$
  为
  \[
      \Phi_U(X)=UXU^{-1}.
  \]
  因为 $U$ 是酉矩阵,所以 $(UXU^{-1})^*=(UXU^*)^*=UXU^{-1}$,
  并且 $\tr(UXU^{-1})=\tr(X)=0$,所以 $\Phi_U(X)$ 确实是 $V$ 的元素。

  注意到
  \[
      \frac{1}{2}\tr\left(
        \Phi_U(X_1)\Phi_U(X_2)
      \right)=\frac{1}{2}\tr\left(UX_1X_2U^{-1}\right)
      =\frac{1}{2}\tr(X_1X_2),
  \]
  所以 $\Phi_U$ 保内积,即 $\Phi_U\in\Orth(3)$ 是正交变换。
  故我们定义了一个映射 $\SU(2)\to\Orth(3)$ 满足 $U\mapsto\Phi_U$。
  不难验证 $\Phi_{U_1U_2}=\Phi_{U_1}\Phi_{U_2}$,所以这是一个群同态。
  因为 $\SU(2)$ 同胚于球面 $\mathbb S^3$ 是连通的,所以上述映射的像集
  处于 $\Orth(3)$ 的单位分支中,即这实际上是一个 $\SU(2)\to\SO(3)$
  的群同态。

  下面我们说明这是一个满同态。任取 $R\in\SO(3)$,根据正交变换的性质,
  $R$ 有一个特征值为 $1$ 的特征向量,设 $X\in V$ 使得 $RX=X$。
  不妨设 $X$ 模长为 $1$。将其扩充为 $V$ 的一组标准正交基 $\{X,Y,Z\}$,
  那么 $R$ 相当于 $Y,Z$-平面的旋转。也即 $R$ 在这组基下有表示矩阵 
  \[
      \begin{pmatrix}
      1 & 0 & 0 \\
      0 &  \cos\theta & -\sin\theta \\
      0 & \sin\theta & \cos\theta 
      \end{pmatrix}.
  \]
  设 $X$ 为 \eqref{eq:expr of X} 式。令 
  \[
      U=\begin{pmatrix}
        e^{i\theta/2} & 0 \\
        0 & e^{-i\theta/2}
      \end{pmatrix},
  \]
  那么
  \[
      UXU^{-1}=\begin{pmatrix}
        x_1' & x_2'+ix_3' \\
        x_2'-ix_3' & -x_1'
      \end{pmatrix},
  \]
  其中 $x_1'=x_1$,
  \[
      x_2'+ix_3'=(x_2\cos\theta-x_3\sin\theta) +i(x_2\sin\theta+x_3\cos\theta).
  \]
  这就表明 $\Phi_U$ 的表示矩阵和 $R$ 相同,所以 $R=\Phi_U$,即这是一个满同态。

  若 $\Phi_U=\mathrm{id}_V$,设 
  \[
      U=\begin{pmatrix}
        \alpha & \beta \\
        -\bar\beta & \bar\alpha
      \end{pmatrix},\quad |\alpha|^2+|\beta|^2=1,
  \]
  那么 
  \[
      U\begin{pmatrix}
        1 & 0 \\
        0 & -1
      \end{pmatrix}U^{-1}=
      % \begin{pmatrix}
      %   \alpha & -\beta \\
      %   -\bar\beta & -\bar\alpha
      % \end{pmatrix}
      % \begin{pmatrix}
      %   \bar\alpha & -\beta \\
      %   \bar\beta & \alpha
      % \end{pmatrix}
      \begin{pmatrix}
        |\alpha|^2-|\beta|^2 & -2\alpha\beta \\
        -2\bar\alpha\bar\beta & |\beta|^2-|\alpha|^2
      \end{pmatrix}=
      \begin{pmatrix}
        1 & 0 \\
        0 & -1
      \end{pmatrix},
  \]
  这表明 $\beta=0$,$|\alpha|=1$。又因为
  以及
  \[
      U\begin{pmatrix}
        0 & 1 \\
        1 & 0
      \end{pmatrix}U^{-1}=
      % \begin{pmatrix}
      %   0 & \alpha \\
      %   \bar\alpha & 0
      % \end{pmatrix}
      % \begin{pmatrix}
      %   \bar\alpha & 0 \\
      %   0 & \alpha
      % \end{pmatrix}
      \begin{pmatrix}
        0 & \alpha^2 \\
        \bar\alpha^2 & 0
      \end{pmatrix}
      =
      \begin{pmatrix}
        0 & 1 \\
        1 & 0
      \end{pmatrix},
  \]
  所以 $\alpha=\pm 1$,故 $U=\pm I_2$。所以同态核为 $\{\pm I_2\}$。
\end{example}

\begin{example}
  令 $\Phi:\SU(2)\to\SO(3)$ 是上例的群同态。那么诱导一个李代数同态
  $\phi:\lie{su}(2)\to\lie{so}(3)$,满足
  \[
    \phi(E_j)=F_j,\quad j=1,2,3.
  \]
  其中 $\{E_1,E_2,E_3\}$ 和 $\{F_1,F_2,F_3\}$ 是 \autoref{exa:base of su2 and so3}
  中的基矩阵。我们看到 $\phi$ 把基送到基,所以是李代数同构,虽然 $\Phi$
  并不是李群同构。
\end{example}
\begin{proof}
  任取 $X\in\lie{su}(2)$,$Y\in V$,其中 $V$ 是 \autoref{exa:homo of SU2 and SO3}
  中的向量空间。那么
  \[
    \phi(X)Y=\frac{d}{dt}\bigg|_{t=0}\Phi(e^{tX})Y
    =\frac{d}{dt}\bigg|_{t=0} e^{tX}Ye^{-tX}=[X,Y],
  \]
  这表明 $\phi(X):Y\mapsto [X,Y]$ 是 $V\to V$ 的线性映射。
  若 $X=E_1$,那么
  \[
    \left[E_1,
    \begin{pmatrix}
      x_1 & x_2+ix_3\\
      x_2-ix_3 & -x_1 
    \end{pmatrix}
    \right]=\begin{pmatrix}
      0 & -x_3+ix_2\\
      -x_3-ix_2 & 0
    \end{pmatrix},
  \]
  这表明 $\phi(E_1)$ 对应矩阵
  \[
    F_1=\begin{pmatrix}
      0 & 0 & 0 \\
      0 & 0 & -1 \\
      0 & 1 & 0
    \end{pmatrix},
  \] 
  故 $\phi(E_1)=F_1$。对于其余两个同理。
\end{proof}

\begin{proposition}
  假设 $G,H,K$ 是矩阵李群,$\Phi:H\to K$ 和 $\Psi:G\to H$ 是李群同态。
  令 $\Lambda:G\to K$ 是复合 $\Psi\circ\Phi$,$\phi,\psi,\lambda$
  分别是 $\Phi,\Psi,\Lambda$ 诱导的李代数同态,那么我们有
  \[
    \lambda=\psi\circ \phi.
  \]
\end{proposition}
\begin{proof}
  任取 $X\in\lie g$,那么
  \[
    \lambda(X)=\frac{d}{dt}\bigg|_{t=0}\Lambda(e^{tX})
    =\frac{d}{dt}\bigg|_{t=0}\Psi(\Phi(e^{tX}))
    =\frac{d}{dt}\bigg|_{t=0}\Psi(e^{t\phi(X)})
    =\psi(\phi(X)),
  \]
  即 $\lambda=\psi\circ\phi$。
\end{proof}

\begin{definition}
  令 $G$ 是矩阵李群,有李代数 $\lie g$。那么对于每个 $A\in G$,
  定义线性映射 $\Ad_A:\lie g\to\lie g$ 为
  \[
    \Ad_A(X)=AXA^{-1}.
  \]
\end{definition}

\begin{proposition}
  令 $G$ 是矩阵李群,有李代数 $\lie g$。那么映射 $A\mapsto \Ad_A$
  是 $G\to\GL(\lie g)$ 的同态。此外,对于每个 $A\in G$,
  $\Ad_A$ 满足 $\Ad_A([X,Y])=[\Ad_AX,\Ad_AY]$。
\end{proposition}

由于 $\Ad:G\to\GL(\lie g)$ 是李群同态,所以诱导一个李代数同态
$\ad:\lie g\to \lie{gl}(\lie g)$,记为 $X\mapsto \ad_X$。那么它们满足
\begin{equation}\label{eq:exp(ad)}
  e^{\ad_X}=\Ad_{e^X}.
\end{equation} 

\begin{proposition}
  令 $G$ 是矩阵李群,$\lie g$ 是李代数。那么对于 $X,Y\in\lie g$ 有
  \[
    \ad_X(Y)=[X,Y].
  \]
\end{proposition}
\begin{proof}
  我们有
  \[
    \ad_X=\frac{d}{dt}\bigg|_{t=0}\Ad({e^{tX}}),
  \]
  所以
  \[
    \ad_X(Y)=\frac{d}{dt}\bigg|_{t=0}\Ad({e^{tX}})(Y)
    =\frac{d}{dt}\bigg|_{t=0} e^{tX}Ye^{-tX}=[X,Y].\qedhere
  \]
\end{proof}

\begin{proposition}
  对于任意 $X\in M_n(\mathbb{C})$,令 $\ad_X:M_n(\mathbb{C})\to M_n(\mathbb{C})$
  为 $\ad_XY=[X,Y]$。那么对于任意 $Y\in M_n(\mathbb{C})$,我们有
  \[
    e^XYe^{-X}=\Ad_{e^X}(Y)=e^{\ad_X}Y,
  \]
  其中
  \[
    e^{\ad_X}(Y)=Y+[X,Y]+\frac{1}{2}[X,[X,Y]]+\cdots.
  \]
\end{proposition}

\section{实李代数的复化}

\begin{definition}
  如果 $V$ 是有限维实向量空间,那么定义 $V$ 的\emph{复化} $V_{\mathbb{C}}$
  是所有的形式线性组合
  \[
    v_1+iv_2\quad v_1,v_2\in V
  \]
  的集合。这显然构成一个实向量空间。我们定义 
  \[
    i(v_1+iv_2)=-v_2+iv_1,
  \]
  此时 $V_{\mathbb{C}}$ 是一个复向量空间。
\end{definition}

\begin{proposition}
  令 $\lie g$ 是一个有限维的实李代数,$\lie g_{\mathbb{C}}$ 是复化。
  那么 $\lie g$ 上的李括号在 $\lie g_{\mathbb{C}}$ 上有一个唯一的延拓,
  使得 $\lie g_{\mathbb{C}}$ 是一个复李代数。此时 $\lie g_{\mathbb{C}}$
  被称为 $\lie g$ 的\emph{复化}。
\end{proposition}
\begin{proof}
  唯一性是显然的,因为根据双线性性,必须有
  \[
    [X_1+iX_2,Y_1+iY_2]=[X_1,Y_1]-[X_2,Y_2]+i\bigl(
      [X_1,Y_2]+[X_2,Y_1]
    \bigr).
  \]
  下面我们只需要验证上述定义满足反对称性和 Jacobi 恒等式。
  反对称性也是显然的。对于 Jacobi 恒等式,利用复线性性,
  当 $X\in\lie g_{\mathbb{C}},Y,Z\in\lie g$ 的时候有
  \[
    [X,[Y,Z]]+[Y,[Z,X]]+[Z,[X,Y]]=0,
  \]
  同理可得一般情况下的 Jacobi 恒等式。
\end{proof}

\begin{proposition}
  设 $\lie g\subseteq M_n(\mathbb{C})$ 是实李代数并且对于非零的 $X\in\lie g$
  有 $iX\notin\lie g$。那么 $\lie g_{\mathbb{C}}$ 同构于集合
  \[
    \bigl\{X+iY\bigm| X,Y\in\lie g\bigr\}.
  \]
\end{proposition}




