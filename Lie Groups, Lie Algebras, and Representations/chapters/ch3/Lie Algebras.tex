\chapter{李代数}

\section{定义和初步例子}

\begin{definition}
  一个\emph{有限维实或者复李代数}指的是一个有限维的实或者复向量空间 $\lie g$,
  配备一个映射 $[\cdot,\cdot]:\lie g\times \lie g\to\lie g$,
  满足:
  \begin{enumerate}
    \item $[\cdot,\cdot]$ 是双线性的。
    \item $[\cdot,\cdot]$ 是反对称的:对于任意 $X,Y\in\lie g$ 有 $[X,Y]=-[Y,X]$。
    \item Jacobi 恒等式:对于任意 $X,Y,Z\in \lie g$ 有 
    \[
      [X,[Y,Z]]+[Y,[Z,X]]+[Z,[X,Y]]=0.
    \]
  \end{enumerate}
  若 $[X,Y]=0$,那么我们说 $X,Y$ 是\emph{可交换的}。如果对于所有 $X,Y\in\lie g$
  都有 $[X,Y]=0$,那么我们说 $\lie g$ 是\emph{可交换的}。
\end{definition}

$[\cdot,\cdot]$ 通常被称为 $\lie g$ 上的李括号。注意到反对称性表明
$[X,X]=0$。李括号运算通常不满足结合律,然而 Jacobi 恒等式可以被视为结合律
的替代方案。

\begin{example}
  令 $\lie g= \mathbb{R}^3$,$[\cdot,\cdot]:\mathbb{R}^3\times \mathbb{R}^3\to \mathbb{R}^3$
  定义为
  \[
    [x,y]=x\times y,
  \]
  其中 $x\times y$ 是向量叉乘。那么 $\lie g$ 是一个李代数。
\end{example}
\begin{proof}
  双线性性和反对称性是显然的。根据双线性性,只需要对基向量 $e_1,e_2,e_3$
  验证 Jacobi 恒等式即可。如果 $j,k,l$ 互不相同,那么 $e_j,e_k,e_l$ 
  中任意两个的叉乘等于第三个或者第三个的相反方向,所以 Jacobi 恒等式中每一项
  都是 $0$。于是只需要验证 $j,k,l$ 中有两个相同的情况即可,通过重新排序,
  只需要验证
  \[
    [e_j,[e_j,e_k]]+[e_j,[e_k,e_j]]+[e_k,[e_j,e_j]]=0,
  \]
  上式的前两项相反,第三项为零,故叉乘满足 Jacobi 恒等式。
\end{proof}

\begin{example}\label{exa:lie algebra of associative algebra}
  令 $\mathcal A$ 是结合代数,$\lie g$ 是 $\mathcal A$ 的一个子空间,使得
  任意的 $X,Y\in\lie g$ 有 $XY-YX\in\lie g$。那么 $\lie g$ 是一个李代数,有李括号
  \[
    [X,Y]=XY-YX.
  \]
\end{example}
\begin{proof}
  双线性性和反对称性是显然的。对于 Jacobi 恒等式,每个双层李括号会产生 4 项,
  所以总共有 12 项,即
  \[
    [X,[Y,Z]]=[X,YZ-ZY]=XYZ-XZY-YZX+ZYX,
  \]
  对 $X,Y,Z$ 进行轮换,那么正项负项刚好抵消,故这是一个李代数。
\end{proof}

如果我们仔细观察 Jacobi 恒等式的证明,我们会发现 $XYZ$ 实际上以两种方式出现,
一种是 $X(YZ)$,一种是 $(XY)Z$。所以代数 $\mathcal A$ 的结合性是重要的。
对于任意李代数,Jacobi 恒等式意味着李括号的行为\emph{就像}在某个结合代数
中的 $XY-YX$ 一样,即使这个李括号本身不是这样定义的(比如叉乘)。
实际上,可以证明每个李代数 $\lie g$ 都可以嵌入到一个结合代数 $\mathcal A$
中,使得其李括号变成 $XY-YX$。

\begin{example}
  令 $\lie{sl}(n,\mathbb{C})$ 是所有满足 $\tr X=0$ 的 $X\in M_n(\mathbb{C})$
  构成的空间。那么 $\lie{sl}(n,\mathbb{C})$ 是李代数,有李括号
  $[X,Y]=XY-YX$。
\end{example}
\begin{proof}
  我们有
  \[
    \tr(XY-YX)=\tr(XY)-\tr(YX)=0,
  \]
  所以可以应用 \autoref{exa:lie algebra of associative algebra}。
\end{proof}

\begin{definition}
  实或者复李代数 $\lie g$ 的一个\emph{子代数}指的是一个子空间 $\lie h$
  使得任取 $H_1,H_2\in\lie h$ 有 $[H_1,H_2]\in\lie h$。如果 $\lie g$
  是复李代数,$\lie h$ 是 $\lie g$ 的实子空间并且对李括号封闭,那么
  $\lie h$ 被称为 $\lie g$ 的\emph{实子代数}。

  李代数 $\lie g$ 的一个子代数 $\lie h$ 被称为 $\lie g$ 中的\emph{理想},
  如果对于任意 $H\in \lie h,X\in\lie g$ 有 $[X,H]\in\lie h$。

  李代数 $\lie g$ 的\emph{中心}指的是一些 $X\in\lie g$ 的集合,对于每个
  $X$,其使得任取 $Y\in\lie g$,有 $[X,Y]=0$。
\end{definition}

\begin{definition}
  如果 $\lie g,\lie h$ 是李代数,线性映射 $\phi:\lie g\to\lie h$ 
  满足 $\phi([X,Y])=[\phi(X),\phi(Y)]$,那么 $\phi$ 被称为\emph{李代数同态}。
  此外,如果 $\phi$ 是双射,那么 $\phi$ 被称为\emph{李代数同构}。
\end{definition}

\begin{definition}
  如果 $\lie g$ 是李代数,$X\in\lie g$,定义线性映射 $\ad_X:\lie g\to\lie g$
  为 
  \[
    \ad_X(Y)=[X,Y].
  \]
  映射 $X\mapsto \ad_X$ 被称为\emph{伴随映射}或者\emph{伴随表示}。
\end{definition}

虽然 $\ad_X(Y)$ 就是 $[X,Y]$,但是 $\ad$ 的记号是有方便的。例如,我们可以把
\[ 
[X,[X,[X,[X,Y]]]]
\]  
写为 $(\ad_X)^4(Y)$。此外,映射 $X\mapsto \ad_X$ 可以视为 $\lie g\to\End(\lie g)$
的映射。Jacobi 恒等式等价于 $\ad_X$ 是李括号的\emph{导子}:
\begin{equation}\label{eq:Jacobi identity of ad}
  \ad_X([Y,Z])=[\ad_X(Y),Z]+[Y,\ad_X(Z)].
\end{equation}

\begin{proposition}
  如果 $\lie g$ 是李代数,那么
  \[
    \ad_{[X,Y]}=\ad_X\ad_Y-\ad_Y\ad_X=[\ad_X,\ad_Y],
  \]
  也就是说 $\ad:\lie g\to\End(\lie g)$ 是李代数同态。
\end{proposition}
\begin{proof}
  注意到
  \[
    \ad_{[X,Y]}(Z)=[[X,Y],Z],
  \]
  并且
  \[
    [\ad_X,\ad_Y](Z)=[X,[Y,Z]]-[Y,[X,Z]],
  \]
  所以上式等价于 Jacobi 恒等式。
\end{proof}

\begin{definition}
  如果 $\lie g_1,\lie g_2$ 是李代数,那么直和 $\lie g_1\oplus\lie g_2$
  也是李代数,配备李括号
  \[
    [(X_1,X_2),(Y_1,Y_2)]=([X_1,Y_1],[X_2,Y_2]).
  \]
  如果 $\lie g$ 是李代数,$\lie g_1,\lie g_2$ 是两个子代数,
  作为向量空间有 $\lie g=\lie g_1\oplus\lie g_2$ 并且对于 $X_1\in\lie g_1,X_2\in\lie g_2$
  有 $[X_1,X_2]=0$,那么我们说 $\lie g$ 分解为 $\lie g_1$ 和 $\lie g_2$
  的直和。
\end{definition}

\begin{definition}
  令 $\lie g$ 是有限维实或者复李代数,$X_1,\dots,X_N$ 是 $\lie g$ 的一组基,
  那么有唯一的常数 $c_{jkl}$ 使得 
  \[
    [X_j,X_k]=\sum_{l=1}^N c_{jkl}X_l,
  \]
  $c_{jkl}$ 被称为 $\lie g$ 的\emph{结构常数}。
\end{definition}

虽然我们不会经常遇到结构常数,但是在物理课程中会经常使用。结构常数满足下面两个恒等式:
对于 $j,k,l,m$ 有 
\begin{align*}
  c_{jkl}+c_{kjl}&=0,\\
  \sum_n (c_{jkn}c_{nlm}+c_{kln}c_{njm}+c_{ljn}c_{nkm})&=0,
\end{align*}
第一个式子来源于反对称性,第二个式子来源于 Jacobi 恒等式。

\section{单、可解和幂零的李代数}

\begin{definition}
  一个李代数 $\lie g$ 被称为\emph{不可约的},如果 $\lie g$ 中的理想只有
  $\lie g$ 和 $\{0\}$。$\lie g$ 被称为\emph{单的},如果 $\lie g$
  是不可约的且 $\dim \lie g\geq 2$。
\end{definition}

一维的李代数一定是不可约的,因为它没有非平凡的子空间,所以没有非平凡的子代数,
进而没有非平凡的理想。但是,根据定义,一维的李代数不被认为是单的!

此外,还可以注意到一维李代数 $\lie g$ 一定是可交换的,因为对于任意
$X\in\lie g$ 和标量 $a,b$ 都有 $[aX,bX]=ab[X,X]=0$。另一方面,如果 $\lie g$
是可交换的,那么 $\lie g$ 的任意子空间都是理想,所以对于可交换的李代数而言,
只有一维的情况才是不可约的。因此,单李代数的等价定义是其\emph{不可约且不交换}。

显然,这些概念在群论中有对应的类比。其中子群类比于子代数,正规子群类比于理想。
(例如,李代数同态的核总是是一个理想,群同态的核总是为正规子群)。
群论中没有非平凡正规子群的群被称为单群,李代数中没有非平凡理想的李代数被称为
单李代数。

\begin{proposition}
  李代数 $\lie{sl}(2,\mathbb{C})$ 是单的。
\end{proposition}
\begin{proof}
  我们使用下列 $\lie{sl}(2,\mathbb{C})$ 的基:
  \[
    X=\begin{pmatrix}
      0 & 1 \\ 0 & 0
    \end{pmatrix},\quad
    Y=\begin{pmatrix}
      0 & 0 \\ 1 & 0
    \end{pmatrix},\quad
    H=\begin{pmatrix}
      1 & 0 \\ 0 & -1
    \end{pmatrix}.
  \]
  计算可知它们满足 $[X,Y]=H,[H,X]=2X,[H,Y]=-2Y$。设 $\lie h$
  是 $\lie{sl}(2,\mathbb{C})$ 中的理想并且 $\lie h$ 包含元素 $Z=aX+bH+cY$,
  其中 $a,b,c\in \mathbb{C}$ 是不全为零的复数。
  首先假设 $c\neq 0$,那么
  \[
    [X,[X,Z]]=[X,-2bX+cH]=-2cX
  \]
  是 $X$ 的非零倍数。$\lie h$ 是理想表明 $X\in\lie h$。另一方面,
  有 $[Y,X]=-H$ 以及 $[Y,[Y,X]]=2Y$,所以 $Y,H\in \lie h$。
  这表明此时 $\lie h=\lie{sl}(2,\mathbb{C})$。

  现在假设 $c=0,b\neq 0$。那么 $[X,Z]=-2bX$ 表明 $X\in\lie h$,
  然后同样可得 $\lie h=\lie{sl}(2,\mathbb{C})$。
  最后,如果 $c=b=0$ 但是 $a\neq 0$,那么 $X=Z/a\in\lie h$,仍然得到 
  $\lie h=\lie{sl}(2,\mathbb{C})$。这就表明 $\lie{sl}(2,\mathbb{C})$
  是单李代数。
\end{proof}

\begin{definition}
  如果 $\lie g$ 是李代数,那么 $\lie g$ 中的\emph{换位子理想} $[\lie g,\lie g]$
  定义为所有换位子的线性组合张成的空间,即 $Z\in[\lie g,\lie g]$ 当且仅当
  \[
    Z=c_1[X_1,Y_1]+\cdots+c_m[X_m,Y_m].
  \]
\end{definition}

对于任意 $X,Y\in\lie g$,换位子 $[X,Y]\in[\lie g,\lie g]$,这表明
$[\lie g,\lie g]$ 确实是一个理想。

\begin{definition}
  对于李代数 $\lie g$,我们定义一个子代数序列 $\lie g_0,\lie g_1,\lie g_2,\dots$
  为:$\lie g_0=\lie g$,$\lie g_1=[\lie g_0,\lie g_0]$,
  $\lie g_2=[\lie g_1,\lie g_1]$,等等。这些子代数被称为 $\lie g$ 的\emph{导出列}。
  如果对于某个 $j$ 使得 $\lie g_j=\{0\}$,那么我们说 $\lie g$ 是\emph{可解的}。
\end{definition}

利用 Jacobi 恒等式不难证明每个 $\lie g_j$ 都是 $\lie g$ 的理想,例如对于
$[X,Y]\in\lie g_2$,其中 $X,Y\in\lie g_1$,那么对于任意 $Z\in \lie g$,有
\[
  [Z,[X,Y]]=-[X,[Y,Z]]-[Y,[Z,X]]\in\lie g_2.
\]

\begin{definition}
  对于任意李代数 $\lie g$,定义理想序列 $\lie g^j$ 为:
  $\lie g^0=\lie g$,$\lie g^{j+1}$ 为所有的形如 $[X,Y]$ 的换位子的线性组合
  构成的空间,其中 $X\in\lie g$ 以及 $Y\in\lie g^j$。这些子代数被称为
  $\lie g$ 的\emph{上中心列}。如果对于某个 $j$ 有 $\lie g^j=\{0\}$,那么
  我们说 $\lie g$ 是\emph{幂零的}。
\end{definition}

等价地说,$\lie g^j$ 由所有的 $j$-重换位子张成:
\[
  [X_1,[X_2,[X_3,\dots,[X_j,X_{j+1}]\dots]]].
\]
注意到 $j$-重换位子也是 $(j-1)$-重换位子,所以 $\lie g^{j-1}\supseteq \lie g^j$。
对于任意 $X\in\lie g$ 和 $Y\in\lie g^j$,我们有 $[X,Y]\in \lie g^{j+1}\subseteq\lie g^j$,
所以 $\lie g^j$ 是 $\lie g$ 的理想。此外,显然有 $\lie g_j\subseteq \lie g^j$,
因此幂零李代数都是可解的。

\begin{proposition}
  如果 $\lie g\subseteq M_3(\mathbb{R})$ 是 $3\times 3$ 上三角矩阵并且对角线为零。
  那么 $\lie g$ 满足 \autoref{exa:lie algebra of associative algebra},
  并且是一个幂零李代数。
\end{proposition}
\begin{proof}
  显然 $\lie g$ 是李代数。我们选取基
  \[
    X=\begin{pmatrix}
      0 & 1 & 0 \\
      0 & 0 & 0 \\
      0 & 0 & 0 
    \end{pmatrix},\quad 
    Y=\begin{pmatrix}
      0 & 0 & 1 \\
      0 & 0 & 0 \\
      0 & 0 & 0 
    \end{pmatrix},\quad
    Z=\begin{pmatrix}
      0 & 0 & 0 \\
      0 & 0 & 1 \\
      0 & 0 & 0 
    \end{pmatrix}.
  \]
  直接计算得 $[X,Y]=Z$ 以及 $[X,Z]=[Y,Z]=0$。
  故 $[\lie g,\lie g]$ 由 $Z$ 张成,进而 $[\lie g,[\lie g,\lie g]]=0$,
  所以 $\lie g$ 是幂零的。
\end{proof}

\begin{proposition}
  如果 $\lie g\subseteq  M_2(\mathbb{C})$ 是形如
  \[
    \begin{pmatrix}
      a & b \\ 0 & c
    \end{pmatrix}\quad a,b,c\in \mathbb{C}
  \]
  的 $2\times 2$ 矩阵,那么 $\lie g$ 满足 \autoref{exa:lie algebra of associative algebra},
  并且是可解但不幂零的李代数。
\end{proposition}
\begin{proof}
  直接计算得 
  \[
    \left[\begin{pmatrix}
      a & b \\ 0 & c
    \end{pmatrix},\begin{pmatrix}
      d & e \\ 0 & f
    \end{pmatrix}\right]=\begin{pmatrix}
      0 & h \\ 0 & 0 
    \end{pmatrix},
  \]
  其中 $h=ae+bf-bd-ce$,这表明 $\lie g$ 是一个子代数。此外,
  还表明换位子理想 $[\lie g,\lie g]$ 是一维的,所以是可交换的。
  因此 $\lie g_2=0$,故 $\lie g$ 是可解的。另一方面,考虑
  \[
    H=\begin{pmatrix}
      1 & 0 \\ 0 & -1
    \end{pmatrix},\quad
    X=\begin{pmatrix}
      0 & 1 \\ 0 & 0
    \end{pmatrix}.
  \]
  我们有 $[H,X]=2X$,所以
  \[
    [H,[H,[H,\dots,[H,X]\dots]]]
  \]
  永远是 $X$ 的非零倍数,所以对于任意 $j$ 都有 $\lie g^j\neq 0$,故 $\lie g$ 不是幂零的。
\end{proof}

\section{矩阵李群的李代数}

\begin{definition}
  令 $G$ 是一个矩阵李群。$G$ 的李代数 $\lie g$ 定义为所有矩阵 $X$ 的集合,
  其中 $X$ 使得对于任意实数 $t$,指数 $e^{tX}\in G$。
\end{definition}

熟悉流形理论的读者可以发现,这实际上就是再说 $G$ 在单位元处的切空间,因为
$\gamma(t)=e^{tX}$ 是以单位元为起点的切向量为 $X$ 的光滑曲线。

\begin{proposition}
  令 $G$ 是矩阵李群,$X\in\lie g$,那么 $e^X$ 是 $G$ 的单位分支 $G_0$
  中的元素。
\end{proposition}
\begin{proof}
  根据定义,$e^{tX}$ 就是连接单位元和 $e^X$ 的道路。
\end{proof}

\begin{theorem}\label{thm:lie algebra of matrix group}
  令 $G$ 是矩阵李群,有李代数 $\lie g$。如果 $X,Y\in\lie g$,那么
  \begin{enumerate}
    \item 对于任意 $A\in G$ 有 $AXA^{-1}\in\lie g$。
    \item 对于实数 $s$ 有 $sX\in\lie g$。
    \item $X+Y\in\lie g$。
    \item $XY-YX\in\lie g$。
  \end{enumerate}
\end{theorem}

注意到 \autoref{thm:lie algebra of matrix group} 的第二点表明即使
$G$ 的元素是复矩阵,$\lie g$ 也不需要是复向量空间。不过,在一些情况下
$\lie g$ 确实是一个复向量空间。

\begin{definition}
  矩阵李群 $G$ 被称为\emph{复的},如果其李代数 $\lie g$ 是复向量空间,
  也就是说,对于所有的 $X\in\lie g$ 有 $iX\in\lie g$。
\end{definition}

\begin{proposition}
  如果 $G$ 是可交换的,那么 $\lie g$ 是可交换的。
\end{proposition}
\begin{proof}
  对于任意两个 $X,Y\in M_n(\mathbb{C})$,那么换位子为
  \[
    [X,Y]=\frac{d}{dt}\bigg|_{t=0}\left(\frac{d}{dt}\bigg|_{s=0}e^{tX}e^{sY}e^{-tX}\right),
  \]
  如果 $G$ 可交换,那么 $e^{tX}e^{sY}e^{-tX}=e^{sY}$ 与 $t$ 无关,所以 
  $[X,Y]=0$。
\end{proof}

\section{示例}






 




